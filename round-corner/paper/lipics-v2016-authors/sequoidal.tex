\documentclass[a4paper,UKenglish]{lipics-v2016}
%This is a template for producing LIPIcs articles. 
%See lipics-manual.pdf for further information.
%for A4 paper format use option "a4paper", for US-letter use option "letterpaper"
%for british hyphenation rules use option "UKenglish", for american hyphenation rules use option "USenglish"
% for section-numbered lemmas etc., use "numberwithinsect"
 
\usepackage{microtype}%if unwanted, comment out or use option "draft"

%\graphicspath{{./graphics/}}%helpful if your graphic files are in another directory

\bibliographystyle{plainurl}% the recommended bibstyle

% Author macros::begin %%%%%%%%%%%%%%%%%%%%%%%%%%%%%%%%%%%%%%%%%%%%%%%%
\title{Sequoidal Categories and Transfinite Games: Towards a Coalgebraic Approach to Linear Logic\footnote{This work was partially supported by [TODO: find the appropriate grant codes]}}
\titlerunning{Sequoidal Categories and Transfinite Games} %optional, in case that the title is too long; the running title should fit into the top page column

%% Please provide for each author the \author and \affil macro, even when authors have the same affiliation, i.e. for each author there needs to be the  \author and \affil macros
\author[1]{William John Gowers}
\author[2]{James Laird}
\affil[1]{Department of Computer Science, University of Bath, Claverton Down, Bath.  BA2 7AY.  United Kingdom\\
  \texttt{W.J.Gowers@bath.ac.uk}}
\affil[2]{Department of Computer Science, University of Bath, Claverton Down, Bath.  BA2 7AY.  United Kingdom\\
  \texttt{jiml@cs.bath.ac.uk}}
\authorrunning{W.\,J. Gowers and J. Laird} %mandatory. First: Use abbreviated first/middle names. Second (only in severe cases): Use first author plus 'et. al.'

\Copyright{William John Gowers and James Laird}%mandatory, please use full first names. LIPIcs license is "CC-BY";  http://creativecommons.org/licenses/by/3.0/

\subjclass{F3.2.2 Denotational Semantics}% mandatory: Please choose ACM 1998 classifications from http://www.acm.org/about/class/ccs98-html . E.g., cite as "F.1.1 Models of Computation". 
\keywords{Game semantics, Linear logic, Transfinite games, Sequoid operator, Exponentials}% mandatory: Please provide 1-5 keywords
% Author macros::end %%%%%%%%%%%%%%%%%%%%%%%%%%%%%%%%%%%%%%%%%%%%%%%%%

%Editor-only macros:: begin (do not touch as author)%%%%%%%%%%%%%%%%%%%%%%%%%%%%%%%%%%
\EventEditors{John Q. Open and Joan R. Acces}
\EventNoEds{2}
\EventLongTitle{42nd Conference on Very Important Topics (CVIT 2016)}
\EventShortTitle{CVIT 2016}
\EventAcronym{CVIT}
\EventYear{2016}
\EventDate{December 24--27, 2016}
\EventLocation{Little Whinging, United Kingdom}
\EventLogo{}
\SeriesVolume{42}
\ArticleNo{23}
% Editor-only macros::end %%%%%%%%%%%%%%%%%%%%%%%%%%%%%%%%%%%%%%%%%%%%%%%

% Author macros and packages
\usepackage{IEEEtrantools} % for \IEEEeqnarray
\usepackage{pbox} % for \pbox
\usepackage{cmll} % Linear logic symbols!
\usepackage{wasysym} % For left and rigth moons
\usepackage{mathtools}

%% Graphics %%
\usepackage{tikz}
\usetikzlibrary{cd}
\usetikzlibrary{patterns}
\usetikzlibrary{calc}
\usetikzlibrary{decorations.pathmorphing}

%% Get the sqsubsetneqq character from the mathabx package
\DeclareFontFamily{U}{mathb}{\hyphenchar\font45}
\DeclareFontShape{U}{mathb}{m}{n}{
      <5> <6> <7> <8> <9> <10> gen * mathb
      <10.95> mathb10 <12> <14.4> <17.28> <20.74> <24.88> mathb12
      }{}
\DeclareSymbolFont{mathb}{U}{mathb}{m}{n}

\DeclareMathSymbol{\sqsubsetneq}    {3}{mathb}{"88}
\DeclareMathSymbol{\varsqsubsetneq} {3}{mathb}{"8A}
\DeclareMathSymbol{\varsqsubsetneqq}{3}{mathb}{"92}
\DeclareMathSymbol{\sqsubsetneqq}   {3}{mathb}{"90}

%% Theorems %%
\theoremstyle{plain}
\newtheorem{proposition}[theorem]{Proposition}
\theoremstyle{definition}
\newtheorem{notation}[theorem]{Notation}

%% Other macros
\newcommand*\from{\colon}
\def \inv {^{-1}}
\DeclareMathOperator{\id}{id}
\DeclareMathOperator{\pr}{pr}
\newcommand{\tensor}{\otimes}
\newcommand{\sequoid}{\oslash}
\renewcommand{\implies}{\multimap}
\newcommand{\comp}[2]{#1 \circ #2}
\newcommand{\cprd}{\sqcup}
\newcommand{\bigcprd}{\bigsqcup}
\newcommand{\C}{\mathcal C}
\newcommand{\F}{\mathcal F}
\newcommand{\G}{\mathcal G}
\newcommand{\suchthat}{\;\colon\;}
\newcommand{\varsuchthat}{\;\mid\;}
\newcommand{\OP}{\{O,P\}}
\newcommand{\emptyplay}{\epsilon}
\newcommand{\prefix}{\sqsubseteq}
\newcommand{\pprefix}{\sqsubsetneqq}
\newcommand{\ppprefix}{\sqsubset}
\DeclareMathOperator{\assoc}{assoc}
\DeclareMathOperator{\lunit}{lunit}
\DeclareMathOperator{\runit}{runit}
\DeclareMathOperator{\sym}{sym}
\newcommand{\braid}{\sym}
\newcommand{\blank}{\;\underline{\hspace{1.5ex}}\;}
\newcommand{\der}{{\mathtt{der}}}
\newcommand{\mult}{{\mathtt{mult}}}
\DeclareMathOperator{\wk}{wk}
\newcommand{\toisom}{{\xrightarrow{\cong}}}
\newcommand{\passoc}{{\mathtt{passoc}}}
\newcommand{\pcomm}{{\mathtt{pcomm}}}
\newcommand{\run}{{\mathtt{r}}}
\newcommand{\lun}{{\mathtt{l}}}
\newcommand{\fcoal}[1]{{\leftmoon #1 \rightmoon}}
\renewcommand{\subset}{\subseteq}
\let\oldemptyset\emptyset
\let\emptyset\varnothing
\newcommand{\s}{\mathfrak{s}}
\newcommand{\dist}{{\mathtt{dist}}}
\newcommand{\dec}{{\mathtt{dec}}}
\renewcommand{\int}{{\mathtt{int}}}
\DeclareMathOperator{\CCom}{CCom}

%% Constant width xrightarrows
\newlength{\arrow}
\settowidth{\arrow}{\scriptsize$1000$}
\newcommand*{\constantwidthxrightarrow}[1]{\xrightarrow{\mathmakebox[\arrow]{#1}}}

%% Landscape pages
\usepackage{everypage}
\usepackage{environ}
\usepackage{pdflscape}
\newcounter{abspage}

\makeatletter
\newcommand{\newSFPage}[1]% #1 = \theabspage
  {\global\expandafter\let\csname SFPage@#1\endcsname\null}

\NewEnviron{SidewaysFigure}{\begin{figure}[p]
\protected@write\@auxout{\let\theabspage=\relax}% delays expansion until shipout
  {\string\newSFPage{\theabspage}}%
\ifdim\textwidth=\textheight
  \rotatebox{90}{\parbox[c][\textwidth][c]{\linewidth}{\BODY}}%
\else
  \rotatebox{90}{\parbox[c][\textwidth][c]{\textheight}{\BODY}}%
\fi
\end{figure}}

\AddEverypageHook{% check if sideways figure on this page
  \ifdim\textwidth=\textheight
    \stepcounter{abspage}% already in landscape
  \else
    \@ifundefined{SFPage@\theabspage}{}{\global\pdfpageattr{/Rotate 0}}%
    \stepcounter{abspage}%
    \@ifundefined{SFPage@\theabspage}{}{\global\pdfpageattr{/Rotate 90}}%
  \fi}
\makeatother

\begin{document}

\maketitle

\begin{abstract}
  In \cite{laird02}, Laird introduces the concept of a \emph{sequoidal category} as a formalization of causality in game semantics.  A sequoidal category is like a monoidal category with an extra connective, $\sequoid$, that allows one to construct an exponential object as a final coalgebra \cite{martinsthesis},  Under a further hypothesis, it is possible to show that this final coalgebra allows one to construct cofree commutative comonoids in the category, giving us a model of the exponential connective $\oc$ from linear logic.  In the first part of this note, we review the coalgebraic arguments for constructing the cofree commutative comonoid, which are known but do not yet appear in print.  In the second part, we show that the extra hypotheses are necessary by outlining a definition of \emph{transfinite game}, in which the carriers of the sequoidal coalgebra and the cofree commutative comonoid do not coincide.
\end{abstract}

\section{Sequoidal categories}

\subsection{Game semantics and the sequoidal operator}

We shall present a form of game semantics in the style of \cite{hyland1997games} and \cite{abramskyjagadeesangames}.  A game will be given by a tuple
\[
  A = (M_A, \lambda_A, b_A, P_A)
  \]
where
\begin{itemize}
  \item $M_A$ is a set of moves.
  \item $\lambda_A\from M_A\to\OP$ is a function designating each move as either an \emph{$O$-move} or a \emph{$P$-move}.
  \item $b_A\in\OP$ is a choice of starting player.
  \item $P_A\subset M_A^*$ is a prefix-closed set of alternating plays (so if $sab\in P_A$ then $\lambda_A(a)=\neg\lambda_A(b)$) such that if $as\in P_A$ then $\lambda_A(a)=b_A$.
\end{itemize}

We call $sa\in P_A$ a \emph{$P$-position} if $a$ is a $P$-move and an \emph{$O$-position} if $a$ is an $O$-move.

A \emph{strategy} for player $P$ for a game $A$ is identified with the set of positions that may arise when playing according to that strategy.  Namely, it is a non-empty prefix-closed subset $\sigma\subset P_A$ satisfying the two conditions:
\begin{description}
  \item[(sO)] If $s\in\sigma$ is a $P$-position and $a$ is an $O$-move such that $sa\in P_A$, then $sa\in\sigma$.
  \item[(sP)] If $sa,sb\in\sigma$ are $P$-positions, then $a=b$.
\end{description}

We shall now concentrate on games $A$ for which $b_A=O$, called \emph{negative games}.  We shall informally describe the standard connectives on negative games:

\begin{description}
  \item[Product] If $A$ and $B$ are negative games then the \emph{product} $A\times B$ is the game given by placing the game trees for $A$ and $B$ side by side: that is, player $O$ may play his first move either in $A$ or in $B$.  Thereafter, play continues in the game that player $O$ has chosen.
  \item[Tensor Product] The tensor product $A\tensor B$ is also played by playing the games $A$ and $B$ in parallel, but this time player $O$ may elect to switch games whenever it is his turn and continue play in the game he has switched to.
  \item[Linear implication] The implication $A\implies B$ is played by playing the game $B$ in parallel with the \emph{negation} of $A$ - that is, the game formed by switching the roles of players $P$ and $O$ in $A$.  Since play in the negation of $A$ starts with a $P$-move, player $O$ is forced to make his first move in the game $B$.  Thereafter, player $P$ may switch games whenever it is her turn.
\end{description}

If $A,B,C$ are negative games, $\sigma$ is a strategy for $A\implies B$ and $\tau$ is a strategy for $B\implies C$, then we may form a strategy $\comp\tau\sigma$ for $A\implies C$ by setting
\[
  \sigma\|\tau = \{\s\in (M_A\cprd M_B\cprd M_C)^*\suchthat \s\vert_{A,B}\in\sigma,\;s\vert_{B,C}\in\tau\}
  \]
and then defining
\[
  \comp\tau\sigma = \{\s\vert_{A,C}\suchthat\s\in\comp\tau\sigma\}
  \]
It is well known (see, for example, \cite{abramskyjagadeesangames}) that $\comp\tau\sigma$ is indeed a strategy for $A\implies C$ and that this form of composition is associative and has an identity.  It is also well known that the resulting category $\G$ of games and strategies has products given by the operator $\times$ and a symmetric monoidal closed structure given by the operations $\tensor$ and $\implies$.  

We turn now to the non-standard \emph{sequoid} connective $\sequoid$.  If $A$ and $B$ are negative games, then the sequoid $A\sequoid B$ is similar to the tensor product $A\tensor B$, but with the restriction that player $O$'s first move must take place in the game $A$.  We observe immediately that we have structural isomorphisms
\begin{gather*}
  \dist\from A\tensor B\toisom (A\sequoid B)\times(B\sequoid A)\\
  \dec\from(A\times B)\sequoid C\toisom (A\sequoid C)\times (B\sequoid C)\\
  \passoc\from (A\sequoid B)\sequoid C\toisom A\sequoid (B\tensor C)
\end{gather*}

One further question to ask is: does the sequoid operator give rise to a functor $\blank\sequoid\blank\from\G\times\G\to\G$, as the tensor operator does?  The answer is no: indeed, let $A,B,C,D$ be negative games, let $\sigma$ be a strategy for $A\implies C$ and let $\tau$ be a strategy for $B\implies D$.  Our aim is to construct a natural strategy $\sigma\sequoid\tau$ for $(A\sequoid B)\implies (C\sequoid D)$.  There is an obvious way to try and do this: player $P$ should play according to the strategy $\sigma$ whenever player $O$'s last move was in $A$ or $C$, and according to $\tau$ whenever player $O$'s last move was in $B$ or $D$.

We show that this does not in general give us a strategy for $(A\sequoid B)\implies (C\sequoid D)$.  Suppose that $\sigma$ is such that player $P$'s response to some opening move in $C$ is another move in $C$ and suppose that $\tau$ is such that player $P$'s response to some opening move in $D$ is a move in $B$ (for example, $\tau$ is a copycat strategy).  Then we end up with the following sequence of events in the game $(A\sequoid B)\implies(C\sequoid D)$  :
\begin{enumerate}
  \item Player $O$ starts with a move in $C$ (as he must).
  \item Player $P$ responds according to $\sigma$ with another move in $C$.
  \item Player $O$ decides to switch games and play a move in $D$.
  \item Player $P$ responds according to $\tau$ with a move in $B$.
\end{enumerate}

But now player $P$'s last move is not a legal move in $(A\sequoid B)\implies(C\sequoid D)$, since no moves have been played in $A$ yet.

We get round this problem by requiring that the strategy $\sigma$ be \emph{strict} -- that is, whatever player $O$'s opening move in $C$ is, player $P$'s reply must be a move in $A$.  

\begin{definition}
  Let $N,L$ be negative games and let $\sigma$ be a strategy for $N\implies L$.  We say that $\sigma$ is \emph{strict} if player $P$'s reply to an opening move in $L$ is always a move in $N$.  
\end{definition}

Identity strategies are strict and the composition of two strict strategies is strict, so we get a full-on-objects subcategory $\G_s$ of $\G$ where the morphisms are strict strategies.  Then the sequoid operator gives rise to a functor:
\[
  \blank\sequoid\blank\from \G_s\times\G\to\G_s
  \]

\subsection{Sequoidal categories}

We now have the motivation required to give the definition of a \emph{sequoidal category} from \cite{laird02}.  

\begin{definition}
  A \emph{sequoidal category} consists of the following data:
  \begin{itemize}
    \item A symmetric monoidal category $\C$ with monoidal product $\tensor$ and tensor unit $I$, associators $\assoc_{A,B,C}\from(A\tensor B)\tensor C\toisom A\tensor(B\tensor C)$, unitors $\runit_A\from A\tensor I\toisom A$ and $\lunit_A\from I\tensor A\toisom A$ and braiding $\sym_{A,B}\from A\tensor B\to B\tensor A$.
    \item A category $\C_s$
    \item A right monoidal category action of $\C$ on the category $\C_s$.  That is, a functor
      \[
        \blank\sequoid\blank\from\C_s\times\C\to\C_s
        \]
      together natural isomorphisms
      \[
        \passoc_{A,B,C}\from A\sequoid(B\tensor C)\to (A\sequoid B)\sequoid C
        \]
      and
      \[
        \run_A\from A\sequoid I\toisom A
        \]
      subject to the following coherence conditions:
      \begin{equation*}
        \begin{tikzcd}
          A\sequoid(B\tensor(C\tensor D)) \arrow[r, "\passoc_{A,B,C\tensor D}" yshift=0.3em] \arrow[d, "\id_A\sequoid \assoc_{B,C,D}"']
            & (A \sequoid B) \sequoid (C\tensor D) \arrow[r, "\passoc_{A\sequoid B,C,D}" yshift=0.3em]
              & ((A\sequoid B)\sequoid C) \sequoid D \\
          A\sequoid((B\tensor C)\tensor D) \arrow[r, "\passoc_{A,B\tensor C,D}"' yshift=-0.3em]
            & (A\sequoid (B\tensor C)) \sequoid D \arrow[ur, "\passoc_{A,B,C}\sequoid\id_D"']
              &
        \end{tikzcd}
      \end{equation*}
      \begin{equation*}
        \begin{tikzcd}
          A \sequoid (I\tensor B) \arrow[r, "\passoc_{A,I,B}" yshift=0.3em] \arrow[d, "\id_A\sequoid\lunit_B"']
            & (A\sequoid I)\sequoid B \arrow[dl, "\run_A\sequoid\id_B"] \\
          A \sequoid B
            &
        \end{tikzcd}
        \quad
        \begin{tikzcd}
          A \sequoid (B\tensor I) \arrow[r, "\passoc_{A,B,I}" yshift=0.3em] \arrow[d, "\id_A\sequoid\runit_B"']
            & (A\sequoid B) \sequoid I \arrow[dl, "\run_{A\sequoid B}"] \\
          A\sequoid B
            &
        \end{tikzcd}
      \end{equation*}

      \item A functor $J\from \C_s\to\C$ (in the games example, this is the inclusion functor $\G_s\to\G$)

      \item A natural transformation $\wk_{A,B}\from J(A)\tensor B\to J(A\sequoid B)$ satisfying the coherence conditions:
      \begin{equation*}
        \begin{tikzcd}
          A \tensor I \arrow[r, "\runit_A" yshift=0.3em] \arrow[d, "\wk_{A,I}"']
            & A \\
          A \sequoid I \arrow[ur, "J(\run_A)"']
            &
        \end{tikzcd}
        \quad
        \begin{tikzcd}
          (A \tensor B) \tensor C \arrow[r, "\wk_{A,B}\tensor\id_C" yshift=0.3em] \arrow[d, "\assoc_{A,B,C}"']
            & (A \sequoid B) \tensor C \arrow[r, "\wk_{A\sequoid B, C}" yshift=0.3em]
              & (A \sequoid B) \sequoid C\\
          A \tensor (B \tensor C) \arrow[r, "\wk_{A,B\tensor C}"']
            & A \sequoid (B \tensor C) \arrow[ur, "J(\passoc_{A,B,C})"']
              &
        \end{tikzcd}
      \end{equation*}
  \end{itemize}
\end{definition}

Our category of games satisfies further conditions:

\begin{definition}
  Let $\C=(\C,\C_s,J,\wk)$ be a sequoidal category.  We say that $\C$ is an \emph{inclusive sequoidal category} if $\C_s$ is a full-on-objects subcategoryof $\C$ containing the monoidal isomorphisms and the morphisms $\wk_{A,B}$, $J$ is the inclusion functor and $J$ reflects isomorphisms.

  If $\C$ is an inclusive sequoidal category, we say that $\C$ is \emph{Cartesian} if $\C_s$ has all products and these are preserved by $J$.  In that case, we say that $\C$ is \emph{decomposable} if the natural transformations
  \begin{gather*}
    \dec_{A,B} = \langle \wk_{A,B}, \comp{\wk_{A,B}}{\sym_{A,B}}\rangle\from A\tensor B\to (A\sequoid B)\times (B\sequoid A) \\
    \dec^0 \from I \to 1
  \end{gather*}
  are isomorphisms and we say that $\C$ is \emph{distributive} if the natural transformations
  \begin{gather*}
    \dist_{A,B,C} = \langle \pr_1\sequoid \id_C,\pr_2\sequoid\id_C\rangle\from (A\times B)\sequoid C\to (A\sequoid C)\times (B\sequoid C) \\
    \dist_{A,0}\from 1\sequoid A\to 1
  \end{gather*}
  are isomorphisms.
\end{definition}

We have one further piece of structure available to us:

\begin{definition}
  Let $\C=(\C,\C_s,J,\wk)$ be an inclusive sequoidal category.  We say that $\C$ is a \emph{sequoidal closed category} if $\C$ is monoidal closed (with internal hom $\implies$ and currying $\Lambda_{A,B,C}\from \C(A\tensor B,C)\toisom \C(A,B\implies C)$) and if the map $f\mapsto\Lambda(\comp{f}{\wk})$ gives rise to a natural transformation
  \[
    \Lambda_{A,B,C,s}\from \C_s(A\sequoid B, C) \to \C_s(A,B\implies C)
    \]
\end{definition}

It can be shown (see for example \cite{martinsthesis}) that our category $\G$ of games has all this structure.

\begin{theorem}
  If $A,B$ are games, let $J$ be the inclusion functor $\G_s\to\G$ and let $\wk_{A,B}\from A\tensor B\to A\sequoid B$ be the natural copycat strategy.  Then
  \[
    (\G,\G_s,J,\wk)
    \]
  is an inclusive, Cartesian, decomposable, distributive sequoidal closed category.
\end{theorem}

\subsection{The sequoidal exponential}

There are several ways to add exponentials to the basic category of games.  We shall use the definition based on countably many copies of the base game (see \cite{laird02}, for example):

\begin{definition}
  Let $A$ be a negative game.  The \emph{exponential} of $A$ is the game $\oc A = (M_{\oc A}, \lambda_{\oc A}, b_{\oc A}, P_{\oc A})$, where $M_{\oc A}, \lambda_{\oc A}, b_{\oc A}, P_{\oc A}$ are defined as follows:
  \begin{itemize}
    \item $M_{\oc A} = M_A \times \omega$
    \item $\lambda_{\oc A} = \lambda_A \circ \pr_1$
    \item $b_{\oc A} = O$
    \item Given a sequence $s\in M_{\oc A}^\omega$, we write $s\vert_n$ for the largest sequence $a_1a_2\dots a_k\in M_A^*$ such that $(a_1,n),(a_2,n),\dots(a_k,n)$ is a subsequence of $s$.  Then $P_{\oc A}$ is the set of all sequence $s\in M_{\oc A}^\omega$ that are alternating with respect to $\lambda_{\oc A}$, such that $s\vert_n\in P_A$ for all $n$ and such that if $m<n$ and $(a,n)$ occurs in $s$ then $(b,m)$ must occur earlier in $s$ for some move $b$: in other words, player $O$ can start infinitely many copies of the game $A$, but he must start them in order.
  \end{itemize}
\end{definition}

This last condition on the order in which games may be opened is very important, as it allows us to define morphisms that give $\oc A$ the semantics of the exponential from linear logic.  For example, we have a natural morphism $\mu\from\oc A\to\oc A\tensor\oc A$, given by the copycat strategy that starts a new copy of $A$ on the left whenever one is started on the right.  Because of the condition on the order in which copies of $A$ may be started, there is a unique way to do this.

\begin{proposition}
  $\mu$ exhibits $\oc A$ as a comonoid in the monoidal category $(\G, \tensor, I)$.  
\end{proposition}
\begin{proof}
  $\mu$ shall be the comultiplication in our comonoid.  The counit is given by the empty strategy $\eta\from\oc A\to I$.  We just need to check that $\mu$ is associative and that $\eta$ is a counit for $\mu$.  

  For associativity, we need to show that the following diagram commutes:
  \[
    \begin{tikzcd}
      \oc A \arrow[rr, "\mu"] \arrow[d, "\mu"']
        &
          & \oc A \tensor \oc A \arrow[d, "\id_{\oc A}\tensor\mu"] \\
      \oc A \tensor \oc A \arrow[r, "\mu\tensor\id_{\oc A}" yshift=0.2em]
        & (\oc A \tensor \oc A) \tensor \oc A \arrow[r, "\assoc_{\oc A,\oc A,\oc A}"' yshift=-0.4em]
          & \oc A \tensor (\oc A \tensor \oc A)
    \end{tikzcd}
    \]
  This is easy to see when we notice that both branches of the square are copycat strategies on $\oc A\implies\oc A\tensor(\oc A\tensor \oc A)$; since copies of $A$ in $\oc A$ must be started in sequence, there is a unique such strategy, and so the square commutes.

  For the counit, we need to show that the follosing two diagrams commute:
  \[
    \begin{tikzcd}
      \oc A \arrow[r, "\mu"] \arrow[dr, "\runit_A\inv"']
        & \oc A \tensor \oc A \arrow[d, "\id_{\oc A}\tensor \eta"] \\
      %
        & \oc A \tensor I
    \end{tikzcd}
    \quad\quad\quad
    \begin{tikzcd}
      \oc A \arrow[r, "\mu"] \arrow[dr, "\lunit_A\inv"']
        & \oc A \tensor \oc A \arrow[d, "\eta\tensor\id_{\oc A}"] \\
      %
        & I \tensor \oc A
    \end{tikzcd}
    \]
  Once again, these diagrams commute because both branches are copycat strategies for $\oc A\implies\oc A\tensor I$ or $\oc A\implies I\tensor\oc A$ and there is a unique such strategy in each case.
\end{proof}

We shall later show that $(\oc A, \mu, \eta)$ is in fact the \emph{cofree commutative comonoid} on $A$ in the monoidal category $(\G, \tensor, I)$.  

We shall call the exponential $\oc A$ the \emph{sequoidal exponential}.  The following proposition explains the name:

\begin{proposition}
  Let $A$ be a negative game.  Then we get an endofunctor $A\sequoid\blank$ on $\G$ given by sending $B$ to $A\sequoid B$.  

  The sequoidal exponential $\oc A$, together with the obvious copycat strategy $\alpha\from\oc A\to A\sequoid\oc A$, is the final coalgebra for the endofunctor $A\sequoid\blank$.  In other words, if $B$ is a negative game and $\sigma\from B\to A\sequoid B$ is a morphism then there is a unique morphism $\fcoal{\sigma}\from B\to \oc A$ such that the following diagram commutes:
  \[
    \begin{tikzcd}
      B \arrow[r, "\sigma"] \arrow[d, "\fcoal{\sigma}"']
        & A \sequoid B \arrow[d, "\id_A\sequoid\fcoal{\sigma}"] \\
      \oc A \arrow[r, "\alpha"']
        & A \sequoid \oc A
    \end{tikzcd}
    \]
\end{proposition}
\begin{proof}
  See \cite{martinsthesis}.  We shall shortly give a proof in the more general case.
\end{proof}

\subsection{Constructing cofree commutative comonoids in sequoidal categories}

We want to deduce the fact that $\oc A\xrightarrow{\mu}\oc A\tensor\oc A$ is the cofree commutative comonoid on $A$ from the fact that $\oc A$ is the final coalgebra for $A\sequoid\blank$.  It turns out that we shall need one more fact about the category of games to prove this.

\begin{notation}
  We shall sometimes make the monoidal structure of the Cartesian product explicit by writing $\sigma\times\tau$ for $\langle\comp\sigma{\pr_1},\comp\tau{\pr_2}\rangle$.
\end{notation}

\begin{definition}
  Let $A,B$ be objects of an inclusive, Cartesian, distributive sequoidal category $(\C,\C_s,J,\wk)$ with final coalgebras $\oc A\xrightarrow{\alpha_A} A\sequoid\oc A$ for all endofunctors of the form $A\sequoid\blank$.  Let $A,B$ be objects of $C$.  Then we have a composite $\kappa_{A,B}\from\oc A \tensor\oc B\to(A\times B)\sequoid(\oc A\tensor \oc B)$:
  \settowidth{\arrow}{\scriptsize$\id_{A\sequoid(\oc A\tensor\oc B)}\times (\id_B\sequoid\sym_{\oc B,\oc A})$}
  \begin{IEEEeqnarray*}{RCL}
    \kappa_{A,B}=\oc A \tensor \oc B & \constantwidthxrightarrow{\langle\id_{\oc A\tensor\oc B},\; \sym_{\oc A,\oc B}\rangle} & (\oc A \tensor \oc B) \times (\oc B \tensor \oc A) \\
    \cdots & \constantwidthxrightarrow{(\alpha_A\tensor\id_{\oc B})\times(\alpha_B\tensor\id_{\oc A})} & ((A \sequoid \oc A) \tensor \oc B) \times ((B \sequoid \oc B) \tensor \oc A) \\
    \cdots  & \constantwidthxrightarrow{\wk_{A\sequoid\oc A,\oc B}\times\wk_{B\sequoid\oc B,\oc A}} & ((A \sequoid \oc A) \sequoid \oc B) \times ((B \sequoid \oc B) \sequoid \oc A) \\
      \cdots & \constantwidthxrightarrow{\passoc_{A,\oc A,\oc B}\inv\times\passoc_{B,\oc B,\oc A}\inv} & (A \sequoid (\oc A \tensor \oc B)) \times (B \sequoid (\oc B \tensor \oc A) \\
      \cdots & \constantwidthxrightarrow{\id_{A\sequoid(\oc A\tensor\oc B)}\times (\id_B\sequoid\sym_{\oc B,\oc A})} & (A \sequoid (\oc A \tensor \oc B)) \times (B \sequoid (\oc A \tensor \oc B))
  \end{IEEEeqnarray*}
inducing a morphism
\[
  \oc A\tensor \oc B \xrightarrow{\kappa_{A,B}} (A \sequoid(\oc A\tensor\oc B)) \times (B \sequoid(\oc A\tensor \oc B)) \xrightarrow{\dist\inv}
    (A\times B)\sequoid(\oc A\tensor\oc B)
    \]
  Remembering that our category has a final coalgebra $\oc(A\times B)$ for the functor $(A\times B)\sequoid\blank$, we write $\int_{A,B}$ for the unique morphism $\oc A\tensor\oc B\to \oc(A\times B)$ making the following diagram commute
  \begin{equation}\label{definitionOfInt}
    \begin{tikzcd}
      \oc A \tensor \oc B \arrow[r, "\kappa_{A,B}"] \arrow[d, "\int_{A,B}"']
        & (A\sequoid(\oc A \tensor \oc B) \times (B \sequoid (\oc A \tensor \oc B)) \arrow[r, "\dist\inv"]
          & (A \times B) \sequoid (\oc A \tensor \oc B) \arrow[d, "\id_{A\times B}\sequoid\int_{A,B}"] \\
      \oc(A\times B) \arrow[rr, "\alpha_{A\times B}"']
        &
          & (A \times B) \sequoid \oc(A\times B)
    \end{tikzcd}\tag{$\star$}
  \end{equation}
\end{definition}

\begin{proposition}
  In the category of games, the morphism $\int_{A,B}$ is an isomorphism for all negative games $A,B$.
\end{proposition}
\begin{proof}
  Observe that the morphism $\int_{A,B}$ is the copycat strategy on $\oc A\tensor\oc B\implies \oc (A\times B)$ that starts a copy of $A$ on the left whenever a copy of $A$ is started on the right and starts a copy of $B$ on the left whenever a copy of $B$ is started on the right (indeed, the morphisms in the diagram above are all copycat morphisms, so the copycat strategy we have just described must make that diagram commute.  Since there are infinitely many copies of both $A$ and $B$ available in $\oc (A\times B)$, and since a new copy of $A$ or $B$ may be started at any time, we may define an inverse copycat strategy on $\oc(A\times B)\implies \oc A\tensor\oc B$.
\end{proof}

Our first main result for this section will be the following:
\begin{theorem}
  \label{StrongMonoidalFunctor}
  Let $(\C,\C_s,J,\wk)$ be an inclusive, Cartesian, distributive and decomposable sequoidal category with a final coalgebra $\oc A\xrightarrow{\alpha_A}A\sequoid\oc A$ for each endofunctor of the form $A\sequoid\blank$.  Suppose further that the morphism $\int_{A,B}$ as defined above is an isomorphism for all objects $A,B$.  $A\mapsto\oc A$ gives rise to a strong symmetric monoidal functor from the monoidal category $(\C, \times, 1)$ to the monoidal category $(\C, \tensor, I)$.  
\end{theorem}

We start off by defining a morphism $\mu\from\oc A\to\oc A\tensor\oc A$.  This will turn out to be the comultiplication for the cofree commutative comonoid over $A$.  First, we note that we have the following composite:
\[
  \oc A \xrightarrow{\alpha_A} A \sequoid \oc A \xrightarrow{\Delta} (A\sequoid\oc A)\times (A\sequoid\oc A)\xrightarrow{\dist\inv} (A\times A)\sequoid\oc A
  \]
where $\Delta$ is the diagonal map on the product.  There is therefore a unique morphism $\sigma_A=\fcoal{\comp{\dist\inv}{\comp{\Delta}{\alpha_A}}}$ making the following diagram commute:
\begin{equation}\label{definitionOfSigma}
  \begin{tikzcd}
    \oc A \arrow[r, "\alpha_A"] \arrow[d, "\sigma_A"']
      & A \sequoid \oc A \arrow[r, "\Delta"]
        & (A \sequoid \oc A)\times(A \sequoid \oc A) \arrow[r, "\dist\inv"]
          & (A \times A) \sequoid \oc A \arrow[d, "\id_{A\times A}\sequoid\sigma_A"] \\
    \oc(A\times A) \arrow[rrr, "\alpha_{A\times A}"]
      &
        &
          & (A \times A) \sequoid \oc(A \times A)
  \end{tikzcd}\tag{\dag}
\end{equation}
and we may set $\mu_A=\comp{\int_{A,A}\inv}{\sigma_A}$.

We also define a morphism $\der_A\from \oc A \to A$.  Note that since $I$ is isomorphic to $1$, we have a unique morphism $*_A\from A\to I$ for each $A$.  We define $\der_A$ to be the composite
\[
  \begin{tikzcd}
    \oc A \arrow[r, "\alpha_A"]
      & A \sequoid \oc A \arrow[r, "\id_A \sequoid *_{\oc A}" yshift=0.3em]
        & A \sequoid I \arrow[r, "\run_A"]
          & A
  \end{tikzcd}
  \]

We define the action of $\oc$ on morphisms as follows: suppose that $\sigma\from A \to B$ is a morphism in $\C$.  Then we have a composite
\[
  \begin{tikzcd}
    \oc A \arrow[r, "\mu"]
      & \oc A \tensor \oc A \arrow[r, "\der_A\tensor\id_{\oc A}"]
        & A \tensor \oc A \arrow[r, "\sigma\tensor\id_{\oc A}"]
          & B \tensor \oc A \arrow[r, "\wk_{B,\oc A}"]
            & B \sequoid \oc A
  \end{tikzcd}
  \]
There is therefore a unique morphism $\oc\sigma\from\oc A \to\oc B$ making the following diagram commute:
\[
  \begin{tikzcd}
    \oc A \arrow[r, "\mu"] \arrow[d, "\oc\sigma"']
      & \oc A \tensor \oc A \arrow[r, "\der_A\tensor\id_{\oc A}"]
        & A \tensor \oc A \arrow[r, "\sigma\tensor\id_{\oc A}"]
          & B \tensor \oc A \arrow[r, "\wk_{B,\oc A}"]
            & B \sequoid \oc A \arrow[d, "\id_B\sequoid\oc\sigma"] \\
    \oc B \arrow[rrrr, "\alpha_B"]
      &
        &
          &
            & B \sequoid \oc B
  \end{tikzcd}
  \]

\begin{proposition}\label{StrongMonoidalFunctorActual}
  $\sigma\mapsto\oc\sigma$ respects composition, so $\oc$ is a functor.  Moreover, $\oc$ is a strong symmetric monoidal functor from the Cartesian category $(\C, \times, 1)$ to the symmetric monoidal category $(\C, \tensor, I)$, witnessed by $\int$ and $\dec^0$.
\end{proposition}
\begin{proof}
  See Appendix.
\end{proof}

This completes the proof of Theorem \ref{StrongMonoidalFunctor}.

Since $\oc$ is a strong monoidal functor, it induces a functor $\CCom(\oc)$ from the category $\CCom(\C,\times,1)$ of comonoids over $(\C, \times, 1)$ to the category $\CCom(\C,\tensor,I)$ of comonoids over $(\C, \tensor, I)$ making the following diagram commute:
\[
  \begin{tikzcd}
    \CCom(\C,\times,1) \arrow[r, "\F"] \arrow[d, "\CCom(\oc)"']
      & (\C, \times, 1) \arrow[d, "\oc"] \\
    \CCom(\C, \tensor, I) \arrow[r, "\F"]
      & (\C, \tensor, I)
  \end{tikzcd}
  \]
where $\F$ is the forgetful functor.

Let $A$ be an object of $\C$.  Since $(\C, \times, 1)$ is Cartesian, the diagonal map $\Delta\from A\to A\times A$ is the cofree commutative comonoid over $A$ in $(\C,\times,1)$.  

\begin{proposition}
  $\CCom(\oc)\left(A\xrightarrow{\Delta}A\times A\right) = \oc A \xrightarrow{\mu_A} \oc A \tensor \oc A$
\end{proposition}
\begin{proof}
  By the (standard) construction of the functor $\CCom(\oc)$, the comultiplication in $\CCom(\oc)\left(A\xrightarrow{\Delta}A\times A\right)$ is given by the composite
  \[
    \begin{tikzcd}
      \oc A \arrow[r, "\oc\Delta"]
        & \oc (A\times A) \arrow[r, "\int_{A,A}\inv"]
          & \oc A \tensor \oc A
    \end{tikzcd}
    \]
\end{proof}

In order to show that $\sigma\mapsto\oc\sigma$ respects composition, we need the following lemma:

\begin{lemma}\label{aFormulaForAlpha}
  Let $A$ be an object of $\C$.  Then $\alpha_A\from\oc A\to A\sequoid\oc A$ is equal to the following composite:
  \[
    \begin{tikzcd}
      \oc A \arrow[r, "\mu_A"]
        & \oc A \tensor \oc A \arrow[r, "\der_A\tensor\id_{\oc A}"]
          & A \tensor \oc A \arrow[r, "\wk_{A,\oc A}"]
            & A \sequoid \oc A
    \end{tikzcd}
    \]
\end{lemma}

\begin{proof}
  We may paste together diagrams \eqref{definitionOfInt} and \eqref{definitionOfSigma} to form the following diagram (where we shall omit subscripts where there is no ambiguity):
  \begin{equation*}
    \begin{tikzcd}[cramped]
      \oc A \arrow[r, "\alpha"] \arrow[d, "\sigma_A"']
        & A \sequoid \oc A \arrow[r, "\Delta"]
          &[-24pt] (A \sequoid \oc A) \times (A \sequoid \oc A) \arrow[r, "\dist\inv"]
            & (A \times A) \sequoid \oc A \arrow[d, "\id_{A\times A}\sequoid\sigma_A"] \\
      \oc(A\times A) \arrow[rrr, "\alpha"]
        &
          &
            & (A \times A) \sequoid \oc(A\times A) \\
      \oc A \tensor \oc A \arrow[rr, "\kappa_{A,A}"] \arrow[u, "\int_A"]
        &
          & (A \sequoid (\oc A \tensor \oc A)) \times (A \sequoid (\oc A \tensor \oc A)) \arrow[r, "\dist\inv"]
            & (A\times A)\sequoid(\oc A \tensor \oc A) \arrow[u, "\id_{A\times A}\sequoid\int_A"']
    \end{tikzcd}
  \end{equation*}
  where we observe that the composites down the left and right hand sides (after inverting the lower arrows) are $\mu_A$ and $\id_{A\times A}\sequoid\mu_A$.

  Now note that we have the following commutative square:
  \[
    \begin{tikzcd}
      (A \times A) \sequoid \oc A \arrow[r, "\dist"] \arrow[d, "\id_{A\times A}\sequoid\mu_A"']
        & (A \sequoid \oc A) \times (A \sequoid \oc A) \arrow[d, "(\id\sequoid\mu)\times(\id\sequoid\mu)"] \\
      (A \times A) \sequoid (\oc A \tensor \oc A) \arrow[r, "\dist"']
        & (A \sequoid (\oc A \tensor \oc A)) \times (A \sequoid (\oc A \tensor \oc A)
    \end{tikzcd}
    \]

  (using the definition of $\dist$).  Putting this together with the diagram above, we get the following commutative diagram:
  \[
    \begin{tikzcd}
      \oc A \arrow[r, "\alpha"] \arrow[d, "\mu_A"']
        & A \sequoid \oc A \arrow[r, "\Delta"]
          & (A \sequoid \oc A) \times (A \sequoid \oc A) \arrow[d, "\id\sequoid\mu_A\times\id\sequoid\mu_A"] \\
      \oc A \tensor \oc A \arrow[rr, "\kappa_{A,A}"]
        & \cdots \arrow[r]
          & (A \sequoid (\oc A \tensor \oc A)) \times (A \sequoid (\oc A \tensor \oc A))
    \end{tikzcd}
    \]
  We now expand the definition of $\kappa_{A,A}$  and take the projections on to the first and second components, yielding the following two commutative diagrams:
  \[
    \begin{tikzcd}\label{firstProjectionDiagram}
      \oc A \arrow[rrr, "\alpha"] \arrow[d, "\mu_A"']
        &
          &
            & A \sequoid \oc A \arrow[d, "\id\sequoid\mu_A"] \\
      \oc A \tensor \oc A \arrow[r, "\alpha\tensor\id"]
        & (A \sequoid \oc A) \tensor \oc A \arrow[r, "\wk"]
          & (A \sequoid \oc A) \sequoid \oc A \arrow[r, "\passoc\inv"]
            & A \sequoid (\oc A \tensor \oc A)
    \end{tikzcd}\tag{\textbf{1}}
    \]

  \begin{equation*}\label{secondProjectionDiagram}
    \begin{tikzcd}[cramped]
     \oc A \arrow[rrrr, "\alpha"] \arrow[d, "\mu_A"']
        &
          &
            &
              & A \sequoid \oc A \arrow[d, "\id\sequoid\mu_A"] \\
      \oc A \tensor \oc A \arrow[r, "\sym"']
        & \oc A \tensor \oc A \arrow[r, "\alpha\tensor\id" yshift=0.4em]
          & (A \sequoid \oc A) \tensor \oc A \arrow[r, "\comp{\passoc\inv}{\wk}" yshift=0.4em]
            & A \sequoid (\oc A \tensor \oc A) \arrow[r, "\id\sequoid\sym"']
              & A \sequoid (\oc A \tensor \oc A)
    \end{tikzcd}\tag{\textbf{2}}
  \end{equation*}

  From diagram \eqref{firstProjectionDiagram}, we construct the following commutative diagram:
  \[
    \begin{tikzcd}
      \oc A \arrow[rrr, "\alpha"] \arrow[d, rightsquigarrow, "\mu_A"'] \arrow[phantom, drrr, "\textbf{a}" xshift=0.9em]
        &
          &
            & A \sequoid \oc A \arrow[d, "\id\sequoid\mu_A"] \\
      \oc A \tensor \oc A \arrow[r, "\alpha\tensor\id"] \arrow[ddr, rightsquigarrow, "\der_A\tensor\id"'] \arrow[phantom, dr, "\textbf{b}" xshift=0.6em]
        & (A \sequoid \oc A) \tensor \oc A \arrow[r, "\wk"] \arrow[d, "(\id\sequoid*)\tensor\id"] \arrow[phantom, dr, "\textbf{c}" xshift=0.3em]
          & (A \sequoid \oc A) \sequoid \oc A \arrow[r, "\passoc\inv"] \arrow[d, "(\id\sequoid*)\sequoid\id"] \arrow[phantom, dr, "\textbf{e}" xshift=0.3em]
            & A \sequoid (\oc A \sequoid \oc A) \arrow[d, "\id\sequoid(*\tensor\id)"] \\
      %
        & (A \sequoid I) \tensor \oc A \arrow[r, "\wk"] \arrow[d, "\run\tensor\id"] \arrow[phantom, dr, "\textbf{d}" xshift=0.3em]
          & (A \sequoid I) \sequoid \oc A \arrow[r, "\passoc\inv"], \arrow[d, "\run\tensor\id"] \arrow[phantom, r, "\textbf{f}" xshift=-0.6em, yshift=-1.0em]
            & A \sequoid (I \tensor \oc A) \arrow[dl, "\id\sequoid\lunit"] \\
      %
        & A \tensor \oc A \arrow[r, rightsquigarrow, "\wk"]
          & A \sequoid \oc A
            &
    \end{tikzcd}
    \]

  \textbf{a} is diagram \eqref{firstProjectionDiagram}.

  \textbf{b} commutes by the definition of $\der_A$.

  \textbf{c} and \textbf{d} commute because $\wk$ is a natural transformation.

  \textbf{e} commutes because $\passoc$ is a natural transformation.

  \textbf{f} commutes by one of the coherence conditions in the definition of a sequoidal category.

  We now observe that the composite of the three squiggly arrows is the composite we are trying to show is equal to $\alpha$; we have $\alpha$ along the top, so it will suffice to show that the composite
  \[
    \begin{tikzcd}
      \xi_A\;=\;\oc A \arrow[r, "\mu_A"]
        & \oc A \tensor \oc A \arrow[r, "*\tensor\id"]
          & I \tensor \oc A \arrow[r, "\lunit"]
            & \oc A
    \end{tikzcd}
    \]
  is equal to the identity.  We do this using diagram \eqref{secondProjectionDiagram}.  First we construct the diagram shown in Figure \ref{hugeDiagram1}.

  \begin{SidewaysFigure}
    \[
      \begin{tikzcd}[ampersand replacement=\&]
        \oc A \arrow[rrrrr, "\alpha"] \arrow[d, "\mu_A"'] \arrow[phantom, drrrrr, "\textbf{a}"]
          \&
            \&
              \&
                \& 
                  \& A \sequoid \oc A \arrow[d, "\id\sequoid\mu_A"] \\
        \oc A \tensor \oc A  \arrow[r, "\sym"] \arrow[d, "*\tensor\id"'] \arrow[phantom, dr, "\textbf{b}"]
          \& \oc A \tensor \oc A \arrow[r, "\alpha\tensor\id"] \arrow[d, "\id\tensor*"]
            \& (A \sequoid \oc A) \tensor \oc A \arrow[r, "\wk"] \arrow[d, "\id\sequoid*"] \arrow[phantom, dr, "\textbf{d}"]
              \& (A \sequoid \oc A) \sequoid \oc A \arrow[r, "\passoc\inv"] \arrow[d, "\id\sequoid*"] \arrow[phantom, dr, "\textbf{e}"]
                \& A \sequoid (\oc A \tensor \oc A) \arrow[r, "\id\sequoid\sym"] \arrow[d, "\id\sequoid(\id\sequoid*)"] \arrow[phantom, dr, "\textbf{c}"]
                  \& A \sequoid (\oc A \tensor \oc A) \arrow[d, "\id\sequoid (*\sequoid\id)"] \\
        I \tensor \oc A \arrow[r, "\sym"] \arrow[d, "\lunit"'] \arrow[phantom, dr, "\textbf{g}"]
          \& \oc A \tensor I \arrow[r, "\alpha\tensor\id"] \arrow[d, "\runit"] \arrow[phantom, dr, "\textbf{f}"]
            \& (A \sequoid \oc A) \tensor I \arrow[r, "\wk"] \arrow[d, "\runit"] \arrow[phantom, dr, "\textbf{i}"]
              \& (A \sequoid \oc A) \sequoid I \arrow[r, "\passoc\inv"] \arrow[d, "\run"] \arrow[phantom, dr, "\textbf{j}"]
                \& A \sequoid (\oc A \tensor I) \arrow[r, "\id\sequoid\sym"] \arrow[d, "\id\sequoid\runit"] \arrow[phantom, dr, "\textbf{h}"]
                  \& A \sequoid (I \tensor \oc A) \arrow[d, "\id\sequoid\lunit"] \\
        \oc A \arrow[r, "\id"']
          \& \oc A \arrow[r, "\alpha"]
            \& A \sequoid \oc A \arrow[r, "\id"']
              \& A \sequoid \oc A \arrow[r, "\id"']
                \& A \sequoid \oc A \arrow[r, "\id"']
                  \& A \sequoid \oc A
      \end{tikzcd}
      \]
    \caption{
      \textbf{a} is diagram \eqref{secondProjectionDiagram}. \\[\baselineskip]
      \textbf{b} and \textbf{c} commute because $\sym$ is a natural transformation, \textbf{d} commutes because $\wk$ is a natural transformation and \textbf{e} commutes because $\passoc$ is a natural transformation.  \textbf{f} commutes because $\runit$ is a natural transformation. \\[\baselineskip]
      \textbf{g} and \textbf{h} commute by one of the coherence conditions for a symmetric monoidal category.  \textbf{i} commutes by one of the coherence conditions for $\wk$ in the definition of a sequoidal category and \textbf{j} commutes by one of the coherence conditions for $\passoc$ in the definition of a sequoidal category.
    }\label{hugeDiagram1}
  \end{SidewaysFigure}

  Now observe that the composite $\xi_A$ is running along the left hand side of Figure \ref{hugeDiagram1}, while $\id\sequoid\xi$ is running along the right.  Since we have $\alpha$ along the bottom, it follows by the uniqueness of $\fcoal{\cdot}$ that $\xi=\fcoal{\alpha}=\id_{\oc A}$.  
\end{proof}

Now we are ready to show that $\sigma\mapsto\oc \sigma$ respects composition.  Let $A,B,C$ be objects, let $\sigma$ be a morphism from $A$ to $B$ and let $\tau$ be a morphism from $B$ to $C$.   Using Lemma \ref{aFormulaForAlpha} and the definition of $\oc\sigma$, $\oc\tau$, we may construct a commutative diagram:
\[
  \begin{tikzcd}
    \oc A \arrow[r, "\mu"] \arrow[d, "\oc\sigma"']
      & \oc A \tensor \oc A \arrow[r, "\der\tensor\id"]
        & A \tensor \oc A \arrow[r, "\sigma\tensor\id"]
          & B \tensor \oc A \arrow[r, "\wk"] \arrow[d, dotted, "\id\tensor\oc\sigma"]
            & B \sequoid \oc A \arrow[d, "\id\sequoid\oc\sigma"] \\
    \oc B \arrow[r, "\mu"] \arrow[dd, "\oc \tau"']
      & \oc B \tensor \oc B \arrow[rr, "\der\tensor\id"]
        &
          & B \tensor \oc B \arrow[r, "\wk"] \arrow[d, "\tau\tensor\id"]
            & B \sequoid \oc B \\
    %
      &
        &
          & C \tensor \oc B \arrow[r, "\wk"] \arrow[d, dotted, "\id\tensor\oc\tau"']
            & C \sequoid \oc B \arrow[d, "\id\sequoid\oc\tau"] \\
    \oc C \arrow[r, "\mu"]
      & \oc C \tensor \oc C \arrow[rr, "\der\tensor\id"]
        &
          & C \tensor \oc C \arrow[r, "\wk"]
            & C \sequoid \oc C
  \end{tikzcd}
  \]
Here, the outermost (solid) shapes commute by the definition of $\oc\sigma$, $\oc\tau$ (after we have replaced $\alpha_B$, $\alpha_C$ with the composite from Lemma \ref{aFormulaForAlpha}).  The smaller squares on the right hand side commute because $\wk$ is a natural transformation.  Now observe that $\wk_{X,Y}=\comp{\pr_1}{\dec_{X,Y}}$ is the composition of epimorphisms, so is an epimorphism for all $X,Y$.  It follows that the two rectangles on the left commute as well.

Throwing away the right hand squares and adding some new arrows at the right, we arrive at the following commutative diagram:
\[
  \begin{tikzcd}
    \oc A \arrow[r, "\comp{(\sigma\tensor\id)}{\comp{(\der\tensor\id)}\mu}"] \arrow[d, "\oc\sigma"']
      &[72pt] B \tensor \oc A \arrow[r, "\tau\tensor\id"] \arrow[d, "\id\tensor\oc\sigma"']
        & C \tensor \oc A \arrow[d, "\id\tensor\oc\sigma"] \\
    \oc B \arrow[d, "\oc\tau"']
      & B \tensor\oc B \arrow[r, "\tau\tensor\id"] \arrow[d, "\tau\tensor\oc\tau"']
        & C \tensor \oc B \arrow[dl, "\id\tensor\oc\tau"] \\
    \oc C \arrow[r, "\comp{(\der\tensor\id)}{\mu}"]
      & C \tensor \oc C
        &
  \end{tikzcd}
  \]
We have just shown that the square on the left commutes.  The shapes on the right commute by inspection.  We now throw away the internal arrows and re-apply $\wk$ on the right hand side:
\[
  \begin{tikzcd}
    \oc A \arrow[r, "\comp{((\comp\tau\sigma)\tensor\id)}{\comp{(\der\tensor\id)}\mu}"] \arrow[d, "\oc\sigma"']
      &[72pt] C \tensor \oc A \arrow[d, "\id\tensor\oc\sigma"'] \arrow[r, "\wk"]
        & C \sequoid \oc A \arrow[d, "\id\sequoid\oc\sigma"] \\
    \oc B \arrow[d, "\oc\tau"']
      & C \tensor \oc B \arrow[d, "\id\tensor\oc\tau"'] \arrow[r, "\wk"]
        & \oc C \sequoid \oc B \arrow[d, "\id\sequoid\oc\tau"] \\
    \oc C \arrow[r, "\comp{(\der\tensor\id)}{\mu}"]
      & C \tensor \oc C \arrow[r, "\wk"]
        & C \sequoid\oc C
  \end{tikzcd}
  \]
By Lemma \ref{aFormulaForAlpha}, the composite along the bottom is equal to $\alpha_C$.  Therefore, by uniqueness of $\fcoal{\cdot}$, we have
\[
  \comp{\oc\tau}{\oc\sigma} = \fcoal{\comp{\wk}{\comp{((\comp\tau\sigma)\tensor\id)}{\comp{(\der\tensor\id)}\mu}}} = \oc(\comp\tau\sigma)
  \]
Therefore, $\oc$ is indeed a functor.

We now want to show that $\oc$ has the structure of a strong symmetric monoidal functor from $(\C,\times,1)$ to $(\C,\tensor,I)$.  The relevant morphisms are:
\[
  \int_{A,B}\from \oc A \tensor \oc B \to \oc (A \times B)
  \quad
  \dec^0\from I\to 1
  \]
By hypothesis, these are both isomorphisms.  We just need to show that the appropriate coherence diagrams commute.

\begin{theorem}
  \label{Coalgebra__CoCoCo}
  Let $(\C,\C_s,J,\wk)$ be a sequoidal category satisfying all the conditions from Theorem \ref{StrongMonoidalFunctor}.  Let $A$ be an object of $\C$ (equivalently, of $\C_s$).  Then $\oc A$ is the carrier for the cofree commutative comonoid over $A$.
\end{theorem}

%%
%% Bibliography
%%

%% Either use bibtex (recommended), 

\bibliography{../../../common/phd_bibliography}

%% .. or use the thebibliography environment explicitely

\end{document}
