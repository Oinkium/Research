\documentclass[11pt]{article}

\usepackage[utf8]{inputenc}

\usepackage{graphicx} % support the \includegraphics command and options

\usepackage{parskip} % Activate to begin paragraphs with an empty line rather than an indent

%%% PACKAGES
\usepackage{booktabs} % for much better looking tables
\usepackage{array} % for better arrays (eg matrices) in maths
\ifdefined\BEAMER
\else
\usepackage{paralist} % very flexible & customisable lists (eg. enumerate/itemize, etc.)\prefix\t$.
\fi
\usepackage{verbatim} % adds environment for commenting out blocks of text & for better verbatim
\ifdefined\BEAMER
\else
\ifdefined\THESIS
\usepackage{subcaption}
\else
\usepackage{subfig} % make it possible to include more than one captioned figure/table in a single float
\fi
\fi
\usepackage{mathtools} % for the all important \coloneqq symbol
\usepackage{hyperref} % for hyperreferences
\usepackage{IEEEtrantools} % for \IEEEeqnarray
\usepackage{pbox} % for \pbox
\usepackage{multirow,bigdelim} % for \multirow
\usepackage{lettrine} % For the drop cap
\usepackage{mathpartir} % for \inferrule, \inferrule* and the mathpar environment
\usepackage{listings}

\usepackage{caption}
\captionsetup{singlelinecheck=off}

\ifdefined\NOTARTICLE
\else

%%% ToC (table of contents) APPEARANCE
\usepackage[nottoc,notlof,notlot]{tocbibind} % Put the bibliography in the ToC
\usepackage[titles,subfigure]{tocloft} % Alter the style of the Table of Contents
\renewcommand{\cftsecfont}{\rmfamily\mdseries\upshape}
\renewcommand{\cftsecpagefont}{\rmfamily\mdseries\upshape} % No bold!

\fi

%% Font things %%
\usepackage{amssymb}
\usepackage{cmll} % Linear logic symbols!
\ifdefined\FEWFONTS
\else
\usepackage{bm} % for bold Greek letters
\fi
\usepackage{stmaryrd}
\usepackage{bbm}

%% Get the sqsubsetneqq character from the mathabx package
\DeclareFontFamily{U}{mathb}{\hyphenchar\font45}
\DeclareFontShape{U}{mathb}{m}{n}{
      <5> <6> <7> <8> <9> <10> gen * mathb
      <10.95> mathb10 <12> <14.4> <17.28> <20.74> <24.88> mathb12
      }{}
\DeclareSymbolFont{mathb}{U}{mathb}{m}{n}

\DeclareMathSymbol{\sqsubsetneq}    {3}{mathb}{"88}
\DeclareMathSymbol{\varsqsubsetneq} {3}{mathb}{"8A}
\DeclareMathSymbol{\varsqsubsetneqq}{3}{mathb}{"92}
\DeclareMathSymbol{\sqsubsetneqq}   {3}{mathb}{"90}

%% Get the left and right moons from the wasysym package

\DeclareFontFamily{U}{wasy}{}
\DeclareFontShape{U}{wasy}{m}{n}{ <5> <6> <7> <8> <9> gen * wasy
      <10> <10.95> <12> <14.4> <17.28> <20.74> <24.88>wasy10  }{}
\DeclareFontShape{U}{wasy}{b}{n}{ <-10> sub * wasy/m/n
 <10> <10.95> <12> <14.4> <17.28> <20.74> <24.88>wasyb10 }{}
\DeclareFontShape{U}{wasy}{bx}{n}{ <-> sub * wasy/b/n}{}

\def\wasyfamily{\fontencoding{U}\fontfamily{wasy}\selectfont}
\def\leftmoon   {\mbox{\wasyfamily\char36}}
\def\rightmoon  {\mbox{\wasyfamily\char37}}

%% Lists %%
\usepackage{enumerate}

%% Graphics %%
\usepackage{tikz}
\usetikzlibrary{cd}
\usetikzlibrary{patterns}
\usetikzlibrary{calc}
\usetikzlibrary{decorations.pathmorphing}
\usetikzlibrary{positioning}

\tikzset{inlinearrows/.style={anchor=base,baseline,x=0.6\baselineskip,y=0.6\baselineskip}}

\ifdefined\BEAMER
\else

%% Theorems! %%
\usepackage{amsthm}
\theoremstyle{plain} % Theorems, lemmas, propositions etc.
\newtheorem{theorem}{Theorem}[section]
\newtheorem{lemma}[theorem]{Lemma}
\newtheorem{proposition}[theorem]{Proposition}
\newtheorem{corollary}[theorem]{Corollary}
\newtheorem{fact}[theorem]{Fact}
\newtheorem{construction}[theorem]{Construction}
\theoremstyle{definition} % Definitions etc.  
\newtheorem{definition}[theorem]{Definition}
\newtheorem{notation}[theorem]{Notation}
\theoremstyle{remark} % Remarks
\newtheorem{remark}[theorem]{Remark}
\newtheorem{remarks}[theorem]{Remarks}
\newtheorem{example}[theorem]{Example}
\newtheorem{question}[theorem]{Question}
\newtheorem{slogan}[theorem]{Slogan}

\newtheoremstyle{note} {3pt} {3pt} {\itshape} {} {\itshape} {:} {.5em} {} % For short notes
\theoremstyle{note}
\newtheorem{note}[theorem]{Note}

\fi

%% Exercises and answers %%
\usepackage{answers}

\newtheoremstyle{exercisestyle}% name
  {6pt}   % ABOVESPACE
  {6pt}   % BELOWSPACE
  {\itshape}  % BODYFONT
  {0pt}       % INDENT (empty value is the same as 0pt)
  {\bfseries} % HEADFONT
  {.}         % HEADPUNCT
  {3pt} % HEADSPACE
  {}          % CUSTOM-HEAD-SPEC

\theoremstyle{exercisestyle}
\newtheorem{exercise}{Exercise}
\newtheorem{answerthm}{Exercise}

\Newassociation{answer}{answerthm}{answers}
\newcommand{\answerthmparams}{}

%% Changes to enumerate things so they look better %%\sigma$

\makeatletter
\def\enumfix{%
\if@inlabel
 \noindent \par\nobreak\vskip-\topsep\hrule\@height\z@
\fi}

\let\olditemize\itemize
\def\itemize{\enumfix\olditemize}
\let\oldenumerate\enumerate
\def\enumerate{\enumfix\oldenumerate}

%% Random crap %%
\usepackage{xifthen}

\makeatletter
\def\thm@space@setup{%
  \thm@preskip=\parskip \thm@postskip=0pt
}
\makeatother

\makeatletter
\newcommand*{\relrelbarsep}{.386ex}
\newcommand*{\relrelbar}{%
  \mathrel{%
    \mathpalette\@relrelbar\relrelbarsep
  }%
}
\newcommand*{\@relrelbar}[2]{%
  \raise#2\hbox to 0pt{$\m@th#1\relbar$\hss}%
  \lower#2\hbox{$\m@th#1\relbar$}%
}
\providecommand*{\rightrightarrowsfill@}{%
  \arrowfill@\relrelbar\relrelbar\rightrightarrows
}
\providecommand*{\leftleftarrowsfill@}{%
  \arrowfill@\leftleftarrows\relrelbar\relrelbar
}
\providecommand*{\xrightrightarrows}[2][]{%
  \ext@arrow 0359\rightrightarrowsfill@{#1}{#2}%
}
\providecommand*{\xleftleftarrows}[2][]{%
  \ext@arrow 3095\leftleftarrowsfill@{#1}{#2}%
}
\makeatother

\newcommand{\catname}[1]{{\normalfont\textbf{#1}}}
\newcommand{\Rings}{\catname{CRing}}
\newcommand{\CAT}{\catname{CAT}}
%\newcommand{\Top}{\catname{Top}}
\newcommand{\Set}{\catname{Set}}
\newcommand{\Cat}{\catname{Cat}}
\newcommand{\MonCat}{\catname{MonCat}}
\newcommand{\SymmMonCat}{\catname{SymmMonCat}}
\newcommand{\Cont}{\catname{Cont}}
\newcommand{\Sch}{\catname{Sch}}
\newcommand{\Rel}{\catname{Rel}}
\newcommand{\Coh}{\catname{Coh}}
\newcommand{\Inj}{\catname{Inj}}
\newcommand{\Dcpo}{\catname{Dcpo}}
\newcommand{\Mod}[1][]{\ifthenelse{\isempty{#1}}{\catname{Mod}}{#1\catname{mod}}}
\DeclareMathOperator{\sh}{Sh}
\newcommand{\Sh}[1][]{\ifthenelse{\isempty{#1}}{\sh}{\sh(#1)}}
\newcommand{\map}[3]{#2\xrightarrow{#1} #3}
\newcommand*\from{\colon}
\newcommand*\bigto{\Rightarrow}
\newcommand{\cmap}[3]{#1\from{}#2\to{}#3}
\newcommand\oppcat[1]{#1^{\mathrm{op}}}
\newcommand{\object}{\colon}
\DeclareRobustCommand{\vmap}[3] {\begin{tikzcd} #2 \arrow[d, "#1"] \\ #3 \end{tikzcd}}
\newcommand{\partref}[1]{(\ref{#1})}
\newcommand{\intgrpd}[4] {#1 \xrightrightarrows[#3]{#4} #2}
\DeclareRobustCommand{\bigintgrpd}[4] {\begin{tikzcd}[ampersand replacement=\&] #1 \arrow[r, shift left=0.5ex, "#3"] \arrow[r, shift right=0.5ex, "#4"'] \& #2 \end{tikzcd}}

\usepackage{xspace}

\newcommand{\etale}{\'{e}tale\xspace}
\newcommand{\Etale}{\'{E}tale\xspace}

\def \inv {^{-1}}

\DeclareMathOperator{\id}{id}
\DeclareMathOperator{\op}{op}
\DeclareMathOperator{\pr}{pr}
\DeclareMathOperator{\inj}{in}
\DeclareMathOperator{\pre}{{pre}}
\DeclareMathOperator{\et}{{\acute{e}t}}

\DeclareMathOperator{\Hom}{Hom}
\DeclareMathOperator{\Spec}{Spec}

\DeclareMathOperator{\ol}{ol}

\def\presuper#1#2%
  {\mathop{}%
   \mathopen{\vphantom{#2}}^{#1}%
   \kern-\scriptspace%
   #2}
\def\presub#1#2%
  {\mathop{}%
   \mathopen{\vphantom{#2}}_{#1}%
   \kern-\scriptspace%
   #2}

\newsavebox{\overlongequation}
\newenvironment{longdiagram}
 {\begin{displaymath}\begin{lrbox}{\overlongequation}$\displaystyle}
 {$\end{lrbox}\makebox[0pt]{\usebox{\overlongequation}}\end{displaymath}}

%% Our things %%

\newcommand{\neggame}[1]{\presuper{\perp}{#1}}
\newcommand{\tensor}{\otimes}
\newcommand{\Tensor}{\bigotimes}
\newcommand{\sequoid}{\oslash}
\newcommand{\varsequoid}{\vartriangleleft}
\renewcommand{\implies}{\multimap}
\newcommand{\iimpl}{\Longrightarrow}
\newcommand{\comp}[2]{#1 \circ #2}
\newcommand{\icomp}[2]{\comp{#1}{#2}}
\newcommand{\cprd}{\sqcup}
\newcommand{\bigcprd}{\bigsqcup}
\newcommand{\G}{\mathcal G}
\newcommand{\W}{\mathcal W}
\newcommand{\suchthat}{\;\colon\;}
\newcommand{\varsuchthat}{\;\mid\;}
\newcommand{\esuchthat}{\;.\;}
\newcommand{\OP}{\{O,P\}}
\newcommand{\QA}{\{Q,A\}}
\renewcommand{\L}{\mathcal L}
\newcommand{\F}{\mathcal F}
\newcommand{\U}{\mathcal U}
\newcommand{\s}{\mathfrak s}
\renewcommand{\t}{\mathfrak t}
\renewcommand{\u}{\mathfrak u}
\renewcommand{\d}{\mathfrak d}
\newcommand{\e}{\mathfrak e}
\newcommand{\emptyplay}{\epsilon}
\newcommand{\bracketed}[1]{\left({#1}\right)}
\newcommand{\bneggame}[1]{{\bracketed{\neggame{#1}}}}
\newcommand{\prefix}{\sqsubseteq}
\newcommand{\ppprefix}{\sqsubset}
\newcommand{\pprefix}{\sqsubsetneqq}
\renewcommand{\ss}{\mathbf{s}}
\newcommand{\bN}{\mathbb{N}}
\newcommand{\bC}{\mathbb{C}}
\newcommand{\bB}{\mathbb{B}}
\newcommand{\bP}{\mathbb{P}}
\newcommand{\pfun}{\rightharpoonup}
\newcommand{\grel}[1]{\underline{#1}}
\DeclareMathOperator{\length}{length}
\renewcommand{\b}{\mathfrak b}
\renewcommand{\r}{\mathfrak r}
\newcommand{\bbeta}{{\bm{\beta}}}
\newcommand{\st}{{\Sigma^*}}
\let\sec\S
\renewcommand{\S}{{\mathfrak{S}}}
\DeclareMathOperator{\cc}{cc}
\DeclareMathOperator{\subs}{subs}
\DeclareMathOperator{\ret}{ret}
\DeclareMathOperator{\zz}{zz}
\newcommand{\aaa}{\mathbf{a}}
\newcommand{\bbb}{\mathbf{b}}
\newcommand{\ccc}{\mathbf{c}}
\newcommand{\ddd}{\mathbf{d}}
\newcommand{\B}{\mathcal B}
\newcommand{\BB}{\mathbf B}
\renewcommand{\H}{\mathcal H}
\DeclareMathOperator{\assoc}{assoc}
\DeclareMathOperator{\lunit}{lunit}
\DeclareMathOperator{\runit}{runit}
\DeclareMathOperator{\dom}{dom}
\DeclareMathOperator{\sym}{sym}
\newcommand{\braid}{\sym}
\newcommand{\blank}{\,\underline{\hspace{1.5ex}}\,}
\DeclareMathOperator{\cn}{cn}
\newcommand{\impliescn}{\protect\overset{\cn}{\implies}}
\newcommand{\C}{{\mathcal{C}}}
\newcommand{\D}{{\mathcal{D}}}
\newcommand{\E}{{\mathcal{E}}}
\newcommand{\V}{{\mathcal{V}}}
\newcommand{\EE}{{\mathbf{E}}}
\DeclareMathOperator{\ev}{ev}
\newcommand{\der}{{\mathtt{der}}}
\newcommand{\mult}{{\mathtt{mult}}}
\DeclareMathOperator{\wk}{wk}
\newcommand{\toisom}{{\xrightarrow{\cong}}}
\DeclareMathOperator{\passoc}{{\mathsf{passoc}}}
\DeclareMathOperator{\pcomm}{{\mathsf{pcomm}}}
\DeclareMathOperator{\run}{{\mathsf{r}}}
\DeclareMathOperator{\lun}{{\mathsf{l}}}
\newcommand{\fcoal}[1]{{\leftmoon #1 \rightmoon}}
\DeclareMathSymbol{\co}{\mathord}{operators}{"3C}
\DeclareMathSymbol{\nw}{\mathord}{operators}{"3E}
\newcommand{\T}{\mathfrak{T}}
\renewcommand{\subset}{\subseteq}
\newcommand{\Ord}{\catname{Ord}}
\newcommand{\FS}{\mathcal{FS}}
\DeclareMathOperator{\rank}{rank}
\DeclareMathOperator{\dist}{{\mathsf{dist}}}
\DeclareMathOperator{\dec}{{\mathsf{dec}}}
\DeclareMathOperator{\str}{str}
\DeclareMathOperator{\weak}{weak}
\DeclareMathOperator{\Strat}{Strat}
\DeclareMathOperator{\OppStrat}{OppStrat}
\newcommand{\seqs}[1]{{\overline{{#1}^{*}}}}
\def\flushRight{\leftskip0pt plus 1fill\rightskip0pt}
\def\Centering{\relax\ifvmode\centering\fi}
\newcommand{\deno}[1]{\left\llbracket#1\right\rrbracket}
\newcommand{\converges}{\Downarrow}
\newcommand{\diverges}{\Uparrow}
\newcommand{\mustconverge}{\converges^{\text{must}}}
\newcommand{\Iflt}{\mathtt{If{<}\;}}
\newcommand{\Ifgt}{\mathtt{If{>}\;}}
\newcommand{\inr}{{\mathsf{inr}}}
\newcommand{\inl}{{\mathsf{inl}}}
\newcommand{{\Na}}{\bN}
\newcommand{{\cell}}{{\mathsf{cell}}}
\newcommand{\fix}{{\mathsf{fix}}}
\newcommand{\eq}{{\mathsf{eq}}}
\DeclareMathOperator{\CCom}{CCom}
\newcommand{\power}{\mathfrak P}

% Slanty things
\newcommand*{\xslant}[2][76]{%
  \begingroup
    \sbox0{#2}%
    \pgfmathsetlengthmacro\wdslant{\the\wd0 + cos(#1)*\the\wd0}%
    \leavevmode
    \hbox to \wdslant{\hss
      \tikz[
        baseline=(X.base),
        inner sep=0pt,
        transform canvas={xslant=cos(#1)},
      ] \node (X) {\usebox0};%
      \hss
      \vrule width 0pt height\ht0 depth\dp0 %
    }%
  \endgroup
}

\makeatletter
\newcommand*{\xslantmath}{}
\def\xslantmath#1#{%
  \@xslantmath{#1}%
}
\newcommand*{\@xslantmath}[2]{%
  % #1: optional argument for \xslant including brackets
  % #2: math symbol
  \ensuremath{%
    \mathpalette{\@@xslantmath{#1}}{#2}%
  }%
}
\newcommand*{\@@xslantmath}[3]{%
  % #1: optional argument for \xslant including brackets
  % #2: math style
  % #3: math symbol
  \xslant#1{$#2#3\m@th$}%
}
\makeatother

\newcommand{\seqdeno}[1]{\xslantmath{\llbracket}#1\xslantmath{\rrbracket}}

% Empty set etc.

\let\oldemptyset\emptyset
\let\emptyset\varnothing

%% Constant width xrightarrows
\newlength{\arrow}
\settowidth{\arrow}{\scriptsize$1000$}
\newcommand*{\constantwidthxrightarrow}[1]{\xrightarrow{\mathmakebox[\arrow]{#1}}}

%% Landscape pages
\usepackage{everypage}
\usepackage{environ}
\usepackage{pdflscape}
\newcounter{abspage}

\ifdefined\NOTARTICLE

\else

\makeatletter
\newcommand{\newSFPage}[1]% #1 = \theabspage
  {\global\expandafter\let\csname SFPage@#1\endcsname\null}

\NewEnviron{SidewaysFigure}{\begin{figure}[p]
\protected@write\@auxout{\let\theabspage=\relax}% delays expansion until shipout
  {\string\newSFPage{\theabspage}}%
\ifdim\textwidth=\textheight
  \rotatebox{90}{\parbox[c][\textwidth][c]{\linewidth}{\BODY}}%
\else
  \rotatebox{90}{\parbox[c][\textwidth][c]{\textheight}{\BODY}}%
\fi
\end{figure}}

\AddEverypageHook{% check if sideways figure on this page
  \ifdim\textwidth=\textheight
    \stepcounter{abspage}% already in landscape
  \else
    \@ifundefined{SFPage@\theabspage}{}{\global\pdfpageattr{/Rotate 0}}%
    \stepcounter{abspage}%
    \@ifundefined{SFPage@\theabspage}{}{\global\pdfpageattr{/Rotate 90}}%
  \fi}
\makeatother

\fi

%% PCF Things

\newcommand{\nat}{{\mathtt{nat}}}
\newcommand{\bool}{{\mathtt{bool}}}

\newcommand{\Y}{\mathbf{Y}}
\newcommand{\opto}{\longrightarrow}
\newcommand{\oopto}{\dashrightarrow}
\newcommand{\n}{{\mathtt{n}}}
\DeclareMathOperator{\IfO}{{\mathsf{If0}}}
\DeclareMathOperator{\suc}{{\mathsf{succ}}}
\DeclareMathOperator{\pred}{{\mathsf{pred}}}
\newcommand{\0}{{\mathtt{0}}}

\newcommand{\iter}{{\mathtt{iter}}}
\newcommand{\rec}{\iter}
\newcommand{\Var}{{\mathtt{Var}}}
\DeclareMathOperator{\Varr}{Var}
\newcommand{\new}{{\mathtt{new}}}
\newcommand{\case}{{\mathtt{case}}}

\newcommand{\lmam}{\mathrel{\sqsubseteq_{m\&m}}}
\newcommand{\emam}{\mathrel{\equiv_{m\&m}}}
\newcommand{\lst}{\mathrel{\lesssim}}
\newcommand{\smam}{\mathrel{\sim_{m\&m}}}
\newcommand{\amam}{\mathrel{\approx_{m\&m}}}

\newcommand{\oes}{\sim}

%% Idealized Algol things

\newcommand{\com}{{\mathtt{com}}}
\newcommand{\skipp}{{\mathsf{skip}}}
\DeclareMathOperator{\seq}{{\mathsf{seq}}}
\DeclareMathOperator{\neww}{{\mathsf{new}}}
\DeclareMathOperator{\mkvar}{{\mathsf{mkvar}}}
\newcommand{\deref}{\texttt{@}}
\DeclareMathOperator{\dereff}{\mathsf{deref}}
\DeclareMathOperator{\assign}{\mathsf{assign}}
\newcommand{\ia}[2]{\langle #1 , #2 \rangle}
\newcommand{\stup}[3]{\langle #1 \mid #2 \mapsto #3 \rangle}

%% Hyland-Ong games things

\newbox\gnBoxA
\newdimen\gnCornerHgt
\setbox\gnBoxA=\hbox{$\ulcorner$}
\global\gnCornerHgt=\ht\gnBoxA
\newdimen\gnArgHgt
\def\pv #1{%
    \setbox\gnBoxA=\hbox{$#1$}%
    \gnArgHgt=\ht\gnBoxA%
    \ifnum     \gnArgHgt<\gnCornerHgt \gnArgHgt=0pt%
    \else \advance \gnArgHgt by -\gnCornerHgt%
    \fi \raise\gnArgHgt\hbox{$\ulcorner$} \box\gnBoxA %
    \raise\gnArgHgt\hbox{$\urcorner$}}
\def\ov #1{%
    \setbox\gnBoxA=\hbox{$#1$}%
    \gnArgHgt=\ht\gnBoxA%
    \ifnum     \gnArgHgt<\gnCornerHgt \gnArgHgt=0pt%
    \else \advance \gnArgHgt by -\gnCornerHgt%
    \fi \raise\gnArgHgt\hbox{$\llcorner$} \box\gnBoxA %
    \raise\gnArgHgt\hbox{$\lrcorner$}}
\newcommand{\ct}[1]{\lceil#1\rceil}
\DeclareMathOperator{\Int}{int}

%% Nondeterministic Factorization things

\newcommand{\code}{\mathsf{code}}
\newcommand{\Det}{\mathsf{Det}}

%% Flexible strategy things

\newcommand{\stle}{{\;\le_s\;}}
\newcommand{\steq}{{\;=_s\;}}
\newcommand{\exle}{\sqsubseteq}
\newcommand{\exlub}{\bigsqcup}
\newcommand{\dv}{{\text{\lightning}}}
\DeclareMathOperator{\pocl}{pocl}
\newcommand{\plot}{\mathrel{\triangleleft}}
\newcommand{\shad}{\mathfrak{S}}
%\newcommand{\tree}{\mathfrak{T}}
\newcommand{\Tau}{T}
\newcommand{\Epsilon}{E}
\newcommand{\sw}{\triangleleft}

%% Roman numerals

\newcommand{\RN}[1]{%
  \textup{\uppercase\expandafter{\romannumeral#1}}%
}
\newcommand{\RNl}[1]{%
  \mathrel{\raisebox{1pt}{$\overline{\underline{#1}}$}}
}

%% Game language things

\newcommand{\ul}[1]{{\underline{#1}}}
\newcommand{\A}{{\mathcal{A}}}
\renewcommand{\P}{\mathcal P}
\newcommand{\M}{\mathcal M}
\newcommand{\N}{\mathcal N}
\newcommand{\X}{\mathcal X}
\newcommand{\YY}{\mathcal Y}
\newcommand{\hole}{\blank}
\newcommand{\Tct}{\xrightarrow{T}t}
\newcommand{\teamconverge}[2]{\xrightarrow{#1}#2}

%% Inference rule things
\newcommand{\rulename}[1]{\LeftTirNameStyle{#1}}
\newcommand{\ts}{\mathbin{\vdash}}
\newcommand{\nts}{\mathbin{\not\vdash}}

%% Double category things
\newcommand{\hc}[2]{\left({#1}\middle|{#2}\right)}
\newcommand{\vc}[2]{\left(\frac{#1}{#2}\right)}

%% What is going on?
\DeclareMathOperator{\Kl}{Kl}
\DeclareMathOperator{\Mell}{Mell}
\newcommand{\powerset}{\mathcal P}
\DeclareMathOperator{\ask}{{\mathsf{ask}}}
\newcommand{\sleep}{{\mathsf{sleep}}}
\newcommand{\true}{\mathbbm{t}}
\newcommand{\false}{\mathbbm{f}}
\DeclareMathOperator{\If}{\mathsf{If}}
\newcommand{\Then}{\mathrel{\mathsf{then}}}
\newcommand{\Else}{\mathrel{\mathsf{else}}}
\newcommand\cat{\mathbin{+\mkern-10mu+}}

%% Profunctor arrows

\makeatletter
\def\slashedarrowfill@#1#2#3#4#5{%
  $\m@th\thickmuskip0mu\medmuskip\thickmuskip\thinmuskip\thickmuskip
   \relax#5#1\mkern-7mu%
   \cleaders\hbox{$#5\mkern-2mu#2\mkern-2mu$}\hfill
   \mathclap{#3}\mathclap{#2}%
   \cleaders\hbox{$#5\mkern-2mu#2\mkern-2mu$}\hfill
   \mkern-7mu#4$%
}
\def\rightslashedarrowfill@{%
  \slashedarrowfill@\relbar\relbar\mapstochar\rightarrow}
\newcommand\xslashedrightarrow[2][]{%
  \ext@arrow 0055{\rightslashedarrowfill@}{#1}{#2}}
\makeatother
\newcommand{\pto}{{\xslashedrightarrow{} }}

%% Profunctors 
\DeclareMathOperator{\Prof}{Prof}
\DeclareMathOperator{\End}{End}
\DeclareMathOperator{\Endoprof}{Endoprof}

%% Our

\def\searchmacro#1{
  \AtBeginOfFiles{%
    \ifdefined#1
      \expandafter\def\csname \currfilename:found\endcsname{}%
    \fi}
  \AtEndOfFiles{%
    \ifdefined#1
      \unless\ifcsname \currfilename:found\endcsname
        \immediate\write\finder{found in '\currfilename'}%
    \fi\fi}}

%% Isomorphism arrows on commutative diagrams %%
\tikzset{Isom/.style={every to/.append style={edge node={node [sloped, above, allow upside down, auto=false]{$\cong$}}}},
         Isom'/.style={every to/.append style={edge node={node [sloped, above, allow upside down, auto=false, rotate=180]{$\cong$}}}},
         Sim/.style={every to/.append style={edge node={node [sloped, above, allow upside down, auto=false]{$\sim$}}}},
         Sim'/.style={every to/.append style={edge node={node [sloped, above, allow upside down, auto=false, rotate=180]{$\sim$}}}}}

%% Adjunctions
\newcommand{\adjunction}[4]{%
  {#1} \underset{\underset{#3}{\longleftarrow}}{\overset{\overset{#2}{\longrightarrow}}{\bot}} {#4}}        

%% Important!
\newcommand\Mellies{Melli\`{e}s\xspace}

\makeatletter
\newcommand{\colim@}[2]{%
  \vtop{\m@th\ialign{##\cr
    \hfil$#1\operator@font colim$\hfil\cr
    \noalign{\nointerlineskip\kern1.5\ex@}#2\cr
    \noalign{\nointerlineskip\kern-\ex@}\cr}}%
}
\newcommand{\colim}{%
  \mathop{\mathpalette\colim@{\rightarrowfill@\textstyle}}\nmlimits@
}
\makeatother

\makeatletter
\newcommand{\laxcolim@}[2]{%
  \vtop{\m@th\ialign{##\cr
    \hfil$#1\operator@font colim_l$\hfil\cr
    \noalign{\nointerlineskip\kern1.5\ex@}#2\cr
    \noalign{\nointerlineskip\kern-\ex@}\cr}}%
}
\newcommand{\laxcolim}{%
  \mathop{\mathpalette\laxcolim@{\rightarrowfill@\textstyle}}\nmlimits@
}
\makeatother

\DeclareMathOperator{\Colim}{colim}

\DeclareMathOperator{\DG}{DG}
\DeclareMathOperator{\RV}{RV}
\newcommand{\Rv}{\catname{Rv}}

\let\choose\undefined
\DeclareMathOperator{\choose}{\mathsf{choose}}
\DeclareMathOperator{\tr}{tr}
\DeclareMathOperator{\test}{test}

%% Slot game things %%
\newcommand{\circled}[1]{\raisebox{.5pt}{\textcircled{\raisebox{-.9pt} {#1}}}}
\newcommand{\slot}{{\circled{\$}}}

\DeclareMathOperator{\may}{may}
\DeclareMathOperator{\must}{must}

\newcommand{\encode}[1]{\lceil#1\rceil}
\DeclareMathOperator{\app}{\mathsf{app}}
\DeclareMathOperator{\lett}{\mathsf{let}}
\newcommand{\inn}{\mathrel{\mathsf{in}}}
\DeclareMathOperator{\byval}{\mathsf{byval}}

\DeclareMathOperator{\rread}{read}
\DeclareMathOperator{\wwrite}{write}

\DeclareSymbolFont{bbsymbol}{U}{bbold}{m}{n}
\DeclareMathSymbol{\bbsemicolon}{\mathbin}{bbsymbol}{"3B}
\newcommand{\semicom}{\bbsemicolon}

\newcommand{\ms}{\makebox[-1pt]{}}

\DeclareMathOperator{\Acc}{Acc}
\DeclareMathOperator{\im}{Im}
\DeclareMathOperator{\wit}{wit}

%%% END Article customizations



\renewcommand{\int}{{\mathtt{int}}}

\title{Rounding off a corner}

\author{John Gowers}

\begin{document}

\maketitle

\begin{abstract}
  In \cite{laird02}, Laird introduces the concept of a \emph{sequoidal category}, a certain extension of a monoidal category, and gives an example constructed using game semantics.  As noted in \cite{martinsthesis}, cofree commutative comonoids can be constructed coalgebraically in any sequoidal category subject to certain extra hypotheses.  In the first part of this note, we review the coalgebraic construction of cofree commutative comonoids in sequoidal categories.  In the second part, we shall show that the extra hypotheses are necessary as well as sufficient and we shall use transfinite game semantics to construct a sequoidal category in which these hypotheses do not hold.
\end{abstract}

\section{Sequoidal categories}

\subsection{Game semantics and the sequoidal operator}

We shall present a form of game semantics in the style of \cite{hyland1997games} and \cite{abramskyjagadeesangames}.  A game will be given by a tuple
\[
  A = (M_A, \lambda_A, b_A, P_A)
  \]
where
\begin{itemize}
  \item $M_A$ is a set of moves.
  \item $\lambda_A\from M_A\to\OP$ is a function designating each move as either an \emph{$O$-move} or a \emph{$P$-move}.
  \item $b_A\in\OP$ is a choice of starting player.
  \item $P_A\subset M_A^*$ is a prefix-closed set of alternating plays (so if $sab\in P_A$ then $\lambda_A(a)=\neg\lambda_A(b)$) such that if $as\in P_A$ then $\lambda_A(a)=b_A$.
\end{itemize}

We call $sa\in P_A$ a \emph{$P$-position} if $a$ is a $P$-move and an \emph{$O$-position} if $a$ is an $O$-move.

A \emph{strategy} for player $P$ for a game $A$ is identified with the set of positions that may arise when playing according to that strategy.  Namely, it is a non-empty prefix-closed subset $\sigma\subset P_A$ satisfying the two conditions:
\begin{description}
  \item[(sO)] If $s\in\sigma$ is a $P$-position and $a$ is an $O$-move such that $sa\in P_A$, then $sa\in\sigma$.
  \item[(sP)] If $sa,sb\in\sigma$ are $P$-positions, then $a=b$.
\end{description}

We shall now concentrate on games $A$ for which $b_A=O$, called \emph{negative games}.  We shall informally describe the standard connectives on negative games:

\begin{description}
  \item[Product] If $A$ and $B$ are negative games then the \emph{product} $A\times B$ is the game given by placing the game trees for $A$ and $B$ side by side: that is, player $O$ may play his first move either in $A$ or in $B$.  Thereafter, play continues in the game that player $O$ has chosen.
  \item[Tensor Product] The tensor product $A\tensor B$ is also played by playing the games $A$ and $B$ in parallel, but this time player $O$ may elect to switch games whenever it is his turn and continue play in the game he has switched to.
  \item[Linear implication] The implication $A\implies B$ is played by playing the game $B$ in parallel with the \emph{negation} of $A$ - that is, the game formed by switching the roles of players $P$ and $O$ in $A$.  Since play in the negation of $A$ starts with a $P$-move, player $O$ is forced to make his first move in the game $B$.  Thereafter, player $P$ may switch games whenever it is her turn.
\end{description}

If $A,B,C$ are negative games, $\sigma$ is a strategy for $A\implies B$ and $\tau$ is a strategy for $B\implies C$, then we may form a strategy $\comp\tau\sigma$ for $A\implies C$ by setting
\[
  \sigma\|\tau = \{\s\in (M_A\cprd M_B\cprd M_C)^*\suchthat \s\vert_{A,B}\in\sigma,\;s\vert_{B,C}\in\tau\}
  \]
and then defining
\[
  \comp\tau\sigma = \{\s\vert_{A,C}\suchthat\s\in\comp\tau\sigma\}
  \]
It is well known (see, for example, \cite{abramskyjagadeesangames}) that $\comp\tau\sigma$ is indeed a strategy for $A\implies C$ and that this form of composition is associative and has an identity.  It is also well known that the resulting category $\G$ of games and strategies has products given by the operator $\times$ and a symmetric monoidal closed structure given by the operations $\tensor$ and $\implies$.  

We turn now to the non-standard \emph{sequoid} connective $\sequoid$.  If $A$ and $B$ are negative games, then the sequoid $A\sequoid B$ is similar to the tensor product $A\tensor B$, but with the restriction that player $O$'s first move must take place in the game $A$.  We observe immediately that we have structural isomorphisms
\begin{gather*}
  \dist\from A\tensor B\toisom (A\sequoid B)\times(B\sequoid A)\\
  \dec\from(A\times B)\sequoid C\toisom (A\sequoid C)\times (B\sequoid C)\\
  \passoc\from (A\sequoid B)\sequoid C\toisom A\sequoid (B\tensor C)
\end{gather*}

One further question to ask is: does the sequoid operator give rise to a functor $\blank\sequoid\blank\from\G\times\G\to\G$, as the tensor operator does?  The answer is no: indeed, let $A,B,C,D$ be negative games, let $\sigma$ be a strategy for $A\implies C$ and let $\tau$ be a strategy for $B\implies D$.  Our aim is to construct a natural strategy $\sigma\sequoid\tau$ for $(A\sequoid B)\implies (C\sequoid D)$.  There is an obvious way to try and do this: player $P$ should play according to the strategy $\sigma$ whenever player $O$'s last move was in $A$ or $C$, and according to $\tau$ whenever player $O$'s last move was in $B$ or $D$.

We show that this does not in general give us a strategy for $(A\sequoid B)\implies (C\sequoid D)$.  Suppose that $\sigma$ is such that player $P$'s response to some opening move in $C$ is another move in $C$ and suppose that $\tau$ is such that player $P$'s response to some opening move in $D$ is a move in $B$ (for example, $\tau$ is a copycat strategy).  Then we end up with the following sequence of events in the game $(A\sequoid B)\implies(C\sequoid D)$  :
\begin{enumerate}
  \item Player $O$ starts with a move in $C$ (as he must).
  \item Player $P$ responds according to $\sigma$ with another move in $C$.
  \item Player $O$ decides to switch games and play a move in $D$.
  \item Player $P$ responds according to $\tau$ with a move in $B$.
\end{enumerate}

But now player $P$'s last move is not a legal move in $(A\sequoid B)\implies(C\sequoid D)$, since no moves have been played in $A$ yet.

We get round this problem by requiring that the strategy $\sigma$ be \emph{strict} -- that is, whatever player $O$'s opening move in $C$ is, player $P$'s reply must be a move in $A$.  

\begin{definition}
  Let $N,L$ be negative games and let $\sigma$ be a strategy for $N\implies L$.  We say that $\sigma$ is \emph{strict} if player $P$'s reply to an opening move in $L$ is always a move in $N$.  
\end{definition}

Identity strategies are strict and the composition of two strict strategies is strict, so we get a full-on-objects subcategory $\G_s$ of $\G$ where the morphisms are strict strategies.  Then the sequoid operator gives rise to a functor:
\[
  \blank\sequoid\blank\from \G_s\times\G\to\G_s
  \]

\subsection{Sequoidal categories}

We now have the motivation required to give the definition of a \emph{sequoidal category} from \cite{laird02}.  

\begin{definition}
  A \emph{sequoidal category} consists of the following data:
  \begin{itemize}
    \item A symmetric monoidal category $\C$ with monoidal product $\tensor$ and tensor unit $I$, associators $\assoc_{A,B,C}\from(A\tensor B)\tensor C\toisom A\tensor(B\tensor C)$, unitors $\runit_A\from A\tensor I\toisom A$ and $\lunit_A\from I\tensor A\toisom A$ and braiding $\sym_{A,B}\from A\tensor B\to B\tensor A$.
    \item A category $\C_s$
    \item A right monoidal category action of $\C$ on the category $\C_s$.  That is, a functor
      \[
        \blank\sequoid\blank\from\C_s\times\C\to\C_s
        \]
      together natural isomorphisms
      \[
        \passoc_{A,B,C}\from A\sequoid(B\tensor C)\to (A\sequoid B)\sequoid C
        \]
      and
      \[
        \run_A\from A\sequoid I\toisom A
        \]
      subject to the following coherence conditions:
      \begin{longdiagram}
        \begin{tikzcd}
          A\sequoid(B\tensor(C\tensor D)) \arrow[r, "\passoc_{A,B,C\tensor D}" yshift=0.3em] \arrow[d, "\id_A\sequoid \assoc_{B,C,D}"']
            & (A \sequoid B) \sequoid (C\tensor D) \arrow[r, "\passoc_{A\sequoid B,C,D}" yshift=0.3em]
              & ((A\sequoid B)\sequoid C) \sequoid D \\
          A\sequoid((B\tensor C)\tensor D) \arrow[r, "\passoc_{A,B\tensor C,D}"' yshift=-0.3em]
            & (A\sequoid (B\tensor C)) \sequoid D \arrow[ur, "\passoc_{A,B,C}\sequoid\id_D"']
              &
        \end{tikzcd}
      \end{longdiagram}
      \begin{longdiagram}
        \begin{tikzcd}
          A \sequoid (I\tensor B) \arrow[r, "\passoc_{A,I,B}" yshift=0.3em] \arrow[d, "\id_A\sequoid\lunit_B"']
            & (A\sequoid I)\sequoid B \arrow[dl, "\run_A"] \\
          A \sequoid B
            &
        \end{tikzcd}
        \quad
        \begin{tikzcd}
          A \sequoid (B\tensor I) \arrow[r, "\passoc_{A,B,I}" yshift=0.3em] \arrow[d, "\id_A\sequoid\runit_B"']
            & (A\sequoid B) \sequoid I \arrow[dl, "\run_{A\sequoid B}"] \\
          A\sequoid B
            &
        \end{tikzcd}
      \end{longdiagram}

      \item A functor $J\from \C_s\to\C$ (in the games example, this is the inclusion functor $\G_s\to\G$)

      \item A natural transformation $\wk_{A,B}\from J(A)\tensor B\to J(A\sequoid B)$ satisfying the coherence conditions:
      \begin{longdiagram}
        \begin{tikzcd}
          A \tensor I \arrow[r, "\runit_A" yshift=0.3em] \arrow[d, "\wk_{A,I}"']
            & A \\
          A \sequoid I \arrow[ur, "J(\run_A)"']
            &
        \end{tikzcd}
        \quad
        \begin{tikzcd}
          (A \tensor B) \tensor C \arrow[r, "\wk_{A,B}\tensor\id_C" yshift=0.3em] \arrow[d, "\assoc_{A,B,C}"']
            & (A \sequoid B) \tensor C \arrow[r, "\wk_{A\sequoid B, C}" yshift=0.3em]
              & (A \sequoid B) \sequoid C\\
          A \tensor (B \tensor C) \arrow[r, "\wk_{A,B\tensor C}"']
            & A \sequoid (B \tensor C) \arrow[ur, "J(\passoc_{A,B,C})"']
              &
        \end{tikzcd}
      \end{longdiagram}
  \end{itemize}
\end{definition}

Our category of games satisfies further conditions:

\begin{definition}
  Let $\C=(\C,\C_s,J,\wk)$ be a sequoidal category.  We say that $\C$ is an \emph{inclusive sequoidal category} if $\C_s$ is a full-on-objects subcategoryof $\C$ containing the monoidal isomorphisms and the morphisms $\wk_{A,B}$, $J$ is the inclusion functor and $J$ reflects isomorphisms.

  If $\C$ is an inclusive sequoidal category, we say that $\C$ is \emph{Cartesian} if $\C_s$ has all products and these are preserved by $J$.  In that case, we say that $\C$ is \emph{decomposable} if the natural transformations
  \begin{gather*}
    \dec_{A,B} = \langle \wk_{A,B}, \comp{\wk_{A,B}}{\sym_{A,B}}\rangle\from A\tensor B\to (A\sequoid B)\times (B\sequoid A) \\
    \dec^0 \from I \to 1
  \end{gather*}
  are isomorphisms and we say that $\C$ is \emph{distributive} if the natural transformations
  \begin{gather*}
    \dist_{A,B,C} = \langle \pr_1\sequoid \id_C,\pr_2\sequoid\id_C\rangle\from (A\times B)\sequoid C\to (A\sequoid C)\times (B\sequoid C) \\
    \dist_{A,0}\from 1\sequoid A\to 1
  \end{gather*}
  are isomorphisms.
\end{definition}

We have one further piece of structure available to us:

\begin{definition}
  Let $\C=(\C,\C_s,J,\wk)$ be an inclusive sequoidal category.  We say that $\C$ is a \emph{sequoidal closed category} if $\C$ is monoidal closed (with internal hom $\implies$ and currying $\Lambda_{A,B,C}\from \C(A\tensor B,C)\toisom \C(A,B\implies C)$) and if the map $f\mapsto\Lambda(\comp{f}{\wk})$ gives rise to a natural transformation
  \[
    \Lambda_{A,B,C,s}\from \C_s(A\sequoid B, C) \to \C_s(A,B\implies C)
    \]
\end{definition}

It can be shown (see for example \cite{martinsthesis}) that our category $\G$ of games has all this structure.

\begin{theorem}
  If $A,B$ are games, let $J$ be the inclusion functor $\G_s\to\G$ and let $\wk_{A,B}\from A\tensor B\to A\sequoid B$ be the natural copycat strategy.  Then
  \[
    (\G,\G_s,J,\wk)
    \]
  is an inclusive, Cartesian, decomposable, distributive sequoidal closed category.
\end{theorem}

\subsection{The sequoidal exponential}

There are several ways to add exponentials to the basic category of games.  We shall use the definition based on countably many copies of the base game (see \cite{laird02}, for example):

\begin{definition}
  Let $A$ be a negative game.  The \emph{exponential} of $A$ is the game $\oc A = (M_{\oc A}, \lambda_{\oc A}, b_{\oc A}, P_{\oc A})$, where $M_{\oc A}, \lambda_{\oc A}, b_{\oc A}, P_{\oc A}$ are defined as follows:
  \begin{itemize}
    \item $M_{\oc A} = M_A \times \omega$
    \item $\lambda_{\oc A} = \lambda_A \circ \pr_1$
    \item $b_{\oc A} = O$
    \item Given a sequence $s\in M_{\oc A}^\omega$, we write $s\vert_n$ for the largest sequence $a_1a_2\dots a_k\in M_A^*$ such that $(a_1,n),(a_2,n),\dots(a_k,n)$ is a subsequence of $s$.  Then $P_{\oc A}$ is the set of all sequence $s\in M_{\oc A}^\omega$ that are alternating with respect to $\lambda_{\oc A}$, such that $s\vert_n\in P_A$ for all $n$ and such that if $m<n$ and $(a,n)$ occurs in $s$ then $(b,m)$ must occur earlier in $s$ for some move $b$: in other words, player $O$ can start infinitely many copies of the game $A$, but he must start them in order.
  \end{itemize}
\end{definition}

This last condition on the order in which games may be opened is very important, as it allows us to define morphisms that give $\oc A$ the semantics of the exponential from linear logic.  For example, we have a natural morphism $\mu\from\oc A\to\oc A\tensor\oc A$, given by the copycat strategy that starts a new copy of $A$ on the left whenever one is started on the right.  Because of the condition on the order in which copies of $A$ may be started, there is a unique way to do this.

\begin{proposition}
  $\mu$ exhibits $\oc A$ as a comonoid in the monoidal category $(\G, \tensor, I)$.  
\end{proposition}
\begin{proof}
  $\mu$ shall be the comultiplication in our comonoid.  The counit is given by the empty strategy $\eta\from\oc A\to I$.  We just need to check that $\mu$ is associative and that $\eta$ is a counit for $\mu$.  

  For associativity, we need to show that the following diagram commutes:
  \[
    \begin{tikzcd}
      \oc A \arrow[rr, "\mu"] \arrow[d, "\mu"']
        &
          & \oc A \tensor \oc A \arrow[d, "\id_{\oc A}\tensor\mu"] \\
      \oc A \tensor \oc A \arrow[r, "\mu\tensor\id_{\oc A}"' yshift=-0.4em]
        & (\oc A \tensor \oc A) \tensor \oc A \arrow[r, "\assoc_{\oc A,\oc A,\oc A}"' yshift=-0.4em]
          & \oc A \tensor (\oc A \tensor \oc A)
    \end{tikzcd}
    \]
  This is easy to see when we notice that both branches of the square are copycat strategies on $\oc A\implies\oc A\tensor(\oc A\tensor \oc A)$; since copies of $A$ in $\oc A$ must be started in sequence, there is a unique such strategy, and so the square commutes.

  For the counit, we need to show that the follosing two diagrams commute:
  \[
    \begin{tikzcd}
      \oc A \arrow[r, "\mu"] \arrow[dr, "\runit_A\inv"']
        & \oc A \tensor \oc A \arrow[d, "\id_{\oc A}\tensor \eta"] \\
      %
        & \oc A \tensor I
    \end{tikzcd}
    \quad\quad\quad
    \begin{tikzcd}
      \oc A \arrow[r, "\mu"] \arrow[dr, "\lunit_A\inv"']
        & \oc A \tensor \oc A \arrow[d, "\eta\tensor\id_{\oc A}"] \\
      %
        & I \tensor \oc A
    \end{tikzcd}
    \]
  Once again, these diagrams commute because both branches are copycat strategies for $\oc A\implies\oc A\tensor I$ or $\oc A\implies I\tensor\oc A$ and there is a unique such strategy in each case.
\end{proof}

We shall later show that $(\oc A, \mu, \eta)$ is in fact the \emph{cofree commutative comonoid} on $A$ in the monoidal category $(\G, \tensor, I)$.  

We shall call the exponential $\oc A$ the \emph{sequoidal exponential}.  The following proposition explains the name:

\begin{proposition}
  Let $A$ be a negative game.  Then we get an endofunctor $A\sequoid\blank$ on $\G$ given by sending $B$ to $A\sequoid B$.  

  The sequoidal exponential $\oc A$, together with the obvious copycat strategy $\alpha\from\oc A\to A\sequoid\oc A$, is the final coalgebra for the endofunctor $A\sequoid\blank$.  In other words, if $B$ is a negative game and $\sigma\from B\to A\sequoid B$ is a morphism then there is a unique morphism $\fcoal{\sigma}\from B\to \oc A$ such that the following diagram commutes:
  \[
    \begin{tikzcd}
      B \arrow[r, "\sigma"] \arrow[d, "\fcoal{\sigma}"']
        & A \sequoid B \arrow[d, "\id_A\sequoid\fcoal{\sigma}"] \\
      \oc A \arrow[r, "\alpha"']
        & A \sequoid \oc A
    \end{tikzcd}
    \]
\end{proposition}
\begin{proof}
  See \cite{martinsthesis}.  We shall shortly give a proof in the more general case.
\end{proof}

\subsection{Constructing cofree commutative comonoids in sequoidal categories}

We want to deduce the fact that $\oc A\xrightarrow{\mu}\oc A\tensor\oc A$ is the cofree commutative comonoid on $A$ from the fact that $\oc A$ is the final coalgebra for $A\sequoid\blank$.  It turns out that we shall need one more fact about the category of games to prove this.

\begin{definition}
  Let $A,B$ be objects of an inclusive, Cartesian, distributive sequoidal category with final coalgebras $\oc A\xrightarrow{\alpha_A} A\sequoid\oc A$ for all endofunctors of the form $A\sequoid\blank$.  Then we have morphisms:
  \[
    \begin{tikzcd}
      \oc A \tensor \oc B \arrow[r, "\alpha_A\tensor\id_{\oc B}"] \arrow[d, "\sym"]
        & (A \sequoid\oc A)\tensor \oc B \arrow[r, "\wk"]
          & (A\sequoid\oc A)\sequoid\oc B \arrow[r, "\passoc"]
            & A \sequoid (\oc A \tensor\oc B)
              & \\
      \oc B \tensor \oc A \arrow[r, "\alpha_B\tensor\id_{\oc A}"]
        & (B\sequoid\oc B)\tensor\oc A \arrow[r, "\wk"]
          & (B\sequoid\oc B)\sequoid \oc A \arrow[r, "\passoc"]
            & B \sequoid (\oc B\tensor\oc A) \arrow[r, "\id_B\sequoid\sym"]
              & B \sequoid (\oc A\tensor\oc B)
    \end{tikzcd}
    \]
inducing a morphism
\[
  \oc A\tensor \oc B \to (A \sequoid(\oc A\tensor\oc B)) \times (B \sequoid(\oc A\tensor \oc B)) \xrightarrow{\dist\inv}
    (A\times B)\sequoid(\oc A\tensor\oc B)
    \]
  Remembering that our category has a final coalgebra $\oc(A\times B)$ for the functor $(A\times B)\sequoid\blank$, we write $\int_{A,B}$ for the unique morphism $\oc A\tensor\oc B\to \oc(A\times B)$ making the following diagram commute
  \[
    \begin{tikzcd}
      \oc A\tensor\oc B \arrow[r] \arrow[d, "\int_{A,B}"']
        & (A\sequoid(\oc A\tensor\oc B))\times(B\sequoid(\oc A\tensor\oc B))\arrow[r,"\dist\inv"]
          & (A\times B)\sequoid(\oc A\tensor\oc B) \arrow[d, "\id_{A\times B}\sequoid\int_{A,B}"] \\
      \oc(A\times B) \arrow[rr, "\alpha_{A\times B}"']
        &
          & (A\times B)\sequoid\oc(A\times B)
    \end{tikzcd}
    \]
\end{definition}

\begin{proposition}
  In the category of games, the morphism $\int_{A,B}$ is an isomorphism for all negative games $A,B$.
\end{proposition}
\begin{proof}
  Observe that the morphism $\int_{A,B}$ is the copycat strategy on $\oc A\tensor\oc B\implies \oc (A\times B)$ that starts a copy of $A$ on the left whenever a copy of $A$ is started on the right and starts a copy of $B$ on the left whenever a copy of $B$ is started on the right (indeed, the morphisms in the diagram above are all copycat morphisms, so the copycat strategy we have just described must make that diagram commute.  Since there are infinitely many copies of both $A$ and $B$ available in $\oc (A\times B)$, and since a new copy of $A$ or $B$ may be started at any time, we may define an inverse copycat strategy on $\oc(A\times B)\implies \oc A\tensor\oc B$.
\end{proof}

Our main result for this section will be the following:
\begin{theorem}
  Let $(\C,\C_s,J)$ be an inclusive, Cartesian, distributive and decomposable sequoidal category with a final coalgebra $\oc A\xrightarrow{\alpha_A}A\sequoid\oc A$ for each endofunctor of the form $A\sequoid\blank$.  Suppose further that the morphism $\int_{A,B}$ as defined above is an isomorphism for all objects $A,B$.  Then $\oc A$ is the carrier of a cofree commutative comonoid over $A$ in the monoidal category $(\C,\tensor,I)$.  
\end{theorem}

The first step will be to define the comultiplication and counits for our comonoid over $\oc A$.  For the comultiplication, note that we have a morphism
\[
  \oc A \xrightarrow{\alpha_A} A \sequoid \oc A \xrightarrow{\Delta} (A\sequoid\oc A)\times (A\sequoid\oc A)\xrightarrow{\dist\inv} (A\times A)\sequoid\oc A
  \]

\bibliographystyle{alpha}
\bibliography{../common/phd_bibliography}

\end{document}
