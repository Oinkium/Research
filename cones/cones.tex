\documentclass{article}

\usepackage[utf8]{inputenc}

\usepackage{graphicx} % support the \includegraphics command and options

\usepackage{parskip} % Activate to begin paragraphs with an empty line rather than an indent

%%% PACKAGES
\usepackage{booktabs} % for much better looking tables
\usepackage{array} % for better arrays (eg matrices) in maths
\ifdefined\BEAMER
\else
\usepackage{paralist} % very flexible & customisable lists (eg. enumerate/itemize, etc.)\prefix\t$.
\fi
\usepackage{verbatim} % adds environment for commenting out blocks of text & for better verbatim
\ifdefined\BEAMER
\else
\ifdefined\THESIS
\usepackage{subcaption}
\else
\usepackage{subfig} % make it possible to include more than one captioned figure/table in a single float
\fi
\fi
\usepackage{mathtools} % for the all important \coloneqq symbol
\usepackage{hyperref} % for hyperreferences
\usepackage{IEEEtrantools} % for \IEEEeqnarray
\usepackage{pbox} % for \pbox
\usepackage{multirow,bigdelim} % for \multirow
\usepackage{lettrine} % For the drop cap
\usepackage{mathpartir} % for \inferrule, \inferrule* and the mathpar environment
\usepackage{listings}

\usepackage{caption}
\captionsetup{singlelinecheck=off}

\ifdefined\NOTARTICLE
\else

%%% ToC (table of contents) APPEARANCE
\usepackage[nottoc,notlof,notlot]{tocbibind} % Put the bibliography in the ToC
\usepackage[titles,subfigure]{tocloft} % Alter the style of the Table of Contents
\renewcommand{\cftsecfont}{\rmfamily\mdseries\upshape}
\renewcommand{\cftsecpagefont}{\rmfamily\mdseries\upshape} % No bold!

\fi

%% Font things %%
\usepackage{amssymb}
\usepackage{cmll} % Linear logic symbols!
\ifdefined\FEWFONTS
\else
\usepackage{bm} % for bold Greek letters
\fi
\usepackage{stmaryrd}
\usepackage{bbm}

%% Get the sqsubsetneqq character from the mathabx package
\DeclareFontFamily{U}{mathb}{\hyphenchar\font45}
\DeclareFontShape{U}{mathb}{m}{n}{
      <5> <6> <7> <8> <9> <10> gen * mathb
      <10.95> mathb10 <12> <14.4> <17.28> <20.74> <24.88> mathb12
      }{}
\DeclareSymbolFont{mathb}{U}{mathb}{m}{n}

\DeclareMathSymbol{\sqsubsetneq}    {3}{mathb}{"88}
\DeclareMathSymbol{\varsqsubsetneq} {3}{mathb}{"8A}
\DeclareMathSymbol{\varsqsubsetneqq}{3}{mathb}{"92}
\DeclareMathSymbol{\sqsubsetneqq}   {3}{mathb}{"90}

%% Get the left and right moons from the wasysym package

\DeclareFontFamily{U}{wasy}{}
\DeclareFontShape{U}{wasy}{m}{n}{ <5> <6> <7> <8> <9> gen * wasy
      <10> <10.95> <12> <14.4> <17.28> <20.74> <24.88>wasy10  }{}
\DeclareFontShape{U}{wasy}{b}{n}{ <-10> sub * wasy/m/n
 <10> <10.95> <12> <14.4> <17.28> <20.74> <24.88>wasyb10 }{}
\DeclareFontShape{U}{wasy}{bx}{n}{ <-> sub * wasy/b/n}{}

\def\wasyfamily{\fontencoding{U}\fontfamily{wasy}\selectfont}
\def\leftmoon   {\mbox{\wasyfamily\char36}}
\def\rightmoon  {\mbox{\wasyfamily\char37}}

%% Lists %%
\usepackage{enumerate}

%% Graphics %%
\usepackage{tikz}
\usetikzlibrary{cd}
\usetikzlibrary{patterns}
\usetikzlibrary{calc}
\usetikzlibrary{decorations.pathmorphing}
\usetikzlibrary{positioning}

\tikzset{inlinearrows/.style={anchor=base,baseline,x=0.6\baselineskip,y=0.6\baselineskip}}

\ifdefined\BEAMER
\else

%% Theorems! %%
\usepackage{amsthm}
\theoremstyle{plain} % Theorems, lemmas, propositions etc.
\newtheorem{theorem}{Theorem}[section]
\newtheorem{lemma}[theorem]{Lemma}
\newtheorem{proposition}[theorem]{Proposition}
\newtheorem{corollary}[theorem]{Corollary}
\newtheorem{fact}[theorem]{Fact}
\newtheorem{construction}[theorem]{Construction}
\theoremstyle{definition} % Definitions etc.  
\newtheorem{definition}[theorem]{Definition}
\newtheorem{notation}[theorem]{Notation}
\theoremstyle{remark} % Remarks
\newtheorem{remark}[theorem]{Remark}
\newtheorem{remarks}[theorem]{Remarks}
\newtheorem{example}[theorem]{Example}
\newtheorem{question}[theorem]{Question}
\newtheorem{slogan}[theorem]{Slogan}

\newtheoremstyle{note} {3pt} {3pt} {\itshape} {} {\itshape} {:} {.5em} {} % For short notes
\theoremstyle{note}
\newtheorem{note}[theorem]{Note}

\fi

%% Exercises and answers %%
\usepackage{answers}

\newtheoremstyle{exercisestyle}% name
  {6pt}   % ABOVESPACE
  {6pt}   % BELOWSPACE
  {\itshape}  % BODYFONT
  {0pt}       % INDENT (empty value is the same as 0pt)
  {\bfseries} % HEADFONT
  {.}         % HEADPUNCT
  {3pt} % HEADSPACE
  {}          % CUSTOM-HEAD-SPEC

\theoremstyle{exercisestyle}
\newtheorem{exercise}{Exercise}
\newtheorem{answerthm}{Exercise}

\Newassociation{answer}{answerthm}{answers}
\newcommand{\answerthmparams}{}

%% Changes to enumerate things so they look better %%\sigma$

\makeatletter
\def\enumfix{%
\if@inlabel
 \noindent \par\nobreak\vskip-\topsep\hrule\@height\z@
\fi}

\let\olditemize\itemize
\def\itemize{\enumfix\olditemize}
\let\oldenumerate\enumerate
\def\enumerate{\enumfix\oldenumerate}

%% Random crap %%
\usepackage{xifthen}

\makeatletter
\def\thm@space@setup{%
  \thm@preskip=\parskip \thm@postskip=0pt
}
\makeatother

\makeatletter
\newcommand*{\relrelbarsep}{.386ex}
\newcommand*{\relrelbar}{%
  \mathrel{%
    \mathpalette\@relrelbar\relrelbarsep
  }%
}
\newcommand*{\@relrelbar}[2]{%
  \raise#2\hbox to 0pt{$\m@th#1\relbar$\hss}%
  \lower#2\hbox{$\m@th#1\relbar$}%
}
\providecommand*{\rightrightarrowsfill@}{%
  \arrowfill@\relrelbar\relrelbar\rightrightarrows
}
\providecommand*{\leftleftarrowsfill@}{%
  \arrowfill@\leftleftarrows\relrelbar\relrelbar
}
\providecommand*{\xrightrightarrows}[2][]{%
  \ext@arrow 0359\rightrightarrowsfill@{#1}{#2}%
}
\providecommand*{\xleftleftarrows}[2][]{%
  \ext@arrow 3095\leftleftarrowsfill@{#1}{#2}%
}
\makeatother

\newcommand{\catname}[1]{{\normalfont\textbf{#1}}}
\newcommand{\Rings}{\catname{CRing}}
\newcommand{\CAT}{\catname{CAT}}
%\newcommand{\Top}{\catname{Top}}
\newcommand{\Set}{\catname{Set}}
\newcommand{\Cat}{\catname{Cat}}
\newcommand{\MonCat}{\catname{MonCat}}
\newcommand{\SymmMonCat}{\catname{SymmMonCat}}
\newcommand{\Cont}{\catname{Cont}}
\newcommand{\Sch}{\catname{Sch}}
\newcommand{\Rel}{\catname{Rel}}
\newcommand{\Coh}{\catname{Coh}}
\newcommand{\Inj}{\catname{Inj}}
\newcommand{\Dcpo}{\catname{Dcpo}}
\newcommand{\Mod}[1][]{\ifthenelse{\isempty{#1}}{\catname{Mod}}{#1\catname{mod}}}
\DeclareMathOperator{\sh}{Sh}
\newcommand{\Sh}[1][]{\ifthenelse{\isempty{#1}}{\sh}{\sh(#1)}}
\newcommand{\map}[3]{#2\xrightarrow{#1} #3}
\newcommand*\from{\colon}
\newcommand*\bigto{\Rightarrow}
\newcommand{\cmap}[3]{#1\from{}#2\to{}#3}
\newcommand\oppcat[1]{#1^{\mathrm{op}}}
\newcommand{\object}{\colon}
\DeclareRobustCommand{\vmap}[3] {\begin{tikzcd} #2 \arrow[d, "#1"] \\ #3 \end{tikzcd}}
\newcommand{\partref}[1]{(\ref{#1})}
\newcommand{\intgrpd}[4] {#1 \xrightrightarrows[#3]{#4} #2}
\DeclareRobustCommand{\bigintgrpd}[4] {\begin{tikzcd}[ampersand replacement=\&] #1 \arrow[r, shift left=0.5ex, "#3"] \arrow[r, shift right=0.5ex, "#4"'] \& #2 \end{tikzcd}}

\usepackage{xspace}

\newcommand{\etale}{\'{e}tale\xspace}
\newcommand{\Etale}{\'{E}tale\xspace}

\def \inv {^{-1}}

\DeclareMathOperator{\id}{id}
\DeclareMathOperator{\op}{op}
\DeclareMathOperator{\pr}{pr}
\DeclareMathOperator{\inj}{in}
\DeclareMathOperator{\pre}{{pre}}
\DeclareMathOperator{\et}{{\acute{e}t}}

\DeclareMathOperator{\Hom}{Hom}
\DeclareMathOperator{\Spec}{Spec}

\DeclareMathOperator{\ol}{ol}

\def\presuper#1#2%
  {\mathop{}%
   \mathopen{\vphantom{#2}}^{#1}%
   \kern-\scriptspace%
   #2}
\def\presub#1#2%
  {\mathop{}%
   \mathopen{\vphantom{#2}}_{#1}%
   \kern-\scriptspace%
   #2}

\newsavebox{\overlongequation}
\newenvironment{longdiagram}
 {\begin{displaymath}\begin{lrbox}{\overlongequation}$\displaystyle}
 {$\end{lrbox}\makebox[0pt]{\usebox{\overlongequation}}\end{displaymath}}

%% Our things %%

\newcommand{\neggame}[1]{\presuper{\perp}{#1}}
\newcommand{\tensor}{\otimes}
\newcommand{\Tensor}{\bigotimes}
\newcommand{\sequoid}{\oslash}
\newcommand{\varsequoid}{\vartriangleleft}
\renewcommand{\implies}{\multimap}
\newcommand{\iimpl}{\Longrightarrow}
\newcommand{\comp}[2]{#1 \circ #2}
\newcommand{\icomp}[2]{\comp{#1}{#2}}
\newcommand{\cprd}{\sqcup}
\newcommand{\bigcprd}{\bigsqcup}
\newcommand{\G}{\mathcal G}
\newcommand{\W}{\mathcal W}
\newcommand{\suchthat}{\;\colon\;}
\newcommand{\varsuchthat}{\;\mid\;}
\newcommand{\esuchthat}{\;.\;}
\newcommand{\OP}{\{O,P\}}
\newcommand{\QA}{\{Q,A\}}
\renewcommand{\L}{\mathcal L}
\newcommand{\F}{\mathcal F}
\newcommand{\U}{\mathcal U}
\newcommand{\s}{\mathfrak s}
\renewcommand{\t}{\mathfrak t}
\renewcommand{\u}{\mathfrak u}
\renewcommand{\d}{\mathfrak d}
\newcommand{\e}{\mathfrak e}
\newcommand{\emptyplay}{\epsilon}
\newcommand{\bracketed}[1]{\left({#1}\right)}
\newcommand{\bneggame}[1]{{\bracketed{\neggame{#1}}}}
\newcommand{\prefix}{\sqsubseteq}
\newcommand{\ppprefix}{\sqsubset}
\newcommand{\pprefix}{\sqsubsetneqq}
\renewcommand{\ss}{\mathbf{s}}
\newcommand{\bN}{\mathbb{N}}
\newcommand{\bC}{\mathbb{C}}
\newcommand{\bB}{\mathbb{B}}
\newcommand{\bP}{\mathbb{P}}
\newcommand{\pfun}{\rightharpoonup}
\newcommand{\grel}[1]{\underline{#1}}
\DeclareMathOperator{\length}{length}
\renewcommand{\b}{\mathfrak b}
\renewcommand{\r}{\mathfrak r}
\newcommand{\bbeta}{{\bm{\beta}}}
\newcommand{\st}{{\Sigma^*}}
\let\sec\S
\renewcommand{\S}{{\mathfrak{S}}}
\DeclareMathOperator{\cc}{cc}
\DeclareMathOperator{\subs}{subs}
\DeclareMathOperator{\ret}{ret}
\DeclareMathOperator{\zz}{zz}
\newcommand{\aaa}{\mathbf{a}}
\newcommand{\bbb}{\mathbf{b}}
\newcommand{\ccc}{\mathbf{c}}
\newcommand{\ddd}{\mathbf{d}}
\newcommand{\B}{\mathcal B}
\newcommand{\BB}{\mathbf B}
\renewcommand{\H}{\mathcal H}
\DeclareMathOperator{\assoc}{assoc}
\DeclareMathOperator{\lunit}{lunit}
\DeclareMathOperator{\runit}{runit}
\DeclareMathOperator{\dom}{dom}
\DeclareMathOperator{\sym}{sym}
\newcommand{\braid}{\sym}
\newcommand{\blank}{\,\underline{\hspace{1.5ex}}\,}
\DeclareMathOperator{\cn}{cn}
\newcommand{\impliescn}{\protect\overset{\cn}{\implies}}
\newcommand{\C}{{\mathcal{C}}}
\newcommand{\D}{{\mathcal{D}}}
\newcommand{\E}{{\mathcal{E}}}
\newcommand{\V}{{\mathcal{V}}}
\newcommand{\EE}{{\mathbf{E}}}
\DeclareMathOperator{\ev}{ev}
\newcommand{\der}{{\mathtt{der}}}
\newcommand{\mult}{{\mathtt{mult}}}
\DeclareMathOperator{\wk}{wk}
\newcommand{\toisom}{{\xrightarrow{\cong}}}
\DeclareMathOperator{\passoc}{{\mathsf{passoc}}}
\DeclareMathOperator{\pcomm}{{\mathsf{pcomm}}}
\DeclareMathOperator{\run}{{\mathsf{r}}}
\DeclareMathOperator{\lun}{{\mathsf{l}}}
\newcommand{\fcoal}[1]{{\leftmoon #1 \rightmoon}}
\DeclareMathSymbol{\co}{\mathord}{operators}{"3C}
\DeclareMathSymbol{\nw}{\mathord}{operators}{"3E}
\newcommand{\T}{\mathfrak{T}}
\renewcommand{\subset}{\subseteq}
\newcommand{\Ord}{\catname{Ord}}
\newcommand{\FS}{\mathcal{FS}}
\DeclareMathOperator{\rank}{rank}
\DeclareMathOperator{\dist}{{\mathsf{dist}}}
\DeclareMathOperator{\dec}{{\mathsf{dec}}}
\DeclareMathOperator{\str}{str}
\DeclareMathOperator{\weak}{weak}
\DeclareMathOperator{\Strat}{Strat}
\DeclareMathOperator{\OppStrat}{OppStrat}
\newcommand{\seqs}[1]{{\overline{{#1}^{*}}}}
\def\flushRight{\leftskip0pt plus 1fill\rightskip0pt}
\def\Centering{\relax\ifvmode\centering\fi}
\newcommand{\deno}[1]{\left\llbracket#1\right\rrbracket}
\newcommand{\converges}{\Downarrow}
\newcommand{\diverges}{\Uparrow}
\newcommand{\mustconverge}{\converges^{\text{must}}}
\newcommand{\Iflt}{\mathtt{If{<}\;}}
\newcommand{\Ifgt}{\mathtt{If{>}\;}}
\newcommand{\inr}{{\mathsf{inr}}}
\newcommand{\inl}{{\mathsf{inl}}}
\newcommand{{\Na}}{\bN}
\newcommand{{\cell}}{{\mathsf{cell}}}
\newcommand{\fix}{{\mathsf{fix}}}
\newcommand{\eq}{{\mathsf{eq}}}
\DeclareMathOperator{\CCom}{CCom}
\newcommand{\power}{\mathfrak P}

% Slanty things
\newcommand*{\xslant}[2][76]{%
  \begingroup
    \sbox0{#2}%
    \pgfmathsetlengthmacro\wdslant{\the\wd0 + cos(#1)*\the\wd0}%
    \leavevmode
    \hbox to \wdslant{\hss
      \tikz[
        baseline=(X.base),
        inner sep=0pt,
        transform canvas={xslant=cos(#1)},
      ] \node (X) {\usebox0};%
      \hss
      \vrule width 0pt height\ht0 depth\dp0 %
    }%
  \endgroup
}

\makeatletter
\newcommand*{\xslantmath}{}
\def\xslantmath#1#{%
  \@xslantmath{#1}%
}
\newcommand*{\@xslantmath}[2]{%
  % #1: optional argument for \xslant including brackets
  % #2: math symbol
  \ensuremath{%
    \mathpalette{\@@xslantmath{#1}}{#2}%
  }%
}
\newcommand*{\@@xslantmath}[3]{%
  % #1: optional argument for \xslant including brackets
  % #2: math style
  % #3: math symbol
  \xslant#1{$#2#3\m@th$}%
}
\makeatother

\newcommand{\seqdeno}[1]{\xslantmath{\llbracket}#1\xslantmath{\rrbracket}}

% Empty set etc.

\let\oldemptyset\emptyset
\let\emptyset\varnothing

%% Constant width xrightarrows
\newlength{\arrow}
\settowidth{\arrow}{\scriptsize$1000$}
\newcommand*{\constantwidthxrightarrow}[1]{\xrightarrow{\mathmakebox[\arrow]{#1}}}

%% Landscape pages
\usepackage{everypage}
\usepackage{environ}
\usepackage{pdflscape}
\newcounter{abspage}

\ifdefined\NOTARTICLE

\else

\makeatletter
\newcommand{\newSFPage}[1]% #1 = \theabspage
  {\global\expandafter\let\csname SFPage@#1\endcsname\null}

\NewEnviron{SidewaysFigure}{\begin{figure}[p]
\protected@write\@auxout{\let\theabspage=\relax}% delays expansion until shipout
  {\string\newSFPage{\theabspage}}%
\ifdim\textwidth=\textheight
  \rotatebox{90}{\parbox[c][\textwidth][c]{\linewidth}{\BODY}}%
\else
  \rotatebox{90}{\parbox[c][\textwidth][c]{\textheight}{\BODY}}%
\fi
\end{figure}}

\AddEverypageHook{% check if sideways figure on this page
  \ifdim\textwidth=\textheight
    \stepcounter{abspage}% already in landscape
  \else
    \@ifundefined{SFPage@\theabspage}{}{\global\pdfpageattr{/Rotate 0}}%
    \stepcounter{abspage}%
    \@ifundefined{SFPage@\theabspage}{}{\global\pdfpageattr{/Rotate 90}}%
  \fi}
\makeatother

\fi

%% PCF Things

\newcommand{\nat}{{\mathtt{nat}}}
\newcommand{\bool}{{\mathtt{bool}}}

\newcommand{\Y}{\mathbf{Y}}
\newcommand{\opto}{\longrightarrow}
\newcommand{\oopto}{\dashrightarrow}
\newcommand{\n}{{\mathtt{n}}}
\DeclareMathOperator{\IfO}{{\mathsf{If0}}}
\DeclareMathOperator{\suc}{{\mathsf{succ}}}
\DeclareMathOperator{\pred}{{\mathsf{pred}}}
\newcommand{\0}{{\mathtt{0}}}

\newcommand{\iter}{{\mathtt{iter}}}
\newcommand{\rec}{\iter}
\newcommand{\Var}{{\mathtt{Var}}}
\DeclareMathOperator{\Varr}{Var}
\newcommand{\new}{{\mathtt{new}}}
\newcommand{\case}{{\mathtt{case}}}

\newcommand{\lmam}{\mathrel{\sqsubseteq_{m\&m}}}
\newcommand{\emam}{\mathrel{\equiv_{m\&m}}}
\newcommand{\lst}{\mathrel{\lesssim}}
\newcommand{\smam}{\mathrel{\sim_{m\&m}}}
\newcommand{\amam}{\mathrel{\approx_{m\&m}}}

\newcommand{\oes}{\sim}

%% Idealized Algol things

\newcommand{\com}{{\mathtt{com}}}
\newcommand{\skipp}{{\mathsf{skip}}}
\DeclareMathOperator{\seq}{{\mathsf{seq}}}
\DeclareMathOperator{\neww}{{\mathsf{new}}}
\DeclareMathOperator{\mkvar}{{\mathsf{mkvar}}}
\newcommand{\deref}{\texttt{@}}
\DeclareMathOperator{\dereff}{\mathsf{deref}}
\DeclareMathOperator{\assign}{\mathsf{assign}}
\newcommand{\ia}[2]{\langle #1 , #2 \rangle}
\newcommand{\stup}[3]{\langle #1 \mid #2 \mapsto #3 \rangle}

%% Hyland-Ong games things

\newbox\gnBoxA
\newdimen\gnCornerHgt
\setbox\gnBoxA=\hbox{$\ulcorner$}
\global\gnCornerHgt=\ht\gnBoxA
\newdimen\gnArgHgt
\def\pv #1{%
    \setbox\gnBoxA=\hbox{$#1$}%
    \gnArgHgt=\ht\gnBoxA%
    \ifnum     \gnArgHgt<\gnCornerHgt \gnArgHgt=0pt%
    \else \advance \gnArgHgt by -\gnCornerHgt%
    \fi \raise\gnArgHgt\hbox{$\ulcorner$} \box\gnBoxA %
    \raise\gnArgHgt\hbox{$\urcorner$}}
\def\ov #1{%
    \setbox\gnBoxA=\hbox{$#1$}%
    \gnArgHgt=\ht\gnBoxA%
    \ifnum     \gnArgHgt<\gnCornerHgt \gnArgHgt=0pt%
    \else \advance \gnArgHgt by -\gnCornerHgt%
    \fi \raise\gnArgHgt\hbox{$\llcorner$} \box\gnBoxA %
    \raise\gnArgHgt\hbox{$\lrcorner$}}
\newcommand{\ct}[1]{\lceil#1\rceil}
\DeclareMathOperator{\Int}{int}

%% Nondeterministic Factorization things

\newcommand{\code}{\mathsf{code}}
\newcommand{\Det}{\mathsf{Det}}

%% Flexible strategy things

\newcommand{\stle}{{\;\le_s\;}}
\newcommand{\steq}{{\;=_s\;}}
\newcommand{\exle}{\sqsubseteq}
\newcommand{\exlub}{\bigsqcup}
\newcommand{\dv}{{\text{\lightning}}}
\DeclareMathOperator{\pocl}{pocl}
\newcommand{\plot}{\mathrel{\triangleleft}}
\newcommand{\shad}{\mathfrak{S}}
%\newcommand{\tree}{\mathfrak{T}}
\newcommand{\Tau}{T}
\newcommand{\Epsilon}{E}
\newcommand{\sw}{\triangleleft}

%% Roman numerals

\newcommand{\RN}[1]{%
  \textup{\uppercase\expandafter{\romannumeral#1}}%
}
\newcommand{\RNl}[1]{%
  \mathrel{\raisebox{1pt}{$\overline{\underline{#1}}$}}
}

%% Game language things

\newcommand{\ul}[1]{{\underline{#1}}}
\newcommand{\A}{{\mathcal{A}}}
\renewcommand{\P}{\mathcal P}
\newcommand{\M}{\mathcal M}
\newcommand{\N}{\mathcal N}
\newcommand{\X}{\mathcal X}
\newcommand{\YY}{\mathcal Y}
\newcommand{\hole}{\blank}
\newcommand{\Tct}{\xrightarrow{T}t}
\newcommand{\teamconverge}[2]{\xrightarrow{#1}#2}

%% Inference rule things
\newcommand{\rulename}[1]{\LeftTirNameStyle{#1}}
\newcommand{\ts}{\mathbin{\vdash}}
\newcommand{\nts}{\mathbin{\not\vdash}}

%% Double category things
\newcommand{\hc}[2]{\left({#1}\middle|{#2}\right)}
\newcommand{\vc}[2]{\left(\frac{#1}{#2}\right)}

%% What is going on?
\DeclareMathOperator{\Kl}{Kl}
\DeclareMathOperator{\Mell}{Mell}
\newcommand{\powerset}{\mathcal P}
\DeclareMathOperator{\ask}{{\mathsf{ask}}}
\newcommand{\sleep}{{\mathsf{sleep}}}
\newcommand{\true}{\mathbbm{t}}
\newcommand{\false}{\mathbbm{f}}
\DeclareMathOperator{\If}{\mathsf{If}}
\newcommand{\Then}{\mathrel{\mathsf{then}}}
\newcommand{\Else}{\mathrel{\mathsf{else}}}
\newcommand\cat{\mathbin{+\mkern-10mu+}}

%% Profunctor arrows

\makeatletter
\def\slashedarrowfill@#1#2#3#4#5{%
  $\m@th\thickmuskip0mu\medmuskip\thickmuskip\thinmuskip\thickmuskip
   \relax#5#1\mkern-7mu%
   \cleaders\hbox{$#5\mkern-2mu#2\mkern-2mu$}\hfill
   \mathclap{#3}\mathclap{#2}%
   \cleaders\hbox{$#5\mkern-2mu#2\mkern-2mu$}\hfill
   \mkern-7mu#4$%
}
\def\rightslashedarrowfill@{%
  \slashedarrowfill@\relbar\relbar\mapstochar\rightarrow}
\newcommand\xslashedrightarrow[2][]{%
  \ext@arrow 0055{\rightslashedarrowfill@}{#1}{#2}}
\makeatother
\newcommand{\pto}{{\xslashedrightarrow{} }}

%% Profunctors 
\DeclareMathOperator{\Prof}{Prof}
\DeclareMathOperator{\End}{End}
\DeclareMathOperator{\Endoprof}{Endoprof}

%% Our

\def\searchmacro#1{
  \AtBeginOfFiles{%
    \ifdefined#1
      \expandafter\def\csname \currfilename:found\endcsname{}%
    \fi}
  \AtEndOfFiles{%
    \ifdefined#1
      \unless\ifcsname \currfilename:found\endcsname
        \immediate\write\finder{found in '\currfilename'}%
    \fi\fi}}

%% Isomorphism arrows on commutative diagrams %%
\tikzset{Isom/.style={every to/.append style={edge node={node [sloped, above, allow upside down, auto=false]{$\cong$}}}},
         Isom'/.style={every to/.append style={edge node={node [sloped, above, allow upside down, auto=false, rotate=180]{$\cong$}}}},
         Sim/.style={every to/.append style={edge node={node [sloped, above, allow upside down, auto=false]{$\sim$}}}},
         Sim'/.style={every to/.append style={edge node={node [sloped, above, allow upside down, auto=false, rotate=180]{$\sim$}}}}}

%% Adjunctions
\newcommand{\adjunction}[4]{%
  {#1} \underset{\underset{#3}{\longleftarrow}}{\overset{\overset{#2}{\longrightarrow}}{\bot}} {#4}}        

%% Important!
\newcommand\Mellies{Melli\`{e}s\xspace}

\makeatletter
\newcommand{\colim@}[2]{%
  \vtop{\m@th\ialign{##\cr
    \hfil$#1\operator@font colim$\hfil\cr
    \noalign{\nointerlineskip\kern1.5\ex@}#2\cr
    \noalign{\nointerlineskip\kern-\ex@}\cr}}%
}
\newcommand{\colim}{%
  \mathop{\mathpalette\colim@{\rightarrowfill@\textstyle}}\nmlimits@
}
\makeatother

\makeatletter
\newcommand{\laxcolim@}[2]{%
  \vtop{\m@th\ialign{##\cr
    \hfil$#1\operator@font colim_l$\hfil\cr
    \noalign{\nointerlineskip\kern1.5\ex@}#2\cr
    \noalign{\nointerlineskip\kern-\ex@}\cr}}%
}
\newcommand{\laxcolim}{%
  \mathop{\mathpalette\laxcolim@{\rightarrowfill@\textstyle}}\nmlimits@
}
\makeatother

\DeclareMathOperator{\Colim}{colim}

\DeclareMathOperator{\DG}{DG}
\DeclareMathOperator{\RV}{RV}
\newcommand{\Rv}{\catname{Rv}}

\let\choose\undefined
\DeclareMathOperator{\choose}{\mathsf{choose}}
\DeclareMathOperator{\tr}{tr}
\DeclareMathOperator{\test}{test}

%% Slot game things %%
\newcommand{\circled}[1]{\raisebox{.5pt}{\textcircled{\raisebox{-.9pt} {#1}}}}
\newcommand{\slot}{{\circled{\$}}}

\DeclareMathOperator{\may}{may}
\DeclareMathOperator{\must}{must}

\newcommand{\encode}[1]{\lceil#1\rceil}
\DeclareMathOperator{\app}{\mathsf{app}}
\DeclareMathOperator{\lett}{\mathsf{let}}
\newcommand{\inn}{\mathrel{\mathsf{in}}}
\DeclareMathOperator{\byval}{\mathsf{byval}}

\DeclareMathOperator{\rread}{read}
\DeclareMathOperator{\wwrite}{write}

\DeclareSymbolFont{bbsymbol}{U}{bbold}{m}{n}
\DeclareMathSymbol{\bbsemicolon}{\mathbin}{bbsymbol}{"3B}
\newcommand{\semicom}{\bbsemicolon}

\newcommand{\ms}{\makebox[-1pt]{}}

\DeclareMathOperator{\Acc}{Acc}
\DeclareMathOperator{\im}{Im}
\DeclareMathOperator{\wit}{wit}

%%% END Article customizations



\begin{document}

\subsection{Preliminaries}

\subsubsection{Surjective strategies}

Let $A$, $B$ be games.  
We say that a strategy $\sigma$ for $A\implies B$ is \emph{surjective} if for all positions $b\in P_B$ there exists some $s\in \sigma$ such that $s\vert_B=b$.  

The reason we are interested in surjective strategies from the point of view of nondeterminism is that composition with them preserves $\top$: if $\sigma\from A\implies B$ is surjective, then $\top_A;\sigma=\top_B$.  

A result of Hyland and Schalk \cite{hylandSchalkGames} sheds some further light on this definition:

\begin{proposition}[Hyland-Schalk functor]
  There is a functor $\G\to\Rel$ given by sending a game $A$ to its set $P_A$ of positions and sending a strategy $\sigma\from A\implies B$ to the set
  \[
    \grel{\sigma} = \{(s\vert_A,s\vert_B)\in P_A\times P_B\suchthat s\in^P\sigma\}
    \]
  This functor is lax monoidal with respect to the Cartesian product on $\Rel$ and the tensor product $\tensor$ on $\G$ and linearly distributive with respect to the finite multisets comonad $W$ on $\Rel$ and the exponential comonad $\oc$ on $\G$.  
  Moreover, the composite $\G_{det}\hookrightarrow\G\to\Rel$ is faithful.  
\end{proposition}

Using this result, we can rephrase our original definition: a strategy $\sigma$ is \emph{surjective} if its image $\grel\sigma$ under the Hyland-Schalk functor is a surjective relation. 

We will principally be interested in surjective strategies that are deterministic and total.  
If a deterministic surjective strategy $\sigma\from A\implies B$ is not total, then it introduces the possibility of divergence, and so we will no longer have the equation $\top_A;\sigma=\top_B$, since the composite on the left-hand side will contain divergent positions.  
The problem is that the composition of total strategies is not total in general.  
In order to get around this definition, we place a small restriction on strategies:

\begin{definition}
  \label{def:winning}
  Let $A,B$ be games.  
  Let $\sigma$ be a strategy for $A\implies B$.  
  We write $\bar\sigma$ for the closure of $\sigma$ inside $M_{A\implies B}^{\omega}$; i.e., the set of all possibly infinite sequences of moves, all of whose finite prefixes lie in $\sigma$.  

  We say that $\sigma$ is \emph{winning} if $\sigma$ is total and if whenever $s\in\bar\sigma$ is such that $s\vert_A$ is infinite, then $s\vert_B$ is infinite.  
\end{definition}

\begin{remark}
  It is then straightforward to prove that the composition of winning strategies is winning.  
  One way to see this is to observe that our definition says precisely that $\sigma$ is a \emph{winning strategy} in the sense of \cite{abramskyjagadeesangames} in the case that $W_A=P_A$ and $W_B=P_B$.  Then the category of games and winning strategies (in the sense of Definition \ref{def:winning}) is seen to be the full subcategory of the category of games and winning strategies from \cite{abramskyjagadeesangames} consisting of all those games $X$ that satisfy $W_X=P_X$.  

  This subcategory is not monoidal \emph{closed}, but neither is the category of games and surjective strategies, so we do not lose anything by adopting this definition.
\end{remark}

\begin{remark}
  Another, perhaps more compact definition of a winning strategy that does not require us to talk about infinite positions is as follows: for every play $s\in\sigma$ ending in an $O$-move from $B$, $s$ is a prefix of some $t\in\sigma$ ending in a $P$-move from $B$.
\end{remark}

\begin{proposition}
  There is a sub-monoidal-exponential category $\G_{sw}$ of $\G$ consisting of all those strategies that are deterministic, surjective and winning.
\end{proposition}

\begin{proof}
  The first step is to show that the composition of deterministic, surjective and winning strategies is deterministic, surjective and winning.

  \begin{itemize}
    \item That the composition of deterministic winning strategies is deterministic and winning is shown in \cite{abramskyjagadeesangames}.
    \item Preservation of surjectivity by composition comes down to the analogous result in $\Rel$, coupled with the functoriality of the Hyland-Schalk construction.
  \end{itemize}

  Therefore, we have a category (the identity is obviously deterministic, surjective and winning).

  We will ignore `deterministic' from here on, because it is well known that the category of games and deterministic strategies has all the structure we are talking about.  We will write `s.w. strategy' for `surjective winning strategy'.

  Let $A,B,C,D$ be games, let $\sigma$ be a s.w. strategy for $A\implies C$ and let $\tau$ be a s.w. strategy for $B\implies D$.  
  It is then easy to see that $\sigma\tensor\tau$ is a winning strategy for $A\tensor B\implies C\tensor D$.

  Next, we need to show that $\sigma\tensor\tau$ is surjective.  
  To do this, we use the monoidality of the Hyland-Schalk functor.  
  Indeed, we have a commutative diagram
  \[
    \begin{tikzcd}
      P_A \times P_B \arrow[r, "\grel{\sigma}\times\grel{\tau}"] \arrow[d, "m_{A,B}"']
        & P_C \times P_D \arrow[d, "m_{C,D}"] \\
      P_{A \tensor B} \arrow[r, "\grel{\sigma\tensor\tau}"]
        & P_{C \tensor D}
    \end{tikzcd}
    \]

  The relation $m_{C,D}$ is the one that relates $(s\vert_C,s\vert_D)$ to $s$ for all $s\in P_{C\tensor D}$.  
  As such, it is certainly surjective.  
  Moreover, since $\grel{\sigma}$ and $\grel{\tau}$ are both surjective relations, then their product is surjective.  
  It follows that $\grel{\sigma\tensor\tau}$ must be surjective, and therefore that $\sigma\tensor\tau$ is a surjective strategy.

  We use an almost identical argument to establish that the bang of a surjective strategy is surjective, this time using the linear distributivity of the Hyland-Schalk functor.  
  In this case, our argument rests on the fact that the coherence $\lambda_A \from W A \to P_{\oc A}$ is surjective; indeed, this is the relation that, for all plays $s\in P_{\oc A}$, relates $s$ with the finite multiset consisting of the restrictions of $s$ to the various subgames.

%  Let $e\in P_{C\tensor D}$.  
%  Then, since $\sigma$, $\tau$ are surjective, there exists some $s\in\sigma$, $t\in\tau$ such that $s\vert_C=e\vert_C$ and $t\vert_D=e\vert_D$.  
%
%  Moreover, since player $O$ switches games in $C\tensor D$, we know that one out of $e\vert_C$, $e\vert_D$ must end in a $P$-move.
%  Assume without loss of generality that $e\vert_C$ ends in a $P$-move, and that the final segment of $e$ consists of moves in $D$.
%  Then we can assume that $s$ ends in a $P$-move.  
%
%  We may now write:
%  \begin{align*}
%    &s' = c_1 a^*_1 C_1 c_2 a^*_2 C_2 \cdots c_m a^*_m C_m \\
%    &t' = d_1 b^*_1 D_1 d_2 b^*_2 D_2 \cdots d_n b^*_n D_n?
%  \end{align*}
%  where each $c_i$ is an $O$-move in $C$, each $C_i$ a $P$-move in $C$, each $d_j$ an $O$-move in $D$, each $D_j$ a $P$-move in $D$, each $a^*_i$ a (possibly empty) sequence of moves in $A$ and each $b^*_j$ a (possibly empty) sequence of moves in $B$.  
%  Lastly, the final move $D_n$ of $t'$ may or may not appear.
%
%  Now we know that $e$ is made up of some sequence of pairs of the form $c_iC_i$ or $d_jD_j$, in some order, and that it ends with either $d_nD_n$ (if $D_n$ is present) or $d_n$ (if not).  
%  If we replace each $c_iC_i$ in $e$ with $c_ia^*_iC_i$ and each $d_jD_j$ with $d_jb^*_jD_j$, then we end up with a valid sequence $e'\in\sigma\tensor\tau$, and we have $e'\vert_{C\tensor D} = e$.  
%  This completes the proof that the tensor product of surjective morphisms is surjective.
%
%  A similar argument shows us that if $\sigma$ is a s. w. strategy then its exponential $\oc\sigma$ is surjective and winning.  

  Lastly, we observe that if $A,B$ are games then the comultiplication $\oc A \to \oc\oc A$, dereliction $\oc A \to A$, weakening $\oc A\to I$, contraction $\oc A \to \oc A \tensor \oc A$ and monoidal coherence $\oc A\tensor \oc B \to \oc (A\tensor B)$ are all surjective, winning strategies.
\end{proof}

\subsubsection{The double category \MonCat}

The natural setting for much of the work we will do is the double category \MonCat, introduced in \cite{AdjointsInDoubleCategories}.  
A \emph{double category} is an internal category in $\Cat$, but it is more useful to think of it as a new sort of category made up of the following data:
\begin{itemize}
  \item It has \emph{objects}.
  \item It has two classes of \emph{morphisms}: the \emph{horizontal} morphisms and the \emph{vertical} morphisms, which go between objects.  
    Horizontal morphisms can be composed with other horizontal morphisms, and vertical morphisms with vertical morphisms.
  \item It has \emph{cells} $\alpha$, which are associated to squares of the form
    \[
      \begin{tikzcd}[row sep=large, column sep=large]
        A \arrow[r, "F", ""'{name=F}] \arrow[d, "g"']
          & B \arrow[d, "f"] \\
        C \arrow[r, "G"', ""{name=G}]
          & D
        \arrow[Rightarrow, from=F, to=G, "\alpha"]
      \end{tikzcd}
      \]
    The cell $\alpha$ is associated to the square in the same way that an ordinary morphism is associated to its source and target objects, so there may be more than one cell corrseponding to a particular square.  
    We may compose cells either horizontally or vertically, as long as the squares in question can be pasted together along an edge.  
    For example, horizontal composition is shown in Figure \ref{fig:horizontal-cell-composition}.
\end{itemize}

\begin{figure}
  \begin{mathpar}
    \begin{tikzcd}[row sep=large, column sep=large]
      A' \arrow[r, "F", ""'{name=F}] \arrow[d, "g"']
        & B' \arrow[r, "H", ""'{name=H}] \arrow[d, "f"]
          & C' \arrow[d, "h"] \\
      A \arrow[r, "G"', ""{name=G}]
        & B \arrow[r, "F'"', ""{name=Fprime}]
          & C
      \arrow[Rightarrow, from=F, to=G, "\alpha"]
      \arrow[Rightarrow, from=H, to=Fprime, "\beta"]
    \end{tikzcd}
    \and
    \begin{tikzcd}[row sep=large, column sep=large]
      A' \arrow[r, "F;H", ""'{name=FH}] \arrow[d, "g"']
        & C' \arrow[d, "h"] \\
      A \arrow[r, "G;F'"', ""{name=GFprime}]
        & C
      \arrow[Rightarrow, from=FH, to=GFprime, "(\alpha|\beta)"]
    \end{tikzcd}
  \end{mathpar}
  \caption{We can compose cells between compatible squares horizontally or vertically.  
  Here, we are composing cells horizontally.}
  \label{fig:horizontal-cell-composition}
\end{figure}

The particular double category we are going to be interested in is the double category $\MonCat$ of monoidal categories.  
The objects of $\MonCat$ are monoidal categories, horizontal morphisms are lax monoidal functors and vertical morphisms are oplax monoidal functors.  
The definition of a cell in $\MonCat$ is a bit less familiar, but appears in \cite{AdjointsInDoubleCategories}.  
  Let $\C'$,$C'$,$\D'$,$\D$ be monoidal categories, let $f\from\C'\to\C$, $g\from \D'\to \D$ be oplax monoidal functors and let $F\from\D'\to\C'$, $G\from\D\to\C$ be lax monoidal functors, as in Figure \ref{fig:generalized-monoidal-setup}.

  \begin{figure}
    \[
      \begin{tikzcd}[row sep=large, column sep=large]
        \D' \arrow[r, "F", ""'{name=F}] \arrow[d, "g"']
          & \C' \arrow[d, "f"] \\
        \D \arrow[r, "G"', ""{name=G}]
          & \C
        \arrow[Rightarrow, from=F, to=G, "\phi"]
      \end{tikzcd}
      \]
    \caption{A generalized monoidal natural transformation $\phi$ is a particular sort of natural transformation between composites of lax and oplax monoidal functors.}
    \label{fig:generalized-monoidal-setup}
  \end{figure}

We say that a natural transformation $\phi\from fF\bigto Gg$ is a \emph{generalized monoidal natural transformation} if the following two diagrams commute for all objects $X,Y$ of $\D'$:
\begin{mathpar}
  \begin{tikzcd}
    f F X \tensor f F Y \arrow[r, "\phi_X \tensor \phi_Y"]
      & G g X \tensor G g Y \arrow[d, "m_G"] \\
    f (F X \tensor F Y) \arrow[u, "m_f"] \arrow[d, "f m_F"']
      & G (g X \tensor g Y) \\
    f F (X \tensor Y) \arrow[r, "\phi_{X \tensor Y}"]
      & G g (X \tensor Y) \arrow[u, "G m_g"']
  \end{tikzcd}
  \and
  \begin{tikzcd}[column sep=small, row sep=small]
    %
      & f F I_{\D'} \arrow[rr, "\phi_I"]
        &[-1.2em]
          &[-1.2em] G g I_{\D'} \arrow[dr, "G \epsilon_g"]
            & \\
    f I_{\C'} \arrow[ur, "f \epsilon_F"] \arrow[drr, "\epsilon_f"]
      &
        &
          &
            & G I_{\D} \\
    %
      &
        & I_{\C} \arrow[urr, "\epsilon_G"]
          &
            &
  \end{tikzcd}
\end{mathpar}

We will, in fact, be more interested in the sub-double category $\SymmMonCat$ of $\MonCat$, whose objects are the symmetric monoidal categories, and where the horizontal and vertical morphisms are lax and oplax \emph{symmetric} monoidal functors.  
The definition of a generalized monoidal natural transformation is unchanged in the symmetric case.

\subsubsection{Multiplicative natural transformations}

We observed that if $\sigma$ is a surjective strategy from a game $A$ to a game $B$, then $\top_A;\sigma=\top_B$.  
In other words, $\top_A$ defines a natural transformation from $I$ to the inclusion of $\G_{sw}$ into our model of nondeterministic PCF.  
We shall see that this is in fact a generalized monoidal natural transformation.

Rather than use this natural transformation $I \to j(X)$ directly, however, it will be technically convenient to consider instead the related natural transformation from $A \to j(X) \tensor A$.

\begin{proposition}
  Let $\C',\C,\C''$ be monoidal categories, let $j$ be an oplax monoidal functor from $\C'$ to $\C$ and let $J$ be a lax monoidal functor from $\C$ to $\C''$.  Let $\phi_X\from I \to JjX$ be a generalized monoidal natural transformation and define a natural transformation $\phi_{X,A}^\infty\from JA \to J (jX \tensor A)$ to be given by the following composite:
  \[
    \begin{tikzcd}
      J A \arrow[r, "\lunit"]
        & I \tensor J A \arrow[r, "\phi_X\tensor J A"]
          & J j X \tensor J A \arrow[r, "m_J"]
            & J (j X \tensor A)
    \end{tikzcd}\,.
    \]
  Then $\phi^\infty$ makes the following diagrams commute:
  \begin{mathpar}
    \begin{tikzcd}
      J A \arrow[r, "\phi^\infty_{X,A}"] \arrow[d, "\phi^\infty_{Y\tensor X,A}"']
        & J (j X \tensor A) \arrow[r, "\phi^\infty_{Y,j X \tensor A}"]
          &[24pt] J (jY \tensor (j X \tensor A)) \\
      J (j (Y \tensor X) \tensor A) \arrow[rr, "J (m_j \tensor A)"]
        &
          & J ((j Y \tensor j X) \tensor A) \arrow[u, "J \assoc"']
    \end{tikzcd}
    \and
    \begin{tikzcd}
      JA \arrow[r, "\phi^\infty_{I,A}"] \arrow[dr, "\lunit"']
        & J (j I \tensor A) \arrow[d, "J(\epsilon_j\tensor A)"] \\
      %
        & I \tensor J A
    \end{tikzcd}\,.
  \end{mathpar}
\end{proposition}

We say that $\phi^\infty$ is a \emph{multiplicative natural transformation}.

\begin{definition}
  Let $\C',\C$ be monoidal categories and let $j$ be an oplax monoidal functor from $\C'$ to $\C$.  Let $\C''$ be a category and let $J$ be a functor from $\C$ to $\C''$.  
  Let $\psi_{X,A}\from J A \to J(j X \tensor A)$ be a natural transformation.  
  We say that $\psi$ is \emph{multiplicative} if the following diagrams commute for all objects $X,Y$ of $\C'$ and all objects $A$ of $\C$:
\end{definition}

One reason why we are interested in the rather strange-looking definition of a dimonoidal natural transformation is that they give rise to multiplicative natural transformations.

\begin{proposition}
  Let $\C',\C,\C''$ be monoidal categories, let $j$ be an oplax monoidal functor from $\C'$ to $\C$ and let $J$ be a lax monoidal functor from $\C$ to $\C''$.  
  Let $\phi_X\from I \to J j X$ be a dimonoidal natural transformation:
  \[
    \begin{tikzcd}[row sep=24pt, column sep=36pt]
      \C' \arrow[r, "j"] \arrow[d, "()"']
        & \C \arrow[d, "J"] \\
      I \arrow[r, "I"]
        \arrow[ur,
               Rightarrow,
               start anchor={[xshift=6pt, yshift=4pt]},
               end anchor={[xshift=-6pt, yshift=-4pt]},
               "\phi"]
        & \C''
    \end{tikzcd}\,.
    \]
  Define a natural transformation $\phi^\infty_{X,A}\from J A \to J (j X \tensor A)$ as follows:
  \[
    \phi^\infty_{X,A} =
    \begin{tikzcd}
      J A \arrow[r, "\lunit\inv"]
        & I \tensor J A \arrow[r, "\phi_X\tensor J A"]
          & J j X \tensor J A \arrow[r, "m_J"]
            & J (j X \tensor A)
    \end{tikzcd}
    \]
  Then $\phi^\infty$ is multiplicative.
\end{proposition}

We can also diagonally compose a multiplicative natural transformation with a dimonoidal natural transformation.

\begin{proposition}
  Let $\C'$, $\C$, $\C''$, $\D'$, $\D$ be monoidal categories, let $F'\from \D'\to\C'$, $G \from \D \to \C$ and $J\from \C\to\C''$ be lax monoidal functors and let $g\from D'\to D$ and $j\from \C' \to \C$ be oplax monoidal functors, as in Figure \ref{fig:dimonoidal-multiplicative}.

  \begin{figure}
    \begin{mathpar}
      \begin{tikzcd}[row sep=24pt, column sep=36pt]
        \D' \arrow[r, "g"] \arrow[d, "F"']
          & \D \arrow[d, "G"] \\
        \C' \arrow[r, "j"']
            \arrow[ur,
                   Rightarrow,
                   start anchor={[xshift=6pt, yshift=4pt]},
                   end anchor={[xshift=-6pt, yshift=-4pt]},
                   "\phi"]
          & \C
      \end{tikzcd}
      \and
      \begin{tikzcd}[row sep=24pt, column sep=36pt]
        \C'\times \C \arrow[r, "j\times\id"] \arrow[d, "\pr_2"']
          & \C \times \C \arrow[r, "\_\tensor\_"]
            & \C \arrow[d, "J"] \\
        \C \arrow[rr, "J"']
           \arrow[urr,
                  Rightarrow,
                  start anchor={[xshift=12pt, yshift=8pt]},
                  end anchor={[xshift=-12pt, yshift=-8pt]},
                  "\psi"]
          &
            & \C''
      \end{tikzcd}
    \end{mathpar}
    \caption{We can compose a multiplicative natural transformation $\psi_{X,A}\from J A \bigto J (j X \tensor A)$ with a dimonoidal natural transformation $\phi \from j F \bigto G g$.}
    \label{fig:dimonoidal-multiplicative}
  \end{figure}

  Let $\psi_{X,A}\from J A \bigto J (jX \tensor A)$ be a multiplicative natural transformation and let $\phi\from j F \bigto G g$ be a dimonoidal natural transformation.  
  Then the natural transformation $(\psi\bbslash\phi)_{X,A}\from J G A \bigto J G (g X \tensor A)$ defined by
  \[
    \begin{tikzcd}
      J G A \arrow[r, "\psi_{F X, G A}"]
        &[12pt] J (j F X \tensor G A) \arrow[r, "J (\phi_X \tensor G A)"]
          &[24pt] J (G g X \tensor G A) \arrow[r, "J m_G"]
            & J G (g X \tensor A)
    \end{tikzcd}
    \]
  is multiplicative.
\end{proposition}

\begin{remark}
  In the case that $\psi=(\phi')^\infty$, it is easy to check that 
  \[
    \psi\bbslash\phi = (\phi')^\infty \bbslash \phi = (\phi'\bbslash\phi)^\infty\,.
    \]
\end{remark}

\subsection{Cones}

\subsubsection{Nondeterminism and deterministic factorization}

In this section, we will write $\G_{ND}$ to refer to a putative model of nondeterministic PCF containing the usual game semantics model of PCF as a sub-linear category.  

We observed that the important feature of surjective winning strategies is that they preserve the nondeterministic oracle $\top$: if $\sigma\from A\implies B$ is surjective, then $\top_A;\sigma = \top_B$.  

Another way of expressing this is that $\top$ is a natural transformation in the following diagram:

\[
  \begin{tikzcd}
    \G_{sw} \arrow[r, hook] \arrow[d, "()"']
      & \G \arrow[d, hook] \\
    I \arrow[r, "I"'] \arrow[ur, "\top", Rightarrow]
      & \G_{ND}
  \end{tikzcd}
  \]

I.e., $\top$ is a natural transformation from the constant $I$ functor from $\G_{sw}$ to $\G_{ND}$ to the inclusion functor.

We will make one more assumption about the category $\G_{ND}$, which will capture the fact that $\G_{ND}$ is the smallest category obtained by adjoining the nondeterministic strategies $\top$ on to our original category $\G$.  

\begin{definition}[Factorization]
  Let $\C', \C, \C''$ be monoidal categories, let
  \[
    \begin{tikzcd}
      \C' \arrow[r, "j"]
        & \C \arrow[r, "J"]
          & \C''
    \end{tikzcd}
    \]
  be functors, where $J$ is lax monoidal, and let $\phi$ be a natural transformation from $I$ to $J\circ j$.

  Then we say that $\C''$ satisfies $\phi$-\emph{factorization} if:
  \begin{itemize}
    \item The functor $J$ is essentially surjective.  
    \item If $A$, $B$ are objects of $\C$, then every morphism from $JA$ to $JB$ in $\C''$ is of the form
      \[
        \begin{tikzcd}
          JA \arrow[r, "\lunit\inv"]
            & I \tensor JA \arrow[r, "\phi_X \tensor \id"]
              & JjX \tensor JA \arrow[r, "m_J"]
                & J(jX \tensor A) \arrow[r, "Jf"]
                  & JB
        \end{tikzcd}
        \]
      where $X$ is some object of $\C'$ and $f$ is some morphism in $\G$ from $jX\tensor A$ to $B$.
  \end{itemize}
\end{definition}

\begin{example}
  \begin{enumerate}
    \item In \cite{mcCHFiniteND}, Harmer and McCusker construct a game semantics model $\G_{fnd}$ of finite nondeterminism and prove that if $A$ is any game in that model, then every nondeterministic strategy $\sigma$ for $A$ can be written as the composition of some strategy $Det(\sigma)\from (\bN\to\bN)\to A$ with a fixed strategy $oracle\from I\to(\bN\to\bN)$.  
      
      In our setting, this is the case where $\C'=I$, $\C''$ is the category $\G_{fnd}$ and $\C$ is its subcategory of games and \emph{deterministic} strategies.  
      In this case, the functor $j$ is the (oplax monoidal) functor sending $I$ to $\bN\to\bN$, and the natural transformation $\phi$ has the strategy $oracle$ as its unique component.

    \item For a model of PCF with \emph{countable} nondeterminism, that is, PCF extended by a `deterministic oracle' $\wn\from\bN$, we might expect our category to satisfy a factorization rule such that every strategy might be written as the composition of some deterministic strategy with the denotation $\top_{\bN}$ of $\wn$.  

      In this case, we have $\C'=I$ again, $\C''$ is our desired model of PCF with countable nondeterminism and $\C$ is its subcategory of deterministic strategies.  
      The functor $j$ sends $I$ to $\bN$ and the unique component of the natural transformation $\phi$ is the strategy $\top_{\bN}$.  

    \item An example not relating to nondeterminism.  
      In the paper \cite{SamsonGuyIAPassive}, Abramsky and McCusker create a model of Idealized Algol based upon a category of games and \emph{visible} strategies.  
      One of the results they prove in the paper is that any visible strategy $\sigma\from I\to A$ may be written as the composition of an \emph{innocent} strategy $\tau\from \oc\Var[X]\implies A$ with the strategy $\cell_{X,x_0}\from I\implies\oc\Var[X]$, where $X$ is some set and $x_0\in X$.

      In this case, $\C'$ is the category of pointed sets, $\C$ is the category of games and innocent strategies and $\C''$ is the category of games and visible strategies.  
      The functor $J$ sends $(X, x_0)$ to the game $\Var[X]$, while the natural transformation $\phi$ is given by $\phi_{(X, x_0)} = \cell_{X,x_0}$.
  \end{enumerate}
\end{example}

We can phrase this factorization definition in terms similar to the ones we used above.

\begin{proposition}
  Let $\C',\C,\C''$ be monoidal categories and let $j\from \C'\to \C$ and $J\from \C\to \C''$ be functors, where $J$ is lax monoidal.  

  Let $\phi$ be a dimonoidal natural transformation from $I$ to $J\circ j$ and suppose that $\C''$ satisfies $\phi$-factorization.

  Now suppose that $\D$ is some other category and that we have a lax monoidal functor $\H\from \C \to \D$ and a dimonoidal natural transformation $\psi\from I \to \H\circ J$.  
  Suppose also that we have a lax monoidal functor $\F\from \D\to\C''$ such that $\F\circ I = I$ and $\F\circ\H=J$.

  Then $\F$ is essentially surjective and full.
\end{proposition}

\subsubsection{Cones}

Let $\C'$, $\C$ be monoidal categories and let $j\from \C'\to \C$ be an oplax monoidal functor.  

\begin{proposition}
  \label{prop:cone}
  There is a monoidal category $C_j$ together with a lax monoidal functor $J\from\C \to C_j$ and a dimonoidal natural transformation $\Phi$ from $I$ to $J\circ j$.  
  Moreover, $C_j$, $J$ and $\Phi$ are universal in the sense that if $\D$, $\H$ and $\psi$ are a similar triple, then there is a functor $\F\from C_j\to \D$ making ... commute.
\end{proposition}

\begin{proof}
  We first define the category $C_j$.  
  The objects of $C_j$ are the objects of $\C$.
  If $A,B$ are objects of $\C$, then the set of morphisms from $A$ to $B$ is the colimit of the diagram
  \[
    \begin{tikzcd}
      \oppcat{\C'} \arrow[r, "\oppcat j"]
        & \oppcat\C \arrow[r, "{\C[\blank\tensor A, B]}"]
          &[24pt] \Set
    \end{tikzcd}\,.
    \]
  That is, a morphism from $A$ to $B$ is a pair $(X, f)$, where $X$ is an object of $\C'$, $f$ is a morphism from $jX \tensor A \to B$ and $(X, f)$ and $(X', f')$ are considered to be equivalent if there is some morphism $h\from X'\to X$ in $\C'$ making the following diagram commute:
  \[
    \begin{tikzcd}
      j X' \tensor A \arrow[r, "f'"] \arrow[d, "j h \tensor A"']
        & B \\
      j X \tensor A \arrow[ur, "f"']
        &
    \end{tikzcd}\,.
    \]
  We define composition as follows: let $A,B,C$ be objects of $\C$, let $X,Y$ be objects of $C'$, and let $f\from X\to A \implies B$, $g\from Y\tensor B\to C$ be morphisms in $\C$, so that $(X,\sigma)$ is a morphism from $A$ to $B$ and $(Y,\tau)$ a morphism from $B$ to $C$ in $C_j$.
  Then the composition of $\sigma$ with $\tau$ is given by the pair $(Y\tensor X,f;g)$, where $f;g$ is the following composite:
  \[
    \begin{tikzcd}[cramped]
      j(Y\tensor X)\tensor A \arrow[r, "m_j\tensor A"]
        & (jY \tensor jX) \tensor A \arrow[r, "\assoc"]
          &[-2pt] jY \tensor (jX \tensor A) \arrow[r, "jY \tensor f"]
            & jY \tensor B \arrow[r, "g"]
              &[-10pt] C
    \end{tikzcd}\,.
    \]
  Well-definedness and associativity of this composition are by standard coherence theorems for monoidal categories and lax monoidal functors.

  We will not define the monoidal structure on $C_j$ yet.  
  Instead, we will first prove its universal property, and then use that to derive the monoidal structure from that of $\C'$ and $\C$.  

  First we define the functor $J\from \C\to C_j$ and the natural transformation $\Phi\from I\Rightarrow J\circ j$.  
  Given an object $A$ of $\C$, we set $J(A) = A$.  
  Given a morphism $f\from A \to B$ in $\C$, we set $J(f)$ to be the morphism $(I, \tilde{f})$ in $C_j$, where $\tilde{f}$ is the composite
  \[
    \begin{tikzcd}
      jI\tensor A \arrow[r, "\epsilon_j\tensor A"]
        & I\tensor A \arrow[r, "\lunit_A"]
          & A \arrow[r, "f"]
            & B
    \end{tikzcd}\,.
    \]

  If $X$ is an object of $\C'$, then we define a morphism $\Phi_X\from I \to JjX=jX$ to be given by $\Phi_X = (X, \phi_X)$, where $\phi_X$ is the morphism
  \[
    \begin{tikzcd}
      jX \tensor I \arrow[r, "\runit"]
        & jX
    \end{tikzcd}\,.
    \]

  We need to show that $\Phi$ is indeed a natural transformation.  
  In order to show that, let $X,Y$ be two morphisms and let $f\from X \to Y$ be a morphism.  
  We need to show that the following diagram commutes:
  \[
    \begin{tikzcd}
      I \arrow[r, "\phi_X"] \arrow[dr, "\phi_Y"']
        & J j X \arrow[d, "J j f"] \\
      %
        & J j Y
    \end{tikzcd}
    \]

  Computed inside the category $\C$, the top morphism is equal to the composite
  \[
    \begin{tikzcd}
      j (I \tensor X) \tensor I \arrow[r, "m_j\tensor I"]
        & (j I \tensor j X) \tensor I \arrow[r, "\assoc"]
          & j I \tensor (j X \tensor I) \arrow[r, "j I \tensor \runit"]
            & j I \tensor j X \arrow[r, "\epsilon_j\tensor j X"]
              & I \tensor j X \arrow[r, "\lunit"]
                & j X \arrow[r, "j f"]
                  & j Y
    \end{tikzcd}
    \]

  Suppose that we have a monoidal category $\D$, a monoidal functor $\H\from\C\to\D$ and a monoidal natural transformation $\psi\from I \to \H\circ j$.  
  We define a functor $\F\from C_j\to \D$ as follows: if $A$ is an object of $C_j$ then $A$ is, \emph{a priori}, an object of $\C$.  
  We set $\F(A) = \H(A)$.  

  Meanwhile, if $f\from jX \tensor A \to B$ is a morphism from $A$ to $B$ in $C_j$, then we set $\F(f)$ to be the composite
  \[
    \begin{tikzcd}
      \H A \arrow[r, "\lunit\inv"]
        & I \tensor \H A \arrow[r, "\psi_X\tensor \H A"]
          &[18pt] \H j X \tensor \H A \arrow[r, "m_\H"]
            & \H(jX\tensor A) \arrow[r, "Hf"]
              & \H B
    \end{tikzcd}\,.
    \]
  \begin{remark}
    Here we see the connection with factorization results: the functor $\F$ is, in effect, `substituting in' the special morphisms $\psi_X$ to the arbitrary hole $X$ that we have introduced into the formula.
  \end{remark}

  \begin{SidewaysFigure}
    \[
      \begin{tikzcd}[ampersand replacement=\&, row sep=48pt, column sep=36pt, labels={description}]
        HjX \tensor HA \arrow[rrr, "m_H"] \arrow[rrd, "\lunit\inv"]
          \&
            \&
              \& H(jX \tensor A) \arrow[r, "Hf"] \arrow[ddd, "\lunit\inv"]
                \& HB \arrow[ddd, "\lunit\inv"] \\[24pt]
        %
          \& I \tensor (I \tensor HA) \arrow[r, "I\tensor(\psi_X\tensor HA)"' auto] \arrow[d, "\assoc\inv"] \arrow[ddr, "\psi_Y\tensor(\psi_X\tensor HA)"]
            \& I \tensor (HjX \tensor HA) \arrow[dd, "\psi_Y\tensor(HjX\tensor HA)"] \arrow[ddr, "I\tensor m_H"]
              \&
                \& \\
        %
          \& (I \tensor I) \tensor HA \arrow[d, "(\psi_Y\tensor\psi_X)\tensor HA"]
            \&
              \&
                \& \\
        I \tensor HA \arrow[uuu, "\psi_X\tensor HA"] \arrow[uur, "\lunit\inv"] \arrow[ur, "\lunit\inv\tensor H_A" description] \arrow[d, "\psi_{Y\tensor X}\tensor HA"] \arrow[dr, phantom, "\kreuz" yshift=6pt]
          \& (HjY \tensor HjX) \tensor HA \arrow[r, "\assoc"] \arrow[d, "m_H\tensor HA"]
            \& HjY \tensor (HjX \tensor HA) \arrow[d, "HjY\tensor m_H"]
              \& I \tensor H(jX \tensor A) \arrow[dl, "\psi_Y\tensor\id"] \arrow[r, "I\tensor Hf" auto]
                \& I \tensor HB \arrow[dl, "\psi_Y\tensor HB"] \\
        Hj(Y \tensor X) \tensor HA \arrow[r, "H(m_j)\tensor HA" auto] \arrow[d, "m_H"]
          \& H(jY \tensor jX) \tensor HA \arrow[d, "m_H"]
            \& HjY \tensor H(jX \tensor A) \arrow[r, "HjY\tensor Hf" auto] \arrow[d, "m_H"]
              \& HjY \tensor HB \arrow[d, "m_H"]
                \& \\
        H(j(Y \tensor X) \tensor A) \arrow[r, "H(m_j\tensor A)" auto]
          \& H((jY \tensor jX) \tensor A) \arrow[r, "H(\assoc)" auto]
            \& H(jY \tensor (jX \tensor A)) \arrow[r, "H(jY\tensor f)" auto]
              \& H(jY \tensor B)
                \&
      \end{tikzcd}
      \]
    2q~
    2t
    \caption{A diagram showing that the functor $\F$ we defined in the proof of Proposition \ref{prop:cone} preserves composition.  
    The trapezium marked \kreuz{} commutes because $\psi$ is a dimonoidal natural transformation; commutativity of the other cells comes from the standard monoidal theory.}
  \end{SidewaysFigure}

  We need to check that $\F\circ J = \H$.  
  This is certainly the case on objects.  
  For morphisms, let $A,B$ be objects of $\C$ and let $f\from A\to B$ be a morphism.  
  Then $\F J(f)$ is the following composite:
  \[
    \begin{tikzcd}
      \H A \arrow[d, "\lunit\inv"']
        &
          &
            & 
              & \H B \\
      I\tensor \H A \arrow[r, "\psi_I\tensor\H A"]
        & \H j I \tensor \H A \arrow[r, "m_\H"]
          & \H(jI \tensor A) \arrow[r, "\H(jI \tensor A)"]
            & \H(I \tensor A) \arrow[r, "\H\lunit"]
              & \H A \arrow[u, "\H f"']
    \end{tikzcd}\,.
    \]
  In order to show that this is in fact equal to $\H f$, it is sufficient to show that the composition of the middle four morphisms is equal to the left unitor $\lunit_{\H A}$:
  \[
    \begin{tikzcd}
      \H A
        &
          & \H(I\tensor A) \arrow[ll, "\H(\lunit_A)"', squiggly] \\
      I \tensor \H A \arrow[u, "\lunit_{\H A}"] \arrow[r, "\epsilon_\H\tensor\H A"] \arrow[dr, "\psi_I\tensor\H A"', squiggly]
        & \H I \tensor \H A \arrow[ur, "m_\H"]
          & \\
      %
        & \H j I \tensor \H A \arrow[u, "\H\epsilon_j\tensor\H A"'] \arrow[r, "m_\H", squiggly]
          & \H (jI \tensor A) \arrow[uu, "\H(\epsilon_j\tensor A)"', squiggly]
    \end{tikzcd}\,.
    \]
  Here, commutativity of the two trapezia comes from the fact that $\H$ is a monoidal functor and commutativity of the lower-left triangle is because $\psi$ is a dimonoidal natural transformation.

\end{proof}

\section{Denotational Semantics}

Let $\G$ be our game semantics model of PCF and let $\bN_{sw}$ be the one-object subcategory on the object $\oc \bN$ where the morphisms are the surjective winning strategies $\oc\bN\to\oc\bN$ and the tensor product of strategies is given by their composition.  

Let $j\from \G_{sw}\to\G$ be the functor induced by the comonoid structure on $\oc \bN$.

Then the natural functor $J\from\G\to C_j$ induces a semantics for PCF on $C_j$.  

Now given some term $M\from T$ of nondeterministic PCF, we may write $M$ as $C[\wn]$, where $C$ is a PCF context, where the hole may appear multiple times.  
We define the denotation $\deno{M}=\deno{C[\wn]}$ to be the following composite in $C_j$:
\[
  \deno{C[\wn]} =
  \begin{tikzcd}
    I \arrow[r, "\runit"]
      & I \tensor I \arrow[r, "\Phi_{\oc \bN}\tensor \deno{\lambda n.C[n]}"]
        &[48pt] \oc \bN \tensor (\oc \bN \implies \deno{T}) \arrow[r, "\text{eval}"]
          & \deno{T}
  \end{tikzcd}\,.
  \]
In other words, we have taken the denotation of the term $\lambda n.C[n]$ and substituted in the strategy $\Phi_{\oc \bN}$, which is our representation of the nondeterministic choice $\wn$.

$\deno{C[\wn]}$ is a morphism from $I$ to $\deno{T}$ in $C_j$, but it is an equivalence class of morphisms from $\oc N$ to $\deno{T}$ in $\G$.  
A particular representative of that equivalence class is given by the denotation of the term $\lambda n.C[n]$.  
We call this representative the \emph{canonical presentation} of $\deno{M}$.  

\begin{lemma}
  \label{lem:sound-ad-lemma}
  Let $M=C[\wn]$ be a term of nondeterministic PCF of type $\nat$ and let $\pi$ be some evaluation of $M$.  
  Let $\sigma_M$ be the canonical presentation of $\deno{M}$; i.e., the denotation of $\lambda n.C[n]$.

  Then there is some total strategy $\sigma_\pi\from \oc\bN$ such that the composite $\sigma_\pi;\sigma_M$ captures the result of the evaluation $\pi$ as follows:
  \begin{itemize}
    \item If $\pi$ terminates at some value $m$, then $\sigma_\pi;\sigma_M$ is the strategy for $\bN$ that responds with the value $m$.
    \item If $\pi$ is infinite, then $\sigma_\pi;\sigma_M$ is the non-total strategy $\bot$ for $\bN$.
  \end{itemize}
\end{lemma}

\begin{proof}
  We use the fact that our base category $\G$ of games gives rise to a sound model for Idealized Algol.  
  For Hyland-Ong games, this model may be defined so that it is fully abstract \cite{SamsonGuyIAPassive}.  
  However, we will not require full abstraction for the model, so for Abramsky-Jagadeesan games we can use Laird's semantics for a PCF-like language with coroutines \cite{LairdCofCommCom}, since IA may be faithfully modelled within that language.

  Suppose first that $\pi$ is a finite evaluation.  
  Then $\pi$ gives rise to a finite sequence $n_1,\cdots, n_k$ of natural numbers corrseponding to the different choices of nondeterminism made during the evaluation of $\pi$.  
  We can now simulate the behaviour of $\wn$ during this evaluation by creating a (deterministic) device that will return these values in order each time it is called:
  \begin{align*}
    &N_{n_1,\cdots,n_k,d} \from (\nat \to \nat) \to \nat \\
    &N_{n_1,\cdots,n_k,d} = \lambda f . \new (\lambda v.v\coloneqq 0; f(v\coloneqq(\suc\deref v); \case_{k+1}\;\deref v\;\Omega\;n_1\;\cdots\;n_k\;d))\,.
  \end{align*}
  Here, $\case_k\;a\;n_0\;\cdots\;n_k\;d$ evaluates to $n_i$ if $a$ evaluates to $i$ and evaluates to $d$ if $a$ evaluates to a value greater than $k$.  
  This function maintains a new local variable $v$, which it increments each time it is called before looking up the appropriate value in the lookup table.

  By working through the construction of the denotational semantics of $\new$ as in \cite{SamsonGuyIAPassive}, we see that, for any term $F\from\nat\to\nat$, the denotation of $N_{n_1,\cdots,n_k,d}M$ is given by the composite
  \[
    I \xrightarrow{\cell_{\bN,0}}
      \oc \Var[\bN] \xrightarrow{\deno{\lambda v.v\coloneqq (\suc \deref v);\;\case_{k+1}\;\deref v\;\Omega\;n_1\;\cdots\;n_k\;d}}
        \oc \bN \xrightarrow{\deno{F}}
          \bN
    \]

  We write $\sigma_\pi\from I\to \oc\bN$ for the composite of the two arrows on the left.  
  Here, $d$ is some arbitrary natural number.  
  Its value does not affect the the composition $\sigma_\pi;\sigma_M$ and it is only included to satisfy the requirement that $\sigma_\pi$ be a total strategy.  
  Plays in $\sigma_\pi$ take the form
  \[
    q\;n_1\;q\;n_2\;\cdots\;q\;n_k\;q\;d\;q\;d\;\cdots
    \]
  or some subsequence thereof.  

  Now $\sigma_\pi;\sigma_M$ is the denotation of the term $N_{n_1,\cdots,n_k}(\lambda n.C[n])$, which corresponds to evaluation of the term-in-context
  \[
    v\from\Var[\bN]\vdash v\coloneqq 0;\;C[v\coloneqq (\suc\deref v); \case_{k+1}\;\deref v\;\Omega\;n_1\;\cdots\;n_k\;d]\,.
    \]
  This term evaluates just as $C[\wn]$ does, except that we use the device we have constructed to simulate the nondeterministic numbers.  
  By following the evaluation of this term in parallel with the evaluation $\pi$ of $C[\wn]$, we can see that both evaluations have the same behaviour: they either both converge to the same number, or they both diverge.  
  By Soundness and Computational Adequacy for the model of Idealized Algol, this means that $\sigma_\pi;\sigma_M$ is the strategy for $\bN$ that captures the eventual behaviour of the evaluation path $\pi$.

  \begin{remark}
    This is the same as the common Software Engineering practice of `mocking' a random number generator for testing purposes by forcing it to always produce the same sequence of numbers.
  \end{remark}

  In the case that $\pi$ evaluates $\wn$ infinitely many times, producing numbers $n_1,n_2,\cdots$, we define approximants $N_{n_1,\cdots,n_k}$, which produce the numbers $n_1$ up to $n_k$ and then diverge the next time they are called.  
  We set $\sigma_\pi^{(k)}$ to be the denotation of $N_{n_1,\cdots,n_k,\Omega}$, considered as a strategy $I \to \oc \bN$ as above.  
  This time, the mocked nondeterminism will answer with $n_k$ each of the first $k$ times it is called, but will diverge on the $(k + 1)$-th time.
  Since $\pi$ evaluates $\wn$ infinitely many times, this means that $\sigma_\pi^{(k)};\sigma_M=\bot$.

  Set $\sigma_\pi$ to be the supremum of the $\sigma_\pi^{(k)}$, under the inclusion order.  
  Then, since composition is continuous with respect to inclusion in deterministic Game Semantics, $\sigma_\pi;\sigma_M$ must also be the divergent strategy $\bot$.

  It remains to show that $\sigma_\pi$ is a total strategy.  
  We can see this combinatorially by observing that each $\sigma_\pi^{(k)}$ has unique maximal play
  \[
    q\;n_1\;\cdots\;q\;n_k\;q\,,
    \]
  after which point the strategy has no reply.  
  This means that the supremum $\sigma_\pi$ has unique maximal play
  \[
    q\;n_1\;q\;n_2\;\cdots\,,
    \]
  which is infinite, and so the strategy is total.
\end{proof}

From this lemma, we may deduce Soundness and Computational Adequacy for our model.

\begin{proposition}[Soundness]
  Let $M=C[\wn]\from \nat$ be a term of nondeterministic PCF and let $\sigma\from\oc \bN \to\bN$ be its denotation.  If $M\mustconverge$ then $\sigma$ is a winning strategy.
\end{proposition}

\begin{proof}
  Suppose that $\sigma$ is not a winning strategy.  
  Then there is some play $qs\in\sigma$ such that $qstn\not\in\sigma$ for any sequence $t$ of moves in $\oc\bN$ and any $n\in\bN$.  
  This sequence $s$ is contained inside some strategy $\sigma_\pi$ for $\oc\bN$, which can be viewed, as in the proof of Lemma \ref{lem:sound-ad-lemma}, as capturing the behaviour of a particular evaluation path $\pi$ of the term $M$.  
  Clearly, we have $\sigma_\pi;\sigma=\bot$, and so Lemma \ref{lem:sound-ad-lemma} tells us that the evaluation path $\pi$ must diverge.  
  So $M\not\mustconverge$.
\end{proof}

\begin{proposition}[Computational Adequacy]
  Let $M\from \nat$ be a term of nondeterministic PCF and let $\sigma\from\oc \bN \to\bN$ be its denotation.  If $\sigma$ is a winning strategy then $M\mustconverge$.
\end{proposition}

\begin{proof}
  Suppose that $\sigma\from\oc \bN\to \bN$ is a winning strategy.  
  Let $\pi$ be an evaluation path for $M$.  
  We claim that $\pi$ converges to some value.  
  Indeed, there is some strategy $\sigma_\pi$ such that $\pi$ converges to a value if and only if $\sigma_\pi;\sigma$ is a winning strategy.  
  But $\sigma_\pi;\sigma$ is the composition of winning strategies, so it is winning.
\end{proof}

\bibliographystyle{alpha}
\bibliography{../common/phd_bibliography}

\end{document}
