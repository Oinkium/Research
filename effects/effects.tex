%%%%%%%%%%%%%%%%%%%% author.tex %%%%%%%%%%%%%%%%%%%%%%%%%%%%%%%%%%%
%
% sample root file for your "contribution" to a proceedings volume
%
% Use this file as a template for your own input.
%
%%%%%%%%%%%%%%%% Springer %%%%%%%%%%%%%%%%%%%%%%%%%%%%%%%%%%


\documentclass{svproc}
%
% RECOMMENDED %%%%%%%%%%%%%%%%%%%%%%%%%%%%%%%%%%%%%%%%%%%%%%%%%%%
%

% to typeset URLs, URIs, and DOIs
\usepackage{url}
\usepackage{amsmath}
\usepackage{amssymb}
\usepackage{tikz}
\usepackage{xspace}
\usepackage{IEEEtrantools}
\usetikzlibrary{cd}
\usepackage{mathpartir}
\def\UrlFont{\rmfamily}

\newcommand\C{\mathcal{C}}
\newcommand\D{\mathcal{D}}
\newcommand\F{\mathcal{F}}
\newcommand\G{\mathcal{G}}
\newcommand\M{\mathcal{M}}
\DeclareMathOperator\pr{pr}
\DeclareMathOperator\id{id}
\newcommand{\suchthat}{\,\colon\,}
\newcommand{\esuchthat}{\,.\,}
\newcommand\from{\,\colon\,}
\newcommand\object\colon
\DeclareMathOperator{\ac}{ac}
\DeclareMathOperator{\End}{End}
\newcommand{\passoc}{\texttt{passoc}}
\newcommand{\assoc}{\texttt{assoc}}
\newcommand\tensor\otimes
\newcommand\run{\texttt{r}}
\newcommand\lun{\texttt{l}}
\newcommand\lunit{\texttt{lunit}}
\newcommand\runit{\texttt{runit}}
\renewcommand\implies\multimap
\newcommand{\pfc}{\boldsymbol{\cdot}}
\newcommand\Mellies{Melli\`{e}s\xspace}

\newcommand{\catname}[1]{\mathbf{#1}}
\newcommand{\Set}{\catname{Set}}
\newcommand{\Surj}{\catname{Surj}}
\newcommand{\Ret}{\catname{Ret}}
\newcommand{\Cat}{\catname{Cat}}
\newcommand{\powerset}{\mathcal P}

\DeclareMathOperator{\Prof}{Prof}
\DeclareMathOperator{\Kl}{Kl}

\makeatletter
\newcommand{\colim@}[2]{%
  \vtop{\m@th\ialign{##\cr
    \hfil$#1\operator@font colim$\hfil\cr
    \noalign{\nointerlineskip\kern1.5\ex@}#2\cr
    \noalign{\nointerlineskip\kern-\ex@}\cr}}%
}
\newcommand{\colim}{%
  \mathop{\mathpalette\colim@{\rightarrowfill@\textstyle}}\nmlimits@
}
\makeatother

\begin{document}
\mainmatter              % start of a contribution
%
\title{A Kleisli-like construction for Parametric Monads, with Examples}
%
\titlerunning{Dum-di-dum}  % abbreviated title (for running head)
%                                     also used for the TOC unless
%                                     \toctitle is used
%
\author{}
%
\authorrunning{} % abbreviated author list (for running head)
%
%%%% list of authors for the TOC (use if author list has to be modified)
\tocauthor{}
%
\institute{University of Bath, Claverton Down Road, Bath.  
BA2 7AY,\\
\email{wjg27@bath.ac.uk},\\ WWW home page:
\texttt{http://people.bath.ac.uk/wjg27}}

\maketitle              % typeset the title of the contribution

\begin{abstract}
  A well-known principle in denotational semantics is that adding an effect to a categorical model should correspond to the Kleisli category construction for some suitable monad.  
  Our point of departure is the observation that much recent work in semantics, particularly in game semantics, has broken away from this principle by modelling effects using categories that are not Kleisli categories.  
  Our aim is to extend the theory of monadic effects to these models in a systematic way.

  An important part of our approach will be to study parametric monads, otherwise known as lax actions of monoidal categories.
  These generalize monads: a monad is precisely an action of the trivial category.
  We introduce a construction on parametric monads which takes a parametric monad on a category and gives us back a new category.  
  As with the Kleisli category, the objects of this category will be the objects of the original category, while the morphisms are obtained from morphisms in the original category in a formal manner brought about by the action.  
  In the case that the action is actually a monad, we get the usual Kleisli category.

  We show that this new construction is closely related to existing models of effects in game semantics, allowing us to reason systematically about adequacy and full abstraction proofs.
  We illustrate the relevance of the construction by showing that a special case gives us a fully abstract game semantics for Probabilistic Algol, which can be related to the existing model given by Danos and Harmer.
\keywords{denotational semantics, category theory, game semantics}
\end{abstract}

\section{Introduction}

\subsection{Monads and Kleisli categories}

This paper is about general frameworks for modelling effects in a categorical semantics.  
The seminal paper in this field is Moggi's \emph{Computational Lambda-Calculus and Monads} \cite{Moggi}, which first put forward the idea that if $\C$ is a categorical model of a programming language, then we can simulate an effect in that language via a suitable \emph{monad} on $\C$.  
Given such a monad, its Kleisli category $Kl_M\C$ is then a model of a language formed by adding the effect to the original language.  

Spivey \cite{Spivey} and Wadler \cite{Wadler1,Wadler2} have transferred this theory to the real world by using monads as programming tools, particularly in the Haskell programming language.  

Recent progress in the field of game semantics, however, has shown that monads and Kleisli categories are not the be-all and end-all of modelling effects.  
There are perfectly game semantic models of effects such as state and nondeterminism that do not arise as the Kleisli category for a monad on some more basic category of games.  
These models are typically built up using intuition -- for example, by relaxing the determinism condition on strategies to give a model of a nondeterministic language -- and do not readily fit into any category theoretic framework the way Kleisli categories do.  
The goal of this paper is to examine some possible frameworks that will allow us to study these and other models in a systematic way.  

A closely related problem is that of using a common technique to model a particular effect in different settings.  
Many of our favourite monads (e.g., the powerset monad on $\Set$ for nondeterminism) live in some particular category and cannot readily be set up in other categories. 

Before we start, it is worth mentioning one example of a Kleisli category arising naturally in game semantics.  
If we read Ghica's work on \emph{slot games} \cite{SlotGames}, it is not difficult to show that the category he constructs is in fact isomorphic to the Kleisli category for the \emph{reader monad} $\mathbb{C}\to \_$, where $\mathbb{C}$ is the game corresponding to a singleton set.  

Since the reader monad is entirely built out of the Cartesian closed structure of the category, we can perform a similar trick in many different settings, although perhaps not as successfully as in game semantics.  

\subsection{Promonads}

It is instructive at this point to see what happens if we generalize from functors to profunctors.  
We define a \emph{promonad} on a category $\C$ to be a monoid in the monoidal category $\Prof[\C,\C]$ of endo-profunctors on $\C$.  
This concept generalizes that of a monad, since an endofunctor $M\from \C\to\C$ may be identified with the profunctor $\C[\_,M(\_)]\from \C^{op}\times\C\to\Set$.

Since the set $\C[A,MB]$ is precisely the set of Kleisli morphisms from $A$ to $B$, it is natural to wonder whether we can perform a Kleisli-type construction on a promonad as well.  
it turns out that we can: if $\D\from \C^{op}\times\C\to\Set$ is a promonad, then we may define a category (also called $\D$) whose objects are the objects of $\C$, and where the morphisms from $A$ to $B$ are the elements of the set $\D(A,B)$.
The promonad action
\[
  (\D\pfc\D)(A,C) = \int^{B}\D(A,B)\times\D(B,C) \to \D(A,C)
  \]
gives us composition in our category and the unit
\[
  1(A,B)=\C[A,B]\to \D[A,B]
  \]
gives us a functor $\C\to\D$ (and, in particular, the identity morphisms).

In fact, the converse is true: if $\C$ and $\D$ are categories with the same objects and $J\from \C\to\D$ is an identity-on-objects functor, then the functor $\D(JA,JB)=\D(A,B)\from\C^{op}\times\C\to\Set$ may be given the structure of a promonad on $\C$.  
Thus, the rather complicated definition of a promonad is in fact equivalent to the much simpler idea of an identity-on-objects functor.
We shall therefore use the word `promonad' to mean both things.

In one sense this is very exciting for us, because we can see that promonads generalize not only monads, but also a very large number of the models developed in Game Semantics.  
For example, the inclusion of the category of games and deterministic strategies into the category of games and nondeterministic strategies is the identity on objects, so it is a promonad.  
So is the inclusion of the category of games and innocent strategies into the category of games and visible strategies.  

More generally, this fits in with the idea that effects should not normally create new types, but should instead give us more flexibility when it comes to writing programs -- exactly what is modelled by an identity-on-objects functor.

On the other hand, the sheer generality of promonads means that they don't give us any ways to actually create new models, since the data required to specify one is the entire category that we are trying to build.

One way of describing the purpose of this work, then, is to discover classes intermediate between monads and promonads that are more general than pure monads but can still be specified in a concise way that allows us to transfer them between different settings.

\subsection{Monads on presheaf categories}

We can think of an endo-profunctor $\F\from \C^{op}\times\C\to \Set$ as a functor $\C\to [\C^{op},\Set]$.  
Since the Yoneda embedding exhibits $[\C^{op},\Set]$ as the free cocompletion of $\Set$, this can alternatively be characterized as a colimit-preserving endofunctor on $[\C^{op},\Set]$.  
Similarly, any promonad on $\C$ can alternatively be considered as a monad on $[\C^{op},\Set]$ whose underlying endofunctor preserves colimits.

\subsection{Parametric monads}

\Mellies \cite{ParametricMonads}, following work by Smirnov \cite{Smirnov2008}, has demonstrated that the concept of a lax monoidal category action is a useful generalization of that of a monad.  
A lax action of a monoidal category $\M$ upon a category $\C$ is a lax monoidal functor $\M\to \End[\C]$; monads are the special case when $\M$ is trivial.  
For this reason, \Mellies renames lax monoidal actions to `parametric monads', where we think of the action of $\M$ as a kind of monad on $\C$ that is parameterized by the objects of $\M$.

This idea has been further generalized by Katsumata \cite{Katsu}, who has given a definition of a `Kleisli-like resolution' of a parametric monad.  
This particular construction, which takes a parametric monad on a category $\C$ parameterized by a monoidal category $\M$ and yields a new category, generalizes many important properties of the Kleisli category -- in particular, it gives us an adjunction with the original category -- but it does not fit into our framework, since the objects of the category it creates are not the objects of the original category $\C$, but pairs of an object of $\C$ and an object of $\M$.

Once again, it is useful to generalize to profunctors.  
First, if $\C$ is a category and $\M$ a monoidal category, we define a \emph{parametric promonad on $\C$ parameterized by $\M$} to be a lax monoidal functor $\M\to\Prof[\C,\C]$.

As with promonads, we can now look at what we have defined and try to recast it as a particular kind of category.  
Suppose that $\D\from \M\times\C^{op}\times\C$ is a parametric promonad on $\C$ parameterized by $\M$.  
After some thought, we can see that $\D$ corresponds to a category whose objects are the objects of $\C$ and where the morphisms from $A$ to $B$ are labelled by objects of $\M$ in such a way that the composition of a morphism labelled with $X$ and a morphism labelled with $Y$ is a morphism labelled with $X\tensor Y$. 

One way to state this is as follows.  
Given a parametric promonad $\D\from \M\times\C^{op}\times\C\to \Set$, we define a bicategory $\D^2$:
\begin{itemize}
  \item whose objects are the objects of $\C$,
  \item where the $1$-morphisms from $A$ to $B$ are pairs $(X,f)$, where $X$ is an object of $\M$ and $f\in\D(X,A,B)$
  \item and where the $2$-morphisms from $(X,f)$ to $(Y,g)$ are morphisms $h\from X\to Y$ in $\M$ such that $\D(h,A,B)(f)=g$.
\end{itemize}
This bicategory comes equipped with two important bifunctors:
\begin{itemize}
  \item a bifunctor $J\from \C\to\D^2$ (where we consider $\C$ as a bicategory with only identity $2$-morphisms;
  \item a bifunctor $l\from \D^2\to \M$ given by the `labelling' (considering $\M$ as a bicategory with a single object).
\end{itemize}

In order to get a promonad resolution of the parametric promonad, then, we need to convert this bicategory into a $1$-category $\D$, which we do by quotienting out the $1$-morphisms in $\D^2$ by the action of the $2$-morphisms: i.e., the smallest equivalence relation such that two $1$-morphisms are related if there is a $2$-morphism from one to the other.

More explicitly, the set of $1$-morphisms from $A$ to $B$ in our new category will be given by the following colimit.
\[
  \D(A,B) = \colim_{X\object\M} \D(X,A,B)
  \]
This category $\D$ now admits an identity-on-objects functor from $\C$, meaning that we have found a natural collapse from parametric promonads to ordinary promonads.  

In the particular case that the parametric promonad is in fact a parametric \emph{monad}, then we will call the category we obtain $\C/\M$.  
The morphisms in $\C/\M$ are given by
\[
  \C/\M(A,B)=\colim_{X\object\M} \C(A,X.B)\,,
  \]
with the composition of $f\from A\to X.B$ and $g\from B\to Y.C$ being given by the composite
\[
  A \xrightarrow{f}
  X.B \xrightarrow{X.g}
  X.Y.C \to
  (X\tensor Y).C\,.
  \]
Note that if $\M$ is the trivial category (so that the action is in fact a monad), then the category $\C/\M$ is the usual Kleisli category.

\subsection{Effects, monads and colimits}

The colimit that appears in the definition of the category $\C/\M$ is instructive, because it is linked to the fact that many important monads may be expressed as colimits.  
For example, let $\Surj$ be the category of sets and surjections.  
Then, if $A$ is a set, we have
\[
  \powerset(A) \cong \colim_{X\object\Surj^{op}} [X,A]\,,
  \]
naturally in $A$.

This is useful, because the body of the colimit -- i.e., the function set $[X,A]$ -- has a natural equivalent inside any Cartesian closed category that admits a functor out of $\Set$.  
Moreover, the functor $[X,A]$ is actually an \emph{action} of $\Surj^{op}$ (with the Cartesian product) upon the category of sets.  
So even though we cannot define the powerset \emph{monad} inside arbitrary Cartesian closed categories, we can still define this action of $\Surj^{op}$.  

Unfortunately, when we try to apply our construction to this action in the category of sets, we do not get the usual Kleisli category for the powerset monad.  
To see why, note that from the discussion above, we have
\begin{IEEEeqnarray*}{rCl}
  \Set/\Surj^{op}(A,B) & = & \colim_{X\object\Surj^{op}}[A,[X,B]] \\
  & \cong & \colim_{X\object\Surj^{op}}[X,[A,B]] \\
  & \cong & \powerset([A,B])\,,
\end{IEEEeqnarray*}
while the morphisms in the Kleisli category take a different form:
\[
  \Kl_{\powerset}\Set(A,B) = [A,\powerset(B)]\,.
  \]
Nevertheless, there are still a few things we can take from this.  
The sets $\powerset([A,B])$ and $[A,\powerset(B)]$ are not equal, but there is still a natural function
\[
  \powerset([A,B]) \to [A,\powerset(B)]\,,
  \]
that sends a set $\F$ of functions to the function $h$ (the \emph{amalgamation} of $\F$) where
\[
  h(a) = \{f(a)\suchthat f \in \F\}\,.
  \]
This amalgamation function is not \emph{quite} surjective: for example, suppose that $g\from \{0,1\}\to \powerset(\{0\})$ sends $0$ to $\{0\}$ and $1$ to the empty set.  
Then $g$ is not the amalgamation of any set of functions $\{0,1\}\to\{0\}$.  
However, if we use the \emph{nonempty powerset monad} $\powerset^+$ (which corresponds to taking the colimit over the opposite of the category $\Surj^+$ of nonempty sets and surjections), then the amalgamation function gives us a surjection
\[
  \powerset^+([A,B])\twoheadrightarrow [A,\powerset^+(B)]\,.
  \]
This means that we can recover the Kleisli category for the nonempty powerset monad on $\Set$ by taking the quotient by a suitable equivalence relation to the category $\Set/(\Surj^{+\,op}$).  

In fact, this surjection has a natural right inverse, formed by sending a function $h\from A\to\powerset^+(B)$ to the \emph{largest} possible set $\F\subseteq[A,B]$ such that for all $f\in \F$ and $a\in A$, $f(a)\in h(a)$.
The composition of these inverses sends a set of functions $A\to B$ to its `amalgamation closure':
\[
  \ac(\F) = \{g\from A \to B\suchthat\forall a\in A\esuchthat g(a)=f(a)\text{ for some $f\in \F$}\}\,.
  \]
By restricting our set of morphisms, then, to those that are fixed points of this operator, we may recover the Kleisli category for $\powerset^+$.  

In the next section, we will see how to generalize this idea to other monads, and to other categories.

\subsection{Weakly filtered colimits}

Suppose we have a lax action of a monoidal category $\M$ upon a cocomplete category $\C$.  
Then we may copy the collapse from parametric promonads to promonads and define a monad on $\C$ given by:
\[
  M A = \colim_{X\object\M}A.X\,.
  \]
Unfortunately, as in the case of the powerset monad above, this transformation does not commute with the collapse from parametric promonads to promonads.  
Indeed, if we consider our action as a parametric promonad and apply the collapse, then we get a category where the morphisms from $A$ to $B$ are given by the colimit
\[
  \colim_{X\object\M} \C(A,X.B)\,,
  \]
while the morphisms from $A$ to $B$ in the Kleisli category for the monad $M$ are given by
\[
  \C(A,\colim_{X\object\M} X.B)\,.
  \]
There is a natural morphism from the first colimit into the second expression, but it is not an isomorphism in general.
If $\C$ is the category of sets, there is a well known result that states that if $\M$ is a $\kappa$-filtered category and $|A|<\kappa$, then the colimit will commute with the functor $[A,\_]$, but this does not help us, since the category $\M$ is not filtered in most of the cases we are interested in (e.g., the category $\Surj^{op}$ has not got morphisms coequalizing parallel pairs, since such maps cannot be chosen to be surjective in general).  

Instead, we try for the next best thing, which is to show that the natural map $\colim_{X\object\M}\C(A,X.B)\to\C(A,\colim_{X\object\M}X.B)$ is a surjection if $\M$ satisfies some weaker conditions.  

We define (see, for example, \cite[Def. 16]{WeaklyFiltered}) a \emph{$\kappa$-weakly filtered category} to be one in which every \emph{discrete} diagram of size $<\kappa$ admits a cocone; i.e., a non-empty category such that whenever $(X_\alpha)_{\alpha<\kappa}$ is a collection of objects we have some object $Y$ and morphisms $X_\alpha\to Y$ for each $\alpha$.  

\begin{example}
  The category $\Surj^{+\,op}$ is $\kappa$-weakly filtered for arbitrary $\kappa$, because if $(X_\alpha)_{\alpha<\kappa}$ is a collection of sets, then we can take their Cartesian product, and the projections will be surjective.

  The category $\Surj^{op}$ is not weakly filtered, because if there is a surjection from an object $X$ to the empty set, then there is no surjection from $X$ to any other set.
\end{example}

The importance of $\kappa$-weakly filtered limits is shown by the following result.

\begin{proposition}
  \label{prop:weakly-filtered}
  If $\M$ is a $\kappa$-weakly filtered category, $D\from \M\to \Set$ is a functor and $A$ is a set with cardinality less than $\kappa$, then the natural function
  \[
    \colim_{X\object\M}[A,D(X)]\to [A,\colim_{X\object\M}D(X)]
    \]
  is a surjection.
\end{proposition}
\begin{proof}
  Given a function $h\from A \to \colim_{X\object\M}D(X)$, for each $a\in A$, choose some $X_a\object\M$ such that $h(a)\in D(X_a)$.  
  Since $\M$ is $\kappa$-weakly filtered, there is some object $Y$ of $\M$ that admits morphisms out of all of the $X_a$, and therefore $h$ may be considered as a function $A \to D(Y)$. \qed
\end{proof}

By setting $D(X)=X.B$, this means that if we have an action of $\M$ on $\Set$, where the monoidal category $\M$ is $\kappa$-weakly filtered for any $\kappa$, then we get a full functor $\Set/\M\to\Kl_M\Set$.

Moreover, if the colimit satisfies additional properties, then the $Y$ in the proof of Proposition \ref{prop:weakly-filtered} can be chosen in a functorial manner, so that we get a faithful right inverse to this full functor.

\begin{example}
  If $f\from A \to \colim_{X\object\Surj^{+\,op}}[X,B]$ is a function with $f(a)\in [X_a,B]$ for each $a$, then $f$ is equivalent to a function
  \[
    \tilde{f}\from A \to \left[\prod_{a\in A}X_a,B\right]
    \]
  via the surjections $\prod_{a\in A}X_a\to X_b$ for each $b\in A$.  
  The operator $f\mapsto\tilde{f}$ is well-defined in this case, and gives rise to the amalgamation functor that we considered in the previous section.
\end{example}

The point of these last two sections can be summarized as follows.  
If a monad $M$ on $\Set$ can be expressed as the colimit of an action of a weakly filtered monoidal category $\M$ on $\Set$, then we do not lose very much by considering the category $\Set/\M$ obtained from the action rather than the Kleisli category $\Kl_M\Set$.  
More specifically, there is a full functor $\Kl_M\Set\to \Set/\M$ that is the identity on objects, so we may recover the Kleisli category by applying a suitable equivalence relation to the set of morphisms.  

The benefit of reasoning about the action of $\M$ rather than the monad $M$ is that the action might be realizable in a much larger class of categories.  
For instance, the action of $\Surj^{+\,op}$ on $\Set$ may be replicated inside any monoidal closed category that admits a functor out of the category of sets (for example, most models of PCF or Idealized Algol).

\subsection{Plan for the paper}

The rest of this paper will consist in two sections.  
In the first, we will give a more detailed account of the construction that allows us to build a promonad (i.e., a category) out of a parametric monad.  

In the second section, we will give a fully abstract game semantics for the language Probabilistic Algol (PA).  
This is by no means a new result -- the paper \cite{DanosHarmer} that introduced PA also gave it a fully abstract game semantics.  
Instead, the point of this section will be to demonstrate how the ideas we have outlined can be applied systematically to yield fully abstract models of programming languages.

\section{A Kleisli-style construction for parametric monads}

\subsection{Parametric monads}

Let $\M$ be a monoidal category and let $\C$ be a category.  
Then a \emph{lax left action} of $\M$ on $\C$ is a functor $\_.\_\from \M\times\C\to \C$ that gives rise (through currying) to a lax monoidal functor $\M\to \End[\C,\C]$.  
In other words, we have natural transformations $\passoc_{X,Y,A}\from X.Y.A\to (X\tensor Y).A$ and $\lun_A\from A \to I.A$ making the following diagrams commute for all objects $A$ of $\C$ and $X,Y,Z$ of $\M$.

\begin{mathpar}
  \begin{tikzcd}[column sep=6em]
    X.Y.Z.A \arrow[dr,"X.\passoc_{Y,Z,A}"'] \arrow[r, "\passoc_{X,Y,Z.A}"]
      & (X \tensor Y).Z.A \arrow[r, "\passoc_{X\tensor Y,Z,A}"]
        & ((X \tensor Y) \tensor Z).A \arrow[d, "\assoc_{X,Y,Z}.A"] \\
    %
      & X.(Y\tensor Z).A \arrow[r, "\passoc_{X,Y\tensor Z,A}"]
        & (X\tensor(Y\tensor Z)).A
  \end{tikzcd}
  \and
  \begin{tikzcd}
    X.A \arrow[r, "\lun_{X.A}"] \arrow[dr, "\lunit_X.A"']
      & I.X.A \arrow[d, "\passoc_{I,X,A}"] \\
    %
      & (I \tensor X) . A
  \end{tikzcd}
  \and
  \begin{tikzcd}
    X.A \arrow[r, "X.\lun_A"] \arrow[dr, "\runit_X.A"']
      & X.I.A \arrow[d, "\passoc_{X,I,A}"] \\
    %
      & (X\tensor I).A
  \end{tikzcd}
\end{mathpar}

\begin{example}
  \begin{itemize}
    \item Any monad $M$ is an action of the trivial category on its underlying category $\C$, regarding $MA$ as the action $I.A$.
      The natural transformation $\passoc_{A,I,I}$ is precisely the monad action on $A$:
      \[
        I.I.A = MMA \to M A = I.A = (I\tensor I).A\,.
        \]
      Dually, any lax action gives rise to a monad on the category being acted upon, by setting $MA = A.I$.
    \item If $\C$ is also a monoidal category and $J\from \M\to \C$ is a lax monoidal functor with monoidal coherence $\mu$, then there is an action of $\M^{co}$ (i.e., $\M$ with the opposite monoidal product) on $\C$ given by
      \[
        A.X = A \tensor JX\,,
        \]
      and
      \[
        \passoc_{X,Y,A}=
        (A \tensor JY) \tensor JX \xrightarrow{\assoc_{A,JX,JY}}
        A \tensor (JY \tensor JX) \xrightarrow{A \tensor \mu_{X,Y}}
        A \tensor J(Y\tensor X)\,.
        \]
      We sometimes refer to a left action of the opposite category $\M^{co}$ on $\C$ as a \emph{right action} of $\M$ on $\C$.  
      In that case, we write $X.A$ instead of $A.X$, so that the coherences become
      \begin{gather*}
        \passoc_{A,X,Y}\from A.X.Y \to A.(X\tensor Y)\\
        \run_A\from A \to A.I\,.
      \end{gather*}

    \item If $\C$ is a monoidal closed category, and $j$ an oplax monoidal functor with coherence $\nu$, then there is an action of $\M$ on $\C$ given by
      \[
        A.X = jX \implies A
        \]
      and
      \[
        \passoc_{A,X,Y}=jY \implies (jX \implies A) \to
        (jY \tensor jX) \implies A \xrightarrow{\nu_{Y,X}\implies A}
        j(Y \tensor X) \implies A\,.
        \]
    \item The intersection of the first two examples is the \emph{writer monad} given by $M_WX = X\tensor W$ for any monoid $W$ in $\C$.  
      The intersection of the first and third examples is the \emph{reader monad} given by $M^RX = R \implies X$ for any comonoid $R$ in $\C$ (and, in particular, for any object $R$ if $\C$ if the monoidal product in $\C$ is Cartesian).
      
      We shall therefore call an action of the form $A \tensor JX$ a \emph{writer-style action} and one of the form $jX \implies A$ a \emph{reader-style action}.
    \item We can define an \emph{oplax action} of $\M$ on $\C$ to be a lax action of $\M$ upon the opposite category $\C^{op}$.  
      In this case, the coherence $\passoc$ goes from $A.(X\tensor Y)$ to $A.X.Y$.  
      As monads are lax actions of the trivial category, so are comonads oplax actions of the trivial category.

      Since the functors $\_\tensor U$ and $U\implies \_$ are adjoint, any reader-style lax action $jX \implies A$ may also be regarded as an oplax action $A \tensor jX$.  
      We shall refer to these oplax actions as \emph{reader-style} too.
  \end{itemize}
\end{example}

\subsection{The $\C/\M$ construction}

Suppose we have a lax action of a monoidal category $\M$ upon a category $\C$.  
We define a new category $\C/\M$ as follows.
The objects of $\C/\M$ are the objects of $\C$, while the morphisms are given by the following formula.
\[
  \C/\M(A,B) = \colim_{X\object\M}\C(A,X.B)
  \]
That is, if $A$ and $B$ are objects of $\C$, then a morphism from $A$ to $B$ in $\C/\M$ is an equivalence class of pairs $(X,f)$, where $X$ is an object of $\M$ and $f\from A\to X.B$ a morphism in $\C$, and where $(X,f)$ and $(Y,g)$ are said to be equivalent if there is a morphism $h\from X\to Y$ in $\M$ making the following diagram commute.
\[
  \begin{tikzcd}
    A \arrow[r, "f"] \arrow[dr, "g"']
      & X.B \arrow[d, "h.B"] \\
    %
      & Y.B
  \end{tikzcd}
  \]
Note that this does not automatically define an equivalence relation in general, so there may be pairs $(X,f)$ and $(Y,g)$ that are not related by some $h\from X\to Y$ but are nevertheless equivalent because there is some chain of such relations from one to the other.

For the rest of the paper, we will refer to the pair $(X,f)$ using the morphism $f$, since the object $X$ should usually be clear from context.

Let $A,B,C$ be objects of $\C$, and let $f\from A\to X.B,g\from B\to Y.C$ be morphisms $A\to B$ and $B\to C$ in $\C/\M$.  
We define the composition of $f$ with $g$ to be given by the following composite in $\C$.
\[
  A \xrightarrow{f}
  X.B \xrightarrow{X.g}
  X.Y.C \xrightarrow{\passoc_{X,Y,C}}
  (X\tensor Y).C
  \]
The identity morphism $A\to A$ is given by the morphism $\lun_A\from A \to I.A$ in $\C$.
The coherence conditions for the action guarantee that this is an identity and that our composition is associative.

There is a natural functor $J\from \C \to \C/\M$ that is the identity on objects and that sends the morphism $f\from A \to B$ in $\C$ to the composite
\[
  A \xrightarrow{f}
  B \xrightarrow{\lun_B}
  I.B\,,
  \]
considered as a morphism $A\to B$ in $\C/\M$.

\subsection{Universal property of $\C/\M$}

The category $\C/\M$ has one other piece of structure besides the functor out of $\C$.  
If $X$ is an object of $M$ and $A$ and object of $\C$, then the identity morphism $X.A\to X.A$ may be considered as a morphism $\phi_{X,A}\from J(X.A) \to JA$ in $\C/\M$.  
$\phi_{X,A}$ is natural in $X$ and $A$ and also makes the following diagram commute.
\[
  \begin{tikzcd}
    J(X.Y.A) \arrow[r, "\phi_{X,Y.A}"] \arrow[d, "J\passoc_{X,Y,A}"']
      & J(Y.A) \arrow[d, "\phi_{Y,A}"] \\
    J((X\tensor Y).A) \arrow[r, "\phi_{X\tensor Y,A}"]
      & JA
  \end{tikzcd}
  \]
We call a natural transformation $\psi_{X,A}\from J(X.A) \to JA$ \emph{multiplicative} if it satisfies the same commutative diagram.

$\C/\M$, $J$ and $\phi_{X,A}$ are universal in the following sense.

\begin{proposition}
  i) Let $\C/\M$, $J$ and $\phi_{X,A}$ be as above.  
  Let $\D$ be a category, $F$ a functor $\C\to \D$ and $\psi_{X,A}\from F(X.A) \to FA$ a multiplicative natural transformation in $\D$.  
  Then there is a unique functor $H\from \C/\M\to \D$ such that $F=H J$ and $\psi=H\phi$.

  ii) Let $\D$ be a category and let $H,K\from \C/\M\to \D$ be two functors.  
  Let $\alpha\from HJ \to KJ$ be a natural transformation making the following diagram commute.
  \[
    \begin{tikzcd}
      HJ A \arrow[r, "\alpha_A"] \arrow[d, "H\phi_{X,A}"']
        & KJ A \arrow[d, "K\phi_{X,A}"] \\
      HJ X.A \arrow[r, "\alpha_{X.A}"]
        & KJ A
    \end{tikzcd}
    \]
  Then there is a unique natural transformation $\beta\from H\to K$ such that $\alpha=\beta J$.
  \label{quotient-universal}
\end{proposition}

\begin{remark}
  The functor $H$ in Proposition \ref{quotient-universal}(i) sends the object $JA=A$ to the object $FA$.  
  To define $H$ on morphisms, we use the fact that every morphism $f\from A \to X.B$ from $A$ to $B$ in $\C/\M$ may be factorized as:
  \[
    f = A \xrightarrow{Jf}
    X.B \xrightarrow{\phi_{X,A}}
    B\,,
    \]
  where we have considered $f$ both as a morphism $A\to B$ in $\C/\M$ and as a morphism $A \to X.B$ in $\C$.  
  It therefore suffices to define $H$ on morphisms of the form $Jf$ (which must be sent to $Ff$, since $F=HJ$) and on morphisms of the form $\phi_{X,A}$ (which must be sent to $\psi_{X,A}$, since $\psi=H\phi$).  
  The proof of part (ii) also follows from this factorization.

  Thus, one way of thinking of the category $\C/\M$ is as a general theory of categories that extend $\C$, and admit a factorization result whereby every morphism is the composition of a morphism from $\C$ with one of the distinguished morphisms $\phi_{X,A}$.

  For example, in \cite{SamsonGuyIAPassive}, it is shown that any morphism $\sigma \from A\to B$ in the category $\G_{vis}$ of games and visible strategies may be factorized as
  \[
    \sigma = A \xrightarrow{\text{Inn}(\sigma)}
    (\text{Var[S]} \to B) \xrightarrow{\textsf{new}_S}
    B\,,
    \]
  where $\text{Inn}(\sigma)$ is a morphism in the category of $\G_{inn}$ of games and innocent strategies, $S$ is a set and $\textsf{new}$ is a special morphism that introduces stateful behaviour.  
  The $\text{Var}[S]$ construction, which is the game corresponding to a storage cell holding values of the set $S$, may be regarded as an oplax monoidal functor from the category $\Ret$ of sets and retractions into the category of games, and so $\text{Var}[S]\to B$ is a reader-style action.  
  Moreover, it can be shown that $\textsf{new}$ gives rise to a multiplicative natural transformation, inducing a functor $\G_{inn}/\Ret\to \G_{vis}$.  
  Then the factorization result we have quoted says precisely that this functor is full.
\end{remark}
\begin{remark}
  If $\M$ is the trivial category, so our action is a monad $M$, then this universal property can be used to prove the usual universal property for the Kleisli category; namely, that it is initial among all adjunctions giving rise to $M$.  
  To see this, suppose that $\C \xrightarrow{F} \D \xrightarrow{G} \C$ are functors such that $F$ is left adjoint to $G$ and $M=GF$.  
  Then we have a natural transformation
  \[
    FMA = FGF A \xrightarrow{\epsilon_{F A}} F A\,
    \]
  which satisfies the hypotheses of Proposition \ref{quotient-universal}(i).  
  Therefore, Proposition \ref{quotient-universal} tells us that the Kleisli category is not only initial amongst adjunctions giving rise to $M$, but is initial among all functors $F\from \C \to \D$ that admit a natural transformation $FMA \to FA$.
  Note that for more general $\M$, we do not in general get an adjunction between $\C$ and $\C/\M$.
\end{remark}
\begin{remark}
  The quotient notation we have used is because the universal property we have outlined exhibits $\C/\M$ as a particular weighted coequalizer of the functors $\pr_2,\_.\_\from \M\times\C \rightrightarrows \C$ in $\Cat$.
  It may therefore be regarded as a lax $2$-dimensional analogue of the quotient $X/G$ of a set $X$ by a group or monoid action.  
  
  This is a strange idea, since we are not used to thinking of, for example, a Kleisli category as quotient of the original category.  
  The reason for this is the laxness: we do not identify the objects $A$ and $X.A$ with an isomorphism, but with a morphism in one direction.  
  One way in which a Kleisli category does act like a quotient is that it allows us to transfer effects that are usually associated with higher types on to lower types.  
  For example, the type $W\to [A,W]$ encodes a state transformer with state $W$.  
  Working in the Kleisli category for the state monad allows us to associate this stateful behaviour to the object $A$.
\end{remark}
\begin{remark}
  Suppose we have an oplax right action of a monoidal category $\M$ on a category $\C$.  
  This may be regarded as a lax left action of $\M^{co}$ upon $\C^{op}$, and therefore we get a category in the same way in which the objects are the objects of $\C$ and the morphisms from $A$ to $B$ are equivalence classes of morphisms $A.X \to B$ in $\C$.

  In the case that the action is an oplax reader-style action given by $A.X = A\tensor jX$ for some oplax monoidal functor $j\from \M \to \C$, then the natural transformation $\phi_{A,X}$ goes from $J(A\tensor jX)$ to $JA$, and may instead be regarded as a natural transformation $I\to jX$ in $\C$.  
  In that case, the category $\C/\M$ that we obtain is what \cite[4.3.11]{lurie2009} calls the \emph{cone} over the functor $j$.  
  This is because the universal property in that case is a lax version of the property satisfied by the mapping cone in homotopy theory.
\end{remark}

\subsection{Monoidal and monoidal closed structure of $\C/\M$}

We saw in the last section that the $\C/\M$ construction may be regarded as a kind of lax $2$-coequalizer.  
In the category of sets, if a pair of morphisms $f,g\from A \rightrightarrows B$ is \emph{reflexive} -- i.e., if there is a morphism $h\from B \to A$ such that $f\circ h = g\circ h = \id_B$, then the coequalizer of $f$ and $g$ commutes with finite products.  

In our setting, the two functors $\pr_2$ and $\_.\_$ from $\M\times\C \to \C$ form a reflexive pair, since they have a common section $h\from \C \to \M\times \C$ given by $h(A) = (I, A)$.  
It turns out that a similar result then applies.

\begin{proposition}
  Given an action of a monoidal category $\M$ on a category $\C$, there is an obvious way to define an action of $\M\times\M$ on $\C\times\C$.  
  Then the functor
  \[
    (\C\times\C)/(\M\times\M) \to (\C/\M) \times (\C/\M)
    \]
  induced as in Proposition \ref{quotient-universal} is an isomorphism.
  \label{monoidality}
\end{proposition}

Now suppose that $\C$ is itself a monoidal category.  
We say that the action of $\M$ on $\C$ is \emph{monoidal} if it is a lax monoidal functor $\M\times\C\to\C$, which means that we need to have a natural transformation 
\[
  X.A\tensor Y.B\to (X\tensor Y).(A\tensor B)\,,
  \]
together with appropriate coherences.  

If the action is reader-style, given by $X.A = jX \implies A$ for an oplax monoidal functor $j\from \M\to \C$, then it is sufficient for $\C$ to be a \emph{symmetric} monoidal category, since then we have a natural transformation
\[
  (j X \implies A) \tensor (j Y \implies B)
  \rightarrow
  (j X \tensor j Y) \implies (A \tensor B)
  \rightarrow
  j (X\tensor Y) \implies (A \tensor B)\,.
  \]
Given a monoidal action, we get a natural transformation
\[
  (X\tensor Y).(A\tensor B)
  \rightarrow
  (X.A)\tensor Y.B
  \xrightarrow{\phi_{X,A}\tensor \phi_{Y,B}}
  A\tensor B\,,
  \]
which is multiplicative, inducing a functor $(\C\times\C)/(\M\times\M)\to \C/\M$.  
Composing with the isomorphism from the start of this section, we get a functor $(\C/\M)\times(\C/\M)\to\C/\M$.  
It turns out that this gives $\C/\M$ the structure of a monoidal category and that in this case the functor $J\from \C\to\C/\M$ is a strict monoidal functor.

This generalizes the case of a monoidal monad, whose the Kleisli category is always a monoidal category.

Now, if $\C$ is monoidal closed and the action of $\M$ on $\C$ is a \emph{reader-type action}, then $\C/\M$ also inherits the monoidal closed structure from $\C$.  
Note that this may not be true for arbitrary actions, even for monads.  
For example, the Kleisli category for the power set monad on $\Set$ (which is better known as the category of sets and relations) is monoidal closed, but with inner function space given by the Cartesian product, and not by the function space from $\Set$.

\bibliographystyle{spbasic}
\bibliography{effects}

\end{document}
