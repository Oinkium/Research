%%%%%%%%%%%%%%%%%%%% author.tex %%%%%%%%%%%%%%%%%%%%%%%%%%%%%%%%%%%
%
% sample root file for your "contribution" to a proceedings volume
%
% Use this file as a template for your own input.
%
%%%%%%%%%%%%%%%% Springer %%%%%%%%%%%%%%%%%%%%%%%%%%%%%%%%%%


\documentclass{svproc}
%
% RECOMMENDED %%%%%%%%%%%%%%%%%%%%%%%%%%%%%%%%%%%%%%%%%%%%%%%%%%%
%

% to typeset URLs, URIs, and DOIs
\usepackage{url}
\usepackage{amsmath}
\usepackage{amssymb}
\usepackage{tikz}
\usepackage{xspace}
\usepackage{IEEEtrantools}
\usepackage{mathtools}
\usetikzlibrary{cd}
\usepackage{mathpartir}
\usepackage{stmaryrd}
\def\UrlFont{\rmfamily}

\newcommand*\circled[1]{\tikz[baseline=(char.base)]{
    \node[shape=circle,draw,inner sep=2pt] (char) {#1};}}
\newcommand{\deno}[1]{\left\llbracket#1\right\rrbracket}
\renewcommand{\ts}{{\;\vdash}}

\newcommand\C{\mathcal{C}}
\newcommand\D{\mathcal{D}}
\newcommand\F{\mathcal{F}}
\newcommand\G{\mathcal{G}}
\newcommand\M{\mathcal{M}}
\DeclareMathOperator\pr{pr}
\DeclareMathOperator\id{id}
\newcommand{\inv}{{}^{-1}}
\newcommand{\suchthat}{\,\colon\,}
\newcommand{\esuchthat}{\,.\,}
\newcommand\object\colon
\DeclareMathOperator{\ac}{ac}
\DeclareMathOperator{\End}{End}
\newcommand{\passoc}{\texttt{passoc}}
\newcommand{\assoc}{\texttt{assoc}}
\newcommand\tensor\otimes
\newcommand\run{\texttt{r}}
\newcommand\lun{\texttt{l}}
\newcommand\lunit{\texttt{lunit}}
\newcommand\runit{\texttt{runit}}
\renewcommand\implies\multimap
\newcommand{\pfc}{\boldsymbol{\cdot}}
\newcommand\Mellies{Melli\`{e}s\xspace}

\newcommand{\test}{\texttt{test}}

\let\oldemptyset\emptyset
\let\emptyset\varnothing
\newcommand{\nat}{{\mathtt{nat}}}
\newcommand*\from{\colon}
\newcommand{\Y}{\mathbf{Y}}
\newcommand{\opto}{\longrightarrow}
\newcommand{\n}{{\mathtt{n}}}
\newcommand{\x}{{\mathtt{x}}}
\newcommand{\IfO}{{\mathtt{If0}\;}}
\newcommand{\If}{{\mathtt{If}\;}}
\newcommand{\suc}{{\mathtt{suc\;}}}
\newcommand{\pred}{{\mathtt{pred\;}}}
\newcommand{\0}{{\mathtt{0}}} \newcommand{\com}{{\mathtt{com}}}
\newcommand{\skipp}{{\mathsf{skip}}}
\newcommand{\neww}{{\mathsf{new}}}
\newcommand{\mkvar}{{\mathsf{mkvar}\;}}
\newcommand{\deref}{\texttt{@}}
\newcommand{\ia}[2]{\langle #1 , #2 \rangle}
\newcommand{\stup}[3]{\langle #1 \mid #2 \mapsto #3 \rangle}
\newcommand{\Var}{{\mathtt{Var}}}
\newcommand{\new}{{\mathtt{new}}}
\renewcommand{\case}{{\mathtt{case}}}
\newcommand{\blank}{\,\underline{\hspace{1.5ex}}\,}
\newcommand{\converges}{\Downarrow}

\newcommand{\catname}[1]{\mathbf{#1}}
\newcommand{\Set}{\catname{Set}}
\newcommand{\Surj}{\catname{Surj}}
\newcommand{\Ret}{\catname{Ret}}
\newcommand{\Cat}{\catname{Cat}}
\newcommand{\DProb}{\catname{DProb}}
\newcommand{\BProb}{\catname{BProb}}
\newcommand{\powerset}{\mathcal P}
\usepackage{bbm}
\newcommand{\true}{\mathbbm{t}}
\newcommand{\false}{\mathbbm{f}}

\newcommand{\bR}{\mathbb{R}}
\newcommand{\bN}{\mathbb{N}}
\newcommand{\bB}{\mathbb{B}}
\newcommand{\bC}{\mathbb{C}}
\newcommand{\bP}{\mathbb{P}}

\DeclareMathOperator{\Prof}{Prof}
\DeclareMathOperator{\Kl}{Kl}

\newcommand{\coin}{\textsf{coin}}
\newcommand{\polydie}{\textsf{polydie}}
\newcommand{\multicoin}{\textsf{multicoin}}
\newcommand{\bool}{\textsf{bool}}

\renewcommand{\emptyset}{\varnothing}

\makeatletter
\newcommand{\colim@}[2]{%
  \vtop{\m@th\ialign{##\cr
    \hfil$#1\operator@font colim$\hfil\cr
    \noalign{\nointerlineskip\kern1.5\ex@}#2\cr
    \noalign{\nointerlineskip\kern-\ex@}\cr}}%
}
\newcommand{\colim}{%
  \mathop{\mathpalette\colim@{\rightarrowfill@\textstyle}}\nmlimits@
}
\makeatother

\begin{document}
\mainmatter              % start of a contribution
%
\title{A Kleisli-style construction for Parametric Monads, with Examples}
%
\titlerunning{A Kleisli-Style Construction}  % abbreviated title (for running head)
%                                     also used for the TOC unless
%                                     \toctitle is used
%
\author{W. John Gowers}
%
\authorrunning{W. J. Gowers} % abbreviated author list (for running head)
%
%%%% list of authors for the TOC (use if author list has to be modified)
\tocauthor{}
%
\institute{University of Bath, Claverton Down Road, Bath.  
BA2 7AY,\\
\email{wjg27@bath.ac.uk}}

\maketitle              % typeset the title of the contribution

\begin{abstract}
  A well-known principle in denotational semantics is that adding an effect to a categorical model should correspond to the Kleisli category construction for some suitable monad.  
  Our point of departure is the observation that much recent work in semantics, particularly in game semantics, has broken away from this principle by modelling effects using categories that are not Kleisli categories.  
  Our aim is to extend the theory of monadic effects to these models in a systematic way.

  An important part of our approach will be to study parametric monads, otherwise known as lax actions of monoidal categories.
  These generalize monads: a monad is precisely an action of the trivial category.
  We introduce a construction on parametric monads which takes a parametric monad on a category and gives us back a new category.  
  As with the Kleisli category, the objects of this category will be the objects of the original category, while the morphisms are obtained from morphisms in the original category in a formal manner brought about by the action.  
  If the action is actually a monad, we get the usual Kleisli category.

  We show that this new construction is closely related to existing models of effects in game semantics, allowing us to reason systematically about adequacy and full abstraction proofs.
  We illustrate the relevance of the construction by showing that a special case gives us a fully abstract game semantics for Probabilistic Algol, which can be related to the existing model given by Danos and Harmer.
\keywords{denotational semantics, category theory, game semantics}
\end{abstract}

\section{Introduction}

This paper is about general frameworks for modelling effects in a categorical semantics.  
The seminal paper in this field is Moggi's \emph{Computational Lambda-Calculus and Monads} \cite{Moggi}, which first put forward the idea that if $\C$ is a categorical model of a programming language, then we can simulate an effect in that language via a suitable \emph{monad} on $\C$.  
Given a monad, its Kleisli category $\Kl_M\C$ is then a model of a language formed by adding the effect to the original one.  

Spivey \cite{Spivey} and Wadler \cite{Wadler1,Wadler2} have transferred this theory to the real world by using monads as programming tools, particularly in Haskell.

There are other systematic treatments of effects -- for example, Lawvere theories \cite{Lawvere}.  
More recent work in game semantics, however, has produced models of effects -- such as the well-known game semantics for scoped state \cite{SamsonGuyIAPassive} and nondeterminism \cite{mcCHFiniteND} -- do not arise by applying any systematic procedure to a base category of games that lacks the effect.  
Instead, these models are typically built up using intuition -- for example, by relaxing the determinism condition on strategies to give a model of a nondeterministic language -- and do not readily fit into any category theoretic framework the way Kleisli categories do.  
The goal of this paper is to examine some possible frameworks that will allow us to study these and other models in a systematic way.  

Before we start, it is worth mentioning one example of a Kleisli category arising naturally in game semantics.  
The \emph{slot games} defined in \cite{SlotGames} are defined like ordinary games, but their strategies are allowed to contain extra moves, called \emph{tokens}, which are meant to represent a point at which the computation takes up time or resources.  
For example, a strategy for a term of natural number type that returns the value $5$ after two CPU cycles could be represented by the strategy shown on the left of Figure \ref{slot-games}.  
Now, we can alternatively represent this strategy as being a normal strategy inside the Kleisli category for the reader monad given by the \emph{command} or \emph{singleton} type, as on the right of Figure \ref{slot-games}.  
The translation works by replacing each token $\circled{\$}$ with a request to the argument to the function: since this argument has singleton type, there is only one possible value that can be passed in, but the game semantics will record that we have made the request.  
What is more, this translation respects composition in the two categories, both the Kleisli composition and the composition defined synthetically for slot games.
\begin{figure}
  \begin{mathpar}
    \begin{array}{c}
      \mathbf{\bN} \\
      q \\
      \circled{\$} \\
      \circled{\$} \\
      5 \\
    \end{array}
    \and
    \begin{array}{ccc}
      \mathbf{\bC} & \mathbf{\implies} & \mathbf{\bN} \\
      && q \\
      q && \\
      a && \\
      q && \\
      a && \\
      && 5
    \end{array}
  \end{mathpar}
  \label{slot-games}
  \caption{The `slot-games' denotation of a program that uses resources can also be represented in a Kleisli category.}
\end{figure}

\section{Promonads and Parametric monads}

\subsection{Promonads}

One clue comes from generalizing from functors to profunctors.  
A monad is a monoid in the category $\End[\C]$ of endofunctors on $\C$, with monoidal product given by composition. So we can define a \emph{promonad} to be a monoid in the category $\Prof[\C,\C]$ of endoprofunctors on $\C$ (i.e., functors $\C^{op}\times\C\to \Set$), where the monoidal product is profunctor composition.
A functor $\F\from \C\to \D$ may be identified with the profunctor $\D[F(\blank),\blank]\from\C^{op}\times\D\to\Set$, so promonads generalize monads.

A little thought, however, shows that a promonad is a vast generalization of a monad.  
Any promonad $\D\from\C^{op}\times\C\to\Set$ on a category $\C$ may be identified \cite{Promonad} with a category (also called $\D$) where the objects are the objects of $\C$ and the morphisms from $A$ to $B$ are elements of the set $\D(A,B)$.  
The promonad structure gives us composition and identities in the category $\D$, and also a functor $J\from \C\to\D$ that is the identity on objects.

In the other direction, if we have such a $\C\xrightarrow{J}\D$, then the hom functor $\D(J(\blank),J(\blank)$ is a promonad on $\C$.  
Thus, promonads on $\C$ generalize the Kleisli construction to such an extent that they incorporate, among others, any category formed from $\C$ by keeping the same objects and adding new morphisms, such as the game semantics models of state \cite{SamsonGuyIAPassive} and nondeterminism \cite{mcCHFiniteND}.  

On the other hand, what promonads capture is quite a natural simple way of modelling effects: keep the same types (objects), but give new ways to write programs (morphisms).  
Although effectful types are important (e.g., in work on call-by-push-value \cite{Cbpv}), we shall adopt this paradigm here.
So one way to state our goal is that we are looking for systematic ways to construct promonads on a category that are more general than the existing Kleisli construction for monads.

\subsection{Parametric monads}

\Mellies \cite{ParametricMonads}, following work by Smirnov \cite{Smirnov2008}, has demonstrated that the concept of a lax monoidal category action is a useful generalization of that of a monad.  
A lax action of a monoidal category $\M$ upon a category $\C$ is a lax monoidal functor $\M\to \End[\C]$ (considered as a functor $\blank.\blank\from\M\times\C\to\C)$; monads are the special case when $\M$ is trivial.  
For this reason, \Mellies renames lax monoidal actions to `parametric monads', where we think of the action of $\M$ as a kind of monad on $\C$ that is parameterized by the objects of $\M$.

This idea has been further generalized by Katsumata \cite{Katsu}, who has given a definition of a `Kleisli-like resolution' of a parametric monad.  
This particular construction, which takes a parametric monad on a category $\C$ parameterized by a monoidal category $\M$ and yields a new category, generalizes many important properties of the Kleisli category -- in particular, it gives us an adjunction with the original category -- but it does not fit into our framework, since the objects of the category it creates are not the objects of the original category $\C$, but pairs of an object of $\C$ and an object of $\M$.

Once again, it is useful to generalize to profunctors.  
First, if $\C$ is a category and $\M$ a monoidal category, we define a \emph{parametric promonad on $\C$ parameterized by $\M$} to be a lax monoidal functor $\M\to\Prof[\C,\C]$.
In a similar way to before, if we have a parametric monad $\blank.\blank\from \M\times\C\to\C$ then we may identify it with the parametric promonad given by $\D(X,A,B)=\C(A,X.B)$.

As with promonads, we can now take a parametric promonad $\D\from \M\times\C^{op}\times\C\to\Set$ and try to recast it as a particular kind of category.  
There are various ways to do it, but one is to view it as a bicategory whose objects are the objects of $\C$, where the elements of the set $\D(X,A,B)$ are considered as $1$-morphisms from $A$ to $B$ and where morphisms $X\to Y$ in $\M$ give rise to $2$-morphisms between the elements of $\D(X,A,B)$ and the elements of $\D(Y,A,B)$.  

A natural idea is to turn this bicategory into a $2$-category by quotienting out by the action of the $2$-morphisms, which gives us the following collapse from parametric promonads to promonads.
\[
  \D \mapsto \colim_{X\object\M}\D(X,\blank,\blank)
  \]

In the particular case that the parametric promonad is in fact a parametric \emph{monad} (so $\D(X,A,B) = \C(A,X.B)$), then we will call the category we obtain $\C/\M$.  
The morphisms in $\C/\M$ are given by
\[
  \C/\M(A,B)=\colim_{X\object\M} \C(A,X.B)\,,
  \]
with the composition of $f\from A\to X.B$ and $g\from B\to Y.C$ given by the composite
\[
  A \xrightarrow{f}
  X.B \xrightarrow{X.g}
  X.Y.C \to
  (X\tensor Y).C\,.
  \]
Note that if $\M$ is the trivial category (so that the action is in fact a monad), then the category $\C/\M$ is the usual Kleisli category.

\subsection{Effects, monads and colimits}

The colimit that appears in the definition of the category $\C/\M$ is instructive, because it is linked to the fact that many important monads may be expressed as colimits.  
For example, let $\Surj^+$ be the category of non-empty sets and surjections.  
Then, if $A$ is a set, we have (naturally in $A$):
\begin{IEEEeqnarray*}{CCC}
  \colim_{X\object\Surj^{+\,op}} [X,A] & \cong & \powerset(A) \\[6pt]
  \left(X \xrightarrow{f} A\right) & \mapsto & \text{Im}(f)\,.
\end{IEEEeqnarray*}

This is useful, because the body of the colimit -- i.e., the function set $[X,A]$ -- has a natural equivalent inside any Cartesian closed category $\C$ that admits a functor $\F\from\Set\to\C$; i.e., the internal hom $\F(X)\implies A$.
Moreover, the functor $[X,A]$ is actually a lax action of $\Surj^{+,op}$ (with the Cartesian product) upon the category of sets, and the monad structure of the powerset may be deduced from the structure of this action via the colimit.
So even though we cannot define the powerset \emph{monad} inside arbitrary CCCs, we can still define this action of $\Surj^{op}$.  

The idea, then, is that we might try to model nondeterminism in other categories (where we might not be able to define the powerset) using this parametric monad.  
One unfortunate fact is that these two alternatives (powerset monad vs lax action of $\Surj^{+,op}$) do not give rise to the same promonad in the category of sets.  
Indeed, if we have a general action of a monoidal category $\M$ on $\Set$, then (as long as the possibly large colimits exist) we get a monad $M$ on $\Set$ given by
\[
  M A = \colim_{X\object\M} X.A\,
  \]
and the morphisms between sets $A$ and $B$ in the Kleisli category are given by
\[
  [A, \colim_{X\object\M} X.B]\,.
  \]
Meanwhile, the morphisms between $A$ and $B$ in the category $\C/\M$ are given by
\[
  \colim_{X\object\M} [A,X.B]\,,
  \]
and the two sets need not be the same, since the colimit might not commute with the functor $[A,\blank]$.

However, if the category $\M$ satisfies certain additional properties, we can get some relation back.  
There is always a natural identity-on-objects functor between the two categories, given by the natural map
\[
  \colim_{X\object\M}[A,X.B] \to [A, \colim_{X\object\M} X.B]\,.
  \]
If the category $\M$ admits a cocone over every discrete diagram (i.e., if whenever $(X_i)$ is a collection of objects of $\M$ there is some $Y$ such that there is a morphism $f_i\from X_i\to Y$ for all $i$), then this function is surjective, meaning that the functor will be full.
The category $\Surj^{+,op}$ has this property (via the Cartesian product and its projections, which are always surjective for non-empty sets).  
In the case of the action of $\Surj^{+,op}$ that gives rise to the powerset monad, the natural map given above is the surjection $\powerset[A,B]\to [A,\powerset B]$ that takes a set $\F$ of functions $A\to B$ and forms the function $A\to\powerset B$ that sends an element $a$ to the set $\{f(a)\suchthat f\in\F\}$.
So we may recover the Kleisli category from the category given by the action by applying a further equivalence relation on morphisms.

In general, if a monad $M$ on $\Set$ can be expressed as the colimit of a lax action of a monoidal category $\M$ on $\Set$, where $\M$ admits a cocone over every discrete diagram, then we do not lose very much by considering the category $\Set/\M$ obtained from the action rather than the Kleisli category $\Kl_M\Set$, since there is a full functor $\Kl_M\Set\to \Set/\M$ that is the identity on objects, letting us recover the Kleisli category by applying a suitable equivalence relation on morphisms.
The benefit of reasoning about the action of $\M$ rather than the monad $M$ is that the action might be realizable in a much larger class of categories, as with our action of $\Surj^{+\,op}$.
We will see later that the non-empty powerset monad has a probabilistic counterpart, where the set of discrete probability distributions on a set may be obtained as a colimit over the category of probability spaces and probability-preserving maps, which satisfies similar properties, allowing us to model discrete probability in categories other than $\Set$.
\section{A Kleisli-style construction for parametric monads}

\subsection{Parametric monads}

Let $\M$ be a monoidal category and let $\C$ be a category.  
Then a \emph{lax left action} of $\M$ on $\C$ is a functor $\blank.\blank\from \M\times\C\to \C$ that gives rise (through currying) to a lax monoidal functor $\M\to \End[\C,\C]$.  
In other words, we have natural transformations $\passoc_{X,Y,A}\from X.Y.A\to (X\tensor Y).A$ and $\lun_A\from A \to I.A$ making the following diagrams commute for all objects $A$ of $\C$ and $X,Y,Z$ of $\M$.

\begin{mathpar}
  \begin{tikzcd}[column sep=6em]
    X.Y.Z.A \arrow[dr,"X.\passoc_{Y,Z,A}"'] \arrow[r, "\passoc_{X,Y,Z.A}"]
      & (X \tensor Y).Z.A \arrow[r, "\passoc_{X\tensor Y,Z,A}"]
        & ((X \tensor Y) \tensor Z).A \arrow[d, "\assoc_{X,Y,Z}.A"] \\
    %
      & X.(Y\tensor Z).A \arrow[r, "\passoc_{X,Y\tensor Z,A}"]
        & (X\tensor(Y\tensor Z)).A
  \end{tikzcd}
  \and
  \begin{tikzcd}
    X.A \arrow[r, "\lun_{X.A}"] \arrow[dr, "\lunit_X.A"']
      & I.X.A \arrow[d, "\passoc_{I,X,A}"] \\
    %
      & (I \tensor X) . A
  \end{tikzcd}
  \and
  \begin{tikzcd}
    X.A \arrow[r, "X.\lun_A"] \arrow[dr, "\runit_X.A"']
      & X.I.A \arrow[d, "\passoc_{X,I,A}"] \\
    %
      & (X\tensor I).A
  \end{tikzcd}
\end{mathpar}

\begin{example}
  \begin{itemize}
    \item Any monad $M$ is an action of the trivial category on its underlying category $\C$, regarding $MA$ as the action $I.A$.
      The natural transformation $\passoc_{A,I,I}$ is precisely the monad action on $A$:
      \[
        I.I.A = MMA \to M A = I.A = (I\tensor I).A\,.
        \]
    \item If $\C$ is also a monoidal category and $J\from \M\to \C$ is a lax monoidal functor with monoidal coherence $\mu$, then there is an action of $\M^{co}$ (i.e., $\M$ with the opposite monoidal product) on $\C$ given by
      \[
        A.X = A \tensor JX\,,
        \]
      where $\passoc_{X,Y,A}$ is given by the composite
      \[
        (A \tensor JY) \tensor JX \xrightarrow{\assoc_{A,JX,JY}}
        A \tensor (JY \tensor JX) \xrightarrow{A \tensor \mu_{X,Y}}
        A \tensor J(Y\tensor X)\,.
        \]
      and the unit $\lun_A$ similarly.  
      We sometimes refer to a left action of the opposite category $\M^{co}$ on $\C$ as a \emph{right action} of $\M$ on $\C$.  
      In that case, we write $X.A$ instead of $A.X$, so that the coherences become
      \begin{gather*}
        \passoc_{A,X,Y}\from A.X.Y \to A.(X\tensor Y)\\
        \run_A\from A \to A.I\,.
      \end{gather*}

    \item If $\C$ is a monoidal closed category, and $j$ an oplax monoidal functor with coherence $\nu$, then there is an action of $\M^{op}$ on $\C$ given by
      \[
        A.X = jX \implies A
        \]
      where $\passoc_{A,X,Y}$ is given by the composite
      \[
        \passoc_{A,X,Y}=jY \implies (jX \implies A) \to
        (jY \tensor jX) \implies A \xrightarrow{\nu_{Y,X}\implies A}
        j(Y \tensor X) \implies A\,.
        \]
    \item The intersection of the first two examples is the \emph{writer monad} given by $M_WX = X\tensor W$ for any monoid $W$ in $\C$.  
      The intersection of the first and third examples is the \emph{reader monad} given by $M^RX = R \implies X$ for any comonoid $R$ in $\C$ (and, in particular, for any object $R$ if $\C$ if the monoidal product in $\C$ is Cartesian).
      We therefore call an action of the form $A \tensor JX$ a \emph{writer-style action} and one of the form $jX \implies A$ a \emph{reader-style action}.
    \item We can define an \emph{oplax action} of $\M$ on $\C$ to be a lax action of $\M$ upon the opposite category $\C^{op}$.  
      In this case, the coherence $\passoc$ goes from $A.(X\tensor Y)$ to $A.X.Y$.  
      As monads are lax actions of the trivial category, so are comonads oplax actions of the trivial category.

      Since the functors $\blank\tensor U$ and $U\implies \blank$ are adjoint, any reader-style lax action $jX \implies A$ may also be regarded as an oplax action $A \tensor jX$.  
      We shall refer to these oplax actions as \emph{reader-style} too.
  \end{itemize}
\end{example}

\subsection{The $\C/\M$ construction}

Suppose we have a lax action of a monoidal category $\M$ upon a category $\C$.  
We define a new category $\C/\M$ as follows.
The objects of $\C/\M$ are the objects of $\C$, while the morphisms are given by the following formula.
\[
  \C/\M(A,B) = \colim_{X\object\M}\C(A,X.B)
  \]
That is, if $A$ and $B$ are objects of $\C$, then a morphism from $A$ to $B$ in $\C/\M$ is an equivalence class of pairs $(X,f)$, where $X$ is an object of $\M$ and $f\from A\to X.B$ a morphism in $\C$, and where $(X,f)$ and $(Y,g)$ are said to be equivalent if there is a morphism $h\from X\to Y$ in $\M$ making the following diagram commute.
\[
  \begin{tikzcd}
    A \arrow[r, "f"] \arrow[dr, "g"']
      & X.B \arrow[d, "h.B"] \\
    %
      & Y.B
  \end{tikzcd}
  \]
Note that this does not automatically define an equivalence relation in general, so there may be pairs $(X,f)$ and $(Y,g)$ that are not related by some $h\from X\to Y$ but are nevertheless equivalent because there is some zigzag of such relations from one to the other.

For the rest of the paper, we will refer to the pair $(X,f)$ using the morphism $f$, since the object $X$ should usually be clear from context.

Let $A,B,C$ be objects of $\C$, and let $f\from A\to X.B,g\from B\to Y.C$ be morphisms $A\to B$ and $B\to C$ in $\C/\M$.  
The composition of $f$ with $g$ is given in $\C$ by:
\[
  A \xrightarrow{f}
  X.B \xrightarrow{X.g}
  X.Y.C \xrightarrow{\passoc_{X,Y,C}}
  (X\tensor Y).C
  \]
The identity morphism $A\to A$ is given by the morphism $\lun_A\from A \to I.A$ in $\C$.
The coherence conditions for the action guarantee that this is an identity and that our composition is associative.

If we consider the case that $\M$ is trivial, so the action is a monad $M$ given by $MA = I.A$, then this composition becomes
\[
  A \xrightarrow{f}
  M B \xrightarrow{M g}
  M M C \xrightarrow{\passoc_{I,I,C}}
  M C\,,
  \]
which is precisely the usual Kleisli composition.

There is a natural functor $J\from \C \to \C/\M$ that is the identity on objects and that sends the morphism $f\from A \to B$ in $\C$ to the composite
\[
  A \xrightarrow{f}
  B \xrightarrow{\lun_B}
  I.B\,,
  \]
considered as a morphism $A\to B$ in $\C/\M$.

\subsection{Universal property of $\C/\M$}

The category $\C/\M$ has one other piece of structure besides the functor out of $\C$.  
If $X$ is an object of $M$ and $A$ and object of $\C$, then the identity morphism $X.A\to X.A$ may be considered as a morphism $\phi_{X,A}\from J(X.A) \to JA$ in $\C/\M$.  
$\phi_{X,A}$ is natural in $X$ and $A$ and also makes the following diagram commute.
\[
  \begin{tikzcd}
    J(X.Y.A) \arrow[r, "\phi_{X,Y.A}"] \arrow[d, "J\passoc_{X,Y,A}"']
      & J(Y.A) \arrow[d, "\phi_{Y,A}"] \\
    J((X\tensor Y).A) \arrow[r, "\phi_{X\tensor Y,A}"]
      & JA
  \end{tikzcd}
  \]
We call a natural transformation $\psi_{X,A}\from J(X.A) \to JA$ \emph{multiplicative} if it satisfies the same commutative diagram.

$\C/\M$, $J$ and $\phi_{X,A}$ are universal in the following sense.

\begin{proposition}
  i) Let $\C/\M$, $J$ and $\phi_{X,A}$ be as above.  
  Let $\D$ be a category, $F$ a functor $\C\to \D$ and $\psi_{X,A}\from F(X.A) \to FA$ a multiplicative natural transformation in $\D$.  
  Then there is a unique functor $H\from \C/\M\to \D$ such that $F=H J$ and $\psi=H\phi$.

  ii) Let $H,K\from \C/\M\to \D$ be two functors.  
  Let $\alpha\from HJ \to KJ$ be a natural transformation making the following square commute.
  \[
    \begin{tikzcd}
      HJ A \arrow[r, "\alpha_A"] \arrow[d, "H\phi_{X,A}"']
        & KJ A \arrow[d, "K\phi_{X,A}"] \\
      HJ X.A \arrow[r, "\alpha_{X.A}"]
        & KJ A
    \end{tikzcd}
    \]
  Then there is a unique natural transformation $\beta\from H\to K$ such that $\alpha=\beta J$.
  \label{quotient-universal}
\end{proposition}

\begin{proof}
  The proof, which we will not give full details of, comes down to the fact that we can factorize any morphism $f\from A \to X.B$ from $A$ to $B$ in $\C/\M$ as:
  \[
    f = A \xrightarrow{Jf}
    X.B \xrightarrow{\phi_{X,A}}
    B\,,
    \]
  where we have considered $f$ both as a morphism $A\to B$ in $\C/\M$ and as a morphism $A \to X.B$ in $\C$.  
  It therefore suffices to define $H$ on morphisms of the form $Jf$ and on morphisms of the form $\phi_{X,A}$.  
  The conditions in the statement mean that $H$ must be defined by sending the above factorization to the composite
  \[
    FA \xrightarrow{Ff}
    F(X.B) \xrightarrow{\psi_{X,A}}
    FB\,,
    \]
  so it suffices to check that this is indeed a functor.
  Part (ii) is proved from the factorization in a very similar way.\qed
\end{proof}

\begin{remark}
  If $\M$ is the trivial category, so our action is a monad $M$, then this universal property can be used to prove the usual universal property for the Kleisli category; namely, that it is initial among all adjunctions giving rise to $M$.  
  To see this, suppose that $\C \xrightarrow{F} \D \xrightarrow{G} \C$ are functors such that $F$ is left adjoint to $G$ and $M=GF$.  
  Then we have a natural transformation
  \[
    FMA = FGF A \xrightarrow{\epsilon_{F A}} F A\,
    \]
  which satisfies the hypotheses of Proposition \ref{quotient-universal}(i).  
  Therefore, Proposition \ref{quotient-universal} tells us that the Kleisli category is not only initial amongst adjunctions giving rise to $M$, but is initial among all functors $F\from \C \to \D$ that admit a natural transformation $FMA \to FA$.
  Note that for more general $\M$, we do not in general get an adjunction between $\C$ and $\C/\M$.

  The `$/$' notation we have used is because the universal property we have outlined exhibits $\C/\M$ as a particular weighted coequalizer of the functors $\pr_2,\blank.\blank\from \M\times\C \rightrightarrows \C$ in $\Cat$.
  It may therefore be regarded as a lax $2$-dimensional analogue of the quotient $X/G$ of a set $X$ by a monoid action.  
  
  When the action is a monad, this is a strange idea, since we are not used to thinking of a Kleisli category as quotient of the original category.  
  The reason for this is the laxness: we do not identify the objects $A$ and $X.A$ with an isomorphism, but with a morphism in one direction.  
  One way in which a Kleisli category does act like a quotient is that it allows us to transfer effects that are usually associated with higher types on to lower types.  
  For example, the type $W\to [A,W]$ encodes a state transformer with state $W$.  
  In the Kleisli category for the state monad, we can associate this stateful behaviour to the object $A$.
\end{remark}
\begin{remark}
  The proof of Proposition \ref{quotient-universal} tells us that one way of thinking of the category $\C/\M$ is as a general theory of categories that extend $\C$, and admit a factorization result whereby every morphism is the composition of a morphism from $\C$ with one of the distinguished morphisms $\phi_{X,A}$.

  Factorization results are very important in the game semantics of effects.  
  What usually happens is that when we extend a model $\G$ to a new model $\G'$, we wish to show that each morphism in $\G'$ may be written as the composite of a morphism from $\G$ composed with one of a small collection of new morphisms that are readily definable in the language -- for example, nondeterministic oracles, or simple storage cells.  
  This allows us to lift a definability result from the old category to the new one.  
  If these new morphisms can be cast as the components of some natural transformation $\psi_{X,A}\from X.A \to A$ for some action of some monoidal category $\M$ on $\G$ (and often they can), then Proposition \ref{quotient-universal}(i) gives us a functor from $\G/\M$ to $\G'$.  
  The factorization result then tells us that this functor is full, and so the model $\G'$ can also be defined as a quotient of $\G/\M$.
\end{remark}

\subsection{Monoidal and monoidal closed structure of $\C/\M$}

We saw in the last section that the $\C/\M$ construction may be regarded as a kind of lax $2$-coequalizer.  
In the category of sets, if a pair of morphisms $f,g\from A \rightrightarrows B$ is \emph{reflexive} -- i.e., if there is a morphism $h\from B \to A$ such that $f\circ h = g\circ h = \id_B$, then the coequalizer of $f$ and $g$ commutes with finite products.  

In our setting, the two functors $\pr_2$ and $\blank.\blank$ from $\M\times\C \to \C$ form a reflexive pair, since they have a common section $h\from \C \to \M\times \C$ given by $h(A) = (I, A)$.  
It turns out that a similar result then applies.

\begin{proposition}
  Given an action of a monoidal category $\M$ on a category $\C$, and an action of a monoidal category $\M'$ on a category $\C'$, there is an obvious way to define an action of $\M\times\M'$ on $\C\times\C'$.  
  Then the functor
  \[
    (\C\times\C')/(\M\times\M') \to (\C/\M) \times (\C'/\M')
    \]
  induced as in Proposition \ref{quotient-universal} is an isomorphism.
  \label{monoidality}
\end{proposition}

Now suppose that $\C$ is itself a monoidal category.  
We say that the action of $\M$ on $\C$ is \emph{monoidal} if it is a lax monoidal functor $\M\times\C\to\C$, which means that we need to have a natural transformation 
\[
  X.A\tensor Y.B\to (X\tensor Y).(A\tensor B)\,,
  \]
together with appropriate coherences.  

If the action is reader-style, given by $X.A = jX \implies A$ for an oplax monoidal functor $j\from \M\to \C$, then it is sufficient for $\C$ to be a \emph{symmetric} monoidal category, since then we have a natural transformation
\[
  (j X \implies A) \tensor (j Y \implies B)
  \rightarrow
  (j X \tensor j Y) \implies (A \tensor B)
  \rightarrow
  j (X\tensor Y) \implies (A \tensor B)\,.
  \]
Given a monoidal action, we get a natural transformation
\[
  (X\tensor Y).(A\tensor B)
  \rightarrow
  (X.A)\tensor Y.B
  \xrightarrow{\phi_{X,A}\tensor \phi_{Y,B}}
  A\tensor B\,,
  \]
which is multiplicative, inducing a functor $(\C\times\C)/(\M\times\M)\to \C/\M$.  
Composing with the isomorphism from the start of this section, we get a functor $(\C/\M)\times(\C/\M)\to\C/\M$.  
It turns out that this gives $\C/\M$ the structure of a monoidal category such that the functor $J\from \C\to\C/\M$ is a strict monoidal functor.

This generalizes the case of a monoidal monad, whose the Kleisli category is always a monoidal category.

Now, if $\C$ is monoidal closed and the action of $\M$ on $\C$ is a \emph{reader-type action}, then $\C/\M$ also inherits the monoidal closed structure from $\C$.  
Note that this may not be true for arbitrary actions, even for monads.  
For example, the Kleisli category for the powerset monad on $\Set$ (which is better known as the category of sets and relations) is monoidal closed, but with inner function space given by the Cartesian product, and not by the function space from $\Set$.

\section{Game Semantics for Probabilistic Algol}

We will now apply the ideas from the previous section to give a fully abstract game semantics for Probabilistic Algol (PA), where the model will be derived using a parametric monad.

\subsection{The discrete probability monad}

Given a set $X$, we define a \emph{discrete probability measure} on $X$ to be a function $\bP\from\powerset(X)\to[0,1]$ satisfying the following conditions.
\begin{description}
  \item[Empty set and whole space] $\bP(\emptyset) = 0$ and $\bP(X)=1$.
  \item[Countable additivity] If $\{E_i\}_{i=0}^\infty$ are pairwise-disjoint subsets of $X$, then
    \[
      \bP\left(\bigcup_{i=0}^\infty E_i\right) = \sum_{i=0}^\infty \bP(E_i)\,.
      \]
\end{description}

If instead we have $\bP(X)\le 1$, we say that $\bP$ is a \emph{sub-probability measure}

Given a set $X$, we define $P(X)$ to be the set of all discrete probability measures on $X$.  
Then we have natural maps
\begin{mathpar}
  \delta_X\from X \to P(X) \and m_X\from P(P(X)) \to X
\end{mathpar}
\begin{mathpar}
  \delta_X(x)(A) = \begin{cases}
    1 & x\in A \\
    0 & x\not\in A
  \end{cases} \and
  m_X(\tau)(A) = \sum_{\bP\in P(X)} \tau(\{\bP\})\bP(A)
\end{mathpar}
$\delta_X$ and $m_X$ give $P$ the structure of a monad on $\Set$, which we call the \emph{discrete probability monad}.  
This is a discrete form of the monad developed in \cite{Giry}.

We want to exhibit this monad as the colimit of an action, as we did for the powerset monad earlier.  
We define a \emph{discrete probability space} to be a pair $(X,\bP_X)$, where $X$ is a set and $\mu$ a discrete probability measure on $X$.  
We will normally refer to this pair using the set $X$.
Given discrete probability spaces $X$ and $Y$, a \emph{measure-preserving map} from $X$ to $Y$ is a function $f\from X \to Y$ such that for all $A\subseteq Y$ we have $\bP(X)(f\inv(A)) = \bP_Y(A)$.
It is clear that discrete probability spaces and measure-preserving maps form a category $\DProb$.

Moreover, $\DProb$ is a monoidal category: given discrete probability spaces $X$ and $Y$, we define a probability measure on the Cartesian product $X\times Y$ by
\[
  \bP_{X\times Y} (A) = \sum_{(x,y)\in A} \bP_X(\{x\})\bP_Y(\{y\})\,.
  \]

Then we have a natural isomorphism
\[
  P(X) \cong \colim_{(Y,\bP_Y)\object\DProb^{op}} [Y,X]\,,
  \]
such that the monad structure on $P$ is induced from this reader-style action of $\DProb$ on $\Set$.

This action is closely related to the action of $\Surj^{+,op}$ that we defined earlier: $\Surj^+$ is equivalent to the category of discrete probability spaces taking probabilities in the semiring $\{0,1\}$ with countable summation given by maximum.

By cutting down the category $\DProb$ we can obtain slightly different notions of probability.  
We will be using the full subcategory $\BProb$ where the objects $(X,\bP_X)$ are those that satisfy the following two additional conditions.
\begin{description}
  \item[Countability] The set $X$ is countable.
  \item[Finite support] There is a finite subset $E\subseteq X$ such that $\bP_X(E)=1$.
\end{description}

\subsection{Probabilistic Algol}

PA was introduced in \cite{DanosHarmer}, where it was also given a fully abstract game semantics.
The base language is the language Idealized Algol (IA) from \cite{SamsonGuyIAPassive}, and it has been extended with a constants $\coin\from\bool$.  
To make things a little more interesting, we will extend this further to incorporate an infinite family of constants $\coin_p\from\bool$, where $p$ ranges over the real interval $[0,1]$.  

We give the language a small-step operational semantics.  
The rules for the deterministic terms may be derived from the operational semantics given in \cite{SamsonGuyIAPassive} (see also \cite{JimGuySmallStep}); for example:
\[
  \inferrule*{ }{\ia s {v \coloneqq n} \opto \ia {\stup s v n} \skipp}\,.
  \]

We add two small-step rules for the $\coin_p$ primitives:

\begin{mathpar}
  \inferrule*[left={heads${}_p$}]{ }{\ia s \coin_p \opto \ia s \true}
  \and
  \inferrule*[left={tails${}_{1-p}$}]{ }{\ia s \coin_p \opto \ia s \false}\,.
\end{mathpar}

These are the only nondeterministic parts of the operational semantics.  

Given a small-step evaluation $\pi = M \to M_1 \to \cdots \to \x$ of a term $M$ of ground type $o$, we define the \emph{probability} $p(\pi)$ of $\pi$ to be the product of all the indices $q$ belonging to \LeftTirName{heads${}_q$} or \LeftTirName{tails${}_q$} rules.

This defines a discrete sub-probability measure on the set corresponding to the ground type $o$ given by
\[
  \bP(\{x\}) = \sum_{\substack{\text{evaluations}\\\pi = M \to\dots \x}}p(\pi)\,.
  \]
This is only a subprobability measure, because some evaluations may diverge.

\cite{DanosHarmer} shows how we can define other probabilistic terms using $\coin$.  
Specifically, they define a term $\polydie\from\bN\to\bN$, where the behaviour of $\polydie\,n$ is to return the result of tossing an $n$-sided die (with sides numbered $0$ to $n-1$), with each outcome occurring with probability $1/n$.  
The definition of $\polydie\,n$ in terms of $\coin$ works as follows.  
We choose some number $k$ such that $2^k\ge n$ and then toss the coin $n$ times to give one of $2^k$ outcomes.  
If an outcome corresponds (under some fixed correspondence $n\hookrightarrow 2^k$) to a natural number below $n$, then we return that number; otherwise, we repeat the process until we get such a number.  

This process will terminate with probability $1$, and since each $m<n$ occurs with equal probability, they must all occur with probability $1/n$.

\subsection{Extending a semantics for IA to a semantics for PA}

Let $\G$ be a category that admits an adequate semantics for Idealized Algol that satisfies finite definability.
The reader may assume that we are dealing with the category of games and visible strategies given in \cite{SamsonGuyIAPassive}.

We assume that $\G$ is symmetric monoidal closed and there is a lax monoidal functor $\Set\to \G$, giving the denotation of the ground types $\nat$,$\bool$ and $\com$ and their structural functions.
So we get a lax monoidal functor $\BProb \to \G$ by composing this with the forgetful functor $\BProb\to\Set$.

This gives rise to a reader-style action of $\BProb^{op}/\G$, and the resulting category $\G/\BProb^{op}$ is therefore a symmetric monoidal closed category.  

The equivalence relation on morphisms is still too weak for our purposes.  
For example, if the boolean type $\bB$ carries a Bernoulli distribution with coefficient $p$, then the diagonal map $\Delta\from \bB\to \bB\times\bB$ should be equivalent to the identity $\id\from\bB\times\bB\to\bB\times\bB$ as terms of type $\bB\times\bB$, because the projections on to each component are equivalent.  
However, they are not yet equivalent in our category.  

To improve the equivalence relation, we use the concept of \emph{testing morphism} for an object $A$ that is the denotation of an Idealized Algol term, which the author introduced in \cite{CslPaper}.
This is a term of type $(\bN \to A) \to A$ that is the denotation of an Idealized Algol term of the form
\[
  \test_{n_1,\cdots,n_k} = \lambda f.\new\;v\coloneqq 0 \text{ in } f(v \coloneqq \deref v + 1; \case\,v\,\blank\,n_1\,\cdots\,n_k\,\Omega)\,,
  \]
where $n_1,\cdots,n_k$ is a sequence of numbers.  
The effect of this term is to call $f$, and to pass in the numbers $n_1,\cdots,n_k$ in turn, and diverging if $f$ calls its argument more than $k$ times.
In game semantics, the sequence $n_1,\cdots,n_k$ is more easily recognized as a sequence of moves occurring in the game $\bN$ on the left.  

Given a morphism $\sigma\from \bN \to A$, we say that the sequence $n_1,\cdots,n_k$ is a \emph{play} of $\sigma$ if it is minimal (with respect to the prefix ordering) such that $\test_{n_1,\cdots,n_k}(\sigma)$ represents a non-diverging term.
In that case, the \emph{result} of the play is the strategy $\test_{n_1,\cdots,n_k}(\sigma)$ for $A$.

Now, if the set $\bN$ carries a probability distribution, then this gives rise to a sub-probability distribution on the set of plays for $\sigma$, where the probability associated to the sequence $n_1,\cdots,n_k$ is the product of the probabilities of the $n_i$.  
This induces a discrete probability distribution on the set of morphisms $1\to A$.

We say that two morphisms $\sigma,\tau\from \bN\to A$ from $1$ to $A$ in $\G/\BProb$ are \emph{probabilistically equivalent} if they give rise to the same probability distribution on the set of strategies for $A$.  
This equivalence relation refines our earlier one.

Our category is symmetric monoidal closed, but not yet Cartesian closed.  
Since every object of our category has a natural comonoid structure given by the `diagonal' $A\to A\times A$, we may make our category Cartesian closed by cutting the morphisms down to the ones that are comonoid homomorphisms.  
Note that we need to apply this equivalence relation in order to make sure that the natural morphisms $\phi_{X,A}$ are comonoid homomorphisms.

\subsection{Denotational semantics and Full Abstraction}

We may now inductively build up a denotational semantics for PA.  
The deterministic terms may all be defined inside the base category $\G$ and transported into our new category via the natural inclusion.  
The probabilistic term $\coin_p$ is given by the natural morphism $\phi_{B_p, 1}$, where $B_p$ is a Bernoulli space with coefficient $p$.  
That is to say, it is given by the pair
\[
  ((\{H,T\}, \mu(H)=p, \mu(T)=1-p), 1 \to (\{H,T\} \to \bB))\,,
  \]
where the function on the right sends $H$ to $\true$ and $T$ to $\false$.

Note now that the denotation of a term $M = C[\coin_{p_1},\cdots,\coin_{p_j}]\from T$ is given by the following morphism.
\[
  \bB^j
  \xrightarrow{\deno{b_1,\cdots,b_j\ts C[b_1,\cdots,b_j]}}
  \deno{T}\,,
  \]
where the distribution on $\bB^j$ is the product of Bernoulli distributions.

Fix an injection $r\from\bB^j\to \bN$ (for instance, the one given by binary representation) and projection functions $\pr_i \from \bN \to \bB$ such that $\pr_i(r(v))$ is the $i$-th entry of $v$.
Let $\multicoin\from\nat$ be the term (depending on the $p_i$) given by
\[
  (r\;\coin_{p_1}\,\cdots\,\coin_{p_j})\,.
  \]
Now the operational semantics of $\pr_i(\multicoin)$ is the same as that of $\coin_{p_i}$, since the projections from the product are probability-preserving.  
This means that any evaluation $\pi$ of the term $M$ may be pegged to an evaluation of the term formed by replacing each instance of $\coin_{p_i}$ in $M$ with $\pr_i(\multicoin)$.

\begin{proposition}[Computational Adequacy]
  Let $M = C[\multicoin]\from \com$ be a term of Idealized Algol, where $C$ is a multi-holed deterministic IA context.
  Let $p$ be the sum of the probabilities of all terminating evaluation sequences of $M$.

  The denotation of $M$ is a morphism $\bN \to \bC$ as outlined above.

  Then we get an induced sub-probability distribution on the set of morphisms $1\to A$.  
  The adequacy result is that $p$ is the same as the induced probability of the morphism $\deno{\skipp}$.
\end{proposition}
\begin{proof}
  Each terminating evaluation sequence of $M$ corresponds to a finite sequence $n_1,\cdots,n_k$ of the different values that $\multicoin$ takes on over the course of the evaluation of the term.  
  By inspection, we may peg this evaluation of $M$ exactly to the evaluation of the term $\test_{n_1,\cdots,n_k} (\lambda n.C[n])$ in pure deterministic Idealized Algol, and therefore this IA term must converge.  
  By the Adequacy result for Idealized Algol \cite{SamsonGuyIAPassive}, an IA term converges to $\skipp$ if and only if its denotation is $\deno{\skipp}$.  
  Therefore, there is bijective matching between terminating evaluation sequences of $M$ and plays of the morphism $\deno{M}$, with each one getting the same probability.  
  The result follows.\qed
\end{proof}

To prove full abstraction for our model, we define a further equivalence on morphisms.  
Say that two morphisms $f,g\from A \to B$ are \emph{intrinsically equivalent} if and only if for all morphisms $\alpha\from (A\to B)\to\bC$, $\alpha(f)$ and $\alpha(g)$ are probabilistically equivalent in the sense given above.
Then our full abstraction result is:

\begin{proposition}[Full Abstraction]
  Let $M,N\from T$ be two PA terms.  
  Then $M$ and $N$ are observationally equivalent if and only if their denotations $\deno{M}$ and $\deno{N}$ are intrinsically equivalent.
\end{proposition}

One direction (soundness) follows immediately from our Computational Adequacy result.  
To prove the other direction, it suffices to show that any compact strategy is definable \cite{CurienFullAbstraction}, where we say that a morphism $\sigma\from A \to (X \to B)$ from $A$ to $B$ is \emph{compact} if it is compact when considered as a morphism in $\G$.
We prove this using a factorization result:
\begin{lemma}[Deterministic factorization]
  Let $\sigma\from A \to B$ be a morphism in $\G/\BProb^{op}$.  
  Then $\sigma$ may be factorized as $\text{Det}(\sigma)(\multicoin)$, where $\text{Det}(\sigma)\from \bN \to A \to B$ is a strategy in the image of the inclusion functor $J\from \G \to \G/\BProb^{op}$, and $\multicoin$ is defined for some values $p_1,\cdots,p_k$.
\end{lemma}
\begin{proof}
  Treat $\sigma$ as a morphism $A \to (X \to B)$ in $\G$.  
  Since $X$ is finitely supported, there is a probability-preserving function $\bN \to X$, for some finitely supported distribution on $\bN$.  
  So $\sigma$ is equivalent to a morphism $A \to (\bN \to B)$, and then the factorization is automatic.\qed
\end{proof}
The most striking thing about the computational adequacy and deterministic factorization results is that they are proved entirely inside the original category $\G$.  
This demonstrates what we have been trying to achieve: to model effects using a systematic method that allows us to prove results formulaically.  
The factorization result, in particular, follows immediately from the way we have set the model up, whereas the corresponding result in traditional game semantics is often intricate and combinatorial.

\bibliographystyle{spphys}
\bibliography{effects}

\end{document}
