

\subsection{Extended example: game semantics of time complexity}
This is an extended example of using a reader monad, in the style of Example \ref{ExReaderMonadKleisli}.
Let $\G$ be the category of games and visible single-threaded strategies as in \cite{SamsonGuyIAActive}.  
So $\G$ is a sound and adequate model of Idealized Algol satisfying compact definability.
Let $\bC$ be the denotation of the command type $\com$ of Idealized Algol.  
Then the Kleisli category for the reader monad over $\bC$ is freely generated over $\G$ by a single morphism $1 \to \bC$.  

Let us extend Idealized Algol with a single term, $\sleep$, of type $\com$.
This should be thought of as a command to the computer to pause for some fixed interval of time -- let us say a second -- before resuming execution.
By Theorem \ref{FunctionalCompletenessCcc}, our Kleisli category is Cartesian closed.
So it gives us a denotational semantics of IA+$\sleep$, in which the terms of IA are interpreted within $\G$ as in \cite{SamsonGuyIAActive} and the new term $\sleep$ is interpreted by the new morphism $1 \to \bC$.

Let us now give an operational semantics to terms of IA+$\sleep$.  
Let $\Gamma$ be an IA-context.
Then a \emph{store} $s$ over $\Gamma$ is a list of pairs $(v\mapsto x)$, where each $v$ is the name for a $\Var$-type variable occurring in $\Gamma$ and $x$ is a value held in $v$.  
We write $()$ for the empty store, $(s\vert v\mapsto x)$ for the store obtained from $s$ by adding or updating the variable $v$ to hold the value $x$ and $(s\vert v\mapsto \bot)$ for the store obtained by removing the variable $v$ from $s$.
Given disjoint stores $s,t$ (i.e., stores whose sets of variables are disjoint), we write $s+t$ for the store obtained by concatenating together $s$ and $t$.  

Given a term $\Gamma\ts M$ in context, stores $s$,$s'$ over $\Gamma$, a value $c$ and a natural number $n$, we define a big-step relation
\[
  \Gamma,s\ts M\converges_n c,s'\,,
  \]
read as `in the context $\Gamma$, $s,M$ converges to $c,s'$ in time $n$'.  
The rule for the constant $\sleep$ is given by
\[
  \inferrule*
  { }
  {\Gamma,s\ts \sleep\converges_1 \skipp,s}\,,
  \]
indicating that execution of the $\sleep$ command takes up an additional time step.  
The other rules are the same as those given in \cite{SamsonGuyIAActive} for the $\converges$ relation, except that we have to modify them in order to take account of the new information we have added to the reduction relation (i.e., the time taken to converge).
For example:
\begin{mathpar}
  \inferrule*
  { }
  {\Gamma,s\ts c \converges_0 c,s}
  \and
  \inferrule*
  {\Gamma,s\ts M\converges_m\skipp,s'\\
  \Gamma,s'\ts N\converges_n\skipp,s''}
  {\Gamma,s\ts M;N\converges_{m+n}\skipp,s''}\,.
\end{mathpar}
If $P$ is a program (i.e., a term of type $\com$) then we write $P\converges_n$ as a shorthand for $,()\ts P \converges_n \skipp,()$.
We say that $P$ \emph{diverges} and write $P\diverges$ if there is no $n$ such that $P\converges_n$.

We now want a way to capture this operational behaviour within the denotational semantics.
First, we note the Computational Adequacy result proved for the game semantics of Idealized Algol.

\begin{proposition}[{\cite[21]{SamsonGuyIAActive}}]
  Let $P$ be a term of type $\com$.  
  Then $P\converges$ if and only if $\deno{P}\ne\bot$.
  \label{SamsonGuyAdequacy}
\end{proposition}

To make sense of this strategy, we need to know that $\G$ is enriched in algebraic cpos.  
Then $\bot$ is the bottom element of $\G(1,\bC)$.

Before proceeding further, we prove an important lemma.  
This lemma tells us that the operational semantics of a term of IA+$\sleep$ may be modelled within Idealized Algol; i.e., that IA+$\sleep$ is a conservative extension of the original language.

\begin{lemma}
  Let $P\from T$ be a term of IA+$\sleep$.
  Suppose that
  \[
    \Gamma,s\ts P\converges_n c,s'
    \]
  in IA+$\sleep$.


  Then for all $k\ge 0$, all contexts $\Delta$ disjoint from $\Gamma$, all stores $t$ over $\Gamma,\Delta$ disjoint from $s$ and all variables $v$ not in $\Gamma,\Delta$:
  \[
    \Delta,\Gamma,v,(t{+}s\vert v \mapsto k)\ts P[v \gets \suc !v/\sleep] \converges c,(t{+}s\vert v \mapsto k+n)
    \]
  in Idealized Algol.
  \label{IaSleepsSoundnessLemma}
\end{lemma}
\begin{proof}
  Induction on the derivation that $\Gamma,s\ts P\converges_n c,s'$.
  There are several possibilities for the last step in the derivation.
  \begin{description}
    \item[Structural congruence]
      Suppose that the last step in the derivation is
      \[
        \inferrule*{P\equiv P'\\
        \Gamma,s\ts P' \converges_n \skipp, s'}
        {\Gamma,s\ts P\converges_n \skipp, s'}\,.
        \]
      It is clear then by the definition of structural congruence that $P'[v\gets\suc !v/\sleep]\equiv P[v\gets\suc !v/\sleep]$.  
      So the claim follows by the inductive hypothesis applied to $P'$ and the structural congruence rule for Idealized Algol.
    \item[$\sleep$]
      Suppose that the last step in the derivation is
      \[
        \inferrule*{ }
        {\Gamma,s\ts \sleep \converges_1 \skipp,s}\,.
        \]
      Fix a context $\Delta$ disjoint from $\Gamma$ and a store $t$ over $\Gamma,\Delta$ that is disjoint from $s$.  
      Fix $k\ge n$.  
      Then it suffices to show that the following judgement may be proved within the operational semantics of Idealized Algol.
      \[
        \Delta,\Gamma,v,(t{+}s\vert v\mapsto k)\ts v \gets \suc !v \converges \skipp, (t{+}s\vert v\mapsto k+1)
        \]
      A derivation for this obvious fact is given in Figure \ref{SleepIADerivation}.
      \begin{SidewaysFigure}
        \centering
        \begin{subfigure}{\textheight}
          \[
            \inferrule*
            {
              \inferrule*
              {
                \inferrule*
                {
                  \inferrule*
                  {
                  }
                  {
                    \Delta,\Gamma,v,(t{+}s\vert v\mapsto k)\ts v \converges v,(t{+}s\vert v\mapsto k)
                  }
                }
                {
                  \Delta,\Gamma,v,(t{+}s\vert v\mapsto k) \ts !v \converges k,(t{+}s\vert v\mapsto k)
                }
              }
              {
                \Delta,\Gamma,v,(t{+}s\vert v\mapsto k)\ts \suc !v \converges k+1,(t{+}s\vert v\mapsto k)
              } \\
              \inferrule*
              {
              }
              {
                \Delta,\Gamma,v,(t{+}s\vert v\mapsto k)\ts v \converges v,(t{+}s\vert v\mapsto k)
              }
            }
            {
              \Delta,\Gamma,v,(t{+}s\vert v\mapsto k)\ts v\gets\suc !v\converges\skipp,(t{+}s\vert v\mapsto k+1)
            }
            \]
          \caption{The crucial part of Lemma \ref{IaSleepsSoundnessLemma} pertaining to the $\sleep$ constant is a routine Idealized Algol derivation.}
          \label{SleepIADerivation}
        \end{subfigure}
        \par\vspace{48pt}
        \begin{subfigure}{\textheight}
          \[
            \inferrule*{\Delta,\Gamma_1,v,(t{+}s\vert v\mapsto k) \ts M_1[v \gets \suc !v/\sleep] \converges c_1,(t{+}s\vert v \mapsto k + n_1) \\
            \cdots \\
            \Delta,\Gamma_p,v,(t{+}s\vert v\mapsto k + n_1 + \cdots + n_{p-1}) \ts M_p[v \gets \suc !v/\sleep] \converges c_1,(t{+}s\vert v \mapsto k+n_1+\cdots+n_p)}
            {\Delta,\Gamma,v,(t{+}s^{(0)}\vert v \mapsto k + n_1 + \cdots + n_p)\ts M[v\gets \suc !v/\sleep] \converges c,(t{+}s^{(p)}\vert v \mapsto k)}
            \]
          \caption{The big-step rules of IA+$\sleep$ give rise to big-step rules of Idealized Algol via our translation.}
          \label{GeneralIADerivation}
        \end{subfigure}
        \caption{The proof of Lemma \ref{IaSleepsSoundnessLemma} works by translating big-step derivations from IA+$\sleep$ into big-step derivations of Idealized Algol.}
        \label{IADerivations}
      \end{SidewaysFigure}
    \item[Canonical forms]
      Suppose that the last step in the derivation is
      \[
        \inferrule*{ }
        {\Gamma,s\ts_0 c\converges c,s}\,.
        \]
      Then, since $\sleep$ is not a canonical form, we have $c[v\gets \suc !v/\sleep]=c$.  
      Then we have the derivation
      \[
        \inferrule*{ }
        {\Delta,\Gamma,(t{+}s\vert v\mapsto k)\ts c \converges c,(t{+}s\vert v\mapsto k)}\,.
        \]
    \item[Sequencing]
      Suppose instead that the last step in the derivation is
      \[
        \inferrule*{\Gamma,s\ts M \converges_m \skipp, s'\\
        \Gamma,s'\ts N \converges_n \skipp, s''}
        {\Gamma,s\ts M;N \converges_{m+n} \skipp, s''}\,.
        \]
      Fix $k\ge 0$.  
      By induction on $M,N$, we have
      \begin{mathpar}
        \Delta,\Gamma,v,(t{+}s\vert v\mapsto k)\ts M[v\gets\pred !v/\sleep]\converges \skipp,(t{+}s'\vert  v\mapsto k+m)\,;
        \and
        \Delta,\Gamma,v,(t{+}s'\vert  v\mapsto k+m) \hspace{144pt} \par\vspace{-8pt} \\ \par \hspace{144pt} \ts N[v\gets\pred !v/\sleep]\converges \skipp,(t{+}s''\vert v\mapsto k+m+n)\,.
      \end{mathpar}
      It follows that
      \begin{mathpar}
        \Delta,\Gamma,v,(t{+}s\vert v\mapsto k) \hspace{180pt} \par\vspace{-8pt} \\ \par \hspace{180pt} \ts (M;N)[v\gets\pred !v/\sleep] \converges \skipp, (t{+}s''\vert v\mapsto k+m+n)\,,
      \end{mathpar}
      by the sequencing rule of Idealized Algol.
    \item[Remaining rules]
      Having given the rule for canonical forms and sequencing explicitly, we now give a general technique by which we can deal with all the remaining rules.

      Each rule is of the form
      \[
        \inferrule*{ \Gamma_1,s^{(0)}\ts z_1\converges_{n_1}\gamma_1,s^{(1)} \\
        \cdots \\
        \Gamma_p,s^{(p - 1)}\ts z_p \converges_{n_p} \gamma_p,s^{(p)}}
        { \Gamma,s^{(0)}\ts z \converges_{n_1+\cdots+n_p} \gamma,s^{(p)}}
        \]
      (for $p\in\{0,1,2,3\}$), where
      \[
        \inferrule*{ \Gamma_1,s^{(0)}\ts z_1\converges \gamma_p,s^{(1)} \\
        \cdots \\
        \Gamma_p,s^{(p - 1)}\ts z_p \converges \gamma_1,s^{(p)}}
        { \Gamma,s^{(0)}\ts z \converges \gamma,s^{(p)}}
        \]
      is a rule for Idealized Algol.

      Here, each $z_i,z,\gamma_i,\gamma$ is a term of Idealized Algol (without $\sleep$) that may have free variables that are not included in the $\Gamma_i,\Gamma$.  
      These free variables $(x_\alpha)$ may be shared between the $z_i,z,\gamma_i,\gamma$, introducing a dependence between them.
      For example, in the rule for sequencing, the terms on the top are single-variable terms of IA, where `$M$' and `$N$' are themselves free variables.  
      These variables are bound in the expression `$M;N$' on the bottom of the rule.

      Fix some such rule, and suppose that it is used in the last step of the deduction.  
      So the last step looks like this.
      \[
        \inferrule*{ \Gamma_1,s^{(0)}\ts M_1\converges_{n_1}c_1,s^{(1)} \\
        \cdots \\
        \Gamma_p,s^{(p - 1)}\ts M_p \converges_{n_p} c_p,s^{(p)}}
        { \Gamma,s^{(0)}\ts M \converges_{n_1+\cdots+n_p} c,s^{(p)}}
        \]
      Here, the $M_i,M,c_i,c$ fit the pattern given by the $z_i,z,\gamma_i,\gamma$.  
      By induction, for each $i=1,\cdots,p$, the following judgement may be proved in Idealized Algol.
      \begin{mathpar}
        \Delta,\Gamma,v,(t{+}s^{(i)}\vert v \mapsto k + n_1 + \cdots + n_{i-1})
        \hspace{144pt} \par\vspace{-20pt} \\ \par \hspace{144pt}
        \ts
        M[v \gets \suc !v] \converges c_i,(t{+}s^{(i)}\vert v \mapsto k + n_1 + \cdots + n_i)
      \end{mathpar}

      Now we know that the $M_i,M,c_i,c$ may be obtained from the $z_i,z,\gamma_i,\gamma$ by substituting in actual terms $N_\alpha$ for the free variables $x_\alpha$ in these expressions.  
      Moreover, since the primitive $\sleep$ is not mentioned anywhere in these rules, it follows that we have
      \begin{IEEEeqnarray*}{rCl}
        M_i[v\gets\suc !v/\sleep] & = & (z_i[N\alpha/x_\alpha])[v\gets \suc !v/\sleep] \\
        & = & z_i[(N[v\gets \suc !v/\sleep])/x_\alpha]\,,
      \end{IEEEeqnarray*}
      and similarly for the $z,\gamma_i,\gamma$.
      Therefore, the $M_i,M,c_i,c$ still fit the pattern given by the $z_i,z,\gamma_i,\gamma$, even after making our substitution.  
      It follows that we may derive
      \begin{mathpar}
        \Delta,\Gamma,v,(t{+}s^{(0)}\vert v \mapsto k)\hspace{180pt}\par\vspace{-8pt}\\ \par
        \hspace{180pt}\ts M[v\gets \suc !v/\sleep] \converges c,(t{+}s^{(p)}\vert v \mapsto k+n_1+\cdots+n_p)
      \end{mathpar}
      using the derivation shown in Figure \ref{GeneralIADerivation}.
  \end{description}
  This completes the induction.
\end{proof}

In light of this result, we can make a definition relating the denotational and operational semantics of terms of type $\nat$.  
First, observe that since the $\beta$ rule is valid in our semantics, we may write the denotation of a term $M\from\com$ of IA+$\sleep$ as the composite
\[
  1 \xrightarrow{\sleep}
  \bC \xrightarrow{\deno{c\ts M[c/\sleep]}}
  \bC
  \]
in $\Kl_{R_\bC}$.

Working in the original category $\G$, however, we may identify this composite with the morphism
\[
  \deno{\lambda c.M[c/\sleep]} \from 1 \to (\bC \to \bC)\,.
  \]
Now we have a morphism
\begin{mathpar}
  \eta = \deno{f\from \com\to \com \ts \lambda v.f(v\gets \suc!v);!v}\from (\bC \to \bC) \to (\Var \to \bN)\,.
\end{mathpar}
If we form the composite
\[
  1 \xrightarrow{\deno{\lambda c.M[c/\sleep]}}
  (\bC \to \bC) \xrightarrow{\eta}
  (\Var \to \bN) \xrightarrow{\new_\bN}
  \bN
  \]
then the morphism we get will be the denotation in $\G$ of the term
\[
  \ts \new\;v=0\text{ in } M[v\gets \suc !v/\sleep];!v\from \nat
  \]
(since the $\beta$ rule is valid for the semantics in $\G$).
Lemma \ref{IaSleepsSoundnessLemma} tells us that if $M\converges_n\skipp$ then this term converges to $n$ in Idealized Algol.

Before we introduce our adequacy result, we need one more definition in order to take account of the fact that Proposition \ref{SamsonGuyAdequacy} applies to terms of type $\com$, not $\nat$.

\newcommand{\test}{{\mathtt{test}}}
\begin{definition}
  Define terms $\test_n\from\nat\to\com$ inductively by
  \begin{gather*}
    \test_0 = \lambda m.\IfO m\text{ then }\skipp\text{ else }\Omega\,; \\
    \test_{n+1} = \lambda m.\IfO m\text{ then }\Omega\text{ else }\test_n\,. (\pred m)
  \end{gather*}
  So $\test_n$ converges if its input evaluates to $n$ and diverges otherwise.
  
  Let $t_n\from\bN \to \bC$ be the denotation in $\G$ of $\test_n$.
\end{definition}

\begin{remark}
  In the model $\G$ of Idealized Algol, every morphism $1 \to \bN$ is definable, so if $\sigma\from 1 \to \bN$ is such a morphism, then there must be at most one $n\in\omega$ such that the composite $\sigma;t_n$ is not equal to $\bot$.
\end{remark}

\newcommand{\tit}{\mathrm{tt}}
\begin{definition}
  Let $\sigma\from 1 \to \bC$ be a Kleisli morphism in $\Kl_{R_{\bC}}$, considered as a morphism $1 \to (\bC \to \bC)$ in $\G$.  

  If there is some (necessarily unique) $n\in\omega$ such that the composite
  \[
    1 \xrightarrow{\sigma}
    (\bC \to \bC) \xrightarrow{\eta}
    (\Var \to \bN) \xrightarrow{\new_\bN}
    \bN \xrightarrow{t_n}
    \bC
    \]
  is not equal to $\bot$, then we say that $n$ is the \emph{time taken for $\sigma$ to converge}, or $\tit(\sigma)$.
  If there is no such $n$, we say that $\tit(\sigma)=\infty$.
  \label{DefTimeTaken}
\end{definition}

From our discussion above, we have proved the following result.

\begin{proposition}[Soundness for IA+$\sleep$]
  Let $P\from \com$ be a term of IA+$\sleep$ and let $\sigma\from \bC \to \bC$ be its denotation in $\Kl_{R_\bC}$, considered as a morphism in $\G$.
  If $P\converges_n$ then $\tit(\sigma)=n$.
  \label{IaSleepSoundness}
\end{proposition}

In order to prove the other direction (adequacy), we need to prove a kind of converse to Lemma \ref{IaSleepsSoundnessLemma}.

\begin{lemma}
  Let $\Gamma\ts M\from \com$ be a term of IA+$\sleep$ in context.
  Let $s,s'$ be stores, let $a,b\in\bN$, let $c$ be a canonical form and let $v$ be a variable not included in $\Gamma$.  
  Suppose that
  \[
    \Gamma,v,(s\vert v\mapsto a) \ts M \converges c,(s'\vert v\mapsto b)
    \]
  in Idealized Algol.
  Then there exists $n$ such that
  \[
    \Gamma,s \ts M \converges_n c,s'
    \]
  in IA+$\sleep$.
\end{lemma}
\begin{proof}
  Induction on the derivation that $\Gamma,v,(s\vert v\mapsto a)\ts M\converges c,(s'\vert v\mapsto b)$.  
  Most cases are self-explanatory.  
  For example, suppose that the last step in the derivation is an instance of the sequencing rule:
  \[
    \inferrule{\Gamma,v,(s\vert v \mapsto a) \ts M[v\gets\suc !v/\sleep] \converges \skipp, s' \\
    \Gamma,v,s' \ts N[v\gets\suc !v/\sleep] \converges c, (s''\vert v \mapsto b)}
    {\Gamma,v(s\vert v \mapsto a) \ts (M;N)[v\gets\suc !v/\sleep] \converges c,(s''\vert v \mapsto b)}\,.
    \]
  Since there is no provable sequent of Idealized Algol in which a defined variable on the left ceases to be defined on the right, we may deduce that $s'$ can be written as $(t\vert v \mapsto d)$.  
  Therefore, we may apply the inductive hypothesis to $M$ and $N$, which tells us that there must be $m,n$ such that
  \begin{mathpar}
    \Gamma,s \ts M\converges_m\skipp,t\,;
    \and
    \Gamma,t \ts M\converges_n c,s''\,.
  \end{mathpar}
  From this it follows that
  \[
    \Gamma,s \ts M;N \converges_{m+n} c,s''\,.
    \]
  Crucial to this argument is the fact that
  \[
    (M[v\gets\suc !v/\sleep]);(N[v\gets\suc !v/\sleep]) = (M;N)[v\gets\suc!v/\sleep]\,.
    \]
  For almost all the other rules, a similar equality holds.  
  The only case where there is a problem is the first rule for $\blank\gets\blank$, since this may introduce a new copy of $v\gets\suc!v$ on the bottom that does not arise from copies of $v\gets\suc!v$ on the top.

  The offending term occurs when $M=\sleep$ and so $M[v\gets\suc!v/\sleep]=v\gets\suc!v$.
  In that case, the inference is
  \[
    \inferrule{\Gamma,(s\vert v \mapsto a)\ts \suc ! v \converges b,s'\\
    \Gamma,s'\ts v \converges v,s''}
    {\Gamma,(s\vert v\mapsto a)\ts v \gets\suc!v \converges \skipp,(s''\vert v\mapsto b)}\,.
    \]
  By following the derivation upwards, we may deduce that in fact $s=s'=s''$.
  But in that case the derivation of $\Gamma,s\ts_n M\converges \skipp,s''$ is one of the base rules:
  \[
    \inferrule{ }
    {\Gamma,s \ts \sleep \converges_1 \skipp,s}\,.
    \]
  The other cases are routine.  
  This completes the induction.
  \label{IaSleepAdequacyLemma}
\end{proof}

\begin{theorem}[Computational adequacy for IA+$\sleep$]
  Let $P\from\com$ be a term of IA+$\sleep$ and let $\sigma\from \bC\to \bC$ be its denotation in $\Kl_{R_\bC}$, considered as a morphism in $\G$.
  Then $P\converges_n$ if and only if $\tit(\sigma)=n$.
  \label{IaSleepComputationalAdequacy}
\end{theorem}
\begin{proof}
  The forward direction is Proposition \ref{IaSleepSoundness}.  
  For the reverse direction, note that if $\tit(\sigma)=n$, it means that
  \[
    \deno{\test_n(\new v=0\text{ in }P[v\gets \suc!v/\sleep];!v])} \ne \bot\,,
  \]
  which, by Proposition \ref{SamsonGuyAdequacy}, means that
  \[
    \test_n(\new v=0\text{ in } P[v\gets\suc!v/\sleep];!v])\converges
  \]
  in Idealized Algol, and therefore that
  \[
    v,(v\mapsto 0) \ts P[v\gets\suc!v/\sleep]\converges \skipp,(v\mapsto n)\,.
    \]
  From Lemma \ref{IaSleepAdequacyLemma}, we deduce that we must have
  \[
    ,() \ts P \converges_m \skipp,()
    \]
  in IA+$\sleep$ for some $m$, and by Lemma \ref{IaSleepsSoundnessLemma} we must have $m=n$.  
  Therefore, $P\converges_n$.
\end{proof}

The last definition we need to make is the standard intrinsic equivalence relation.

\begin{definition}
  Let $\sigma,\tau\from A \to B$ be morphisms in $\Kl_{R_\bC}$.  
  By currying, we may consider $\sigma,\tau$ to be morphisms $1 \to (A \to B)$.  
  We say that $\sigma\sim\tau$ if for all morphisms $\alpha\from (A \to B) \to \bC$, we have $\tit(\sigma;\alpha)=\tit(\sigma;\beta)$.
\end{definition}

We now have all the tools necessary to prove that $Kl_{R_\bC}$ is fully abstract for IA+$\sleep$.
The technique we use is standard and relies on a factorization result, as with the proofs given in \cite{SamsonGuyIAActive} and \cite{mcCHFiniteND}.  
The difference here is that our factorization result is immediate, and does not rely on any combinatorial arguments.

\begin{theorem}
  Let $M,N\from T$ be two terms of IA+$\sleep$.  
  Then $M$ and $N$ are observationally equivalent if and only if $\deno{M}\sim\deno{N}$.
  \label{IaSleepTimeComplexity}
\end{theorem}
\begin{proof}
  First suppose that $M$ and $N$ are not observationally equivalent.  
  So there is some context $-\from T\ts C[-]\from\com$ such that $C[M]$ and $C[N]$ have different operational behaviours in IA+$\sleep$.
  By Theorem \ref{IaSleepComputationalAdequacy} we know that for any program $P$ we have $P\converges_n$ if and only if $\tit(\deno P)=n$ and $P\diverges$ if and only if $\tit(\deno P)=\infty$.
  Therefore, we must have $\tit(\deno{C[M]})\ne \tit(\deno{C[n]})$.  
  But since $\deno{C[M]}=\deno{M};\deno{C}$ and $\deno{C[N]}=\deno{N};\deno{C}$, we must have $\deno M\not\sim \deno N$.

  Conversely, suppose that $\deno M \not\sim \deno N$.  
  So there is some $\alpha\from \deno{T} \to \bC$ such that $\deno{M};\alpha\ne\deno{N};\alpha$.

  Let us now work in the category $\G$.  
  So $\deno{M},\deno{N}$ may be regarded as morphisms $1 \to (\bC \to \deno{T})$ and $\alpha$ may be regarded as a morphism $\deno{T} \to (\bC \to \bC)$.
  Then the composites $\deno M;\alpha$ and $\deno N;\alpha$ within $\Kl_{R_\bC}$ are given by the following composites in $\G$.
  \begin{mathpar}
    1 \xrightarrow{\deno M}
    (\bC \to \deno T) \xrightarrow{\bC \to \alpha}
    (\bC \to (\bC \to \bC)) \xrightarrow{\mu}
    (\bC \to \bC)
    \and
    1 \xrightarrow{\deno N}
    (\bC \to \deno T) \xrightarrow{\bC \to \alpha}
    (\bC \to (\bC \to \bC)) \xrightarrow{\mu}
    (\bC \to \bC)
  \end{mathpar} 
  Without loss of generality, suppose that there exists some $n$ such that
  \begin{mathpar}
    \tit(\deno{M};(\bC \to \alpha);\mu)=n \and \tit(\deno{N};(\bC \to \alpha);\mu)\ne n\,.
  \end{mathpar}
  This means that we must have
  \begin{mathpar}
    \deno M;(\bC \to \alpha);\mu;\eta;\new_\bC;t_n \ne \bot
    \and
    \deno N;(\bC \to \alpha);\mu;\eta;\new_\bC;t_n = \bot
  \end{mathpar}
  in $\G$ (where $\eta$ is as defined in Definition \ref{DefTimeTaken}).

  $\G$ is enriched in algebraic cpos, so $\alpha$ is the least upper bound of its compact approximants.  
  It follows that there is some compact $\alpha'\subset \alpha$ such that
  \begin{mathpar}
    \deno M;(\bC \to \alpha');\mu;\eta;\new_\bC;t_n \ne \bot
    \and
    \deno N;(\bC \to \alpha');\mu;\eta;\new_\bC;t_n = \bot\,.
  \end{mathpar}
  In other words:
  \begin{mathpar}
    \tit(\deno M;(\bC \to \alpha');\mu) = n\,;
    \and
    \tit(\deno N;(\bC \to \alpha');\mu) \ne n\,.
  \end{mathpar}
  Now the compact definability result for Idealized Algol \cite[17]{SamsonGuyIAActive} tells us that $\alpha'$ must be the denotation of some IA term $x\from T \ts C[x] \from \com \to \com$, which is therefore the denotation of $x\from T \ts C[x]\,\sleep \from \com$ in $\Kl_{R_\bC}$.
  So we get
  \begin{mathpar}
    \tit(\deno{C[M]\,\sleep}) = n\,;
    \and
    \tit(\deno{C[N]\,\sleep}) \ne n\,.
  \end{mathpar}
  By Theorem \ref{IaSleepComputationalAdequacy}, $C[M]\,\sleep\converges_n$ and $C[N]\,\sleep\not\converges_n$.  
  Therefore, $M$ and $N$ are observationally inequivalent in IA+$\sleep$.
\end{proof}
