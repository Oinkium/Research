\documentclass[11pt]{report}

\def\FEWFONTS{1}
\def\THESIS{1}
\usepackage[utf8]{inputenc}

\usepackage{graphicx} % support the \includegraphics command and options

\usepackage{parskip} % Activate to begin paragraphs with an empty line rather than an indent

%%% PACKAGES
\usepackage{booktabs} % for much better looking tables
\usepackage{array} % for better arrays (eg matrices) in maths
\ifdefined\BEAMER
\else
\usepackage{paralist} % very flexible & customisable lists (eg. enumerate/itemize, etc.)\prefix\t$.
\fi
\usepackage{verbatim} % adds environment for commenting out blocks of text & for better verbatim
\ifdefined\BEAMER
\else
\ifdefined\THESIS
\usepackage{subcaption}
\else
\usepackage{subfig} % make it possible to include more than one captioned figure/table in a single float
\fi
\fi
\usepackage{mathtools} % for the all important \coloneqq symbol
\usepackage{hyperref} % for hyperreferences
\usepackage{IEEEtrantools} % for \IEEEeqnarray
\usepackage{pbox} % for \pbox
\usepackage{multirow,bigdelim} % for \multirow
\usepackage{lettrine} % For the drop cap
\usepackage{mathpartir} % for \inferrule, \inferrule* and the mathpar environment
\usepackage{listings}

\usepackage{caption}
\captionsetup{singlelinecheck=off}

\ifdefined\NOTARTICLE
\else

%%% ToC (table of contents) APPEARANCE
\usepackage[nottoc,notlof,notlot]{tocbibind} % Put the bibliography in the ToC
\usepackage[titles,subfigure]{tocloft} % Alter the style of the Table of Contents
\renewcommand{\cftsecfont}{\rmfamily\mdseries\upshape}
\renewcommand{\cftsecpagefont}{\rmfamily\mdseries\upshape} % No bold!

\fi

%% Font things %%
\usepackage{amssymb}
\usepackage{cmll} % Linear logic symbols!
\ifdefined\FEWFONTS
\else
\usepackage{bm} % for bold Greek letters
\fi
\usepackage{stmaryrd}
\usepackage{bbm}

%% Get the sqsubsetneqq character from the mathabx package
\DeclareFontFamily{U}{mathb}{\hyphenchar\font45}
\DeclareFontShape{U}{mathb}{m}{n}{
      <5> <6> <7> <8> <9> <10> gen * mathb
      <10.95> mathb10 <12> <14.4> <17.28> <20.74> <24.88> mathb12
      }{}
\DeclareSymbolFont{mathb}{U}{mathb}{m}{n}

\DeclareMathSymbol{\sqsubsetneq}    {3}{mathb}{"88}
\DeclareMathSymbol{\varsqsubsetneq} {3}{mathb}{"8A}
\DeclareMathSymbol{\varsqsubsetneqq}{3}{mathb}{"92}
\DeclareMathSymbol{\sqsubsetneqq}   {3}{mathb}{"90}

%% Get the left and right moons from the wasysym package

\DeclareFontFamily{U}{wasy}{}
\DeclareFontShape{U}{wasy}{m}{n}{ <5> <6> <7> <8> <9> gen * wasy
      <10> <10.95> <12> <14.4> <17.28> <20.74> <24.88>wasy10  }{}
\DeclareFontShape{U}{wasy}{b}{n}{ <-10> sub * wasy/m/n
 <10> <10.95> <12> <14.4> <17.28> <20.74> <24.88>wasyb10 }{}
\DeclareFontShape{U}{wasy}{bx}{n}{ <-> sub * wasy/b/n}{}

\def\wasyfamily{\fontencoding{U}\fontfamily{wasy}\selectfont}
\def\leftmoon   {\mbox{\wasyfamily\char36}}
\def\rightmoon  {\mbox{\wasyfamily\char37}}

%% Lists %%
\usepackage{enumerate}

%% Graphics %%
\usepackage{tikz}
\usetikzlibrary{cd}
\usetikzlibrary{patterns}
\usetikzlibrary{calc}
\usetikzlibrary{decorations.pathmorphing}
\usetikzlibrary{positioning}

\tikzset{inlinearrows/.style={anchor=base,baseline,x=0.6\baselineskip,y=0.6\baselineskip}}

\ifdefined\BEAMER
\else

%% Theorems! %%
\usepackage{amsthm}
\theoremstyle{plain} % Theorems, lemmas, propositions etc.
\newtheorem{theorem}{Theorem}[section]
\newtheorem{lemma}[theorem]{Lemma}
\newtheorem{proposition}[theorem]{Proposition}
\newtheorem{corollary}[theorem]{Corollary}
\newtheorem{fact}[theorem]{Fact}
\newtheorem{construction}[theorem]{Construction}
\theoremstyle{definition} % Definitions etc.  
\newtheorem{definition}[theorem]{Definition}
\newtheorem{notation}[theorem]{Notation}
\theoremstyle{remark} % Remarks
\newtheorem{remark}[theorem]{Remark}
\newtheorem{remarks}[theorem]{Remarks}
\newtheorem{example}[theorem]{Example}
\newtheorem{question}[theorem]{Question}
\newtheorem{slogan}[theorem]{Slogan}

\newtheoremstyle{note} {3pt} {3pt} {\itshape} {} {\itshape} {:} {.5em} {} % For short notes
\theoremstyle{note}
\newtheorem{note}[theorem]{Note}

\fi

%% Exercises and answers %%
\usepackage{answers}

\newtheoremstyle{exercisestyle}% name
  {6pt}   % ABOVESPACE
  {6pt}   % BELOWSPACE
  {\itshape}  % BODYFONT
  {0pt}       % INDENT (empty value is the same as 0pt)
  {\bfseries} % HEADFONT
  {.}         % HEADPUNCT
  {3pt} % HEADSPACE
  {}          % CUSTOM-HEAD-SPEC

\theoremstyle{exercisestyle}
\newtheorem{exercise}{Exercise}
\newtheorem{answerthm}{Exercise}

\Newassociation{answer}{answerthm}{answers}
\newcommand{\answerthmparams}{}

%% Changes to enumerate things so they look better %%\sigma$

\makeatletter
\def\enumfix{%
\if@inlabel
 \noindent \par\nobreak\vskip-\topsep\hrule\@height\z@
\fi}

\let\olditemize\itemize
\def\itemize{\enumfix\olditemize}
\let\oldenumerate\enumerate
\def\enumerate{\enumfix\oldenumerate}

%% Random crap %%
\usepackage{xifthen}

\makeatletter
\def\thm@space@setup{%
  \thm@preskip=\parskip \thm@postskip=0pt
}
\makeatother

\makeatletter
\newcommand*{\relrelbarsep}{.386ex}
\newcommand*{\relrelbar}{%
  \mathrel{%
    \mathpalette\@relrelbar\relrelbarsep
  }%
}
\newcommand*{\@relrelbar}[2]{%
  \raise#2\hbox to 0pt{$\m@th#1\relbar$\hss}%
  \lower#2\hbox{$\m@th#1\relbar$}%
}
\providecommand*{\rightrightarrowsfill@}{%
  \arrowfill@\relrelbar\relrelbar\rightrightarrows
}
\providecommand*{\leftleftarrowsfill@}{%
  \arrowfill@\leftleftarrows\relrelbar\relrelbar
}
\providecommand*{\xrightrightarrows}[2][]{%
  \ext@arrow 0359\rightrightarrowsfill@{#1}{#2}%
}
\providecommand*{\xleftleftarrows}[2][]{%
  \ext@arrow 3095\leftleftarrowsfill@{#1}{#2}%
}
\makeatother

\newcommand{\catname}[1]{{\normalfont\textbf{#1}}}
\newcommand{\Rings}{\catname{CRing}}
\newcommand{\CAT}{\catname{CAT}}
%\newcommand{\Top}{\catname{Top}}
\newcommand{\Set}{\catname{Set}}
\newcommand{\Cat}{\catname{Cat}}
\newcommand{\MonCat}{\catname{MonCat}}
\newcommand{\SymmMonCat}{\catname{SymmMonCat}}
\newcommand{\Cont}{\catname{Cont}}
\newcommand{\Sch}{\catname{Sch}}
\newcommand{\Rel}{\catname{Rel}}
\newcommand{\Coh}{\catname{Coh}}
\newcommand{\Inj}{\catname{Inj}}
\newcommand{\Dcpo}{\catname{Dcpo}}
\newcommand{\Mod}[1][]{\ifthenelse{\isempty{#1}}{\catname{Mod}}{#1\catname{mod}}}
\DeclareMathOperator{\sh}{Sh}
\newcommand{\Sh}[1][]{\ifthenelse{\isempty{#1}}{\sh}{\sh(#1)}}
\newcommand{\map}[3]{#2\xrightarrow{#1} #3}
\newcommand*\from{\colon}
\newcommand*\bigto{\Rightarrow}
\newcommand{\cmap}[3]{#1\from{}#2\to{}#3}
\newcommand\oppcat[1]{#1^{\mathrm{op}}}
\newcommand{\object}{\colon}
\DeclareRobustCommand{\vmap}[3] {\begin{tikzcd} #2 \arrow[d, "#1"] \\ #3 \end{tikzcd}}
\newcommand{\partref}[1]{(\ref{#1})}
\newcommand{\intgrpd}[4] {#1 \xrightrightarrows[#3]{#4} #2}
\DeclareRobustCommand{\bigintgrpd}[4] {\begin{tikzcd}[ampersand replacement=\&] #1 \arrow[r, shift left=0.5ex, "#3"] \arrow[r, shift right=0.5ex, "#4"'] \& #2 \end{tikzcd}}

\usepackage{xspace}

\newcommand{\etale}{\'{e}tale\xspace}
\newcommand{\Etale}{\'{E}tale\xspace}

\def \inv {^{-1}}

\DeclareMathOperator{\id}{id}
\DeclareMathOperator{\op}{op}
\DeclareMathOperator{\pr}{pr}
\DeclareMathOperator{\inj}{in}
\DeclareMathOperator{\pre}{{pre}}
\DeclareMathOperator{\et}{{\acute{e}t}}

\DeclareMathOperator{\Hom}{Hom}
\DeclareMathOperator{\Spec}{Spec}

\DeclareMathOperator{\ol}{ol}

\def\presuper#1#2%
  {\mathop{}%
   \mathopen{\vphantom{#2}}^{#1}%
   \kern-\scriptspace%
   #2}
\def\presub#1#2%
  {\mathop{}%
   \mathopen{\vphantom{#2}}_{#1}%
   \kern-\scriptspace%
   #2}

\newsavebox{\overlongequation}
\newenvironment{longdiagram}
 {\begin{displaymath}\begin{lrbox}{\overlongequation}$\displaystyle}
 {$\end{lrbox}\makebox[0pt]{\usebox{\overlongequation}}\end{displaymath}}

%% Our things %%

\newcommand{\neggame}[1]{\presuper{\perp}{#1}}
\newcommand{\tensor}{\otimes}
\newcommand{\Tensor}{\bigotimes}
\newcommand{\sequoid}{\oslash}
\newcommand{\varsequoid}{\vartriangleleft}
\renewcommand{\implies}{\multimap}
\newcommand{\iimpl}{\Longrightarrow}
\newcommand{\comp}[2]{#1 \circ #2}
\newcommand{\icomp}[2]{\comp{#1}{#2}}
\newcommand{\cprd}{\sqcup}
\newcommand{\bigcprd}{\bigsqcup}
\newcommand{\G}{\mathcal G}
\newcommand{\W}{\mathcal W}
\newcommand{\suchthat}{\;\colon\;}
\newcommand{\varsuchthat}{\;\mid\;}
\newcommand{\esuchthat}{\;.\;}
\newcommand{\OP}{\{O,P\}}
\newcommand{\QA}{\{Q,A\}}
\renewcommand{\L}{\mathcal L}
\newcommand{\F}{\mathcal F}
\newcommand{\U}{\mathcal U}
\newcommand{\s}{\mathfrak s}
\renewcommand{\t}{\mathfrak t}
\renewcommand{\u}{\mathfrak u}
\renewcommand{\d}{\mathfrak d}
\newcommand{\e}{\mathfrak e}
\newcommand{\emptyplay}{\epsilon}
\newcommand{\bracketed}[1]{\left({#1}\right)}
\newcommand{\bneggame}[1]{{\bracketed{\neggame{#1}}}}
\newcommand{\prefix}{\sqsubseteq}
\newcommand{\ppprefix}{\sqsubset}
\newcommand{\pprefix}{\sqsubsetneqq}
\renewcommand{\ss}{\mathbf{s}}
\newcommand{\bN}{\mathbb{N}}
\newcommand{\bC}{\mathbb{C}}
\newcommand{\bB}{\mathbb{B}}
\newcommand{\bP}{\mathbb{P}}
\newcommand{\pfun}{\rightharpoonup}
\newcommand{\grel}[1]{\underline{#1}}
\DeclareMathOperator{\length}{length}
\renewcommand{\b}{\mathfrak b}
\renewcommand{\r}{\mathfrak r}
\newcommand{\bbeta}{{\bm{\beta}}}
\newcommand{\st}{{\Sigma^*}}
\let\sec\S
\renewcommand{\S}{{\mathfrak{S}}}
\DeclareMathOperator{\cc}{cc}
\DeclareMathOperator{\subs}{subs}
\DeclareMathOperator{\ret}{ret}
\DeclareMathOperator{\zz}{zz}
\newcommand{\aaa}{\mathbf{a}}
\newcommand{\bbb}{\mathbf{b}}
\newcommand{\ccc}{\mathbf{c}}
\newcommand{\ddd}{\mathbf{d}}
\newcommand{\B}{\mathcal B}
\newcommand{\BB}{\mathbf B}
\renewcommand{\H}{\mathcal H}
\DeclareMathOperator{\assoc}{assoc}
\DeclareMathOperator{\lunit}{lunit}
\DeclareMathOperator{\runit}{runit}
\DeclareMathOperator{\dom}{dom}
\DeclareMathOperator{\sym}{sym}
\newcommand{\braid}{\sym}
\newcommand{\blank}{\,\underline{\hspace{1.5ex}}\,}
\DeclareMathOperator{\cn}{cn}
\newcommand{\impliescn}{\protect\overset{\cn}{\implies}}
\newcommand{\C}{{\mathcal{C}}}
\newcommand{\D}{{\mathcal{D}}}
\newcommand{\E}{{\mathcal{E}}}
\newcommand{\V}{{\mathcal{V}}}
\newcommand{\EE}{{\mathbf{E}}}
\DeclareMathOperator{\ev}{ev}
\newcommand{\der}{{\mathtt{der}}}
\newcommand{\mult}{{\mathtt{mult}}}
\DeclareMathOperator{\wk}{wk}
\newcommand{\toisom}{{\xrightarrow{\cong}}}
\DeclareMathOperator{\passoc}{{\mathsf{passoc}}}
\DeclareMathOperator{\pcomm}{{\mathsf{pcomm}}}
\DeclareMathOperator{\run}{{\mathsf{r}}}
\DeclareMathOperator{\lun}{{\mathsf{l}}}
\newcommand{\fcoal}[1]{{\leftmoon #1 \rightmoon}}
\DeclareMathSymbol{\co}{\mathord}{operators}{"3C}
\DeclareMathSymbol{\nw}{\mathord}{operators}{"3E}
\newcommand{\T}{\mathfrak{T}}
\renewcommand{\subset}{\subseteq}
\newcommand{\Ord}{\catname{Ord}}
\newcommand{\FS}{\mathcal{FS}}
\DeclareMathOperator{\rank}{rank}
\DeclareMathOperator{\dist}{{\mathsf{dist}}}
\DeclareMathOperator{\dec}{{\mathsf{dec}}}
\DeclareMathOperator{\str}{str}
\DeclareMathOperator{\weak}{weak}
\DeclareMathOperator{\Strat}{Strat}
\DeclareMathOperator{\OppStrat}{OppStrat}
\newcommand{\seqs}[1]{{\overline{{#1}^{*}}}}
\def\flushRight{\leftskip0pt plus 1fill\rightskip0pt}
\def\Centering{\relax\ifvmode\centering\fi}
\newcommand{\deno}[1]{\left\llbracket#1\right\rrbracket}
\newcommand{\converges}{\Downarrow}
\newcommand{\diverges}{\Uparrow}
\newcommand{\mustconverge}{\converges^{\text{must}}}
\newcommand{\Iflt}{\mathtt{If{<}\;}}
\newcommand{\Ifgt}{\mathtt{If{>}\;}}
\newcommand{\inr}{{\mathsf{inr}}}
\newcommand{\inl}{{\mathsf{inl}}}
\newcommand{{\Na}}{\bN}
\newcommand{{\cell}}{{\mathsf{cell}}}
\newcommand{\fix}{{\mathsf{fix}}}
\newcommand{\eq}{{\mathsf{eq}}}
\DeclareMathOperator{\CCom}{CCom}
\newcommand{\power}{\mathfrak P}

% Slanty things
\newcommand*{\xslant}[2][76]{%
  \begingroup
    \sbox0{#2}%
    \pgfmathsetlengthmacro\wdslant{\the\wd0 + cos(#1)*\the\wd0}%
    \leavevmode
    \hbox to \wdslant{\hss
      \tikz[
        baseline=(X.base),
        inner sep=0pt,
        transform canvas={xslant=cos(#1)},
      ] \node (X) {\usebox0};%
      \hss
      \vrule width 0pt height\ht0 depth\dp0 %
    }%
  \endgroup
}

\makeatletter
\newcommand*{\xslantmath}{}
\def\xslantmath#1#{%
  \@xslantmath{#1}%
}
\newcommand*{\@xslantmath}[2]{%
  % #1: optional argument for \xslant including brackets
  % #2: math symbol
  \ensuremath{%
    \mathpalette{\@@xslantmath{#1}}{#2}%
  }%
}
\newcommand*{\@@xslantmath}[3]{%
  % #1: optional argument for \xslant including brackets
  % #2: math style
  % #3: math symbol
  \xslant#1{$#2#3\m@th$}%
}
\makeatother

\newcommand{\seqdeno}[1]{\xslantmath{\llbracket}#1\xslantmath{\rrbracket}}

% Empty set etc.

\let\oldemptyset\emptyset
\let\emptyset\varnothing

%% Constant width xrightarrows
\newlength{\arrow}
\settowidth{\arrow}{\scriptsize$1000$}
\newcommand*{\constantwidthxrightarrow}[1]{\xrightarrow{\mathmakebox[\arrow]{#1}}}

%% Landscape pages
\usepackage{everypage}
\usepackage{environ}
\usepackage{pdflscape}
\newcounter{abspage}

\ifdefined\NOTARTICLE

\else

\makeatletter
\newcommand{\newSFPage}[1]% #1 = \theabspage
  {\global\expandafter\let\csname SFPage@#1\endcsname\null}

\NewEnviron{SidewaysFigure}{\begin{figure}[p]
\protected@write\@auxout{\let\theabspage=\relax}% delays expansion until shipout
  {\string\newSFPage{\theabspage}}%
\ifdim\textwidth=\textheight
  \rotatebox{90}{\parbox[c][\textwidth][c]{\linewidth}{\BODY}}%
\else
  \rotatebox{90}{\parbox[c][\textwidth][c]{\textheight}{\BODY}}%
\fi
\end{figure}}

\AddEverypageHook{% check if sideways figure on this page
  \ifdim\textwidth=\textheight
    \stepcounter{abspage}% already in landscape
  \else
    \@ifundefined{SFPage@\theabspage}{}{\global\pdfpageattr{/Rotate 0}}%
    \stepcounter{abspage}%
    \@ifundefined{SFPage@\theabspage}{}{\global\pdfpageattr{/Rotate 90}}%
  \fi}
\makeatother

\fi

%% PCF Things

\newcommand{\nat}{{\mathtt{nat}}}
\newcommand{\bool}{{\mathtt{bool}}}

\newcommand{\Y}{\mathbf{Y}}
\newcommand{\opto}{\longrightarrow}
\newcommand{\oopto}{\dashrightarrow}
\newcommand{\n}{{\mathtt{n}}}
\DeclareMathOperator{\IfO}{{\mathsf{If0}}}
\DeclareMathOperator{\suc}{{\mathsf{succ}}}
\DeclareMathOperator{\pred}{{\mathsf{pred}}}
\newcommand{\0}{{\mathtt{0}}}

\newcommand{\iter}{{\mathtt{iter}}}
\newcommand{\rec}{\iter}
\newcommand{\Var}{{\mathtt{Var}}}
\DeclareMathOperator{\Varr}{Var}
\newcommand{\new}{{\mathtt{new}}}
\newcommand{\case}{{\mathtt{case}}}

\newcommand{\lmam}{\mathrel{\sqsubseteq_{m\&m}}}
\newcommand{\emam}{\mathrel{\equiv_{m\&m}}}
\newcommand{\lst}{\mathrel{\lesssim}}
\newcommand{\smam}{\mathrel{\sim_{m\&m}}}
\newcommand{\amam}{\mathrel{\approx_{m\&m}}}

\newcommand{\oes}{\sim}

%% Idealized Algol things

\newcommand{\com}{{\mathtt{com}}}
\newcommand{\skipp}{{\mathsf{skip}}}
\DeclareMathOperator{\seq}{{\mathsf{seq}}}
\DeclareMathOperator{\neww}{{\mathsf{new}}}
\DeclareMathOperator{\mkvar}{{\mathsf{mkvar}}}
\newcommand{\deref}{\texttt{@}}
\DeclareMathOperator{\dereff}{\mathsf{deref}}
\DeclareMathOperator{\assign}{\mathsf{assign}}
\newcommand{\ia}[2]{\langle #1 , #2 \rangle}
\newcommand{\stup}[3]{\langle #1 \mid #2 \mapsto #3 \rangle}

%% Hyland-Ong games things

\newbox\gnBoxA
\newdimen\gnCornerHgt
\setbox\gnBoxA=\hbox{$\ulcorner$}
\global\gnCornerHgt=\ht\gnBoxA
\newdimen\gnArgHgt
\def\pv #1{%
    \setbox\gnBoxA=\hbox{$#1$}%
    \gnArgHgt=\ht\gnBoxA%
    \ifnum     \gnArgHgt<\gnCornerHgt \gnArgHgt=0pt%
    \else \advance \gnArgHgt by -\gnCornerHgt%
    \fi \raise\gnArgHgt\hbox{$\ulcorner$} \box\gnBoxA %
    \raise\gnArgHgt\hbox{$\urcorner$}}
\def\ov #1{%
    \setbox\gnBoxA=\hbox{$#1$}%
    \gnArgHgt=\ht\gnBoxA%
    \ifnum     \gnArgHgt<\gnCornerHgt \gnArgHgt=0pt%
    \else \advance \gnArgHgt by -\gnCornerHgt%
    \fi \raise\gnArgHgt\hbox{$\llcorner$} \box\gnBoxA %
    \raise\gnArgHgt\hbox{$\lrcorner$}}
\newcommand{\ct}[1]{\lceil#1\rceil}
\DeclareMathOperator{\Int}{int}

%% Nondeterministic Factorization things

\newcommand{\code}{\mathsf{code}}
\newcommand{\Det}{\mathsf{Det}}

%% Flexible strategy things

\newcommand{\stle}{{\;\le_s\;}}
\newcommand{\steq}{{\;=_s\;}}
\newcommand{\exle}{\sqsubseteq}
\newcommand{\exlub}{\bigsqcup}
\newcommand{\dv}{{\text{\lightning}}}
\DeclareMathOperator{\pocl}{pocl}
\newcommand{\plot}{\mathrel{\triangleleft}}
\newcommand{\shad}{\mathfrak{S}}
%\newcommand{\tree}{\mathfrak{T}}
\newcommand{\Tau}{T}
\newcommand{\Epsilon}{E}
\newcommand{\sw}{\triangleleft}

%% Roman numerals

\newcommand{\RN}[1]{%
  \textup{\uppercase\expandafter{\romannumeral#1}}%
}
\newcommand{\RNl}[1]{%
  \mathrel{\raisebox{1pt}{$\overline{\underline{#1}}$}}
}

%% Game language things

\newcommand{\ul}[1]{{\underline{#1}}}
\newcommand{\A}{{\mathcal{A}}}
\renewcommand{\P}{\mathcal P}
\newcommand{\M}{\mathcal M}
\newcommand{\N}{\mathcal N}
\newcommand{\X}{\mathcal X}
\newcommand{\YY}{\mathcal Y}
\newcommand{\hole}{\blank}
\newcommand{\Tct}{\xrightarrow{T}t}
\newcommand{\teamconverge}[2]{\xrightarrow{#1}#2}

%% Inference rule things
\newcommand{\rulename}[1]{\LeftTirNameStyle{#1}}
\newcommand{\ts}{\mathbin{\vdash}}
\newcommand{\nts}{\mathbin{\not\vdash}}

%% Double category things
\newcommand{\hc}[2]{\left({#1}\middle|{#2}\right)}
\newcommand{\vc}[2]{\left(\frac{#1}{#2}\right)}

%% What is going on?
\DeclareMathOperator{\Kl}{Kl}
\DeclareMathOperator{\Mell}{Mell}
\newcommand{\powerset}{\mathcal P}
\DeclareMathOperator{\ask}{{\mathsf{ask}}}
\newcommand{\sleep}{{\mathsf{sleep}}}
\newcommand{\true}{\mathbbm{t}}
\newcommand{\false}{\mathbbm{f}}
\DeclareMathOperator{\If}{\mathsf{If}}
\newcommand{\Then}{\mathrel{\mathsf{then}}}
\newcommand{\Else}{\mathrel{\mathsf{else}}}
\newcommand\cat{\mathbin{+\mkern-10mu+}}

%% Profunctor arrows

\makeatletter
\def\slashedarrowfill@#1#2#3#4#5{%
  $\m@th\thickmuskip0mu\medmuskip\thickmuskip\thinmuskip\thickmuskip
   \relax#5#1\mkern-7mu%
   \cleaders\hbox{$#5\mkern-2mu#2\mkern-2mu$}\hfill
   \mathclap{#3}\mathclap{#2}%
   \cleaders\hbox{$#5\mkern-2mu#2\mkern-2mu$}\hfill
   \mkern-7mu#4$%
}
\def\rightslashedarrowfill@{%
  \slashedarrowfill@\relbar\relbar\mapstochar\rightarrow}
\newcommand\xslashedrightarrow[2][]{%
  \ext@arrow 0055{\rightslashedarrowfill@}{#1}{#2}}
\makeatother
\newcommand{\pto}{{\xslashedrightarrow{} }}

%% Profunctors 
\DeclareMathOperator{\Prof}{Prof}
\DeclareMathOperator{\End}{End}
\DeclareMathOperator{\Endoprof}{Endoprof}

%% Our

\def\searchmacro#1{
  \AtBeginOfFiles{%
    \ifdefined#1
      \expandafter\def\csname \currfilename:found\endcsname{}%
    \fi}
  \AtEndOfFiles{%
    \ifdefined#1
      \unless\ifcsname \currfilename:found\endcsname
        \immediate\write\finder{found in '\currfilename'}%
    \fi\fi}}

%% Isomorphism arrows on commutative diagrams %%
\tikzset{Isom/.style={every to/.append style={edge node={node [sloped, above, allow upside down, auto=false]{$\cong$}}}},
         Isom'/.style={every to/.append style={edge node={node [sloped, above, allow upside down, auto=false, rotate=180]{$\cong$}}}},
         Sim/.style={every to/.append style={edge node={node [sloped, above, allow upside down, auto=false]{$\sim$}}}},
         Sim'/.style={every to/.append style={edge node={node [sloped, above, allow upside down, auto=false, rotate=180]{$\sim$}}}}}

%% Adjunctions
\newcommand{\adjunction}[4]{%
  {#1} \underset{\underset{#3}{\longleftarrow}}{\overset{\overset{#2}{\longrightarrow}}{\bot}} {#4}}        

%% Important!
\newcommand\Mellies{Melli\`{e}s\xspace}

\makeatletter
\newcommand{\colim@}[2]{%
  \vtop{\m@th\ialign{##\cr
    \hfil$#1\operator@font colim$\hfil\cr
    \noalign{\nointerlineskip\kern1.5\ex@}#2\cr
    \noalign{\nointerlineskip\kern-\ex@}\cr}}%
}
\newcommand{\colim}{%
  \mathop{\mathpalette\colim@{\rightarrowfill@\textstyle}}\nmlimits@
}
\makeatother

\makeatletter
\newcommand{\laxcolim@}[2]{%
  \vtop{\m@th\ialign{##\cr
    \hfil$#1\operator@font colim_l$\hfil\cr
    \noalign{\nointerlineskip\kern1.5\ex@}#2\cr
    \noalign{\nointerlineskip\kern-\ex@}\cr}}%
}
\newcommand{\laxcolim}{%
  \mathop{\mathpalette\laxcolim@{\rightarrowfill@\textstyle}}\nmlimits@
}
\makeatother

\DeclareMathOperator{\Colim}{colim}

\DeclareMathOperator{\DG}{DG}
\DeclareMathOperator{\RV}{RV}
\newcommand{\Rv}{\catname{Rv}}

\let\choose\undefined
\DeclareMathOperator{\choose}{\mathsf{choose}}
\DeclareMathOperator{\tr}{tr}
\DeclareMathOperator{\test}{test}

%% Slot game things %%
\newcommand{\circled}[1]{\raisebox{.5pt}{\textcircled{\raisebox{-.9pt} {#1}}}}
\newcommand{\slot}{{\circled{\$}}}

\DeclareMathOperator{\may}{may}
\DeclareMathOperator{\must}{must}

\newcommand{\encode}[1]{\lceil#1\rceil}
\DeclareMathOperator{\app}{\mathsf{app}}
\DeclareMathOperator{\lett}{\mathsf{let}}
\newcommand{\inn}{\mathrel{\mathsf{in}}}
\DeclareMathOperator{\byval}{\mathsf{byval}}

\DeclareMathOperator{\rread}{read}
\DeclareMathOperator{\wwrite}{write}

\DeclareSymbolFont{bbsymbol}{U}{bbold}{m}{n}
\DeclareMathSymbol{\bbsemicolon}{\mathbin}{bbsymbol}{"3B}
\newcommand{\semicom}{\bbsemicolon}

\newcommand{\ms}{\makebox[-1pt]{}}

\DeclareMathOperator{\Acc}{Acc}
\DeclareMathOperator{\im}{Im}
\DeclareMathOperator{\wit}{wit}

%%% END Article customizations


\usepackage{lua-visual-debug}

\begin{document}
\begin{titlepage}
  \begin{center}
    \Huge
    \textbf{The Crossroads of Categorical Algebra and Game Semantics}
 
    \vspace{0.2cm}
    \LARGE
    An investigation into the application of Kleisli categories and related constructions to the study of Full Abstraction for nondeterministic effects in Algol-like languages
 
    \vspace{0.3cm}
 
    \textbf{William John Gowers}
 
    A thesis submitted for the degree of\\
    Doctor of Philosophy
 
    \vspace{0.1cm}
 
    \Large
    Department of Computer Science\\
    University of Bath\\
    United Kingdom\\
    August 2019

    \textbf{COPYRIGHT}
    Attention is drawn to the fact that copyright of this thesis rests with the author. A copy of this thesis has
    been supplied on condition that anyone who consults it is understood to recognise that its copyright rests
    with the author and that they must not copy it or use material from it except as permitted by law or with
    the consent of the author. 
    
    This thesis may be made available for consultation within
    the University Library and may be photocopied or lent to other libraries
    for the purposes of consultation. \ul{\hspace{64pt}}
 
  \end{center}
\end{titlepage}

\let\OldClearpage\clearpage
\let\clearpage\relax

\tableofcontents
\listoffigures

\let\clearpage\OldClearpage
\newpage

\section*{Acknowledgements}

There are many people and organizations to whom I owe a debt of gratitude, and without whom this thesis would not have been written.  
My first thanks go to EPSRC research grant EP/K018868/1 for funding my PhD, keeping me fed and housed for three and a half years, and sponsoring several trips to conferences and schools at home and abroad.  
I am also indebted to the organizers of those conferences and schools, including but not limited to 
\begin{itemize}
  \item Paul Blain Levy, Juha Kontinen, Marina Lenisa, Ugo dal Lago and Gabriel Sandu, chairs of various iterations of GaLoP (and, in Paul's case, organizer of the Midlands Graduate School 2016);
  \item Filippo Bonchi and Barbara K\"{o}nig, organizers of CALCO 2017;
  \item Matthias Felleisen, Robby Findler, Matthew Flatt, Shriram Krishnamurthi, Jay McCarthy, and Justin Pombrio, teachers at the Racket Summer School 2017;
  \item Moshe Vardi, Daniel Kroening and Marta Kwiatkowska, co-chairs of FloC 2018; and
  \item Dan Ghica and Achim Jung, co-chairs of CSL 2018.
\end{itemize}
I would also like to express gratitude to those people whom I have had productive discussions with at conferences and during departmental visits: in particular, Martin Hyland, Paul-Andr\'{e} \Mellies, Paul Levy and Pierre Clairambault.

Thank you to the anonymous reviewers from CALCO 2017, LICS 2018, CSL 2018 and FoSSaCS 2019 for your careful comments, all of which have shaped this thesis in some way.

Thank you to the whole Computer Science Department at the University of Bath, particularly the Mathematical Foundations team, for all your help over the years.  
Congratulations to John Power on the conclusion of a momentous career, and thanks for helping me with my category theory.  
Thank you too to Ben, David and Alessio for organizing the Departmental Seminar at Bath and for being such good colleagues.

Thank you to my family for all your support during my PhD.  
Thank you to Ellie for your patience and love, and for enduring a three-year long-distance relationship while I was in Bath.  
This thesis is dedicated to you.

Lastly, there would be no thesis at all without my PhD supervisor, Jim Laird, who took me on when I had little to no knowledge of the subject and has carefully guided me throughout.  
That I am able to submit this now is testament to Jim's compendious knowledge of the subject and his patience in imparting some of it to me.  
At all stages of my PhD, I have been able to rely on our weekly meetings for insight when I was stuck or unsure about the correct direction to investigate.  
In addition, Jim often encouraged me to submit papers to workshops and conferences, even in the early stages of my PhD, and the experience of doing so has shaped the direction of my research, and helped me form valuable relationships with other researchers.

\begin{abstract}
  Starting with a Fully Abstract denotational semantics for an Algol-like programming language, we investigate how we can use techniques of categorical algebra to adapt the semantics when the original language is extended with various nondeterministic effects.  
  Our running example for the base denotational semantics is Abramsky and McCusker's game semantics for Idealized Algol \cite{SamsonGuyIAActive}.  
  We give a full presentation of this game semantics, including an alternative proof of Computational Adequacy that uses Laird's concept of a sequoidal category \cite{laird02} rather than the combinatorial proof from Abramsky and McCusker's original paper.

  We introduce the familiar concepts of monads and Kleisli categories, and prove a Full Abstraction result that shows how the process of passing to certain Kleisli categories corresponds to particular language extensions.  
  We link these language extensions to may- and must-testing for finite and countable nondeterminism, showing how we can construct Fully Abstract models of these effects using Kleisli categories.

  We introduce a generalization of monads: lax actions, also known as parametric monads.
  We investigate various corresponding generalizations of Kleisli categories, and prove a Full Abstraction result for one such generalization, showing how it too can be used to construct a model of an extended version of our original Algol-like language.
  We show how a special case of this construction can be adapted to model a probabilistic language.
\end{abstract}

\documentclass[11pt]{report}

\usepackage[utf8]{inputenc}

\usepackage{graphicx} % support the \includegraphics command and options

\usepackage{parskip} % Activate to begin paragraphs with an empty line rather than an indent

%%% PACKAGES
\usepackage{booktabs} % for much better looking tables
\usepackage{array} % for better arrays (eg matrices) in maths
\ifdefined\BEAMER
\else
\usepackage{paralist} % very flexible & customisable lists (eg. enumerate/itemize, etc.)\prefix\t$.
\fi
\usepackage{verbatim} % adds environment for commenting out blocks of text & for better verbatim
\ifdefined\BEAMER
\else
\ifdefined\THESIS
\usepackage{subcaption}
\else
\usepackage{subfig} % make it possible to include more than one captioned figure/table in a single float
\fi
\fi
\usepackage{mathtools} % for the all important \coloneqq symbol
\usepackage{hyperref} % for hyperreferences
\usepackage{IEEEtrantools} % for \IEEEeqnarray
\usepackage{pbox} % for \pbox
\usepackage{multirow,bigdelim} % for \multirow
\usepackage{lettrine} % For the drop cap
\usepackage{mathpartir} % for \inferrule, \inferrule* and the mathpar environment
\usepackage{listings}

\usepackage{caption}
\captionsetup{singlelinecheck=off}

\ifdefined\NOTARTICLE
\else

%%% ToC (table of contents) APPEARANCE
\usepackage[nottoc,notlof,notlot]{tocbibind} % Put the bibliography in the ToC
\usepackage[titles,subfigure]{tocloft} % Alter the style of the Table of Contents
\renewcommand{\cftsecfont}{\rmfamily\mdseries\upshape}
\renewcommand{\cftsecpagefont}{\rmfamily\mdseries\upshape} % No bold!

\fi

%% Font things %%
\usepackage{amssymb}
\usepackage{cmll} % Linear logic symbols!
\ifdefined\FEWFONTS
\else
\usepackage{bm} % for bold Greek letters
\fi
\usepackage{stmaryrd}
\usepackage{bbm}

%% Get the sqsubsetneqq character from the mathabx package
\DeclareFontFamily{U}{mathb}{\hyphenchar\font45}
\DeclareFontShape{U}{mathb}{m}{n}{
      <5> <6> <7> <8> <9> <10> gen * mathb
      <10.95> mathb10 <12> <14.4> <17.28> <20.74> <24.88> mathb12
      }{}
\DeclareSymbolFont{mathb}{U}{mathb}{m}{n}

\DeclareMathSymbol{\sqsubsetneq}    {3}{mathb}{"88}
\DeclareMathSymbol{\varsqsubsetneq} {3}{mathb}{"8A}
\DeclareMathSymbol{\varsqsubsetneqq}{3}{mathb}{"92}
\DeclareMathSymbol{\sqsubsetneqq}   {3}{mathb}{"90}

%% Get the left and right moons from the wasysym package

\DeclareFontFamily{U}{wasy}{}
\DeclareFontShape{U}{wasy}{m}{n}{ <5> <6> <7> <8> <9> gen * wasy
      <10> <10.95> <12> <14.4> <17.28> <20.74> <24.88>wasy10  }{}
\DeclareFontShape{U}{wasy}{b}{n}{ <-10> sub * wasy/m/n
 <10> <10.95> <12> <14.4> <17.28> <20.74> <24.88>wasyb10 }{}
\DeclareFontShape{U}{wasy}{bx}{n}{ <-> sub * wasy/b/n}{}

\def\wasyfamily{\fontencoding{U}\fontfamily{wasy}\selectfont}
\def\leftmoon   {\mbox{\wasyfamily\char36}}
\def\rightmoon  {\mbox{\wasyfamily\char37}}

%% Lists %%
\usepackage{enumerate}

%% Graphics %%
\usepackage{tikz}
\usetikzlibrary{cd}
\usetikzlibrary{patterns}
\usetikzlibrary{calc}
\usetikzlibrary{decorations.pathmorphing}
\usetikzlibrary{positioning}

\tikzset{inlinearrows/.style={anchor=base,baseline,x=0.6\baselineskip,y=0.6\baselineskip}}

\ifdefined\BEAMER
\else

%% Theorems! %%
\usepackage{amsthm}
\theoremstyle{plain} % Theorems, lemmas, propositions etc.
\newtheorem{theorem}{Theorem}[section]
\newtheorem{lemma}[theorem]{Lemma}
\newtheorem{proposition}[theorem]{Proposition}
\newtheorem{corollary}[theorem]{Corollary}
\newtheorem{fact}[theorem]{Fact}
\newtheorem{construction}[theorem]{Construction}
\theoremstyle{definition} % Definitions etc.  
\newtheorem{definition}[theorem]{Definition}
\newtheorem{notation}[theorem]{Notation}
\theoremstyle{remark} % Remarks
\newtheorem{remark}[theorem]{Remark}
\newtheorem{remarks}[theorem]{Remarks}
\newtheorem{example}[theorem]{Example}
\newtheorem{question}[theorem]{Question}
\newtheorem{slogan}[theorem]{Slogan}

\newtheoremstyle{note} {3pt} {3pt} {\itshape} {} {\itshape} {:} {.5em} {} % For short notes
\theoremstyle{note}
\newtheorem{note}[theorem]{Note}

\fi

%% Exercises and answers %%
\usepackage{answers}

\newtheoremstyle{exercisestyle}% name
  {6pt}   % ABOVESPACE
  {6pt}   % BELOWSPACE
  {\itshape}  % BODYFONT
  {0pt}       % INDENT (empty value is the same as 0pt)
  {\bfseries} % HEADFONT
  {.}         % HEADPUNCT
  {3pt} % HEADSPACE
  {}          % CUSTOM-HEAD-SPEC

\theoremstyle{exercisestyle}
\newtheorem{exercise}{Exercise}
\newtheorem{answerthm}{Exercise}

\Newassociation{answer}{answerthm}{answers}
\newcommand{\answerthmparams}{}

%% Changes to enumerate things so they look better %%\sigma$

\makeatletter
\def\enumfix{%
\if@inlabel
 \noindent \par\nobreak\vskip-\topsep\hrule\@height\z@
\fi}

\let\olditemize\itemize
\def\itemize{\enumfix\olditemize}
\let\oldenumerate\enumerate
\def\enumerate{\enumfix\oldenumerate}

%% Random crap %%
\usepackage{xifthen}

\makeatletter
\def\thm@space@setup{%
  \thm@preskip=\parskip \thm@postskip=0pt
}
\makeatother

\makeatletter
\newcommand*{\relrelbarsep}{.386ex}
\newcommand*{\relrelbar}{%
  \mathrel{%
    \mathpalette\@relrelbar\relrelbarsep
  }%
}
\newcommand*{\@relrelbar}[2]{%
  \raise#2\hbox to 0pt{$\m@th#1\relbar$\hss}%
  \lower#2\hbox{$\m@th#1\relbar$}%
}
\providecommand*{\rightrightarrowsfill@}{%
  \arrowfill@\relrelbar\relrelbar\rightrightarrows
}
\providecommand*{\leftleftarrowsfill@}{%
  \arrowfill@\leftleftarrows\relrelbar\relrelbar
}
\providecommand*{\xrightrightarrows}[2][]{%
  \ext@arrow 0359\rightrightarrowsfill@{#1}{#2}%
}
\providecommand*{\xleftleftarrows}[2][]{%
  \ext@arrow 3095\leftleftarrowsfill@{#1}{#2}%
}
\makeatother

\newcommand{\catname}[1]{{\normalfont\textbf{#1}}}
\newcommand{\Rings}{\catname{CRing}}
\newcommand{\CAT}{\catname{CAT}}
%\newcommand{\Top}{\catname{Top}}
\newcommand{\Set}{\catname{Set}}
\newcommand{\Cat}{\catname{Cat}}
\newcommand{\MonCat}{\catname{MonCat}}
\newcommand{\SymmMonCat}{\catname{SymmMonCat}}
\newcommand{\Cont}{\catname{Cont}}
\newcommand{\Sch}{\catname{Sch}}
\newcommand{\Rel}{\catname{Rel}}
\newcommand{\Coh}{\catname{Coh}}
\newcommand{\Inj}{\catname{Inj}}
\newcommand{\Dcpo}{\catname{Dcpo}}
\newcommand{\Mod}[1][]{\ifthenelse{\isempty{#1}}{\catname{Mod}}{#1\catname{mod}}}
\DeclareMathOperator{\sh}{Sh}
\newcommand{\Sh}[1][]{\ifthenelse{\isempty{#1}}{\sh}{\sh(#1)}}
\newcommand{\map}[3]{#2\xrightarrow{#1} #3}
\newcommand*\from{\colon}
\newcommand*\bigto{\Rightarrow}
\newcommand{\cmap}[3]{#1\from{}#2\to{}#3}
\newcommand\oppcat[1]{#1^{\mathrm{op}}}
\newcommand{\object}{\colon}
\DeclareRobustCommand{\vmap}[3] {\begin{tikzcd} #2 \arrow[d, "#1"] \\ #3 \end{tikzcd}}
\newcommand{\partref}[1]{(\ref{#1})}
\newcommand{\intgrpd}[4] {#1 \xrightrightarrows[#3]{#4} #2}
\DeclareRobustCommand{\bigintgrpd}[4] {\begin{tikzcd}[ampersand replacement=\&] #1 \arrow[r, shift left=0.5ex, "#3"] \arrow[r, shift right=0.5ex, "#4"'] \& #2 \end{tikzcd}}

\usepackage{xspace}

\newcommand{\etale}{\'{e}tale\xspace}
\newcommand{\Etale}{\'{E}tale\xspace}

\def \inv {^{-1}}

\DeclareMathOperator{\id}{id}
\DeclareMathOperator{\op}{op}
\DeclareMathOperator{\pr}{pr}
\DeclareMathOperator{\inj}{in}
\DeclareMathOperator{\pre}{{pre}}
\DeclareMathOperator{\et}{{\acute{e}t}}

\DeclareMathOperator{\Hom}{Hom}
\DeclareMathOperator{\Spec}{Spec}

\DeclareMathOperator{\ol}{ol}

\def\presuper#1#2%
  {\mathop{}%
   \mathopen{\vphantom{#2}}^{#1}%
   \kern-\scriptspace%
   #2}
\def\presub#1#2%
  {\mathop{}%
   \mathopen{\vphantom{#2}}_{#1}%
   \kern-\scriptspace%
   #2}

\newsavebox{\overlongequation}
\newenvironment{longdiagram}
 {\begin{displaymath}\begin{lrbox}{\overlongequation}$\displaystyle}
 {$\end{lrbox}\makebox[0pt]{\usebox{\overlongequation}}\end{displaymath}}

%% Our things %%

\newcommand{\neggame}[1]{\presuper{\perp}{#1}}
\newcommand{\tensor}{\otimes}
\newcommand{\Tensor}{\bigotimes}
\newcommand{\sequoid}{\oslash}
\newcommand{\varsequoid}{\vartriangleleft}
\renewcommand{\implies}{\multimap}
\newcommand{\iimpl}{\Longrightarrow}
\newcommand{\comp}[2]{#1 \circ #2}
\newcommand{\icomp}[2]{\comp{#1}{#2}}
\newcommand{\cprd}{\sqcup}
\newcommand{\bigcprd}{\bigsqcup}
\newcommand{\G}{\mathcal G}
\newcommand{\W}{\mathcal W}
\newcommand{\suchthat}{\;\colon\;}
\newcommand{\varsuchthat}{\;\mid\;}
\newcommand{\esuchthat}{\;.\;}
\newcommand{\OP}{\{O,P\}}
\newcommand{\QA}{\{Q,A\}}
\renewcommand{\L}{\mathcal L}
\newcommand{\F}{\mathcal F}
\newcommand{\U}{\mathcal U}
\newcommand{\s}{\mathfrak s}
\renewcommand{\t}{\mathfrak t}
\renewcommand{\u}{\mathfrak u}
\renewcommand{\d}{\mathfrak d}
\newcommand{\e}{\mathfrak e}
\newcommand{\emptyplay}{\epsilon}
\newcommand{\bracketed}[1]{\left({#1}\right)}
\newcommand{\bneggame}[1]{{\bracketed{\neggame{#1}}}}
\newcommand{\prefix}{\sqsubseteq}
\newcommand{\ppprefix}{\sqsubset}
\newcommand{\pprefix}{\sqsubsetneqq}
\renewcommand{\ss}{\mathbf{s}}
\newcommand{\bN}{\mathbb{N}}
\newcommand{\bC}{\mathbb{C}}
\newcommand{\bB}{\mathbb{B}}
\newcommand{\bP}{\mathbb{P}}
\newcommand{\pfun}{\rightharpoonup}
\newcommand{\grel}[1]{\underline{#1}}
\DeclareMathOperator{\length}{length}
\renewcommand{\b}{\mathfrak b}
\renewcommand{\r}{\mathfrak r}
\newcommand{\bbeta}{{\bm{\beta}}}
\newcommand{\st}{{\Sigma^*}}
\let\sec\S
\renewcommand{\S}{{\mathfrak{S}}}
\DeclareMathOperator{\cc}{cc}
\DeclareMathOperator{\subs}{subs}
\DeclareMathOperator{\ret}{ret}
\DeclareMathOperator{\zz}{zz}
\newcommand{\aaa}{\mathbf{a}}
\newcommand{\bbb}{\mathbf{b}}
\newcommand{\ccc}{\mathbf{c}}
\newcommand{\ddd}{\mathbf{d}}
\newcommand{\B}{\mathcal B}
\newcommand{\BB}{\mathbf B}
\renewcommand{\H}{\mathcal H}
\DeclareMathOperator{\assoc}{assoc}
\DeclareMathOperator{\lunit}{lunit}
\DeclareMathOperator{\runit}{runit}
\DeclareMathOperator{\dom}{dom}
\DeclareMathOperator{\sym}{sym}
\newcommand{\braid}{\sym}
\newcommand{\blank}{\,\underline{\hspace{1.5ex}}\,}
\DeclareMathOperator{\cn}{cn}
\newcommand{\impliescn}{\protect\overset{\cn}{\implies}}
\newcommand{\C}{{\mathcal{C}}}
\newcommand{\D}{{\mathcal{D}}}
\newcommand{\E}{{\mathcal{E}}}
\newcommand{\V}{{\mathcal{V}}}
\newcommand{\EE}{{\mathbf{E}}}
\DeclareMathOperator{\ev}{ev}
\newcommand{\der}{{\mathtt{der}}}
\newcommand{\mult}{{\mathtt{mult}}}
\DeclareMathOperator{\wk}{wk}
\newcommand{\toisom}{{\xrightarrow{\cong}}}
\DeclareMathOperator{\passoc}{{\mathsf{passoc}}}
\DeclareMathOperator{\pcomm}{{\mathsf{pcomm}}}
\DeclareMathOperator{\run}{{\mathsf{r}}}
\DeclareMathOperator{\lun}{{\mathsf{l}}}
\newcommand{\fcoal}[1]{{\leftmoon #1 \rightmoon}}
\DeclareMathSymbol{\co}{\mathord}{operators}{"3C}
\DeclareMathSymbol{\nw}{\mathord}{operators}{"3E}
\newcommand{\T}{\mathfrak{T}}
\renewcommand{\subset}{\subseteq}
\newcommand{\Ord}{\catname{Ord}}
\newcommand{\FS}{\mathcal{FS}}
\DeclareMathOperator{\rank}{rank}
\DeclareMathOperator{\dist}{{\mathsf{dist}}}
\DeclareMathOperator{\dec}{{\mathsf{dec}}}
\DeclareMathOperator{\str}{str}
\DeclareMathOperator{\weak}{weak}
\DeclareMathOperator{\Strat}{Strat}
\DeclareMathOperator{\OppStrat}{OppStrat}
\newcommand{\seqs}[1]{{\overline{{#1}^{*}}}}
\def\flushRight{\leftskip0pt plus 1fill\rightskip0pt}
\def\Centering{\relax\ifvmode\centering\fi}
\newcommand{\deno}[1]{\left\llbracket#1\right\rrbracket}
\newcommand{\converges}{\Downarrow}
\newcommand{\diverges}{\Uparrow}
\newcommand{\mustconverge}{\converges^{\text{must}}}
\newcommand{\Iflt}{\mathtt{If{<}\;}}
\newcommand{\Ifgt}{\mathtt{If{>}\;}}
\newcommand{\inr}{{\mathsf{inr}}}
\newcommand{\inl}{{\mathsf{inl}}}
\newcommand{{\Na}}{\bN}
\newcommand{{\cell}}{{\mathsf{cell}}}
\newcommand{\fix}{{\mathsf{fix}}}
\newcommand{\eq}{{\mathsf{eq}}}
\DeclareMathOperator{\CCom}{CCom}
\newcommand{\power}{\mathfrak P}

% Slanty things
\newcommand*{\xslant}[2][76]{%
  \begingroup
    \sbox0{#2}%
    \pgfmathsetlengthmacro\wdslant{\the\wd0 + cos(#1)*\the\wd0}%
    \leavevmode
    \hbox to \wdslant{\hss
      \tikz[
        baseline=(X.base),
        inner sep=0pt,
        transform canvas={xslant=cos(#1)},
      ] \node (X) {\usebox0};%
      \hss
      \vrule width 0pt height\ht0 depth\dp0 %
    }%
  \endgroup
}

\makeatletter
\newcommand*{\xslantmath}{}
\def\xslantmath#1#{%
  \@xslantmath{#1}%
}
\newcommand*{\@xslantmath}[2]{%
  % #1: optional argument for \xslant including brackets
  % #2: math symbol
  \ensuremath{%
    \mathpalette{\@@xslantmath{#1}}{#2}%
  }%
}
\newcommand*{\@@xslantmath}[3]{%
  % #1: optional argument for \xslant including brackets
  % #2: math style
  % #3: math symbol
  \xslant#1{$#2#3\m@th$}%
}
\makeatother

\newcommand{\seqdeno}[1]{\xslantmath{\llbracket}#1\xslantmath{\rrbracket}}

% Empty set etc.

\let\oldemptyset\emptyset
\let\emptyset\varnothing

%% Constant width xrightarrows
\newlength{\arrow}
\settowidth{\arrow}{\scriptsize$1000$}
\newcommand*{\constantwidthxrightarrow}[1]{\xrightarrow{\mathmakebox[\arrow]{#1}}}

%% Landscape pages
\usepackage{everypage}
\usepackage{environ}
\usepackage{pdflscape}
\newcounter{abspage}

\ifdefined\NOTARTICLE

\else

\makeatletter
\newcommand{\newSFPage}[1]% #1 = \theabspage
  {\global\expandafter\let\csname SFPage@#1\endcsname\null}

\NewEnviron{SidewaysFigure}{\begin{figure}[p]
\protected@write\@auxout{\let\theabspage=\relax}% delays expansion until shipout
  {\string\newSFPage{\theabspage}}%
\ifdim\textwidth=\textheight
  \rotatebox{90}{\parbox[c][\textwidth][c]{\linewidth}{\BODY}}%
\else
  \rotatebox{90}{\parbox[c][\textwidth][c]{\textheight}{\BODY}}%
\fi
\end{figure}}

\AddEverypageHook{% check if sideways figure on this page
  \ifdim\textwidth=\textheight
    \stepcounter{abspage}% already in landscape
  \else
    \@ifundefined{SFPage@\theabspage}{}{\global\pdfpageattr{/Rotate 0}}%
    \stepcounter{abspage}%
    \@ifundefined{SFPage@\theabspage}{}{\global\pdfpageattr{/Rotate 90}}%
  \fi}
\makeatother

\fi

%% PCF Things

\newcommand{\nat}{{\mathtt{nat}}}
\newcommand{\bool}{{\mathtt{bool}}}

\newcommand{\Y}{\mathbf{Y}}
\newcommand{\opto}{\longrightarrow}
\newcommand{\oopto}{\dashrightarrow}
\newcommand{\n}{{\mathtt{n}}}
\DeclareMathOperator{\IfO}{{\mathsf{If0}}}
\DeclareMathOperator{\suc}{{\mathsf{succ}}}
\DeclareMathOperator{\pred}{{\mathsf{pred}}}
\newcommand{\0}{{\mathtt{0}}}

\newcommand{\iter}{{\mathtt{iter}}}
\newcommand{\rec}{\iter}
\newcommand{\Var}{{\mathtt{Var}}}
\DeclareMathOperator{\Varr}{Var}
\newcommand{\new}{{\mathtt{new}}}
\newcommand{\case}{{\mathtt{case}}}

\newcommand{\lmam}{\mathrel{\sqsubseteq_{m\&m}}}
\newcommand{\emam}{\mathrel{\equiv_{m\&m}}}
\newcommand{\lst}{\mathrel{\lesssim}}
\newcommand{\smam}{\mathrel{\sim_{m\&m}}}
\newcommand{\amam}{\mathrel{\approx_{m\&m}}}

\newcommand{\oes}{\sim}

%% Idealized Algol things

\newcommand{\com}{{\mathtt{com}}}
\newcommand{\skipp}{{\mathsf{skip}}}
\DeclareMathOperator{\seq}{{\mathsf{seq}}}
\DeclareMathOperator{\neww}{{\mathsf{new}}}
\DeclareMathOperator{\mkvar}{{\mathsf{mkvar}}}
\newcommand{\deref}{\texttt{@}}
\DeclareMathOperator{\dereff}{\mathsf{deref}}
\DeclareMathOperator{\assign}{\mathsf{assign}}
\newcommand{\ia}[2]{\langle #1 , #2 \rangle}
\newcommand{\stup}[3]{\langle #1 \mid #2 \mapsto #3 \rangle}

%% Hyland-Ong games things

\newbox\gnBoxA
\newdimen\gnCornerHgt
\setbox\gnBoxA=\hbox{$\ulcorner$}
\global\gnCornerHgt=\ht\gnBoxA
\newdimen\gnArgHgt
\def\pv #1{%
    \setbox\gnBoxA=\hbox{$#1$}%
    \gnArgHgt=\ht\gnBoxA%
    \ifnum     \gnArgHgt<\gnCornerHgt \gnArgHgt=0pt%
    \else \advance \gnArgHgt by -\gnCornerHgt%
    \fi \raise\gnArgHgt\hbox{$\ulcorner$} \box\gnBoxA %
    \raise\gnArgHgt\hbox{$\urcorner$}}
\def\ov #1{%
    \setbox\gnBoxA=\hbox{$#1$}%
    \gnArgHgt=\ht\gnBoxA%
    \ifnum     \gnArgHgt<\gnCornerHgt \gnArgHgt=0pt%
    \else \advance \gnArgHgt by -\gnCornerHgt%
    \fi \raise\gnArgHgt\hbox{$\llcorner$} \box\gnBoxA %
    \raise\gnArgHgt\hbox{$\lrcorner$}}
\newcommand{\ct}[1]{\lceil#1\rceil}
\DeclareMathOperator{\Int}{int}

%% Nondeterministic Factorization things

\newcommand{\code}{\mathsf{code}}
\newcommand{\Det}{\mathsf{Det}}

%% Flexible strategy things

\newcommand{\stle}{{\;\le_s\;}}
\newcommand{\steq}{{\;=_s\;}}
\newcommand{\exle}{\sqsubseteq}
\newcommand{\exlub}{\bigsqcup}
\newcommand{\dv}{{\text{\lightning}}}
\DeclareMathOperator{\pocl}{pocl}
\newcommand{\plot}{\mathrel{\triangleleft}}
\newcommand{\shad}{\mathfrak{S}}
%\newcommand{\tree}{\mathfrak{T}}
\newcommand{\Tau}{T}
\newcommand{\Epsilon}{E}
\newcommand{\sw}{\triangleleft}

%% Roman numerals

\newcommand{\RN}[1]{%
  \textup{\uppercase\expandafter{\romannumeral#1}}%
}
\newcommand{\RNl}[1]{%
  \mathrel{\raisebox{1pt}{$\overline{\underline{#1}}$}}
}

%% Game language things

\newcommand{\ul}[1]{{\underline{#1}}}
\newcommand{\A}{{\mathcal{A}}}
\renewcommand{\P}{\mathcal P}
\newcommand{\M}{\mathcal M}
\newcommand{\N}{\mathcal N}
\newcommand{\X}{\mathcal X}
\newcommand{\YY}{\mathcal Y}
\newcommand{\hole}{\blank}
\newcommand{\Tct}{\xrightarrow{T}t}
\newcommand{\teamconverge}[2]{\xrightarrow{#1}#2}

%% Inference rule things
\newcommand{\rulename}[1]{\LeftTirNameStyle{#1}}
\newcommand{\ts}{\mathbin{\vdash}}
\newcommand{\nts}{\mathbin{\not\vdash}}

%% Double category things
\newcommand{\hc}[2]{\left({#1}\middle|{#2}\right)}
\newcommand{\vc}[2]{\left(\frac{#1}{#2}\right)}

%% What is going on?
\DeclareMathOperator{\Kl}{Kl}
\DeclareMathOperator{\Mell}{Mell}
\newcommand{\powerset}{\mathcal P}
\DeclareMathOperator{\ask}{{\mathsf{ask}}}
\newcommand{\sleep}{{\mathsf{sleep}}}
\newcommand{\true}{\mathbbm{t}}
\newcommand{\false}{\mathbbm{f}}
\DeclareMathOperator{\If}{\mathsf{If}}
\newcommand{\Then}{\mathrel{\mathsf{then}}}
\newcommand{\Else}{\mathrel{\mathsf{else}}}
\newcommand\cat{\mathbin{+\mkern-10mu+}}

%% Profunctor arrows

\makeatletter
\def\slashedarrowfill@#1#2#3#4#5{%
  $\m@th\thickmuskip0mu\medmuskip\thickmuskip\thinmuskip\thickmuskip
   \relax#5#1\mkern-7mu%
   \cleaders\hbox{$#5\mkern-2mu#2\mkern-2mu$}\hfill
   \mathclap{#3}\mathclap{#2}%
   \cleaders\hbox{$#5\mkern-2mu#2\mkern-2mu$}\hfill
   \mkern-7mu#4$%
}
\def\rightslashedarrowfill@{%
  \slashedarrowfill@\relbar\relbar\mapstochar\rightarrow}
\newcommand\xslashedrightarrow[2][]{%
  \ext@arrow 0055{\rightslashedarrowfill@}{#1}{#2}}
\makeatother
\newcommand{\pto}{{\xslashedrightarrow{} }}

%% Profunctors 
\DeclareMathOperator{\Prof}{Prof}
\DeclareMathOperator{\End}{End}
\DeclareMathOperator{\Endoprof}{Endoprof}

%% Our

\def\searchmacro#1{
  \AtBeginOfFiles{%
    \ifdefined#1
      \expandafter\def\csname \currfilename:found\endcsname{}%
    \fi}
  \AtEndOfFiles{%
    \ifdefined#1
      \unless\ifcsname \currfilename:found\endcsname
        \immediate\write\finder{found in '\currfilename'}%
    \fi\fi}}

%% Isomorphism arrows on commutative diagrams %%
\tikzset{Isom/.style={every to/.append style={edge node={node [sloped, above, allow upside down, auto=false]{$\cong$}}}},
         Isom'/.style={every to/.append style={edge node={node [sloped, above, allow upside down, auto=false, rotate=180]{$\cong$}}}},
         Sim/.style={every to/.append style={edge node={node [sloped, above, allow upside down, auto=false]{$\sim$}}}},
         Sim'/.style={every to/.append style={edge node={node [sloped, above, allow upside down, auto=false, rotate=180]{$\sim$}}}}}

%% Adjunctions
\newcommand{\adjunction}[4]{%
  {#1} \underset{\underset{#3}{\longleftarrow}}{\overset{\overset{#2}{\longrightarrow}}{\bot}} {#4}}        

%% Important!
\newcommand\Mellies{Melli\`{e}s\xspace}

\makeatletter
\newcommand{\colim@}[2]{%
  \vtop{\m@th\ialign{##\cr
    \hfil$#1\operator@font colim$\hfil\cr
    \noalign{\nointerlineskip\kern1.5\ex@}#2\cr
    \noalign{\nointerlineskip\kern-\ex@}\cr}}%
}
\newcommand{\colim}{%
  \mathop{\mathpalette\colim@{\rightarrowfill@\textstyle}}\nmlimits@
}
\makeatother

\makeatletter
\newcommand{\laxcolim@}[2]{%
  \vtop{\m@th\ialign{##\cr
    \hfil$#1\operator@font colim_l$\hfil\cr
    \noalign{\nointerlineskip\kern1.5\ex@}#2\cr
    \noalign{\nointerlineskip\kern-\ex@}\cr}}%
}
\newcommand{\laxcolim}{%
  \mathop{\mathpalette\laxcolim@{\rightarrowfill@\textstyle}}\nmlimits@
}
\makeatother

\DeclareMathOperator{\Colim}{colim}

\DeclareMathOperator{\DG}{DG}
\DeclareMathOperator{\RV}{RV}
\newcommand{\Rv}{\catname{Rv}}

\let\choose\undefined
\DeclareMathOperator{\choose}{\mathsf{choose}}
\DeclareMathOperator{\tr}{tr}
\DeclareMathOperator{\test}{test}

%% Slot game things %%
\newcommand{\circled}[1]{\raisebox{.5pt}{\textcircled{\raisebox{-.9pt} {#1}}}}
\newcommand{\slot}{{\circled{\$}}}

\DeclareMathOperator{\may}{may}
\DeclareMathOperator{\must}{must}

\newcommand{\encode}[1]{\lceil#1\rceil}
\DeclareMathOperator{\app}{\mathsf{app}}
\DeclareMathOperator{\lett}{\mathsf{let}}
\newcommand{\inn}{\mathrel{\mathsf{in}}}
\DeclareMathOperator{\byval}{\mathsf{byval}}

\DeclareMathOperator{\rread}{read}
\DeclareMathOperator{\wwrite}{write}

\DeclareSymbolFont{bbsymbol}{U}{bbold}{m}{n}
\DeclareMathSymbol{\bbsemicolon}{\mathbin}{bbsymbol}{"3B}
\newcommand{\semicom}{\bbsemicolon}

\newcommand{\ms}{\makebox[-1pt]{}}

\DeclareMathOperator{\Acc}{Acc}
\DeclareMathOperator{\im}{Im}
\DeclareMathOperator{\wit}{wit}

%%% END Article customizations



\begin{document}

\chapter{Introduction}

\section{Denotational Semantics and Program Equivalence}

Given two pieces of computer code, in what circumstances can we say that they are interchangeable?
Clearly, the two pieces of code should return the same output for any choice of input values. 
But -- depending on the expressive power of the language -- this might not be enough.

For example, the following two Haskell functions appear to do the same thing.

\begin{lstlisting}[language=haskell]
f :: Int -> Int
f n = if (n == 0) then 0 else 0

g :: Int -> Int
g n = 0
\end{lstlisting}

However, if we introduce a non-terminating function
\begin{lstlisting}[language=haskell]
diverge :: Int -> Int
diverge x = diverge x
\end{lstlisting}
then it becomes clear that \lstinline[language=haskell]{f} and \lstinline[language=haskell]{g} are not interchangeable: since \lstinline[language=haskell]{f} always evaluates its argument \lstinline[language=haskell]{n}, \lstinline[language=haskell]{f (diverge 0)} will fail to terminate, whereas \lstinline[language=haskell]{g (diverge 0)} will give us \lstinline[language=haskell]{0}, since inputs to functions are evaluated lazily in Haskell.

We can have another go at answering our original question, then, by adding the requirement that the two programs should behave in the same way if they are passed non-terminating inputs.
Thus, we add to each datatype an extra distinguished value $\bot$ representing non-termination (so, for example, the type of integers is represented by the set $\bZ_\bot = \bZ + \{\bot\}$), and then function types are interpreted as functions between these sets -- so a term of type \lstinline[language=haskell]{Int -> Int} is reprepresented by a function $\bZ_\bot \to \bZ_\bot$.

We need to be careful, though, since not every such function arises from a program in this way.
For example, we cannot write a program corresponding to the function $\chi\from\bZ_\bot \to \bZ_\bot$ that sends $\bot$ to $0$ and all other values to $1$since $\chi$ is not constant, such a program would have to evaluate its argument, and would consequently fail to terminate if that argument did not terminate.

If our language admits higher types, then it becomes especially important to exclude such `impossible' functions from our model.
For example, if $F$ and $G$ are two programs of type \lstinline[language=haskell]{(Int -> Int) -> Int} -- i.e., functions that take in a function from integers to integers and return an integer -- then we do not want to declare that $F$ and $G$ are different on the basis that $F(\chi)\ne G(\chi)$.  

In order to rule out these `impossible' functions, we define a partial order on the sets $T_\bot$ corresponding to types, defined by setting $x\le x$ and $\bot\le x$ for all $x$.
We then require that functions should be monotonic and continuous with respect to this order. 
For example, a monotonic function $\bZ_\bot \to \bZ_\bot$ is either constant (correponding to a function that does not evaluate its argument at all) or sends $\bot$ to $\bot$ and is otherwise unconstrained (corresponding to a function that evaluates its argument).
It turns out that if we order functions pointwise, then we get the correct constraints at higher types as well.

Even if we get round the problems with divergence, there are other language features that we may need to consider if we want to determine whether two pieces of code are equivalent. 
If our functions have access to global variables, then we need to check that these variables end up taking the same final value, whatever their initial values were.
If we have IO calls in our language, then we need to check that the functions print out the same text, whatever the user input was.
If we have a random number generator, then we need to check that our functions return the same \emph{set} of values, whatever the input.

What we have been doing in all these examples is \emph{denotational semantics}: the art of using mathematical objects to study logic and programming languages.  
In the first case, our denotational semantics was expressed through the mathematics of sets and functions, whereby we captured the behaviour of a (programming language) function via a (mathematical) function.

Then, following Scott \cite{ScottDomains}, we refind this model to one based on partially-ordered sets -- more specifically, \emph{Scott domains} -- in which we modelled the behaviour or a program via an associated Scott-continuous functions between domains.

In our other examples, we need to come up with further refinements to our model in order to incorporate the new computational effects. 
For example, to handle nondeterminism, we might want to switch to using nondeterministic functions, or \emph{relations}, instead of ordinary functions.

The advantages in all of these cases is that the mathematical objects we use are often fairly simple, whereas computer programs, even in simple `toy' languages, are very complicated to study. 
A program is, at its heart, a string of symbols governed by a collection of operational rules that govern how such strings should behave. 
Such an object is very fiddly to reason with directly; indeed, the only way to think about it is as some kind of `function' from input to outputs.
Denotational takes this basic intuition further, and aims to capture features of programming languages through a diverse collection of different mathematical models.

A word of warning: the principal mode of denotational semantics which we shall be studying in this thesis is game semantics, which is much more complicated than the semantics of sets and functions. 
However, it is still a very valuable tool for determining equivalence of programs.

\section{Computational Adequacy and Full Abstraction}

In order for a denotational semantics to tell us anything, we first need to prove some results that relate it to the language we are studying. 
For example, if we are hoping to model a programming language with sets and functions, then we need to define a mapping $\deno{-}$ (the \emph{denotation}) that takes program types to sets and program functions to functions between those sets, and we also need to prove that this denotation respects the operational rules of the language: for example, we might want to prove that if $f \from \Int \to \Int$ is a function and $M \from \Int$ is a term that evaluates to the integer $n$, then the term $f M$ will evaluate to the integer $\deno{f}(n)$.
This type of result is called \emph{Computational Adequacy}, and relates to programs of a ground or observable type (e.g., a program that returns an integer has ground type, and we can observe that it either returns an integer or fails to teminate, whereas a program that takes in an integer and returns an integer has a function type: it is not possible to `observe' its behaviour without substituting in values).

Briefly speaking, a Computational Adequacy result tells us that the behaviour of a program of ground type may be deduced exactly from its denotation.
For example, in a domain-theoretic semantics, we might want to say that a program $M$ evaluates to a value $v$ if and only if $\deno M=v$ and that $M$ fails to terminate if and only if $\deno M = \bot$.

Such a computational adequacy result extends readily to terms not of ground type.  
Given two programs $M$ and $N$ of the same type, we say that $M$ and $N$ are \emph{observationally equivalent} if $C[M]$ and $C[N]$ have the same behaviour for any one-holed context $C[-]$ of ground type.
If our semantics is \emph{compositional} -- so that the denotation of $C[M]$ is obtained by `applying' the denotation of $C[-]$ to the denotation of $M$ -- and computationally adequate, then it follows that the semantics is \emph{equationally sound}: if two terms $M$ and $N$ have the same denotation, then they are observationally equivalent.

If we have an effective way of computing denotations, then this can give us an easy way to prove observational equivalence of terms.  
However, there is no guarantee that we are able to use such a trick: the terms $M$ and $N$ might be observationally equivalent despite having distinct denotations.  
The gold standard of denotational semantics -- \emph{Full Abstraction} -- asserts in addition that the converse of equational soundness holds, so that the denotational semantics completely captures the observational equivalence relation.  

An important early success in this direction came with Plotkin's introduction of the stateless sequential programming language PCF \cite{PlotkinPcf}.  
Plotkin was unable to provide a fully abstract denotational semantics for PCF itself, but he showed that if we add a simple parallel construct\footnote{Specifically, `parallel or', which evaluates its two boolean arguments in parallel, and is thus able to return true if either the left or the right argument returns true, even if the other fails to terminate.} to PCF, then a denotational semantics based on Scott domains is fully abstract.  
This astounding result presents us with a world in which we can practically and systematically check observational equivalence (for terms of this extended version of PCF) by computing denotations.  
This vision is a little rosey-eyed -- deciding observational equivalence is stronger than the halting problem and is hence undecidable in general -- but if the programs in question are finitely presentable in some sense, then we really can use the denotational semantics to check observational equivalence.

Unfortunately, this stop working for PCF itself.  
PCF is a sub-language of the parallelized version, but this also means that the observational equivalence relation is coarser: there may be terms that can be distinguished by a context including the parallel construct that cannot be distinguished inside any purely sequential context.  
The enterprise was brought down to earth by Ralph Loader's 2001 theorem that observational equivalence for PCF is undecidable, even if we restrict ourselves to a finitary version of the language with no infinite datatypes or recursion beyond a simple non-termination primitive $\bot$.
This in particular tells us that no concretely presentable denotational semantics for PCF can possibly be fully abstract, or it would give us an algorithm for deciding observational equivalence in this finite version.

Nevertheless there were, roughly contemporaneous with Loader's result, several fully abstract models of PCF published, in a watershed moment for the subject.
The models published by Nickau \cite{NickauPcf} and O'Hearn and Riecke \cite{OHearnRieckePcf} were more or less along domain-theoretic lines, while those of Abramsky, Jagadeesan and Malacaria \cite{ajmPcf} and Hyland and Ong \cite{hoPcf} used the relatively new Game Semantics.

These models took a slightly oblique approach to Full Abstraction.  
First, they defined the notion of \emph{intrinsic equivalence} of terms of the same type $T$ in a denotational model, where two elements $\sigma$ and $\tau$ of the denotation of $T$ are intrinsically equivalent if $\alpha(\sigma)=\alpha(\tau)$ for all functions $\alpha\from\deno{T}\to\deno{o}$ from the denotation of $T$ to the denotation of some fixed ground type $o$.
This definition is very closely linked to that of observational equivalence; indeed, if two terms $M$ and $N$ are observationally equivalent, and we are working in a computationally adequate and compositional denotational semantics, then the denotations of $M$ and $N$ will be intrinsically equivalent, since for any ground-type context $C[-]$, we can take $\alpha$ to be the denotation of $C[-]$ in the above definition.

Proving the converse -- that intrinsic equivalence of denotations implies observational equivalence -- entails going in the opposite direction; i.e., starting with some element $\alpha$ in the model and coming up with a context $C[-]$ in the language whose denotation is $\alpha$.  
Thus, proving this direction normally reduces to some kind of \emph{definability} result.  
Typically, we only need to prove definability for a restricted class of elements $\alpha$ of the denotational model -- the \emph{compact} elements.

If we can prove, for some denotational model of a language, that observational equivalence of terms is equivalent to intrinsic equivalence of their denotations, then we can form a fully abstract model by passing to equivalence classes under the intrinsic equivalence relation.  
This is the approach taken by the fully abstract semantics that have been given for PCF; there is no contradiction of Loader's theorem, because the intrinsic equivalence relation is itself undecidable.

In this thesis, we shall skip the final step of passing to equivalence classes and declare a denotational semantics to be fully abstract for a language if we can prove that observational equivalence of terms is equivalent to intrinsic equivalence of their denotations.  
Thus, the Full Abstraction results that we prove will have three main ingredients: compositionality, computational adequacy and definability.

\section{Categorical Semantics}

There is a close link between (typed) programming langauges and categories.
Programming languages have things called \emph{types}, and they have \emph{functions} that go from one type to another. 
Typically, it will be possible to compose two functions together in the language in an associative way, giving us a category.
It should come as no surprise, then, that a very important branch of denotational semantics is \emph{categorical semantics}, in which we take some existing category from mathematics, and use its objects and morphisms to represent the types and terms of a programming language.

Typically, each type $T$ of the language will correspond to some object $\deno T$ of the category, while a term of type $T$ will correspond to a morphism $1 \to \deno T$, where $1$ is some fixed object in the category (usually a terminal object).

Particularly important \cite{Lambek} are the Cartesian closed categories, which have a number of properties making them suitable for denotational semantics:
\begin{description}
  \item[Product and function spaces] Given types $S$ and $T$, we can define the denotations of the product type $S \times T$ and the function type $S \to T$ to be given by $\deno S \times \deno T$ and ${\deno T}^{\deno S}$.
  \item[Compositionality] Given types $S$ and $T$, and corresponding objects $\deno S$ and $\deno T$ of the category, we can define the denotation of the function type $S \to T$ to be given by the exponentiation ${\deno T}^{\deno S}$ as above.
    Then we automatically have a recipe for substituting a term of type $S$ into a function of type $S\to T$ via the canonical morphism
    \[
      {\deno T}^{\deno S} \times \deno S \to \deno T\,.
      \]
  \item[Abstraction] Given a morphism $\sigma \from A \times B \to C$, we may form a morphism $\Lambda(\sigma) \from A \to C^B$.
    This gives us the semantics for $\lambda$-abstraction, whereby we pass from a term-in-context
    \[
      \Gamma,x\from S \ts M \from T
      \]
    to the term-in-context
    \[
      \Gamma \ts \lambda x.M\from S \to T\,.
      \]
\end{description}
These rules allow us to build up a model of the simply-typed $\lambda$-calculus within any Cartesian closed category, which means we get a large part of the denotation (and the subsequent proof of Computational Adequacy) for free.  

This alone would be a good justification for using category theory in denotational semantics, but the benefits go further.  
The development of programming languages such as Haskell has been strongly influenced by category-theoretic concepts.  
For example, Moggi's 19xx observation \cite{Moggi} that monads on categories provide us with a way of modelling computational effects influenced work that led directly to the introduction of support for monads in Haskell \cite{Haskell}, where they have become the primary tool for abstracting out effectful computation.

Monads will be particularly important in this thesis, so it is worth dwelling on them a little further.  
A \emph{monad} on a category $\C$ is given by a functor $M\from \C \to \C$, together with natural transformations
\begin{mathpar}
  e\from\id_\C \Rightarrow M \and m\from M\circ M\Rightarrow M
\end{mathpar}
that endow $M$ with an algebraic structure.
One example is the non-empty powerset functor on the category of sets, together with the natural transformations given by
\begin{mathpar}
  \begin{IEEEeqnarraybox*}{rCcCl}
    e & \from & A & \to & \powerset(A) \\
    && a & \mapsto & \{a\}
  \end{IEEEeqnarraybox*}
  \and
  \begin{IEEEeqnarraybox*}{rCcCl}
    m & \from & \powerset(\powerset(A)) & \to & \powerset(A) \\
    && \mathcal A & \mapsto & \bigcup_{A \in \mathcal A} A\,.
  \end{IEEEeqnarraybox*}
\end{mathpar}
This powerset monad indicates some kind of nondeterministic choice between elements of $A$, particularly if we modify the construction to the \emph{non-empty powerset} functor $\powerset_+$.

Another example in the category of sets is the functor $A \mapsto A + \{\bot\}$, that appends an additional element on to a set.  
We have natural functions $A \to A + \{\bot\}$ and $A + \{\bot\} + \{\bot\} \to A + \{\bot\}$ that make this into a monad as well.
In the study of programming languages, this is often called the \emph{maybe monad}, because $A + \{\bot\}$ indicates an element of $A$ that may or may not be present (with the distinguished value $\bot$ indicating no value).

Given a monad $M$ on a category $\C$, we can form a new category $\Kl_M\C$ -- the \emph{Kleisli category} of $M$ -- whose objects are the objects of $\C$ and where a morphism from $A$ to $B$ is given by a morphism $A \to MB$ in $\C$.  
The monadic coherence gives us the correct notion of composition: given Kleisli morphisms $\sigma \from A \to MB$ and $\tau\from B \to MC$, we may compose them to give a morphism $A \to MC$ via the following formula.
\[
  A \xrightarrow{\sigma} MB \xrightarrow{M\tau} M M C \xrightarrow{m} MC
  \]
The Kleisli category of the powerset monad is the category of sets and relations, while the Kleisli category of the maybe monad is the category of sets and partial functions.

There are numerous other monads that can be used to model computational effects, such as the state monad and the exception monad.  
Work by Plotkin and Power \cite{PlotkinPower} makes this more precise, by studying monads that can be built up via algebraic operations and equations.  
For example, we might want to model nondeterministic choice on a set $A$ via an operation $\sum$ that takes in infinitely many elements of $A$ -- so $\sum a_i$ gives us a choice between the $a_i$.  
We then impose some axioms on this operation.
\begin{description}
  \item[Idempotence] If $a_i=a$ for all $i$, then $\sum a_i=a$;
  \item[Commutativity] $\sum a_i = \sum_{a_{\pi(i)}}$ for any permutation $\pi$; and
  \item[Associativity] $\sum_i (\sum_j a_{ij}) = \sum_{i,j} a_{ij}$.
\end{description}
This is an algebraic theory akin to the theory of groups, and its category of free algebras is isomorphic to the Kleisli category of the powerset monad.

\section{Game Semantics}

Game Semantics gives us a particularly fruitful categorical semantics for programming languages.  
The underlying idea is that a computer program behaves like a strategy for a two-player game, in that it needs to respond to arbitrary inputs (opponent moves) with its own behaviours (proponent moves).  
Thus, we represent a programming language type by an idealized two-player game and represent a term of that type by a strategy for that game.

The precise definition of a strategy that we use depends on the language that we are trying to model -- a language with less expressive power can realize fewer strategies.  
For example, the denotations of terms in a stateless langauge such as PCF are \emph{history-free} or \emph{innocent}, in which the proponent's moves can only depend on a particular subsequence of the current sequence of moves -- the $P$-view -- rather than on the whole sequence.
So for a lot of computational effects, particularly ones that have something to do with state, adding that effect corresponds to a \emph{relaxing} of conditions on strategies.

We model other types of effects by extending the definition of a strategy.  
For example, if we want to provide a semantics for a language with nondeterminism, then we modify the definition of a strategy so that the propoonent can have multiple replies to each opponent move, as in the work of Harmer and McCusker \cite{mcCHFiniteND}.
If we want to model a probabilistic language, then we decorate these different moves with probabilities, as in the work of Danos and Harmer \cite{DanosHarmer}.

When choosing a definition of a strategy, the aim is to prove a definability result, so that we can prove Full Abstraction.  
The original proofs of definability of compact innocent strategies in PCF from \cite{ajmPcf} and \cite{hoPcf} were intricate and technical.  
Subsequent work on languages that extend PCF tends to try to prove definability via a \emph{factorization result}, in which we show that every strategy in an extended category of games may be written as the composition of a strategy in an original category of games with some fixed strategy in the new category.  
Then, if we have a definability result for the original semantics, we can extend it to a definability result in the new category.

For example, the language Idealized Algol is an extended version of PCF that adds some stateful primitives.
For example, Abramsky and McCusker's proof of compact definability for Idealized Algol in \cite{SamsonGuyIAActive} first proves that each compact strategy in their model may be written as the composite of a compact innocent strategy with the (non-innocent) denotation of one of the new stateful constants.  
Thus, they can deduce compact definability for Idealized Algol from Hyland and Ong's result that every compact innocent strategy is the denotation of a term of PCF.  

Harmer and McCusker develop in \cite{mcCHFiniteND} a model of game semantics in which strategies can be nondeterministic.  
They show that every such strategy can be written as the composite of a deterministic strategy with some particular fixed nondeterministic strategy.  
Then, to prove compact definability for nondeterministic Idealized Algol, it suffices for them to exhibit a nondeterministic term whose denotation is that fixed strategy.

\bibliographystyle{alpha2}
\bibliography{../common/phd_bibliography}

\end{document}

\documentclass[11pt]{report}

\def\FEWFONTS{1}
\def\THESIS{1}
\usepackage[utf8]{inputenc}

\usepackage{graphicx} % support the \includegraphics command and options

\usepackage{parskip} % Activate to begin paragraphs with an empty line rather than an indent

%%% PACKAGES
\usepackage{booktabs} % for much better looking tables
\usepackage{array} % for better arrays (eg matrices) in maths
\ifdefined\BEAMER
\else
\usepackage{paralist} % very flexible & customisable lists (eg. enumerate/itemize, etc.)\prefix\t$.
\fi
\usepackage{verbatim} % adds environment for commenting out blocks of text & for better verbatim
\ifdefined\BEAMER
\else
\ifdefined\THESIS
\usepackage{subcaption}
\else
\usepackage{subfig} % make it possible to include more than one captioned figure/table in a single float
\fi
\fi
\usepackage{mathtools} % for the all important \coloneqq symbol
\usepackage{hyperref} % for hyperreferences
\usepackage{IEEEtrantools} % for \IEEEeqnarray
\usepackage{pbox} % for \pbox
\usepackage{multirow,bigdelim} % for \multirow
\usepackage{lettrine} % For the drop cap
\usepackage{mathpartir} % for \inferrule, \inferrule* and the mathpar environment
\usepackage{listings}

\usepackage{caption}
\captionsetup{singlelinecheck=off}

\ifdefined\NOTARTICLE
\else

%%% ToC (table of contents) APPEARANCE
\usepackage[nottoc,notlof,notlot]{tocbibind} % Put the bibliography in the ToC
\usepackage[titles,subfigure]{tocloft} % Alter the style of the Table of Contents
\renewcommand{\cftsecfont}{\rmfamily\mdseries\upshape}
\renewcommand{\cftsecpagefont}{\rmfamily\mdseries\upshape} % No bold!

\fi

%% Font things %%
\usepackage{amssymb}
\usepackage{cmll} % Linear logic symbols!
\ifdefined\FEWFONTS
\else
\usepackage{bm} % for bold Greek letters
\fi
\usepackage{stmaryrd}
\usepackage{bbm}

%% Get the sqsubsetneqq character from the mathabx package
\DeclareFontFamily{U}{mathb}{\hyphenchar\font45}
\DeclareFontShape{U}{mathb}{m}{n}{
      <5> <6> <7> <8> <9> <10> gen * mathb
      <10.95> mathb10 <12> <14.4> <17.28> <20.74> <24.88> mathb12
      }{}
\DeclareSymbolFont{mathb}{U}{mathb}{m}{n}

\DeclareMathSymbol{\sqsubsetneq}    {3}{mathb}{"88}
\DeclareMathSymbol{\varsqsubsetneq} {3}{mathb}{"8A}
\DeclareMathSymbol{\varsqsubsetneqq}{3}{mathb}{"92}
\DeclareMathSymbol{\sqsubsetneqq}   {3}{mathb}{"90}

%% Get the left and right moons from the wasysym package

\DeclareFontFamily{U}{wasy}{}
\DeclareFontShape{U}{wasy}{m}{n}{ <5> <6> <7> <8> <9> gen * wasy
      <10> <10.95> <12> <14.4> <17.28> <20.74> <24.88>wasy10  }{}
\DeclareFontShape{U}{wasy}{b}{n}{ <-10> sub * wasy/m/n
 <10> <10.95> <12> <14.4> <17.28> <20.74> <24.88>wasyb10 }{}
\DeclareFontShape{U}{wasy}{bx}{n}{ <-> sub * wasy/b/n}{}

\def\wasyfamily{\fontencoding{U}\fontfamily{wasy}\selectfont}
\def\leftmoon   {\mbox{\wasyfamily\char36}}
\def\rightmoon  {\mbox{\wasyfamily\char37}}

%% Lists %%
\usepackage{enumerate}

%% Graphics %%
\usepackage{tikz}
\usetikzlibrary{cd}
\usetikzlibrary{patterns}
\usetikzlibrary{calc}
\usetikzlibrary{decorations.pathmorphing}
\usetikzlibrary{positioning}

\tikzset{inlinearrows/.style={anchor=base,baseline,x=0.6\baselineskip,y=0.6\baselineskip}}

\ifdefined\BEAMER
\else

%% Theorems! %%
\usepackage{amsthm}
\theoremstyle{plain} % Theorems, lemmas, propositions etc.
\newtheorem{theorem}{Theorem}[section]
\newtheorem{lemma}[theorem]{Lemma}
\newtheorem{proposition}[theorem]{Proposition}
\newtheorem{corollary}[theorem]{Corollary}
\newtheorem{fact}[theorem]{Fact}
\newtheorem{construction}[theorem]{Construction}
\theoremstyle{definition} % Definitions etc.  
\newtheorem{definition}[theorem]{Definition}
\newtheorem{notation}[theorem]{Notation}
\theoremstyle{remark} % Remarks
\newtheorem{remark}[theorem]{Remark}
\newtheorem{remarks}[theorem]{Remarks}
\newtheorem{example}[theorem]{Example}
\newtheorem{question}[theorem]{Question}
\newtheorem{slogan}[theorem]{Slogan}

\newtheoremstyle{note} {3pt} {3pt} {\itshape} {} {\itshape} {:} {.5em} {} % For short notes
\theoremstyle{note}
\newtheorem{note}[theorem]{Note}

\fi

%% Exercises and answers %%
\usepackage{answers}

\newtheoremstyle{exercisestyle}% name
  {6pt}   % ABOVESPACE
  {6pt}   % BELOWSPACE
  {\itshape}  % BODYFONT
  {0pt}       % INDENT (empty value is the same as 0pt)
  {\bfseries} % HEADFONT
  {.}         % HEADPUNCT
  {3pt} % HEADSPACE
  {}          % CUSTOM-HEAD-SPEC

\theoremstyle{exercisestyle}
\newtheorem{exercise}{Exercise}
\newtheorem{answerthm}{Exercise}

\Newassociation{answer}{answerthm}{answers}
\newcommand{\answerthmparams}{}

%% Changes to enumerate things so they look better %%\sigma$

\makeatletter
\def\enumfix{%
\if@inlabel
 \noindent \par\nobreak\vskip-\topsep\hrule\@height\z@
\fi}

\let\olditemize\itemize
\def\itemize{\enumfix\olditemize}
\let\oldenumerate\enumerate
\def\enumerate{\enumfix\oldenumerate}

%% Random crap %%
\usepackage{xifthen}

\makeatletter
\def\thm@space@setup{%
  \thm@preskip=\parskip \thm@postskip=0pt
}
\makeatother

\makeatletter
\newcommand*{\relrelbarsep}{.386ex}
\newcommand*{\relrelbar}{%
  \mathrel{%
    \mathpalette\@relrelbar\relrelbarsep
  }%
}
\newcommand*{\@relrelbar}[2]{%
  \raise#2\hbox to 0pt{$\m@th#1\relbar$\hss}%
  \lower#2\hbox{$\m@th#1\relbar$}%
}
\providecommand*{\rightrightarrowsfill@}{%
  \arrowfill@\relrelbar\relrelbar\rightrightarrows
}
\providecommand*{\leftleftarrowsfill@}{%
  \arrowfill@\leftleftarrows\relrelbar\relrelbar
}
\providecommand*{\xrightrightarrows}[2][]{%
  \ext@arrow 0359\rightrightarrowsfill@{#1}{#2}%
}
\providecommand*{\xleftleftarrows}[2][]{%
  \ext@arrow 3095\leftleftarrowsfill@{#1}{#2}%
}
\makeatother

\newcommand{\catname}[1]{{\normalfont\textbf{#1}}}
\newcommand{\Rings}{\catname{CRing}}
\newcommand{\CAT}{\catname{CAT}}
%\newcommand{\Top}{\catname{Top}}
\newcommand{\Set}{\catname{Set}}
\newcommand{\Cat}{\catname{Cat}}
\newcommand{\MonCat}{\catname{MonCat}}
\newcommand{\SymmMonCat}{\catname{SymmMonCat}}
\newcommand{\Cont}{\catname{Cont}}
\newcommand{\Sch}{\catname{Sch}}
\newcommand{\Rel}{\catname{Rel}}
\newcommand{\Coh}{\catname{Coh}}
\newcommand{\Inj}{\catname{Inj}}
\newcommand{\Dcpo}{\catname{Dcpo}}
\newcommand{\Mod}[1][]{\ifthenelse{\isempty{#1}}{\catname{Mod}}{#1\catname{mod}}}
\DeclareMathOperator{\sh}{Sh}
\newcommand{\Sh}[1][]{\ifthenelse{\isempty{#1}}{\sh}{\sh(#1)}}
\newcommand{\map}[3]{#2\xrightarrow{#1} #3}
\newcommand*\from{\colon}
\newcommand*\bigto{\Rightarrow}
\newcommand{\cmap}[3]{#1\from{}#2\to{}#3}
\newcommand\oppcat[1]{#1^{\mathrm{op}}}
\newcommand{\object}{\colon}
\DeclareRobustCommand{\vmap}[3] {\begin{tikzcd} #2 \arrow[d, "#1"] \\ #3 \end{tikzcd}}
\newcommand{\partref}[1]{(\ref{#1})}
\newcommand{\intgrpd}[4] {#1 \xrightrightarrows[#3]{#4} #2}
\DeclareRobustCommand{\bigintgrpd}[4] {\begin{tikzcd}[ampersand replacement=\&] #1 \arrow[r, shift left=0.5ex, "#3"] \arrow[r, shift right=0.5ex, "#4"'] \& #2 \end{tikzcd}}

\usepackage{xspace}

\newcommand{\etale}{\'{e}tale\xspace}
\newcommand{\Etale}{\'{E}tale\xspace}

\def \inv {^{-1}}

\DeclareMathOperator{\id}{id}
\DeclareMathOperator{\op}{op}
\DeclareMathOperator{\pr}{pr}
\DeclareMathOperator{\inj}{in}
\DeclareMathOperator{\pre}{{pre}}
\DeclareMathOperator{\et}{{\acute{e}t}}

\DeclareMathOperator{\Hom}{Hom}
\DeclareMathOperator{\Spec}{Spec}

\DeclareMathOperator{\ol}{ol}

\def\presuper#1#2%
  {\mathop{}%
   \mathopen{\vphantom{#2}}^{#1}%
   \kern-\scriptspace%
   #2}
\def\presub#1#2%
  {\mathop{}%
   \mathopen{\vphantom{#2}}_{#1}%
   \kern-\scriptspace%
   #2}

\newsavebox{\overlongequation}
\newenvironment{longdiagram}
 {\begin{displaymath}\begin{lrbox}{\overlongequation}$\displaystyle}
 {$\end{lrbox}\makebox[0pt]{\usebox{\overlongequation}}\end{displaymath}}

%% Our things %%

\newcommand{\neggame}[1]{\presuper{\perp}{#1}}
\newcommand{\tensor}{\otimes}
\newcommand{\Tensor}{\bigotimes}
\newcommand{\sequoid}{\oslash}
\newcommand{\varsequoid}{\vartriangleleft}
\renewcommand{\implies}{\multimap}
\newcommand{\iimpl}{\Longrightarrow}
\newcommand{\comp}[2]{#1 \circ #2}
\newcommand{\icomp}[2]{\comp{#1}{#2}}
\newcommand{\cprd}{\sqcup}
\newcommand{\bigcprd}{\bigsqcup}
\newcommand{\G}{\mathcal G}
\newcommand{\W}{\mathcal W}
\newcommand{\suchthat}{\;\colon\;}
\newcommand{\varsuchthat}{\;\mid\;}
\newcommand{\esuchthat}{\;.\;}
\newcommand{\OP}{\{O,P\}}
\newcommand{\QA}{\{Q,A\}}
\renewcommand{\L}{\mathcal L}
\newcommand{\F}{\mathcal F}
\newcommand{\U}{\mathcal U}
\newcommand{\s}{\mathfrak s}
\renewcommand{\t}{\mathfrak t}
\renewcommand{\u}{\mathfrak u}
\renewcommand{\d}{\mathfrak d}
\newcommand{\e}{\mathfrak e}
\newcommand{\emptyplay}{\epsilon}
\newcommand{\bracketed}[1]{\left({#1}\right)}
\newcommand{\bneggame}[1]{{\bracketed{\neggame{#1}}}}
\newcommand{\prefix}{\sqsubseteq}
\newcommand{\ppprefix}{\sqsubset}
\newcommand{\pprefix}{\sqsubsetneqq}
\renewcommand{\ss}{\mathbf{s}}
\newcommand{\bN}{\mathbb{N}}
\newcommand{\bC}{\mathbb{C}}
\newcommand{\bB}{\mathbb{B}}
\newcommand{\bP}{\mathbb{P}}
\newcommand{\pfun}{\rightharpoonup}
\newcommand{\grel}[1]{\underline{#1}}
\DeclareMathOperator{\length}{length}
\renewcommand{\b}{\mathfrak b}
\renewcommand{\r}{\mathfrak r}
\newcommand{\bbeta}{{\bm{\beta}}}
\newcommand{\st}{{\Sigma^*}}
\let\sec\S
\renewcommand{\S}{{\mathfrak{S}}}
\DeclareMathOperator{\cc}{cc}
\DeclareMathOperator{\subs}{subs}
\DeclareMathOperator{\ret}{ret}
\DeclareMathOperator{\zz}{zz}
\newcommand{\aaa}{\mathbf{a}}
\newcommand{\bbb}{\mathbf{b}}
\newcommand{\ccc}{\mathbf{c}}
\newcommand{\ddd}{\mathbf{d}}
\newcommand{\B}{\mathcal B}
\newcommand{\BB}{\mathbf B}
\renewcommand{\H}{\mathcal H}
\DeclareMathOperator{\assoc}{assoc}
\DeclareMathOperator{\lunit}{lunit}
\DeclareMathOperator{\runit}{runit}
\DeclareMathOperator{\dom}{dom}
\DeclareMathOperator{\sym}{sym}
\newcommand{\braid}{\sym}
\newcommand{\blank}{\,\underline{\hspace{1.5ex}}\,}
\DeclareMathOperator{\cn}{cn}
\newcommand{\impliescn}{\protect\overset{\cn}{\implies}}
\newcommand{\C}{{\mathcal{C}}}
\newcommand{\D}{{\mathcal{D}}}
\newcommand{\E}{{\mathcal{E}}}
\newcommand{\V}{{\mathcal{V}}}
\newcommand{\EE}{{\mathbf{E}}}
\DeclareMathOperator{\ev}{ev}
\newcommand{\der}{{\mathtt{der}}}
\newcommand{\mult}{{\mathtt{mult}}}
\DeclareMathOperator{\wk}{wk}
\newcommand{\toisom}{{\xrightarrow{\cong}}}
\DeclareMathOperator{\passoc}{{\mathsf{passoc}}}
\DeclareMathOperator{\pcomm}{{\mathsf{pcomm}}}
\DeclareMathOperator{\run}{{\mathsf{r}}}
\DeclareMathOperator{\lun}{{\mathsf{l}}}
\newcommand{\fcoal}[1]{{\leftmoon #1 \rightmoon}}
\DeclareMathSymbol{\co}{\mathord}{operators}{"3C}
\DeclareMathSymbol{\nw}{\mathord}{operators}{"3E}
\newcommand{\T}{\mathfrak{T}}
\renewcommand{\subset}{\subseteq}
\newcommand{\Ord}{\catname{Ord}}
\newcommand{\FS}{\mathcal{FS}}
\DeclareMathOperator{\rank}{rank}
\DeclareMathOperator{\dist}{{\mathsf{dist}}}
\DeclareMathOperator{\dec}{{\mathsf{dec}}}
\DeclareMathOperator{\str}{str}
\DeclareMathOperator{\weak}{weak}
\DeclareMathOperator{\Strat}{Strat}
\DeclareMathOperator{\OppStrat}{OppStrat}
\newcommand{\seqs}[1]{{\overline{{#1}^{*}}}}
\def\flushRight{\leftskip0pt plus 1fill\rightskip0pt}
\def\Centering{\relax\ifvmode\centering\fi}
\newcommand{\deno}[1]{\left\llbracket#1\right\rrbracket}
\newcommand{\converges}{\Downarrow}
\newcommand{\diverges}{\Uparrow}
\newcommand{\mustconverge}{\converges^{\text{must}}}
\newcommand{\Iflt}{\mathtt{If{<}\;}}
\newcommand{\Ifgt}{\mathtt{If{>}\;}}
\newcommand{\inr}{{\mathsf{inr}}}
\newcommand{\inl}{{\mathsf{inl}}}
\newcommand{{\Na}}{\bN}
\newcommand{{\cell}}{{\mathsf{cell}}}
\newcommand{\fix}{{\mathsf{fix}}}
\newcommand{\eq}{{\mathsf{eq}}}
\DeclareMathOperator{\CCom}{CCom}
\newcommand{\power}{\mathfrak P}

% Slanty things
\newcommand*{\xslant}[2][76]{%
  \begingroup
    \sbox0{#2}%
    \pgfmathsetlengthmacro\wdslant{\the\wd0 + cos(#1)*\the\wd0}%
    \leavevmode
    \hbox to \wdslant{\hss
      \tikz[
        baseline=(X.base),
        inner sep=0pt,
        transform canvas={xslant=cos(#1)},
      ] \node (X) {\usebox0};%
      \hss
      \vrule width 0pt height\ht0 depth\dp0 %
    }%
  \endgroup
}

\makeatletter
\newcommand*{\xslantmath}{}
\def\xslantmath#1#{%
  \@xslantmath{#1}%
}
\newcommand*{\@xslantmath}[2]{%
  % #1: optional argument for \xslant including brackets
  % #2: math symbol
  \ensuremath{%
    \mathpalette{\@@xslantmath{#1}}{#2}%
  }%
}
\newcommand*{\@@xslantmath}[3]{%
  % #1: optional argument for \xslant including brackets
  % #2: math style
  % #3: math symbol
  \xslant#1{$#2#3\m@th$}%
}
\makeatother

\newcommand{\seqdeno}[1]{\xslantmath{\llbracket}#1\xslantmath{\rrbracket}}

% Empty set etc.

\let\oldemptyset\emptyset
\let\emptyset\varnothing

%% Constant width xrightarrows
\newlength{\arrow}
\settowidth{\arrow}{\scriptsize$1000$}
\newcommand*{\constantwidthxrightarrow}[1]{\xrightarrow{\mathmakebox[\arrow]{#1}}}

%% Landscape pages
\usepackage{everypage}
\usepackage{environ}
\usepackage{pdflscape}
\newcounter{abspage}

\ifdefined\NOTARTICLE

\else

\makeatletter
\newcommand{\newSFPage}[1]% #1 = \theabspage
  {\global\expandafter\let\csname SFPage@#1\endcsname\null}

\NewEnviron{SidewaysFigure}{\begin{figure}[p]
\protected@write\@auxout{\let\theabspage=\relax}% delays expansion until shipout
  {\string\newSFPage{\theabspage}}%
\ifdim\textwidth=\textheight
  \rotatebox{90}{\parbox[c][\textwidth][c]{\linewidth}{\BODY}}%
\else
  \rotatebox{90}{\parbox[c][\textwidth][c]{\textheight}{\BODY}}%
\fi
\end{figure}}

\AddEverypageHook{% check if sideways figure on this page
  \ifdim\textwidth=\textheight
    \stepcounter{abspage}% already in landscape
  \else
    \@ifundefined{SFPage@\theabspage}{}{\global\pdfpageattr{/Rotate 0}}%
    \stepcounter{abspage}%
    \@ifundefined{SFPage@\theabspage}{}{\global\pdfpageattr{/Rotate 90}}%
  \fi}
\makeatother

\fi

%% PCF Things

\newcommand{\nat}{{\mathtt{nat}}}
\newcommand{\bool}{{\mathtt{bool}}}

\newcommand{\Y}{\mathbf{Y}}
\newcommand{\opto}{\longrightarrow}
\newcommand{\oopto}{\dashrightarrow}
\newcommand{\n}{{\mathtt{n}}}
\DeclareMathOperator{\IfO}{{\mathsf{If0}}}
\DeclareMathOperator{\suc}{{\mathsf{succ}}}
\DeclareMathOperator{\pred}{{\mathsf{pred}}}
\newcommand{\0}{{\mathtt{0}}}

\newcommand{\iter}{{\mathtt{iter}}}
\newcommand{\rec}{\iter}
\newcommand{\Var}{{\mathtt{Var}}}
\DeclareMathOperator{\Varr}{Var}
\newcommand{\new}{{\mathtt{new}}}
\newcommand{\case}{{\mathtt{case}}}

\newcommand{\lmam}{\mathrel{\sqsubseteq_{m\&m}}}
\newcommand{\emam}{\mathrel{\equiv_{m\&m}}}
\newcommand{\lst}{\mathrel{\lesssim}}
\newcommand{\smam}{\mathrel{\sim_{m\&m}}}
\newcommand{\amam}{\mathrel{\approx_{m\&m}}}

\newcommand{\oes}{\sim}

%% Idealized Algol things

\newcommand{\com}{{\mathtt{com}}}
\newcommand{\skipp}{{\mathsf{skip}}}
\DeclareMathOperator{\seq}{{\mathsf{seq}}}
\DeclareMathOperator{\neww}{{\mathsf{new}}}
\DeclareMathOperator{\mkvar}{{\mathsf{mkvar}}}
\newcommand{\deref}{\texttt{@}}
\DeclareMathOperator{\dereff}{\mathsf{deref}}
\DeclareMathOperator{\assign}{\mathsf{assign}}
\newcommand{\ia}[2]{\langle #1 , #2 \rangle}
\newcommand{\stup}[3]{\langle #1 \mid #2 \mapsto #3 \rangle}

%% Hyland-Ong games things

\newbox\gnBoxA
\newdimen\gnCornerHgt
\setbox\gnBoxA=\hbox{$\ulcorner$}
\global\gnCornerHgt=\ht\gnBoxA
\newdimen\gnArgHgt
\def\pv #1{%
    \setbox\gnBoxA=\hbox{$#1$}%
    \gnArgHgt=\ht\gnBoxA%
    \ifnum     \gnArgHgt<\gnCornerHgt \gnArgHgt=0pt%
    \else \advance \gnArgHgt by -\gnCornerHgt%
    \fi \raise\gnArgHgt\hbox{$\ulcorner$} \box\gnBoxA %
    \raise\gnArgHgt\hbox{$\urcorner$}}
\def\ov #1{%
    \setbox\gnBoxA=\hbox{$#1$}%
    \gnArgHgt=\ht\gnBoxA%
    \ifnum     \gnArgHgt<\gnCornerHgt \gnArgHgt=0pt%
    \else \advance \gnArgHgt by -\gnCornerHgt%
    \fi \raise\gnArgHgt\hbox{$\llcorner$} \box\gnBoxA %
    \raise\gnArgHgt\hbox{$\lrcorner$}}
\newcommand{\ct}[1]{\lceil#1\rceil}
\DeclareMathOperator{\Int}{int}

%% Nondeterministic Factorization things

\newcommand{\code}{\mathsf{code}}
\newcommand{\Det}{\mathsf{Det}}

%% Flexible strategy things

\newcommand{\stle}{{\;\le_s\;}}
\newcommand{\steq}{{\;=_s\;}}
\newcommand{\exle}{\sqsubseteq}
\newcommand{\exlub}{\bigsqcup}
\newcommand{\dv}{{\text{\lightning}}}
\DeclareMathOperator{\pocl}{pocl}
\newcommand{\plot}{\mathrel{\triangleleft}}
\newcommand{\shad}{\mathfrak{S}}
%\newcommand{\tree}{\mathfrak{T}}
\newcommand{\Tau}{T}
\newcommand{\Epsilon}{E}
\newcommand{\sw}{\triangleleft}

%% Roman numerals

\newcommand{\RN}[1]{%
  \textup{\uppercase\expandafter{\romannumeral#1}}%
}
\newcommand{\RNl}[1]{%
  \mathrel{\raisebox{1pt}{$\overline{\underline{#1}}$}}
}

%% Game language things

\newcommand{\ul}[1]{{\underline{#1}}}
\newcommand{\A}{{\mathcal{A}}}
\renewcommand{\P}{\mathcal P}
\newcommand{\M}{\mathcal M}
\newcommand{\N}{\mathcal N}
\newcommand{\X}{\mathcal X}
\newcommand{\YY}{\mathcal Y}
\newcommand{\hole}{\blank}
\newcommand{\Tct}{\xrightarrow{T}t}
\newcommand{\teamconverge}[2]{\xrightarrow{#1}#2}

%% Inference rule things
\newcommand{\rulename}[1]{\LeftTirNameStyle{#1}}
\newcommand{\ts}{\mathbin{\vdash}}
\newcommand{\nts}{\mathbin{\not\vdash}}

%% Double category things
\newcommand{\hc}[2]{\left({#1}\middle|{#2}\right)}
\newcommand{\vc}[2]{\left(\frac{#1}{#2}\right)}

%% What is going on?
\DeclareMathOperator{\Kl}{Kl}
\DeclareMathOperator{\Mell}{Mell}
\newcommand{\powerset}{\mathcal P}
\DeclareMathOperator{\ask}{{\mathsf{ask}}}
\newcommand{\sleep}{{\mathsf{sleep}}}
\newcommand{\true}{\mathbbm{t}}
\newcommand{\false}{\mathbbm{f}}
\DeclareMathOperator{\If}{\mathsf{If}}
\newcommand{\Then}{\mathrel{\mathsf{then}}}
\newcommand{\Else}{\mathrel{\mathsf{else}}}
\newcommand\cat{\mathbin{+\mkern-10mu+}}

%% Profunctor arrows

\makeatletter
\def\slashedarrowfill@#1#2#3#4#5{%
  $\m@th\thickmuskip0mu\medmuskip\thickmuskip\thinmuskip\thickmuskip
   \relax#5#1\mkern-7mu%
   \cleaders\hbox{$#5\mkern-2mu#2\mkern-2mu$}\hfill
   \mathclap{#3}\mathclap{#2}%
   \cleaders\hbox{$#5\mkern-2mu#2\mkern-2mu$}\hfill
   \mkern-7mu#4$%
}
\def\rightslashedarrowfill@{%
  \slashedarrowfill@\relbar\relbar\mapstochar\rightarrow}
\newcommand\xslashedrightarrow[2][]{%
  \ext@arrow 0055{\rightslashedarrowfill@}{#1}{#2}}
\makeatother
\newcommand{\pto}{{\xslashedrightarrow{} }}

%% Profunctors 
\DeclareMathOperator{\Prof}{Prof}
\DeclareMathOperator{\End}{End}
\DeclareMathOperator{\Endoprof}{Endoprof}

%% Our

\def\searchmacro#1{
  \AtBeginOfFiles{%
    \ifdefined#1
      \expandafter\def\csname \currfilename:found\endcsname{}%
    \fi}
  \AtEndOfFiles{%
    \ifdefined#1
      \unless\ifcsname \currfilename:found\endcsname
        \immediate\write\finder{found in '\currfilename'}%
    \fi\fi}}

%% Isomorphism arrows on commutative diagrams %%
\tikzset{Isom/.style={every to/.append style={edge node={node [sloped, above, allow upside down, auto=false]{$\cong$}}}},
         Isom'/.style={every to/.append style={edge node={node [sloped, above, allow upside down, auto=false, rotate=180]{$\cong$}}}},
         Sim/.style={every to/.append style={edge node={node [sloped, above, allow upside down, auto=false]{$\sim$}}}},
         Sim'/.style={every to/.append style={edge node={node [sloped, above, allow upside down, auto=false, rotate=180]{$\sim$}}}}}

%% Adjunctions
\newcommand{\adjunction}[4]{%
  {#1} \underset{\underset{#3}{\longleftarrow}}{\overset{\overset{#2}{\longrightarrow}}{\bot}} {#4}}        

%% Important!
\newcommand\Mellies{Melli\`{e}s\xspace}

\makeatletter
\newcommand{\colim@}[2]{%
  \vtop{\m@th\ialign{##\cr
    \hfil$#1\operator@font colim$\hfil\cr
    \noalign{\nointerlineskip\kern1.5\ex@}#2\cr
    \noalign{\nointerlineskip\kern-\ex@}\cr}}%
}
\newcommand{\colim}{%
  \mathop{\mathpalette\colim@{\rightarrowfill@\textstyle}}\nmlimits@
}
\makeatother

\makeatletter
\newcommand{\laxcolim@}[2]{%
  \vtop{\m@th\ialign{##\cr
    \hfil$#1\operator@font colim_l$\hfil\cr
    \noalign{\nointerlineskip\kern1.5\ex@}#2\cr
    \noalign{\nointerlineskip\kern-\ex@}\cr}}%
}
\newcommand{\laxcolim}{%
  \mathop{\mathpalette\laxcolim@{\rightarrowfill@\textstyle}}\nmlimits@
}
\makeatother

\DeclareMathOperator{\Colim}{colim}

\DeclareMathOperator{\DG}{DG}
\DeclareMathOperator{\RV}{RV}
\newcommand{\Rv}{\catname{Rv}}

\let\choose\undefined
\DeclareMathOperator{\choose}{\mathsf{choose}}
\DeclareMathOperator{\tr}{tr}
\DeclareMathOperator{\test}{test}

%% Slot game things %%
\newcommand{\circled}[1]{\raisebox{.5pt}{\textcircled{\raisebox{-.9pt} {#1}}}}
\newcommand{\slot}{{\circled{\$}}}

\DeclareMathOperator{\may}{may}
\DeclareMathOperator{\must}{must}

\newcommand{\encode}[1]{\lceil#1\rceil}
\DeclareMathOperator{\app}{\mathsf{app}}
\DeclareMathOperator{\lett}{\mathsf{let}}
\newcommand{\inn}{\mathrel{\mathsf{in}}}
\DeclareMathOperator{\byval}{\mathsf{byval}}

\DeclareMathOperator{\rread}{read}
\DeclareMathOperator{\wwrite}{write}

\DeclareSymbolFont{bbsymbol}{U}{bbold}{m}{n}
\DeclareMathSymbol{\bbsemicolon}{\mathbin}{bbsymbol}{"3B}
\newcommand{\semicom}{\bbsemicolon}

\newcommand{\ms}{\makebox[-1pt]{}}

\DeclareMathOperator{\Acc}{Acc}
\DeclareMathOperator{\im}{Im}
\DeclareMathOperator{\wit}{wit}

%%% END Article customizations


\usepackage{lua-visual-debug}

\begin{document}

\chapter{A Fully Abstract Game Semantics for Idealized Algol}

To introduce our material, we will go back over some old ground, namely the fully abstract game semantics for Idealized Algol developed by Abramsky and McCusker in \cite{SamsonGuyIAActive}.  
In keeping with the spirit of this thesis, we will aim to use category theoretic methods, and so our proofs of soundness and adequacy will depart from those given by Abramsky and McCusker, and will instead involve coalgebraic ideas developed by Laird in \cite{laird02} and \cite{LairdCofCommCom}.

\section{Idealized Algol}

The ground types of Idealized Algol are called $\com$, $\bool$, $\nat$ and $\Var$.  
The first three are data types corresponding to the sets $\bC = \{a\}$, $\bB = \{\true,\false\}$ and $\bN = \{0,1,2,\cdots\}$.
$\com$ takes the role of a command or void type; typically, although the return value of a function $T \to \com$ will not convey any information, the function will have side effects that \emph{do} make a difference.

The type $\Var$ is the type of a variable that holds elements of $\bN$.
It is best understood as corresponding to the following pseudo-Java `interface'.

\begin{minipage}{\linewidth}
\begin{lstlisting}[language=Java, morekeywords={nat,com}]
public interface Var
{
  nat read();
  com write(nat value);
}
\end{lstlisting}
\end{minipage}

We now present the typing rules for the language.  
Here, $\Gamma$ will represent a \emph{context}; i.e., a list $x_1\from T_1,\cdots,x_n\from T_n$ of variable names together with their types.

First, we have the usual rules for the simply typed lambda calculus.

\begin{mathpar}
  \inferrule*{ }{\Gamma,x\from T \ts x \from T}
  \and
  \inferrule*{\Gamma\ts M \from S \to T \\ \Gamma \ts N \from S}{\Gamma\ts MN \from T}
  \and
  \inferrule*{\Gamma,x\from S \ts M \from T}{\Gamma \ts \lambda x^S.M \from S \to T}
\end{mathpar}

We then have rules for each of the base types.  
At type $\com$ we have:
\begin{mathpar}
  \inferrule*{ }{\Gamma\ts\skipp\from\com}
  \and
  \inferrule*[right={$T\in\{\com,\bool,\nat\}$}]{\Gamma\ts M\from\com \\ \Gamma\ts N \from T}{\Gamma\ts M;N\from T}\,.
\end{mathpar}
Here, $\skipp$ is a generic command with no side-effects that returns the unique element of the singleton set $\bC$.  
$M;N$ represents the sequential composition of $M$ with $N$; i.e., the term that first evaluates $M$, performing any of its side-effects, and then evaluates $N$ and returns the result.

At type $\bool$ we have true/false values and conditionals.
\begin{mathpar}
  \inferrule*{ }{\Gamma\ts\true\from\bool}
  \and
  \inferrule*{ }{\Gamma\ts\false\from\bool}
  \\
  \inferrule*[right={$T\in\{\com,\bool,\nat\}$}]{\Gamma\ts M \from \bool \\ \Gamma \ts N \from T \\ \Gamma \ts P \from T}{\Gamma\ts \If M \Then N \Else P \from T}
\end{mathpar}

At type $\nat$ we have numerals, arithmetic operators and a conditional that tests whether a number is equal to $0$ or not.
\begin{mathpar}
  \inferrule*{ }{\Gamma\ts n\from\nat}
  \and
  \inferrule*{\Gamma\ts M \from \nat}{\Gamma\ts \suc M \from \nat}
  \and
  \inferrule*{\Gamma\ts M \from \nat}{\Gamma\ts \pred M\from \nat}
  \\
  \inferrule*[right={$T\in\{\com,\bool,\nat\}$}]{\Gamma\ts M \from \nat \\ \Gamma\ts N \from T \\ \Gamma \ts P \from T}{\Gamma \ts \IfO M \Then N \Else P \from T}
\end{mathpar}

At type $\Var$, we have terms that call the read and write `methods' to dereference the variable or to assign a new value to it.
\begin{mathpar}
  \inferrule*{\Gamma\ts V \from \Var}{\Gamma \ts \oc V \from \nat}
  \and
  \inferrule*{\Gamma\ts V \from \Var \\ \Gamma \ts E \from \nat}{\Gamma \ts V\gets E \from \com}
\end{mathpar}

We also have the ability to create a new variable.
\begin{mathpar}
  \inferrule*[right={$T\in\{\com,\bool,\nat\}$}]{\Gamma,x\from \Var \ts M \from T}{\Gamma\ts \neww_T \lambda x.M\from T}
\end{mathpar}
The idea here is that if $M$ is a term that refers to some free variable $x$ of type $\Var$; then $\neww \lambda x.M$ makes $x$ behave like an actual storage cell (so, for instance, the result of the computation $\neww_\nat \lambda x.(x\gets 5);!x$ will be $5$).

We have another way of creating variables, using the $\mkvar$ keyword.  
If we think back to our illustration of the $\Var$ type as an interface, this becomes clearer.  
$\mkvar$ creates a new anonymous instance of the $\Var$ interface, using the `methods' supplied through its arguments.
\begin{mathpar}
  \inferrule*{\Gamma \ts M \from \nat \\ \Gamma \ts N \from \nat \to \com}{\Gamma\ts \mkvar M N \from \Var}
\end{mathpar}

Lastly, we have fixpoint combinators at all types that we use to implement recursion.
\begin{mathpar}
  \inferrule*{\Gamma \ts M \from T \to T}{\Gamma \ts \Y_T M \from T}
\end{mathpar}

\section{Games and Strategies}

We adopt the game semantics from \cite{SamsonGuyIAActive}; these are based on the game semantics developed in \cite{hoPcf}, with a modification to make them into a linear category.

\begin{definition}
  An \emph{arena} is a tuple $A=(M_A,\lambda_A,\ts_A)$, where
  \begin{itemize}
    \item $M_A$ is a set of \emph{moves},
    \item $\lambda_A \from M_A \to \OP \times \QA$ is a function that identifies each move as either an \emph{$O$-move} or a \emph{$P$-move}, and as either a \emph{question} or an \emph{answer}, and
    \item $\ts_A$ is a relation between $M_A+\{*\}$ and $M_A$ such that
      \begin{itemize}
        \item if $*\ts_A a$, then $\lambda_A(a)=(O,Q)$, and if $b\ts_A a$ then $b=*$,
        \item if $a\ts_A b$ and $a$ is an answer, then $b$ is a question, and
        \item if $a \ts_A b$ and $a\ne *$, then either $a$ is an $O$-move and $b$ a $P$-move, or the other way round.
      \end{itemize}
  \end{itemize}
  If $*\ts_A a$, then we say that $a$ is an \emph{initial move} in $A$.  
  If $a \ts_A b$, the we say that $a$ \emph{enables} $b$.
\end{definition}

As a shorthand, we write $\lambda_A^{OP}\from M_A \to \OP$ for $\pr_1\circ\lambda_A$ and $\lambda_A^{QA}\from M_A \to \QA$ for $\pr_2\circ\lambda_A$.

\begin{definition}
  A \emph{justified sequence} in an arena $A$ is a finite sequence $s$ of moves together with, for each non-initial move $a$ occurring in $s$, a pointer back to some move $b$ occurring earlier in $s$ such that $b\ts_A a$.
  We say that \emph{$b$ justifies $a$} or that $b$ is the \emph{justifier} of $a$.

  Given such a justified sequence, we define the \emph{$P$-view} $\pv s$ and \emph{$O$-view} $\ov s$ of $s$ inductively as follows.
  \begin{IEEEeqnarray*}{RCL"s}
    \pv{\epsilon} & = & \epsilon & \\
    \pv{sa} & = & \pv{s}a & if $a$ is a $P$-move \\
    \pv{sa} & = & a & if $a$ is initial \\
    \pv{sbta} & = & \pv{s}ba & if $a$ is an $O$-move justified by $b$ \\
    & & & \\
    \ov{\epsilon} & = & \epsilon & \\
    \ov{sa} & = & \ov{s}a & if $a$ is an $O$-move \\
    \ov{sbta} & = & \ov{s}ba & if $a$ is a $P$-move justified by $b$
  \end{IEEEeqnarray*}
  
  A justified sequence $s$ is \emph{well-bracketed} if whenever a question $q$ justifies some answer $a$, then any question $q'$ occurring after $q$ and before $a$ must justify some answer $a'$ occurring between $q'$ and $a$, and moreover $a$ is the only answer justified by $q$.
  We say that a justified sequence $s$ is \emph{alternating} if it alternates between $O$-moves and $P$-moves, and that it is \emph{well-formed} if it is both well-bracketed and alternating.

  We say that a well-formed justified sequence is \emph{visible} if whenever $ta\prefix s$, and $a$ is a $P$-move, then the justifier of $a$ occurs in the $P$-view of $t$, and if whenever $tb\prefix s$, and $b$ is a non-initial $O$-move, then the justifier of $b$ occurs in the $O$-view of $t$.

  We say that a justified sequence $s$ is \emph{legal} if it is well-formed and visible, and write $L_A$ for the set of legal sequences occurring in $A$.
\end{definition}

Note that since every non-initial move in a justified sequence $s$ must be justified by some previous move, then the first move in the sequence must be initial and therefore an $O$-question.  
If $s$ is alternating, this means that $s$ ends with an $O$-move if it has odd length and with a $P$-move if it has even length.  

\begin{definition}
  Given a legal sequence $s\in L_A$, and a move $b$ in $s$, we say that a move $a$ in $s$ is \emph{hereditarily justified by $b$} if there is a chain of justification pointers going back from $a$ to $b$.

  We write $s\vert_b$ for the subsequence of $s$ given by all moves in $s$ that are hereditarily justified by $b$.
  Given a set $I$ of initial moves, we write $s\vert_I$ for the subsequence of $s$ given by all moves that are hereditarily justified by some $b\in I$.

  A \emph{game} is given by a tuple $A=(M_A,\lambda_A,\ts_A,P_A)$ where $(M_A,\lambda_A,\ts_A)$ is an arena and $P_A$ is a non-empty prefix-closed subset of $L_A$ such that if $s\in P_A$ and $I$ is a set of initial moves, then $s\vert_I\in P_A$.
\end{definition}

We shall call an odd-length sequence $s\in P_A$ an \emph{$O$-position} and an even-length sequence a \emph{$P$-position}

\begin{example}[Empty game]
  The \emph{empty game} $I$ is given by the tuple
  \[
    (\emptyset,\emptyset,\emptyset,\{\epsilon\})\,,
    \]
  where $\epsilon$ is the empty sequence.
  \label{ExEmptyGame}
\end{example}

\begin{example}[Data-type games]
  Let $X$ be some set.  
  Then we have a game, which we shall also call $X$, given by:
  \begin{itemize}
    \item $M_X = \{q\} + X$, 
    \item $\lambda_X(q)=(O,Q)$ and $\lambda_X(x)=(P,A)$ for all $x\in X$, 
    \item $q\ts_X x$ for each $x\in X$, and
    \item $P_X = \{\epsilon,q\}\cup\{qx\suchthat x\in X\}$, where the $x$ in $qx$ is justified by $q$.
  \end{itemize}

  In particular, we have games $\bC$, $\bB$ and $\bN$, which we shall use to model the datatypes $\com$, $\bool$ and $\nat$ of Idealized Algol.
\end{example}

\begin{definition}
  Let $A$ be a game.  
  Then a \emph{strategy} for $A$ is a non-empty even-prefix-closed set $\sigma\subset P_A$ of $P$-positions in $A$ such that if $sab,sac\in \sigma$ then $b=c$ and the justifier of $b$ is the justifier of $c$.  
\end{definition}
Here, we have identified a strategy for a game with the set of $P$-positions that can occur when player $P$ plays according to that strategy.  
So the condition we have given is one of \emph{determinism}: in any $O$-position $sa$ that can occur in the strategy, player $P$ must have at most one reply.

Note that there may be $O$-positions for which player $P$ has no reply at all; we use these to model non-terminating computations.

We write $\sigma\from A$ to denote that $\sigma$ is a strategy for the game $A$.

\begin{definition}
  A strategy $\sigma$ for a game $A$ is called \emph{innocent} if player $P$'s moves only depend on the current $P$-view; i.e., if whenever $sab\in\sigma$, $t\in\sigma$ and $ta\in P_A$ such that $\pv{sa}=\pv{ta}$, then we have $tab\in\sigma$.
\end{definition}

\section{Connectives on Games}
\label{SecConnectives}

In the \emph{product} $A\times B$ of games $A$ and $B$, player $O$ chooses either $A$ or $B$ on the first move and subsequent play is that game.

\begin{definition}
  Given games $A$,$B$, define a game $A\times B$ by
  \begin{itemize}
    \item $M_{A\times B} = M_A + M_B$,
    \item $\lambda_{A\times B} = [\lambda_A,\lambda_B]$,
    \item $* \ts_{A\times B} a$ if and only if $*\ts_A a$ or $*\ts_B a$ and $a \ts_{A\times B} b$ if and only if $a \ts_A b$ or $a\ts_B b$, and
    \item $P_{A\times B} = \{s\in L_{A\times B} \suchthat \text{$s\vert_A\in P_A$ \& $s\vert_B=\epsilon$ or $s\vert_A=\epsilon$ \& $s\vert_B\in P_B$}\}$.
  \end{itemize}
  We extend this to arbitrary products $\prod_i A_i$ in the obvious way.
  In particular, the product $1$ of the empty collection is the same as the empty game $I$ defined in Example \ref{ExEmptyGame}.
  \label{DefProduct}
\end{definition}

Here, we have written $s\vert_A$ for the subsequence of $s$ consisting of all moves from $M_A$ and $s\vert_B$ for the subsequence consisting of all moves from $M_B$.

In the \emph{tensor product} $A\tensor B$ of games $A$ and $B$, the games $A$ and $B$ are played in parallel, and player $O$ may switch between games when it is his turn.

\begin{definition}
  Given games $A$,$B$, define a game $A\tensor B$ by
  \begin{itemize}
    \item $M_{A\tensor B} = M_A + M_B$,
    \item $\lambda_{A\tensor B} = [\lambda_A,\lambda_B]$,
    \item $* \ts_{A\tensor B} a$ if and only if $*\ts_A a$ or $*\ts_B a$ and $a \ts_{A\tensor B} b$ if and only if $a \ts_A b$ or $a\ts_B b$, and
    \item $P_{A\tensor B} = \{s\in L_{A\tensor B} \suchthat \text{$s\vert_A\in P_A$ and $s\vert_B\in P_B$}\}$.
  \end{itemize}
\end{definition}

In the \emph{linear implication} $A\implies B$, the game $B$ is played in parallel with a version of $A$ in which the two players' roles have been switched around, and player $P$ may switch between the two games when it is her turn.

\begin{definition}
  Given games $A$,$B$, define a game $A\implies B$ by
  \begin{itemize}
    \item $M_{A\implies B} = M_A + M_B$,
    \item $\lambda_{A\implies B} = [\neg\circ\lambda_A,\lambda_B]$,
    \item $* \ts_{A\implies B} a$ if and only if $*\ts_B a$, and $a \ts_{A\implies B} b$ if and only if $a \ts_A b$ or $a\ts_B b$, or if $a$ is initial in $B$ and $b$ is initial in $a$, and
    \item $P_{A\implies B} = \{s\in L_{A\implies B} \suchthat \text{$s\vert_A\in P_A$ and $s\vert_B\in P_B$}\}$.
  \end{itemize}
\end{definition}

Here, $\neg\from \OP\times\QA \to \OP\times \QA$ is the function that reverses $O$ and $P$, while leaving $\QA$ unchanged.

In the \emph{exponential} of a game $A$, infinitely many copies of $A$ are played in parallel, and player $O$ may switch between copies whenever it is his move.

\begin{definition}
  Given a game $A$, define a game $\oc A$ by
  \begin{itemize}
    \item $M_{\oc A}=M_A$,
    \item $\lambda_{\oc A} = \lambda_A$,
    \item $\ts_{\oc A} = \ts_A$ and
    \item $P_{\oc A} = \{s\in L_{\oc A} \suchthat \text{$s\vert_b\in P_A$ for each initial move $b$ occurring in $s$}\}$.
  \end{itemize}
\end{definition}

Lastly, the \emph{sequoid} $A\sequoid B$ of two games $A$ and $B$ behaves like the tensor product $A\tensor B$, except that the opening move must take place in $A$.

\begin{definition}
  Given games $A,B$, define a game $A\sequoid B$ by
  \begin{itemize}
    \item $M_{A\sequoid B} = M_{A\tensor B}$, 
    \item $\lambda_{A\sequoid B} = \lambda_{A\tensor B}$, 
    \item $\ts_{A\sequoid B} = \ts_{A \tensor B}$ and
    \item $P_{A\sequoid B} = \{s\in P_{A\tensor B}\suchthat\text{$s=\epsilon$ or $s$ begins with a move from $A$}\}$.
  \end{itemize}
\end{definition}

\section{Composition of strategies}

\begin{definition}
  Let $A,B,C$ be arenas.  
  An \emph{interaction sequence} between $A,B,C$ is a justified sequence $\s$ of moves drawn from $M_A$, $M_B$ and $M_C$ such that $\s\vert_{A,B}\in L_{A\implies B}$ and $\s\vert_{B,C}\in L_{B\implies C}$.  
  Here, $\s\vert_{A,B}$ is the subsequence of $\s$ consisting of those moves from $\s$ that occur in $A$ or $B$, together with all justification pointers between moves in $A$ and $B$, and $\s\vert_{B,C}$ is defined similarly.

  We write $\Int(A,B,C)$ for the set of all interaction sequences between $A,B,C$.

  Given $\s\in\Int(A,B,C)$, we write $\s\vert_{A,C}$ for the subsequence of $\s$ consisting of those moves from $\s$ that occur in $A$ or $B$.  
  A move $b$ in $\s\vert_{A,C}$ justifies a move $a$ either if $b$ justifies $a$ in either the $A$ or the $C$ components, or if $b$ justifies in $\s$ some initial move $c$ in $B$, which itself justifies $a$.
\end{definition}

\begin{definition}
  Let $A,B,C$ be games, let $\sigma$ be a strategy for $A\implies B$ and let $\tau$ be a strategy for $B\implies C$.  
  We define $\sigma\|\tau$ to be given by the set
  \[
    \{\s\in\Int(A,B,C)\suchthat \text{$\s\vert_{A,B}\in\sigma$ and $\s\vert_{B,C}\in\tau$}\}\,.
    \]
  Then we define the \emph{composition} $\sigma;\tau$ of $\sigma$ and $\tau$ to be given by the set
  \[
    \{\s\vert_{A,C}\suchthat \s\in\sigma\|\tau\}\,.
    \]
\end{definition}

We need some small lemmata and definitions to help us show that this is a strategy.

\begin{lemma}
  We extend the function $\lambda_A^{OP}$ to sequences of moves by
  \begin{itemize}
    \item $\lambda_A^{OP}(\epsilon)=P$ and
    \item $\lambda_A^{OP}(sa)=\lambda_A(a)$.
  \end{itemize}

  If $s\in P_{A\implies B}$, then $\lambda_{A\implies B}^{OP}(s)=(\lambda_A^{OP}(s\vert_A)\Rightarrow \lambda_B^{OP}(s\vert_B))$, where $\Rightarrow$ is the binary operation on $\OP$ defined by
  \[
    \begin{array}{cc|c}
      P & Q & P\Rightarrow Q \\
      \hline
      P & P & P \\
      O & P & P \\
      P & O & O \\
      O & O & P
    \end{array}\,.
    \]
  Moreover, if $\lambda_A^{OP}(s\vert_A)=O$ then $\lambda_A^{OP}(s\vert_B)=O$.
  \label{LemCompositionLemma}
\end{lemma}
\begin{proof}
  Induction on the length of $s$.
  If $s=\epsilon$, then $s\vert_A=s\vert_B=\epsilon$, and so $(\lambda_A^{OP}(s\vert_A)\Rightarrow \lambda_B^{OP}(s\vert_B)) = (P\Rightarrow P) = P = \lambda_{A\implies B}^{OP}(s)$.  

  Suppose then that $s=ta$, and that $\lambda_{A\implies B}^{OP}(t)=O$.  
  This means that $\lambda_A^{OP}(t\vert_A)=P$ and $\lambda_B^{OP}(t\vert_B)=O$.
  Then, whether $a$ is a move in $A$ or a move in $B$, adding it will flip exactly one of these components -- so $\lambda_{A\implies B}(s\vert_A)=O$ and $\lambda_{A\implies B}^{OP}(s\vert_B)=O$ if $a$ is a move in $A$ and $\lambda_{A\implies B}(s\vert_A)=P$ and $\lambda_{A\implies B}^{OP}(s\vert_B)=P$ if $a$ is a move in $C$.

  Suppose instead that $\lambda_{A\implies B}^{OP}(t)=P$.  
  By induction, this means that either $\lambda_A^{OP}(t\vert_A)=P$ and $\lambda_B^{OP}(t\vert_B)=P$ or that $\lambda_A^{OP}(t\vert_A)=O$ and $\lambda_B^{OP}(t\vert_B)=O$.
  In the first case, this means that either $t\vert_A$ is empty or its last move is a $P$-move in $A$ (and therefore an $O$-move in $A\implies B$), and so the move $a$ must take place in $C$, meaning that $\lambda_A^{OP}(s\vert_A)=P$ and $\lambda_B^{OP}(s\vert_B)=O$.  

  Similarly, in the second case, the last move in $t\vert_C$ must be an $O$-move in $B$ (and therefore an $O$-move in $A\implies B$, and so the move $a$ must take place in $A$, meaning that $\lambda_A^{OP}(s\vert_A)=P$ and $\lambda_B^{OP}(s\vert_B)=O$.  
\end{proof}

It follows that

\begin{corollary}[Switching condition]
  Only player $P$ may switch between games in $A\implies B$; i.e., if $tab\in P_{A\implies B}$, and $a$ occurs in $A$ and $b$ in $B$, or if $a$ occurs in $B$ and $b$ in $A$, then $b$ is a $P$-move.
  \label{CorSwitchingCondition}
\end{corollary}
\begin{proof}
  Otherwise, $\lambda_{A\implies B}(t)=O$, so $\lambda_A(t\vert_A)=P$ and $\lambda_B(t\vert_B)=O$.  
  But we must also have $\lambda_{A\implies B}(tab)=O$, so $\lambda_A(tab\vert_A)=P$ and $\lambda_B(tab\vert_B)=O$.  
  But this is a contradiction, since $tab\vert_A$ and $tab\vert_B$ are both one move longer than the plays $t\vert_A$ and $t\vert_B$.
\end{proof}

\begin{definition}[{\cite[\sec 3.1]{Harmer2006InnocentGS}}]
  Given $\s\in \Int(A,B,C)$, we define the \emph{$P$-view} $\pv{\s}$ of $\s$ inductively as follows.
  \begin{IEEEeqnarray*}{RCL?s}
    \pv{\epsilon} & = & \epsilon & \\
    \pv{\s a} & = & \pv{\s}a & if $a$ is a move in $B$, an $O$-move in $A$ or a $P$-move in $C$ \\
    \pv{\s c} & = & c & if $c$ is an initial move of $C$ \\
    \pv{\s b\t a} & = & \pv{\s}ba & \parbox[t][][t]{200pt}{if $a$ is a $P$-move of $A$ or an $O$-move of $C$ and is justified by $b$} \\
  \end{IEEEeqnarray*}
\end{definition}

\begin{lemma}
  If $\s\in \Int(A,B,C)$, then $\pv{\s}\vert_{A,C}=\pv{\s\vert_{A,C}}$.
  \label{LemHarmerRestriction}
\end{lemma}
\begin{proof}
  Induction on the length of $\s$.  
  This is clear if $\s=\epsilon$.  
  
  If $a$ is an $O$-move in $A$ or a $P$-move in $C$, then $a$ is a $P$-move in $A\implies C$.  
  We have $\pv{\s a}\vert_{A,C}=\pv{\s}\vert_{A,C}a$, which by the inductive hypothesis is equal to $\pv{\s\vert_{A,C}}a$, which is the same as $\pv{\s a\vert_{A,C}}$.
  If $b$ is a move in $B$, then $\pv{\s b}\vert_{A,C}=\pv{\s}b\vert_{A,C}=\pv{\s}\vert_{A,C}=\pv{\s\vert_{A,C}}=\pv{\s b\vert_{A,C}}$, by the inductive hypothesis.

  If $c$ is initial in $C$, then $\pv{\s c}\vert_{A,C}=c=\pv{\s c\vert_{A,C}}$.

  Suppose $a$ is a $P$-move of $A$ or an $O$-move of $C$ -- so $a$ is an $O$-move in $A\implies C$ -- and suppose that $a$ is justified by $b$ in the sequence $\s b\t a$.  
  Since $a$ cannot be an initial move in $A$, $b$ must occur in the same game as $a$, and in particular must not occur in $B$.
  Then we have $\pv{\s b\t a}\vert_{A,C}=\pv{\s} ba\vert_{A,C}=\pv{\s}\vert_{A,C}ba$, which by the inductive hypothesis is equal to $\pv{\s\vert_{A,C}}ba=\pv{\s ba\vert_{A,C}}$.
\end{proof}

\begin{lemma}[{\cite[\sec 3.1]{Harmer2006InnocentGS}}]
  Let $\s a\in \Int(A,B,C)$ (so, in particular, $\s\vert_{A,B}$ and $\s\vert_{B,C}$ satisfy the visibility condition).  
  If $a$ is a move in $B$, an $O$-move in $A$ or a $P$-move in $C$, then $\pv{\s a}\in \Int(A,B,C)$.
  \label{LemHarmersLemma}
\end{lemma}
\begin{proof}
  Induction on the length of $\s$.  
  If $\s=\epsilon$, then this is clear.  
  Otherwise, suppose that $\s$ is non-empty.

  First, we claim that $\pv{\s}\in\Int(A,B,C)$.  
  If $\s$ ends with a move in $B$, an $O$-move in $A$ or a $P$-move in $C$, then this follows immediately from the inductive hypothesis.  
  Otherwise, suppose that $\s$ ends with a $P$-move in $A$ or an $O$-move in $C$.  
  If this last move is initial, then $\pv{\s}$ is a single move, so the claim is trivial.  
  Otherwise, write $\s=\t p\u r$, where $p$ justifies $r$.  
  By the inductive hypothesis, we have $\pv{\t p}\in \Int(A,B,C)$, and then $\pv{s}=\pv{\t p\u r}=\pv{\t}pr=\pv{\t p}r\in \Int(A,B,C)$.  

  Now, since $a$ is a $P$-move in $A\implies B$ or in $B\implies C$, its predecessor $b$ is an $O$-move and has some justifier $c$ contained in $\pv{\s\vert_X}$, where $X\in\{A\implies B,B\implies C\}$ is that component in which $a$ is a $P$-move.  
  Then this $c$ is preceded by some other $O$-move $b'$, which is necessarily also contained in $\pv{\s}$, and so has some justifier $c'$, contained in $\pv{\s\vert_X}$ by visibility.  
  Continuing in this way until we reach an initial move, we build up the whole of the sequence $\pv{\s\vert_X}$ as a subsequence of $\pv{\s}$.  
  Therefore, the justifier of $a$ must be contained in $\pv{\s}$, and so $\pv{\s a}=\pv{\s}a\in\Int(A,B,C)$.
\end{proof}

\begin{lemma}[$O$-views in the linear implication, {\cite[4.2,4.3]{hoPcf}}]
  Let $A,B$ be games, and let $bs$ be a non-empty play in $A\implies B$ beginning with an initial move $b$ in $B$.

  i) If $bs$ ends with a $P$-move in $B$, then $\ov{bs}_{A\implies B}=\ov{bs\vert_B}_B$.

  ii) If $bs$ ends with a $P$-move in $A$, then $\ov{bs}_{A\implies B} = b\pv{s\vert_A}^A$.
  \label{LemProjectionLemma}
\end{lemma}
\begin{proof}
  Induction on the length of $s$.
  If $s=\epsilon$, then $bs$ ends with an $O$-move in $B$, and we have $\pv{b}^{A\implies B} = b = \pv{b}^{B}$.  

  Otherwise, suppose that $bs$ ends with a $P$-move $c$ in $B$.  
  Let $d$ be the justifier of $c$.  
  Then $d$ must be an $O$-move in $B$.
  Write $bs=tduc$, where $t,u$ are sequences.  
  Then $\ov{tduc}_{A\implies B}=\ov{t}_{A\implies B}dc$ and $\ov{tduc\vert_B}_B=\ov{t\vert_B}_Bdc$.
  By Corollary \ref{CorSwitchingCondition}, $t$ must end with a $P$-move in $B$, or be empty, so by the inductive hypothesis we have $\ov{t}_{A\implies B}=\ov{t\vert_B}_B$.
  Therefore, $\ov{bs}_{A\implies B}=\ov{tduc}_{A\implies B}=\ov{t}_Bdc = \ov{tduc\vert_B}_B=\ov{bs\vert_B}_B$.

  Next, suppose that $bs$ ends with a $P$-move $a$ in $A$.
  Let $c$ be the justifier of $a$.  
  Then $c$ must be an $O$-move in $A$.  
  Write $s=tcua$, where $t,u$ are sequences.  
  Then $\ov{btcua}_{A\implies B}=\ov{bt}_{A\implies B}ca$ and $\pv{tcua\vert_A}^A = \pv{t\vert_A}^Aca$, since the roles are reversed in $A$.
  By Corollary \ref{CorSwitchingCondition}, $t$ must end in a $P$-move in $A$, or be empty, so by the inductive hypothesis we have $\ov{bt}_{A\implies B} = b\pv{t\vert_A}^A$.  
  Therefore, $\ov{bs}_{A\implies B}=\ov{btcua}_{A\implies B}=\ov{bt}_{A\implies B}ca=b\pv{t\vert_A}^Aca=b\pv{tcua\vert_A}^A=b\pv{s\vert_A}^A$.  
\end{proof}

\begin{proposition}
  $\sigma;\tau$ is a strategy for $A\implies C$.  
\end{proposition}
\begin{proof}
  First, we claim that $\s\vert_{A,C}\in P_{A\implies B}$ for any $\s\in \sigma\|\tau$.
  Since we certainly have $\s\vert_{A,C}\vert_A = \s\vert_{A,B}\vert_A\in P_A$ and $\s\vert_{A,C}\vert_C = \s\vert_C = \s\vert_{B,C}\vert_C\in P_C$, it suffices to show that $\s\vert_{A,C}\in L_{A\implies C}$.

  Suppose that $ta\prefix \s\vert_{A,C}$.  We claim that $\lambda_{A\implies C}(t) = \neg\lambda_{A\implies C}(a)$.
  By Lemma \ref{LemCompositionLemma}, we are in one of the following configurations.
  \small
  \[
    \begin{array}{ccc|ccc}
      \lambda_A^{OP}(t\vert_A) & \lambda_B^{OP}(t\vert_B) & \lambda_C^{OP}(t\vert_C) & \lambda_{A\implies B}^{OP}(t\vert_{A,B}) & \lambda_{B\implies C}^{OP}(t\vert_{B,C}) & \lambda_{A\implies C}^{OP}(t\vert_{A,C}) \\
      \hline
      P & P & P & P & P & P \\
      P & P & O & P & O & O \\
      P & O & O & O & P & O \\
      O & O & O & P & P & P 
    \end{array}
    \]
  \normalsize
  In the configuration $PPP$, the move $a$ cannot be a move in $A$, since that would leave $ta\vert_{A\implies B}$ in the configuration $OP$, which is impossible by Lemma \ref{LemCompositionLemma}.  
  Therefore, it must be a move in $C$, and must therefore be an $O$-move in $C$ and hence an $O$-move in $A\implies C$.

  In the configuration $PPO$, once again the move $a$ cannot take place in $A$, since this would leave $ta\vert_{A\implies B}$ in an illegal configuration.  
  Therefore, it must occur in $C$, and must be a $P$-move in $C$ and hence a $P$-move in $A\implies C$.

  In the configuration $POO$, the move $a$ cannot take place in $C$, or it would leave $ta\vert_{B,C}$ in the illegal configuration $OP$, so the move $a$ takes place in $A$.  
  Therefore, it must be an $O$-move in $A$ and hence a $P$-move in $A\implies C$.  

  Lastly, in the configuration $OOO$, the move $a$ cannot occur in $C$, or it would leave $ta\vert_{B,C}$ in the configuration $OP$, and so it must take place in $A$.  
  Therefore, it must be a $P$-move in $A$, and hence an $O$-move in $A\implies C$.

  Having established that $\s\vert_{A,C}$ is alternating, we now show that it is well-bracketed.
  Suppose that a question move $q$ in $\s\vert_{A,C}$ justifies some answer move $a$.
  $q$ and $a$ must occur in the same component, since the only case in which a move from one of $A$ and $C$ can justify a move in the other is when both moves are initial, and hence questions.
  Suppose first that $q$ and $a$ both occur in the game $C$.  
  Suppose that some other question move $q'$ occurs between $q$ and $a$ in $\s\vert_{A,C}$.  
  If $q'$ occurs in $C$, then it must be answered by some $a'$ occurring between $q'$ and $a$, since $\s\vert_C$ is a well-bracketed sequence.  
  Otherwise, suppose that $q'$ occurs in $A$.  

  By examining the table above, we see that there must be some move in $B$ occurring between $q$ and $q'$ in $\s$, since moves in $A$ move between configurations $OOO$ and $POO$, while moves in $C$ move us between configurations $PPP$ and $PPO$.
  Let $b$ be the earliest such move.  
  Then $b$ must be a question; indeed, if it is an answer, then it is non-initial and so can only be justified by questions in $B$.  
  But such a question must occur earlier in $\s\vert_{B,C}$ than $q$, which would mean that $q$ was an unanswered question when the move $b$ was played, contradicting well-bracketedness of $\s\vert_{B,C}$.  
  Since $b$ is a question, it must be answered by some $a''$ occurring between $b$ and $a$.  
  Therefore, since $\s\vert_{A,B}$ is well-bracketed, the move $q'$ must be answered by some $a'$ occurring between $a'$ and $a''$ in $\s\vert_{A,B}$, and therefore between $a'$ and $a$ in $\s\vert_{A,C}$.

  The case when $q$ and $a$ both occur in $A$ is similar.  

  Lastly, we need to show that $\s\vert_{A,C}$ satisfies the visibility condition.
  Let $ta\prefix\s\vert_{A,C}$.
  Choose some $\t\prefix\s$ such that $\t\vert_{A,C}=t$.  

  Suppose $a$ is a $P$-move.
  Then by Lemma \ref{LemHarmersLemma}, $\pv{\t a}\in\Int(A,B,C)$.  
  By Lemma \ref{LemHarmerRestriction}, $\pv{t}a=\pv{ta}=\pv{\t a}\vert_{A,C}$, and therefore that the justifier of $a$ must be inside $\pv{t}$.

  Secondly, suppose that $a$ is an $O$-move.
  If $a$ is an $O$-move in $C$, then either it is initial or $t$ ends with some $P$-move in $B$, and therefore $\ov{t}_{A\implies C}=\ov{t\vert_B}_B = \ov{\t\vert_{B,C}}_B$.  
  Therefore, since $t\vert_{B,C}$ satisfies visibility, the justifier of $a$ must lie in $\ov{t}_{A\implies C}$.  
  If $a$ is an $O$-move in $A$, then write $t=cu$ and $\t=c\u$, where $c$ is the starting move in $C$.
  We have $\ov{cua}_{A\implies C}=c\pv{u\vert_A a}^A = \ov{c\u\vert_{A,B}}_{A\implies B}$.  
  Therefore, the justifier of $a$ must lie in $\ov{t}_{A\implies C}$.

  Therefore, $\s\vert_{A,C}\in L_{A\implies C}$, so $\s\vert_{A,C}\in P_{A\implies C}$.

  It is fairly clear that $\sigma;\tau$ is even-prefix closed, since $\sigma$ and $\tau$ are.  
  Indeed, if $\s\vert_{A,C}\in\sigma;\tau$ and $t\prefix\s\vert_{A,C}$, then we may choose some prefix $\t$ of $\s$ such that $t=\t\vert_{A,C}$.  
  Then $\t\vert_{A,B}\prefix\s\vert_{A,B}\in\sigma$ and $\t\vert_{B,C}\prefix\s\vert_{B,C}\in\tau$, so $\t\in\sigma\|\tau$.

  We claim that every sequence in $\sigma;\tau$ has even length.  
  Indeed, if $\s\vert_{A,B}\in\sigma$ and $\s\vert_{B,C}\in\tau$, then both $\s\vert_{A,B}$ and $\s\vert_{B,C}$ must have even length, so must be in configuration $OO$ or $PP$.  
  This means that $\s$ as a whole must be in configuration $OOO$ or $PPP$, and so $\s\vert_{A,C}$ must be in configuration $OO$ or $PP$, so must have even length.

  Lastly, we need to show that $\sigma;\tau$ is deterministic.  
  Suppose that $sab,sac\in\sigma;\tau$, and suppose that $b\ne c$.  
  Suppose that $\s\vert_{A,C}=sab$ and $\t\vert_{A,C}=sac$, for $\s,\t\in\sigma\|\tau$, and let $\u$ be the longest common prefix of $\s,\t$.
  $\s$ and $\t$ are certainly incomparable under the prefix ordering, since $\s\vert_{A,C}$ and $\t\vert_{A,C}$ are, so we have $\u p\prefix \s$ and $\u q\prefix \t$, where $p\ne q$.
  Now $p$ and $q$ cannot be $O$-moves in $A$, $P$-moves in $C$ or moves in $B$, or they would have to be equal by determinism of $\sigma$ and $\tau$.  
  Therefore, they are $P$-moves in $A$ or $O$-moves in $C$, but this contradicts $\s\vert_{A,C}=sab$ and $\t\vert_{A,C}=sac$.

  Therefore, the composition $\sigma;\tau$ is a strategy.  
\end{proof}

We also want to show that the composition of innocent strategies is innocent.
We follow the proof given in \cite{Harmer2006InnocentGS}.  
First, we use a lemma.

\begin{lemma}[{\cite[3.3.3]{Harmer2006InnocentGS}}]
  Let $\s a\in\Int(A,B,C)$.  

  i) If $a$ is a $P$-move of $A$ or an $O$-move of $B$, then $\pv{\s a\vert_{A,B}}=\pv{\pv{\s a}\vert_{A,B}}$.

  ii) If $a$ is a $P$-move of $B$ or an $O$-move of $C$, then $\pv{\s a\vert_{B,C}}=\pv{\pv{\s a}\vert_{B,C}}$.
  \label{LemHarmerProjection}
\end{lemma}
\begin{proof}
  Induction on the length of $\s$.
  We prove (i); the proof of (ii) is exactly the same.  

  If $a$ is a $P$-move of $A$ or an $O$-move of $B$, then it is an $O$-move of $A\implies B$.
  If $a$ is an initial move of $A\implies B$, then we have $\pv{\s\vert_{A,B}a}=a=\pv{a}\vert_{A,B}=\pv{\pv{\s a}}\vert_{A,B}$.  
  Otherwise, write $\s=\t b\u$, where $b$ justifies $a$.  
  Then $\pv{\s a \vert_{A,B}} = \pv{\t\vert_{A,B} b\u\vert_{A,B} a} = \pv{t\vert_{A,B}}ba$, which by the inductive hypothesis is equal to $\pv{\pv{\t}\vert_{A,B}}ba$, which is equal to $\pv{\pv{\t b \u a}\vert_{A,B}} = \pv{\pv{\s a}\vert_{A,B}}$.
\end{proof}

\begin{proposition}
  If $\sigma\from A \implies B$ and $\tau\from B \implies C$ are innocent strategies, then $\sigma;\tau\from A\implies C$ is innocent.
\end{proposition}
\begin{proof}
  Suppose there are $sab,t\in\sigma;\tau$ such that $ta\in P_{A\implies C}$, $\pv{sa}=\pv{ta}$.
  Let $\s' b$ be such that $\s' b\vert_{A,C}=sab$ and choose the minimal prefix $\s\prefix \s'$ such that $\s a\vert_{A,C}=sa$.
  
  Let $\t a$ be such that $\t a\vert_{A,C}=ta$.
  Since $\pv{sa}=\pv{ta}$, we have $\pv{\s a}\vert_{A,C}=\pv{\s a\vert_{A,C}}=\pv{sa}=\pv{ta}=\pv{\t a\vert_{A,C}}=\pv{\t a}\vert_{A,C}$ by Lemma \ref{LemHarmerRestriction}.  
  Let $\u$ be the longest common prefix of $\pv{\s a}$ and $\pv{\t a}$.  
  If $\s a$ and $\t a$ are not equal, then without loss of generality there is some $\u p\prefix \s$, where $\u p\not\prefix \t$.  
  Then, by determinism of $\sigma$ and $\tau$, this $p$ cannot be a $P$-move in either $A\implies B$ or $B\implies C$, so it must be a $P$-move in $A$ or an $O$-move in $C$, and is therefore preceded by another move in $A$ or $C$, which contradicts $\pv{\s a}\vert_{A,C}=\pv{\t a}\vert_{A,C}$.  
  Therefore, $\pv{\s a}=\pv{\t a}$.

  Now write $\s'=\s a b_1 \cdots b_n b$, where each $b_i$ is a move in $B$.  
  We show by induction that $\t a b_1 \cdots b_j\in\sigma\|\tau$.  
  Indeed, if $\t a b_1 \cdots b_{j-1}\in\sigma\|\tau$, then $b_j$ (or $b$) is a $P$-move in either $A\implies B$ or $B\implies C$, and $b_{j-1}$ is an $O$-move in that same component.  
  Write $X$ for the component ($A\implies B$ or $B\implies C$) in which $b_j$ is a $P$-move.
  Repeating the argument above, we see that $\pv{\t a b_1\cdots b_{j-1}} = \pv{\s a b_1\cdots b_{j-1}}$, and so we have that $\pv{\t a b_1\cdots b_{j-1}\vert_X}=\pv{\s a b_1\cdots b_{j-1}\vert_X}$ by Lemma \ref{LemHarmerProjection}.  
  Therefore, by innocence of $\sigma$ (if $X=A\implies B$) or $\tau$ (if $X=B\implies C$), we see that $\t a b_1 \cdots b_j\in \sigma\|\tau$.  
  It follows that $\t a b_1 \cdots b_n b\in\sigma\|\tau$, and therefore that $t a b\in\sigma;\tau$.
\end{proof}

\section{Associativity of composition}

In this section, we shall prove that composition of strategies is associative; i.e., that if $\sigma\from A \implies B$, $\tau\from B\implies C$ and $\upsilon \from C \implies D$ are strategies, then $(\sigma;\tau);\upsilon=\sigma;(\tau;\upsilon)$.  
To do this, if $A,B,C,D$ are arenas, we define the set $\Int(A,B,C,D)$ to be the set of all sequences $\u$ of moves such that $\u\vert_{A,B}\in L_{A\implies B}$, $\u\vert_{B,C}\in L_{B\implies C}$ and $\u\vert_{C,D}\in L_{C \implies D}$.  
Given such a sequence $\u$, we define $\u\vert_{A,D}$ as before; i.e., we take all moves from $\u$ occurring in $A$ and $D$, together with justification pointers within these games, and if an initial move in $A$ is justified by an initial move in $B$, which is justified by an initial move in $C$, which is justified by an initial move in $D$, then we add a justification pointer from that move in $A$ to that move in $D$.

Given strategies $\sigma,\tau,\upsilon$ as above, we define $\sigma\|\tau\|\upsilon$ to be the set of all $\u\in\Int(A,B,C,D)$ such that $\u\vert_{A,B}\in\sigma$, $\u\vert_{B,C}\in\tau$ and $\u\vert_{C,D}\in\upsilon$.
We then claim that:

\begin{lemma}
  \[
    (\sigma;\tau);\upsilon = \{\u\vert_{A,C}\suchthat \u\in\sigma\|\tau\|\upsilon\} = \sigma;(\tau;\upsilon)\,.
    \]
\end{lemma}
\begin{proof}
  Firstly, if $\u\in\sigma\|\tau\|\upsilon$, then it is clear to see that $\u\vert_{A,B,C}\in\sigma\|\tau$ and that $\u\vert_{B,C,D}\in\tau\|\upsilon$, and therefore that $\{\u\vert_{A,C}\suchthat \u\in\sigma\|\tau\|\upsilon\}\subset (\sigma;\tau);\upsilon$ and $\{\u\vert_{A,C}\suchthat \u\in\sigma\|\tau\|\upsilon\}\subset \sigma;(\tau;\upsilon)$. 

  Conversely, suppose that $\t\in(\sigma;\tau)\|\upsilon$, so that $\t\vert_{A,C}\in\sigma;\tau$ and $\t\vert_{C,D}\in\upsilon$, and choose some $\s\in\sigma\|\tau$ such that $\s\vert_{A,C}=\t\vert_{A,C}$.
  We may write 
  \[
    \s=\ccc_1\bbb_1\aaa_1\cdots \ccc_n\bbb_n\aaa_n
    \]
  for some (possibly empty) sequences of moves $\aaa_i$ from $A$, $\bbb_i$ from $B$ and $\ccc_i$ from $C$.  
  We may then write 
  \[
    \t=\ddd_1\ccc_1\aaa_1\cdots\ddd_n\ccc_n\aaa_n
    \]
  (for the same $\aaa_i$, $\ccc_i$), and we can therefore interleave these sequences into the sequence
  \[
    \u = \ddd_1\ccc_1\bbb_1\aaa_1\cdots\ddd_n\ccc_n\bbb_n\aaa_n\,,
    \]
  which is in $\sigma\|\tau\|\upsilon$.
  Then we have $\u\vert_{A,D}=\t\vert_{A,D}$, and it follows that $(\sigma;\tau);\upsilon\subset\{\u\vert_{A,C}\suchthat \u\in\sigma\|\tau\|\upsilon\}$, and the case for $\sigma;(\tau;\upsilon)$ is identical.
\end{proof}

\section{Copycat strategies}
\label{SecCopycat}

\begin{definition}
  Let $A,B$ be games.  
  Then a \emph{subset inclusion} of $A$ into $B$ is a partial injection $i\from M_A\hookrightarrow M_B$ such that
  \begin{itemize}
    \item if $i$ is defined at $a$ and $b$ then $*\ts_A a$ if and only if $*\ts_B i(a)$, and $a\ts_A b$ if and only if $i(a)\ts_B i(b)$;
    \item $i(a)$ is defined for every move $a$ occurring in a play in $P_A$; and
    \item $i_*(s)\in P_B$ for every $s\in P_A$.
  \end{itemize}
  Here, $i_*(s)$ means the function $i$ applied pointwise to the elements of the string $s$.

  If $i$ is a subset incusion of $A$ into $B$, then we get an innocent strategy $\subs_i\from B \implies A$ defined by
  \[
    \subs_i = \{s\in P_{B\implies A}\suchthat\text{for all even-length $t\prefix s$, $t\vert_B=i_*(t\vert_A)$}\}\,.
    \]
  If $P_B=\{i_*(s)\suchthat s\in P_A\}$, then we call it a \emph{structural isomorphism}, and we write $\cc_i$ (`copycat') for $\subs_i$.
\end{definition}

\begin{proposition}
  $\subs_i$ is an innocent strategy.

  Moreover, if $\sigma\from C \implies B$ is a strategy, then
  \[
    \sigma;\subs_i = \{[\id_{M_C},i\inv]_*(s)\suchthat s\in\sigma,\,s\vert_B\in i_*(P_A)\}\,,
    \]
  where $i\inv\from M_B \pfun M_A$ is the canonical partial right-inverse to $i$, and if $\tau \from A \implies D$ is a strategy, then
  \[
    \subs_i;\tau = \{[i,\id_{M_D}]_*(s)\suchthat s\in\tau\}\,.
    \]
  \label{PropCopycat}
\end{proposition}
\begin{proof}
  $\subs_i$ is clearly prefix-closed by definition.  
  Suppose that $sab,sac\in\subs_i$; then $s\vert_A=i_*(s\vert_B)$ and $sab\vert_A=i_*(sab\vert_B)$.  
  It follows that $ab\vert_A=i_*(ab\vert_B)$, so either $a$ is a move in $A$ and $b=i(a)$ or $a$ is a move in $B$ and $a=i(b)$.  
  Since the same applies to $c$, and since $i$ is injective, we have $b=c$.

  This argument also shows that $\subs_i$ is \emph{history-free} -- i.e., that its reply to an $O$-position is entirely determined by the last $O$-move -- and therefore it is certainly innocent.

  Now let $\sigma \from C \implies B$ be a strategy.  
  Suppose that $\s\in\sigma\|\subs_i$.  
  Then $\s\vert_{C,B}\in\sigma$ and $\s\vert_B=i_*(\s\vert_A)$; i.e., $\s\vert_{C,B}=[\id_{M_C},i]_*(\s\vert_{C,A})$, and therefore $\s\vert_{C,A}=[\id_{M_C},i]_*(\s\vert_{C,B})$, where $\s\vert_B\in i_*(P_A)$.

  Conversely, given $s\in\sigma$, where $s\vert_B\in i_*(P_A)$, for each $P$-move $b=i(a)$ in $s$ occurring in the component $B$, insert the move $a$ immediately after it, and for each $O$-move $b'=i(a')$ in $s$ occurring in the component $B$, insert the move $a'$ immediately before it.
  Let these extra moves in $B$ be justified according to the original moves in $A$, and let all initial moves in $B$ be justified by the initial moves in $A$ that occur immediately before them.
  Then the resulting sequence $\s$ is contained in $\sigma\|\subs_i$, and $\s\vert_{A,C}=[\id_{M_C},i\inv]_*(s)$.

  The case for composition in the other direction is similar.
\end{proof}

An easy corollary of this fact is that composition of copycat strategies respects composition of the underlying subset inclusions.

\begin{corollary}
  Let $i$ be a subset inclusion from $A$ to $B$ and let $j$ be a subset inclusion from $B$ to $C$.  
  Then $j\circ i$ is a subset inclusion from $A$ to $C$ and $\cc_{j\circ i}=\subs_j;\subs_i\from C \implies A$.
\end{corollary}

It is also easy to see from Proposition \ref{PropCopycat} that the identity function $\id\from M_A \to M_A$ is a structural isomorphism from $A$ to itself, and that the resulting copycat strategy $\cc_{\id}$ is an identity for composition.  
Combining this with our result for associativity in the previous chapter, we get that
\begin{theorem}
  The collection of games forms a category $\G$, where the morphisms $A \to B$ are strategies for $A\implies B$, composition is as above and the identity morphisms are the copycat strategies induced from the identity functions on moves.
\end{theorem}

In this setting, Proposition \ref{PropCopycat} tells us that a structural isomorphism gives rise to an isomorphism in $\G$.
\begin{proposition}
  Let $f$ be a structural isomorphism from a game $A$ to a game $B$.  
  Then $\cc_f$ is an isomorphim in $\G$ from $A$ to $B$.
\end{proposition}
\begin{proof}
  The underlying partial injection $f\from M_A \hookrightarrow M_B$ has an inverse partial injection $f\inv\from M_B \to M_A$, inducing a structural isomorphism from $B$ to $A$.  
  Then Proposition \ref{PropCopycat} tells us that $\cc_f$ and $\cc_{f\inv}$ are inverses in $\G$.
\end{proof}

General subset inclusions are not, of course, isomorphisms, but we can still say something category-theoretic about them.

\begin{proposition}
  Let $i$ be a subset inclusion from a game $A$ to a game $B$.  
  Then the strategy $\subs_i$ is an epimorphism from $B$ to $A$.
  \label{PropSubsetInclusionEpic}
\end{proposition}
\begin{proof}
  In fact, it is a split epimorphism: we can define a retract
  \[
    \ret_i = \{s\in P_{A\implies B}\suchthat\text{for all even-length $t\prefix s$, $t\vert_A=i_*(t\vert_B)$}\}\,.
    \]
  The same argument as in Proposition \ref{PropCopycat} tells us that this is indeed a strategy for $A\implies B$.
  Note that although $\subs_i$ is always a total strategy (i.e., if $s\in\subs_i$ and $sa\in P_{B\implies A}$, then there is always $sab\in\subs_i$ for some $b$), the same is not in general true about $\ret_i$.  

  In any case, if $\s\in\ret_i\|\subs_i$, then $\s\vert_{A^L} = i_*(\s\vert_B) = \s\vert_{A^R}$, and the same is true of any even-length substring of $\s$, and so $\s\vert_{A,A}\in\id_A$.  
  Conversely, given any $s\in \id_A$, we can form some $\s\in\ret_i\|\subs_i$ such that $\s\vert_{A,A}=s$ as in Proposition \ref{PropCopycat}.

  We can also prove that $\subs_i$ is an epimorphism directly, which might be useful, for example, in a setting in which non-total strategies such as $\ret_i$ are disallowed.  
  In this setting, let $\sigma,\tau\from A \implies C$ be strategies such that $\subs_i;\sigma=\subs_i;\tau$.  
  Then, by Proposition \ref{PropCopycat}, we know that
  \[
    \{[i,\id_{M_C}]_*(s)\suchthat s\in\sigma\}
    =
    \{[i,\id_{M_C}]_*(s)\suchthat s\in\tau\}\,.
    \]
  Then, since the function $[i,\id_{M_D}]_* \from P_{A\implies C} \to P_{B\implies C}$ is an injection, we deduce that $\sigma=\tau$.
\end{proof}

\section{$\G$ as a Symmetric Monoidal Category}

We now claim that the tensor product connective $\tensor$ makes $\G$ into a symmetric monoidal closed category, with internal hom given by $\implies$.  

\begin{definition}
  Let $\sigma\from A \implies B$ and $\tau\from C \implies D$ be strategies.  
  We define a strategy $\sigma\tensor\tau\from (A\tensor C) \implies (B \tensor D)$ by
  \[
    \sigma\tensor\tau = \{s\in P_{(A\tensor C)\implies (B\tensor D)}\suchthat s\vert_{A,B}\in\sigma\text{ and }s\vert_{C,D}\in\tau\}\,.
    \]
\end{definition}

To prove that this is a strategy, we prove a lemma analogous to our Lemma \ref{LemCompositionLemma}.  

\begin{lemma}
  Let $s\in P_{A\tensor B}$.  
  Then $\lambda^{OP}_{A\tensor B}(s)=\lambda^{OP}_A(s\vert_A)\wedge\lambda^{OP}_B(s\vert_B)$, where $\wedge$ is the binary operator on $\OP$ given by
  \[
    \begin{array}{cc|c}
      p & q & p\wedge q \\
      \hline
      P & P & P \\
      O & P & O \\
      P & O & O \\
      O & O & O
    \end{array}\,.
    \]
  Moreover, either $\lambda_A^{OP}(s\vert_A)=P$ or $\lambda_B^{OP}(s\vert_B)=P$.
  \label{LemTensorAnalogue}
\end{lemma}
\begin{proof}
  Mutual induction on the length of $s$.  
  This is obvious if $s$ is empty.  
  Suppose that $sa\in P_{A\tensor B}$, where $a$ is an $O$-move.  
  By induction, since $\lambda_{A\tensor B}(s)=P$, we must have $\lambda_{A\tensor B}(s\vert_A)=P$ and $\lambda_{A\tensor B}(s\vert_B)=P$.  
  Therefore, depending on which game $a$ is played in, either $\lambda_A(sa\vert_A)=O$ and $\lambda_B(sa\vert_B)=P$ or $\lambda_A(sa\vert_A)=P$ and $\lambda_B(sa\vert_B)=O$.  

  If $sb\in P_{A\tensor B}$, where $b$ is a $P$-move, then by induction either $\lambda_A(s\vert_A)=O$ and $\lambda_B(s\vert_B)=P$ or $\lambda_A(s\vert_A)=P$ and $\lambda_B(s\vert_B)=O$.  
  In either case, player $P$ must play in whichever game is currently in an $O$-position, returning us to configuration $PP$.
\end{proof}

The above proof gives us the following result, which is analogous to Corollary \ref{CorSwitchingCondition}.

\begin{corollary}[Switching condition for {$\tensor$}]
  Player $O$ switches games in $A\tensor B$; i.e., if $sab\in P_{A\tensor B}$, where $a$ and $b$ take place in different games (i.e., $a$ in $A$ and $b$ in $B$ or $a$ in $B$ and $b$ in $A$), then $b$ is an $O$-move.
\end{corollary}

\begin{proposition}
  $\sigma\tensor\tau$ is a strategy for $(A\tensor C)\implies (B\tensor D)$.
  \label{PropTensorWellDefined}
\end{proposition}
\begin{proof}
  $\sigma\tensor\tau$ is certainly an even-prefix-closed subset of $P_{(A\tensor C)\implies (B\tensor D)}^{\textit{even}}$.  

  Let $s$ be a play of $P_{(A\tensor B)\implies (C\tensor D)}$.  
  We consider the possible configurations of $s$; i.e., the tuples $(\lambda_A(s\vert_A),\lambda_B(s\vert_B),\lambda_C(s\vert_C),\lambda_D(s\vert_D))$.

  By Lemma \ref{LemCompositionLemma} we must avoid the overall configuration $OP$ for the linear implication, and by Lemma \ref{LemTensorAnalogue} we must avoid the configuration $OO$ inside either tensor product, so we end up with the following possibilities.
  \scriptsize
  \[
\arraycolsep=4.8pt\def\arraystretch{1.5}
    \begin{array}{cccc|cc|c}
      \lambda_A(s\vert_A) & \lambda_C(s\vert_C) & \lambda_B(s\vert_B) & \lambda_D(s\vert_D) & \lambda_{A\tensor C}(s\vert_{A,C}) & \lambda_{B\tensor D}(s\vert_{B,C}) & \lambda_{(A\tensor C)\implies (B\tensor D)}(s) \\
      \hline
      P & P & P & P & P & P & P \\
      P & P & P & O & P & O & O \\
      P & P & O & P & P & O & O \\
      P & O & P & O & O & O & P \\
      O & P & O & P & O & O & P \\
      P & O & O & P & O & O & P \\
      O & P & P & O & O & O & P \\
    \end{array}
    \]
  \normalsize
  Now, if $s\in\sigma\tensor\tau$, or an odd-length sequence formed by adding an $O$-move to the end of a sequence in $\sigma\tensor\tau$, then we also know that $s\vert_{A,B}\in\sigma\subset P_{A\implies B}$ and that $s\vert_{C,D}\in\tau\subset P_{C\implies D}$.  
  This means that we can discount the last two configurations in the table above, since one contains the illegal configuration $OP$ in $C\implies D$ and the other contains the illegal configuration $OP$ in $A \implies B$.

  Now suppose that $sab,sac\in\sigma\tensor\tau$.  
  Then $sa$ is an $O$-position in $(A\tensor C)\implies (B\tensor D)$, and is therefore in configuration $PPPO$ or $PPOP$.  
  By inspecting the table above, we see that if $sa$ is in configuration $PPPO$, then $b$ and $c$ must both occur either in $C$ or in $D$, and that if $sa$ is in configuration $PPOP$, then $b$ and $c$ must both occur either in $A$ or in $B$.  
  In either case, we must have $b=c$, by determinism of $\tau$ (in the first case) or of $\sigma$ (in the second case).
\end{proof}

We need a lemma to prove that the tensor product of two innocent strategies is innocent.

\begin{lemma}
  Let $s\in\sigma\tensor\tau$.  

  i) If $s$ ends with a move in $A$ or $B$, then $\pv{s}^{(A\tensor C)\implies (B\tensor D)}=\pv{s\vert_{A,B}}^{A\implies B}$.

  ii) If $s$ ends with a move in $C$ or $D$, then $\pv{s}^{(A\tensor C)\implies (B\tensor D)}=\pv{s\vert_{C,D}}^{C\implies D}$.
  \label{LemTensorViewLemma}
\end{lemma}
\begin{proof}
  Induction on the length of $s$.  
  We prove (i); (ii) is exactly the same.

  If $a$ is a $P$-move, then we have $\pv{sa}=\pv{s}a$.  
  By our analysis in the proof of Proposition \ref{PropTensorWellDefined}, player $P$ only switches moves between $A$ and $B$, and between $C$ and $D$, so $s$ must end with a move from $A$ or $B$.  
  Therefore, by the inductive hypothesis, $\pv{s}=\pv{s\vert_{A,B}}$.  
  Then $\pv{sa}=\pv{s}a=\pv{s\vert_{A,B}}a=\pv{sa\vert_{A,B}}$.  

  If $a$ is an initial move, then $\pv{sa}=a=\pv{sa\vert_{A,B}}$.

  If $a$ is an $O$-move justified by $b$ in $sbta$, then $\pv{sbta}=\pv{s}ba$.  
  Then $b$ is a $P$-move, so $s$ must end with a move in $A$ or $B$, as before.  
  Therefore, by the inductive hypothesis, $\pv{s}=\pv{s\vert_{A,B}}$.  
  Then $\pv{sbta}=\pv{s}ba=\pv{s\vert_{A,B}}ba=\pv{sbta\vert_{A,B}}$.
\end{proof}

\begin{proposition}
  Let $\sigma\from A \to B$, $\tau\from C \to D$ be innocent strategies.  
  Then $\sigma\tensor\tau$ is innocent.
  \label{PropTensorInnocent}
\end{proposition}
\begin{proof}
  Suppose $sab,t\in\sigma\tensor\tau$ such that $ta\in P_{(A\tensor C)\implies (B\tensor D)}$ and $\pv{sa}=\pv{ta}$.  
  Suppose without loss of generality that $a$ is a move in $A$ or $B$.  
  Then $\pv{sa}=\pv{sa\vert_{A,B}}$ and $\pv{ta}=\pv{ta\vert_{A,B}}$ by Lemma \ref{LemTensorViewLemma}, and therefore $tab\vert_{A,B}\in\sigma$ by innocence of $\sigma$, and so $tab\in\sigma\tensor\tau$.  
\end{proof}

The most important thing we need to prove is that $\tensor$ is a functor.

\begin{proposition}
  Let $\sigma'\from A'' \implies A'$, $\sigma\from A' \implies A$, $\tau'\from B'' \implies B'$ and $\tau \from B' \implies B$ be strategies.  
  Then $(\sigma'\tensor\tau');(\sigma\tensor\tau)=(\sigma';\sigma)\tensor(\tau';\tau)$.

  Moreover, if $A',A,B',B$ are games, $i$ is a subset inclusion from $A$ to $A'$ and $j$ is a structural isomorphism from $B$ to $B'$, then $\subs_i\tensor\subs_j = \subs_{[i,j]}$.
  In particular, if $A$ and $B$ are games, then $\id_A\tensor \id_B=\id_{A\tensor B}$.
  \label{PropTensorProductIsFunctor}
\end{proposition}
\begin{proof}
  First suppose that $s\in(\sigma'\tensor\tau');(\sigma\tensor\tau)$; so $s=\s\vert_{A'',B'',A,B}$, where $\s\in(\sigma'\tensor\tau')\|(\sigma\tensor\tau)$.  
  Then $\s\vert_{A'',A'}\in\sigma'$ and $\s\vert_{A',A}\in\sigma$, so $\s\vert_{A'',A',A}\in\sigma'\|\sigma$ and therefore $s\vert_{A'',A}=\s\vert_{A'',A}\in\sigma';\sigma$.  
  Similarly, $s\vert_{B'',B}\in\tau';\tau$, and therefore $s\in(\sigma';\sigma)\tensor(\tau';\tau)$.

  Conversely, suppose that $s\in(\sigma';\sigma)\tensor (\tau';\tau)$.  
  Choose some $\s\in\sigma'\|\sigma$, $\t\in\tau'\|\tau$ such that $s\vert_{A'',A}=\s\vert_{A'',A}$ and $s\vert_{B'',B}=\s\vert_{B'',B}$.  
  By our analysis, the only time we switch from the $A'',A$-component to the $B'',B$ component in $s$, or \emph{vice versa}, is when player $O$ switches between the games $A$ and $B$.  
  Thus, we may divide $s$ up into blocks, each starting and ending with a move in the outer component $A \tensor B$.  
  This then gives us a way to divide up $\s$ and $\t$ into blocks, such that each block of $\s$ or $\t$ projects on to a block of $s$.  
  Lastly, we can string these blocks together to give us some $\u\in(\sigma'\tensor\tau')\|(\sigma\tensor\tau)$ such that $\u\vert_{A'',B'',A,B}=s$.

  For the second part, let $A',A,B',B$ be games, let $i$ be a structural isomorphism from $A$ to $A'$ and let $j$ be a structural isomorphism from $B$ to $B'$.  
  Suppose that $s\in\subs_i\tensor \subs_j$.  
  Then $s\vert_{A',A}\in\subs_i$ and $s\vert_{B',B}\in\subs_j$ -- so if $u\prefix s\vert_{A,A}$ has even length, then $u\vert_{A'}=i_*(u\vert_A)$, and if $v\prefix s\vert_{B,B}$ has even length, then $v\vert_{B'}=i_*(v\vert_B)$.  
  Suppose that $t\prefix s$ is of even length.  
  Then, since only player $O$ swtiches between the $A',A$-component and the $B',B$-component, both $t\vert_{A',A}$ and $t\vert_{B',B}$ are of even length, it follows that $t\vert_{A',B'}=[i,j]_*(t\vert_{A,B})$.  
  Since $t$ was arbitrary, this means that $s\in \subs_{[i,j]}$.

  Conversely, suppose that $s\in\subs_{[i,j]}$.  
  Then for all even-length $t\prefix s$, $t\vert_{A'}=i_*(t\vert_{A})$ and $t\vert_{B'}=j_*(t\vert_{B})$.  
  Since any play in $\sigma$ or in $\tau$ is itself a play of $\sigma\tensor\tau$, then if $u\prefix s\vert_{A',A}$ has even length, then $u\vert_{A'}=i_*(u\vert_{A})$, and if $v\prefix s\vert_{B',B}$, then $v\vert_{B'}=j_*(v\vert_{B})$.  
  It follows that $s\vert_{A',A}\in \subs_i$ and $s\vert_{B',B}\in\subs_j$, and therefore that $s\in \subs_i\tensor\subs_j$.  
\end{proof}

Now it is fairly clear that if $A,B,C$ are games, then we have structural isomorphisms
\begin{mathpar}
  (A \tensor B) \tensor C \cong A \tensor (B\tensor C)
  \\
  A \cong A \tensor I \and A \cong I \tensor A
  \\
  A \tensor B \cong B \tensor A\,,
\end{mathpar}
induced by the associators, unitors and symmetry of the category of sets with coproduct.
We claim that these are natural transformations.

\begin{proposition}
  The families of morphisms
  \begin{mathpar}
    \cc_{\assoc_{M_A,M_B,M_C}} \from (A \tensor B) \tensor C \to A \tensor (B \tensor C)
    \\
    \cc_{\lunit_{M_A}} \from A \to I \tensor A \and \cc_{\runit_{M_A}} \from A \to A \tensor I
    \\
    \cc_{\sym{M_A,M_B}} \from A \tensor B \to B \tensor A
  \end{mathpar}
  are natural transformations in $\G$.
  \label{PropCoherencesAreNatural}
\end{proposition}
\begin{proof}
  We prove this for the associator; the other cases are similar.

  Let $\sigma\from A' \implies A$, $\tau\from B' \implies B$, $\upsilon\from C' \implies C$ be strategies.  
  By Proposition \ref{PropCopycat}, we have
  \begin{IEEEeqnarray*}{Cl}
    & ((\sigma\tensor\tau)\tensor\upsilon);\cc_{\assoc_{M_{A},M_{B},M_{C}}} \\
    = & \{[\id_{M_{(A'\tensor B')\tensor C'}},\assoc_{M_A,M_B,M_C}]_*(s)\suchthat s\in (\sigma\tensor\tau)\tensor\upsilon\} \\
    = & \left\{[\id_{M_{(A'\tensor B')\tensor C'}},\assoc_{M_A,M_B,M_C}]_*(s) \,\middle|\,\mbox{\pbox{\textwidth}{
      $s\in P_{((A'\tensor B')\tensor C')\implies ((A \tensor B) \tensor C)}$\\
      $s\vert_{A',A}\in\sigma$, $s\vert_{B',B}\in\tau$, $s\vert_{C',C}\in\upsilon$
    }}\right\} \\
    = & \{s\in P_{((A'\tensor B') \tensor C') \implies (A \tensor (B \tensor C))} \suchthat s\vert_{A',A}\in\sigma,s\vert_{B',B}\in\tau,s\vert_{C',C}\in\upsilon\} \\
    = & \left\{[\assoc_{M_{A'},M_{B'},M_{C'}},\id_{M_{A\tensor (B\tensor C)}}]_*(s) \,\middle|\,\mbox{\pbox{\textwidth}{
      $s\in P_{(A'\tensor (B'\tensor C'))\implies (A \tensor (B \tensor C))}$\\
      $s\vert_{A',A}\in\sigma$, $s\vert_{B',B}\in\tau$, $s\vert_{C',C}\in\upsilon$
    }}\right\} \\
    = & \{[\assoc_{M_{A'},M_{B'},M_{C'}},\id_{M_{A\tensor (B\tensor C)}}]_*(s)\suchthat s\in \sigma\tensor (\tau\tensor \upsilon)\} \\
    = & \cc_{\assoc_{M_{A'},M_{B'},M_{C'}}};(\sigma\tensor(\tau\tensor\upsilon))\,.\hspace{1pt plus 1fill} \qedhere
  \end{IEEEeqnarray*}
\end{proof}

Then, by Proposition \ref{PropCopycat} again, these natural transformations satisfy the same coherence diagrams (pentagon, triangles, hexagon etc.) satisfied by the original associators, unitors and symmetry in $(\Set,+)$.  

It follows that $\tensor$ makes $\G$ into a symmetric monoidal category.

\section{$\G$ as a Symmetric Monoidal Closed Category}

\begin{definition}
  Let $A,B,C,D$ be games, let $\sigma$ be a strategy for $A\implies B$ and let $\tau$ be a strategy for $C\implies D$.  
  Then we define a strategy $\sigma\implies \tau\from (B\implies C) \implies (A\implies D)$ by
  \[
    \sigma\implies \tau = \{s\in P_{(B\implies C) \implies (A \implies D)}\suchthat \text{$s\vert_{A,B}\in\sigma$, $s\vert_{C,D}\in\tau$}\}\,.
    \]
\end{definition}

\begin{proposition}
  $\sigma\implies\tau$ is a strategy for $(B\implies C) \implies (A \implies D)$.
  \label{PropImpliesWellDefined}
\end{proposition}
\begin{proof}
  $\sigma\implies\tau$ is certainly a prefix-closed subset of $P_{(B\implies C)\implies (A \implies D)}^{\textit{even}}$.
  
  We examine the sign configuration of a play in $(B\implies C)\implies (A\implies D)$, using Lemma \ref{LemCompositionLemma}.  
  Since we must avoid the configuration $OP$ in either $B\implies C$, $A\implies D$ or in $(B\implies C)\implies (A\implies D)$, we arrive at the following list of possibilities.
  \scriptsize
  \[
\arraycolsep=2.6pt\def\arraystretch{1.5}
    \begin{array}{cccc|cc|c}
      \lambda_B^{OP}(s\vert_B) & \lambda_C^{OP}(s\vert_C) & \lambda_A^{OP}(s\vert_A) & \lambda_D^{OP}(s\vert_D) & \lambda_{B\implies C}^{OP}(s\vert_{B,C}) & \lambda_{A\implies D}^{OP}(s\vert_{A,D}) & \lambda_{(B\implies C)\implies (A\implies D)}^{OP}(s) \\
      \hline
      P & O & P & O & O & O & P \\
      P & P & P & O & P & O & O \\
      O & O & P & O & P & O & O \\
      P & P & P & P & P & P & P \\
      O & O & O & O & P & P & P \\
      P & P & O & O & P & P & P \\
      O & O & P & P & P & P & P
    \end{array}
    \]
  \normalsize
  If $s\in\sigma\implies\tau$, then we can immediately discount the last two of these possibilities, since one includes the illegal configuration $OP$ in $A\implies B$, and the other includes the illegal configuration $OP$ in $B\implies D$.

  By examining the remaining possibilities, we arrive at the conclusion that any $O$-position in configuration $PPPO$ constrains player $P$ to play in $C$ (to reach configuration $POPO$) or to play in $D$ (to reach configuration $PPPP$), and that any $O$-position in configuration $OOPO$ constrains player $P$ to play in $A$ (to reach configuration $OOOO$) or to play in $C$ (to reach configuration $POPO$).  

  Now suppose that $sab,sac\in\sigma\implies\tau$.  
  Then, by our above analysis, $b$ and $c$ must either both take place in the $B,A$-component, in which case $b = c$ by determinism of $\sigma$, or both in the $C,D$-component, in which case $b = c$ by determinism of $\tau$.  
\end{proof}

To prove that $\sigma\implies\tau$ is innocent if $\sigma$ and $\tau$ are, we need a lemma analogous to Lemma \ref{LemTensorViewLemma}.

\begin{lemma}
  Let $s\in \sigma\implies \tau$.  

  i) If $s$ ends with a move in $A$ or $B$, then $\pv{s}^{(B\implies C) \implies (A\implies D)} = \pv{s\vert_{A,B}}^{A\implies B}$.  

  ii) If $s$ ends with a move in $C$ or $D$, then $\pv{s}^{(B\implies C) \implies (A\implies D)} = \pv{s\vert_{C,D}}^{C\implies D}$.
  \label{LemImpliesViewLemma}
\end{lemma}
\begin{proof}
  Exactly the same as in Lemma \ref{LemTensorViewLemma}, using the analysis from the proof of Proposition \ref{PropImpliesWellDefined} to show that player $P$ only switches moves between $A$ and $B$, and between $C$ and $D$, in $\sigma\implies\tau$.
\end{proof}

\begin{proposition}
  Let $\sigma\from A \implies B$, $\tau\from C\implies D$ be innocent strategies.  
  Then $\sigma\implies\tau$ is innocent.
\end{proposition}
\begin{proof}
  Suppose $sab,t\in\sigma\implies\tau$ such that $ta\in P_{(B\implies C)\implies (A \implies D)}$ and $\pv{sa}=\pv{ta}$.  
  Suppose without loss of generality that $a$ is a move in $A$ or $B$.  
  Then $\pv{sa}=\pv{sa\vert_{A,B}}$ and $\pv{ta}=\pv{ta\vert_{A,B}}$ by Lemma \ref{LemImpliesViewLemma}, and therefore $tab\vert_{A,B}\in\sigma$ by innocence of $\sigma$, and so $tab\in\sigma\implies\tau$.
\end{proof}

We now need to prove that $\implies$ is a functor $\oppcat\G\times\G\to\G$.

\begin{proposition}
  Let $\sigma'\from A''\implies A'$, $\sigma\from A'\implies A$, $\tau'\from B'' \implies B'$ and $\tau\from B'\implies B$ be strategies.  
  Then $(\sigma\implies\tau');(\sigma'\implies\tau) = (\sigma';\sigma)\implies(\tau';\tau)$.

  Moreover, if $A',A,B',B$ are games, $f$ is a structural isomorphism from $A'$ to $A$ and $g$ is a structural isomorphism from $B'$ to $B$, then $\cc_f\implies\cc_g = \cc_{[f\inv,g]}$.
  In particular, if $A,B$ are games than $\id_A\implies\id_B=\id_{A\implies B}$.
\end{proposition}
\begin{proof}
  As in Proposition \ref{PropTensorProductIsFunctor}.
\end{proof}

Now it is easy to see that the associator $\assoc_{M_A,M_B,M_C}$ is a structural isomorphism from $(A\tensor B) \implies C$ to $A\implies (B\implies C)$, so it induces a copycat isomorphism $\Lambda_{A,B,C}=\cc_{\assoc_{M_A,M_B,M_C}} \from (A\tensor B) \implies C \to A \implies (B\implies C)$.

\begin{proposition}
  $\Lambda_{A,B,C}$ is natural in $A,B,C$.
\end{proposition}
\begin{proof}
  The same argument as in Proposition \ref{PropCoherencesAreNatural}.
\end{proof}

We have proved the following.

\begin{theorem}
  $\G$ is a symmetric monoidal closed category, with tensor product given by $\tensor$ and internal hom given by $\implies$.
\end{theorem}

\section{Products in $\G$}

\begin{proposition}
  Given some family $A_i$ of games, the game $\prod_i A_i$, as defined in Definition \ref{DefProduct}, is the category-theoretic product of the $A_i$.
  \label{PropProductOfGames}
\end{proposition}
\begin{proof}
  We have natural injections $\inj_j \from M_{A_j} \hookrightarrow M_{\prod_i A_i}$ giving rise to subset inclusions.  
  Then our projections are given by the morphisms 
  \[
    \pr_j \coloneqq \subs_{\inj_j}\from \prod_i A_i \to A_j\,.
    \]
  Now suppose we have some game $B$, and strategies $\sigma_i \from B \implies A_i$ for each $i$.
  Define a strategy
  \[
    \langle \sigma_i \rangle = \bigcup_i [\id_{M_B},\inj_i]_*(\sigma_i)\,.
    \]
  We claim that this is indeed a strategy for $B \implies \prod_i A_i$.
  Indeed, it is certainly a prefix-closed subset of $P_{C\implies\prod_i A_i}$.

  Moreover, if $sab,sac\in\langle \sigma_i \rangle$, then there is some unique $j$ such that $a$ comes from a move in $A_j$, and therefore $sab,sac$ are both plays in $\sigma_j$, so $b=c$.

  Next, we claim that $\langle\sigma_i\rangle;\pr_j = \sigma_j$.  
  Indeed, we have
  \begin{IEEEeqnarray*}{rCl?u}
    \langle\sigma_i\rangle;\pr_j & = & \langle\sigma_i\rangle;\subs_{\inj_j} \\
    & = & \{[\id_{M_B},\inj_j\inv]_*(s)\suchthat s\in\langle\sigma_i\rangle,\,s\vert_{\prod_iAi}\in(\inj_j)_*(P_{A_j})\} & \textit{Prop. \ref{PropCopycat}} \\
    & = & \sigma_j\,.
  \end{IEEEeqnarray*}
  Lastly, suppose $\tau\from B\implies\prod_iA_i$ is a strategy such that $\tau;\pr_j=\sigma_j$ for each $j$.
  We claim that $\tau=\langle\sigma_i\rangle$.
  Indeed, by the argument above, we must have
  \[
    \{[\id_{M_B},\inj_j\inv]_*(s)\suchthat s\in\tau,\,s\vert_{\prod_iA_i}\in(\inj_j)_*(P_{A_j})\} = \sigma_j
    \]
  for each $j$.
  Suppose that $s\in\tau$.  
  Then $s\vert_{\prod_iA_i}\in(\inj_j)_*(P_{A_j})$ for some $j$, by the definition of $\prod_iA_i$.  
  Therefore, $s\in[\id_{M_B},\inj_j]_*(\sigma_j)$.  

  Conversely, let $t\in \sigma_j$.  
  By the above equation, we know that there is some $s\in\tau$ such that $s\vert_{\prod_iA_i}\in(\inj_j)_*(P_{A_j})$ and $[\id_{M_B},\inj_j\inv]_*(s)=t$.  
  It follows that $[\id_{M_B},\inj_j]_*(t)=s\in\tau$.
\end{proof}

An examination of the definitions tells us that
\begin{proposition}
  Let $A_i,B$ be games and let $\phi_i$ be tree embeddings from $A_i$ to $B$.  
  Then $\langle \zz_{\phi_i} \rangle = \zz_\phi$, where $\phi$ is the tree embeddings from $\prod_iA_i$ to $B$ given by
  \[
    \phi(s) = \begin{cases}
      \epsilon & \text{if $s=\epsilon$} \\
      \phi_i(s\vert_{A_i}) & \text{if $s$ starts with a move from $A_i$}
    \end{cases}
    \]
  \label{PropProductOfTreeEmbeddings}
\end{proposition}
\begin{proof}
  The only thing we really need to check is that this is indeed a tree embedding.  
  Let $sb,sc$ be positions in $\prod_iA_i$, where $b,c$ are $P$-moves.
  Then $sb,sc$ must start with the same move, so if $\phi(sb)=\phi(sc)$ then we have $\phi_i(sb)=\phi_i(sc)$ for some $i$ and therefore $b=c$.
\end{proof}

Note that $\langle\sigma_i\rangle$ is not in general innocent, even if all the $\sigma_i$ are, and there is no version of Proposition \ref{PropProductOfTreeEmbeddings} that works for subset inclusion strategies.
Of course, since a subset inclusion is a special case of a tree embedding, then $\langle \subs_i\rangle$ is always a tree embedding strategy.

\section{Sequoidal categories}

We have now given the category-theoretic properties of all the connectives from chapter \ref{SecConnectives}, with the exception of the sequoid $\sequoid$ and the exponential $\oc$.

We would like to say that $\blank\sequoid\blank$ is a functor from $\G \times \G \to \G$, as is the case with the tensor product $\blank\tensor\blank$.  
However, this does not quite work: given strategies $\sigma\from A \implies B$ and $\tau\from C \implies D$, we may not get a well-formed strategy for $(A \sequoid C) \implies (B \sequoid D)$ by `playing according to $\sigma$ in $A$ and $B$ and according to $\tau$ in $C$ and $D$'.  
The reason is that the constraint that player $O$ plays in $B$ before $D$ is not strong enough to force player $P$ to play in $A$ before $C$; indeed, suppose that $\sigma$ tells player $P$ to respond to an initial move in $B$ with another move in $B$.  
Suppose that player $O$ then decides to make a move in $D$.  
If $\tau$ tells player $P$ to respond to this move in $D$ with a move in $C$, then she will be stuck, unable to play this move because no move has yet been played in $A$.

We can fix this problem by imposing some constraints on the strategies $\sigma$ and $\tau$.  
The problem occurs when player $O$'s initial move in $B$ is not reflected by an initial move by player $P$ in $A$; therefore, if $\sigma$ is such that player $P$ always responds to the initial move in $B$ with a move in $A$, then we can form a strategy $\sigma\sequoid\tau$ for $(A\sequoid C)\implies (B\sequoid D)$.  
Moreover, this strategy $\sigma\sequoid\tau$ will inherit this property that the first move on the right is always replied to by a move on the left.

\begin{definition}
  Let $A,B$ be games.  
  A \emph{strict morphism} from $A$ to $B$ is a strategy $\sigma$ for $A\implies B$ such that any player $P$ response to an opening move in $B$ is a move in $A$; i.e., such that if $b$ is an initial $O$-move in $B$ and $ba\in\sigma$, then $a$ is a move in $A$.
\end{definition}

We will call such a $\sigma$ a \emph{strict strategy} for $A\implies B$, although this is a slight abuse of language, since the definition depends on the constituent games $A$ and $B$, which may not be recoverable from $A\implies B$.

It is clear that the composition of strict morphisms is again a strict morphism , as is any morphism of the form $\subs_i$, and so we get a wide subcategory $\G_s$ of $\G$ whose objects are games and where the morphisms are the strict strategies.
We then have a natural inclusion functor $J\from \G_s\to \G$.  

\begin{definition}
  Given games $A,B,C,D$, a strict morphism $\sigma\from A \implies B$ and a strategy $\tau\from C\implies D$, we define a strict morphism $\sigma\sequoid\tau\from (A \sequoid C)\implies (B \sequoid D)$ by
  \[
    \sigma\sequoid\tau = \{s\in P_{(A\sequoid C)\implies (B\sequoid D)}\suchthat s\vert_{A,B}\in\sigma,\,s\vert_{C,D}\in\tau\}\,.
    \]
  \label{DefSequoidOfStrategies}
\end{definition}

\begin{proposition}
  $\sigma\sequoid\tau$ is a strategy.  
\end{proposition}
\begin{proof}
  $\sigma\sequoid\tau$ is certainly a prefix-closed subset of $P_{(A\sequoid C)\implies (B\sequoid D)}$.  
  Moreover, if $sab,sac\in\sigma\sequoid\tau$, then $sab,sac\in\sigma\tensor\tau$, so $b=c$.  
\end{proof}

Of course, $P_{A\sequoid B}$ is a subset of $P_{A\tensor B}$, which means that the identity function $M_A + M_B \to M_A + M_B$ gives us a subset inclusion from $A \sequoid B$ to $A \tensor B$, and hence a strategy $\subs_{\id_{M_A+M_B}}$ for $A\tensor B \implies A \sequoid B$, which we shall refer to as $\wk_{A,B}$.  

\begin{proposition}
  Let $A,B,C,D$ be games, let $\sigma\from A \implies B$ be a strict strategy and let $\tau\from C \implies D$ be a strategy.  
  Then the following diagram commutes.
  \[
    \begin{tikzcd}
      A \tensor C \arrow[r, "\sigma\tensor\tau"] \arrow[d, "{\wk_{A,C}}"']
        & B \tensor D \arrow[d, "{\wk_{B,D}}"] \\
      A \sequoid C \arrow[r, "\sigma\sequoid\tau"]
        & B \sequoid D
    \end{tikzcd}
    \]
  \label{PropWkCommutativeDiagram}
\end{proposition}
\begin{proof}
  By Proposition \ref{PropCopycat} and the definition of $\wk$, we know that
  \begin{IEEEeqnarray*}{rCl}
    \sigma\tensor\tau;\wk_{B,D} & = & \{s\in \sigma\tensor\tau,\,s\vert_{B,D}\in P_{B\sequoid D}\}
    \\
    \wk_{A,C};\sigma\sequoid\tau & = & \sigma\sequoid\tau\,,
  \end{IEEEeqnarray*}
  as sets of plays.

  Now we know that $\sigma\sequoid\tau = \{s\in\sigma\tensor\tau\suchthat s\in P_{(A\sequoid C)\implies (B\sequoid D)}\}$, so it suffices to show that if $s\in\sigma\tensor\tau$ is such that $s\vert_{B,D}\in P_{B\sequoid D}$ then $s\in P_{(A\sequoid C)\implies (B\sequoid D)}$.

  Indeed, if $s\vert_{B,D}\in P_{B\sequoid D}$ then $s$ begins with an initial $O$-move in $B$.  
  Then, \emph{since $\sigma$ is strict}, the next move in $s$ must be a move in $A$, and therefore $s\vert_{A,C}$ begins with a move in $A$.
  Since we also have $s\vert_{A,C}\in P_{A\tensor C}$, we must have that $s\vert_{A,C}\in P_{A\sequoid C}$.
\end{proof}
\begin{remark}
  This is the main place where we have used the assumption that $\sigma$ is a strict strategy: if we drop the strictness requirement from Definition \ref{DefSequoidOfStrategies}, then we get a valid (if nonsensical) strategy that has a partiality (`gives up') if playing according to $\sigma$ and $\tau$ would lead to it creating an invalid play.  
  But such a strategy would not satisfy the conclusion of Proposition \ref{PropWkCommutativeDiagram}, since $\sigma\tensor\tau;\wk_{B,D}$ would contain these extra plays where $\wk_{A,C};\sigma\sequoid\tau$ had `given up'.
\end{remark}
\begin{remark}
  Of course, we would \emph{like} to restate Proposition \ref{PropWkCommutativeDiagram} by saying that $\wk$ is some sort of natural transformation, but that doesn't make sense until we've shown that $\blank\sequoid\blank$ is a functor.
\end{remark}

\begin{proposition}
  If we have strict strategies $\sigma'\from A'' \implies A'$ and $\sigma\from A' \implies A$, and strategies $\tau' \from B'' \implies B'$ and $\tau \from B' \implies B$, then we have
  \[
    (\sigma'\sequoid\tau');(\sigma\sequoid\tau) = (\sigma';\sigma)\sequoid(\tau';\tau)\,.
    \]
  If $A',A,B',B$ are games, $i$ is a subset inclusion from $A$ into $A'$ and $j$ is a subset inclusion from $B$ into $B'$, then
  \[
    \subs_i\sequoid\subs_j = \subs_{[i,j]}\from A'\sequoid B' \implies A \sequoid B\,.
    \]
  In particular, if $A,B$ are games, then $\id_A\sequoid\id_B=\id_{A\sequoid B}$.
  \label{PropSequoidIsFunctor}
\end{proposition}
\begin{proof}
  Let $A'',A',A,B'',B',B$ and $\sigma',\sigma,\tau',\tau$ be as above.  

  We have
  \begin{IEEEeqnarray*}{rCl?u}
    \wk_{A'',B''};(\sigma'\sequoid\tau');(\sigma\sequoid\tau) & = & (\sigma'\tensor\tau');\wk_{A',B'};(\sigma\sequoid\tau) & \textit{Prop. \ref{PropWkCommutativeDiagram}} \\
    & = & (\sigma'\tensor\tau');(\sigma\tensor\tau);\wk_{A,B} & \textit{Prop. \ref{PropWkCommutativeDiagram}} \\
    & = & ((\sigma';\sigma)\tensor(\tau';\tau));\wk_{A,B} & \textit{Prop. \ref{PropTensorProductIsFunctor}} \\
    & = & \wk_{A'',B''};((\sigma';\sigma)\sequoid(\tau';\tau))\,. & \textit{Prop. \ref{PropWkCommutativeDiagram}} \\
  \end{IEEEeqnarray*}
  By Proposition \ref{PropSubsetInclusionEpic}, $\wk_{A'',B''}$ is an epimorphism, and therefore we have that
  \[
    (\sigma'\sequoid\tau');(\sigma\sequoid\tau) = (\sigma';\sigma)\sequoid(\tau';\tau)\,.
    \]

  Now let $A',A,B',B$ be games, let $i$ be a subset inclusion from $A$ into $A'$ and let $j$ be a subset inclusion from $B$ into $B'$.  
  Then, since subset inclusion strategies are automatically strict, we have
  \begin{IEEEeqnarray*}{rCl?u}
    \wk_{A',B'};(\subs_i\sequoid\subs_j) & = & (\subs_i\tensor\subs_j);\wk_{A,B} & \textit{Prop. \ref{PropWkCommutativeDiagram}} \\
    & = & \subs_{[i,j]};\wk_{A,B} & \textit{Prop. \ref{PropTensorProductIsFunctor}} \\
    & = & \subs_{[i,j]} & \textit{Prop. \ref{PropCopycat}} \\
    & = & \wk_{A',B'};\subs_{[i,j]}\,. & \textit{Prop. \ref{PropCopycat}} \\
  \end{IEEEeqnarray*}
  As before, we know from Proposition \ref{PropSubsetInclusionEpic} that $\wk_{A',B'}$ is an epimorphism, and so
  \[
    \subs_i\sequoid\subs_j = \subs_{[i,j]}\,.\qedhere
    \]
\end{proof}

Proposition \ref{PropSequoidIsFunctor} tells us that $\blank\sequoid\blank$ is a functor $\G_s\times\G\to \G$.  
As before, write $J$ for the inclusion functor $\G_s\hookrightarrow\G$.
Then we can restate Proposition \ref{PropWkCommutativeDiagram} in a more \emph{natural} way.

\begin{proposition}
  $\wk_{A,B}$ is a natural transformation $JA \tensor B \to J(A \sequoid B)$.
  \label{PropWkNatural}
\end{proposition}

We have some additional structure on the $\tensor$ and $\sequoid$ operators.  
By inspecting the definitions that if $A,X,Y$ are games then the associator $\assoc_{M_A,M_X,M_Y}$ and unitor $\runit_{M_A}$ in $(\Set,+)$ give rise to structural isomorphisms
\begin{mathpar}
  (A \sequoid X) \sequoid Y \cong A \sequoid (X \tensor Y)
  \and
  A \cong A \sequoid I\,.
\end{mathpar}
Indeed, in the first case, both games are the game in which $A$, $X$ and $Y$ are played in parallel, but where the first move must take place in $A$.  
In the second case, we have $A\sequoid I=A\tensor I$, because there are no moves in $I$ anyway, and the copycat morphism induced from the right unitor in $(\Set,+)$ is the same strategy as the right unitor $A \toisom A \tensor I$.

We formalize the structure we have uncovered so far in the concept of a \emph{sequoidal category}.

\begin{definition}[\cite{laird02}]
  A \emph{sequoidal category} $\C$ is given by
  \begin{itemize}
    \item a symmetric monoidal category $(\C,\tensor, I)$ (with coherences $\assoc,$ $\lunit$, $\runit$, $\sym$);
    \item a (strong) right action of $\C$ on a category $\C_s$; i.e., a functor $\blank\sequoid\blank\from\C_s\times\C\to\C_s$ together with natural isomorphisms
      \begin{mathpar}
        \passoc_{a,x,y}\from (a \sequoid x) \sequoid y \toisom a \sequoid (x \tensor y)
        \and
        \run_a \from a \toisom a \sequoid I
      \end{mathpar}
      that make the diagrams
      \begin{mathpar}
        \begin{tikzcd}[column sep=31pt]
          ((a \sequoid x) \sequoid y) \sequoid z \arrow[r, "{\passoc_{a,x,y}\sequoid z}" yshift=1pt] \arrow[d, "{\passoc_{a\sequoid x,y,z}}"' description]
            & (a \sequoid (x \tensor y)) \sequoid z \arrow[r, "{\passoc_{a,x\tensor y,z}}" yshift=1pt]
              & a \sequoid ((x\tensor y) \tensor z)  \arrow[dl, "{a \sequoid \assoc_{x,y,z}}"] \\
          (a \sequoid x) \sequoid (y \tensor z) \arrow[r, "{\passoc_{a,x,(y\tensor z)}}" yshift=1pt]
            & a \sequoid (x \tensor (y \tensor z))
              &
        \end{tikzcd}
        \and
        \begin{tikzcd}
          a \sequoid x \arrow[r, "{a \sequoid \lunit_x}"] \arrow[d, "\run_a \sequoid x"']
            & a \sequoid (I \tensor x) \\
          (a \sequoid I) \sequoid x \arrow[ur, "{\passoc_{a,I,x}}"']
            &
        \end{tikzcd}
        \and
        \begin{tikzcd}
          a \sequoid x \arrow[r, "{a \sequoid \runit_x}"] \arrow[d, "\run_{a \sequoid x}"']
            & a \sequoid (x \tensor I) \\
          (a \sequoid x) \sequoid I \arrow[ur, "{\passoc_{a,x,I}}"']
            &
        \end{tikzcd}
      \end{mathpar}
      commute; and
    \item a lax morphism of actions from $\blank\sequoid\blank$ to the right tensor multiplication action $\blank\tensor\blank$ of $\C$ on itself; i.e., a functor $J\from \C_s \to \C$ and a natural transformation $\wk_{a,x} \from Ja \tensor x \to J(a \sequoid x)$ that makes the following diagrams commute.
      \begin{mathpar}
        \begin{tikzcd}[column sep=30pt]
          (Ja \tensor x) \tensor y \arrow[r, "{\wk_{a,x}\tensor y}"] \arrow[d, "{\assoc_{Ja,x,y}}"'] & J(a \sequoid x) \tensor y \arrow[r, "{\wk_{a\sequoid x,y}}"]
              & J((a \sequoid x) \sequoid y) \arrow[dl, "{J\passoc_{a,x,y}}"] \\
          Ja \tensor (x \tensor y) \arrow[r, "{\wk_{a,x\tensor y}}"]
            & J(a \sequoid (x \tensor y))
              &
        \end{tikzcd}
        \and
        \begin{tikzcd}
          Ja \arrow[r, "J\run_a"] \arrow[d, "\runit_{Ja}"']
            & J(a \sequoid I) \\
          Ja \tensor I \arrow[ur, "{\wk_{a,I}}"']
            &
        \end{tikzcd}
      \end{mathpar}
  \end{itemize}
  \label{DefSequoidalCategory}
\end{definition}
\begin{remark}
  The definitions of \emph{lax action} can be found at the start of chapter \ref{SecParametricMonads}, whie that of an \emph{oplax morphism of actions} is found at Definition \ref{DefOplaxMorphismOfActions}.  
  The definitions we have used are similar: a \emph{strong action} is a lax action in which the coherences (called $m$ and $e$ in chapter \ref{SecParametricMonads} and $\passoc$ and $\run$ here) are isomorphims.  
  A \emph{lax morphism of actions} is defined in the same way as an oplax morphism, except that the coherence (called $\mu$ in Definition \ref{DefOplaxMorphismOfActions} and $\wk$ here) goes in the opposite direction.
\end{remark}

\begin{proposition}
  The monoidal category $\G$, together with the category $\G_s$, the natural transformations
  \begin{mathpar}
    \passoc_{A,X,Y} = \cc_{\assoc_{M_A,M_X,M_Y}} \from (A \sequoid X) \sequoid Y \toisom A \sequoid (X \tensor Y)
    \and
    \run_A = \cc_{\runit_{M_A}} \from A \toisom A \sequoid I\,,
  \end{mathpar}
  the inclusion functor $J\from \G_s \to G$ and the natural transformation
  \begin{mathpar}
    \wk_{A,X}=\subs_{j} \from JA \tensor X = A \tensor X \to A \sequoid X = J(A \sequoid X)
  \end{mathpar}
  form a sequoidal category.
\end{proposition}
\begin{proof}
  We have shown most of this already; all that remains is to show that $\passoc$ and $\run$ are natural transformations and that the five diagrams in Definition \ref{DefSequoidalCategory} commute.  

  Let us start with the diagrams.  
  By Proposition \ref{PropCopycat}, commutativity of these diagrams follows from commutativity of the diagrams formed from the corresponding subset inclusion functions in $\Set$.  
  For example, to show that the first diagram commutes in $\G$, we must show that the following diagram commutes in $\Set$.
  \scriptsize
  \[
    \begin{tikzcd}[column sep=16pt, row sep=30pt]
      ((M_A + M_X) + M_Y) + M_Z \arrow[r, "{[\assoc_{M_A,M_X,M_Y},\id_{M_Z}]}" yshift=3pt] \arrow[d, "{\assoc_{M_A+M_X,M_Y,Z}}" description]
        & (M_A + (M_X + M_Y)) + M_Z \arrow[r, "{\assoc_{M_A,M_X+M_Y,M_Z}}" yshift=3pt]
          & M_A + ((M_X + M_Y) + M_Z) \arrow[dl, "{[\id_{M_A},\assoc_{M_X,M_Y,M_Z}]}"] \\
      (M_A + M_X) + (M_Y + M_Z) \arrow[r, "{\assoc_{M_A,M_X,M_Y+M_Z}}" yshift=3pt]
        & M_A + (M_X + (M_Y + M_Z))
          &
    \end{tikzcd}
    \]
  \normalsize
  This diagram is, of course, none other than the pentagram diagram for the coproduct $+$ in $\Set$.  
  Similarly, the second and third diagrams in Definition \ref{DefSequoidalCategory} reduce in this case to the triangle diagrams for the coproduct $+$ in $\Set$.

  For the fourth diagram in Definition \ref{DefSequoidalCategory}, since $\wk$ is a subset inclusion strategy induced from an identity map, Proposition \ref{PropCopycat} tells us that both arms of the diagram are the strategy induced by the subset inclusion 
  \[
    \assoc_{M_A,M_X,M_Y}\from (M_A + M_X) + M_Y \to M_A + (M_X + M_Y)
    \]
  from $(A \tensor X) \tensor Y$ to $A \sequoid (X \tensor Y)$.
  Similarly, both arms of the last diagram in Definition \ref{DefSequoidalCategory} are the strategies induced by the subset inclusion $\runit_{M_A} \from M_A \to M_A + \emptyset$ from $A$ to $A \sequoid I$.

  It now remains only to show that $\passoc$ and $\run$ are natural transformations.  
  For $\passoc$, suppose that $A',X',Y',A,X,Y$ are games, that $\sigma\from A'\implies A$ is a strict strategy and that $\tau\from B'\implies B,\upsilon\from C'\implies C$ are strategies.  
  Then we need to show that the following diagram commutes.
  \[
    \begin{tikzcd}[column sep=50pt]
      (A' \sequoid X') \sequoid Y' \arrow[r, "{\passoc_{A',X',Y'}}"] \arrow[d, "(\sigma\sequoid\tau)\sequoid\upsilon"']
        & A' \sequoid (X' \tensor Y') \arrow[d, "\sigma\sequoid (\tau\tensor\upsilon)"] \\
      (A \sequoid X) \sequoid Y \arrow[r, "{\passoc_{A,X,Y}}"]
        & A \sequoid (X \tensor Y)
    \end{tikzcd}
    \]
  Indeed, we have
  \begin{IEEEeqnarray*}{rCl?u}
    && (\wk_{A',X'}\tensor Y');\wk_{A'\sequoid X',Y'};\passoc_{A',X',Y'};(\sigma\sequoid(\tau\tensor\upsilon)) \\
    & = & \assoc_{A',X',Y'};\wk_{A',X'\tensor Y'};(\sigma\sequoid(\tau\tensor\upsilon)) & (see above) \\
    & = & \assoc_{A',X',Y'};(\sigma\tensor(\tau\tensor\upsilon));\wk_{A,X\tensor Y} & \textit{Prop. \ref{PropWkNatural}} \\
    & = & ((\sigma\tensor\tau)\tensor\upsilon);\assoc_{A,X,Y};\wk_{A,X\tensor Y} & \textit{Prop. \ref{PropCoherencesAreNatural}} \\
    & = & ((\sigma\tensor\tau)\tensor\upsilon);(\wk_{A,X}\tensor Y);\wk_{A\sequoid X,Y};\passoc_{A,X,Y} & (see above) \\
    & = & (\wk_{A',X'}\tensor Y');((\sigma\sequoid\tau)\tensor\upsilon);\wk_{A\sequoid X,Y};\passoc_{A,X,Y} & \textit{Prop. \ref{PropWkNatural}} \\
    & = & (\wk_{A',X'}\tensor Y');\wk_{A'\sequoid X',Y'};((\sigma\sequoid\tau)\sequoid\upsilon);\passoc_{A,X,Y}\,. & \textit{Prop. \ref{PropWkNatural}}
  \end{IEEEeqnarray*}
  Now observe that $\wk_{A',X'}\tensor Y=\subs_{[\id_{M_{A'}+M_{X'}},\id_{M_{Y'}}]}$ by Proposition \ref{PropTensorProductIsFunctor}, so it is an epimorphism by Proposition \ref{PropSubsetInclusionEpic}.  
  Proposition \ref{PropSubsetInclusionEpic} also tells us that $\wk_{A'\sequoid X',Y'}$ is an epimorphism.  
  Therefore, we have
  \[
    \passoc_{A',X',Y'};(\sigma\sequoid(\tau\tensor\upsilon)) = ((\sigma\sequoid\tau)\sequoid\upsilon);\passoc_{A,X,Y}
    \]
  for any $A',X',Y',A,X,Y,\sigma,\tau,\upsilon$ as above.  
  It follows that $\passoc$ is a natural transformation.

  The proof that $\run$ is a natural transformation is similar.  
  Let $A',A$ be games and let $\sigma\from A'\implies A$ be a strict strategy.  
  We need to show that the following diagram commutes.
  \[
    \begin{tikzcd}
      A' \arrow[r, "\run_{A'}"] \arrow[d, "\sigma"']
        & A' \sequoid I \arrow[d, "\sigma\sequoid I"] \\
      A \arrow[r, "\run_A"]
        & A \sequoid I
    \end{tikzcd}
    \]
  Indeed, we have
  \begin{IEEEeqnarray*}{rCl?u}
    \run_{A'};(\sigma\sequoid I) & = & \runit_{A'};\wk_{A',I};(\sigma\tensor I) & (see above) \\
    & = & \runit_{A'};(\sigma\tensor I);\wk_{A,I} & \textit{Prop. \ref{PropWkNatural}} \\
    & = & \sigma;\runit_A;\wk_{A,I} & \textit{Prop. \ref{PropCoherencesAreNatural}} \\
    & = & \sigma;\run_A\,. & (see above)
  \end{IEEEeqnarray*}

  Therefore, $\run$ is a natural transformation, which completes our check of the criteria required by Definition \ref{DefSequoidalCategory}.
\end{proof}

\section{Tree Embeddings and Zigzag Strategies}

So far, the strategies we have been considering have all been innocent.  
We now start considering some non-innocent strategies.

We can generalize the subset inclusions of the previous chapter to \emph{tree embeddings}.  
Tree embeddings are similar to subset inclusions, but generated by a function between plays, rather than between moves.  
A consequence of this is that while tree embeddings do give rise to strategies, these strategies are not in general innocent.

\begin{definition}
  Let $A,B$ be games.  
  A \emph{tree embedding} from $A$ to $B$ is a function $\phi\from P_A \hookrightarrow P_B$ such that
  \begin{itemize}
    \item $\phi$ preserves length and justification indices; 
    \item for all sequences $s,t\in P_A$, if $t\prefix s$ then $\phi(t)\prefix\phi(s)$; and
    \item if $\phi(sb)=\phi(sc)$, where $b,c$ are $P$-moves in $A$, then $b=c$.
  \end{itemize}

  Given a tree embedding $\phi$ from $A$ to $B$, we define a strategy $\zz_\phi\from B \implies A$ by
  \[
    \zz_\phi = \{s\in P_{B\implies A}\suchthat\text{for all even-length $t\prefix s$, $t\vert_B=\phi(t\vert_A)$}\}\,.
    \]
\end{definition}
\begin{example}
  If $i$ is a subset inclusion from $A$ to $B$, then $i_*$ is a tree embedding from $A$ to $B$ and $\zz_{i_*}=\subs_i$.
\end{example}

\begin{proposition}
  $\zz_\phi$ is a strategy.  
\end{proposition}
\begin{proof}
  $\zz_\phi$ is a prefix-closed subset of $P_{B\implies A}$ by definition.  
  If $sab,sac\in\zz_\phi$, then we have $s\vert_B=\phi(s\vert_A)$ and $sab\vert_B=\phi(sab\vert_A)$.
  Since $\phi$ is length-preserving, $s\vert_A$ and $s\vert_B$ must have the same length, and the same is true of $sab\vert_A$ and $sab\vert_B$.  
  Therefore, either $a$ is a move in $A$ and $s\vert_Bb=\phi(s\vert_Aa)$ or $a$ is a move in $B$ and $s\vert_Ba=\phi(s\vert_Ab)$.  
  The same applies to $c$: so either $s\vert_Bb=\phi(s\vert_Aa)=s\vert_Bc$ or $\phi(s\vert_Ab)=s\vert_Ba=\phi(s\vert_Ac)$.  
  In either case, we have $b=c$.
\end{proof}

We want an analogue of Proposition \ref{PropCopycat}.

\begin{definition}
  Given a tree embedding from $A$ to $B$, and a play $s\in P_{C\implies A}$ for some $C$, we write $s^\phi$ for the play obtained by replacing the moves of $s\vert_A$ wholesale with the moves of $\phi(s\vert_A)$ (using the fact that $\phi$ preserves length and justification indices).
\end{definition}

\begin{proposition}
  If $\sigma\from C \implies B$ is a strategy, then $\sigma;\zz_\phi$ is given by
  \[
    \{s\in P_{C\implies A}\suchthat s^\phi\in\sigma\}\,.
    \]

  In particular, if $\phi$ is a tree embedding from $A$ to $B$ and $\psi$ is a tree embedding from $B$ to $C$, then $\psi\circ\phi$ is a tree embedding from $A$ to $C$ and $\zz_{\psi\circ\phi}=\zz_\psi;\zz_\phi$.
  \label{PropTree}
\end{proposition}
\begin{proof}
  Suppose that $\s\in\sigma\|\zz_\phi$.  
  Then $\s\vert_{C,B}\in\sigma$ and $t\vert_B=\phi(t\vert_A)$ for all even-length $t\prefix \s\vert_{B,A}$.  
  Then it is clear that $(\s\vert_{C,A})^\phi=\s\vert_{C,B}$.  

  Conversely, suppose that $s\in P_{C\implies A}$ and that $s^\phi\in\sigma$.  
  We construct a sequence $\s\in\sigma\|\zz_phi$ such that $\s\vert_{C,B}=s^\phi$ and $\s\vert_{C,A}=s$ by taking the sequence $s$ and inserting, in order, the elements of the sequence $\phi(s\vert_A)$ immediately after each $O$-move in $s\vert_C$ and immediately before each $P$-move in $C$, leaving the rest of $s$ intact.  
  Then $\s\vert_{C,B}=s^\phi\in\sigma$ and $\s\vert_{B,A}\in\zz_\phi$, by construction.  
  So $s=\s\vert_{C,A}\in \sigma;\zz_\phi$.
\end{proof}

\begin{definition}
  We say that a tree embedding $\phi$ is a \emph{tree isomorphism} if it is a bijection.
\end{definition}

\begin{proposition}
  If $\phi$ is a tree isomorphism from a game $A$ to a game $B$, then $\zz_\phi$ is an isomorphism in $\G$.
\end{proposition}
\begin{proof}
  If $\phi$ is a tree isomorphism, then its inverse $\phi\inv$ is also a tree isomorphism, and Proposition \ref{PropTree} tells us that $\zz_\phi$ and $\zz_{\phi\inv}$ are inverses in $\G$.
\end{proof}

More generally:

\begin{proposition}
  If $\phi$ is a surjection, then $\zz_\phi$ is a monomorphism.
  \label{PropZigzagMono}
\end{proposition}
\begin{proof}
  Let $\sigma,\tau\from C \implies B$ be strategies.
  Then, by Proposition \ref{PropTree}, we have
  \[
    \{s\in P_{C\implies A}\suchthat s^\phi\in\sigma\} = \{s\in P_{C\implies A}\suchthat s^\phi\in\tau\}\,.
    \]
  Let $t\in\sigma$.  
  Then, since $\phi$ is surjective, there is some $u\in P_A$ such that $\phi(u)=t\vert_B$.  
  As before, we may construct some sequence $t'$ such that $t'\vert_A=u$ and $t=(t')^\phi$.  
  Then, since $(t')^\phi=t\in\sigma$, we must have $(t')^\phi\in\tau$; i.e., that $t\in\tau$.  
  So $\sigma\subset\tau$.  
  
  Similarly, $\tau\subset\sigma$, and so $\sigma$ and $\tau$ are equal.
\end{proof}

\section{Sequoidally decomposable categories}

We will now consider some important additional category-theoretic properties of the sequoid operator on games that do not follow from the fact that $\G$ is a sequoidal category.

\begin{definition}[\cite{martinsthesis}]
  Let $\C$ be a sequoidal category such that $C_s$ has arbitrary products (including a terminal object $1$).
  We say that $\C$ is \emph{distributive} if whenever $a_i$ is a collection of objects of $\C'$ and $x$ is an object of $\C$, the morphism
  \[
    \dec_{(a_i),x} = \langle \pr_i\sequoid x \rangle \from \prod_ia_i \sequoid x \to \prod_i (a_i \sequoid x)
    \]
  is an isomorphism.
\end{definition}
\begin{remark}
  In particular, taking $(a_i)$ to be the empty collection, the morphism $\lun_x = () \from 1 \sequoid x \to 1$ is an isomorphism.
\end{remark}

\begin{proposition}
  $\G$ is a distributive sequoidal category.
\end{proposition}
\begin{proof}
  Let $(A_i),X$ be games.
  By Proposition \ref{PropProductOfTreeEmbeddings}, the morphism $\langle \pr_i\sequoid X\rangle$ is given by the tree embedding $\phi\from P_{\prod_i(A_i\sequoid X)}\to P_{\prod_iA_i\sequoid X}$ defined as follows.
  \[
    \phi(s) = \begin{cases}
      \epsilon & \text{if $s=\epsilon$} \\
      [\inj_{A_j},\inj_{X}]_*(s) & \text{if $s$ begins with a move in the $j$-th component}
    \end{cases}
    \]
  When we say $[\inj_{A_j},\inj_{X^j}]_*(s)$, we have considered $s$ as a sequence in $(M_{A_j} + M_{X^j})^*$.

  We claim that $\phi$ is a bijection.  
  Indeed, it is certainly injective, since if $\phi(s)=\phi(t)$, then the first move of $\phi(s)=\phi(t)$ occurs in one of the $A_j$, which means that $s,t$ must both come from the $j$-th component.
  Then, if we have a non-empty sequence $s\in P_{\prod_iA_i \sequoid X)}$, then $s$ must start with a move in some $A_j$, and must thereafter take place in the games $A_j$ and $X$.  
  Then $s=\phi([\inj_{A_j},\inj_{X^j}]_*(s))$, where we have considered $s$ as a sequence in $(M_{A_j} + M_{X^j})^*$.

  Therefore, $\phi$ is a tree isomorphism, so $\dec_{(A_i),X} = \zz_\phi$ is an isomorphism by Proposition \ref{PropTree}.
\end{proof}

We can get a distributivity result in the other direction, but this one is not as strong, since the morphism we get is only a monomorphism, not an isomorphism.
\begin{definition}
  Let $\C$ be a distributive sequoidal category.  
  We say that $\C$ is \emph{strongly distributive} if whenever $(A_i),(B_i)$ are objects of $\C_s$, where $(B_i)$ is a non-empty collection, then the morphism
  \[
    \langle A_1 \sequoid(\cdots \sequoid (A_n \sequoid J(\pr_i))\cdots)\rangle
    \]
  is a monomorphism
  \[
    A_1 \sequoid\left(\cdots\sequoid\left(A_n \sequoid J\left(\prod_iB_i\right)\right)\cdots\right) \to \prod_i (A_1 \sequoid (\cdots \sequoid (A_n \sequoid J(B_i))\cdots))\,.
    \]
\end{definition}

\begin{proposition}
  $\G$ is a strongly distributive sequoidal category.
\end{proposition}
\begin{proof}
  By Proposition \ref{PropProductOfTreeEmbeddings}, the morphism $\langle (A_1\sequoid(\cdots\sequoid(A_n \sequoid\pr_i)\cdots))\rangle$ is given by the tree embedding 
  \[
    \phi\from P_{\prod_i (A_1 \sequoid(\cdots\sequoid(A_n\sequoid B_i)\cdots))} \to P_{A_1\sequoid(\cdots\sequoid(A_n\sequoid \prod_iB_i))}
    \]
  defined as follows.
  \[
    \phi(s) = \begin{cases}
      \epsilon & \text{if $s=\epsilon$} \\
      [\inj_{A_1,\cdots,A_n},\inj_{B_j}]_*(s) & \parbox[t][][t]{180pt}{if $s$ begins with a move in the $j$-th component}
    \end{cases}
    \]
  Note that $\phi$ is not in general injective, since if $s$ occurs entirely inside one of the copies of the $A_i$, then $\phi(s)=\phi(s')$ for any identical sequence $s'$ occurring inside one of the other copies of the $A_i$.

  We claim that $\phi$ is surjective.  
  Indeed, let $t\in P_{A_1\sequoid(\cdots \sequoid(\prod_i B_i))}$ be a non-empty sequence.  
  If $t$ contains moves in one of the $B_j$, then we have $t=\phi([\inj_{A_i^j},\inj_{B_j}]_*(t))$, where $A_i^j$ is the copy of $A_i$ in the $j$-th component of the product and we have considered $t$ as a sequence in $(M_{A_1} + \cdots + M_{A_n} + M_{B_j})^*$.  
  If $t$ only contains moves in the $A_i$, then pick some fixed index $0$; then we have $t=\phi([\inj_{A_i^0}]_*(t))$, where we have considered $t$ as a sequence in $(M_{A_1} + \cdots + M_{A_n})^*$.

  Therefore, $\phi$ is a monomorphism by Proposition \ref{PropZigzagMono}.
\end{proof}

Note that if $\C$ is decomposable, this property of being a monomorphism is automatically preserved by the tensor product; i.e., for any objects $A,(B_i),C$ of $\C_s$ (for $B_i$ a non-empty collection), the morphism
\[
  \langle JC \tensor J(A \sequoid J(\pr_i))\rangle \from JC \tensor J\left(A \sequoid J\left(\prod_iB_i\right)\right) \to \prod_i JC \tensor J(A \sequoid J(B_i))
  \]
is a monomorphism.

\begin{definition}
  A sequoidal category $\C$ is \emph{inclusive} if $\C_s$ is a full-on-objects subcategory of $\C$ containing $\wk$ and all isomorphisms of $\G$, and the functor $J$ is the inclusion functor.
\end{definition}
In such a situation, we will sometimes drop the mention of the functor $J$.

\begin{proposition}
  $\G$ is an inclusive sequoidal category.
\end{proposition}
\begin{proof}
  The only thing we really need to check is that isomorphisms in $\G$ are always strict strategies.  
  Indeed, if $\sigma$ is a strategy for $A \implies B$ and $\tau$ a strategy for $B \implies A$ such that $\sigma;\tau=\id_A$, then for any opening move $a$ in $A$ on the right of $\tau$ there is some $\s\in\int(A,B,A)$ such that $\s\vert_{A,A}=aa$, and therefore the reply to $a$ in $\tau$ must take place in $B$.
\end{proof}

An important fact about the sequoid operator for games is that it gives us a way to decompose the tensor product as
\[
  A \tensor B \cong (A \sequoid B) \times (B \sequoid A)\,.
  \]
Informally, this is because both sides allow player $O$ to start either in $A$ or in $B$, and thereafter to continue in that game.

\begin{definition}
  Let $\C$ be a distributive inclusive sequoidal category, where $\C$ is a symmetric monoidal category.
  We say that $\C$ is \emph{decomposable} if the morphisms
  \begin{mathpar}
    \dec_{a,b}=\langle \wk_{a,Jb},\sym_{Ja,Jb};\wk_{b,Ja} \rangle \from Ja \tensor Jb \to (a \sequoid Jb) \times (b \sequoid Ja)
    \and
    () \from I \to 1
  \end{mathpar}
  are isomorphisms in $\C_s$.
\end{definition}

\begin{proposition}
  Let $\C$ be a decomposable sequoidal category and suppose that $a_1,\cdots,a_n$ is a list of objects of $\C_s$.  
  Then we have an isomorphism
  \[
    a_1 \tensor \cdots \tensor a_n \cong \prod_{i=1}^n (a_i \sequoid (a_1 \tensor \cdots \tensor a_{i-1} \tensor a_{i+1} \tensor \cdots \tensor a_n))\,.
    \]
  \label{PropDecSeq}
\end{proposition}
\begin{proof}
  Induction on $n$.  
  If $n=0$, then we have the isomorphism $() \from I \to 1$.  
  More generally, we have
  \begin{IEEEeqnarray*}{Cl}
    & a_1 \tensor \cdots \tensor a_{n+1}\\
    \cong & (a_1 \tensor \cdots \tensor a_n) \tensor a_{n+1} \\
    \xrightarrow{\dec} & (a_1 \tensor \cdots \tensor a_n) \sequoid a_{n+1} \times a_{n+1} \sequoid (a_1 \tensor \cdots \tensor a_n) \\
    \cong & \left(\prod_{i=1}^n\left(a_i \sequoid \Tensor_{j\ne i}^{j\le n} a_j\right)\right) \sequoid a_{n+1} \times a_{n+1} \sequoid (a_1 \tensor \cdots \tensor a_n) \\
    \xrightarrow{\dist\times\id} & \prod_{i=1}^n\left(\left(a_i \sequoid \Tensor_{j\ne i}^{j\le n} a_j \right) \sequoid a_{n+1}\right) \times a_{n+1} \sequoid (a_1 \tensor \cdots \tensor a_n) \\
    \xrightarrow{\langle\passoc\rangle\times\id} & \prod_{i=1}^n \left(a_i \sequoid \left(\Tensor_{j\ne i}^{j\le n} a_j \tensor a_{n+1}\right)\right) \times a_{n+1} \sequoid (a_1\tensor \cdots \tensor a_n) \\
    \cong & \prod_{i=1}^{n+1} \left(a_i \times \Tensor_{j\ne i}^{j\le n+1} a_j\right) \times a_{n+1} \sequoid (a_1 \tensor \cdots \tensor a_n) \\
    \cong & \prod_{i=1}^{n+1} (a_i \sequoid (a_1 \tensor \cdots \tensor a_{i-1} \tensor a_{i+1} \tensor \cdots \tensor a_{n+1}))\,,
  \end{IEEEeqnarray*}
  where each of the arrows is an isomorphism.
\end{proof}

The specific isomorphism in Proposition \ref{PropDecSeq} is rather complicated at the moment, but we can simplify it.

\begin{definition}
  Given objects $a_1,\cdots,a_n$ of a monoidal category, we write $\sym_i^n$ for the unique symmetric coherence isomorphism
  \[
    a_1 \tensor \cdots \tensor a_n \cong a_i \tensor a_1 \cdots \tensor a_{i-1} \tensor a_{i+1} \tensor a_n\,.
    \]
\end{definition}

\begin{proposition}
  The isomorphism in the proof of Proposition \ref{PropDecSeq} is given by
  \[
    \dec_{(a_i)}^n = \langle \sym_i^{n};\wk_{a_i,a_1\tensor\cdots\tensor a_{i-1}\tensor a_{i+1}\tensor \cdots\tensor a_n}\rangle\,.
    \]
  \label{PropDecSeqFormula}
\end{proposition}
\begin{proof}
  Induction on $n$.
  We will make use of the coherence theorem for symmetric monoidal categories \cite[\sec 11]{WorkingMathematician} to allow us to elide associators.  
  The base case is obviously true, because $()\from I \to 1$ is the unique morphism between these objects.  
  Otherwise, we observe that the morphism into $\prod_{i=1}^{n+1} \left(a_i \sequoid \Tensor_{j\ne i}^{j\le n+1} a_j\right)$ is given component-wise by morphisms $a_1 \tensor \cdots \tensor a_{n+1} \to a_i \sequoid \Tensor_{j\ne i}^{j\le n+1}$ for each $i=1,\cdots,n+1$; we need to check that each of these components is equal to $\sym_i^{n+1};\wk_{a_i,\Tensor_{j\ne i}a_j}$.

  If $i\le n$, then the $i$-th component of the morphism in the proof of Proposition \ref{PropDecSeq} is given by the composite thick dashed arrows in Figure \ref{FigDecSeqFormula}, and is therefore equal to the composite of the solid arrows, which is equal to $\sym_i^{n+1};\wk_{a_i,\Tensor_{j\ne i}a_j}$ as desired.  
  The $n+1$-th component of the morphism in the proof of Proposition \ref{PropDecSeq} is given by the composite
  \small
  \[
    \Tensor_{j=1}^{n+1}a_j \to \Tensor_{j=1}^n a_j \tensor a_{n+1} \xrightarrow{\sym_{\Tensor_{j=1}^na_j,a_{n+1}}} a_{n+1} \tensor \Tensor_{j=1}^n a_j \xrightarrow{\wk_{a_{n+1},\Tensor_{j=1}^n a_j}} a_{n+1} \sequoid \Tensor_{j=1}^n a_j\,,
    \]
  \normalsize
  and then we use the fact that the leftmost two morphisms in this composite compose to give us $\sym_{n+1}^{n+1}$.
  \begin{SidewaysFigure}
    \[
      \begin{tikzcd}[ampersand replacement=\&, column sep=6pt, row sep=30pt]
        \Tensor_{j=1}^{n+1}a_j \arrow[r, thick, dashed] \arrow[dd, "\sym_i^{n+1}"']
          \& \Tensor_{j=1}^n a_j \tensor a_{n+1} \arrow[r, "{\wk_{\Tensor_{j=1}^na_j,a_{n+1}}}", thick, dashed] \arrow[d, "\sym_i^n\tensor a_{n+1}" description, dotted]
            \&[45.5pt] \Tensor_{j=1}^n a_j \sequoid a_{n+1} \arrow[d, "\sym_i^n\sequoid a_{n+1}", thick, dashed]
              \&[31pt] \\
        %
          \& \left(a_i \tensor \Tensor_{j\ne i}^{j\le n} a_j\right)\tensor a_{n+1} \arrow[r, "{\wk_{a_i\tensor\Tensor_{j\ne i}^{j\le n} a_j,a_{n+1}}}" xshift=-10pt, dotted] \arrow[d, "{\assoc_{a_i,\Tensor_{j\ne i}^{j\le n}a_j,a_{n+1}}}" description, dotted] \arrow[dr, "{\wk_{a_i,\Tensor_{j\ne i}^{j\le n}a_j}\tensor a_{n+1}}" description, dotted]
            \& |[alias=Z]| \left(a_i \tensor \Tensor_{j\ne i}^{j\le n} a_j\right)\sequoid a_{n+1} \arrow[dr, from=Z.east, bend left=25, "{\wk_{a_i,\Tensor_{j\ne i}^{j\le n}a_j}\sequoid a_{n+1}}", thick, dashed]
              \& \\
        a_i \tensor \Tensor_{j\ne i}^{j\le n+1} a_j \arrow[ddrrr, bend right=15, "{\wk_{a_i,\Tensor_{j\ne i}^{j\le n+1}a_j}}"']
          \& a_i \tensor \left(\tensor_{j\ne i}^{j\le n}a_j \tensor a_{n+1}\right) \arrow[l, dotted] \arrow[drr, to=Y.west, bend right=10, "{\wk_{a_i,\Tensor_{j\ne i}^{j\le n}a_j\tensor a_{n+1}}}" description, dotted]
            \& \left(a_i \sequoid \Tensor_{j\ne i}^{j\le n}\right)\tensor a_{n+1} \arrow[r, "{\wk_{a_i\sequoid \Tensor_{j\ne i}^{j\le n},a_{n+1}}}" xshift=-10pt, dotted]
              \& \left(a_i \sequoid \Tensor_{j\ne i}^{j\le n} a_j\right) \sequoid a_{n+1} \arrow[d, "{\passoc_{a_i,\Tensor_{j\ne i}^{j\le n}a_j,a_{n+1}}}" {description, yshift=2pt}, thick, dashed] \\
        %
          \&
            \&
              \& |[alias=Y]| a_i \sequoid \left(\Tensor_{j\ne i}^{j\le n} a_j \tensor a_{n+1}\right) \arrow[d, thick, dashed] \\
        %
          \&
            \&
              \& a_i \sequoid \Tensor_{j\ne i}^{j\le n+1}a_j
      \end{tikzcd}
      \]
    \caption[Diagram used in the proof of Proposition \ref{PropDecSeqFormula}]{Diagram used in the proof of Proposition \ref{PropDecSeqFormula}.  
    The pentagon at the heart of the diagram is the coherence diagram for $\passoc$ and $\wk$ from Definition \ref{DefSequoidalCategory}.}
    \label{FigDecSeqFormula}
  \end{SidewaysFigure}
\end{proof}

\begin{proposition}
  $\G$ is a decomposable sequoidal category.
\end{proposition}
\begin{proof}
  Let $A,B$ be games.  
  By Proposition \ref{PropProductOfTreeEmbeddings}, the strategy 
  \[
    \langle \wk_{A,B},\sym_{A,B};\wk_{A,B}\rangle
    \]
  is given by the tree embedding $\phi$ from $(A \sequoid B) \times (B \sequoid A)$ to $A \tensor B)$ given by
  \[
    \phi(s) = \begin{cases}
      \epsilon & \text{if $s=\epsilon$} \\
      s\vert_{A \sequoid B} & \text{if $s$ takes place entirely within $A\sequoid B$} \\
      s\vert_{B \sequoid A} & \text{if $s$ takes place entirely within $B\sequoid A$}
    \end{cases}\,.
    \]
  We claim that this tree embedding is a bijection.  
  Indeed, it is certainly injective.
  Now let $s\in P_{A\tensor B}$ be a non-empty play.  
  Then, if $s$ begins with a move in $A$, we have $s=\phi((\inj_{A\sequoid B})_*(s))$, and if $s$ begins with a move in $B$, we have $s=\phi((\inj_{B\sequoid A})_*(s)$.  
  Therefore, $\phi$ is a tree isomorphism, so $\dec_{A,B}=\zz_\phi$ is an isomorphism in $\G$.

  Lastly, we have $I=1$ in $\G$, and the unique morphism $I\to 1$ is the identity.
\end{proof}

\begin{definition}[\cite{laird02}]
  A \emph{sequoidal closed category} is an inclusive sequoidal category $\C$ such that $\C$ is a monoidal closed category (with inner hom $\implies$) and such that the map $f\mapsto \Lambda(\wk_{A,B};f)$ defines an isomorphism
  \[
    \Lambda_s \from \C_s(A \sequoid B,C) \toisom \C_s(A,B\implies C)\,.
    \]
\end{definition}

\begin{proposition}
  $\G$ is a sequoidal closed category.
\end{proposition}
\begin{proof}
  Since $\wk_{A,B}$ is an epimorphism and $\Lambda$ is a bijection, the map is certainly injective.  
  Showing that it is surjective comes down to proving that uncurrying of a strict strategy for $A\implies (B \implies C)$ is a strict strategy for $(A\sequoid B) \implies C$.
  Indeed, after the opening move in $C$, in both cases player $P$ must play the next move in $A$.
\end{proof}

\section{A Formula for the Exponential}

\begin{definition}
  Let $\C$ be a symmetric monoidal category.  
  Given objects $A_1,\cdots,A_n$ of $\C$ and a permutation $\pi\in \S_n$, there is a unique canonical symmetry isomorphism
  \[
    \sym^\pi \from A_1 \tensor \cdots \tensor A_n \toisom A_{\pi(1)} \tensor \cdots \tensor A_{\pi(n)}\,.
    \]
  Given an object $A$ of $\C$, an \emph{$n$-th symmetrized tensor power} of $A$ is an equalizer $(A^n,\eq^n)$ for the diagram given by all morphisms of the form
  \[
    \sym_\pi \from A ^{\tensor n} \to A ^{\tensor n}\,.
    \]
  We say that the symmetrized tensor power $A^n$ \emph{commutes with the tensor product} if $(B\tensor A^n,B\tensor \eq_n)$ is an equalizer for the diagram given by morphisms of the form
  \[
    B\tensor\sym_\pi \from B\tensor A ^{\tensor n} \to B\tensor A ^{\tensor n}\,.
    \]
\end{definition}

\begin{proposition}[\cite{CalcoPaper}]
  Let $\C$ be an inclusive, strongly distributive, decomposable sequoidal category.  
  Then $\C$ has all symmetrized tensor powers.
  \label{PropSequoidPower}
\end{proposition}
\begin{proof}
  Let $A$ be an object of $\C$ (equivalently, an object of $\C_s$).  
  We inductively define objects $A^{\sequoid n}$ by
  \begin{itemize}
    \item $A^{\sequoid 0} = I$; and
    \item $A^{\sequoid (n+1)}=J(A\sequoid A^{\sequoid n})$.
  \end{itemize}
  We claim that $A^{\sequoid n}$ is a symmetrized tensor power of $A$.

  Given $n$, we inductively define a morphism $\wk^n\from A^{\tensor n} \to A^{\sequoid n}$, where $\wk^0=\id_I$, and $\wk^{n+1}$ is given by the composite
  \[
    A^{\tensor (n+1)} \to A \tensor A^{\tensor n} \xrightarrow{A \tensor \wk^n} A \tensor A^{\sequoid n} \xrightarrow{\wk_{A,A^{\sequoid n}}} A \sequoid A^{\sequoid n}\,.
    \]
  We show by induction on $n$ that if $B$ is an object of $\C$ and $k\ge 0$ then the composite
  \scriptsize
  \begin{mathpar}
    B\tensor (A \sequoid\blank)^kA^{\tensor n}
    \xrightarrow{\langle B\tensor(A \sequoid\blank)^k \sym^\pi \rangle}
    (B\tensor(A \sequoid\blank)^kA^{\tensor n})^{n!}
    \xrightarrow{(B\tensor(A\sequoid\blank)^k\wk^n)^{n!}}
    (B\tensor A^{\sequoid (k+n)})^{n!}
  \end{mathpar}
  \normalsize
  (i.e., the morphism $\langle B\tensor(A\sequoid\blank)^k(\sym^\pi;\wk^n)\rangle$) is a monomorphism.
  In particular, taking $k=0$, we will have shown that $\langle B\tensor (\sym^\pi;\wk^n)\rangle$ is a monomorphism.

  The hypothesis is clearly true for $n=0$; in the general case, we have a composite
  \begin{mathpar}
    B\tensor (A\sequoid\blank)^kA^{\tensor (n+1)}
    \xrightarrow{B\tensor (A\sequoid\blank)^k\langle \sym^{n+1}_i;\wk_{A,A^{\tensor n}}\rangle}
    B\tensor (A \sequoid\blank)^k(A\sequoid A^{\tensor n})^{n+1}
    \xrightarrow{\langle B\tensor (A\sequoid\blank)^k;\pr_i\rangle}
    (B\tensor(A \sequoid\blank)^{k+1} A^{\tensor n})^{n+1}
    \to
    (B\tensor A ^{\sequoid (k + n + 1)})^{(n+1)!}\,,
  \end{mathpar}
  where the last arrow is the tensor product of $B$ with the $(n+1)$-th power of the composite given by
  \footnotesize
  \[
    (A \sequoid\blank)^{k+1}A^{\tensor n}
    \xrightarrow{\langle (A \sequoid \blank)^{k+1}\sym^\sigma\rangle}
    ((A \sequoid\blank)^{k+1}A^{\tensor n})^{n!}
    \xrightarrow{((A \sequoid\blank)^{k+1}\wk^n)^{n!}}
    (A^{\sequoid (k+n+1)})^{n!}\,,
    \]
  \normalsize
  which is a monomorphism by the induction hypothesis.
  Then the previous composite is the composite of monomorphisms (by our assumptions on $\C$), and is therefore itself a monomorphism.  
  Now this composite may be written as
  \[
    \langle B\tensor(A\sequoid\blank)^k(\sym_i^{n+1};\wk_{A,A^{\tensor n}};(A\sequoid\sym^\sigma);(A\sequoid\wk^n)) \rangle\,,
    \]
  which, since $\wk$ is a natural transformation, is equal to
  \[
    \langle B\tensor(A \sequoid\blank)^k(\sym_i^{n+1};(A\tensor\sym^\sigma);(A\tensor\wk^n);\wk_{A,A^{\sequoid n}})\rangle\,,
    \]
  where $\sigma$ ranges over the permutations in $\S_n$.
  Moreover, by the definition of $\wk^n$, this is equal to
  \[
    \langle B\tensor(A \sequoid\blank)^k(\sym_i^{n+1};(A \tensor \sym^\sigma);\wk^{n+1}\rangle\,.
    \]

  Now, given $i\in\{1,\cdots n+1\}$ and $\sigma\in\S_n$, there is a unique permutation $\pi\in\S_{n+1}$ such that $\sym_i^{n+1};(A\tensor\sym^\sigma=\sym^\pi)$; moreover, this defines a bijection from $\{1,\cdots,n+1\}\times\S_n \to \S_{n+1}$.  
  Therefore (after choosing an appropriate enumeration of our permutations), we see that this composite is in fact equal to
  \[
    \langle B\tensor(A \sequoid\blank)^k (\sym^\pi;\wk^{n+1})\rangle\,.
    \]
  Therefore, $\langle B\tensor(A \sequoid\blank)^k(\sym^\pi;\wk^{n+1})\rangle$ is a monomorphism as desired, completing the induction.

  Next, we define morphisms $\eq_n\from A^{\sequoid n} \to A^{\tensor n}$ inductively, where $\eq_0=\id$ and $\eq_{n+1}$ is defined by the following composite
  \[
    A^{\sequoid (n+1)} = A\sequoid A^{\sequoid n} \xrightarrow{\langle (A\sequoid \eq_n)_1^n\rangle} (A \sequoid A^{\tensor n})^n \cong A^{\tensor (n+1)}\,,
    \]
  where the final isomorphism is as in Propositions \ref{PropDecSeq} and \ref{PropDecSeqFormula}.

  First, we show inductively that $\eq_n;\sym^\pi;\wk^n=\id_{A^{\sequoid n}}$ for all permutations $\pi$ of $S_n$.
  This is certainly true for $n=0$; in the general case, let $\pi\in \S_{n+1}$ be a permutation.  
  Let $j=\pi\inv(1)$ be the element sent to $1$ by $\pi$ and let $\sigma$ be the permutation of $1,\cdots,n$ such that applying $\sigma$ to the elements $2,\cdots,n+1$ and composing with $\pi$ gives us the $j$-cycle $(1\,\dots\,j)$.  
  Then we have
  \[
    \sym_j^{n+1};(A\tensor\sym^\sigma) = \sym^\pi\,.
    \]
  Now we get
  \begin{IEEEeqnarray*}{Cl}
    & \eq_{n+1};\sym^\pi;\wk^{n+1} \\
    = & \langle (A \sequoid \eq_n)_1^n\rangle;(\dec_{\vec A}^{n+1})\inv;\sym^\pi;(A \tensor \wk^n);\wk_{A,A^{\sequoid n}} \\
    = & \langle (A \sequoid \eq_n)_1^n\rangle;(\dec_{\vec A}^{n+1})\inv;\sym_j^{n+1};(A\tensor\sym^\sigma);(A \tensor \wk^n);\wk_{A,A^{\sequoid n}} \\
    = & \langle (A \sequoid \eq_n)_1^n\rangle;(\dec_{\vec A}^{n+1})\inv;\sym_j^{n+1};\wk_{A,A^{\tensor n}};(A \sequoid \sym^\sigma);(A \sequoid \wk^n) \\
    = & \langle (A \sequoid \eq_n)_1^n\rangle;(\dec_{\vec A}^{n+1})\inv;\langle \sym_i^{n+1};\wk_{A,A^{\tensor n}}\rangle;\pr_j;(A \sequoid \sym^\sigma);(A \sequoid \wk^n) \\
    = & A \sequoid (\eq_n;\sym^\sigma;\wk^n)\,,
  \end{IEEEeqnarray*}
  which is equal to the identity on $A^{\sequoid (n+1)}$ by the induction hypothesis.

  Now let $\rho$ be a permutation in $\S_n$.  
  We claim that $\eq_n=\eq_n;\sym^{\rho}$.  

  Indeed, we have
  \begin{IEEEeqnarray*}{rCl}
    \eq_n;\langle \sym^{\pi} ; \wk^n \rangle
    & = & \langle \eq_n;\sym^{\pi};\wk^n \rangle \\
    & = & \langle \id \rangle \\
    & = & \langle \eq_n;\sym^{\rho\pi};\wk^n \rangle \\
    & = & \eq_n;\sym^{\rho};\langle \sym^\pi ; \wk^n \rangle\,.
  \end{IEEEeqnarray*}
  Since $\langle \sym^\pi;\wk^n\rangle$ is a monomorphism, this means that $\eq_n=\eq_n;\sym^{\rho}$, as desired.
  Therefore, $\eq_n$ equalizes the morphisms $\eq_n$.  
  We claim that it is an equalizer, and that this equalizer is preserved by the tensor product.

  Indeed, let $B,C$ be objects of $\C$, and let $f \from C \to B \tensor A^{\tensor n}$ be a morphism such that $f=f;(B\tensor\sym^{\pi})$ for all $\pi\in\S_n$.

  Let $\tilde f = f;(B \tensor \wk^n)\from C \to B \tensor A^{\sequoid n}$.  
  We claim that $\tilde f;(B\tensor\eq_n)=f$; indeed, we have
  \begin{IEEEeqnarray*}{rCl}
    \tilde f;(B \tensor\eq_n);\langle B\tensor(\sym^\pi;\wk^n)\rangle 
    & = & \langle f;(B \tensor (\wk^n;\eq_n;\sym^\pi;\wk^n)) \rangle \\
    & = & \langle f;(B\tensor \wk^n)\rangle \\
    & = & f;\langle B \tensor (\sym^\pi;\wk^n)\rangle\,.
  \end{IEEEeqnarray*}
  Therefore, since $\langle B \tensor (\sym^\pi;\wk^n)\rangle$ is a monomorphism, we know that $\tilde f;(B\tensor \eq_n)=f$.

  Now suppose that $h\from C \to B \tensor A^{\sequoid n}$ is such that $h;(B\tensor eq_n)=f$.  
  We claim that $h=\tilde f$.  
  Indeed, we have 
  \[
    \tilde f = f;(B \tensor \wk^n) = h;(B \tensor \eq_n);(B \tensor \wk^n) = h\,.
    \]
  Therefore, $(B\tensor A^{\sequoid n},B\tensor \eq_n)$ is an equalizer of the arrows $B\tensor \sym^\pi\from B \tensor A^{\tensor n} \to B \tensor A^{\tensor n}$, as desired.
\end{proof}

We are interested in symmetrized tensor powers because of an important result of \Mellies, Tabareau and Tasson.
Suppose $\C$ is a monoidal category, and that $\C$ has symmetrized tensor powers that commute with the tensor product.  
Given $n$, we have a morphism
\[
  A^{\tensor n}\tensor() \from A^{\tensor (n+1)} \to A^{\tensor n}\,,
  \]
where $()$ is the unique morphism into the terminal object.  
Then, if $A^n$ and $A^{n+1}$ are the $n$-th and $n+1$-th symmetrized tensor powers of $A$, and $\eq_{n+1},\eq$ the corresponding equalization, for any $\pi\in\S_n$ we have a commutative diagram
\[
  \begin{tikzcd}[column sep=50pt]
    B \tensor A^{n+1} \arrow[r, "B\tensor\eq_{n+1}"]
      & B \tensor A^{\tensor (n+1)} \arrow[r, "B\tensor A^{\tensor n}\tensor()"] \arrow[d, "\sym^{\pi'}"']
        & B \tensor A^{\tensor n} \arrow[d, "\sym^\pi"] \\
    %
      & B \tensor A^{\tensor (n+1)} \arrow[r, "B\tensor A^{\tensor n} \tensor()"]
        & B \tensor A^{\tensor n}
  \end{tikzcd}\,,
  \]
where $\pi'$ is the permutation of $1,\cdots,n+1$ that fixes $1$ and applies $\pi$ to the remaining elements $2,\cdots,n+1$.

This means that for each $\pi\in \S_n$ we have
\begin{IEEEeqnarray*}{rCl}
  (B\tensor\eq_{n+1});(B\tensor A^{\tensor n}\tensor ())
  & = & (B \tensor\eq_{n+1});(B\tensor \sym^{\pi'});(B\tensor A^{\tensor n}\tensor ()) \\
  & = & (B\tensor\eq_{n+1});(B\tensor A^{\tensor n}\tensor());(B\tensor \sym^\pi)\,,
\end{IEEEeqnarray*}
and that there is therefore an induced morphism
\[
  B\tensor A^{n+1} \to B \tensor A^n\,,
  \]
by the universal property of the equalizer.

Note also that if $m,n$ are integers, then any permutations $\sigma$ of $1,\cdots,m$ and $\pi$ of $1,\cdots,n$ induce a permutation $[\sigma,\pi]$ of $1,\cdots,m+n$.  
Then we get morphisms
\[
  A^{m+n} \to A^{\tensor (m+n)} \to A^{\tensor m} \tensor A^{\tensor n}\,,
  \]
which are equalized by all symmetries on $A^{\tensor m}$ and $A^{\tensor n}$ individually.  
Since the equalizers $A^m$ and $A^n$ are preserved by the tensor product, then we get an induced morphisms
\[
  A^{m+n}\to A^m\tensor A^n\,.
  \]

\begin{theorem}[\cite{MelliesCofCommCom}]
  Let $\C$ be a monoidal category such that the monoidal unit for $\C$ is a terminal object.
  Suppose that $\C$ has symmetrized tensor powers that commute with the tensor product.  

  Then, for any objects $A,B$ of $\C$, there is a natural sequence
  \[
    B \leftarrow B\tensor A \leftarrow B\tensor A^2 \leftarrow B \tensor A^3 \leftarrow \cdots\,.
    \]
  In particular, there is a sequence
  \[
    I \leftarrow A \leftarrow A^2 \leftarrow A^3 \leftarrow \cdots\,.
    \]
  Suppose that this sequence has a limit $\oc A$, and that $B\tensor \oc A$ is the limit of the first sequence for all $B$.  

  For each $m$, we can define morphisms
  \[
    \oc A \to A^{m+n} \to A^m \tensor A^n
    \]
  for each $n$, which commute with the morphisms $A^{n+1}\to A^n$ and hence induce a morphism $\oc A \to A^m \tensor \oc A$.  
  Then these morphisms themselves commute with the morphisms $A^{m+1}\tensor \oc A \to A^m \tensor \oc A$, and so we get a morphism $\mu_A\from \oc A \to \oc A \tensor \oc A$.
  
  The morphisms $\mu_A\from \oc A \to \oc A \tensor \oc A$ and $()\from A \to I$ give $\oc A$ the structure of a commutative comonoid in $\C$.  
  In fact, this is the \emph{cofree commutative comonoid} over $A$ in $\C$.  
  The counit $\der_A\from \oc A \to A$ of the adjunction is the map in the limiting cone.
  \label{TheMtt}
\end{theorem}

We want to show that this theorem applies in $\G$.
First, we find an explicit formula for the morphisms $B \tensor A^{\sequoid(n+1)}\to B\tensor A^{\sequoid n}$.

\begin{proposition}
  Let $\C$ be an inclusive, strongly distributive, decomposable sequoidal category -- so $\G$ has symmetrized tensor powers preserved by the tensor product as in Proposition \ref{PropSequoidPower}.  

  Then the canonical morphisms $B \tensor A^{\sequoid (n+1)}\to B\tensor A^{\sequoid n}$ are given by
  \[
    B \tensor (A \sequoid \blank)^n()\,,
    \]
  where $() \from A \to I$ is the unique morphism into the terminal object.
\end{proposition}
\begin{proof}
  First, we show by induction on $n$ that the following diagram commutes.
  \[
    \begin{tikzcd}[column sep=30pt]
      A^{\tensor(n+1)} \arrow[r, "A^{\tensor n}\tensor()"] \arrow[d, "A\tensor\wk^{n+1}"']
        & A^{\tensor n} \arrow[d, "\wk^n"] \\
      A^{\sequoid(n+1)} \arrow[r, "(A\sequoid\blank)^n()"]
        & A^{\sequoid n}
    \end{tikzcd}
    \]
  This is clearly true for $n=0$; in the general case, we have the following commutative diagram --
  \[
    \begin{tikzcd}[column sep=41pt]
      A^{\tensor(n+2)} \arrow[r] \arrow[dd, "\wk^{n+2}"']
        & A \tensor A^{\tensor (n+1)} \arrow[r, "A \tensor (A^{\tensor n}\tensor ())"] \arrow[d, "A \tensor \wk^{n+1}"']
          & A \tensor A^{\tensor n} \arrow[r] \arrow[d, "A\tensor \wk^n"]
            & A^{\tensor(n+1)} \arrow[dd, "\wk^{n+1}"] \\
      %
        & A \tensor A^{\sequoid (n+1)} \arrow[r, "A \tensor (A\sequoid\blank)^n()"] \arrow[dl, "{\wk_{A,A^{\sequoid (n+1)}}}" description]
          & A \tensor A^{\sequoid n} \arrow[dr, "{\wk_{A,A^{\sequoid n}}}" description]
            & \\
      A^{\sequoid(n+1)} \arrow[rrr, "(A \sequoid \blank)^{n+1}()"]
        &
          &
            & A^{\sequoid (n+1)}
    \end{tikzcd}\,\text{--}
    \]
  where the middle square is the inductive hypothesis (tensored by $A$), the outer trapezia are the definitions of $\wk^{n+2}$ and $\wk^{n+1}$, and the bottom trapezium commutes because $\wk$ is a natural transformation.

  Now, by the proof of Proposition \ref{PropSequoidPower}, the canonical morphism $B\tensor A^{\sequoid (n+1)} \to B\tensor A^{\sequoid n}$ must be constructed as the composite
  \[
    B \tensor (\eq_{n+1};(A^{\tensor n}\tensor ());\wk^n)\,,
    \]
  which we have shown is equal to
  \[
    B \tensor (\eq_{n+1};\wk^{n+1};(A \sequoid\blank)^n()) = B \tensor (A \sequoid\blank)^n()\,.\qedhere
    \]
\end{proof}

\begin{definition}
  Let $A$ be a game.  
  We say that $A$ is \emph{well-opened} if initial moves of $A$ can only occur as the very first move in a play.
\end{definition}

It is immediate from the definitions that:
\begin{itemize}
  \item the empty game $I$ and our data-type games $\bC,\bB,\bN$ are well opened;
  \item if $A_i$ are well-opened games, then so is $\prod_iA_i$; and
  \item if $B$ is a well-opened game, then so is $A \implies B$;
\end{itemize}
but that $A \tensor B$, $A\sequoid B$ and $\oc A$ are not in general well-opened, even if $A$ and $B$ are.

\begin{proposition}
  Let $A$ be a well-opened game.  
  Then we have natural morphisms
  \[
    \oc A \to A^{\sequoid n}
    \]
  for each $n$, and these commute with the natural morphisms $A^{\sequoid(n+1)}\to A^{\sequoid n}$ and make $\oc A$ the limit of the sequence
  \[
    I \leftarrow A \leftarrow A^{\sequoid 2} \leftarrow A^{\sequoid 3}\leftarrow\cdots\,.
    \]
  Moreover, $B\tensor\oc A$ is the limit of the sequence
  \[
    B \leftarrow B \tensor A \leftarrow B \tensor A^{\sequoid 2} \leftarrow B \tensor A^{\sequoid 3}\leftarrow \cdots\,.
    \]
\end{proposition}
\begin{proof}
  For the sake of notational simplicity, we will only prove the first part of the Proposition, but the second part (the $B\tensor\cdots$ version) goes through in exactly the same way.

  The morphism in question is the (non-innocent) zigzag strategy given by the tree embedding $\phi_n\from P_{A^{\sequoid n}}\to P_{\oc A}$ defined by
  \[
    \phi_n(s) = \nabla_*(s)\,,
    \]
  where $\nabla\from M_A+\cdots + M_A$ is the co-diagonal.

  We have seen already that the natural morphism $A^{\sequoid (n+1)}\to A^{\sequoid n}$ is the copycat morphism generated by the inclusion $n.(M_A) \to (n+1).(M_A)$, and so it is clear that these commute with $\zz_\phi$ by Proposition \ref{PropTree}.

  Now let $C$ be a game and suppose that there are strategies $\sigma_n \from C \implies A^{\sequoid n}$ that commute with the natural morphisms $A^{\sequoid (n+1)} \to A^{\sequoid n}$.  
  Then we define a morphism $\sigma\from C \to \oc A$ by
  \[
    \sigma = \left\{s\in P_{C\implies \oc A} \,\middle|\, \mbox{\pbox{\textwidth}{
      for some $n$, $s\vert_{\oc A}$ contains \\
      at most $n$ initial moves, and \\
      $[\inj,\inj_{A_1},\cdots,\inj_{A_n}]_*(s)\in\sigma_n$.}}\right\}\,.
    \]
  Here we have used the fact that $A$ is well-opened to tell us that $s\vert_{\oc A}$ is indeed a valid play in $A^{\sequoid n}$.

  We claim that $\sigma$ is indeed a strategy.
  First we show that $\sigma$ is prefix closed.  
  If $s\in \sigma$ and $t\prefix s$, write $n$ for the number of initial moves in $s$.  
  Then $t$ has at most $n$ initial moves; if $[\id_C,\inj_{A_1},\cdots,\inj_{A_n}]_*(s)\in\sigma_n$, then $[\id_C,\inj_{A_1},\cdots,\inj_{A_n}]_*(t)\in\sigma_n$, and therefore $t\in\sigma$.

  Now note that if $s\in C_{C\implies \oc A}$ is such that $s\vert_{\oc A}$ contains $k$ initial moves, and if $m,n\ge k$, then
  \[
    [\inj,\inj_{A_1},\cdots,\inj_{A_m}]_*(s) = [\inj_C,\inj_{A_1},\cdots,\inj_{A_n}]_*(s)\,,
    \]
  since the $\sigma_n$ commute with the natural morphisms $A^{\sequoid (n+1)} \to A^{\sequoid n}$.  
  So, if $sab,sac\in\sigma$, then we can assume that
  \[
    [\inj_C,\inj_{A_1},\cdots,\inj_{A_k}]_*(sab),[\inj_C,\inj_{A_1},\cdots,\inj_{A_k}]_*(sac)\in\sigma_k
  \]
  for some common $k$, and therefore that $b=c$.

  Now we have $\sigma;\zz_{\phi_n}=\sigma_n$ for each $n$ by Proposition \ref{PropTree}.

  Suppose that $\tau$ is some other strategy for $C\implies \oc A$ such that $\tau;\zz_{\phi_n}=\sigma_n$ for each $n$.  
  Cy Proposition \ref{PropTree}, we have
  \[
    \sigma_n = \tau;\zz_{\phi_n} = \{s\in P_{C\implies A^{\sequoid n}}\suchthat s^{\phi_n}\in\tau\}\,.
    \]
  Suppose $s\in\sigma$ and that $s$ contains $n$ initial moves.  
  Then 
  \[
    [\inj_C,\inj_{A_1},\cdots,\inj_{A_n}]_*(s)\in\sigma_n
    \]
  for some $n$.  
  Therefore, $s=([\inj_C,\inj_{A_1},\cdots,\inj_{A_n}]_*(s))^{\phi_n}\in\tau$.  
  So $\sigma\subset\tau$.
  
  Conversely, suppose that $t\in\tau$.  
  Suppose that $t\vert_{\oc A}$ contains $n$ initial moves.  
  Then $t=s^{\phi_n}$ for some sequence $s\in P_{C\implies A^{\sequoid n}}$, and we must have $s\in\sigma_n$.  
  Therefore, $t\in\sigma$.  
  So $\tau\subset\sigma$.
\end{proof}

Therefore, by Theorem \ref{TheMtt}, if $A$ is a well-opened game, then $\oc A$ inherits the structure of a cofree commutative comonoid on $A$.

Let $\G_{wo}$ denote the category of well-opened games and strategies.  
Let $\CCom(\G)$ denote the category of commutative comonoids with respect to the symmetric monoidal structure on $\G$.

In general, given two commutative comonoids $M,N$ in a symmetric monoidal category $\C$, we can form the \emph{tensor product}
\begin{mathpar}
  M \tensor N \to (M \tensor M) \tensor (N \tensor N) \to (M \tensor N) \tensor (M \tensor N)
  \and
  M \tensor N \to I \tensor I \to I\,,
\end{mathpar}
and this makes $\CCom(\C)$ into a \emph{Cartesian} category.  

Now note that we have defined a functor
\[
  \G_{wo} \to \CCom(\G)
  \]
which is a right adjoint on to its image, and therefore preserves products.  
So if $A,B$ are well-opened games, then we get a natural isomorphism of comonoids between the tensor product of the comonoids on $\oc A$ and $\oc B$ and the comonoid on $\oc(A\times B)$.
In particular, we have a natural isomorphism
\[
  \oc A \tensor \oc B \cong \oc (A \times B)\,.
  \]
We define a category $\G_{wo}^{\oc}$ to be the image of the functor $\G_{wo}\to \CCom(\G)$ inside $\CCom(\G)$.  
By our discussion above, $\G_{wo}^{\oc}$ is a Cartesian category.  

A more convenient description of $\G_{wo}^{\oc}$ is that it is the category where the objects are well-opened games and where the morphisms
\[
  A \to B
  \]
are morphisms $\oc A \to B$ in the original category $\G$.
We compose two such morphisms $\sigma \from \oc A \implies B$ and $\tau \from \oc B \implies C$ as
\[
  \oc A \xrightarrow{\sigma^\dag} \oc B \xrightarrow{\tau} C\,,
  \]
where the \emph{promotion} $\sigma^\dag$ of $\sigma$ comes from the description of $\oc A$ as the cofree commutative comonoid on $A$.

Since the functor $\G_{wo}\to \CCom(\G)$ preserves products, $\G_{wo}^{\oc}$ obtains a Cartesian structure given by the category-theoretic product $\times$.  
We claim that it is Cartesian closed, with the function object from $A$ to $B$ given by $\oc A \implies B$.  
Indeed, we have
\begin{IEEEeqnarray*}{rCl}
  \G_{wo}^{\oc}(A,\oc B\implies C)
  & \cong & \G(\oc A,\oc B \implies C) \\
  & \cong & \G(\oc A \tensor \oc B,C) \\
  & \cong & \G(\oc(A \times B),C) \\
  & \cong & \G_{wo}^{\oc}(A\times B,C)\,.
\end{IEEEeqnarray*}

We have one thing left to prove.
\begin{proposition}
  Let $\sigma\from \oc A \implies B$, $\tau\from \oc B \implies C$ be innocent strategies, where $A,B,C$ are well-opened games.  
  Then the composite of $\sigma$ and $\tau$ in $\G_{wo}^{\oc}$ is an innocent strategy.
\end{proposition}
\begin{proof}
  We may write this composite as
  \[
    \oc A \xrightarrow{m}
    \oc \oc A \xrightarrow{\oc\sigma}
    \oc B \xrightarrow{\tau}\,,
    \]
  where $\oc\sigma$ is the strategy formed by playing $\sigma$ in parallel with itself, whereas $m\from \oc A \to \oc \oc A$ comes from the fact that $\oc A$ is the cofree commutative comonoid on $A$.
  $\oc \sigma$ is innocent for the same reasons as in Proposition \ref{PropTensorInnocent}, so it suffices to show that $m$ is innocent.  

  Indeed, it is clear from our definitions that $\oc\oc A = \oc A$ as games.  
  Then, $m$ is in fact the identity strategy between these two games, so is innocent.
\end{proof}

\section{The Exponential as a Final Coalgebra}

\begin{definition}
  Let $F \from \C \to \C$ be a functor.  
  A \emph{coalgebra for $F$} or \emph{$F$-coalgebra} is an object $a$ of $\C$, together with a morphism $f\from a \to Fa$.

  A \emph{coalgebra homomophism} from $(a,f)$ to $(b,g)$ is a morphism $h\from a \to b$ such that the following diagram commutes.
  \[
    \begin{tikzcd}
      a \arrow[r, "f"] \arrow[d, "h"']
        & Fa \arrow[d, "Fh"] \\
      b \arrow[r, "g"]
        & Fb
    \end{tikzcd}
    \]

  Clearly, the coalgebras for a given functor $F$ form a category.  
  A \emph{final coalgebra} for $F$ is a terminal object for this category; i.e., an $F$-coalgebra $(t,\alpha)$ such that for all $F$-coalgebras $(a,f)$ there is a unique morphism $\fcoal f \from a \to t$ such that the following diagram commutes.
  \[
    \begin{tikzcd}
      a \arrow[r, "f"] \arrow[d, "\fcoal f"']
        & Fa \arrow[d, "F\fcoal f"] \\
      t \arrow[r, "\alpha"]
        & Ft
    \end{tikzcd}
    \]
\end{definition}
We call $\fcoal f$ the \emph{anamorphism} of $f$.

We use two standard pieces of theory about coalgebras.  

\begin{theorem}[Lambek's Theorem, \cite{Lambek}]
  If $(t,\alpha)$ is a final coalgebra for a functor $F$, then $\alpha\from t \to Ft$ is an isomorphism with inverse given by $\fcoal{F\alpha}$.
\end{theorem}

\begin{theorem}[Ad{\'{a}}mek's Theorem, {\cite{Adamek}}]
  Suppose $\C$ has a terminal object $1$.  
  By repeatedly applying $F$ to the morphism $F1 \to 1$, we build up a sequence
  \[
    1 \leftarrow F1 \leftarrow F^2 1 \leftarrow F^31 \leftarrow \cdots\,.
    \]
  If this sequence has a limit $F^{\omega}1$, and if the morphism $\beta\from F(F^{\omega}1) \to F^{\omega}1$ induced from the univeral property of the limit is an isomorphism, then $(F^{\omega}1,\beta\inv)$ is a final coalgebra for $F$.
\end{theorem}

Now we have already shown that if $A$ is well-opened, then $\oc A$ is the limit of the sequence
\[
  I \leftarrow A \leftarrow A^{\sequoid 2} \leftarrow A^{\sequoid 3} \leftarrow \cdots\,,
  \]
and this sequence is precisely the sequence from Ad{\'a}mek's Theorem, when $F=J(A\sequoid\blank) \from \G \to \G$.
Moreover, this limit is preserved when taking the sequoid with $A$ on the left, and so we get that
\begin{corollary}
  If $A$ is a well-opened game, then $\oc A$ is the final coalgebra for the functor $A \sequoid \blank$.
\end{corollary}
In this case, the morphism $\oc A \to A \sequoid \oc A$ is the zigzag strategy that plays copycat between the different copies of $A$; i.e., $\zz_\phi$, where $\phi\from P_{A\sequoid \oc A} \to P_{\oc A}$ is the tree isomorphism given by
\[
  \phi(s) = [\inj_{M_A},\id]_*(s)\,.
  \]

One small thing we need to do is to relate the two structures on the exponential.

\begin{proposition}
  The final coalgebra
  \[
    \alpha\from \oc A \to A \sequoid \oc A
    \]
  is given by the composite
  \[
    \oc A \xrightarrow{\mu_A} \oc A \tensor \oc A \xrightarrow{\der_A\tensor \oc A} A \tensor \oc A \xrightarrow{\wk_A,\oc A} A \sequoid \oc A\,.
    \]
  \label{PropFormulaForAlpha}
\end{proposition}
\begin{proof}
  By Theorem \ref{TheMtt}, we can tell that this composite is a copycat strategy between $\oc A$ and $A \sequoid \oc A$, as is $\alpha$.  
  Since $A$ is well-opened, there is a unique such strategy.
\end{proof}

\section{Denotational Semantics of Idealized Algol}

We now come to interpret Idealized Algol within our category $\G_{wo}^{\oc}$.  
The base types $\com$, $\bool$ and $\nat$ are interpreted by the games $\bC$, $\bB$ and $\bN$, while the type $\Var$ is interpreted by the game
\[
  \Varr = \bC^{\bN} \times \bN\,,
  \]
where $\bC^{\bN}$ is the product of $\bN$-many copies of $\bC$.
Given types $S,T$, the denotation $\deno{S\to T}$ of the type of functions from $S$ to $T$ is given by
\[
  \deno{S}\to\deno{T} \coloneqq \oc\deno{S}\implies\deno{T}\,.
  \]
This gives us the denotation of the types of Idealized Algol.  

We inductively define a denotation of terms-in-context $\Gamma\ts M$ of IA, where $\deno{x_1\from T_1,\cdots,x_n\from T_n\ts M\from T}$ is a strategy
\[
  \deno{T_1} \times \cdots \times \deno{T_n} \to \deno{T}\,.
  \]
First note that we have natural innocent strategies $a \from \bC$, $\true,\false\from\bB$ and $n \from \bN$, which give us the denotations of $\Gamma\ts\skipp$, $\Gamma\ts\true$, $\Gamma\ts\false$ and $\Gamma\ts n$.

Moreover, if we have a strategy

\[
  \deno{\Gamma,x\from S \ts M\from T} \from \deno{\Gamma}\times\deno{S} \to \deno{T}\,,
  \]
then, since $\G_{wo}^{\oc}$ is Cartesian closed, we get a strategy
\[
  \deno{\Gamma\ts \lambda x^s.M} \from \deno{\Gamma} \to \deno {S} \to \deno{T}\,.
  \]
In addition, we have natural morphisms
\[
  \deno{\Gamma,x\from T\ts x\from T} = \deno{\Gamma}\times\deno{T} \xrightarrow{\pr_{\deno{T}}} \deno{T}\,.
  \]
Lastly, if we have strategies
\begin{mathpar}
  \deno{\Gamma\ts M\from S \to T} \from \deno{\Gamma} \to \deno{S} \to \deno{T}
  \and
  \deno{\Gamma\ts N \from S} \from \deno{\Gamma} \to \deno{S}\,,
\end{mathpar}
then we get a strategy
\[
  \deno{\Gamma\ts MN\from T} = \deno \Gamma \xrightarrow{\Delta} \deno\Gamma\times\deno\Gamma \xrightarrow{\deno{\Gamma\ts M}\times\deno{\Gamma\ts N}} (\deno S \to \deno T) \times \deno S \xrightarrow{\ev} \deno{S}\,.
  \]

In order to form the denotation of the next lot of terms, we need a new definition.
\begin{definition}
  Let $X$ be a set, and let $(\sigma_x\suchthat x\in X)$ be a collection of strategies for a game $A$.  
  Write $X$ for the datatype game corresponding to $X$.  
  Then we define a strict strategy $(\sigma_x) \from X \to A$ by
  \[
    (\sigma_x) = \{\epsilon\}\cup\{*q\suchthat {*}\in P_A\}\cup \{*qys\suchthat {*}s\in\sigma_y\}\,.
    \]
  In other words, after the initial move in $A$, $(\sigma_x)$ requests some element $y\in X$, and thereafter plays according to $\sigma_y$ in $A$.
\end{definition}

\begin{proposition}
  If the $\sigma_x$ are innocent strategies, then $(\sigma_x)$ is an innocent strategy.
\end{proposition}
\begin{proof}
  If $*s\in\sigma_y$, then $\pv{*qysa}=*qy\pv{sa}$.  
  Then, if $t\in(\sigma_x)$ and $\pv{ta}=*qy\pv{sa}$, we have $t=*qyt'$ for $t'\in\sigma_y$ and $\pv{*t'a}=\pv{*sa}$.  
  So if $*sab\in\sigma_y$, then $*t'ab\in\sigma_y$ and therefore $tab\in(\sigma_x)$.
\end{proof}

The most important feature of strategies of the form $(\sigma_x)$ is the most obvious one: given $X$, and $y\in X$, we have a strategy $y$ for the game $X$ with maximal play $qx$.  
Then $y;(\sigma_x) = \sigma_y$.

Now we define morphisms
\begin{itemize}
  \item $\seq_X = (\id_X) \from \bC \to (X \to X)$;
  \item $\If_X = (\lambda x.\lambda y.x,\lambda x.\lambda y.y) \from \bB \to (X \to X \to X)$;
  \item $\suc = (1, 2, 3, 4,\cdots) \from \bN \to \bN$;
  \item $\pred = (0,0,1,2,\cdots)\from \bN \to \bN$;
  \item $\IfO_X = (\lambda x.\lambda y.x,\lambda x.\lambda y.y,\lambda x.\lambda y.y,\cdots)\from \bN \to (X \to X \to X)$;
  \item $\assign = (\pr_0,\pr_1,\cdots) \from \bN \to (\Varr \to \bC)$; and
  \item $\dereff = \pr_\bN \from \Varr \to \bN$;
  \item $\mkvar = \lambda w.\lambda r.\langle (w\,n)_{n\in\bN},r\rangle \from (\bN \to \bC) \to \bN \to \Varr$.
\end{itemize}
These give us an obvious way to interpret most of the rest of the terms of Idealized Algol.  
For example, if we have strategies
\begin{mathpar}
  \deno{\Gamma\ts V\from \Varr}
  \and
  \deno{\Gamma\ts E\from \bN}\,,
\end{mathpar}
then we get a strategy
\[
  \deno{\Gamma\ts V\gets E \from \bC} = \deno{\Gamma} \xrightarrow{\Delta} \deno{\Gamma}\times\deno{\Gamma} \xrightarrow{\deno{\Gamma\ts E}\times\deno{\Gamma\ts V}} \bN \times \Varr \xrightarrow{\assign} \bC\,.
  \]

\section{Order-Enrichment of $\G$}

The remaining parts of Idealized Algol that we have yet to define are the fixpoint combinator $\Y_T$ and the new variable constructor $\neww$.  

To define $\Y_T$, we use order-enriched properties of $\G$.  

Note that if $A$ is a game, then we can order the strategies for $A$ by subset inclusion.  
Then this order is clearly preserved by composition.  

\begin{proposition}
  The partial order of strategies for $A$, ordered by inclusion, is directed-complete.  
  So is the partial order of innocent strategies for $A$.
\end{proposition}
\begin{proof}
  Let $\Sigma$ be a directed set of strategies for $A$; so if $\sigma,\tau\in\Sigma$ then there is some $\upsilon\in\Sigma$ such that $\sigma\subset\upsilon$ and $\tau\subset\upsilon$.  
  We claim that $\bigcup\Sigma$ is a strategy for $A$.  
  Indeed, it is certainly even-prefix-closed, and if $sab,sac\in\Sigma$, then $sab\in\sigma$ and $sac\in\tau$ for $\sigma,\tau\in\Sigma$, and therefore $sab,sac\in\upsilon$ for some $\upsilon\in\Sigma$ and so $b=c$.

  Now suppose that all $\sigma\in\Sigma$ are innocent.
  Let $sab\in\bigcup\Sigma$ and suppose that $t\in \bigcup\Sigma$ is such that $\pv{ta}=\pv{sa}$.  
  Then, as before, we have $sab,t\in\upsilon$ for some innocent $\upsilon\in\Sigma$, and therefore $tab\in\upsilon\subset\bigcup\Sigma$.
\end{proof}

It is clear then that composition of strategies is Scott-continuous with respect to this ordering.

Writing $\bot = \{\epsilon\}$ for the bottom strategy for a game $A$, if we have a strategy $\sigma\from A \to A$, then the Kleene fixed point theorem tells us that we may construct a fixed point for $\sigma$ as the union of the chain
\[
  \bot \subset \bot;\sigma \subset \bot;\sigma;\sigma \subset \cdots\,.
  \]
Given a game $A$, we define a strategy $\Y_A\from (A \to A) \to A$ as the fixed point of the strategy
\[
  \lambda F.\lambda f.f(Ff) \from ((A \to A) \to A) \to (A \to A) \to A\,.
  \]
We can then use $\Y_A$ to interpret the term $\Gamma\ts\Y_TM\from T$ for any term $\Gamma\ts M\from T\to T$.

We will later require other order-theoretic properties of the set of strategies for a game $A$.  
Recall that an element $\sigma$ of a directed-complete partially ordered set is called \emph{compact} if whenever we have $\sigma = \bigcup\Sigma$ for some directed set $\Sigma$, then $\sigma\in\Sigma$.  

A little thought convinces us that a strategy $\sigma\from A$ is compact if and only if it is finite as a set of plays; indeed, suppose $\sigma$ is a finite set and $\sigma=\bigcup\Sigma$.  
For each $s\in\sigma$, we have $s\in\tau_s$ for some $\tau_s\in\Sigma$; since $\Sigma$ is directed, then there is some $\upsilon\in\Sigma$ such that $\tau_s\in\Sigma$ for each $s$, and therefore $\sigma\subset\upsilon\subset\sigma$.
Conversely, if $\sigma$ is infinite, then by K\"{o}nig's lemma, it either has an infinite branching point (i.e., $s\in\sigma$ such that there are infinitely many plays $sab\in\sigma$) or an infinite branch (i.e., an infinite increasing sequence $s_1\prefix s_2\prefix \cdots$ in $\sigma$).  
In either case, it is easy to construct some directed set $\Sigma$ such that $\sigma=\bigcup\Sigma$ but $\sigma\not\in\Sigma$.

Recall that a directed-complete partial order $P$ is said to be \emph{algebraic} if whenever $p\in P$, the set of compact elements of $P$ lying below $p$ is directed and its supremum is $p$.  

\begin{proposition}
  The set of strategies for a game $A$ is an algebraic directed-complete partial order.
\end{proposition}
\begin{proof}
  Let $\sigma$ be a strategy for a game $A$ and let $\tau_1,\tau_2$ be two finite sub-strategies such that $\tau_1,\tau_2\subset\sigma$.  
  Then $\tau_1\cup\tau_2\subset\sigma$ and is finite; moreover, if $sab,sac\in\tau_1\cup\tau_2$, then $sab,sac\in\sigma$, so $b=c$.  

  Lastly, given any $s\in\sigma$, there is a compact strategy $\sigma_s$ containing $s$; namely
  \[
    \sigma_s = \{t\suchthat \text{$t\prefix s$ has even length}\}\,.\qedhere
    \]
\end{proof}

\section{The Strategy $\cell$}
\label{SecCell}

Now we come to the denotation of $\neww$.  
For this, we shall define a strategy $\cell\from \oc\bN \implies \oc \Varr$ by using the property of $\oc \Varr$ as a final coalgebra.

Given $n\in\bN$, we define a strategy $\wwrite_n\from \oc\bN \implies \bC \sequoid\oc\bN$ by
\[
  \wwrite_n = \oc\bN \xrightarrow{()} I \xrightarrow{\skipp\tensor \oc n} \bC \tensor \oc\bN \xrightarrow{\wk} \bC \sequoid\oc\bN\,.
  \]
Let $\rread \from \oc\bN \implies \bN \sequoid \oc\bN$ be the morphism part $\alpha$ of the limiting coalgebra.  
In other words, by Proposition \ref{PropFormulaForAlpha}, $\rread$ is the composite
\[
  \rread = \oc\bN \xrightarrow{\mu_\bN} \oc\bN \tensor \oc\bN \xrightarrow{\der_\bN\tensor\oc\bN} \bN \tensor \oc\bN \xrightarrow{\wk_{\bN,\oc\bN}} \bN \sequoid\oc\bN\,.
  \]
Then we get a coalgebra $\cell_0\from \oc \bN \implies \Varr \sequoid \oc \bN$ given by
\[
  \oc\bN \xrightarrow{\langle (\wwrite_n)_{n\in\bN}, \rread\rangle}
  (\bC \sequoid \oc\bN)^{\bN} \times (\bN \sequoid \oc\bN) \xrightarrow{\dist_{(\bC)_{\bN},\bN,\oc\bN}\inv}
  (\bC^\bN \times \bN) \sequoid \oc\bN = \Varr \sequoid \oc \bN\,.
  \]
We then take the anamorphism $\cell=\fcoal{\cell_0}\from \oc\bN \implies \oc\Varr$; i.e., $\cell$ is the unique morphism making the following diagram commute.
\[
  \begin{tikzcd}
    \oc\bN \arrow[r, "\cell_0"] \arrow[d, "\cell"']
      & \Varr \sequoid \oc\bN \arrow[d, "\Varr \sequoid \cell"] \\
    \oc\Varr \arrow[r, "\alpha_{\Varr}"]
      & \Varr \sequoid \oc\Varr
  \end{tikzcd}
  \]
Concretely, the strategy $\cell$ behaves as follows.  
When player $O$ plays in $\oc\Var$, he chooses to play either in one of the copies of $\bC$ or in $\bN$.  
If he plays the initial move $q_n$ in the $n$-th copy of $\bC$, player $P$ updates the value she has stored in her head to $n$.  
If he plays the initial move $q$ in $\bN$, then player $P$ replies with this stored value.  
Lastly, if he plays this initial move $q$ without having played in any of the copies of $\bC$, then player $P$ interrogates the argument in order to find out which value to play.

This strategy $\cell$ now gives us a morphism $\neww_A \from \oc(\oc\Varr \implies A) \implies A$ given by
\begin{IEEEeqnarray*}{rCl}
  \oc(\oc\Varr\implies A) & \xrightarrow{\mathmakebox[50pt]{\der}} & (\oc\Varr\implies A) \\
  & \xrightarrow{\mathmakebox[50pt]{\lunit}} & I \tensor (\oc\Varr\implies A) \\
  & \xrightarrow{\mathmakebox[50pt]{\oc0\tensor (\oc\Varr\implies A)}} & \oc\bN \tensor (\oc \Varr\implies A) \\
  & \xrightarrow{\mathmakebox[50pt]{\cell\tensor (\oc\Varr\implies A)}} & \oc\Varr \tensor (\oc \Varr \implies A) \\
  & \xrightarrow{\mathmakebox[50pt]{\ev}} & A\,.
\end{IEEEeqnarray*}

We use this to provide the denotation of the term $\neww$.

\begin{lemma}
  $\cell_0;(\pr_n\sequoid\oc\bN) = \wwrite_n$ for each $n$ and $\cell_0;(\pr_\bN\sequoid\oc\bN) = \rread$, where $\pr_n\from \Varr=\bC^\bN \times \bN \implies \bC$ is the projection on to the $n$-th copy of $\bC$ and $\pr_\bN\from \Varr = \bC^\bN \times \bN \implies \bN$ is the projection on to the copy of $\bN$.
  \label{LemCellProjections}
\end{lemma}
\begin{proof}
  We have
  \[
    \dist_{(\bC)_{\bN},\bN,\oc\bN};\pr_n = \langle(\pr_n\sequoid\oc \bN)_{n\in\bN},\pr_{\bN}\sequoid\oc\bN\rangle;\pr_n = \pr_n\sequoid\oc\bN
    \]
  and
  \[
    \dist_{(\bC)_{\bN},\bN,\oc\bN};\pr_\bN = \langle(\pr_n\sequoid\oc \bN)_{n\in\bN},\pr_{\bN}\sequoid\oc\bN\rangle;\pr_\bN = \pr_\bN\sequoid\oc\bN\,,
    \]
  so
  \begin{IEEEeqnarray*}{Cl}
    & \cell_0;(\pr_n\sequoid\oc\bN) \\
    = & \langle (\wwrite_n)_{n\in\bN},\rread \rangle;\dist_{(\bC)_\bN,\bN,\oc\bN}\inv;(\pr_n\sequoid\oc\bN) \\
    = & \langle (\wwrite_n)_{n\in\bN},\rread \rangle ;\dist_{(\bC)_\bN,\bN,\oc\bN}\inv\dist_{(\bC)_\bN,\bN,\oc\bN};\pr_n \\
    = & \langle (\wwrite_n)_{n\in\bN},\rread\rangle;\pr_n \\
    = & \wwrite_n
  \end{IEEEeqnarray*}
  and
  \begin{IEEEeqnarray*}{Cl}
    & \cell_0;(\pr_\bN\sequoid\oc\bN) \\
    = & \langle (\wwrite_n)_{n\in\bN},\rread \rangle;\dist_{(\bC)_\bN,\bN,\oc\bN}\inv;(\pr_\bN\sequoid\oc\bN) \\
    = & \langle (\wwrite_n)_{n\in\bN},\rread \rangle ;\dist_{(\bC)_\bN,\bN,\oc\bN}\inv\dist_{(\bC)_\bN,\bN,\oc\bN};\pr_\bN \\
    = & \langle (\wwrite_n)_{n\in\bN},\rread\rangle;\pr_\bN \\
    = & \rread\,.\hspace{1em plus 1fill}\qedhere
  \end{IEEEeqnarray*}
\end{proof}

\section{Big-Step Operational Semantics}

We now introduce the operational semantics of Idealized Algol, so that we can prove soundness and adequacy of our semantics for it.

We first define a \emph{canonical form} of the language to be
\begin{itemize}
  \item at type $\com$, the term $\skipp$;
  \item at type $\bool$, the terms $\true$ and $\false$;
  \item at type $\nat$, the numerals $n$; 
  \item at type $\Var$, variable names $x$ of type $\Var$ and expressions of the form $\mkvar W\,R$; and
  \item at type $S\to T$, expressions of the form $\lambda x^S.M$.
\end{itemize}

We define a \emph{$\Var$-context} to be a context $\Gamma$ of the form $x_1\from \Var,\cdots,x_n\from \Var$.
Given a $\Var$-context $\Gamma$, we define a \emph{$\Gamma$-store} to be a function $s$ from the variable names occurring in $\Gamma$ to the natural numbers.  
Given such a store $s$, we write $(s\vert x \mapsto n)$ for the store given by
\[
  (s\vert x\mapsto n)(y) = \begin{cases}
    n & \text{if $y = x$} \\
    s(y) & \text{otherwise}
  \end{cases}\,.
  \]
We now inductively define a relation $\Gamma,s\ts M\converges c,s'$, where
\begin{itemize}
  \item $\Gamma$ is a $\Var$-context; 
  \item $s$ and $s'$ are $\Gamma$-stores;  and
  \item $\Gamma\ts M$, $\Gamma\ts c$ are Idealized Algol terms-in-context, where $c$ is a canonical form.
\end{itemize}
The definition of this relation is shown in Figure \ref{FigIaOpSem}.

\begin{figure}
  \vspace{-27pt}
  \begin{mathpar}
    \inferrule*{ }{\Gamma,s\ts c \converges c,s}
    \and
    \inferrule*{\Gamma,s \ts M \converges \lambda x.M',s' \\ \Gamma,s' \ts M'[N/x] \converges c,s''}{\Gamma,s \ts MN \converges c,s''}
    \and
    \inferrule*{\Gamma,s \ts M(\Y M) \converges c,s'}{\Gamma,s \ts \Y M \converges c,s'}
    \and
    \inferrule*{\Gamma,s\ts M \converges n,s'}{\Gamma,s\ts \suc M \converges n+1,s'}
    \\\and
    \inferrule*{\Gamma,s\ts M \converges n+1,s'}{\Gamma,s\ts \pred M \converges n,s'}
    \and
    \inferrule*{\Gamma,s\ts M \converges 0,s'}{\Gamma,s\ts \pred M \converges 0,s'}
    \and
    \inferrule*{\Gamma,s\ts M \converges \skipp,s' \\ \Gamma,s'\ts N \converges c,s''}{\Gamma,s \ts M;N \converges c,s''}
    \and
    \inferrule*{\Gamma,s\ts M \converges \true,s' \\ \Gamma,s' \ts N \converges c,s''}{\Gamma,s \ts \If M \Then N \Else P \converges c,s''}
    \and
    \inferrule*{\Gamma,s\ts M \converges \false,s' \\ \Gamma,s' \ts P \converges c,s''}{\Gamma,s \ts \If M \Then N \Else P \converges c,s''}
    \and
    \inferrule*{\Gamma,s\ts M \converges 0,s' \\ \Gamma,s' \ts N \converges c,s''}{\Gamma,s \ts \IfO M \Then N \Else P \converges c,s''}
    \and
    \inferrule*{\Gamma,s\ts M \converges n+1,s' \\ \Gamma,s' \ts P \converges c,s''}{\Gamma,s \ts \IfO M \Then N \Else P \converges c,s''}
    \and
    \inferrule*[right=$x\in\Gamma$]{\Gamma,s\ts E \converges n,s' \\ \Gamma,s' \ts V \converges x,s''}{\Gamma,s\ts V \gets E \converges \skipp,(s''\vert x \mapsto n)}
    \and
    \inferrule*[right={$s'(x)=n$}]{\Gamma,s\ts V \converges x,s'}{\Gamma,s\ts !V \converges n,s'}
    \and
    \inferrule*{\Gamma,x\from\Var,(s\vert x\mapsto 0)\ts M \converges c,(s'\vert x\mapsto n)}{\Gamma,s\ts \neww \lambda x.M \converges c,s'}
    \and
    \inferrule*{\Gamma,s\ts E \converges n,s' \\ \Gamma,s'\ts V \converges \mkvar W R,s'' \\ \Gamma,s'' \ts Wn \converges \skipp,s'''}
    {\Gamma,s \ts V\gets E \converges \skipp,s'''}
    \and
    \inferrule*{\Gamma,s\ts V \converges \mkvar W R,s' \\ \Gamma,s'\ts R \converges n,s''}{\Gamma,s\ts !V \converges n,s''}
  \end{mathpar}
  \caption[Operational semantics for Idealized Algol.]{Operational semantics for Idealized Algol. See \cite{RusssThesis} and \cite{SamsonGuyIAActive}.}
  \label{FigIaOpSem}
\end{figure}

\section{Small-Step Operational Semantics}

We also give a small-step operational semantics for Idealized Algol, which will sometimes be easier to work with.  

This time, instead of defining a relation $\Gamma,s\ts M\converges c,s'$, we define a relation $\Gamma,s\ts M\opto \Gamma,\Delta,s'\ts M'$, where
\begin{itemize}
  \item $\Gamma,\Delta$ are disjoint $\Var$-contexts; 
  \item $s$ is a $\Gamma$-store and $s'$ a $\Gamma,\Delta$-store;  and
  \item $\Gamma\ts M$, $\Gamma,\Delta\ts M'$ are Idealized Algol terms-in-context.
\end{itemize}

As an auxiliary definition, we need the notion of an \emph{evaluation context} (see \cite{Felleisen}).
This is a single-holed context defined inductively by the following BNF formula, where $M$ ranges over IA terms (subject to typing rules).
\begin{center}
\parbox{0.8\textwidth}{
\begin{mathpar}
  \EE \Coloneqq - \mid \EE\,M \mid \suc \EE \mid \pred \EE \mid \EE;M \mid \If \EE\Then M \Else M \mid \IfO \EE \Then M \Else M \mid !\EE \mid M \gets \EE \mid \EE \gets n
\end{mathpar}}
\end{center}

Next, we define a relation $\Gamma,s,M\oopto\Gamma,\Delta,s',M'$ as in Figure \ref{FigIaSsOpSem}.

We then define the relation $\opto$ as
\[
  \inferrule*{ \Gamma,s\ts M\oopto \Gamma,\Delta,s'\ts M'}{\Gamma,s\ts \EE[M] \opto \Gamma,\Delta,s'\ts\EE[M']}
  \]
for each evaluation context $\EE$.

\begin{figure}
  \begin{mathpar}
    \Gamma,s\ts (\lambda x.M)N \oopto \Gamma,s\ts M[N/x]
    \and
    \Gamma,s\ts \Y M \oopto \Gamma,s\ts M(\Y M)
    \and
    \Gamma,s\ts \suc n \oopto \Gamma,s\ts n+1
    \\
    \Gamma,s \ts \pred (n+1) \oopto \Gamma,s\ts n
    \and
    \Gamma,s\ts \pred 0 \oopto \Gamma,s\ts 0
    \\
    \Gamma,s\ts \skipp;M \oopto \Gamma,s\ts M
    \\
    \Gamma,s\ts \If \true \Then N \Else P \oopto \Gamma,s\ts N
    \and
    \Gamma,s\ts \If \false \Then N \Else P \oopto \Gamma,s\ts P
    \and
    \Gamma,s\ts \IfO 0 \Then N \Else P \oopto \Gamma,s\ts N
    \and
    \Gamma,s\ts \IfO (n+1) \Then N \Else P \oopto \Gamma,s\ts P
    \and
    x,\Gamma,s\ts x \gets n \oopto x,\Gamma,(s\vert x\mapsto n)\ts \skipp
    \and
    x,\Gamma,s\ts \oc x \oopto x,\Gamma,s\ts s(x)
    \\
    \Gamma,s\ts \neww \lambda x.M \oopto \Gamma,x,(s\vert x\mapsto 0)\ts M
    \\
    \Gamma,s \ts (\mkvar W\,R) \gets n \oopto \Gamma,s\ts Wn
    \and
    \Gamma,s \ts \oc(\mkvar W\,R) \oopto \Gamma,s\ts R
  \end{mathpar}
  \caption{Felleisen-style small-step operational semantics for Idealized Algol.}
  \label{FigIaSsOpSem}
\end{figure}

We need to prove that this is equivalent to our original semantics.
Given a $\Gamma,\Delta$-store $s$, write $s\vert_{\Gamma}$ for the restriction of $s$ to $\Gamma$.

\begin{lemma}
  Suppose that $\Gamma,s\ts M \oopto \Gamma,\Delta,s'\ts M'$ and that $\Gamma,\Delta,s'\ts \EE[M'] \converges c,s''$.  
  Then $\Gamma,s\ts \EE[M]\converges c,s''\vert_\Gamma$.
  \label{LemSmallToBig}
\end{lemma}
\begin{proof}
  Structural induction on $\EE$.
  The base case, when $\EE$ is a hole, covers the interesting cases, so we shall leave it to last.
  The remaining cases are quite similar, so we will show the proof for one of them for illustration.

  If $\EE=\EE'\,N$, for some term $N$, then we have $\Gamma,\Delta,s'\ts \EE'[M']\,N\converges c,s''$.  
  By inspection of the rules in Figure \ref{FigIaOpSem}, the derivation of this must end with a rule of the form
  \[
    \inferrule*{\Gamma,\Delta,s'\ts \EE'[M'] \converges \lambda x.M'',t \\ \Gamma,\Delta,t \ts M''[N/x] \converges c,s''}
    {\Gamma,\Delta,s'\ts \EE'[M']\,N \converges c,s''}\,.
    \]
  Thus, $\Gamma,\Delta,s'\ts \EE'[M']\converges \lambda x.M'',t$ and $\Gamma,\Delta\ts M''[N/x]\converges c,s''$ must be provable for some $M'',t$.
  By the induction hypothesis, this means that $\Gamma,s\ts \EE'[M]\converges \lambda x.M'',t$ is provable.
  Then we have a derivation
  \[
    \inferrule*{\Gamma,\Delta,s\ts \EE'[M] \converges \lambda x.M'',t \\ \Gamma,\Delta,t \ts M''[N/x] \converges c,s''}
    {\Gamma,\Delta,s\ts \EE'[M]\,N\converges c,s''\vert_{\Gamma,\Delta}}\,.
    \]
  Then, because any variables in $\Delta$ are not mentioned in $\EE'[M]\,N$ or in $c$, we have
  \[
    \Gamma,s\ts \EE'[M]\,N\converges c,s''\vert_\Gamma\,.
    \]
  Now, let us suppose that $\EE$ is a hole, so that $\EE[M']=M'$.  

  Then there are a number of cases, depending on the particular $\oopto$ rule we are using.  
  Many of these cases are similar, so we will cover a few of them for the purposes of illustration.
  \begin{itemize}
    \item Suppose that $\Gamma,s\ts M[N/x]\converges c,s'$.  
      Then we have a derivation
      \[
        \inferrule*
        {
          \inferrule*
          {
          }
          {
            \Gamma,s \ts \lambda x.M \converges \lambda x.M,s
          }
          \\
          \Gamma,s\ts M[N/x] \converges c,s'
        }
        {
          \Gamma,s\ts (\lambda x.M)N \converges c,s'
        }\,.
        \]
    \item We have a derivation
      \[
        \inferrule*
        {
          \inferrule*
          {
          }
          {
            \Gamma,s\ts x\converges x,s
          }
          \\
          \inferrule*
          {
          }
          {
            \Gamma,s\ts n\converges n,s
          }
        }
        {
          \Gamma,s\ts x\gets n\converges \skipp,(s\vert x\mapsto n)
        }\,.
        \]
    \item We have a derivation
      \[
        \inferrule*
        {
          \inferrule*
          {
          }
          {
            \Gamma,s\ts x \converges x,s'
          }
        }
        {
          \Gamma,s\ts \oc x\converges s(x),s
        }\,.
        \]
    \item Suppose that $\Gamma,x,(s\vert x\mapsto 0) \ts M\converges c,s'$.  
      Then we have a derivation
      \[
        \inferrule*
        {
          \Gamma,x,(s\vert x\mapsto 0) \ts M \converges c,s'
        }
        {
          \Gamma,s\ts \neww\lambda x.M \converges c,s'\vert_\Gamma
        }\,,
        \]
      since $s'=(s'\vert_\Gamma\vert x\mapsto s'(x))$.\qedhere
  \end{itemize}
\end{proof}

We have proved:

\begin{proposition}
  Suppose that we have a sequence
  \[
    \Gamma_1,s^{(1)},M_1\opto\cdots\opto\Gamma_n,s^{(n)},M_n\,,
    \]
  where $M_n$ is a canonical form.  
  Then $\Gamma_1,s^{(1)} \ts M_1\converges M_n,s^{(n)}\vert_{\Gamma_1}$.  
  \label{PropSmallToBig}
\end{proposition}
\begin{proof}
  Induction on $n$.  
  The inductive step is Lemma \ref{LemSmallToBig}, while the base case ($n=1$) is given by the derivation
  \[
    \inferrule*{ }
    {\Gamma,s\ts c\converges c,s}\,.\qedhere
    \]
\end{proof}

We can also prove the converse.

\begin{proposition}
  Suppose that $\Gamma,s\ts M\converges c,s'$.  
  Then there are sequences $\Gamma=\Gamma_1,\cdots,\Gamma_n=\Gamma,\Delta$, $s=s^{(1)},\cdots,s^{(n)}$, $M=M_1,\cdots,M_n=c$ such that
  \[
    \Gamma_1,s^{(1)}\ts M_1 \opto \cdots \opto \Gamma_n,s^{(n)}\ts M_n\,,
    \]
  and $s^{(n)}\vert_{\Gamma}=s'$.
  \label{PropBigToSmall}
\end{proposition}
\begin{proof}
  Induction on the derivation of $\Gamma,s\ts M\converges c,s'$.  
  Since most of the cases are similar, we cover a selection for illustration.

  \begin{itemize}
    \item Suppose that the last step in the derivation is
      \[
        \inferrule*{ }
        {\Gamma,s\ts c\converges c,s}\,.
        \]
      Then we have the one-element sequence $\Gamma,s\ts c$.  
    \item Suppose that the last step in the derivation is
      \[
        \inferrule*
        {
          \Gamma,s\ts M\converges \lambda x.M',s' \\ \Gamma,s' \ts M'[N/x]\converges c,s''
        }
        {
          \Gamma,s \ts MN\converges c,s''
        }\,.
        \]
      Then, by the inductive hypothesis, we have small-step derivations
      \begin{mathpar}
        \Gamma,s \ts M \opto \cdots \opto \Gamma,\Delta,t',\lambda x.M'
        \and
        \Gamma,s' \ts M'[N/x]\opto \cdots \opto \Gamma,\Delta',t'',c\,,
      \end{mathpar}
      where $t'\vert_\Gamma=s'$ and $t''\vert_\Gamma=s''$.

      If we apply the evaluation context $-N$ pointwise to the first small-step derivation, then we have another valid small-step derivation.  
      Then we can join the two together to get the derivation
      \begin{center}
        \parbox{0.8\textwidth}{
        \begin{mathpar}
          \Gamma,s \ts MN \opto \cdots \opto \Gamma,\Delta,t',(\lambda x.M')N \opto \Gamma,\Delta,t',M'[N/x] \opto \cdots \opto \Gamma,\Delta\cup\Delta',t''\setminus t',c\,,
        \end{mathpar}}
      \end{center}
      where $t''\setminus t'$ is the $\Gamma,\Delta\cup\Delta'$-store that agrees with $t''$ on $\Gamma,\Delta'$ and with $t'$ on $\Delta\setminus\Delta'$.
      Then $(t''\setminus t')\vert_{\Gamma}=s''$.
    \item Suppose that the last step in the derivation is
      \[
        \inferrule*
        {
          \Gamma,s\ts E\converges n,s' \\ \Gamma,s'\ts V\converges x,s''
        }
        {
          \Gamma,s \ts V \gets E \converges\skipp,(s''\vert x\mapsto n)
        }\,.
        \]
      By the inductive hypothesis, we have small-step derivations
      \begin{mathpar}
        \Gamma,s\ts E \opto \cdots \opto \Gamma,\Delta,t',n
        \and
        \Gamma,s'\ts V \opto \cdots \opto \Gamma,\Delta',t'',x\,,
      \end{mathpar}
      where $t'\vert_{\Gamma}=s'$ and $t''\vert_{\Gamma}=s''$.

      We may apply the evaluation context $V\gets-$ pointwise to the first derivation and the evaluation context $-\gets n$ pointwise to the second, and then string the two together to get
      \begin{center}
        \parbox{0.8\textwidth}{
        \begin{mathpar}
          \Gamma,s\ts V\gets E \opto \cdots \opto \Gamma,\Delta,t',V\gets n \opto \cdots \opto \Gamma,\Delta\cup\Delta',t''\setminus t',x\gets n \opto \Gamma,\Delta\cup\Delta',(t''\setminus t'\vert x\mapsto n)\,,
        \end{mathpar}}
      \end{center}
      where we have $(t''\setminus t'\vert x\mapsto n)\vert_\Gamma=(s''\vert x\mapsto n)$.
    \item Suppose that the last step in the derivation is
      \[
        \inferrule*
        {
          \Gamma,s\ts V \converges x,s'
        }
        {
          \Gamma,s\ts \oc V \converges s'(x),s'
        }\,.
        \]
      Then, by the induction hypothesis, we have a small-step derivation
      \[
        \Gamma,s\ts V \opto \cdots \opto \Gamma,\Delta,t'\ts x\,,
        \]
      where $t'\vert_\Gamma=s'$.  
      Then we may compose this derivation pointwise with the evaluation context $\oc-$, and add an extra term on the end, to arrive at the derivation
      \[
        \Gamma,s\ts \oc V \opto \cdots \opto \Gamma,\Delta,t'\ts \oc x \opto \Gamma,\Delta,t'\ts t'(x)\,,
        \]
      where $t'(x)=s'(x)$.
    \item Lastly, suppose that the last step in the derivation is
      \[
        \inferrule*
        {
          \Gamma,x,(s\vert x\mapsto 0) \ts M \converges c,(s'\vert x\mapsto n)
        }
        {
          \Gamma,s \ts \neww \lambda x.M \converges c,s'
        }\,.
        \]
      By the induction hypothesis, we have a small-step derivation
      \[
        \Gamma,x,(s\vert x\mapsto 0) \ts M \opto \cdots \opto \Gamma,\Delta,x,(t'\vert x\mapsto n),c\,,
        \]
      where $t'\vert_\Gamma=s'$.  
      Then we may add a term at the beginning to give us
      \begin{center}
        \parbox{0.6\textwidth}{
        \begin{mathpar}
          \Gamma,s \ts \neww \lambda x.M \opto \Gamma,x,(s\vert x\mapsto 0) \ts M \opto \cdots \opto \Gamma,\Delta,x,(t'\vert x\mapsto n),c\,,
        \end{mathpar}}
      \end{center}
      where $(t'\vert x\mapsto n) \vert_\Gamma=s'$.\qedhere
  \end{itemize}
\end{proof}

\section{Soundness}

To prove soundness of our model, we shall use the small-step formulation.  
Our reason for this is that the most difficult part of the denotational semantics we are using is the part to do with state.  
In the big-step formulation, nearly every rule involves the state changing in some way, whereas in the small-step formulation, only the rules that specifically pertain to the stateful components of the language do.

Given a $\Var$-context $\Gamma$, we will write $S_\Gamma$ for $\bN \times \cdots \times \bN$ and $\V_\Gamma$ for $\Varr \times \cdots \times \Varr$, where in each case the terms of the tensor product are indexed by the variables in $\Gamma$.  
Then, given some term $\Gamma\ts M$ in context, its denotation will be a morphism $\deno{\Gamma\ts M\from T} \from \oc \V_\Gamma \implies \deno{T}$.

Given $\Gamma$, we have a morphism $\cell^\Gamma\from \oc S_\Gamma \implies \oc V_\Gamma$, given by
\[
  \oc S_\Gamma \cong \oc \bN \tensor \cdots \tensor \oc \bN \xrightarrow{\cell \tensor \cdots \tensor \cell} \oc \Varr \tensor \cdots \tensor \oc \Varr \cong \oc V_\Gamma\,.
  \]
\begin{lemma}
  For $j=1,\cdots,|\Gamma|$, the following diagram commutes.
  \[
    \begin{tikzcd}
      \oc S_\Gamma \arrow[r, "\cell^\Gamma"] \arrow[d, "\oc\pr_j"']
        & \oc V_\Gamma \arrow[d, "\oc \pr_j"] \\
      \oc \bN \arrow[r, "\cell"]
        & \oc \Varr
    \end{tikzcd}
    \]
\end{lemma}
\begin{proof}
  We have a commutative diagram
  \[
    \begin{tikzcd}[column sep=40pt, row sep=40pt]
      \oc S_\Gamma \arrow[r, Isom] \arrow[dr, "\oc \pr_j"']
        & \oc \bN \tensor \cdots \tensor \oc \bN \arrow[r, "\cell \tensor \cdots \tensor \cell"] \arrow[d, "() \tensor \cdots \tensor \oc\bN \tensor \cdots \tensor ()"' {description, xshift=10pt}]
          & \oc \Varr \tensor \cdots \tensor \oc \Varr \arrow[r, Isom] \arrow[d, "() \tensor \cdots \tensor \oc\Varr \tensor \cdots \tensor ()" {description, xshift=-10pt}]
            & \oc V_\Gamma \arrow[dl, "\oc \pr_j"] \\
      %
        & \oc\bN \arrow[r, "\cell"]
          & \oc\Varr
            &
    \end{tikzcd}\,,
    \]
  where the outer triangles commute because the vertical arrows are the projections in the tensor product of comonoids.
\end{proof}

Given a $\Gamma$-store $s$, we will write $\deno{s}$ for the corresponding morphism $I \to S_\Gamma$.

We start our proof of soundness with a result about evaluation contexts.
This result captures the fact that the term filling the hole of an evaluation context is the first thing to be computed, using the sequoid operator to capture this notion of precedence.
\begin{lemma}
  Let $\Gamma\ts M\from T$ be an Idealized Algol term-in-context, and let $\EE$ be an evaluation context with a hole of type $T$, where $\EE[M]\from U$.  
  Then there is a game $A$ and strategies $\sigma\from \oc\V_\Gamma \implies A$, $\tau\from\deno{T}\sequoid A \implies \deno{S}$, where $\tau$ is a strict strategy, such that the denotation $\deno{\Gamma\ts\EE[M]}$ factors as
  \[
    \oc\V_\Gamma \xrightarrow{\mu_{\V_\Gamma}} \oc \V_\Gamma \tensor \oc \V_\Gamma \xrightarrow{\deno{\Gamma\ts M}\tensor\sigma} \deno{T}\tensor A \xrightarrow{\wk_{\deno{T},A}} \deno{T} \sequoid A \xrightarrow{\tau} \deno{U}\,.
    \]
  \label{LemEvContexLemma}
\end{lemma}
\begin{proof}
  Structural induction on $\EE$.
  \begin{itemize}
    \item If $\EE=-$ is a hole, then we may take $A=I$, $\sigma=()$ and $\tau=\run_{\deno T}$.  

    \item If $\EE=\EE' N$ for some term $N$, where $\EE'$ has type $S\to T$, $N$ has type $S$, and $M\from S'$ fits into the hole, then the denotation $\deno{\Gamma \ts \EE[M]} = \deno{\Gamma \ts \EE'[M]\,N}$ is given by the composite
      \[
        \oc\V_\Gamma \xrightarrow{\mu_{\V_\Gamma}} \oc\V_\Gamma \tensor \oc\V_\Gamma \xrightarrow{\deno{\Gamma \ts \EE'[M]}\tensor \deno{\Gamma \ts N}} (\oc\deno{S} \implies \deno{T}) \tensor \oc\deno{S} \xrightarrow{\ev_{\oc\deno{S},\deno{T}}} \deno{T}.
        \]
      By the inductive hypothesis, $\deno{\Gamma\ts\EE' M}$ factors as
      \small
      \[
        \oc\V_\Gamma \xrightarrow{\mu_{\V_\Gamma}} \oc \V_\Gamma \tensor \oc \V_\Gamma \xrightarrow{\deno{\Gamma\ts M}\tensor\sigma'} \deno{S'}\tensor A' \xrightarrow{\wk_{\deno{S'},A'}} \deno{S'} \sequoid A' \xrightarrow{\tau'} (\oc\deno{S}\implies\deno{T}),
        \]
      \normalsize
      for appropriate $A',\sigma',\tau'$.  
      Then $\deno{\Gamma\ts\EE[M]} = \deno{\Gamma\ts\EE'[M]\,N}$ is given by the thick dashed arrows in the diagram in Figure \ref{FigEvContextApp}.  
      But this composite is equal to that given by the thin solid arrows in the diagram, which is of the required form, with
      \begin{mathpar}
        A = A' \tensor \oc\deno{S}
        \and
        \sigma = \mu_{\V_\Gamma};(\sigma'\tensor\deno{\Gamma\ts N}^\dag)
        \and
        \tau = \passoc_{\deno{S'},A',\oc\deno{S}}\inv;(\tau'\sequoid\oc\deno{S});\ev_{s\;\oc\deno{S},\deno{T}}\,.
      \end{mathpar}
      \begin{figure}
        \small
        \[
          \begin{tikzcd}[column sep=79pt]
            \oc\V_\Gamma \arrow[d, "\mu_{\V_\Gamma}"' yshift=3pt, thick, dashed] \arrow[r, "\mu_{\V_\Gamma}"]
              & \oc\V_\Gamma \tensor \oc\V_\Gamma \arrow[d, "\oc\V_\Gamma\tensor\mu_{\V_\Gamma}"] \\
            \oc \V_\Gamma \tensor \oc \V_\Gamma \arrow[d, "\mu_{\V_\Gamma}\tensor \oc\V_\Gamma"' yshift=3pt, thick, dashed]
              & \oc \V_\Gamma \tensor (\oc \V_\Gamma \tensor \oc \V_\Gamma) \arrow[d, "\deno{\Gamma \ts M}\tensor (\sigma' \tensor \deno{\Gamma\ts N}^\dag)"] \\
            (\oc \V_\Gamma \tensor \oc \V_\Gamma) \tensor \oc \V_\Gamma \arrow[d, "(\deno{\Gamma\ts M} \tensor \sigma')\tensor\deno{\Gamma\ts N}^\dag"' yshift=3pt, thick, dashed] \arrow[ur, "\assoc_{\V_\Gamma,\V_\Gamma,\V_\Gamma}" description, dotted]
              & \deno{S'} \tensor (A' \tensor \oc\deno{S}) \arrow[d, "{\wk_{\deno{S'},A' \tensor \oc \deno{S}}}"] \\
            (\deno{S'} \tensor A') \tensor \oc\deno{S} \arrow[d, "{\wk_{\deno{S'},A'}\tensor \oc\deno{S}}"' yshift=3pt, thick, dashed] \arrow[ur, "\assoc_{\deno{S'},A',\oc\deno{S}}" description, dotted]
              & \deno{S'} \sequoid (A' \tensor \oc\deno{S}) \arrow[d, "{\passoc_{\deno{S'},A',\oc\deno{S}}\inv}"] \\
            (\deno{S'} \sequoid A') \tensor \oc\deno{S} \arrow[d, "\tau' \tensor\oc\deno S"' yshift=3pt, thick, dashed] \arrow[r, "{\wk_{\deno{S'}\sequoid A',\oc\deno{S}}}" description, dotted]
              & (\deno{S'} \sequoid A') \sequoid \oc\deno{S} \arrow[d, "\tau' \sequoid \oc\deno{S}"] \\
            (\oc \deno{S} \implies \deno{T}) \tensor \oc\deno{S} \arrow[d, "{\ev_{\oc\deno{S},\deno{T}}}"' yshift=3pt, thick, dashed] \arrow[r, "{\wk_{\oc\deno S \implies \deno T,\oc\deno S}}" description, dotted]
              & (\oc\deno{S} \implies \deno{T}) \sequoid \oc\deno{S} \arrow[dl, "{\ev_{s\;\oc\deno{S},\deno{T}}}"] \\
            \deno{T}
              &
          \end{tikzcd}
          \]
        \normalsize
        \caption[The property in Lemma \ref{LemEvContexLemma} is preserved by function application.]{The property in Lemma \ref{LemEvContexLemma} is preserved by function application.  
        Here, $\ev_{s\;\oc\deno{S},\deno{T}} = \Lambda_s\inv(\id_{\oc\deno{S}\implies \deno{T}})$.}
        \label{FigEvContextApp}
      \end{figure}
    \item If $\EE=\suc \EE'$ or $\pred \EE'$, where $\EE'$ is a context of type $\nat$, and $M\from T$ is a term that fits into the hole, then the denotation $\deno{\Gamma\ts\EE[M]}$ is given by the composite
      \[
        \oc \V_\Gamma \xrightarrow{\deno{\Gamma \ts \EE'[M]}} \bN \xrightarrow{\theta} \bN\,,
        \]
      where $\theta$ is either $\pred$ or $\suc$.
      By the inductive hypothesis, $\deno{\Gamma\ts\EE'[M]}$ factors as
      \[
        \oc\V_\Gamma \xrightarrow{\mu_{\V_\Gamma}} \oc \V_\Gamma \tensor \oc \V_\Gamma \xrightarrow{\deno{\Gamma\ts M}\tensor\sigma'} \deno{T}\tensor A' \xrightarrow{\wk_{\deno{T},A'}} \deno{T} \sequoid A' \xrightarrow{\tau'} \bN\,,
        \]
      for appropriate $A',\sigma',\tau'$.  
      Then we can compose on the right by $\theta$, and we are already in the required form, for
      \begin{mathpar}
        A = A' \and \sigma = \sigma' \and \tau = \tau';\theta\,.
      \end{mathpar}
    \item Similarly, if $\EE = \oc\EE'$, where $\EE'$ is a context of type $\Var$, and $M\from T$ is a term that fits into the hole, then the denotation $\deno{\Gamma\ts\EE[M]}$ is given by the composite
      \[
        \oc \V_\Gamma \xrightarrow{\deno{\Gamma \ts \EE'[M]}} \Varr \xrightarrow{\dereff} \bN\,.
        \]
      By the inductive hypothesis, $\deno{\Gamma\ts\EE'[M]}$ factors as
      \[
        \oc\V_\Gamma \xrightarrow{\mu_{\V_\Gamma}} \oc \V_\Gamma \tensor \oc \V_\Gamma \xrightarrow{\deno{\Gamma\ts M}\tensor\sigma'} \deno{T}\tensor A' \xrightarrow{\wk_{\deno{T},A'}} \deno{T} \sequoid A' \xrightarrow{\tau'} \Varr\,,
        \]
      for appropriate $A',\sigma',\tau'$.  
      Then we can compose on the right by $\dereff$, and we are already in the required form, for
      \begin{mathpar}
        A = A' \and \sigma = \sigma' \and \tau = \tau';\dereff\,.
      \end{mathpar}
    \item If $\EE=\EE';N$ for some term $N$ of type $X\in\{\com,\bool,\nat\}$, where $\EE'$ is an evaluation context of type $\com$, and if $M\from T$ fits into the hole in $\EE'$, then the denotation $\deno{\Gamma\ts\EE[M]}$ is given by the composite
      \[
        \oc\V_\Gamma \xrightarrow{\mu_{\V_\Gamma}} \oc\V_\Gamma \tensor \oc\V_\Gamma \xrightarrow{\deno{\Gamma\ts\EE'[M]}\tensor\deno{\Gamma\tensor N}} \bC \times X \xrightarrow{\Lambda\inv(\seq_X)} X\,.
        \]
      If $\EE = N\gets\EE'$, for some term $N$ of type $\Var$, where $\EE'$ is an evaluation context of type $\com$, and if $M\from T$ fits into the hole in $\EE'$, then the denotation $\deno{\Gamma\ts\EE[M]}$ is given by the composite
      \[
        \oc\V_\Gamma \xrightarrow{\mu_{\V_\Gamma}} \oc\V_\Gamma \tensor \oc\V_\Gamma \xrightarrow{\deno{\Gamma\ts\EE'[M]}\tensor\deno{\Gamma\tensor N}} \bN \times \Var \xrightarrow{\Lambda\inv(\assign)} \bC\,.
        \]
      Write $Y=\bC$, $X'=X$, $Z=X$ and $\upsilon=\seq_X$ in the sequencing case, and $Y=\bN$, $X'=\Var$, $Z=\bC$ and $\upsilon=\assign$ in the variable assignment case.
      Then, by the inductive hypothesis, $\deno{\Gamma\ts\EE'[M]}$ factors as
      \[
        \oc\V_\Gamma \xrightarrow{\mu_{\V_\Gamma}} \oc \V_\Gamma \tensor \oc \V_\Gamma \xrightarrow{\deno{\Gamma\ts M}\tensor\sigma'} \deno{T}\tensor A' \xrightarrow{\wk_{\deno{T},A'}} \deno{T} \sequoid A' \xrightarrow{\tau'} Y\,,
        \]
      for suitable $A',\sigma',\tau'$.
      Then $\deno{\Gamma\ts\EE[M]}$ is given by the thick dashed arrows in the diagram in Figure \ref{FigEvContextSeqAss}.  
      But this composite is equal to that given by the thin solid arrows in the diagram, which is of the required form, with
      \begin{mathpar}
        A = A' \tensor X'
        \and
        \sigma = \mu_{\V_\Gamma};(\sigma'\tensor\deno{\Gamma\ts N})
        \and
        \tau = \passoc_{\deno{T},A',X'}\inv;(\tau'\sequoid X');\Lambda_s\inv(\upsilon)\,.
      \end{mathpar}
      \begin{figure}
        \small
        \[
          \begin{tikzcd}[column sep=79pt]
            \oc\V_\Gamma \arrow[d, "\mu_{\V_\Gamma}"' yshift=3pt, thick, dashed] \arrow[r, "\mu_{\V_\Gamma}"]
              & \oc\V_\Gamma \tensor \oc\V_\Gamma \arrow[d, "\oc\V_\Gamma\tensor\mu_{\V_\Gamma}"] \\
            \oc \V_\Gamma \tensor \oc \V_\Gamma \arrow[d, "\mu_{\V_\Gamma}\tensor \oc\V_\Gamma"' yshift=3pt, thick, dashed]
              & \oc \V_\Gamma \tensor (\oc \V_\Gamma \tensor \oc \V_\Gamma) \arrow[d, "\deno{\Gamma \ts M}\tensor (\sigma' \tensor \deno{\Gamma\ts N})"] \\
            (\oc \V_\Gamma \tensor \oc \V_\Gamma) \tensor \oc \V_\Gamma \arrow[d, "(\deno{\Gamma\ts M} \tensor \sigma')\tensor\deno{\Gamma\ts N}"' yshift=3pt, thick, dashed] \arrow[ur, "\assoc_{\V_\Gamma,\V_\Gamma,\V_\Gamma}" description, dotted]
              & \deno{T} \tensor (A' \tensor X') \arrow[d, "{\wk_{\deno{T},A' \tensor X'}}"] \\
            (\deno{T} \tensor A') \tensor X' \arrow[d, "{\wk_{\deno{T},A'}\tensor X'}"' yshift=3pt, thick, dashed] \arrow[ur, "\assoc_{\deno{T},A',X'}" description, dotted]
              & \deno{T} \sequoid (A' \tensor X') \arrow[d, "{\passoc_{\deno{T},A',X'}\inv}"] \\
            (\deno{T} \sequoid A') \tensor X' \arrow[d, "\tau' \tensor X'"' yshift=3pt, thick, dashed] \arrow[r, "{\wk_{\deno{T}\sequoid A',X'}}" description, dotted]
              & (\deno{T} \sequoid A') \sequoid X' \arrow[d, "\tau' \sequoid X'"] \\
            Y \tensor X' \arrow[d, "\Lambda\inv(\upsilon)"' yshift=3pt, thick, dashed] \arrow[r, "{\wk_{Y,X'}}" description, dotted]
              & Y \sequoid X' \arrow[dl, "\Lambda_s\inv(\upsilon)"] \\
            \deno{T}
              &
          \end{tikzcd}
          \]
        \normalsize
        \caption[The property in Lemma \ref{LemEvContexLemma} is preserved by sequencing and variable assignment.]{The property in Lemma \ref{LemEvContexLemma} is preserved by sequencing and variable assignment.
        We use the fact that $\upsilon\in\{\seq_X,\assign\}$ is a strict strategy, so that $\Lambda_s\inv(\upsilon)$ is well-defined.}
        \label{FigEvContextSeqAss}
      \end{figure}
    \item If $\EE = \If \EE' \Then N \Else P$ for a context $\EE'$ of type $\bool$, where $N$ and $P$ are terms of type $X\in\{\bool,\com,\nat\}$, then write $Y=\bool$ and $\eta = \If_X$.  
      If $\EE = \IfO \EE' \Then N \Else P$ for a context $\EE'$ of type $\nat$, where $N$ and $P$ are terms of type $X\in\{\bool,\com,\nat\}$, then write $Y=\nat$ and $\eta=\IfO_X$.  
      In either case, if $M\from T$ is a term that fits into the hole, then the denotation $\deno{\Gamma\ts\EE[M]}$ is given by
      \begin{mathpar}
        \oc\V_\Gamma \xrightarrow{\mu_{V_\Gamma}} \oc\V_\Gamma\tensor\oc\V_\Gamma \xrightarrow{\mu_\Gamma \tensor \V_\Gamma} (\oc \V_\Gamma \tensor \oc \V_\Gamma) \tensor \oc\V_\Gamma \\\xrightarrow{(\deno{\Gamma\ts\EE'[M]} \tensor \deno{\Gamma\ts N}) \tensor \deno{\Gamma \ts P}} (Y \tensor X) \tensor X \xrightarrow{\Lambda\inv(\Lambda\inv(\eta))} X\,.
      \end{mathpar}
      By the inductive hypothesis, $\deno{\Gamma\ts\EE'[M]}$ factors as
      \[
        \oc\V_\Gamma \xrightarrow{\mu_{\V_\Gamma}} \oc \V_\Gamma \tensor \oc \V_\Gamma \xrightarrow{\deno{\Gamma\ts M}\tensor\sigma'} \deno{T}\tensor A' \xrightarrow{\wk_{\deno{T},A'}} \deno{T} \sequoid A' \xrightarrow{\tau'} Y\,,
        \]
      for appropriate $A',\sigma',\tau'$.  
      Then $\deno{\Gamma\ts\EE[M]} = \deno{\Gamma\ts\EE'[M]\,N}$ is given by the thick dashed arrows in the diagram in Figure \ref{FigEvContextCond}.  
      But this composite is equal to that given by the thin solid arrows in the diagram, which is of the required form, with
      \begin{mathpar}
        A = (A' \tensor X)\tensor X
        \and
        \sigma = \mu_{\V_\Gamma};(\mu_{\V_\Gamma}\tensor\oc\V_\Gamma);((\sigma'\tensor\deno{\Gamma\ts N})\tensor \deno{\Gamma\ts P})
        \and
        \tau = \passoc_{\deno T,A'\tensor X,X}\inv;(\passoc_{\deno T,A',X}\inv\sequoid X);((\tau'\sequoid X)\sequoid X);\Lambda_s\inv(\Lambda_s\inv(\eta))\,.
      \end{mathpar}
      \begin{SidewaysFigure}
        \small
        \[
          \begin{tikzcd}[ampersand replacement=\&, column sep=78pt, row sep=30pt]
            \oc \V_\Gamma \arrow[rr, "\mu_{\V_\Gamma}"] \arrow[d, "\mu_{\V_\Gamma}"', thick, dashed]
              \&
                \& \oc \V_\Gamma \tensor \oc \V_\Gamma \arrow[d, "\oc\V_\Gamma \tensor\mu_{\V_\Gamma}"] \\
            \oc \V_\Gamma \tensor \oc\V_\Gamma \arrow[r, "\mu_{\V_\Gamma}\tensor \oc\V_\Gamma", dotted] \arrow[d, "\mu_{\V_\Gamma}\tensor\oc\V_\Gamma"', thick, dashed]
              \& (\oc \V_\Gamma \tensor \oc\V_\Gamma) \tensor \oc\V_\Gamma \arrow[r, "{\assoc_{\oc\V_\Gamma,\oc\V_\Gamma,\oc\V_\Gamma}}", dotted] \arrow[d, "(\oc\V_\Gamma \tensor \mu_{\V_\Gamma}) \tensor \oc\V_\Gamma" description, dotted]
                \& \oc \V_\Gamma \tensor (\oc \V_\Gamma \tensor \oc\V_\Gamma) \arrow[d, "\oc\V_\Gamma \tensor (\mu_{\V_\Gamma}\tensor \oc\V_\Gamma)"] \\
            (\oc \V_\Gamma \tensor \oc\V_\Gamma)\tensor \oc\V_\Gamma \arrow[d, "(\mu_{\V_\Gamma}\tensor\oc\V_\Gamma)\tensor \oc\V_\Gamma"', thick, dashed]
              \& |[alias=Z]| (\oc \V_\Gamma \tensor(\oc\V_\Gamma\tensor \oc\V_\Gamma)) \tensor \oc\V_\Gamma \arrow[r, "{\assoc_{\oc\V_\Gamma,\oc\V_\Gamma\tensor\oc\V_\Gamma,\oc\V_\Gamma}}", dotted] \arrow[d, "(\deno{\Gamma\ts M}\tensor (\sigma'\tensor\deno{\Gamma\ts N})) \tensor \deno{\Gamma\ts P}" description, dotted]
                \& \oc\V_\Gamma \tensor ((\oc\V_\Gamma \tensor \oc\V_\Gamma) \tensor \oc\V_\Gamma) \arrow[d, "\deno{\Gamma\ts M} \tensor ((\sigma'\tensor\deno{\Gamma\ts N})\tensor \deno{\Gamma\ts P})" description] \\
            ((\oc\V_\Gamma \tensor \oc\V_\Gamma) \tensor \oc\V_\Gamma) \tensor \oc\V_\Gamma \arrow[d, "((\deno{\Gamma\ts M} \tensor \sigma') \tensor \deno{\Gamma\ts N}) \tensor \deno{\Gamma\ts P}" {description, xshift=-20pt}, thick, dashed] \arrow[ur, "{\assoc_{\oc\V_\Gamma,\oc\V_\Gamma,\oc\V_\Gamma}\tensor\oc\V_\Gamma}" description, to=Z.west, dotted]
              \& |[alias=Y]| (\deno T \tensor (A' \tensor X)) \tensor X \arrow[r, "{\assoc_{\deno T,A'\tensor X,X}}" description, dotted] \arrow[d, "{\wk_{T,A'\tensor X}\tensor X}" description, dotted]
                \& \deno T \tensor ((A' \tensor X) \tensor X) \arrow[d, "{\wk_{\deno T,(A'\tensor X)\tensor X}}"] \\
            ((\deno T\tensor A') \tensor X)\tensor X \arrow[ur, "{\assoc_{\deno T,A',X}\tensor X}" description, to=Y.west, dotted] \arrow[d, "{(\wk_{\deno T,A'}\tensor X)\tensor X}"', thick, dashed]
              \& (\deno T \sequoid (A' \tensor X))\tensor X \arrow[dr, "{\wk_{\deno T\sequoid (A'\tensor X),X}}" description, dotted] \arrow[d, "{\passoc_{\deno T,A',X}\tensor X}" description, dotted]
                \& \deno T \sequoid ((A' \tensor X)\tensor X) \arrow[d, "{\passoc_{\deno T,A'\tensor X,X}\inv}"] \\
            ((\deno T\sequoid A') \tensor X) \tensor X \arrow[r, "{\wk_{\deno T \sequoid A',X}\tensor X}", dotted] \arrow[d, "(\tau'\tensor X)\tensor X)"', thick, dashed]
              \& ((\deno T \sequoid A') \sequoid X) \tensor X \arrow[dr, "{\wk_{(\deno T\sequoid A') \sequoid X,X}}" description, dotted] \arrow[d, "(\tau'\sequoid X)\tensor X" description, dotted]
                \& (\deno T \sequoid (A' \tensor X)) \sequoid X \arrow[d, "{\passoc_{\deno T,A',X}\inv\sequoid X}"] \\
            (Y \tensor X) \tensor X \arrow[r, "{\wk_{Y,X}\tensor X}", dotted] \arrow[d, "\Lambda\inv(\Lambda\inv(\eta))"', thick, dashed]
              \& (Y \sequoid X) \tensor X \arrow[dr, "{\wk_{Y\sequoid X,X}}" description, dotted] \arrow[dl, "\Lambda\inv(\Lambda_s\inv(\eta))" description, dotted]
                \& ((\deno T \sequoid A') \sequoid X) \sequoid X \arrow[d, "(\tau'\sequoid X) \sequoid X"] \\
            X
              \&
                \& (Y \sequoid X)\sequoid X \arrow[ll, "\Lambda_s\inv(\Lambda_s\inv(\eta))"']
          \end{tikzcd}
          \]
          \normalsize
        \caption[The property in Lemma \ref{LemEvContexLemma} is preserved by conditionals.]{The property in Lemma \ref{LemEvContexLemma} is preserved by conditionals.
        We use the fact that $\eta\in\{\If_X,\IfO_X\}$ is a strict strategy, and that the $\Lambda_s\inv$ is a function from strict strategies to strict strategies, so that $\Lambda_s\inv(\Lambda_s\inv(\eta))$ is well-defined.}
        \label{FigEvContextCond}
      \end{SidewaysFigure}
    \item Lastly, suppose that $\EE = \EE' \gets n$ for some numeral $n$, where $\EE'$ is a context of type $\Var$, and suppose that $M\from T$ fits into the hole.  
      The denotation $\deno{\Gamma\ts \EE[M]}$ is given by
      \[
        \oc\V_\Gamma \xrightarrow{\deno{\Gamma\ts \EE'[M]}} \Varr \xrightarrow{\lunit_{\Varr}} I \tensor \Varr \xrightarrow{n\tensor\Varr} \bN \tensor \Varr \xrightarrow{\assign} \bC\,.
        \]
      By the induction hypothesis, $\deno{\Gamma\ts\EE'[M]}$ takes the form
      \[
        \oc\V_\Gamma \xrightarrow{\mu_{\V_\Gamma}}
        \oc\V_\Gamma\tensor\oc\V_\Gamma \xrightarrow{\deno{\Gamma\ts M}\tensor\sigma'}
        \deno{T}\tensor A' \xrightarrow{\wk_{\deno{T},A'}}
        \deno T\sequoid A' \xrightarrow{\tau'}
        \Varr\,,
        \]
      for suitable $A',\sigma',\tau'$.  
      Then, if we compose on the right by the morphism $\lunit_{\Varr};(n\tensor\Varr);\assign$, in order to give us the denotation $\deno{\Gamma\ts \EE'[M]\gets n} = \deno{\Gamma\ts\EE[M]}$, then we are already in the required form, with
      \begin{mathpar}
        A = A' \and \sigma = \sigma' \and \tau = \tau';\lunit_{\Varr};(n\tensor\Varr);\assign\,.\hspace{1em plus 1fill}\qedhere
      \end{mathpar}
  \end{itemize}
\end{proof}

Our next lemma will help us deal with the base $\oopto$ rules.  
We will then use Lemma \ref{LemEvContexLemma} to extend this to the $\opto$ relation.

\begin{definition}
  Let $\Gamma,s\ts M\from T$ be a term with store.  
  Then the \emph{sequoidal denotation} $\seqdeno{\Gamma,s\ts M}$ is the composite
  \[
    I \xrightarrow{\deno s} \oc S_\Gamma \xrightarrow{\cell^\Gamma} \oc\V_\Gamma \xrightarrow{\mu_{\V_\Gamma}} \oc\V_\Gamma \tensor \oc\V_\Gamma \xrightarrow{\deno{\Gamma\ts M}\tensor\oc\V_\Gamma} \deno T \tensor\oc\V_\Gamma \xrightarrow{\wk_{\deno T,\oc\V_\Gamma}} \deno T \sequoid \oc\V_\Gamma\,.
    \]
\end{definition}

\begin{lemma}
  The relation $\oopto$ betwen triples $\Gamma,s\ts M$ preserves the sequoidal denotation $\seqdeno{\Gamma,s\ts M}$.

  In other words, if $\Gamma,s\ts M \oopto \Gamma,\Delta,s'\ts N$, where $M,N$ have type $T$, then the following diagram commutes.
  \[
    \begin{tikzcd}[column sep=28pt]
      I \arrow[r, "\oc \deno s"] \arrow[d, "\oc \deno s'"']
        & \oc S_\Gamma \arrow[r, "\cell^\Gamma"]
          & \oc \V_\Gamma \arrow[r, "\mu_{\V_\Gamma}"]
            &[22pt] \oc \V_\Gamma \tensor \oc \V_\Gamma \arrow[d, "\deno{\Gamma\ts M}\tensor \oc V_\Gamma"] \\
      {\oc S_{\Gamma,\Delta}} \arrow[d, "{\cell^{\Gamma,\Delta}}"']
        &
          &
            & \deno T \tensor \oc V_\Gamma \arrow[d, "{\wk_{\deno T,\oc \V_{\Gamma}}}"] \\
      {\oc \V_{\Gamma,\Delta}} \arrow[d, "{\mu_{\V_{\Gamma,\Delta}}}"']
        &
          &
            &  \deno{T}\sequoid \oc\V_\Gamma \\
      {\oc \V_{\Gamma,\Delta} \tensor \oc \V_{\Gamma,\Delta}} \arrow[rr, "{\deno{\Gamma,\Delta\ts N} \tensor \oc \V_{\Gamma,\Delta}}"]
        &
          & {\deno T \tensor \oc \V_{\Gamma,\Delta}} \arrow[r, "{\wk_{\deno T,\oc \V_{\Gamma,\Delta}}}"]
            & {{\deno T \sequoid \oc \V_{\Gamma,\Delta}}\,.} \arrow[u, "\deno{T}\sequoid\oc\pr_\Gamma"']
    \end{tikzcd}
    \]
  I.e., if $\Gamma,s\ts M \oopto \Gamma,\Delta,s'\ts N$, then $\seqdeno{\Gamma,s\ts M} = \seqdeno{\Gamma,\Delta,s'\ts N};\oc\pr_\Gamma$.
  \label{LemSoundnessOopto}
\end{lemma}
\begin{proof}
  We prove this on a case-by-case basis.
  \begin{itemize}
    \item For most of the rules, $\Delta=\_$ and $s=s'$; i.e., the rule is of the form
      \[
        \Gamma,s\ts M \oopto \Gamma,s \ts N
        \]
      for some $M,N$.
      In such a case, it suffices to show that $\deno{\Gamma\ts M}=\deno{\Gamma\ts N}$.
      Indeed, we have
      \begin{itemize}
        \item $\deno{\Gamma\ts(\lambda x.M)N} = \deno{\Gamma\ts M[N/x]}$ (by a usual substitution-lemma argument);
        \item $\deno{\Gamma\ts\Y M} = \deno{\Gamma\ts (\lambda F.\lambda f.f(F f)) \Y M} = \deno{\Gamma\ts M(\Y M)}$;
        \item $\deno{\Gamma\ts \suc n} = \deno{\Gamma \ts n+1}$;
        \item $\deno{\Gamma\ts \pred (n+1)} = \deno{\Gamma\ts n}$;
        \item $\deno{\Gamma\ts \pred 0} = \deno{\Gamma\ts 0}$;
        \item $\deno{\Gamma\ts\skipp;M} = \deno{\Gamma\ts M}$;
        \item $\deno{\Gamma\ts\If \true \Then N \Else P} = \deno{\Gamma\ts (\lambda x.\lambda y.x)NP} = \deno{\Gamma\ts N}$;
        \item $\deno{\Gamma\ts\If \false \Then N \Else P} = \deno{\Gamma\ts (\lambda x.\lambda y.y)NP} = \deno{\Gamma\ts P}$;
        \item $\deno{\Gamma\ts\IfO 0 \Then N \Else P} = \deno{\Gamma\ts (\lambda x.\lambda y.x)NP} = \deno{\Gamma\ts N}$;
        \item $\deno{\Gamma\ts\IfO (n+1) \Then N \Else P} = \deno{\Gamma\ts (\lambda x.\lambda y.y)NP} = \deno{\Gamma\ts P}$;
        \item \parbox[t][][t]{0.8\textwidth}{$\deno{\Gamma\ts(\mkvar W\,R)\gets n} = \langle \deno{\Gamma\ts Wn}_{n\in\bN},\deno{\Gamma\ts R}\rangle;\pr_n = \deno{\Gamma\ts Wn}$; and}
        \item $\deno{\Gamma\ts\oc(\mkvar W\,R)} = \langle \deno{\Gamma\ts Wn}_{n\in\bN},\deno{\Gamma\ts R}\rangle;\pr_{\bN} = \deno{\Gamma\ts R}$.
      \end{itemize}
    \item Now consider the rule
      \[
        \Gamma,s\ts\neww \lambda x.M\oopto \Gamma,x,(s\vert x\mapsto 0)\ts M\,,
        \]
      where $\Gamma\ts M\from T$.
      The first observation to make is that the denotation $\deno{\Gamma\ts \neww \lambda x.M}$ may be written as
      \[
        \oc\V_\Gamma \xrightarrow{\runit_{\oc\V_\Gamma}} \oc\V_\Gamma \tensor I \xrightarrow{\oc\V_\Gamma\tensor (0;\cell)} \oc\V_\Gamma \tensor \oc\Varr \to \oc\V_{\Gamma,x} \xrightarrow{\Lambda\inv(\deno{\Gamma \ts \lambda x.M})} \deno{T}\,,
        \]
      where, of course $\Lambda\inv(\deno{\Gamma\ts\lambda x.M})=\deno{\Gamma,x\ts M}$, and that the projection $\pr_\Gamma \from \oc\V_{\Gamma,x}$ is a right inverse for the composite $\runit_{\oc\V_\Gamma};(\oc\V_{\Gamma}\tensor(0;\cell));\cong$, since we have a commutative diagram
      \[
        \begin{tikzcd}[column sep=50pt]
          \oc\V_\Gamma \arrow[r, "\runit_{\oc\V_\Gamma}"] \arrow[d, Rightarrow, no head]
            & \oc\V_\Gamma \tensor I \arrow[r, "\oc\V_\Gamma\tensor(0;\cell)"] \arrow[d, "\id_{\oc\V_\Gamma\tensor I}" description]
              & \oc\V_\Gamma \tensor \oc\Varr \arrow[d] \arrow[dl, "\oc\V_\Gamma\tensor()" description] \\
          \oc\V_\Gamma
            & \oc\V_\Gamma \tensor I \arrow[l, "\runit\inv"']
              & {\oc\V_{\Gamma,x}\,.} \arrow[ll, "\oc\pr_\Gamma", bend left=20]
        \end{tikzcd}
        \]
      Then we may prove this case using the commutative diagram in Figure \ref{FigSoundnessOoptoNew}.
      \begin{figure}[htbp]
        \[
          \begin{tikzcd}[column sep=34pt, row sep=30pt]
            I \arrow[r, "\oc\deno s"] \arrow[d, "{\oc\langle\deno s,0\rangle}" description]
              &[5pt] \oc S_\Gamma \arrow[r, "\cell^\Gamma"]
                &[-5pt] \oc\V_\Gamma \arrow[r, "\mu_{\V_\Gamma}"] \arrow[d, "\mu_{\V_\Gamma}" description]
                  & \oc\V_\Gamma \tensor \oc\V_\Gamma \arrow[d, "\runit_{\oc\V_\Gamma};(\oc\V_\Gamma\tensor (0;\cell));\cong" {description, xshift=-16pt}] \\
            {\oc S_{\Gamma,x}} \arrow[ur, "\oc\pr_\Gamma" description] \arrow[d, "{\cell^{\Gamma,x}}" description]
              &
                & \oc\V_\Gamma\tensor\oc\V_\Gamma
                  & {\oc\V_{\Gamma,x}\tensor\oc\V_\Gamma} \arrow[d, "{\deno{\Gamma,x\ts M}\tensor\oc\V_\Gamma}" description] \arrow[l, "\pr_\Gamma\tensor\oc\V_\Gamma"'] \\
            {\oc\V_{\Gamma,x}} \arrow[uurr, "\oc\pr_\Gamma" description] \arrow[d, "{\mu_{\V_{\Gamma,x}}}"]
              &
                &
                  & \deno T \tensor \oc\V_\Gamma \arrow[d, "{\wk_{\deno T,\oc\V_\Gamma}}" description] \\
            {\oc\V_{\Gamma,x}\tensor\oc\V_{\Gamma,x}} \arrow[uurr, "\oc\pr_\Gamma\tensor\oc\pr_\Gamma" {description, pos=0.7}] \arrow[uurrr, "{\V_{\Gamma,x}\tensor\pr_\Gamma}" {description, pos=0.7}] \arrow[r, "{\deno{\Gamma,x\ts M}\tensor\oc\V_{\Gamma,x}}"' yshift=-3pt]
              & {\deno T \tensor \oc\V_{\Gamma,x}} \arrow[r, "{\wk_{\deno T,\oc\V_{\Gamma,x}}}"' yshift=-3pt]
                & {\deno T \sequoid \oc\V_{\Gamma,x}} \arrow[r, "\deno T \sequoid\pr_\Gamma"' yshift=-3pt]
                  & \deno T \sequoid\oc\V_\Gamma
          \end{tikzcd}
          \]
        \caption{The conclusion of Lemma \ref{LemSoundnessOopto} holds for the $\neww$ rule.}
        \label{FigSoundnessOoptoNew}
      \end{figure}
    \item Now consider the rules
      \begin{mathpar}
        x,\Gamma,s\ts x\gets n \oopto x,\Gamma,(s\vert x\mapsto n) \ts \skipp
        \and
        x,\Gamma,s\ts \oc x\oopto x,\Gamma,s\ts s(x)\,.
      \end{mathpar}
      In each of these cases, the outer context $\Gamma$ is unchanged by the rule; moreover, each term ignores the variables in $\Gamma$.
      So the denotations terms-in-context $x,\Gamma\ts M\from T$ on either side of each rule are of the form
      \[
        \oc\V_{x,\Gamma} \xrightarrow{\oc\pr_1} \oc \Varr \xrightarrow{\deno{x\ts M}} \deno{T}\,.
        \]
      Now the diagram in Figure \ref{FigSoundnessOoptoContext} tells us that each $\seqdeno{\Gamma,x,s\ts M}$ may be written as the composite
      \begin{mathpar}
        \oc\deno{s\vert_\Gamma};\cell^\Gamma;\lunit_{\oc\V_\Gamma};\\((\oc\deno{s\vert_x};\cell;\mu_{\Varr};(\deno{x\ts M}\tensor\oc\Varr);\wk_{\deno T,\oc\Varr})\tensor\oc\V_\Gamma);\\\wk_{\deno T\sequoid \oc\Varr,\oc\V_\Gamma};\passoc_{\deno T,\oc\Varr,\oc\V_\Gamma};\cong
      \end{mathpar}
      (where $\cong$ represents the natural isomorpism between $\deno T\sequoid(\oc\Varr\tensor\oc\V_\Gamma)$ and $\deno T \sequoid\oc\V_{x,\Gamma}$),
      and so is completely determined by the value of the composite
      \[
        \oc\deno{s\vert_x};\cell;\mu_{\Varr};(\deno{x\ts M}\tensor\oc\Varr);\wk_{\deno T,\oc\Varr}\,;
        \]
      i.e., the sequoidal denotation $\seqdeno{x,s\vert_x\ts M}$.

      This tells us that we can assume that $\Gamma$ is the empty context, so that the rules take on the form
      \begin{mathpar}
        x,s\ts x\gets n \oopto x,(x\mapsto n)\ts \skipp
        \and
        x,s \ts \oc x \oopto x,s\ts s(x)\,.
      \end{mathpar}
      \begin{SidewaysFigure}
        \[
          \begin{tikzcd}[ampersand replacement=\&, column sep=5pt]
            I \arrow[r, "\oc\deno{s\vert_\Gamma}"] \arrow[d, "\oc\deno s"' {description, yshift=2pt}, thick, dashed]
              \& \oc S_\Gamma \arrow[r, "\cell^\Gamma"] \arrow[dr, "\lunit_{\oc S_\Gamma}" description, dotted]
                \&[18pt] \oc \V_\Gamma \arrow[r, "\lunit_{\oc\V_\Gamma}"] 
                  \&[17pt] I \tensor \oc\V_\Gamma \arrow[dd, "\oc\deno{s\vert_x}\tensor\oc\V_\Gamma" description, thick] \\
            {\oc S_{x,\Gamma}} \arrow[drr, Isom, dotted] \arrow[dd, "{\cell^{x,\Gamma}}"' description, thick, dashed]
              \&
                \& I \tensor \oc S_\Gamma \arrow[ur, "I \tensor\cell^\Gamma" description, dotted] \arrow[d, "\oc\deno{s\vert_x}\tensor\oc S_\Gamma" description, dotted]
                  \& \\
            %
              \&
                \& \oc\bN\tensor\oc S_\Gamma \arrow[r, "\oc\bN\tensor\cell^\Gamma" description, dotted] \arrow[dr, "\cell\tensor\cell^\Gamma" description, dotted]
                  \& \oc\bN \tensor\oc\V_\Gamma \arrow[d, "\cell\tensor\oc\V_\Gamma" description, thick] \\
            {\oc\V_{x,\Gamma}} \arrow[rrr, Isom, dotted] \arrow[d, "{\mu_{\V_{x,\Gamma}}}"' description, thick, dashed]
              \&
                \&
                  \& \oc\Varr \tensor \oc\V_\Gamma \arrow[d, "\mu_{\Varr}\tensor\oc\V_\Gamma" description, thick] \arrow[dl, "\mu_{\Varr}\tensor\mu_{\V_\Gamma}" description, bend right=15, dotted] \\[10pt]
            {\oc\V_{x,\Gamma}\tensor\oc\V_{x,\Gamma}} \arrow[r, dotted] \arrow[d, "{\oc\pr_1\tensor\oc\V_{x,\Gamma}}" description, dotted] \arrow[dd, "{\deno{x,\Gamma\ts M}\tensor\oc\V_{x,\Gamma}}" description, bend right=68, thick, dashed, end anchor={[xshift=-3pt]}]
              \& (\oc\Varr \tensor \oc\V_\Gamma) \tensor (\oc\Varr \tensor\oc\V_\Gamma) \arrow[r, dotted] \arrow[d, "(\oc\Varr\tensor())\tensor(\oc\Varr\tensor\oc\V_{\Gamma})" description, dotted]
                \& (\oc\Varr \tensor \oc\Varr) \tensor (\oc\V_\Gamma \tensor\oc\V_\Gamma) \arrow[r, "(\oc\Varr\tensor\oc\Varr)\tensor(()\tensor\oc\V_\Gamma)" yshift=5pt, dotted]
                  \& (\oc\Varr \tensor \oc\Varr) \tensor\oc\V_\Gamma \arrow[d, "(\deno{x\ts M}\tensor \oc\Varr) \tensor \oc\V_\Gamma" description, thick] \\[30pt]
            {\oc\Varr \tensor \oc\V_{x,\Gamma}} \arrow[r, Isom, dotted] \arrow[d, "{\deno{x\ts M} \tensor \oc\V_{x,\Gamma}}"' {description, yshift=1pt}, dotted]
              \& \oc\Varr \tensor (\oc\Varr \tensor\oc\V_\Gamma) \arrow[urr, "{\assoc_{\oc\Varr,\oc\Varr,\oc\V_\Gamma}\inv}" {description, xshift=-5pt, yshift=5pt}, start anchor={[xshift=-25pt]}, bend left=5, dotted] \arrow[r, "\deno{x\ts M} \tensor (\oc\Varr\tensor\oc\V_\Gamma)" yshift=2pt, dotted]
                \& \deno{T} \tensor (\oc\Varr \tensor \oc\V_\Gamma) \arrow[r, "{\assoc_{\deno T,\oc\Varr,\oc\V_\Gamma}}" yshift=2pt, dotted] \arrow[ddl, "{\wk_{\deno T,(\oc\Varr\tensor\oc\V_\Gamma)}}" description, dotted]
                  \& (\deno T \tensor \oc\Varr) \tensor \oc\V_\Gamma\arrow[d, "{\wk_{\deno T,\oc\Varr}\tensor\oc\V_\Gamma}" {description, yshift=1pt}, thick] \\
            {\deno T \tensor \oc\V_{x,\Gamma}} \arrow[d, "{\wk_{\deno T,\oc\V_{x,\Gamma}}}"' {description, yshift=2pt}, thick, dashed] \arrow[urr, Isom, dotted]
              \&
                \&
                  \& (\deno T \sequoid \oc\Varr) \tensor \oc\V_\Gamma \arrow[d, "{\wk_{(\deno T\sequoid\oc\Varr),\oc\V_\Gamma}}" {description, yshift=2pt}] \\[5pt]
            {\deno T \sequoid \oc\V_{x,\Gamma}} \arrow[r, Isom]
              \& \deno T \sequoid (\oc\Varr \tensor\oc\V_\Gamma)
                \&
                  \& (\deno T \sequoid \oc\Varr) \sequoid\oc\V_\Gamma \arrow[ll, "{\passoc_{\deno T,\oc\Varr,\oc\V_\Gamma}}"]
          \end{tikzcd}
          \]
          \caption{Diagram proving that if we want to prove the conclusion of Lemma \ref{LemSoundnessOopto} for a small-step rule that does not change the context and only mentions one variable, then it suffices to assume that that variable is the only variable in the context.}
          \label{FigSoundnessOoptoContext}
      \end{SidewaysFigure}

      Now the commutative diagrams in Figure \ref{FigSoundnessOoptoStorage} prove that
      \[
        \seqdeno{x,s\vert_x\ts x\gets n} = \seqdeno{x,(x\mapsto n)\ts \skipp}
        \]
      and that
      \[
        \seqdeno{x,s\vert_x\ts \oc x} = \seqdeno{x,s\vert_x\ts s(x)}\,,
        \]
      completing the proof.\qedhere
      \begin{figure}[htbp]
        \begin{mathpar}
          \begin{tikzcd}[column sep=41pt, row sep=30pt]
            I \arrow[r, "\oc s(x)"] \arrow[d, "\oc n" description] \arrow[dr, Rightarrow, no head]
              & \oc \bN \arrow[r, "\cell"] \arrow[d, "()" description] \arrow[dr, "\cell_0" description]
                & \oc\Varr \arrow[r, "\mu_{\Varr}"] \arrow[dd, "\alpha_{\Varr}" description, controls={+(1,-0.25) and +(0.8,0.75)}, start anchor={[yshift=-1pt]}, end anchor = {[xshift=25pt]}] \arrow[dr, phantom, "\text{P. \ref{PropFormulaForAlpha}}" {xshift=20pt, yshift=-7pt}]
                  & \oc \Varr \tensor \oc \Varr \arrow[d, "\der_{\Varr} \tensor \oc\Varr" {description, near start}]  \\
            \oc\bN \arrow[dd, "\cell" description] \arrow[ddr, "\lunit_{\oc\bN}" description, end anchor={[xshift=-3pt, yshift=-2pt]}] \arrow[ddrr, phantom, "\text{L. \ref{LemCellProjections}}" {xshift=2pt, yshift=20pt}]
              & I \arrow[l, "\oc n" description]
                & \Varr \sequoid \oc\bN \arrow[d, "\Varr\sequoid\cell" description] \arrow[dd, "\pr_n\sequoid\oc\bN" {description, pos=0.44}, controls={+(-2.5,-0.2) and (-0.5,0.5)}]
                  & \Varr \tensor \oc\Varr \arrow[dl, "{\wk_{\Varr,\oc\Varr}}" description] \arrow[dd, "\pr_n\tensor\oc\Varr" description] \\
            %
              & \bC \tensor \oc\bN \arrow[dr, "{\wk_{\bC,\oc\bN}}" description] \arrow[ddr, "\bC \tensor \cell" description]
                & \Varr \sequoid\oc\Varr \arrow[ddr, "\pr_n\sequoid\oc\Varr" description]
                  & \\
            \oc\Varr \arrow[d, "\mu_{\Varr}" description] \arrow[dr, "\lunit_{\oc\Varr}" description]
              & I \tensor \oc\bN \arrow[u, "\skipp\tensor\oc\bN" {description, pos=0.44}] \arrow[d, "I\tensor \cell" description]
                & \bC \sequoid \oc\bN \arrow[dr, "\bC \sequoid \cell" description]
                  & \bC \tensor \oc\Varr \arrow[d, "{\wk_{\bC,\oc\Varr}}" {description, pos=0.44}] \\
            \oc\Varr \tensor \oc\Varr \arrow[r, "()\tensor\oc\Varr"']
              & I \tensor \oc\Varr \arrow[r, "\skipp \tensor \oc\Varr"']
                & \oc\bC \tensor \oc\Varr \arrow[r, "{\wk_{\bC,\oc\Varr}}"']
                  & \bC \sequoid \oc\Varr
          \end{tikzcd}
          \and
          \begin{tikzcd}[row sep=30pt]
            I \arrow[r, "\oc s(x)"] \arrow[d, "\oc s(x)" description] \arrow[dr]
              &[-31pt] \oc\bN \arrow[r, "\cell"] \arrow[dr, "\mu_\bN" description] \arrow[ddr, "\cell_0" {description, near end}, bend left=50, end anchor={[xshift=4pt]}] \arrow[ddd, phantom, "\text{L.\ref{LemCellProjections}}" {xshift=3pt,yshift=-7pt}]
                &[-100pt] \oc\Varr \arrow[r, "\mu_{\Varr}"] \arrow[ddd, "\alpha_{\Varr}" description] \arrow[dr, phantom, "\text{P.\ref{PropFormulaForAlpha}}", {yshift=-5pt}]
                  &[-21pt] \oc\Varr \tensor \oc\Varr \arrow[d, "\der_{\Varr}\tensor\oc\Varr" {description, near start}] \\
            |[alias=Z]| \oc\bN \arrow[d, "\lunit_{\oc\bN}" description] \arrow[dd, "\cell" description]
              & |[left=40pt of Z.east]| I \tensor I \arrow[dl, "I \tensor \oc s(x)" description] \arrow[r, "\oc s(x) \tensor \oc s(x)"] \arrow[d, "s(x)\tensor \oc s(x)" description]
                & |[left=50pt of Z.west]| \oc \bN \tensor \oc\bN \arrow[dl, "\der_\bN \tensor \oc\bN" {description, near start}, bend right=10]
                  & \Varr \tensor \oc\Varr \arrow[ddl, "{\wk_{\Varr,\oc\Varr}}" {description, pos=0.65}] \arrow[dd, "\pr_{\bN}\tensor\oc\Varr" description] \\
            |[right=of Z.west]| I \tensor \oc\bN \arrow[r, "s(x)\tensor\oc\bN"'] \arrow[ddr, "I \tensor \cell" description]
              & \bN \tensor \oc\bN \arrow[d, "{\wk_{\bN,\oc\bN}}" description] \arrow[ddr, "\bN \tensor\cell" {description, pos=0.4, xshift=-5pt}, bend right=30]
                & |[left=of Z]| \Varr \sequoid \oc\bN \arrow[dl, "\pr_\bN \sequoid \oc\bN" {description, pos=0.65}, bend left=30] \arrow[d, "\Varr\sequoid\cell" {description, xshift=-5pt}, bend left=30]
                  & \\
            \oc\Varr \arrow[d, "\mu_{\Varr}" description] \arrow[dr, "\lunit_{\Varr}" description]
              & |[right=of Z]| \bN \sequoid \oc\bN \arrow[drr, "\bN \sequoid \cell" {description, pos=0.4}, controls={+(0.2,-1.2) and +(-0.3,0.2)}, to=Y.north west, end anchor={[yshift=-2pt]}]
                & \Varr \sequoid\oc\Varr \arrow[dr, "\pr_\bN \sequoid \oc\Varr" {description, pos=0.35}]
                  & \bN \tensor \oc\Varr \arrow[d, "{\wk_{\bN},\oc\Varr}" description] \\
            \oc\Varr \tensor \oc\Varr \arrow[r, "()\tensor\oc\Varr"']
              & I \tensor \oc\Varr \arrow[r, "s(x)\tensor\oc\Varr"']
                & \bN \tensor \oc\Varr \arrow[r, "{\wk_{\bN,\oc\Varr}}"']
                  & |[alias=Y]| \bN \sequoid \oc\Varr
          \end{tikzcd}
        \end{mathpar}
        \caption[Diagrams to prove that the conclusion of Lemma \ref{LemSoundnessOopto} holds for the storage cell rules.]{Diagrams to prove that the conclusion of Lemma \ref{LemSoundnessOopto} holds for the storage cell rules.  
        References in the middle of a shape refer to Lemma \ref{LemCellProjections} and Proposition \ref{PropFormulaForAlpha} above.  
        Note the prominent role played in both diagrams by the anamorphism square for $\cell$ as in Section \ref{SecCell}.}
        \label{FigSoundnessOoptoStorage}
      \end{figure}
  \end{itemize}
\end{proof}

Now that we have dealt with the base rules, we can move on to the full relation $\opto$.

\begin{lemma}
  The relation $\opto$ between triples $\Gamma,s\ts M$ preserves the composite
  \[
    \oc\deno s;\cell^\Gamma;\mu_{\V_\Gamma};(\deno{\Gamma\ts M}\tensor\oc\V_\Gamma);\wk_{\deno T,\oc\V_\Gamma}\,.
    \]
  I.e., if $\Gamma,s\ts M \oopto \Gamma,\Delta,s'\ts N$, where $M,N$ have type $T$, and $\EE$ is a context of type $U$ with a hole of type $T$, then the following diagram commutes.
  \[
    \begin{tikzcd}[column sep=24pt]
      I \arrow[r, "\oc \deno s"] \arrow[d, "\oc \deno {s'}"']
        & \oc S_\Gamma \arrow[r, "\cell^\Gamma"]
          & \oc \V_\Gamma \arrow[r, "\mu_{\V_\Gamma}"]
            &[22pt] \oc \V_\Gamma \tensor \oc \V_\Gamma \arrow[d, "\deno{\Gamma\ts \EE[M]}\tensor \oc V_\Gamma"] \\
      {\oc S_{\Gamma,\Delta}} \arrow[d, "{\cell^{\Gamma,\Delta}}"']
        &
          &
            & \deno U \tensor \oc V_\Gamma \arrow[d, "{\wk_{\deno U,\oc \V_{\Gamma}}}"] \\
      {\oc \V_{\Gamma,\Delta}} \arrow[d, "{\mu_{\V_{\Gamma,\Delta}}}"']
        &
          &
            &  \deno{U}\sequoid \oc\V_\Gamma \arrow[d, "{\deno{U} \sequoid (\Gamma+0_\Delta)}"] \\
      {\oc \V_{\Gamma,\Delta} \tensor \oc \V_{\Gamma,\Delta}} \arrow[rr, "{\deno{\Gamma,\Delta\ts \EE[N]} \tensor \oc \V_{\Gamma,\Delta}}"]
        &
          & {\deno U \tensor \oc \V_{\Gamma,\Delta}} \arrow[r, "{\wk_{\deno U,\oc \V_{\Gamma,\Delta}}}"]
            & {{\deno U \sequoid \oc \V_{\Gamma,\Delta}}\,,}
    \end{tikzcd}
    \]
  \label{LemSoundnessOpto}
\end{lemma}
\begin{proof}
  We use Lemma \ref{LemEvContexLemma} to reduce this to Lemma \ref{LemSoundnessOpto}.  
  Indeed, Lemma \ref{LemEvContexLemma} tells us that for any $\Gamma,s\ts M$, $\deno{\Gamma\ts \EE[M]}$ may be written as
  \[
    \oc\V_\Gamma \xrightarrow{\mu_{\V_\Gamma}} \oc\V_\Gamma\tensor\oc\V_\Gamma \xrightarrow{\deno{\Gamma\ts M}\tensor\sigma} \deno T \tensor A \xrightarrow{\wk_{\deno T,A}} \deno T \sequoid A \xrightarrow{\tau} \deno U
    \]
  for suitably chosen $A,\sigma,\tau$.

  Let us write $\seqdeno{\Gamma,s\ts M}$ for the composite
  \[
    \oc\deno s;\cell^\Gamma;\mu_{\V_\Gamma};(\deno{\Gamma\ts M}\tensor\oc\V_\Gamma);\wk_{\deno T,\oc\V_\Gamma};(\deno T \sequoid (\Gamma+0_\Delta))\,.
    \]
  So we are trying to show that $\seqdeno{\Gamma,s\ts \EE[M]}=\seqdeno{\Gamma,\Delta,s'\ts \EE[N]};(\deno U \sequoid \oc\pr_\Gamma)$.

  Now the diagram in Figure \ref{FigSoundnessOpto} shows us that for any $\Gamma,s\ts M$ we may write
  \[
    \seqdeno{\Gamma,s\ts \EE[M]} = \seqdeno{\Gamma,s\ts M};(\deno T \sequoid (\mu_{\V_\Gamma};(\sigma\tensor\oc\V_\Gamma));\passoc_{\deno T,A,\oc\V_\Gamma}\inv;(\tau\sequoid\oc\V_\Gamma)\,.
    \]

  \begin{SidewaysFigure}
    \[
      \begin{tikzcd}[column sep=45pt, row sep=30pt, ampersand replacement=\&]
        \oc S_\Gamma \arrow[d, "\cell^\Gamma"', thick]
          \& I \arrow[l, "\oc\deno s"', thick]
            \&
              \& \\
        \oc\V_\Gamma \arrow[r, "\mu_{\V_\Gamma}", thick] \arrow[d, "\mu_{\V_\Gamma}"', thick, dashed]
          \& \oc\V_\Gamma \tensor \oc\V_\Gamma \arrow[r, "\deno{\Gamma\ts M}\tensor\oc\V_\Gamma", thick] \arrow[dd, "\oc\V_\Gamma\tensor\mu_{\V_\Gamma}" description, dotted]
            \& \deno T \tensor \oc\V_\Gamma \arrow[r, "{\wk_{\deno T,\oc\V_\Gamma}}", thick] \arrow[dd, "\deno T \tensor \mu_{\V_\Gamma}" description, dotted]
              \& \deno T \sequoid \oc\V_\Gamma \arrow[dd, "\deno T \sequoid \mu_{\V_\Gamma}"] \\
        \oc\V_\Gamma \tensor \oc\V_\Gamma \arrow[d, "\mu_{\V_\Gamma}\tensor\oc\V_\Gamma"', thick, dashed]
          \&
            \&
              \& \\
        (\oc\V_\Gamma \tensor \oc\V_\Gamma) \tensor \oc\V_\Gamma \arrow[r, "{\assoc_{\oc\V_\Gamma,\oc\V_\Gamma,\oc\V_\Gamma}}" yshift=3pt, dotted] \arrow[d, "(\deno{\Gamma\ts M}\tensor \sigma) \tensor\oc\V_\Gamma"', thick, dashed]
          \& \oc\V_\Gamma \tensor (\oc\V_\Gamma \tensor \oc\V_\Gamma) \arrow[r, "\deno{\Gamma\ts M} \tensor (\oc\V_\Gamma\tensor \oc\V_\Gamma)" yshift=3pt, dotted] \arrow[d, "\deno{\Gamma\ts M}\tensor (\sigma\tensor\oc\V_\Gamma)" description, dotted]
            \& \deno T \tensor (\oc\V_\Gamma\ \tensor \oc\V_\Gamma) \arrow[r, "{\wk_{\deno T,\oc\V_\Gamma\tensor\oc\V_\Gamma}}" yshift=3pt, dotted] \arrow[dl, "\deno T \tensor (\sigma \tensor\oc\V_\Gamma)" description, dotted]
              \& \deno T \sequoid (\oc\V_\Gamma \tensor \oc\V_\Gamma) \arrow[d, "\deno T \sequoid (\sigma \tensor \oc\V_\Gamma)"] \\
        (\deno T \tensor A) \tensor \oc\V_\Gamma \arrow[d, "{\wk_{\deno T,A}\tensor\oc\V_\Gamma}"', thick, dashed] \arrow[r, "{\assoc_{\deno T,A,\oc\V_\Gamma}}", dotted]
          \& \deno T \tensor (A \tensor \oc\V_\Gamma) \arrow[rr, "{\wk_{\deno T,A\tensor\oc\V_\Gamma}}", dotted]
            \&
              \& \deno T \sequoid (A \tensor \oc\V_\Gamma) \\
        (\deno T \sequoid A) \tensor \oc\V_\Gamma \arrow[r, "{\wk_{\deno T\sequoid A,\oc\V_\Gamma}}", dotted] \arrow[d, "\tau\tensor\oc\V_\Gamma", thick, dashed]
          \& (\deno T \sequoid A) \sequoid \oc\V_\Gamma \arrow[d, "\tau\sequoid\oc\V_\Gamma"] \arrow[urr, "{\passoc_{\deno T,A,\oc\V_\Gamma}}"']
            \&
              \& \\
        \deno U \tensor \oc\V_\Gamma \arrow[r, "{\wk_{\deno U,\oc\V_\Gamma}}", thick, dashed]
          \& \deno U \sequoid \oc\V_\Gamma
            \&
              \&
      \end{tikzcd}
      \]
      \caption{Diagram proving that the conclusion of Lemma \ref{LemSoundnessOopto} can be lifted to the $\opto$ relation.}
      \label{FigSoundnessOpto}
  \end{SidewaysFigure}

  Therefore, Lemma \ref{LemSoundnessOopto} tells us that if $\Gamma,s\ts M \oopto \Gamma,\Delta,s'\ts N$, then we have
  \begin{IEEEeqnarray*}{Cl}
    & \seqdeno{\Gamma,s\ts \EE[M]} \\
    = & \seqdeno{\Gamma,s\ts M};(\deno T\sequoid(\mu_{\V_\Gamma};(\sigma\tensor\oc\V_\Gamma));\passoc\inv;(\tau\sequoid\oc\V_\Gamma) \\
    = & \seqdeno{\Gamma,s\ts N};(\deno T \sequoid \oc\pr_\Gamma);(\deno T\sequoid(\mu_{\V_\Gamma};(\sigma\tensor\oc\V_\Gamma));\passoc\inv;(\tau\sequoid\oc\V_\Gamma) \\
    = & \seqdeno{\Gamma,s\ts N};(\deno T \sequoid (\mu_{\V_{\Gamma,\Delta}};((\pr_\Gamma;\sigma)\tensor\oc\V_{\Gamma,\Delta})));\passoc\inv; \\
    &\qquad(\tau\sequoid\oc\V_\Gamma);(\deno U\sequoid\oc\pr_\Gamma) \\
    = & \seqdeno{\Gamma,s\ts \EE[N]};(\deno U\sequoid\oc\pr_\Gamma)\,,
  \end{IEEEeqnarray*}
  as desired.
\end{proof}

It is now a simple induction to show that we can extend this to the $\converges$ relation.

\begin{lemma}
  Suppose that $\Gamma,s\ts M \converges c,s'$, where $M,c\from T$.
  Then the following diagram commutes.
  \[
    \begin{tikzcd}[column sep=40pt]
      I \arrow[r, "\oc\deno s"] \arrow[d, "\oc\deno{s'}"']
        & \oc S_\Gamma \arrow[r, "\cell^\Gamma"]
          & \oc\V_\Gamma \arrow[r, "\mu_{\V_\Gamma}"]
            & \oc\V_\Gamma\tensor\oc\V_\Gamma \arrow[d, "\deno{\Gamma\ts M}\tensor\oc\V_\Gamma"] \\
      \oc S_\Gamma \arrow[d, "\cell^\Gamma"']
        &
          &
             & \deno{T} \tensor \oc\V_\Gamma \arrow[d, "{\wk_{\deno T,\oc\V_\Gamma}}"] \\
      \oc\V_\Gamma \arrow[r, "\mu_{\V_\Gamma}"]
        & \oc\V_\Gamma \tensor \oc\V_\Gamma \arrow[r, "\deno{\Gamma\ts c}\tensor\oc\V_\Gamma"]
          & \deno T \tensor \oc\V_\Gamma \arrow[r, "{\wk_{\deno T,\oc\V_\Gamma}}"]
            & \deno T \sequoid \oc\V_\Gamma
    \end{tikzcd}
    \]
  \label{LemSoundness}
\end{lemma}
\begin{proof}
  By Proposition \ref{PropBigToSmall}, there are sequences $\Gamma=\Gamma_1,\cdots,\Gamma_n=\Gamma,\Delta$, $s=s^{(1)},\cdots,s^{(n)}$, $M=M_1,\cdots,M_n=c$ such that
  \[
    \Gamma_1,s^{(1)}\ts M_1 \opto \cdots \opto \Gamma_n,s^{(n)} \ts M_n\,,
    \]
  and $s^{(n)}\vert_\Gamma = s'$.

  By inductively applying Lemma \ref{LemSoundnessOpto}, we see that we have a commutative diagram
  \[
    \begin{tikzcd}[column sep=28pt]
      I \arrow[r, "\oc \deno s"] \arrow[d, "\oc \deno {s^{(n)}}"']
        & \oc S_\Gamma \arrow[r, "\cell^\Gamma"]
          & \oc \V_\Gamma \arrow[r, "\mu_{\V_\Gamma}"]
            &[22pt] \oc \V_\Gamma \tensor \oc \V_\Gamma \arrow[d, "\deno{\Gamma\ts M}\tensor \oc V_\Gamma"] \\
      {\oc S_{\Gamma,\Delta}} \arrow[d, "{\cell^{\Gamma,\Delta}}"']
        &
          &
            & \deno T \tensor \oc V_\Gamma \arrow[d, "{\wk_{\deno T,\oc \V_{\Gamma}}}"] \\
      {\oc \V_{\Gamma,\Delta}} \arrow[d, "{\mu_{\V_{\Gamma,\Delta}}}"']
        &
          &
            &  \deno{T}\sequoid \oc\V_\Gamma \\
      {\oc \V_{\Gamma,\Delta} \tensor \oc \V_{\Gamma,\Delta}} \arrow[rr, "{\deno{\Gamma,\Delta\ts c} \tensor \oc \V_{\Gamma,\Delta}}"]
        &
          & {\deno T \tensor \oc \V_{\Gamma,\Delta}} \arrow[r, "{\wk_{\deno T,\oc \V_{\Gamma,\Delta}}}"]
            & {{\deno T \sequoid \oc \V_{\Gamma,\Delta}}\,,} \arrow[u, "\deno T \sequoid \oc\pr_\Gamma"]
    \end{tikzcd}
    \]
  Now, since $s^{(n)}\vert_\Gamma = s'$, we have a commutative diagram
  \[
    \begin{tikzcd}[column sep=50pt]
      I \arrow[r, "\deno {s'}"] \arrow[d, "\deno{s^{(n)}}"']
        & \oc S_\Gamma \arrow[dd, "\cell^\Gamma"] \\
      {\oc S_{\Gamma,\Delta}} \arrow[ur, "\oc\pr_\Gamma" description] \arrow[d, "{\cell^{\Gamma,\Delta}}"']
        & \\
      {\oc\V_{\Gamma,\Delta}} \arrow[r, "\oc\pr_\Gamma"] \arrow[d, "{\mu_{\V_{\Gamma,\Delta}}}"']
        & \oc\V_\Gamma \arrow[d, "\mu_{\V_\Gamma}"] \\
      {\oc\V_{\Gamma,\Delta}\tensor\oc\V_{\Gamma,\Delta}} \arrow[d, "{\deno{\Gamma,\Delta\ts c}\tensor\oc\V_{\Gamma,\Delta}}"'] \arrow[r, "\oc\pr_\Gamma\tensor\oc\pr_\Gamma"]
        & \oc\V_\Gamma \tensor \oc\V_\Gamma \arrow[d, "\deno{\Gamma\ts c}\tensor\oc\V_\Gamma"]  \\
      {\deno T \tensor \oc\V_{\Gamma,\Delta}} \arrow[r, "\deno T \tensor \oc\pr_\Gamma"] \arrow[d, "{\wk_{\deno T,\oc\V_{\Gamma,\Delta}}}"']
        & \deno T \tensor \oc\V_\Gamma \arrow[d, "{\wk_{\deno T,\oc\V_\Gamma}}"] \\
      {\deno T \sequoid \oc\V_{\Gamma,\Delta}} \arrow[r, "\deno T \sequoid \oc\pr_\Gamma"]
        & \deno T \sequoid \oc\V_\Gamma \\
    \end{tikzcd}
    \]
  which, together with the diagram above, gives us the commutative diagram in the statement.
\end{proof}

It is then an easy corollary to show the sense in which our semantics is sound.

\begin{proposition}[\cite{SamsonGuyIAActive}]
  Suppose that $\Gamma,s\ts M\converges c,s'$.  
  Then
  \[
    \deno s;\cell^\Gamma;\deno{\Gamma\ts M} = \deno{s'};\cell^\Gamma;\deno{\Gamma\ts c}\,.
    \]
\end{proposition}
\begin{proof}
  Lemma \ref{LemSoundness}, plus the fact that if $\Gamma,s\ts P\from T$, then we have a commutative diagram
  \[
    \begin{tikzcd}[column sep=14.5pt]
      I \arrow[r, "\deno{s}", thick]
        & \oc S_\Gamma \arrow[r, "\cell^\Gamma", thick]
          & \oc\V_\Gamma \arrow[rr, "\mu_{\V_\Gamma}", thick, dashed] \arrow[ddd, "\deno{\Gamma\ts P}"', thick] \arrow[drr, "\runit_{\oc\V_\Gamma}" description, dotted]
            &
              & \oc\V_\Gamma \tensor \oc\V_\Gamma \arrow[rr, "\deno{\Gamma\ts P}\tensor\oc\V_\Gamma", thick, dashed] \arrow[d, "\oc\V_\Gamma\tensor()" description, dotted]
                &
                  & \deno T \tensor\oc\V_\Gamma \arrow[ddd, "{\wk_{\deno T,\oc\V_\Gamma}}" description, thick, dashed] \arrow[ddl, "\deno T \tensor ()" description, dotted] \\
      %
        &
          &
            &
              & \oc\V_\Gamma \tensor I \arrow[dr, "\deno{\Gamma\ts P}\tensor I" description, dotted] \arrow[dl, "{\wk_{\oc\V_\Gamma,I}}" description, dotted]
                &
                  & \\
      %
        &
          &
            & \oc\V_\Gamma \sequoid I \arrow[dr, "\deno{\Gamma\ts P}\sequoid I" description, dotted] \arrow[uul, "\run_{\oc\V_\Gamma}" description, dotted]
              &
                & \deno T \tensor I \arrow[dl, "{\wk_{\deno T,I}}" description, dotted]
                  & \\
      %
        &
          & \deno T
            &
              & \deno T \sequoid I \arrow[ll, "{\run_{\deno T}}"']
                &
                  & \deno T \sequoid \oc\V_\Gamma \arrow[ll, "\deno T \sequoid()"']\,,
    \end{tikzcd}
    \]
  allowing us to recover $\deno{s};\cell^\Gamma;\deno{\Gamma\ts P}$ from 
  \[
    \seqdeno{\Gamma,s\ts P} = \deno{s};\cell^\Gamma;\mu_{\V_\Gamma};(\deno{\Gamma\ts P}\tensor\oc\V_\Gamma);\wk_{\deno T,\oc\V_\Gamma}\,,
    \]
  for $P=M,c$.
\end{proof}

\section{Computational Adequacy}

Our proof of computational adequacy is based on that from \cite{SamsonGuyIAActive}, but modified to make use of the coalgebraic definition of the $\cell$ strategy.  
As is usual in proofs of computational adequacy, our proof relies on logical relations.

First, we note some additional order-theoretic properties of our model.  
For any game $A$, we have a strategy $\bot_A\from A$, given by $\bot_A=\{\epsilon\}$; i.e., the strategy that has no reply even for the very first move.  
It is clear that $\bot$ is the bottom element of the set of strategies for $A$, ordered by inclusion.  

It is then easy to see the following.
\begin{proposition}
  \begin{itemize}
    \item Given $\sigma\from A \implies B$, $\sigma;\bot_{B\implies C}=\bot_{A\implies C}$.
    \item Given a strict strategy $\tau\from B\implies C$, $\bot_{A\implies B};\tau=\bot_{A\implies C}$.
    \item Given a zigzag (copycat) strategy $\zz_\phi\from B \implies C$ and a strategy $\sigma\from A\implies B$, if $\sigma;\zz_\phi=\bot_{A\implies C}$ then $\sigma=\bot_{A\implies B}$.
  \end{itemize}
\end{proposition}

\begin{definition}
  Given a $\Var$-store $\Gamma$, a strategy $\sigma\from\oc\V_\Gamma \implies A$ and a $\Gamma$-store $s$, we write
  \[
    \seqdeno{s,\sigma}
    \]
  for the composite
  \[
    I \xrightarrow{\oc\deno s} \oc S_\Gamma \xrightarrow{\cell^\Gamma} \oc\V_\Gamma \xrightarrow{\mu_{\V_\Gamma}} \oc\V_\Gamma \tensor \oc\V_\Gamma \xrightarrow{\sigma\tensor\oc\V_\Gamma} A \tensor \oc\V_\Gamma \xrightarrow{\wk_{A,\oc\V_\Gamma}} A \sequoid \oc\V_\Gamma\,.
    \]
  In particular, if $\Gamma\ts M\from T$ is a term in context, then $\seqdeno{s,\deno{\Gamma\ts M}}$ is equal to the sequoidal denotation $\seqdeno{\Gamma,s\ts M}$.
\end{definition}

\begin{definition}
  We inductively define a relation $\plot_T^\Gamma$, where $\Gamma$ is a $\Var$-store and $T$ a type, between strategies for $\oc\V_\Gamma \implies \deno T$ and terms $\Gamma\ts M\from T$ in context as follows.
  \begin{itemize}
    \item If $X\in\{\bC,\bB,\bN\}$ is a datatype, $M\from X$ and $\sigma\from \oc\V_\Gamma\implies X$, then we say that $\sigma \plot_X^\Gamma u$ if for all $\Gamma$-stores $s$, either $\seqdeno{s,\sigma}=\bot_{A\sequoid\oc\V_\Gamma}$ or $\seqdeno{s,\sigma}=\seqdeno{s',u}$ for some $\Gamma$-store $s'$ and some canonical form $u\in X$ such that $\Gamma,s\ts M \converges u,s'$.

    \item If $\sigma\from \oc\V_\Gamma \implies \Varr$ and $\Gamma\ts M\from \Var$, we say that $\sigma\plot_{\Var}^\Gamma M$ if
      \[
        \sigma;\pr_n\plot_{\com}^\Gamma M\gets n
        \]
      for all $n$, and if
      \[
        \sigma;\pr_{\bN}\plot_{\nat}^\Gamma \oc M\,.
        \]
      
    \item If $\sigma \from \oc\V_\Gamma \implies (\oc\deno{S} \implies \deno T)$ and $M\from S \to T$, we say that $\sigma\plot_{S\to T}^\Gamma M$ if whenever $\tau\from \oc\V_\Gamma \implies \deno S$ is a strategy and $N\from S$ is a term such that $\tau\plot_S^\Gamma N$, then
      \[
        \left(\oc\V_\Gamma \xrightarrow{\mu_{\V_\Gamma}} \oc\V_\Gamma \tensor \oc\V_\Gamma \xrightarrow{\sigma\tensor\tau^\dag} (\oc\deno S \implies \deno T) \tensor \oc\deno S \xrightarrow{\ev} \deno T\right) \plot_T^\Gamma M\,N\,,
        \]
      and in addition if $S=\Var$ and $T=X$ for some datatype $X$, then
      \small
      \[
        \left(\oc\V_\Gamma \xrightarrow{\mu_{\V_\Gamma}} \oc\V_\Gamma \tensor \oc\V_\Gamma \xrightarrow{\sigma\tensor \cell} (\oc\Varr \implies X) \tensor \oc\Varr \xrightarrow{\ev} X\right) \plot_X^\Gamma \neww_X (\lambda x.M\,x)\,.
        \]
      \normalsize
  \end{itemize}
\end{definition}

\begin{lemma}
  Let $\Gamma\ts M,N\from T$ be terms in context of Idealized Algol such that
  \[
    \Gamma,s\ts M\opto \Gamma,s\ts N
    \]
  for all $\Gamma$-stores $s$.  
  Suppose $\sigma\from\oc\V_\Gamma\implies\deno T$ is a strategy such that $\sigma\plot_T^\Gamma N$.  
  Then $\sigma\plot_T^\Gamma M$.
\end{lemma}
\begin{proof}
  Induction on $T$.

  Suppose that $\Gamma,s\ts M,N\ts X$, where $X$ is some datatype, and that $\Gamma,s\ts M\opto \Gamma,s\ts N$ for all $\Gamma$-stores $s$.  
  Fix some $\Gamma$-store $s$ and some strategy $\sigma\from \oc\V_\Gamma\implies \deno T$, and suppose that $\sigma\plot_T^\Gamma N$.  

  If $\seqdeno{s,\sigma}\ne\bot_{\oc\V_\Gamma\implies X}$, then by hypothesis it is equal to $\seqdeno{s',u}$ for some $u$ such that $\Gamma,s\ts N\converges u,s'$.  
  Then, by Lemma \ref{LemSmallToBig}, $\Gamma,s\ts M\converges u,s'$.

  If $\Gamma,s\ts M,N\from \Var$ and $\sigma\plot_\Var^\Gamma N$, then we have $\sigma;\pr_n\plot_\com^\Gamma N\gets n$ for each $n$ and $\sigma;\pr_\bN\plot_\nat^\Gamma \oc N$.  
  If $\Gamma,s\ts M\opto \Gamma,s\ts N$, then $\Gamma,s\ts M\gets n\opto\Gamma,s\ts N\gets n$ for each $n$ and $\Gamma,s\ts \oc M\opto\Gamma,s\ts \oc N$.  
  Then, by the previous paragraph, we must have $\sigma;\pr_n\plot_{\com}^\Gamma M\gets n$ for each $n$ and $\sigma;\pr_\bN\plot_{\nat}^\Gamma \oc M$, and therefore $\sigma\plot_{\Var}^\Gamma M$.

  Lastly, suppose that 
\end{proof}

\begin{lemma}
  Let $\Gamma$ be a $\Var$-context, let $\Delta$ be an arbitrary context and let $T$ be an Idealized Algol type.
  Write $\Delta=x_1\from T_1,\cdots,x_n\from T_n$.  
  Suppose that $\sigma_i\from \oc\V_\Gamma \implies \deno{T_i}$ are strategies and $\Gamma\ts N_i\from T_i$ are terms-in-context such that $\sigma_i\plot_{T_i}^\Gamma N_i$ for each $i$.  

  Given a strategy $\sigma\from \oc\V_{\Gamma,\Delta}\implies \deno{T}$, we write
  \[
    (\sigma_i)\semicom \sigma
    \]
  for the composite
  \[
    \oc\V_\Gamma \xrightarrow{\langle \oc\V_\Gamma,\sigma_1^\dag,\cdots,\sigma_n^\dag\rangle} \oc\V_{\Gamma,\Delta} \xrightarrow{\sigma} \deno{T}\,.
    \]

  Then for any term-in-context $\Gamma,\Delta \ts M\from T$, we have
  \[
    (\sigma_i)\semicom\deno{\Gamma\ts M} \plot_T^{\Gamma} M[N_i/x_i]\,.
    \]
\end{lemma}

\bibliographystyle{alpha2}
\bibliography{../common/phd_bibliography}

\end{document}

\documentclass{article}

\let\FEWFONTS=1
\let\THESIS=1
\usepackage[utf8]{inputenc}

\usepackage{graphicx} % support the \includegraphics command and options

\usepackage{parskip} % Activate to begin paragraphs with an empty line rather than an indent

%%% PACKAGES
\usepackage{booktabs} % for much better looking tables
\usepackage{array} % for better arrays (eg matrices) in maths
\ifdefined\BEAMER
\else
\usepackage{paralist} % very flexible & customisable lists (eg. enumerate/itemize, etc.)\prefix\t$.
\fi
\usepackage{verbatim} % adds environment for commenting out blocks of text & for better verbatim
\ifdefined\BEAMER
\else
\ifdefined\THESIS
\usepackage{subcaption}
\else
\usepackage{subfig} % make it possible to include more than one captioned figure/table in a single float
\fi
\fi
\usepackage{mathtools} % for the all important \coloneqq symbol
\usepackage{hyperref} % for hyperreferences
\usepackage{IEEEtrantools} % for \IEEEeqnarray
\usepackage{pbox} % for \pbox
\usepackage{multirow,bigdelim} % for \multirow
\usepackage{lettrine} % For the drop cap
\usepackage{mathpartir} % for \inferrule, \inferrule* and the mathpar environment
\usepackage{listings}

\usepackage{caption}
\captionsetup{singlelinecheck=off}

\ifdefined\NOTARTICLE
\else

%%% ToC (table of contents) APPEARANCE
\usepackage[nottoc,notlof,notlot]{tocbibind} % Put the bibliography in the ToC
\usepackage[titles,subfigure]{tocloft} % Alter the style of the Table of Contents
\renewcommand{\cftsecfont}{\rmfamily\mdseries\upshape}
\renewcommand{\cftsecpagefont}{\rmfamily\mdseries\upshape} % No bold!

\fi

%% Font things %%
\usepackage{amssymb}
\usepackage{cmll} % Linear logic symbols!
\ifdefined\FEWFONTS
\else
\usepackage{bm} % for bold Greek letters
\fi
\usepackage{stmaryrd}
\usepackage{bbm}

%% Get the sqsubsetneqq character from the mathabx package
\DeclareFontFamily{U}{mathb}{\hyphenchar\font45}
\DeclareFontShape{U}{mathb}{m}{n}{
      <5> <6> <7> <8> <9> <10> gen * mathb
      <10.95> mathb10 <12> <14.4> <17.28> <20.74> <24.88> mathb12
      }{}
\DeclareSymbolFont{mathb}{U}{mathb}{m}{n}

\DeclareMathSymbol{\sqsubsetneq}    {3}{mathb}{"88}
\DeclareMathSymbol{\varsqsubsetneq} {3}{mathb}{"8A}
\DeclareMathSymbol{\varsqsubsetneqq}{3}{mathb}{"92}
\DeclareMathSymbol{\sqsubsetneqq}   {3}{mathb}{"90}

%% Get the left and right moons from the wasysym package

\DeclareFontFamily{U}{wasy}{}
\DeclareFontShape{U}{wasy}{m}{n}{ <5> <6> <7> <8> <9> gen * wasy
      <10> <10.95> <12> <14.4> <17.28> <20.74> <24.88>wasy10  }{}
\DeclareFontShape{U}{wasy}{b}{n}{ <-10> sub * wasy/m/n
 <10> <10.95> <12> <14.4> <17.28> <20.74> <24.88>wasyb10 }{}
\DeclareFontShape{U}{wasy}{bx}{n}{ <-> sub * wasy/b/n}{}

\def\wasyfamily{\fontencoding{U}\fontfamily{wasy}\selectfont}
\def\leftmoon   {\mbox{\wasyfamily\char36}}
\def\rightmoon  {\mbox{\wasyfamily\char37}}

%% Lists %%
\usepackage{enumerate}

%% Graphics %%
\usepackage{tikz}
\usetikzlibrary{cd}
\usetikzlibrary{patterns}
\usetikzlibrary{calc}
\usetikzlibrary{decorations.pathmorphing}
\usetikzlibrary{positioning}

\tikzset{inlinearrows/.style={anchor=base,baseline,x=0.6\baselineskip,y=0.6\baselineskip}}

\ifdefined\BEAMER
\else

%% Theorems! %%
\usepackage{amsthm}
\theoremstyle{plain} % Theorems, lemmas, propositions etc.
\newtheorem{theorem}{Theorem}[section]
\newtheorem{lemma}[theorem]{Lemma}
\newtheorem{proposition}[theorem]{Proposition}
\newtheorem{corollary}[theorem]{Corollary}
\newtheorem{fact}[theorem]{Fact}
\newtheorem{construction}[theorem]{Construction}
\theoremstyle{definition} % Definitions etc.  
\newtheorem{definition}[theorem]{Definition}
\newtheorem{notation}[theorem]{Notation}
\theoremstyle{remark} % Remarks
\newtheorem{remark}[theorem]{Remark}
\newtheorem{remarks}[theorem]{Remarks}
\newtheorem{example}[theorem]{Example}
\newtheorem{question}[theorem]{Question}
\newtheorem{slogan}[theorem]{Slogan}

\newtheoremstyle{note} {3pt} {3pt} {\itshape} {} {\itshape} {:} {.5em} {} % For short notes
\theoremstyle{note}
\newtheorem{note}[theorem]{Note}

\fi

%% Exercises and answers %%
\usepackage{answers}

\newtheoremstyle{exercisestyle}% name
  {6pt}   % ABOVESPACE
  {6pt}   % BELOWSPACE
  {\itshape}  % BODYFONT
  {0pt}       % INDENT (empty value is the same as 0pt)
  {\bfseries} % HEADFONT
  {.}         % HEADPUNCT
  {3pt} % HEADSPACE
  {}          % CUSTOM-HEAD-SPEC

\theoremstyle{exercisestyle}
\newtheorem{exercise}{Exercise}
\newtheorem{answerthm}{Exercise}

\Newassociation{answer}{answerthm}{answers}
\newcommand{\answerthmparams}{}

%% Changes to enumerate things so they look better %%\sigma$

\makeatletter
\def\enumfix{%
\if@inlabel
 \noindent \par\nobreak\vskip-\topsep\hrule\@height\z@
\fi}

\let\olditemize\itemize
\def\itemize{\enumfix\olditemize}
\let\oldenumerate\enumerate
\def\enumerate{\enumfix\oldenumerate}

%% Random crap %%
\usepackage{xifthen}

\makeatletter
\def\thm@space@setup{%
  \thm@preskip=\parskip \thm@postskip=0pt
}
\makeatother

\makeatletter
\newcommand*{\relrelbarsep}{.386ex}
\newcommand*{\relrelbar}{%
  \mathrel{%
    \mathpalette\@relrelbar\relrelbarsep
  }%
}
\newcommand*{\@relrelbar}[2]{%
  \raise#2\hbox to 0pt{$\m@th#1\relbar$\hss}%
  \lower#2\hbox{$\m@th#1\relbar$}%
}
\providecommand*{\rightrightarrowsfill@}{%
  \arrowfill@\relrelbar\relrelbar\rightrightarrows
}
\providecommand*{\leftleftarrowsfill@}{%
  \arrowfill@\leftleftarrows\relrelbar\relrelbar
}
\providecommand*{\xrightrightarrows}[2][]{%
  \ext@arrow 0359\rightrightarrowsfill@{#1}{#2}%
}
\providecommand*{\xleftleftarrows}[2][]{%
  \ext@arrow 3095\leftleftarrowsfill@{#1}{#2}%
}
\makeatother

\newcommand{\catname}[1]{{\normalfont\textbf{#1}}}
\newcommand{\Rings}{\catname{CRing}}
\newcommand{\CAT}{\catname{CAT}}
%\newcommand{\Top}{\catname{Top}}
\newcommand{\Set}{\catname{Set}}
\newcommand{\Cat}{\catname{Cat}}
\newcommand{\MonCat}{\catname{MonCat}}
\newcommand{\SymmMonCat}{\catname{SymmMonCat}}
\newcommand{\Cont}{\catname{Cont}}
\newcommand{\Sch}{\catname{Sch}}
\newcommand{\Rel}{\catname{Rel}}
\newcommand{\Coh}{\catname{Coh}}
\newcommand{\Inj}{\catname{Inj}}
\newcommand{\Dcpo}{\catname{Dcpo}}
\newcommand{\Mod}[1][]{\ifthenelse{\isempty{#1}}{\catname{Mod}}{#1\catname{mod}}}
\DeclareMathOperator{\sh}{Sh}
\newcommand{\Sh}[1][]{\ifthenelse{\isempty{#1}}{\sh}{\sh(#1)}}
\newcommand{\map}[3]{#2\xrightarrow{#1} #3}
\newcommand*\from{\colon}
\newcommand*\bigto{\Rightarrow}
\newcommand{\cmap}[3]{#1\from{}#2\to{}#3}
\newcommand\oppcat[1]{#1^{\mathrm{op}}}
\newcommand{\object}{\colon}
\DeclareRobustCommand{\vmap}[3] {\begin{tikzcd} #2 \arrow[d, "#1"] \\ #3 \end{tikzcd}}
\newcommand{\partref}[1]{(\ref{#1})}
\newcommand{\intgrpd}[4] {#1 \xrightrightarrows[#3]{#4} #2}
\DeclareRobustCommand{\bigintgrpd}[4] {\begin{tikzcd}[ampersand replacement=\&] #1 \arrow[r, shift left=0.5ex, "#3"] \arrow[r, shift right=0.5ex, "#4"'] \& #2 \end{tikzcd}}

\usepackage{xspace}

\newcommand{\etale}{\'{e}tale\xspace}
\newcommand{\Etale}{\'{E}tale\xspace}

\def \inv {^{-1}}

\DeclareMathOperator{\id}{id}
\DeclareMathOperator{\op}{op}
\DeclareMathOperator{\pr}{pr}
\DeclareMathOperator{\inj}{in}
\DeclareMathOperator{\pre}{{pre}}
\DeclareMathOperator{\et}{{\acute{e}t}}

\DeclareMathOperator{\Hom}{Hom}
\DeclareMathOperator{\Spec}{Spec}

\DeclareMathOperator{\ol}{ol}

\def\presuper#1#2%
  {\mathop{}%
   \mathopen{\vphantom{#2}}^{#1}%
   \kern-\scriptspace%
   #2}
\def\presub#1#2%
  {\mathop{}%
   \mathopen{\vphantom{#2}}_{#1}%
   \kern-\scriptspace%
   #2}

\newsavebox{\overlongequation}
\newenvironment{longdiagram}
 {\begin{displaymath}\begin{lrbox}{\overlongequation}$\displaystyle}
 {$\end{lrbox}\makebox[0pt]{\usebox{\overlongequation}}\end{displaymath}}

%% Our things %%

\newcommand{\neggame}[1]{\presuper{\perp}{#1}}
\newcommand{\tensor}{\otimes}
\newcommand{\Tensor}{\bigotimes}
\newcommand{\sequoid}{\oslash}
\newcommand{\varsequoid}{\vartriangleleft}
\renewcommand{\implies}{\multimap}
\newcommand{\iimpl}{\Longrightarrow}
\newcommand{\comp}[2]{#1 \circ #2}
\newcommand{\icomp}[2]{\comp{#1}{#2}}
\newcommand{\cprd}{\sqcup}
\newcommand{\bigcprd}{\bigsqcup}
\newcommand{\G}{\mathcal G}
\newcommand{\W}{\mathcal W}
\newcommand{\suchthat}{\;\colon\;}
\newcommand{\varsuchthat}{\;\mid\;}
\newcommand{\esuchthat}{\;.\;}
\newcommand{\OP}{\{O,P\}}
\newcommand{\QA}{\{Q,A\}}
\renewcommand{\L}{\mathcal L}
\newcommand{\F}{\mathcal F}
\newcommand{\U}{\mathcal U}
\newcommand{\s}{\mathfrak s}
\renewcommand{\t}{\mathfrak t}
\renewcommand{\u}{\mathfrak u}
\renewcommand{\d}{\mathfrak d}
\newcommand{\e}{\mathfrak e}
\newcommand{\emptyplay}{\epsilon}
\newcommand{\bracketed}[1]{\left({#1}\right)}
\newcommand{\bneggame}[1]{{\bracketed{\neggame{#1}}}}
\newcommand{\prefix}{\sqsubseteq}
\newcommand{\ppprefix}{\sqsubset}
\newcommand{\pprefix}{\sqsubsetneqq}
\renewcommand{\ss}{\mathbf{s}}
\newcommand{\bN}{\mathbb{N}}
\newcommand{\bC}{\mathbb{C}}
\newcommand{\bB}{\mathbb{B}}
\newcommand{\bP}{\mathbb{P}}
\newcommand{\pfun}{\rightharpoonup}
\newcommand{\grel}[1]{\underline{#1}}
\DeclareMathOperator{\length}{length}
\renewcommand{\b}{\mathfrak b}
\renewcommand{\r}{\mathfrak r}
\newcommand{\bbeta}{{\bm{\beta}}}
\newcommand{\st}{{\Sigma^*}}
\let\sec\S
\renewcommand{\S}{{\mathfrak{S}}}
\DeclareMathOperator{\cc}{cc}
\DeclareMathOperator{\subs}{subs}
\DeclareMathOperator{\ret}{ret}
\DeclareMathOperator{\zz}{zz}
\newcommand{\aaa}{\mathbf{a}}
\newcommand{\bbb}{\mathbf{b}}
\newcommand{\ccc}{\mathbf{c}}
\newcommand{\ddd}{\mathbf{d}}
\newcommand{\B}{\mathcal B}
\newcommand{\BB}{\mathbf B}
\renewcommand{\H}{\mathcal H}
\DeclareMathOperator{\assoc}{assoc}
\DeclareMathOperator{\lunit}{lunit}
\DeclareMathOperator{\runit}{runit}
\DeclareMathOperator{\dom}{dom}
\DeclareMathOperator{\sym}{sym}
\newcommand{\braid}{\sym}
\newcommand{\blank}{\,\underline{\hspace{1.5ex}}\,}
\DeclareMathOperator{\cn}{cn}
\newcommand{\impliescn}{\protect\overset{\cn}{\implies}}
\newcommand{\C}{{\mathcal{C}}}
\newcommand{\D}{{\mathcal{D}}}
\newcommand{\E}{{\mathcal{E}}}
\newcommand{\V}{{\mathcal{V}}}
\newcommand{\EE}{{\mathbf{E}}}
\DeclareMathOperator{\ev}{ev}
\newcommand{\der}{{\mathtt{der}}}
\newcommand{\mult}{{\mathtt{mult}}}
\DeclareMathOperator{\wk}{wk}
\newcommand{\toisom}{{\xrightarrow{\cong}}}
\DeclareMathOperator{\passoc}{{\mathsf{passoc}}}
\DeclareMathOperator{\pcomm}{{\mathsf{pcomm}}}
\DeclareMathOperator{\run}{{\mathsf{r}}}
\DeclareMathOperator{\lun}{{\mathsf{l}}}
\newcommand{\fcoal}[1]{{\leftmoon #1 \rightmoon}}
\DeclareMathSymbol{\co}{\mathord}{operators}{"3C}
\DeclareMathSymbol{\nw}{\mathord}{operators}{"3E}
\newcommand{\T}{\mathfrak{T}}
\renewcommand{\subset}{\subseteq}
\newcommand{\Ord}{\catname{Ord}}
\newcommand{\FS}{\mathcal{FS}}
\DeclareMathOperator{\rank}{rank}
\DeclareMathOperator{\dist}{{\mathsf{dist}}}
\DeclareMathOperator{\dec}{{\mathsf{dec}}}
\DeclareMathOperator{\str}{str}
\DeclareMathOperator{\weak}{weak}
\DeclareMathOperator{\Strat}{Strat}
\DeclareMathOperator{\OppStrat}{OppStrat}
\newcommand{\seqs}[1]{{\overline{{#1}^{*}}}}
\def\flushRight{\leftskip0pt plus 1fill\rightskip0pt}
\def\Centering{\relax\ifvmode\centering\fi}
\newcommand{\deno}[1]{\left\llbracket#1\right\rrbracket}
\newcommand{\converges}{\Downarrow}
\newcommand{\diverges}{\Uparrow}
\newcommand{\mustconverge}{\converges^{\text{must}}}
\newcommand{\Iflt}{\mathtt{If{<}\;}}
\newcommand{\Ifgt}{\mathtt{If{>}\;}}
\newcommand{\inr}{{\mathsf{inr}}}
\newcommand{\inl}{{\mathsf{inl}}}
\newcommand{{\Na}}{\bN}
\newcommand{{\cell}}{{\mathsf{cell}}}
\newcommand{\fix}{{\mathsf{fix}}}
\newcommand{\eq}{{\mathsf{eq}}}
\DeclareMathOperator{\CCom}{CCom}
\newcommand{\power}{\mathfrak P}

% Slanty things
\newcommand*{\xslant}[2][76]{%
  \begingroup
    \sbox0{#2}%
    \pgfmathsetlengthmacro\wdslant{\the\wd0 + cos(#1)*\the\wd0}%
    \leavevmode
    \hbox to \wdslant{\hss
      \tikz[
        baseline=(X.base),
        inner sep=0pt,
        transform canvas={xslant=cos(#1)},
      ] \node (X) {\usebox0};%
      \hss
      \vrule width 0pt height\ht0 depth\dp0 %
    }%
  \endgroup
}

\makeatletter
\newcommand*{\xslantmath}{}
\def\xslantmath#1#{%
  \@xslantmath{#1}%
}
\newcommand*{\@xslantmath}[2]{%
  % #1: optional argument for \xslant including brackets
  % #2: math symbol
  \ensuremath{%
    \mathpalette{\@@xslantmath{#1}}{#2}%
  }%
}
\newcommand*{\@@xslantmath}[3]{%
  % #1: optional argument for \xslant including brackets
  % #2: math style
  % #3: math symbol
  \xslant#1{$#2#3\m@th$}%
}
\makeatother

\newcommand{\seqdeno}[1]{\xslantmath{\llbracket}#1\xslantmath{\rrbracket}}

% Empty set etc.

\let\oldemptyset\emptyset
\let\emptyset\varnothing

%% Constant width xrightarrows
\newlength{\arrow}
\settowidth{\arrow}{\scriptsize$1000$}
\newcommand*{\constantwidthxrightarrow}[1]{\xrightarrow{\mathmakebox[\arrow]{#1}}}

%% Landscape pages
\usepackage{everypage}
\usepackage{environ}
\usepackage{pdflscape}
\newcounter{abspage}

\ifdefined\NOTARTICLE

\else

\makeatletter
\newcommand{\newSFPage}[1]% #1 = \theabspage
  {\global\expandafter\let\csname SFPage@#1\endcsname\null}

\NewEnviron{SidewaysFigure}{\begin{figure}[p]
\protected@write\@auxout{\let\theabspage=\relax}% delays expansion until shipout
  {\string\newSFPage{\theabspage}}%
\ifdim\textwidth=\textheight
  \rotatebox{90}{\parbox[c][\textwidth][c]{\linewidth}{\BODY}}%
\else
  \rotatebox{90}{\parbox[c][\textwidth][c]{\textheight}{\BODY}}%
\fi
\end{figure}}

\AddEverypageHook{% check if sideways figure on this page
  \ifdim\textwidth=\textheight
    \stepcounter{abspage}% already in landscape
  \else
    \@ifundefined{SFPage@\theabspage}{}{\global\pdfpageattr{/Rotate 0}}%
    \stepcounter{abspage}%
    \@ifundefined{SFPage@\theabspage}{}{\global\pdfpageattr{/Rotate 90}}%
  \fi}
\makeatother

\fi

%% PCF Things

\newcommand{\nat}{{\mathtt{nat}}}
\newcommand{\bool}{{\mathtt{bool}}}

\newcommand{\Y}{\mathbf{Y}}
\newcommand{\opto}{\longrightarrow}
\newcommand{\oopto}{\dashrightarrow}
\newcommand{\n}{{\mathtt{n}}}
\DeclareMathOperator{\IfO}{{\mathsf{If0}}}
\DeclareMathOperator{\suc}{{\mathsf{succ}}}
\DeclareMathOperator{\pred}{{\mathsf{pred}}}
\newcommand{\0}{{\mathtt{0}}}

\newcommand{\iter}{{\mathtt{iter}}}
\newcommand{\rec}{\iter}
\newcommand{\Var}{{\mathtt{Var}}}
\DeclareMathOperator{\Varr}{Var}
\newcommand{\new}{{\mathtt{new}}}
\newcommand{\case}{{\mathtt{case}}}

\newcommand{\lmam}{\mathrel{\sqsubseteq_{m\&m}}}
\newcommand{\emam}{\mathrel{\equiv_{m\&m}}}
\newcommand{\lst}{\mathrel{\lesssim}}
\newcommand{\smam}{\mathrel{\sim_{m\&m}}}
\newcommand{\amam}{\mathrel{\approx_{m\&m}}}

\newcommand{\oes}{\sim}

%% Idealized Algol things

\newcommand{\com}{{\mathtt{com}}}
\newcommand{\skipp}{{\mathsf{skip}}}
\DeclareMathOperator{\seq}{{\mathsf{seq}}}
\DeclareMathOperator{\neww}{{\mathsf{new}}}
\DeclareMathOperator{\mkvar}{{\mathsf{mkvar}}}
\newcommand{\deref}{\texttt{@}}
\DeclareMathOperator{\dereff}{\mathsf{deref}}
\DeclareMathOperator{\assign}{\mathsf{assign}}
\newcommand{\ia}[2]{\langle #1 , #2 \rangle}
\newcommand{\stup}[3]{\langle #1 \mid #2 \mapsto #3 \rangle}

%% Hyland-Ong games things

\newbox\gnBoxA
\newdimen\gnCornerHgt
\setbox\gnBoxA=\hbox{$\ulcorner$}
\global\gnCornerHgt=\ht\gnBoxA
\newdimen\gnArgHgt
\def\pv #1{%
    \setbox\gnBoxA=\hbox{$#1$}%
    \gnArgHgt=\ht\gnBoxA%
    \ifnum     \gnArgHgt<\gnCornerHgt \gnArgHgt=0pt%
    \else \advance \gnArgHgt by -\gnCornerHgt%
    \fi \raise\gnArgHgt\hbox{$\ulcorner$} \box\gnBoxA %
    \raise\gnArgHgt\hbox{$\urcorner$}}
\def\ov #1{%
    \setbox\gnBoxA=\hbox{$#1$}%
    \gnArgHgt=\ht\gnBoxA%
    \ifnum     \gnArgHgt<\gnCornerHgt \gnArgHgt=0pt%
    \else \advance \gnArgHgt by -\gnCornerHgt%
    \fi \raise\gnArgHgt\hbox{$\llcorner$} \box\gnBoxA %
    \raise\gnArgHgt\hbox{$\lrcorner$}}
\newcommand{\ct}[1]{\lceil#1\rceil}
\DeclareMathOperator{\Int}{int}

%% Nondeterministic Factorization things

\newcommand{\code}{\mathsf{code}}
\newcommand{\Det}{\mathsf{Det}}

%% Flexible strategy things

\newcommand{\stle}{{\;\le_s\;}}
\newcommand{\steq}{{\;=_s\;}}
\newcommand{\exle}{\sqsubseteq}
\newcommand{\exlub}{\bigsqcup}
\newcommand{\dv}{{\text{\lightning}}}
\DeclareMathOperator{\pocl}{pocl}
\newcommand{\plot}{\mathrel{\triangleleft}}
\newcommand{\shad}{\mathfrak{S}}
%\newcommand{\tree}{\mathfrak{T}}
\newcommand{\Tau}{T}
\newcommand{\Epsilon}{E}
\newcommand{\sw}{\triangleleft}

%% Roman numerals

\newcommand{\RN}[1]{%
  \textup{\uppercase\expandafter{\romannumeral#1}}%
}
\newcommand{\RNl}[1]{%
  \mathrel{\raisebox{1pt}{$\overline{\underline{#1}}$}}
}

%% Game language things

\newcommand{\ul}[1]{{\underline{#1}}}
\newcommand{\A}{{\mathcal{A}}}
\renewcommand{\P}{\mathcal P}
\newcommand{\M}{\mathcal M}
\newcommand{\N}{\mathcal N}
\newcommand{\X}{\mathcal X}
\newcommand{\YY}{\mathcal Y}
\newcommand{\hole}{\blank}
\newcommand{\Tct}{\xrightarrow{T}t}
\newcommand{\teamconverge}[2]{\xrightarrow{#1}#2}

%% Inference rule things
\newcommand{\rulename}[1]{\LeftTirNameStyle{#1}}
\newcommand{\ts}{\mathbin{\vdash}}
\newcommand{\nts}{\mathbin{\not\vdash}}

%% Double category things
\newcommand{\hc}[2]{\left({#1}\middle|{#2}\right)}
\newcommand{\vc}[2]{\left(\frac{#1}{#2}\right)}

%% What is going on?
\DeclareMathOperator{\Kl}{Kl}
\DeclareMathOperator{\Mell}{Mell}
\newcommand{\powerset}{\mathcal P}
\DeclareMathOperator{\ask}{{\mathsf{ask}}}
\newcommand{\sleep}{{\mathsf{sleep}}}
\newcommand{\true}{\mathbbm{t}}
\newcommand{\false}{\mathbbm{f}}
\DeclareMathOperator{\If}{\mathsf{If}}
\newcommand{\Then}{\mathrel{\mathsf{then}}}
\newcommand{\Else}{\mathrel{\mathsf{else}}}
\newcommand\cat{\mathbin{+\mkern-10mu+}}

%% Profunctor arrows

\makeatletter
\def\slashedarrowfill@#1#2#3#4#5{%
  $\m@th\thickmuskip0mu\medmuskip\thickmuskip\thinmuskip\thickmuskip
   \relax#5#1\mkern-7mu%
   \cleaders\hbox{$#5\mkern-2mu#2\mkern-2mu$}\hfill
   \mathclap{#3}\mathclap{#2}%
   \cleaders\hbox{$#5\mkern-2mu#2\mkern-2mu$}\hfill
   \mkern-7mu#4$%
}
\def\rightslashedarrowfill@{%
  \slashedarrowfill@\relbar\relbar\mapstochar\rightarrow}
\newcommand\xslashedrightarrow[2][]{%
  \ext@arrow 0055{\rightslashedarrowfill@}{#1}{#2}}
\makeatother
\newcommand{\pto}{{\xslashedrightarrow{} }}

%% Profunctors 
\DeclareMathOperator{\Prof}{Prof}
\DeclareMathOperator{\End}{End}
\DeclareMathOperator{\Endoprof}{Endoprof}

%% Our

\def\searchmacro#1{
  \AtBeginOfFiles{%
    \ifdefined#1
      \expandafter\def\csname \currfilename:found\endcsname{}%
    \fi}
  \AtEndOfFiles{%
    \ifdefined#1
      \unless\ifcsname \currfilename:found\endcsname
        \immediate\write\finder{found in '\currfilename'}%
    \fi\fi}}

%% Isomorphism arrows on commutative diagrams %%
\tikzset{Isom/.style={every to/.append style={edge node={node [sloped, above, allow upside down, auto=false]{$\cong$}}}},
         Isom'/.style={every to/.append style={edge node={node [sloped, above, allow upside down, auto=false, rotate=180]{$\cong$}}}},
         Sim/.style={every to/.append style={edge node={node [sloped, above, allow upside down, auto=false]{$\sim$}}}},
         Sim'/.style={every to/.append style={edge node={node [sloped, above, allow upside down, auto=false, rotate=180]{$\sim$}}}}}

%% Adjunctions
\newcommand{\adjunction}[4]{%
  {#1} \underset{\underset{#3}{\longleftarrow}}{\overset{\overset{#2}{\longrightarrow}}{\bot}} {#4}}        

%% Important!
\newcommand\Mellies{Melli\`{e}s\xspace}

\makeatletter
\newcommand{\colim@}[2]{%
  \vtop{\m@th\ialign{##\cr
    \hfil$#1\operator@font colim$\hfil\cr
    \noalign{\nointerlineskip\kern1.5\ex@}#2\cr
    \noalign{\nointerlineskip\kern-\ex@}\cr}}%
}
\newcommand{\colim}{%
  \mathop{\mathpalette\colim@{\rightarrowfill@\textstyle}}\nmlimits@
}
\makeatother

\makeatletter
\newcommand{\laxcolim@}[2]{%
  \vtop{\m@th\ialign{##\cr
    \hfil$#1\operator@font colim_l$\hfil\cr
    \noalign{\nointerlineskip\kern1.5\ex@}#2\cr
    \noalign{\nointerlineskip\kern-\ex@}\cr}}%
}
\newcommand{\laxcolim}{%
  \mathop{\mathpalette\laxcolim@{\rightarrowfill@\textstyle}}\nmlimits@
}
\makeatother

\DeclareMathOperator{\Colim}{colim}

\DeclareMathOperator{\DG}{DG}
\DeclareMathOperator{\RV}{RV}
\newcommand{\Rv}{\catname{Rv}}

\let\choose\undefined
\DeclareMathOperator{\choose}{\mathsf{choose}}
\DeclareMathOperator{\tr}{tr}
\DeclareMathOperator{\test}{test}

%% Slot game things %%
\newcommand{\circled}[1]{\raisebox{.5pt}{\textcircled{\raisebox{-.9pt} {#1}}}}
\newcommand{\slot}{{\circled{\$}}}

\DeclareMathOperator{\may}{may}
\DeclareMathOperator{\must}{must}

\newcommand{\encode}[1]{\lceil#1\rceil}
\DeclareMathOperator{\app}{\mathsf{app}}
\DeclareMathOperator{\lett}{\mathsf{let}}
\newcommand{\inn}{\mathrel{\mathsf{in}}}
\DeclareMathOperator{\byval}{\mathsf{byval}}

\DeclareMathOperator{\rread}{read}
\DeclareMathOperator{\wwrite}{write}

\DeclareSymbolFont{bbsymbol}{U}{bbold}{m}{n}
\DeclareMathSymbol{\bbsemicolon}{\mathbin}{bbsymbol}{"3B}
\newcommand{\semicom}{\bbsemicolon}

\newcommand{\ms}{\makebox[-1pt]{}}

\DeclareMathOperator{\Acc}{Acc}
\DeclareMathOperator{\im}{Im}
\DeclareMathOperator{\wit}{wit}

%%% END Article customizations



\begin{document}

\section{Monads and Kleisli categories}

\subsection{Monads}

Let $\C$ be a category.  
Then the category $[\C,\C]$ of functors $\C\to\C$ and natural transformations has a (strict) monoidal structure given by composition.  
A \emph{monad} \cite[\sec VI]{WorkingMathematician} in $\C$ is a monoid in $[\C,\C]$.

In other words, it is a functor $M\from \C\to\C$ together with natural  transformations $m_a \from MMa \to Ma$ and $u_a \from a \to Ma$ such that the following diagrams commute for all objects $a$ of $\C$.

\begin{mathpar}
  \begin{tikzcd}
    MMM a \arrow[r, "Mm_a"] \arrow[d, "m_{Ma}"]
      & MM a \arrow[d, "m_a"] \\
    MM a \arrow[r, "m_a"]
      & M a
  \end{tikzcd}
  \and
  \begin{tikzcd}
    M a \arrow[r, "M u_a"] \arrow[dr, "id"']
      & M M a \arrow[d, "m_a"] \\
    %
      & M a
  \end{tikzcd}
  \and
  \begin{tikzcd}
    M a \arrow[r, "u_{M a}"] \arrow[dr, "id"']
      & M M a \arrow[d, "m_a"] \\
    %
      & M a
  \end{tikzcd}
\end{mathpar}

\begin{example}
  In the category of sets, the \emph{nonempty powerset functor} $\powerset_+$ sends a set $A$ to the set of nonempty subsets of $A$.  
  This has the structure of a monad on $\Set$, since we have a natural transformation (union) from $\powerset_+\powerset_+A \to \powerset_+A$ and a natural transformation (singleton) from $A \to \powerset_+A$ that obey the diagrams given above.
\end{example}
\begin{example}
  Let $\M$ be a monoidal category and let $x$ be a monoid in $\M$.  
  The \emph{writer monad} $W_x$ on $\M$ is defined by $W_x y = y \tensor x$, with natural transformations
  \begin{mathpar}
    m_y \from y \tensor x \tensor x \to y \tensor x
    \and
    u_y \from y \to y \tensor x
  \end{mathpar}
  given by the monoid structure on $x$.

  Going the other way, if $\M$ is monoidal closed with inner hom $\implies$, and if $z$ is a comonoid in $\M$, then the \emph{reader monad} $R_z$ is given by $R_z y = z \implies y$.  
  Then the monadic coherences
  \begin{mathpar}
    m_y \from z \implies z \implies y \to z \implies y
    \and
    u_y \from y \to z \implies y
  \end{mathpar}
  are induced from the comonoid structure on $z$.
  This second example is particularly important in Cartesian closed categories, in which every object has the structure of a comonoid.
\end{example}
\begin{example}
  If $\adjunction{\C}{L}{R}{\D}$ is an adjunction with counit $\epsilon\from LR\to 1$ and unit $\eta\from 1 \to RL$, then the composite $RL\from \C \to \C$ has the structure of a monoid on $\C$, where the multiplication and unit are given by
  \begin{mathpar}
    R\epsilon L \from RLRL \to RL
    \and
    \eta \from 1 \to RL\,.
  \end{mathpar}
  We will see in the next section that every monad is induced by an adjunction in this way.

  As an example, if $\M$ is a monoidal closed category and $w$ is an object of $\M$, then the \emph{state monad} $S_w$ on $\M$ is defined by
  \[
    S_w x = w \implies (x \tensor w)\,.
    \]
\end{example}
\begin{example}
  Another example that arises from an adjunction is the \emph{list monad} on $\Set$ that arises from the adjunction between the category of sets and the category of (set-valued) monoids.  
  The underlying set of the free monoid on a set $A$ is the set $A^*$ of finite lists of elements of $A$, and the functor $A\mapsto A^*$ inherits a monoid structure where the multiplication $m_A\from (A^*)^* \to A^*$ concatenates a list of lists into a single list and the unit $u_a \from A \to A^*$ forms a list with a single element.  
\end{example}
\begin{example}
  A monad on $\oppcat\C$ is called a \emph{comonad} on $\C$.  
  The carrier of a comonad is still a functor $M\from \C\to\C$, but now the multiplication and unit are natural transformations $M\Rightarrow MM$ and $M\Rightarrow 1$, rather than the other way round.  

  An adjunction $\adjunction\C LR\D$ gives rise to a comonad structure on $LR$ in much the same way as it gives rise to a monad structure on $RL$.  
  So, for example, we have the \emph{store comonad} $S_r'$ for any object $r$ of a monoidal closed category $\M$, given by
  \[
    S_r'x = (r \implies x) \tensor x\,.
    \]
\end{example}

\subsection{Kleisli Categories}

Let $\C$ be a category and let $M$ be a monad on $\C$.  
Then \cite{Kleisli} there is a category $\Kl_M$, called the \emph{Kleisli category} of $M$, whose objects are the objects of $\C$ and where a morphism from an object $a$ to an object $b$ is a morphism $a \to Mb$ in $\C$.

Identity arrows are given by the morphisms $u_c\from c \to M c$ (considered as a morphism $c\to c$ in $\Kl_M$) and the composition of arrows $f\from a \to Mb$ and $g \from b \to Mc$ is given by the following composite in $\C$.
\[
  a \xrightarrow{f}
  Mb \xrightarrow{Mg}
  MMc \xrightarrow{m_c}
  M c
  \]
There is a natural identity-on-objects functor $J\from \C \to \Kl_M$ that sends a morphism $f\from a \to b$ in $\C$ to the composite
\[
  a \xrightarrow{f}
  b \xrightarrow{u_b}
  M b\,,
  \]
considered as a morphism $a\to b$ in $\Kl_M$.

In the other direction, we have a functor $ S \from \Kl_M \to \C$ that sends an object $a$ of $\Kl_M$ to the object $Ma$ of $\C$ and sends a morphism $f\from a \to M b$ from $a$ to $b$ in $\Kl_M$ to the composite
\[
  Ma \xrightarrow{M f}
  MMb \xrightarrow{m_b}
  Mb
  \]
in $\C$.  
Note that $ S J=M$, by one of our coherence conditions on $m$ and $u$.
Meanwhile, $J S $ is the functor $\Kl_M\to\Kl_M$ that sends an object $a$ to $Ma$ and sends a morphism $f\from a \to Mb$ from $a$ to $b$ to the morphism $Mf\from Ma \to MMb$ from $Ma$ to $Mb$.
\begin{proposition}[\cite{Kleisli}]
  $ S $ is a right adjoint to $J$.
  The unit of the adjunction is $u\from \id \Rightarrow M$.  
  The counit $e_a\from J( S  a) \to a$ is given by the identity morphism $Ma \to Ma$ in $\C$, considered as a morphism $Ma \to a$ in $\Kl_M$.
  \label{prop:KleisliHasAdjunction}
\end{proposition}

Given a monad $M$ on a category $\C$ and a functor $F\from \C \to \D$, where $\D$ is another category, we say that a natural transormation $\psi_a \from FMa \to Fa$ is \emph{$M$-multiplicative} if it makes the following diagrams commute.
\begin{mathpar}
  \begin{tikzcd}
    FMMa \arrow[r, "\psi_{Ma}"] \arrow[d, "Fm_a"']
      & FMa \arrow[d, "\psi_a"] \\
    FMa \arrow[r, "\psi_a"]
      & Fa
  \end{tikzcd}
  \and
  \begin{tikzcd}
    Fa \arrow[r, "F u_a"] \arrow[dr, "\id"']
      & FMa \arrow[d, "\psi_a"] \\
    %
      & Fa
  \end{tikzcd}
\end{mathpar}

Given two triples $(\D, F,\psi), (\D', F',\psi')$, where $F\from \C \to \D, F'\from \C'\to\D'$ are functors and $\psi\from FM\Rightarrow F, \psi'\from F'M\Rightarrow F'$ are functors, we define a \emph{morphism} from $(\D', F',\psi')$ to $(\D, F, \psi)$ to be a functor $H\from \D' \to \D$ such that $F=HF'$ and $\psi=H\psi'$.  
This gives us a category.

A defining property of the Kleisli category is that it is initial among such triples $(\D,F,\psi)$:

\begin{proposition}[\cite{StreetMonads}]
  i) Given an object $a$ of $\C$, the identity morphism $Ma \to Ma$ may be considered as a morphism $\phi_a \from JMa \to Ja$ in $\Kl_M$.  
  $\phi_a$ is an $M$-multiplicative natural transformation.

  ii) Let $\D$ be a category, let $F\from \C \to \D$ be a functor and suppose that $\psi_a\from FMa \to Ma$ is an $M$-multiplicative natural transformation.
  Then there is a unique functor $\hat{F}\from \Kl_M \to \D$ such that $F=\hat{F}J$ and $\psi = \hat{F}\phi$.
  \label{pKleisli}
\end{proposition}

Another way to characterize the Kleisli category $\Kl_M$ is to say that the the adjunction we described above is initial among all adjunctions giving rise to the monad $M$.  
This can be deduced from Proposition \ref{pKleisli} using the following result.

\begin{lemma}[\cite{StreetMonads}]
  Let $\C$ be a category and let $M$ be a monad on $\C$.  
  If $\adjunction{\C}{L}{R}{\D}$ is an adjunction (with counit $\epsilon$ and unit $\eta$), we say it \emph{gives rise to $M$} if $M=RL$, $m=R\epsilon L$ and $u=\eta$.

  Any such adjunction gives rise to an $M$-multiplicative natural transformation $\psi\from LM \Rightarrow L$.  
  This gives us a fully faithful functor from the category of adjunctions giving rise to $M$ to the category of triples $(\D,F,\psi)$ where $\psi$ is $M$-multiplicative.
\end{lemma}

The proof of Proposition \ref{pKleisli} essentially comes down to the following factorization result.  
If $f\from a \to b$ is a morphism in $\Kl_M$, then $f$ may be factorized as
\[
  f = a \xrightarrow{Jf}
  Mb \xrightarrow{\phi_b}
  b\,,
  \]
where we use `$f$' to refer both to the morphism $a\to b$ in $\Kl_M$ and to the underlying morphism $a \to Mb$ in $\C$.
Indeed, if we compute this composite inside $\C$, we get
\[
  a \xrightarrow{f}
  Mb \xrightarrow{u_{Mb}}
  MMb \xrightarrow{M\id}
  MMb \xrightarrow{m_b}
  Mb\,,
  \]
which is equal to $f$ by the coherence conditions on $m$ and $u$.
This means that the Kleisli category may be thought of as begin freely generated from the original category $\C$ and a multiplicative natural transformation $\phi$.

\begin{example}
  The morphisms in the Kleisli category for the nonempty powerset monad $\powerset_+$ on $\Set$ are functions $A \to \powerset_+B$, which can be thought of as nondeterministic functions.  
  Given a set $A$, the morphism $\phi_A\from \powerset_+A \to A$ in $\Kl_{\powerset_+}$ can be interpreted as a `nondeterministic choice' function that accepts a nonempty set of elements of $A$ and nondeterministically chooses one of them.
  The factorization then means that the category is freely generated over $\C$ by these nondeterministic choice morphisms.
\end{example}
\begin{example}
  Let $\C$ be a Cartesian closed category and let $z$ be some fixed object of $\C$.  
  Then the Kleisli category for the reader monad $R_z$ on $\C$ is generated over $\C$ by a natural transformation $\phi_y\from (z \to y) \to y$.  
  By the enriched Yoneda lemma, such a natural transformation is always given by precomposition with some fixed morphism $\ask\from 1 \to z$.  
  This means that $\Kl_{R_z}$ is suitable for modelling any situation in which we are generally working in $\C$, but need the ability to request a value of type $z$ (for example, a config file, a piece of user input or something else that isn't being passed into the function in question).
  \label{ExReaderMonadKleisli}
\end{example}

A particularly important fact about the reader monad in Cartesian closed categories is the following.

\begin{theorem}[\cite{FunctionalCompleteness}]
  Let $\C$ be a Cartesian closed category and let $z$ be an object of $\C$.  
  Then the Kleisli category $\Kl_{R_z}$ for the reader monad over $z$ on $\C$ is Cartesian closed.
  \label{FunctionalCompletenessCcc}
\end{theorem}

The \emph{functional completeness} theorem \cite{FunctionalCompleteness} can be thought of as a special case of our remarks above.

\subsection{Extended example: game semantics of time complexity}
This is an extended example of using a reader monad, in the style of Example \ref{ExReaderMonadKleisli}.
Let $\G$ be the category of games and visible single-threaded strategies as in \cite{SamsonGuyIAActive}.  
So $\G$ is a sound and adequate model of Idealized Algol satisfying compact definability.
Let $\bC$ be the denotation of the command type $\com$ of Idealized Algol.  
Then the Kleisli category for the reader monad over $\bC$ is freely generated over $\G$ by a single morphism $1 \to \bC$.  

Let us extend Idealized Algol with a single term, $\sleep$, of type $\com$.
This should be thought of as a command to the computer to pause for some fixed interval of time -- let us say a second -- before resuming execution.
By Theorem \ref{FunctionalCompletenessCcc}, our Kleisli category is Cartesian closed.
So it gives us a denotational semantics of IA+$\sleep$, in which the terms of IA are interpreted within $\G$ as in \cite{SamsonGuyIAActive} and the new term $\sleep$ is interpreted by the new morphism $1 \to \bC$.

Let us now give an operational semantics to terms of IA+$\sleep$.  
Let $\Gamma$ be an IA-context.
Then a \emph{store} $s$ over $\Gamma$ is a list of pairs $(v\mapsto x)$, where each $v$ is the name for a $\Var$-type variable occurring in $\Gamma$ and $x$ is a value held in $v$.  
We write $()$ for the empty store, $(s\vert v\mapsto x)$ for the store obtained from $s$ by adding or updating the variable $v$ to hold the value $x$ and $(s\vert v\mapsto \bot)$ for the store obtained by removing the variable $v$ from $s$.
Given disjoint stores $s,t$ (i.e., stores whose sets of variables are disjoint), we write $s+t$ for the store obtained by concatenating together $s$ and $t$.  

Given a term $\Gamma\ts M$ in context, stores $s$,$s'$ over $\Gamma$, a value $c$ and a natural number $n$, we define a big-step relation
\[
  \Gamma,s\ts M\converges_n c,s'\,,
  \]
read as `in the context $\Gamma$, $s,M$ converges to $c,s'$ in time $n$'.  
The rule for the constant $\sleep$ is given by
\[
  \inferrule*
  { }
  {\Gamma,s\ts \sleep\converges_1 \skipp,s}\,,
  \]
indicating that execution of the $\sleep$ command takes up an additional time step.  
The other rules are the same as those given in \cite{SamsonGuyIAActive} for the $\converges$ relation, except that we have to modify them in order to take account of the new information we have added to the reduction relation (i.e., the time taken to converge).
For example:
\begin{mathpar}
  \inferrule*
  { }
  {\Gamma,s\ts c \converges_0 c,s}
  \and
  \inferrule*
  {\Gamma,s\ts M\converges_m\skipp,s'\\
  \Gamma,s'\ts N\converges_n\skipp,s''}
  {\Gamma,s\ts M;N\converges_{m+n}\skipp,s''}\,.
\end{mathpar}
If $P$ is a program (i.e., a term of type $\com$) then we write $P\converges_n$ as a shorthand for $,()\ts P \converges_n \skipp,()$.
We say that $P$ \emph{diverges} and write $P\diverges$ if there is no $n$ such that $P\converges_n$.

We now want a way to capture this operational behaviour within the denotational semantics.
First, we note the Computational Adequacy result proved for the game semantics of Idealized Algol.

\begin{proposition}[{\cite[21]{SamsonGuyIAActive}}]
  Let $P$ be a term of type $\com$.  
  Then $P\converges$ if and only if $\deno{P}\ne\bot$.
  \label{SamsonGuyAdequacy}
\end{proposition}

Readers unfamiliar with game semantics should not worry too much about what $\bot$ is; the point of Proposition \ref{SamsonGuyAdequacy} is that the denotation of a command-type term precisely captures the operational behaviourof that term, in the sense that there is some distinguished strategy $\bot\from 1 \to \bC$ that is the denotation of every divergent program, and no convergent one.

Before proceeding further, we prove an important lemma.  
This lemma tells us that the operational semantics of a term of IA+$\sleep$ may be modelled within Idealized Algol; i.e., that IA+$\sleep$ is a conservative extension of the original language.

\begin{lemma}
  Let $P\from T$ be a term of IA+$\sleep$.
  Suppose that
  \[
    \Gamma,s\ts P\converges_n c,s'
    \]
  in IA+$\sleep$.


  Then for all $k\ge 0$, all contexts $\Delta$ disjoint from $\Gamma$, all stores $t$ over $\Gamma,\Delta$ disjoint from $s$ and all variables $v$ not in $\Gamma,\Delta$:
  \[
    \Delta,\Gamma,v,(t{+}s\vert v \mapsto k)\ts P[v \gets \suc !v/\sleep] \converges c,(t{+}s\vert v \mapsto k+n)
    \]
  in Idealized Algol.
  \label{IaSleepsSoundnessLemma}
\end{lemma}
\begin{proof}
  Induction on the derivation that $\Gamma,s\ts P\converges_n c,s'$.
  There are several possibilities for the last step in the derivation.
  \begin{description}
    \item[Structural congruence]
      Suppose that the last step in the derivation is
      \[
        \inferrule*{P\equiv P'\\
        \Gamma,s\ts P' \converges_n \skipp, s'}
        {\Gamma,s\ts P\converges_n \skipp, s'}\,.
        \]
      It is clear then by the definition of structural congruence that $P'[v\gets\suc !v/\sleep]\equiv P[v\gets\suc !v/\sleep]$.  
      So the claim follows by the inductive hypothesis applied to $P'$ and the structural congruence rule for Idealized Algol.
    \item[$\sleep$]
      Suppose that the last step in the derivation is
      \[
        \inferrule*{ }
        {\Gamma,s\ts \sleep \converges_1 \skipp,s}\,.
        \]
      Fix a context $\Delta$ disjoint from $\Gamma$ and a store $t$ over $\Gamma,\Delta$ that is disjoint from $s$.  
      Fix $k\ge n$.  
      Then it suffices to show that the following judgement may be proved within the operational semantics of Idealized Algol.
      \[
        \Delta,\Gamma,v,(t{+}s\vert v\mapsto k)\ts v \gets \suc !v \converges \skipp, (t{+}s\vert v\mapsto k+1)
        \]
      A derivation for this obvious fact is given in Figure \ref{SleepIADerivation}.
      \begin{SidewaysFigure}
        \centering
        \begin{subfigure}{\textheight}
          \[
            \inferrule*
            {
              \inferrule*
              {
                \inferrule*
                {
                  \inferrule*
                  {
                  }
                  {
                    \Delta,\Gamma,v,(t{+}s\vert v\mapsto k)\ts v \converges v,(t{+}s\vert v\mapsto k)
                  }
                }
                {
                  \Delta,\Gamma,v,(t{+}s\vert v\mapsto k) \ts !v \converges k,(t{+}s\vert v\mapsto k)
                }
              }
              {
                \Delta,\Gamma,v,(t{+}s\vert v\mapsto k)\ts \suc !v \converges k+1,(t{+}s\vert v\mapsto k)
              } \\
              \inferrule*
              {
              }
              {
                \Delta,\Gamma,v,(t{+}s\vert v\mapsto k)\ts v \converges v,(t{+}s\vert v\mapsto k)
              }
            }
            {
              \Delta,\Gamma,v,(t{+}s\vert v\mapsto k)\ts v\gets\suc !v\converges\skipp,(t{+}s\vert v\mapsto k+1)
            }
            \]
          \caption{The crucial part of Lemma \ref{IaSleepsSoundnessLemma} pertaining to the $\sleep$ constant is a routine Idealized Algol derivation.}
          \label{SleepIADerivation}
        \end{subfigure}
        \par\vspace{48pt}
        \begin{subfigure}{\textheight}
          \[
            \inferrule*{\Delta,\Gamma_1,v,(t{+}s\vert v\mapsto k) \ts M_1[v \gets \suc !v/\sleep] \converges c_1,(t{+}s\vert v \mapsto k + n_1) \\
            \cdots \\
            \Delta,\Gamma_p,v,(t{+}s\vert v\mapsto k + n_1 + \cdots + n_{p-1}) \ts M_p[v \gets \suc !v/\sleep] \converges c_1,(t{+}s\vert v \mapsto k+n_1+\cdots+n_p)}
            {\Delta,\Gamma,v,(t{+}s^{(0)}\vert v \mapsto k + n_1 + \cdots + n_p)\ts M[v\gets \suc !v/\sleep] \converges c,(t{+}s^{(p)}\vert v \mapsto k)}
            \]
          \caption{The big-step rules of IA+$\sleep$ give rise to big-step rules of Idealized Algol via our translation.}
          \label{GeneralIADerivation}
        \end{subfigure}
        \caption{The proof of Lemma \ref{IaSleepsSoundnessLemma} works by translating big-step derivations from IA+$\sleep$ into big-step derivations of Idealized Algol.}
        \label{IADerivations}
      \end{SidewaysFigure}
    \item[Canonical forms]
      Suppose that the last step in the derivation is
      \[
        \inferrule*{ }
        {\Gamma,s\ts_0 c\converges c,s}\,.
        \]
      Then, since $\sleep$ is not a canonical form, we have $c[v\gets \suc !v/\sleep]=c$.  
      Then we have the derivation
      \[
        \inferrule*{ }
        {\Delta,\Gamma,(t{+}s\vert v\mapsto k)\ts c \converges c,(t{+}s\vert v\mapsto k)}\,.
        \]
    \item[Sequencing]
      Suppose instead that the last step in the derivation is
      \[
        \inferrule*{\Gamma,s\ts M \converges_m \skipp, s'\\
        \Gamma,s'\ts N \converges_n \skipp, s''}
        {\Gamma,s\ts M;N \converges_{m+n} \skipp, s''}\,.
        \]
      Fix $k\ge 0$.  
      By induction on $M,N$, we have
      \begin{mathpar}
        \Delta,\Gamma,v,(t{+}s\vert v\mapsto k)\ts M[v\gets\pred !v/\sleep]\converges \skipp,(t{+}s'\vert  v\mapsto k+m)\,;
        \and
        \Delta,\Gamma,v,(t{+}s'\vert  v\mapsto k+m) \hspace{144pt} \par\vspace{-8pt} \\ \par \hspace{144pt} \ts N[v\gets\pred !v/\sleep]\converges \skipp,(t{+}s''\vert v\mapsto k+m+n)\,.
      \end{mathpar}
      It follows that
      \begin{mathpar}
        \Delta,\Gamma,v,(t{+}s\vert v\mapsto k) \hspace{180pt} \par\vspace{-8pt} \\ \par \hspace{180pt} \ts (M;N)[v\gets\pred !v/\sleep] \converges \skipp, (t{+}s''\vert v\mapsto k+m+n)\,,
      \end{mathpar}
      by the sequencing rule of Idealized Algol.
    \item[Remaining rules]
      Having given the rule for canonical forms and sequencing explicitly, we now give a general technique by which we can deal with all the remaining rules.

      Each rule is of the form
      \[
        \inferrule*{ \Gamma_1,s^{(0)}\ts z_1\converges_{n_1}\gamma_1,s^{(1)} \\
        \cdots \\
        \Gamma_p,s^{(p - 1)}\ts z_p \converges_{n_p} \gamma_p,s^{(p)}}
        { \Gamma,s^{(0)}\ts z \converges_{n_1+\cdots+n_p} \gamma,s^{(p)}}
        \]
      (for $p\in\{0,1,2,3\}$), where
      \[
        \inferrule*{ \Gamma_1,s^{(0)}\ts z_1\converges \gamma_p,s^{(1)} \\
        \cdots \\
        \Gamma_p,s^{(p - 1)}\ts z_p \converges \gamma_1,s^{(p)}}
        { \Gamma,s^{(0)}\ts z \converges \gamma,s^{(p)}}
        \]
      is a rule for Idealized Algol.

      Here, each $z_i,z,\gamma_i,\gamma$ is a term of Idealized Algol (without $\sleep$) that may have free variables that are not included in the $\Gamma_i,\Gamma$.  
      These free variables $(x_\alpha)$ may be shared between the $z_i,z,\gamma_i,\gamma$, introducing a dependence between them.
      For example, in the rule for sequencing, the terms on the top are single-variable terms of IA, where `$M$' and `$N$' are themselves free variables.  
      These variables are bound in the expression `$M;N$' on the bottom of the rule.

      Fix some such rule, and suppose that it is used in the last step of the deduction.  
      So the last step looks like this.
      \[
        \inferrule*{ \Gamma_1,s^{(0)}\ts M_1\converges_{n_1}c_1,s^{(1)} \\
        \cdots \\
        \Gamma_p,s^{(p - 1)}\ts M_p \converges_{n_p} c_p,s^{(p)}}
        { \Gamma,s^{(0)}\ts M \converges_{n_1+\cdots+n_p} c,s^{(p)}}
        \]
      Here, the $M_i,M,c_i,c$ fit the pattern given by the $z_i,z,\gamma_i,\gamma$.  
      By induction, for each $i=1,\cdots,p$, the following judgement may be proved in Idealized Algol.
      \begin{mathpar}
        \Delta,\Gamma,v,(t{+}s^{(i)}\vert v \mapsto k + n_1 + \cdots + n_{i-1})
        \hspace{144pt} \par\vspace{-20pt} \\ \par \hspace{144pt}
        \ts
        M[v \gets \suc !v] \converges c_i,(t{+}s^{(i)}\vert v \mapsto k + n_1 + \cdots + n_i)
      \end{mathpar}

      Now we know that the $M_i,M,c_i,c$ may be obtained from the $z_i,z,\gamma_i,\gamma$ by substituting in actual terms $N_\alpha$ for the free variables $x_\alpha$ in these expressions.  
      Moreover, since the primitive $\sleep$ is not mentioned anywhere in these rules, it follows that we have
      \begin{IEEEeqnarray*}{rCl}
        M_i[v\gets\suc !v/\sleep] & = & (z_i[N\alpha/x_\alpha])[v\gets \suc !v/\sleep] \\
        & = & z_i[(N[v\gets \suc !v/\sleep])/x_\alpha]\,,
      \end{IEEEeqnarray*}
      and similarly for the $z,\gamma_i,\gamma$.
      Therefore, the $M_i,M,c_i,c$ still fit the pattern given by the $z_i,z,\gamma_i,\gamma$, even after making our substitution.  
      It follows that we may derive
      \begin{mathpar}
        \Delta,\Gamma,v,(t{+}s^{(0)}\vert v \mapsto k)\hspace{180pt}\par\vspace{-8pt}\\ \par
        \hspace{180pt}\ts M[v\gets \suc !v/\sleep] \converges c,(t{+}s^{(p)}\vert v \mapsto k+n_1+\cdots+n_p)
      \end{mathpar}
      using the derivation shown in Figure \ref{GeneralIADerivation}.
  \end{description}
  This completes the induction.
\end{proof}

In light of this result, we can make a definition relating the denotational and operational semantics of terms of type $\nat$.  
First, observe that since the $\beta$ rule is valid in our semantics, we may write the denotation of a term $M\from\com$ of IA+$\sleep$ as the composite
\[
  1 \xrightarrow{\sleep}
  \bC \xrightarrow{\deno{c\ts M[c/\sleep]}}
  \bC
  \]
in $\Kl_{R_\bC}$.

Working in the original category $\G$, however, we may identify this composite with the morphism
\[
  \deno{c\ts M[c/\sleep]} \from \bC \to \bC\,.
  \]
Now we have morphisms
\begin{mathpar}
  \deno{v\ts v \gets \suc !v} \from \Var \to \bC
  \and
  \deno{v\ts !v} \from \Var \to \bN\,,
\end{mathpar}
which together induce the composite
\[
  \Var \xrightarrow{\langle \deno{v\ts v \gets \suc !v},\deno{v\ts !v}\rangle}
  \bC \times \bN \xrightarrow{\deno{c \ts M[c/\sleep]} \times \id}
  \bC \times \bN \xrightarrow{\deno{\blank;\blank}}
  \bN\,.
  \]
Lastly, we may curry this term to give a morphism $1 \to (\Var\to \bN)$ and compose with the morphism
\[
  \new_\bN \from (\Var \to \bN) \to \bN\,.
  \]
The resulting morphism will be the denotation of the term
\[
  \new\;v=0\text{ in } M[v\gets \suc !v/\sleep];!v\from \nat\,.
  \]
Lemma \ref{IaSleepsSoundnessLemma} tells us that if $M\converges_n\skipp$ then this term converges to $n$ in Idealized Algol.

Before we introduce our adequacy result, we need one more definition in order to take account of the fact that Proposition \ref{SamsonGuyAdequacy} applies to terms of type $\com$, not $\nat$.

\newcommand{\test}{{\mathtt{test}}}
\begin{definition}
  Define terms $\test_n\from\nat\to\com$ inductively by
  \begin{gather*}
    \test_0 = \lambda m.\IfO m\text{ then }\skipp\text{ else }\Omega\,; \\
    \test_{n+1} = \lambda m.\IfO m\text{ then }\Omega\text{ else }\test_n\,. (\pred m)
  \end{gather*}
  So $\test_n$ converges if its input evaluates to $n$ and diverges otherwise.
  
  Let $t_n\from\bN \to \bC$ be the denotation in $\G$ of $\test_n$.
\end{definition}

\begin{remark}
  In the model $\G$ of Idealized Algol, every morphism $1 \to \bN$ is definable, so if $\sigma\from 1 \to \bN$ is such a morphism, then there must be at most one $n\in\omega$ such that the composite $\sigma;t_n$ is not equal to $\bot$.
\end{remark}

\newcommand{\tit}{\mathrm{tt}}
\begin{definition}
  Let $\sigma\from 1 \to \bC$ be a Kleisli morphism in $\Kl_{R_{\bC}}$, considered as a morphism $\bC \to \bC$ in $\G$.  
  Define:
  \begin{IEEEeqnarray*}{rCl}
    \sigma' & = & \Var \xrightarrow{\langle \deno{v\ts v \gets \suc !v},\deno{v\ts !v}\rangle}
    \bC \times \bN \xrightarrow{\sigma\times\id}
    \bC \times \bN \xrightarrow{\deno{\blank;\blank}}
    \bN\,; \\
    \sigma'' & = & 1 \xrightarrow{\text{curry }\sigma''}
    (\Var \to \bN) \xrightarrow{\new_\bN}
    \bN\,.
  \end{IEEEeqnarray*}
  
  If there is some (necessarily unique) $n\in\omega$ such that
  \[
    1 \xrightarrow{\sigma''}
    \bN \xrightarrow{t_n}
    \bC
    \]
  is not equal to $\bot$, then we say that $n$ is the \emph{time taken for $\sigma$ to converge}, or $\tit(\sigma)$.
  If there is no such $n$, we say that $\tit(\sigma)=\infty$.
\end{definition}

From our discussion above, we have proved the following result.

\begin{proposition}[Soundness for IA+$\sleep$]
  Let $P\from \com$ be a term of IA+$\sleep$ and let $\sigma\from \bC \to \bC$ be its denotation in $\Kl_{R_\bC}$, considered as a morphism in $\G$.
  If $P\converges_n$ then $\tit(\sigma)=n$.
  \label{IaSleepSoundness}
\end{proposition}

In order to prove the other direction (adequacy), we need to prove a kind of converse to Lemma \ref{IaSleepsSoundnessLemma}.

\begin{lemma}
  Let $\Gamma\ts M\from \com$ be a term of IA+$\sleep$ in context.
  Let $s,s'$ be stores, let $a,b\in\bN$, let $c$ be a canonical form and let $v$ be a variable not included in $\Gamma$.  
  Suppose that
  \[
    \Gamma,v,(s\vert v\mapsto a) \ts M \converges c,(s'\vert v\mapsto b)
    \]
  in Idealized Algol.
  Then there exists $n$ such that
  \[
    \Gamma,s \ts M \converges_n c,s'
    \]
  in IA+$\sleep$.
\end{lemma}
\begin{proof}
  Induction on the derivation that $\Gamma,v,(s\vert v\mapsto a)\ts M\converges c,(s'\vert v\mapsto b)$.  
  Most cases are self-explanatory.  
  For example, suppose that the last step in the derivation is an instance of the sequencing rule:
  \[
    \inferrule{\Gamma,v,(s\vert v \mapsto a) \ts M[v\gets\suc !v/\sleep] \converges \skipp, s' \\
    \Gamma,v,s' \ts N[v\gets\suc !v/\sleep] \converges c, (s''\vert v \mapsto b)}
    {\Gamma,v(s\vert v \mapsto a) \ts (M;N)[v\gets\suc !v/\sleep] \converges c,(s''\vert v \mapsto b)}\,.
    \]
  Since there is no provable sequent of Idealized Algol in which a defined variable on the left ceases to be defined on the right, we may deduce that $s'$ can be written as $(t\vert v \mapsto d)$.  
  Therefore, we may apply the inductive hypothesis to $M$ and $N$, which tells us that there must be $m,n$ such that
  \begin{mathpar}
    \Gamma,s \ts M\converges_m\skipp,t\,;
    \and
    \Gamma,t \ts M\converges_n c,s''\,.
  \end{mathpar}
  From this it follows that
  \[
    \Gamma,s \ts M;N \converges_{m+n} c,s''\,.
    \]
  Crucial to this argument is the fact that
  \[
    (M[v\gets\suc !v/\sleep]);(N[v\gets\suc !v/\sleep]) = (M;N)[v\gets\suc!v/\sleep]\,.
    \]
  For almost all the other rules, a similar equality holds.  
  The only case where there is a problem is the first rule for $\blank\gets\blank$, since this may introduce a new copy of $v\gets\suc!v$ on the bottom that does not arise from copies of $v\gets\suc!v$ on the top.

  The offending term occurs when $M=\sleep$ and so $M[v\gets\suc!v/\sleep]=v\gets\suc!v$.
  In that case, the inference is
  \[
    \inferrule{\Gamma,(s\vert v \mapsto a)\ts \suc ! v \converges b,s'\\
    \Gamma,s'\ts v \converges v,s''}
    {\Gamma,(s\vert v\mapsto a)\ts v \gets\suc!v \converges \skipp,(s''\vert v\mapsto b)}\,.
    \]
  By following the derivation upwards, we may deduce that in fact $s=s'=s''$.
  But in that case the derivation of $\Gamma,s\ts_n M\converges \skipp,s''$ is one of the base rules:
  \[
    \inferrule{ }
    {\Gamma,s \ts \sleep \converges_1 \skipp,s}\,.
    \]
  The other cases are routine.  
  This completes the induction.
  \label{IaSleepAdequacyLemma}
\end{proof}

\begin{theorem}[Computational adequacy for IA+$\sleep$]
  Let $P\from\com$ be a term of IA+$\sleep$ and let $\sigma\from \bC\to \bC$ be its denotation in $\Kl_{R_\bC}$, considered as a morphism in $\G$.
  Then $P\converges_n$ if and only if $\tit(\sigma)=n$.
  \label{IaSleepComputationalAdequacy}
\end{theorem}
\begin{proof}
  The forward direction is Proposition \ref{IaSleepSoundness}.  
  For the reverse direction, note that if $\tit(\sigma)=n$, it means that
  \[
    \deno{\test_n(\new v=0\text{ in }P[v\gets \suc!v/\sleep];!v])} \ne \bot\,,
  \]
  which, by Proposition \ref{SamsonGuyAdequacy}, means that
  \[
    \test_n(\new v=0\text{ in } P[v\gets\suc!v/\sleep];!v])\converges
  \]
  in Idealized Algol, and therefore that
  \[
    v,(v\mapsto 0) \ts P[v\gets\suc!v/\sleep]\converges \skipp,(v\mapsto n)\,.
    \]
  From Lemma \ref{IaSleepAdequacyLemma}, we deduce that we must have
  \[
    ,() \ts P \converges_m \skipp,()
    \]
  in IA+$\sleep$ for some $m$, and by Lemma \ref{IaSleepsSoundnessLemma} we must have $m=n$.  
  Therefore, $P\converges_n$.
\end{proof}

The last definition we need to make is the standard intrinsic equivalence relation.

\begin{definition}
  Let $\sigma,\tau\from A \to B$ be morphisms in $\Kl_{R_\bC}$.  
  By currying, we may consider $\sigma,\tau$ to be morphisms $1 \to (A \to B)$.  
  We say that $\sigma\sim\tau$ if for all morphisms $\alpha\from (A \to B) \to \bC$, we have $\tit(\sigma;\alpha)=\tit(\sigma;\beta)$.
\end{definition}

We now have all the tools necessary to prove that $Kl_{R_\bC}$ is fully abstract for IA+$\sleep$.
The technique we use is standard and relies on a factorization result, as with the proofs given in \cite{SamsonGuyIAActive} and \cite{mcCHFiniteND}.  
The difference here is that our factorization result is immediate, and does not rely on any combinatorial arguments.

\begin{theorem}
  Let $M,N\from T$ be two terms of IA+$\sleep$.  
  Then $M$ and $N$ are observationally equivalent if and only if $\deno{M}\sim\deno{N}$.
  \label{IaSleepTimeComplexity}
\end{theorem}
\begin{proof}
  First suppose that $M$ and $N$ are not observationally equivalent.  
  So there is some context $-\from T\ts C[-]\from\com$ such that $C[M]$ and $C[N]$ have different operational behaviours in IA+$\sleep$.
  By Theorem \ref{IaSleepComputationalAdequacy} we know that for any program $P$ we have $P\converges_n$ if and only if $\tit(\deno P)=n$ and $P\diverges$ if and only if $\tit(\deno P)=\infty$.
  Therefore, we must have $\tit(\deno{C[M]})\ne \tit(\deno{C[n]})$.  
  But since $\deno{C[M]}=\deno{M};\deno{C}$ and $\deno{C[N]}=\deno{N};\deno{C}$, we must have $\deno M\not\sim \deno N$.

  Conversely, suppose that $\deno M \not\sim \deno N$.  
  So there is some $\alpha\from \deno{T} \to \bC$ such that $\deno{M};\alpha\ne\deno{N};\alpha$.

  Let us now work in the category $\G$.  
  So $\deno{M},\deno{N}$ may be regarded as morphisms $\bC \to \deno{T}$ and $\alpha$ may be regarded as a morphism $\deno{T} \to (\bC \to \bC)$, or equivalently as a morphism $\deno{T}\times \bC \to \bC$.
  Then the composites $\deno M;\alpha,\deno N;\alpha$ within $\Kl_{R_\bC}$ are given by the following composites in $\G$.
  \begin{mathpar}
    \bC \xrightarrow{\langle \deno M,\id\rangle}
    \deno T \times \bC \xrightarrow{\alpha}
    \bC
    \and
    \bC \xrightarrow{\langle \deno N,\id\rangle}
    \deno T \times \bC \xrightarrow{\alpha}
    \bC
  \end{mathpar} 
  By assumption, these composites are not equal in $\G$.  
  Then it is sufficient (since this is the case in $\G$) to assume that $\alpha$ is a compact strategy.
  Then, by the compact definability result in \cite{SamsonGuyIAActive}, $\alpha$ must be the denotation (in $\G$) of some term $P\from T \to \com \to \com$, which corresponds to the context
  \[
    C[x] = P\,x\;\sleep
    \]
  in $\Kl_{R_\bC}$.

  It follows that $\deno{C[M]} \ne \deno{C[N]}$.  
  Therefore, by Proposition \ref{IaSleepComputationalAdequacy}, $M$ and $N$ are not observationally equivalent.
\end{proof}

\bibliographystyle{alpha}
\bibliography{../common/phd_bibliography}

\end{document}

\documentclass{article}

\usepackage[utf8]{inputenc}

\usepackage{graphicx} % support the \includegraphics command and options

\usepackage{parskip} % Activate to begin paragraphs with an empty line rather than an indent

%%% PACKAGES
\usepackage{booktabs} % for much better looking tables
\usepackage{array} % for better arrays (eg matrices) in maths
\ifdefined\BEAMER
\else
\usepackage{paralist} % very flexible & customisable lists (eg. enumerate/itemize, etc.)\prefix\t$.
\fi
\usepackage{verbatim} % adds environment for commenting out blocks of text & for better verbatim
\ifdefined\BEAMER
\else
\ifdefined\THESIS
\usepackage{subcaption}
\else
\usepackage{subfig} % make it possible to include more than one captioned figure/table in a single float
\fi
\fi
\usepackage{mathtools} % for the all important \coloneqq symbol
\usepackage{hyperref} % for hyperreferences
\usepackage{IEEEtrantools} % for \IEEEeqnarray
\usepackage{pbox} % for \pbox
\usepackage{multirow,bigdelim} % for \multirow
\usepackage{lettrine} % For the drop cap
\usepackage{mathpartir} % for \inferrule, \inferrule* and the mathpar environment
\usepackage{listings}

\usepackage{caption}
\captionsetup{singlelinecheck=off}

\ifdefined\NOTARTICLE
\else

%%% ToC (table of contents) APPEARANCE
\usepackage[nottoc,notlof,notlot]{tocbibind} % Put the bibliography in the ToC
\usepackage[titles,subfigure]{tocloft} % Alter the style of the Table of Contents
\renewcommand{\cftsecfont}{\rmfamily\mdseries\upshape}
\renewcommand{\cftsecpagefont}{\rmfamily\mdseries\upshape} % No bold!

\fi

%% Font things %%
\usepackage{amssymb}
\usepackage{cmll} % Linear logic symbols!
\ifdefined\FEWFONTS
\else
\usepackage{bm} % for bold Greek letters
\fi
\usepackage{stmaryrd}
\usepackage{bbm}

%% Get the sqsubsetneqq character from the mathabx package
\DeclareFontFamily{U}{mathb}{\hyphenchar\font45}
\DeclareFontShape{U}{mathb}{m}{n}{
      <5> <6> <7> <8> <9> <10> gen * mathb
      <10.95> mathb10 <12> <14.4> <17.28> <20.74> <24.88> mathb12
      }{}
\DeclareSymbolFont{mathb}{U}{mathb}{m}{n}

\DeclareMathSymbol{\sqsubsetneq}    {3}{mathb}{"88}
\DeclareMathSymbol{\varsqsubsetneq} {3}{mathb}{"8A}
\DeclareMathSymbol{\varsqsubsetneqq}{3}{mathb}{"92}
\DeclareMathSymbol{\sqsubsetneqq}   {3}{mathb}{"90}

%% Get the left and right moons from the wasysym package

\DeclareFontFamily{U}{wasy}{}
\DeclareFontShape{U}{wasy}{m}{n}{ <5> <6> <7> <8> <9> gen * wasy
      <10> <10.95> <12> <14.4> <17.28> <20.74> <24.88>wasy10  }{}
\DeclareFontShape{U}{wasy}{b}{n}{ <-10> sub * wasy/m/n
 <10> <10.95> <12> <14.4> <17.28> <20.74> <24.88>wasyb10 }{}
\DeclareFontShape{U}{wasy}{bx}{n}{ <-> sub * wasy/b/n}{}

\def\wasyfamily{\fontencoding{U}\fontfamily{wasy}\selectfont}
\def\leftmoon   {\mbox{\wasyfamily\char36}}
\def\rightmoon  {\mbox{\wasyfamily\char37}}

%% Lists %%
\usepackage{enumerate}

%% Graphics %%
\usepackage{tikz}
\usetikzlibrary{cd}
\usetikzlibrary{patterns}
\usetikzlibrary{calc}
\usetikzlibrary{decorations.pathmorphing}
\usetikzlibrary{positioning}

\tikzset{inlinearrows/.style={anchor=base,baseline,x=0.6\baselineskip,y=0.6\baselineskip}}

\ifdefined\BEAMER
\else

%% Theorems! %%
\usepackage{amsthm}
\theoremstyle{plain} % Theorems, lemmas, propositions etc.
\newtheorem{theorem}{Theorem}[section]
\newtheorem{lemma}[theorem]{Lemma}
\newtheorem{proposition}[theorem]{Proposition}
\newtheorem{corollary}[theorem]{Corollary}
\newtheorem{fact}[theorem]{Fact}
\newtheorem{construction}[theorem]{Construction}
\theoremstyle{definition} % Definitions etc.  
\newtheorem{definition}[theorem]{Definition}
\newtheorem{notation}[theorem]{Notation}
\theoremstyle{remark} % Remarks
\newtheorem{remark}[theorem]{Remark}
\newtheorem{remarks}[theorem]{Remarks}
\newtheorem{example}[theorem]{Example}
\newtheorem{question}[theorem]{Question}
\newtheorem{slogan}[theorem]{Slogan}

\newtheoremstyle{note} {3pt} {3pt} {\itshape} {} {\itshape} {:} {.5em} {} % For short notes
\theoremstyle{note}
\newtheorem{note}[theorem]{Note}

\fi

%% Exercises and answers %%
\usepackage{answers}

\newtheoremstyle{exercisestyle}% name
  {6pt}   % ABOVESPACE
  {6pt}   % BELOWSPACE
  {\itshape}  % BODYFONT
  {0pt}       % INDENT (empty value is the same as 0pt)
  {\bfseries} % HEADFONT
  {.}         % HEADPUNCT
  {3pt} % HEADSPACE
  {}          % CUSTOM-HEAD-SPEC

\theoremstyle{exercisestyle}
\newtheorem{exercise}{Exercise}
\newtheorem{answerthm}{Exercise}

\Newassociation{answer}{answerthm}{answers}
\newcommand{\answerthmparams}{}

%% Changes to enumerate things so they look better %%\sigma$

\makeatletter
\def\enumfix{%
\if@inlabel
 \noindent \par\nobreak\vskip-\topsep\hrule\@height\z@
\fi}

\let\olditemize\itemize
\def\itemize{\enumfix\olditemize}
\let\oldenumerate\enumerate
\def\enumerate{\enumfix\oldenumerate}

%% Random crap %%
\usepackage{xifthen}

\makeatletter
\def\thm@space@setup{%
  \thm@preskip=\parskip \thm@postskip=0pt
}
\makeatother

\makeatletter
\newcommand*{\relrelbarsep}{.386ex}
\newcommand*{\relrelbar}{%
  \mathrel{%
    \mathpalette\@relrelbar\relrelbarsep
  }%
}
\newcommand*{\@relrelbar}[2]{%
  \raise#2\hbox to 0pt{$\m@th#1\relbar$\hss}%
  \lower#2\hbox{$\m@th#1\relbar$}%
}
\providecommand*{\rightrightarrowsfill@}{%
  \arrowfill@\relrelbar\relrelbar\rightrightarrows
}
\providecommand*{\leftleftarrowsfill@}{%
  \arrowfill@\leftleftarrows\relrelbar\relrelbar
}
\providecommand*{\xrightrightarrows}[2][]{%
  \ext@arrow 0359\rightrightarrowsfill@{#1}{#2}%
}
\providecommand*{\xleftleftarrows}[2][]{%
  \ext@arrow 3095\leftleftarrowsfill@{#1}{#2}%
}
\makeatother

\newcommand{\catname}[1]{{\normalfont\textbf{#1}}}
\newcommand{\Rings}{\catname{CRing}}
\newcommand{\CAT}{\catname{CAT}}
%\newcommand{\Top}{\catname{Top}}
\newcommand{\Set}{\catname{Set}}
\newcommand{\Cat}{\catname{Cat}}
\newcommand{\MonCat}{\catname{MonCat}}
\newcommand{\SymmMonCat}{\catname{SymmMonCat}}
\newcommand{\Cont}{\catname{Cont}}
\newcommand{\Sch}{\catname{Sch}}
\newcommand{\Rel}{\catname{Rel}}
\newcommand{\Coh}{\catname{Coh}}
\newcommand{\Inj}{\catname{Inj}}
\newcommand{\Dcpo}{\catname{Dcpo}}
\newcommand{\Mod}[1][]{\ifthenelse{\isempty{#1}}{\catname{Mod}}{#1\catname{mod}}}
\DeclareMathOperator{\sh}{Sh}
\newcommand{\Sh}[1][]{\ifthenelse{\isempty{#1}}{\sh}{\sh(#1)}}
\newcommand{\map}[3]{#2\xrightarrow{#1} #3}
\newcommand*\from{\colon}
\newcommand*\bigto{\Rightarrow}
\newcommand{\cmap}[3]{#1\from{}#2\to{}#3}
\newcommand\oppcat[1]{#1^{\mathrm{op}}}
\newcommand{\object}{\colon}
\DeclareRobustCommand{\vmap}[3] {\begin{tikzcd} #2 \arrow[d, "#1"] \\ #3 \end{tikzcd}}
\newcommand{\partref}[1]{(\ref{#1})}
\newcommand{\intgrpd}[4] {#1 \xrightrightarrows[#3]{#4} #2}
\DeclareRobustCommand{\bigintgrpd}[4] {\begin{tikzcd}[ampersand replacement=\&] #1 \arrow[r, shift left=0.5ex, "#3"] \arrow[r, shift right=0.5ex, "#4"'] \& #2 \end{tikzcd}}

\usepackage{xspace}

\newcommand{\etale}{\'{e}tale\xspace}
\newcommand{\Etale}{\'{E}tale\xspace}

\def \inv {^{-1}}

\DeclareMathOperator{\id}{id}
\DeclareMathOperator{\op}{op}
\DeclareMathOperator{\pr}{pr}
\DeclareMathOperator{\inj}{in}
\DeclareMathOperator{\pre}{{pre}}
\DeclareMathOperator{\et}{{\acute{e}t}}

\DeclareMathOperator{\Hom}{Hom}
\DeclareMathOperator{\Spec}{Spec}

\DeclareMathOperator{\ol}{ol}

\def\presuper#1#2%
  {\mathop{}%
   \mathopen{\vphantom{#2}}^{#1}%
   \kern-\scriptspace%
   #2}
\def\presub#1#2%
  {\mathop{}%
   \mathopen{\vphantom{#2}}_{#1}%
   \kern-\scriptspace%
   #2}

\newsavebox{\overlongequation}
\newenvironment{longdiagram}
 {\begin{displaymath}\begin{lrbox}{\overlongequation}$\displaystyle}
 {$\end{lrbox}\makebox[0pt]{\usebox{\overlongequation}}\end{displaymath}}

%% Our things %%

\newcommand{\neggame}[1]{\presuper{\perp}{#1}}
\newcommand{\tensor}{\otimes}
\newcommand{\Tensor}{\bigotimes}
\newcommand{\sequoid}{\oslash}
\newcommand{\varsequoid}{\vartriangleleft}
\renewcommand{\implies}{\multimap}
\newcommand{\iimpl}{\Longrightarrow}
\newcommand{\comp}[2]{#1 \circ #2}
\newcommand{\icomp}[2]{\comp{#1}{#2}}
\newcommand{\cprd}{\sqcup}
\newcommand{\bigcprd}{\bigsqcup}
\newcommand{\G}{\mathcal G}
\newcommand{\W}{\mathcal W}
\newcommand{\suchthat}{\;\colon\;}
\newcommand{\varsuchthat}{\;\mid\;}
\newcommand{\esuchthat}{\;.\;}
\newcommand{\OP}{\{O,P\}}
\newcommand{\QA}{\{Q,A\}}
\renewcommand{\L}{\mathcal L}
\newcommand{\F}{\mathcal F}
\newcommand{\U}{\mathcal U}
\newcommand{\s}{\mathfrak s}
\renewcommand{\t}{\mathfrak t}
\renewcommand{\u}{\mathfrak u}
\renewcommand{\d}{\mathfrak d}
\newcommand{\e}{\mathfrak e}
\newcommand{\emptyplay}{\epsilon}
\newcommand{\bracketed}[1]{\left({#1}\right)}
\newcommand{\bneggame}[1]{{\bracketed{\neggame{#1}}}}
\newcommand{\prefix}{\sqsubseteq}
\newcommand{\ppprefix}{\sqsubset}
\newcommand{\pprefix}{\sqsubsetneqq}
\renewcommand{\ss}{\mathbf{s}}
\newcommand{\bN}{\mathbb{N}}
\newcommand{\bC}{\mathbb{C}}
\newcommand{\bB}{\mathbb{B}}
\newcommand{\bP}{\mathbb{P}}
\newcommand{\pfun}{\rightharpoonup}
\newcommand{\grel}[1]{\underline{#1}}
\DeclareMathOperator{\length}{length}
\renewcommand{\b}{\mathfrak b}
\renewcommand{\r}{\mathfrak r}
\newcommand{\bbeta}{{\bm{\beta}}}
\newcommand{\st}{{\Sigma^*}}
\let\sec\S
\renewcommand{\S}{{\mathfrak{S}}}
\DeclareMathOperator{\cc}{cc}
\DeclareMathOperator{\subs}{subs}
\DeclareMathOperator{\ret}{ret}
\DeclareMathOperator{\zz}{zz}
\newcommand{\aaa}{\mathbf{a}}
\newcommand{\bbb}{\mathbf{b}}
\newcommand{\ccc}{\mathbf{c}}
\newcommand{\ddd}{\mathbf{d}}
\newcommand{\B}{\mathcal B}
\newcommand{\BB}{\mathbf B}
\renewcommand{\H}{\mathcal H}
\DeclareMathOperator{\assoc}{assoc}
\DeclareMathOperator{\lunit}{lunit}
\DeclareMathOperator{\runit}{runit}
\DeclareMathOperator{\dom}{dom}
\DeclareMathOperator{\sym}{sym}
\newcommand{\braid}{\sym}
\newcommand{\blank}{\,\underline{\hspace{1.5ex}}\,}
\DeclareMathOperator{\cn}{cn}
\newcommand{\impliescn}{\protect\overset{\cn}{\implies}}
\newcommand{\C}{{\mathcal{C}}}
\newcommand{\D}{{\mathcal{D}}}
\newcommand{\E}{{\mathcal{E}}}
\newcommand{\V}{{\mathcal{V}}}
\newcommand{\EE}{{\mathbf{E}}}
\DeclareMathOperator{\ev}{ev}
\newcommand{\der}{{\mathtt{der}}}
\newcommand{\mult}{{\mathtt{mult}}}
\DeclareMathOperator{\wk}{wk}
\newcommand{\toisom}{{\xrightarrow{\cong}}}
\DeclareMathOperator{\passoc}{{\mathsf{passoc}}}
\DeclareMathOperator{\pcomm}{{\mathsf{pcomm}}}
\DeclareMathOperator{\run}{{\mathsf{r}}}
\DeclareMathOperator{\lun}{{\mathsf{l}}}
\newcommand{\fcoal}[1]{{\leftmoon #1 \rightmoon}}
\DeclareMathSymbol{\co}{\mathord}{operators}{"3C}
\DeclareMathSymbol{\nw}{\mathord}{operators}{"3E}
\newcommand{\T}{\mathfrak{T}}
\renewcommand{\subset}{\subseteq}
\newcommand{\Ord}{\catname{Ord}}
\newcommand{\FS}{\mathcal{FS}}
\DeclareMathOperator{\rank}{rank}
\DeclareMathOperator{\dist}{{\mathsf{dist}}}
\DeclareMathOperator{\dec}{{\mathsf{dec}}}
\DeclareMathOperator{\str}{str}
\DeclareMathOperator{\weak}{weak}
\DeclareMathOperator{\Strat}{Strat}
\DeclareMathOperator{\OppStrat}{OppStrat}
\newcommand{\seqs}[1]{{\overline{{#1}^{*}}}}
\def\flushRight{\leftskip0pt plus 1fill\rightskip0pt}
\def\Centering{\relax\ifvmode\centering\fi}
\newcommand{\deno}[1]{\left\llbracket#1\right\rrbracket}
\newcommand{\converges}{\Downarrow}
\newcommand{\diverges}{\Uparrow}
\newcommand{\mustconverge}{\converges^{\text{must}}}
\newcommand{\Iflt}{\mathtt{If{<}\;}}
\newcommand{\Ifgt}{\mathtt{If{>}\;}}
\newcommand{\inr}{{\mathsf{inr}}}
\newcommand{\inl}{{\mathsf{inl}}}
\newcommand{{\Na}}{\bN}
\newcommand{{\cell}}{{\mathsf{cell}}}
\newcommand{\fix}{{\mathsf{fix}}}
\newcommand{\eq}{{\mathsf{eq}}}
\DeclareMathOperator{\CCom}{CCom}
\newcommand{\power}{\mathfrak P}

% Slanty things
\newcommand*{\xslant}[2][76]{%
  \begingroup
    \sbox0{#2}%
    \pgfmathsetlengthmacro\wdslant{\the\wd0 + cos(#1)*\the\wd0}%
    \leavevmode
    \hbox to \wdslant{\hss
      \tikz[
        baseline=(X.base),
        inner sep=0pt,
        transform canvas={xslant=cos(#1)},
      ] \node (X) {\usebox0};%
      \hss
      \vrule width 0pt height\ht0 depth\dp0 %
    }%
  \endgroup
}

\makeatletter
\newcommand*{\xslantmath}{}
\def\xslantmath#1#{%
  \@xslantmath{#1}%
}
\newcommand*{\@xslantmath}[2]{%
  % #1: optional argument for \xslant including brackets
  % #2: math symbol
  \ensuremath{%
    \mathpalette{\@@xslantmath{#1}}{#2}%
  }%
}
\newcommand*{\@@xslantmath}[3]{%
  % #1: optional argument for \xslant including brackets
  % #2: math style
  % #3: math symbol
  \xslant#1{$#2#3\m@th$}%
}
\makeatother

\newcommand{\seqdeno}[1]{\xslantmath{\llbracket}#1\xslantmath{\rrbracket}}

% Empty set etc.

\let\oldemptyset\emptyset
\let\emptyset\varnothing

%% Constant width xrightarrows
\newlength{\arrow}
\settowidth{\arrow}{\scriptsize$1000$}
\newcommand*{\constantwidthxrightarrow}[1]{\xrightarrow{\mathmakebox[\arrow]{#1}}}

%% Landscape pages
\usepackage{everypage}
\usepackage{environ}
\usepackage{pdflscape}
\newcounter{abspage}

\ifdefined\NOTARTICLE

\else

\makeatletter
\newcommand{\newSFPage}[1]% #1 = \theabspage
  {\global\expandafter\let\csname SFPage@#1\endcsname\null}

\NewEnviron{SidewaysFigure}{\begin{figure}[p]
\protected@write\@auxout{\let\theabspage=\relax}% delays expansion until shipout
  {\string\newSFPage{\theabspage}}%
\ifdim\textwidth=\textheight
  \rotatebox{90}{\parbox[c][\textwidth][c]{\linewidth}{\BODY}}%
\else
  \rotatebox{90}{\parbox[c][\textwidth][c]{\textheight}{\BODY}}%
\fi
\end{figure}}

\AddEverypageHook{% check if sideways figure on this page
  \ifdim\textwidth=\textheight
    \stepcounter{abspage}% already in landscape
  \else
    \@ifundefined{SFPage@\theabspage}{}{\global\pdfpageattr{/Rotate 0}}%
    \stepcounter{abspage}%
    \@ifundefined{SFPage@\theabspage}{}{\global\pdfpageattr{/Rotate 90}}%
  \fi}
\makeatother

\fi

%% PCF Things

\newcommand{\nat}{{\mathtt{nat}}}
\newcommand{\bool}{{\mathtt{bool}}}

\newcommand{\Y}{\mathbf{Y}}
\newcommand{\opto}{\longrightarrow}
\newcommand{\oopto}{\dashrightarrow}
\newcommand{\n}{{\mathtt{n}}}
\DeclareMathOperator{\IfO}{{\mathsf{If0}}}
\DeclareMathOperator{\suc}{{\mathsf{succ}}}
\DeclareMathOperator{\pred}{{\mathsf{pred}}}
\newcommand{\0}{{\mathtt{0}}}

\newcommand{\iter}{{\mathtt{iter}}}
\newcommand{\rec}{\iter}
\newcommand{\Var}{{\mathtt{Var}}}
\DeclareMathOperator{\Varr}{Var}
\newcommand{\new}{{\mathtt{new}}}
\newcommand{\case}{{\mathtt{case}}}

\newcommand{\lmam}{\mathrel{\sqsubseteq_{m\&m}}}
\newcommand{\emam}{\mathrel{\equiv_{m\&m}}}
\newcommand{\lst}{\mathrel{\lesssim}}
\newcommand{\smam}{\mathrel{\sim_{m\&m}}}
\newcommand{\amam}{\mathrel{\approx_{m\&m}}}

\newcommand{\oes}{\sim}

%% Idealized Algol things

\newcommand{\com}{{\mathtt{com}}}
\newcommand{\skipp}{{\mathsf{skip}}}
\DeclareMathOperator{\seq}{{\mathsf{seq}}}
\DeclareMathOperator{\neww}{{\mathsf{new}}}
\DeclareMathOperator{\mkvar}{{\mathsf{mkvar}}}
\newcommand{\deref}{\texttt{@}}
\DeclareMathOperator{\dereff}{\mathsf{deref}}
\DeclareMathOperator{\assign}{\mathsf{assign}}
\newcommand{\ia}[2]{\langle #1 , #2 \rangle}
\newcommand{\stup}[3]{\langle #1 \mid #2 \mapsto #3 \rangle}

%% Hyland-Ong games things

\newbox\gnBoxA
\newdimen\gnCornerHgt
\setbox\gnBoxA=\hbox{$\ulcorner$}
\global\gnCornerHgt=\ht\gnBoxA
\newdimen\gnArgHgt
\def\pv #1{%
    \setbox\gnBoxA=\hbox{$#1$}%
    \gnArgHgt=\ht\gnBoxA%
    \ifnum     \gnArgHgt<\gnCornerHgt \gnArgHgt=0pt%
    \else \advance \gnArgHgt by -\gnCornerHgt%
    \fi \raise\gnArgHgt\hbox{$\ulcorner$} \box\gnBoxA %
    \raise\gnArgHgt\hbox{$\urcorner$}}
\def\ov #1{%
    \setbox\gnBoxA=\hbox{$#1$}%
    \gnArgHgt=\ht\gnBoxA%
    \ifnum     \gnArgHgt<\gnCornerHgt \gnArgHgt=0pt%
    \else \advance \gnArgHgt by -\gnCornerHgt%
    \fi \raise\gnArgHgt\hbox{$\llcorner$} \box\gnBoxA %
    \raise\gnArgHgt\hbox{$\lrcorner$}}
\newcommand{\ct}[1]{\lceil#1\rceil}
\DeclareMathOperator{\Int}{int}

%% Nondeterministic Factorization things

\newcommand{\code}{\mathsf{code}}
\newcommand{\Det}{\mathsf{Det}}

%% Flexible strategy things

\newcommand{\stle}{{\;\le_s\;}}
\newcommand{\steq}{{\;=_s\;}}
\newcommand{\exle}{\sqsubseteq}
\newcommand{\exlub}{\bigsqcup}
\newcommand{\dv}{{\text{\lightning}}}
\DeclareMathOperator{\pocl}{pocl}
\newcommand{\plot}{\mathrel{\triangleleft}}
\newcommand{\shad}{\mathfrak{S}}
%\newcommand{\tree}{\mathfrak{T}}
\newcommand{\Tau}{T}
\newcommand{\Epsilon}{E}
\newcommand{\sw}{\triangleleft}

%% Roman numerals

\newcommand{\RN}[1]{%
  \textup{\uppercase\expandafter{\romannumeral#1}}%
}
\newcommand{\RNl}[1]{%
  \mathrel{\raisebox{1pt}{$\overline{\underline{#1}}$}}
}

%% Game language things

\newcommand{\ul}[1]{{\underline{#1}}}
\newcommand{\A}{{\mathcal{A}}}
\renewcommand{\P}{\mathcal P}
\newcommand{\M}{\mathcal M}
\newcommand{\N}{\mathcal N}
\newcommand{\X}{\mathcal X}
\newcommand{\YY}{\mathcal Y}
\newcommand{\hole}{\blank}
\newcommand{\Tct}{\xrightarrow{T}t}
\newcommand{\teamconverge}[2]{\xrightarrow{#1}#2}

%% Inference rule things
\newcommand{\rulename}[1]{\LeftTirNameStyle{#1}}
\newcommand{\ts}{\mathbin{\vdash}}
\newcommand{\nts}{\mathbin{\not\vdash}}

%% Double category things
\newcommand{\hc}[2]{\left({#1}\middle|{#2}\right)}
\newcommand{\vc}[2]{\left(\frac{#1}{#2}\right)}

%% What is going on?
\DeclareMathOperator{\Kl}{Kl}
\DeclareMathOperator{\Mell}{Mell}
\newcommand{\powerset}{\mathcal P}
\DeclareMathOperator{\ask}{{\mathsf{ask}}}
\newcommand{\sleep}{{\mathsf{sleep}}}
\newcommand{\true}{\mathbbm{t}}
\newcommand{\false}{\mathbbm{f}}
\DeclareMathOperator{\If}{\mathsf{If}}
\newcommand{\Then}{\mathrel{\mathsf{then}}}
\newcommand{\Else}{\mathrel{\mathsf{else}}}
\newcommand\cat{\mathbin{+\mkern-10mu+}}

%% Profunctor arrows

\makeatletter
\def\slashedarrowfill@#1#2#3#4#5{%
  $\m@th\thickmuskip0mu\medmuskip\thickmuskip\thinmuskip\thickmuskip
   \relax#5#1\mkern-7mu%
   \cleaders\hbox{$#5\mkern-2mu#2\mkern-2mu$}\hfill
   \mathclap{#3}\mathclap{#2}%
   \cleaders\hbox{$#5\mkern-2mu#2\mkern-2mu$}\hfill
   \mkern-7mu#4$%
}
\def\rightslashedarrowfill@{%
  \slashedarrowfill@\relbar\relbar\mapstochar\rightarrow}
\newcommand\xslashedrightarrow[2][]{%
  \ext@arrow 0055{\rightslashedarrowfill@}{#1}{#2}}
\makeatother
\newcommand{\pto}{{\xslashedrightarrow{} }}

%% Profunctors 
\DeclareMathOperator{\Prof}{Prof}
\DeclareMathOperator{\End}{End}
\DeclareMathOperator{\Endoprof}{Endoprof}

%% Our

\def\searchmacro#1{
  \AtBeginOfFiles{%
    \ifdefined#1
      \expandafter\def\csname \currfilename:found\endcsname{}%
    \fi}
  \AtEndOfFiles{%
    \ifdefined#1
      \unless\ifcsname \currfilename:found\endcsname
        \immediate\write\finder{found in '\currfilename'}%
    \fi\fi}}

%% Isomorphism arrows on commutative diagrams %%
\tikzset{Isom/.style={every to/.append style={edge node={node [sloped, above, allow upside down, auto=false]{$\cong$}}}},
         Isom'/.style={every to/.append style={edge node={node [sloped, above, allow upside down, auto=false, rotate=180]{$\cong$}}}},
         Sim/.style={every to/.append style={edge node={node [sloped, above, allow upside down, auto=false]{$\sim$}}}},
         Sim'/.style={every to/.append style={edge node={node [sloped, above, allow upside down, auto=false, rotate=180]{$\sim$}}}}}

%% Adjunctions
\newcommand{\adjunction}[4]{%
  {#1} \underset{\underset{#3}{\longleftarrow}}{\overset{\overset{#2}{\longrightarrow}}{\bot}} {#4}}        

%% Important!
\newcommand\Mellies{Melli\`{e}s\xspace}

\makeatletter
\newcommand{\colim@}[2]{%
  \vtop{\m@th\ialign{##\cr
    \hfil$#1\operator@font colim$\hfil\cr
    \noalign{\nointerlineskip\kern1.5\ex@}#2\cr
    \noalign{\nointerlineskip\kern-\ex@}\cr}}%
}
\newcommand{\colim}{%
  \mathop{\mathpalette\colim@{\rightarrowfill@\textstyle}}\nmlimits@
}
\makeatother

\makeatletter
\newcommand{\laxcolim@}[2]{%
  \vtop{\m@th\ialign{##\cr
    \hfil$#1\operator@font colim_l$\hfil\cr
    \noalign{\nointerlineskip\kern1.5\ex@}#2\cr
    \noalign{\nointerlineskip\kern-\ex@}\cr}}%
}
\newcommand{\laxcolim}{%
  \mathop{\mathpalette\laxcolim@{\rightarrowfill@\textstyle}}\nmlimits@
}
\makeatother

\DeclareMathOperator{\Colim}{colim}

\DeclareMathOperator{\DG}{DG}
\DeclareMathOperator{\RV}{RV}
\newcommand{\Rv}{\catname{Rv}}

\let\choose\undefined
\DeclareMathOperator{\choose}{\mathsf{choose}}
\DeclareMathOperator{\tr}{tr}
\DeclareMathOperator{\test}{test}

%% Slot game things %%
\newcommand{\circled}[1]{\raisebox{.5pt}{\textcircled{\raisebox{-.9pt} {#1}}}}
\newcommand{\slot}{{\circled{\$}}}

\DeclareMathOperator{\may}{may}
\DeclareMathOperator{\must}{must}

\newcommand{\encode}[1]{\lceil#1\rceil}
\DeclareMathOperator{\app}{\mathsf{app}}
\DeclareMathOperator{\lett}{\mathsf{let}}
\newcommand{\inn}{\mathrel{\mathsf{in}}}
\DeclareMathOperator{\byval}{\mathsf{byval}}

\DeclareMathOperator{\rread}{read}
\DeclareMathOperator{\wwrite}{write}

\DeclareSymbolFont{bbsymbol}{U}{bbold}{m}{n}
\DeclareMathSymbol{\bbsemicolon}{\mathbin}{bbsymbol}{"3B}
\newcommand{\semicom}{\bbsemicolon}

\newcommand{\ms}{\makebox[-1pt]{}}

\DeclareMathOperator{\Acc}{Acc}
\DeclareMathOperator{\im}{Im}
\DeclareMathOperator{\wit}{wit}

%%% END Article customizations


\begin{document}
\section{Parametric monads}
\label{ChapParametricMonads}

Since a monoid in a monoidal category $\mathcal X$ is the same thing as a lax monoidal functor from the unit category into $\mathcal X$, the definition of a monad on a category $\C$ (i.e., a monoid in $[\C,\C]$) may be generalized to that of a \emph{lax action} \cite{LaxActions} of a monoidal category $\X$ on $\C$; i.e., a lax monoidal functor $\X \to [\C,\C]$ (so that a monoid in $\C$ is a lax action of the unit category on $\C$).

Equivalently, a lax action is given by a functor $\blank.\blank \from \X \times \C \to \C$ together with natural transformations
\begin{mathpar}
  m_{x,y,a}\from x.y.a \to (x\tensor y).a
  \and
  e_a \from a \to I.a
\end{mathpar}
such that the following diagrams commute.
\begin{mathpar}
  \begin{tikzcd}[column sep=36pt]
    x.y.z.a \arrow[r, "x.m_{y,z,a}"] \arrow[d, "m_{x,y,z.a}"']
      & x.(y\tensor z).a \arrow[r, "m_{x,y\tensor z,a}"]
        & (x \tensor (y \tensor z)) . a \\
    (x\tensor y).z.a \arrow[r, "m_{x\tensor y,z,a}"]
      & ((x\tensor y)\tensor z) . a \arrow[ur, "\assoc_{x,y,z}.a"']
        &
  \end{tikzcd}
  \and
  \begin{tikzcd}
    x.a \arrow[r, "e_{x.a}"] \arrow[dr, "\lunit_x.a"']
      & I.x.a \arrow[d, "m_{I,x,a}"] \\
    %
      & (I\tensor x).a
  \end{tikzcd}
  \and
  \begin{tikzcd}
    x.a \arrow[r, "x.e_a"] \arrow[dr, "\runit_x.a"']
      & x.I.a \arrow[d, "m_{x,I,a}"] \\
    %
      & (x\tensor I) . a
  \end{tikzcd}
\end{mathpar}
Since lax actions generalize monads, we shall follow \Mellies \cite{ParametricMonads} and refer to them as \emph{parametric monads}.

\begin{example}
  If $\X$ is a monoidal category, $\C$ is a monoidal closed category and $j\from \X \to \C$ is an oplax monoidal functor, then we have a lax action of $\oppcat\X$ on $\C$ given by
  \[
    x.a = jx \implies a\,,
    \]
  together with the natural coherences
  \begin{mathpar}
    jx\implies jy\implies a \xrightarrow{\makebox[24pt]{}} (jx\tensor jy) \implies a \xrightarrow{m^j_{x,y}\implies a} j(x\tensor y) \implies a
    \and
    a \xrightarrow{\makebox[24pt]{}} I \implies a \xrightarrow{e^j\implies a} jI \implies a\,.
  \end{mathpar}
  This generalizes the \emph{reader monads} that we met earlier, so we shall call a parametric monad of this form a \emph{parametric reader monad} or \emph{lax reader action}.
\end{example}

\begin{definition}
  \label{DefOplaxMorphismOfActions}
  Suppose that a monoidal category $\X$ acts on a category $\C$ and that it also acts on a category $\D$.  
  An \emph{oplax morphism of actions} from one action to the other is given by a functor $F \from \C \to \D$ together with a natural transformation $\mu_{x,a} \from F(x.a) \to x.Fa$ that makes the following diagrams commute for all objects $x,y$ of $\X$ and $a$ of $\C$.
  \begin{mathpar}
    \begin{tikzcd}
      F(x.y.a) \arrow[r, "{\mu_{x,y.a}}"] \arrow[d, "{Fm_{x,y,a}}"']
        & x.F(y.a) \arrow[r, "{x.\mu_{y,a}}"]
          & x.y.Fa \arrow[d, "{m_{x,y,Fa}}"] \\
      F((x\tensor y).a) \arrow[rr, "{\mu_{x\tensor y,a}}"]
        &
          & (x \tensor y).Fa
    \end{tikzcd}
    \and
    \begin{tikzcd}
      Fa \arrow[d, "Fe_a"'] \arrow[dr, "e_{Fa}"]
        & \\
      F(I.a) \arrow[r, "{\mu_{I,a}}" xshift=-3pt]
        & I.Fa
    \end{tikzcd}
  \end{mathpar}
\end{definition}

\subsection{The \Mellies category}

The main thing we want to do with lax actions is to perform a construction analogous to that of the Kleisli category of a monad.  
Fujii, Katsumata and \Mellies give a construction called a `Kleisli resolution' in the paper \cite{KleisliResolution}, but we shall prefer an alternative construction due to \Mellies.

Suppose that a monoidal category $\X$ acts on a category $\C$.
\begin{definition}
  A \emph{\Mellies morphism} from $a$ to $b$ is one of the form
  \[
    a \to x.b\,,
    \]
  for some object $x$ of $\X$.
\end{definition}

\begin{definition}
  Given \Mellies morphisms $f$ from $a$ to $b$ and $g$ from $b$ to $c$ given by
  \begin{mathpar}
    \tilde{f} \from a \to x.b
    \and
    \tilde{g} \from b \to y.c\,,
  \end{mathpar}
  their \emph{\Mellies composition} is given by the composite
  \[
    a \xrightarrow{\tilde f}
    x.b \xrightarrow{x.\tilde g}
    x.y.c \xrightarrow{m_{x,y,c}}
    (x\tensor y).c\,,
    \]
  which is a \Mellies morphism from $a$ to $c$.
  \label{DefMelliesComposition}
\end{definition}
\begin{definition}[\cite{ThesisForDays}]
  Let $\X$ be a monoidal category, and let $F,G\from \X \to \Set$ be functors.  
  Then the \emph{Day convolution} of $F$ and $G$ is a functor
  \[
    F \tensor_{\text{Day}} G \from \X \to \Set
    \]
  given by
  \[
    (F \tensor_{\text{Day}} G)(x) = \int^{y,z\from \X} F(y) \times G(z) \times \X(y\tensor z,x)\,.
    \]
  This makes $[\X,\Set]$ into a monoidal category (the monoidal unit is the functor $\X(I,\blank)\from \X \to \Set$).
  \label{DefDayConvolution}
\end{definition}

\begin{definition}[\cite{MelliesCategory}]
  Given a lax action of a monoidal category $\X$ upon a category $\C$, the \emph{\Mellies category} of the action is an $[\X,\Set]$-enriched category $\Mell_\X\C$ whose objects are the objects of $\C$ and where the hom objects are given by
  \[
    \Mell_\X\C(a,b)(x) = \C(a,x.b)\,.
    \]
  The composition is given by the natural transformation
  \begin{IEEEeqnarray*}{Cl}
    & (\Mell_\X\C(a,b) \tensor_{\text{Day}} \Mell_\X\C(b,c))(x)\\
    = & \int^{y,z\from \X} \C(a,y.b) \times \C(b,z.c) \times \X(y\tensor z,x) \\
    \to & \int^{y,z\from \X} \C(a,(y\tensor z).c) \times \X(y\tensor z,x) \\
    \cong & \C(a,x.c) \\
    = & \Mell_\X\C(a,c)(x)\,,
  \end{IEEEeqnarray*}
  where the arrow is induced by the \Mellies composition $\C(a,y.b) \times \C(b,z.c) \to \C(a,(y\tensor z).c)$.

  The identity transformation
  \[
    \C(I,x) \to \C(a,x.a)
    \]
  is the one sending a morphism $f\from I \to x$ to the composite
  \[
    a \xrightarrow{e_a}
    I.a \xrightarrow{f.a}
    x.a\,.
    \]
\end{definition}

\subsection{The category $\C/\X$}

We will mainly be concerned not with the \Mellies category itself, but with a closely related ordinary category.  
The usual method for turning a $\V$-enriched category into an ordinary category is via base change along a monoidal functor $\V \to \Set$: for example, any $\V$-enriched category has an \emph{underlying ordinary category}, obtained via base change along the functor $\V(I, \blank)$.  
For example, if $a$ and $b$ are objects of the base category $\C$, then the set of morphisms in the underlying ordinary category of the \Mellies category is the set of natural transformations
\[
  \C(I,x) \to \C(a,x.b)\,,
  \]
which by the Yoneda lemma is the same as the set $\C(a,I.b)$.  
In other words, this underlying ordinary category is precisely the Kleisli category for the monad on $\C$ given by the composite
\[
  1 \xrightarrow{I}
  \X \xrightarrow{\blank.\blank}
  \End[\C]\,;
  \]
in other words, the monad on $\C$ given by $Ma = I.a$.

This is not very useful for us, since it ignores most of the action and does not give us any new theory beyond the basic theory of monads and Kleisli categories.  
We want a monoidal functor $[\M,\Set]\to\M$ that will preserve more of the structure.  

Let us assume for now that $\X$ is a small category.  
Then we have a functor
\begin{IEEEeqnarray*}{cCc}
  [\X,\Set] & \to & \Set \\
  F & \mapsto & \colim_{x\from \X} F(x)
\end{IEEEeqnarray*}
given by the colimit in $\Set$.  

Moreover, this functor is lax monoidal, via the morphisms
\begin{IEEEeqnarray*}{Cl}
  & \left(\int^y F(y)\right)\times \left(\int^z G(z)\right) \\
  \to & \int^{y,z} F(y) \times G(z) \\
  \to & \int^{y,z} F(y) \times G(z) \times \X(y \tensor z,x) \\
  = & \int^x (F\tensor_{\text{Day}} G)(x)
\end{IEEEeqnarray*}
(where the second morphism sends picks out the inhabitant $\inj_{y\tensor z}(\id_{y\tensor z})$ of $\colim_{x\from \X}\X(y\tensor z,x)$)
and
\begin{IEEEeqnarray*}{cCc}
  1 & \to & \colim_{x\from \X} \X(I,x) \\
  () & \mapsto & \inj_I(\id_I)\,.
\end{IEEEeqnarray*}

Rather than pursue this line of argument to its conclusion, though, we choose to define the base-changed category directly.

\begin{definition}
  Let a monoidal category $\X$ act on a category $\C$ via a lax action.  
  Then we define a new category $\C/\X$ where
  \begin{itemize}
    \item the objects are the objects of $\C$,
    \item given objects $a,b$ of $\C$, the set of morphisms $a \to b$ is given by the colimit
      \[
        \colim_{x\from \X} \C(a,x.b)\,;
        \]
      i.e., a morphism in $\C/\X$ from $a$ to $b$ is an equivalence class of \Mellies morphisms
      \[
        a \to x.b
        \]
      in $\C$, where $x$ ranges over the objects of $\X$, and where the equivalence relation on morphisms is generated by identifying two morphisms $f\from a \to x.b$, $g \from a \to y.b$ if there is a morphism $h\from x\to y$ in $\X$ such that the following diagram commutes (we say that $h$ \emph{mediates} between $f$ and $g$);
      \[
        \begin{tikzcd}
          a \arrow[r, "f"] \arrow[dr, "g"']
            & x.b \arrow[d, "h.b"] \\
          %
            & y.b
        \end{tikzcd}
        \]
    \item composition of morphisms is via the \Mellies composition of Definition \ref{DefMelliesComposition} and
    \item the identity $a\to a$ is given by the equivalence class corresponding to the morphism
      \[
        e_a \from a \to I.a
        \]
      in $\C$.
  \end{itemize}
  \label{DefCX}
\end{definition}

\begin{remark}
  We will not in general assume that $\X$ is a small category, as we do above, but we will assume that the colimits in Definition \ref{DefCX} do exist in $\Set$.  
  If $\X$ is a small category, then they are small colimits so they exist automatically.
  If the colimits do not exist, then we get a category that is not locally small.
\end{remark}

\begin{proposition}
  $\C/\X$ is a well defined category.
\end{proposition}
\begin{proof}
  Let us first check that \Mellies composition is well defined with respect to our equivalence relation.  
  Suppose that
  \begin{gather*}
    f \from a \to x.b \\
    f' \from a \to x'.b \\
    g \from b \to y.c \\
    g' \from b \to y'.c
  \end{gather*}
  are \Mellies morphisms, and that there are morphisms $h \from x \to x'$, $k \from y \to y'$ in $\X$ such that the following diagrams commute.
  \begin{mathpar}
    \begin{tikzcd}
      a \arrow[r, "f"] \arrow[dr, "f'"']
        & x.b \arrow[d, "h.b"] \\
      %
        & x'.b
    \end{tikzcd}
    \and
    \begin{tikzcd}
      b \arrow[r, "g"] \arrow[dr, "g'"']
        & y.c \arrow[d, "k.c"] \\
      %
        & y'.c
    \end{tikzcd}
  \end{mathpar}
  Then we have the following commutative diagram, since $m$ is a natural transformation.
  \[
    \begin{tikzcd}
      a \arrow[r, "f"] \arrow[dr, "f'"']
        & x.b \arrow[r, "x.g"] \arrow[dr, "h.g" description] \arrow[d, "h.b" description]
          & x.y.c \arrow[r, "{m_{x,y,c}}"] \arrow[d, "h.y.c" description] \arrow[dd, bend left=50, "h.k.c"]
            & (x \tensor y).c \arrow[dd, "(h\tensor k).c"] \\
      %
        & x'.b \arrow[r, "x'.g" description] \arrow[dr, "x'.g'"']
          & x'.y.c \arrow[d, "x'.k.c" description]
            & \\
      %
        &
          & x'.y'.c \arrow[r, "{m_{x',y',c}}"]
            & (x' \tensor y').c
    \end{tikzcd}
    \]
  This proves that the \Mellies composition of $f$ and $g$ is equivalent to the \Mellies composition of $f'$ and $g'$.

  We should check that \Mellies composition is associative with respect to our equivalence relation.  
  Let $f \from a \to x.b$, $g\from b \to y.c$, $h \from c \to z.d$ be \Mellies morphisms.  
  Then we have the following commutative diagram.
  \[
    \begin{tikzcd}
      a \arrow[d, "f"' yshift=3pt] \arrow[d, dashed, thick]
        &
          &
            & \\
      x.b \arrow[d, "x.g"' yshift=3pt] \arrow[d, dashed, thick]
        &
          &
            & \\
      x.y.c \arrow[r, "x.y.h", thick, dashed] \arrow[d, "{m_{x,y,c}}"']
        & x.y.z.d \arrow[r, "x.m_{y,z,d}", thick, dashed] \arrow[d, "m_{x,y,z.d}"', dotted]
          & x.(y\tensor z).d \arrow[r, "m_{x,y\tensor z,d}", thick, dashed]
            & (x \tensor (y \tensor z)) . a \\
      (x \tensor y).c \arrow[r, "(x\tensor y).h"]
        & (x\tensor y).z.d \arrow[r, "m_{x\tensor y,z,a}"]
          & ((x\tensor y)\tensor z) . a \arrow[ur, "\assoc_{x,y,z}.d"', dotted]
            &
    \end{tikzcd}
    \]
  Here, the pentagon at the right is one of our conditions for a lax action, while the left-hand square commutes because $m$ is a natural transformation.  
  The composite given by the thick dashed lines is the \Mellies composition $f;(g;h)$, while that given by thin lines is the \Mellies composition $(f;g);h$.  
  The arrow $\assoc_{x,y,z}$ in $\X$ then mediates between these morphisms, so they are equivalent.

  Lastly, we need to check that the identity we have defined is indeed an identity for the composition.  
  The following diagrams show us that the morphism $\lunit_x$ in $\X$ mediates between a \Mellies morphism $f\from a \to x.b$ and the \Mellies composite $e_a;f$, and that the morphism $\runit_x$ in $\X$ mediates between $f$ and the \Mellies composite $f;e_b$.
  \begin{mathpar}
    \begin{tikzcd}
      a \arrow[r, "f"] \arrow[d, "e_a"']
        & x.b \arrow[d, "e_{x.b}"'] \arrow[dr, "\lunit_x.b"]
          & \\
      I.a \arrow[r, "I.f"]
        & I.x.b \arrow[r, "{m_{I,x,b}}" xshift=-3pt]
          & (I\tensor x).b
    \end{tikzcd}
    \and
    \begin{tikzcd}
      a \arrow[r, "f"]
        & x.b \arrow[r, "x.e_b"] \arrow[dr, "\runit_x.b"']
          & x.I.b \arrow[d, "{m_{x,I,b}}"] \\
      %
        &
          & (x\tensor I).b
    \end{tikzcd}
    \and
    \qedhere
  \end{mathpar}
\end{proof}

\begin{remark}
  This generalizes the Kleisli category of a monad.  
  Indeed, if $M$ is a monad on $\C$, then we may consider $M$ as a lax action of the unit category on $\C$.
  Then composition of morphisms $f \from a \to I.b$, $g \from b \to I.c$ in the category $\C/1$ is given by
  \[
    a \xrightarrow{f}
    I.b \xrightarrow{I.g}
    I.I.c \xrightarrow{m_{I,I,c}}
    I.c\,.
    \]
  Identifying $I.\blank$ with $M$, we see that this is precisely the definition of composition in the Kleisli category for $M$ (and the identity is the same thing too).  
  Since the unit category has only an identity morphism, the equivalence relation on morphisms in $\C/1$ is discrete, which is why we do not need to mention it in this context.
\end{remark}

There is an identity-on-objects functor $J \from \C \to \C/\X$ given by sending a morphism $f \from a \to b$ to the \Mellies morphism
\[
  a \xrightarrow{f}
  b \xrightarrow{e_b}
  I.b\,.
  \]
It is easy to show that this is a functor, and we will prove this in a more general context in Proposition \ref{PropGrotLaxLimit}.

A special role will be played by the identity morphisms
\[
  \id_{x.a} \from x.a \to x.a\,,
  \]
considered as \Mellies morphisms $x.a \to a$.  
We will see later that these are the components of a natural transformation $\phi_{x,a}\from x.a \to a$ in $\C/\X$, generalizing the natural transformation $\phi_a \from Ma \to a$ in the Kleisli category for a monad $M$ (see Proposition \ref{pKleisli}).

We will be using the following `factorization result'.  
If $f \from a \to x.b$ is a morphism in $\C$ then the corresponding \Mellies morphism $a \to b$ may be written as the following composite in $\C/\X$.
\[
  a \xrightarrow{Jf}
  x.b \xrightarrow{\phi_{x,b}}
  b\,.
  \]
Indeed, if we compute this composite in $\C$, then we get
\[
  a \xrightarrow{f}
  x.b \xrightarrow{e_{x.b}}
  I.x.b \xrightarrow{\id}
  I.x.b \xrightarrow{m_{I,x,b}}
  (I\tensor x).b\,,
  \]
which is equivalent to $f$ in $\C/\X$, because $e_{x.b};m_{I,x,b}=\lunit_x.b$, so $\lunit_x$ mediates between this morphism and $f$.

\subsection{Lax $2$-colimits}

In the previous section, we approached the category $\C/\X$ via the \Mellies category.  
In this section, we will show that $\C/\X$ has important properties in its own right: namely that it is a certain lax $2$-colimit in $\Cat$.  
We will briefly return to the \Mellies category in Section \ref{SecPromonads}, when we will show why it is a natural idea to consider an $[\X,\Set]$-enriched category.

\begin{definition}[\cite{StreetTwoConstructions}]
  Let $\C,\D$ be bicategories.  
  Then a \emph{lax functor} $F\from\C \to \D$ is given by
  \begin{itemize}
    \item a map $F$ from the objects of $\C$ to the objects of $\D$,
    \item for each pair $a,b$ of objects of $\C$, a functor $F$ from $\C(a,b)$ to $\D(F(a),F(b))$,
    \item for each triple $a,b,c$ of objects of $\C$, a transformation $m_{f,g}\from F(f);F(g) \to F(f;g)$ natural in $f\from a \to b$, $g\from b \to c$ and
    \item for each object $a$ of $\C$, a $2$-cell $e_a\from \id_{Fa} \Rightarrow F(\id_a)\from Fa \to Fa$
  \end{itemize}
  such that for all tuples $a,b,c,d$ of objects and all morphisms $f\from a \to b$, $g\from b \to c$, $h \from c \to d$, the following diagrams commute.
  \begin{mathpar}
    \begin{tikzcd}[column sep=60pt]
      (F(f);F(g));F(h) \arrow[r, "{\assoc_{F(f),F(g),F(h)}}"] \arrow[d, "{m_{f,g};F(h)}"']
        & F(f);(F(g);F(h)) \arrow[d, "{F(f);m_{g,h}}"] \\
      F(f;g);F(h) \arrow[d, "{m_{f;g,h}}"']
        & F(f);F(g;h) \arrow[d, "{m_{f,g;h}}"] \\
      F((f;g);h) \arrow[r, "{F(\assoc_{f,g,h})}"]
        & F(f;(g;h))
    \end{tikzcd}
    \and
    \begin{tikzcd}
      Ff \arrow[r, "\lunit_{Ff}"] \arrow[d, "F\lunit_f"']
        & \id_{Fa} ; Ff \arrow[d, "e_a;Ff"] \\
      F(\id_a;f)
        & F\id_a;Ff \arrow[l, "{m_{\id_a,f}}"']
    \end{tikzcd}
    \and
    \begin{tikzcd}
      Ff \arrow[r, "\runit_{Ff}"] \arrow[d, "F\runit_f"']
        & Ff ; \id_{Fb} \arrow[d, "Ff;e_b"] \\
      F(f;\id_b)
        & Ff;F\id_b \arrow[l, "{m_{f,\id_b}}"']
    \end{tikzcd}
  \end{mathpar}
\end{definition}
\begin{example}
  If $\C = \BB \X$, $\D = \BB \YY$ are the delooping bicategories of monoidal categories $\X, \YY$ -- i.e., $\X$ and $\YY$ considered as bicategories with a single object -- then a lax functor $\C \to \D$ is the same thing as a lax monoidal functor $\X \to \YY$.  

  More generally, a lax functor $F\from \BB \X \to \D$ is the same as a lax monoidal functor from $\X$ to the monoidal category of $1$-cells $F(*) \to F(*)$, where $*$ is the unique object of $\BB\X$.  
  In particular, a lax functor $1\to \Cat$ is the same thing as a monad and a lax functor $\BB\X \to \Cat$ is the same thing as a parametric monad parameterized by $\X$.
\end{example}

\begin{definition}
  Let $F \from \C \to \D$ be a lax functor of bicategories.  
  Then an \emph{oplax cocone under $F$} is given by an object $d$ of $\D$, together with $1$-cells
  \[
    l_c \from F(c) \to d
    \]
  for each object $c$ of $\C$ and $2$-cells
  \[
    \mu_h \from F(h);l_{c'} \Rightarrow l_c \from F(c) \to d
    \]
  for each $1$-cell $h\from c \to c'$ in $\C$, such that for all $2$-cells $\phi\from h' \Rightarrow h \from c \to c'$ the diagram
  \[
    \begin{tikzcd}
      F(h');l_{c'} \arrow[r, "\mu_{h'}", Rightarrow] \arrow[d, "F(\phi);l_{c'}"', Rightarrow]
        & l_c \\
      F(h);l_{c'} \arrow[ur, "\mu_h"', Rightarrow]
        &
    \end{tikzcd}
    \]
  commutes, and such that for all $h\from c \to c'$, $h' \from c' \to c''$, the diagrams
  \begin{mathpar}
    \begin{tikzcd}[column sep=60pt]
      (F(h);F(h'));l_{c''} \arrow[r, "{\assoc_{F(h),F(h'),l_{c''}}}", Rightarrow] \arrow[d, "{m_{h,h'};l_{c''}}"', Rightarrow]
        & F(h);(F(h');l_{c''}) \arrow[r, "F(h);\mu_{h'}", Rightarrow] 
          & F(h);l_{c'} \arrow[d, "\mu_h", Rightarrow] \\
      F(h;h');l_{c''} \arrow[rr, "\mu_{h;h'}", Rightarrow]
        &
          & l_c
    \end{tikzcd}
    \and
    \begin{tikzcd}
      l_c \arrow[r, "\lunit_{l_c}", Rightarrow] \arrow[d, "\id_{l_c}"', Rightarrow]
        & \id_c;l_c \arrow[d, "e_c;l_c", Rightarrow] \\
      l_c
        & F(\id_c);l_c \arrow[l, "\mu_{\id}"', Rightarrow]
    \end{tikzcd}
  \end{mathpar}
  commute.
\end{definition}

\begin{definition}[\cite{Gepner}]
  Let $F \from \C \to \D$ be a lax functor of bicategories.  
  Then a \emph{lax colimit} of $F$ is an oplax cocone $(u,k,\nu)$ under $F$ such that if $(d,l,\mu)$ is any other oplax cocone then there is a unique $1$-cell $\hat{l} \from u \to d$ such that $k_c;\hat{l}=l_c$ for each object $c$ of $\C$ and such that $\nu_h\hat{l}=\mu_h$ for any $1$-cell $h\from c \to c'$ in $\C$.
\end{definition}

For the justification of the terminology we have used whereby a \emph{lax} limit is a limiting \emph{oplax} cocone, see \cite{nlab:2-limit}.

\begin{remark}
  The definition of a lax colimit is normally weaker than the one we have given, in that the equations $k_c;\hat l = l_c$ and $\nu_h\hat l = \mu_h$ are not required to hold exactly, but instead up to coherent isomorphism.  
  For the lax functors we are constructing, however, we will be able to construct lax colimits that make these equations hold exactly.
\end{remark}

\subsection{Lax natural transformations and functoriality of lax colimits}

\begin{definition}[\cite{BasicBicategories}]
  Let $F,G \from \C \to \D$ be lax functors of bicategories.  
  A \emph{oplax natural transformation} $\C\to \D$ is given by a family of $1$-cells
  \[
    t_c \from F(c) \to G(c)\,,
    \]
  for each object $c$ of $\C$, together with a family of $2$-cells
  \[
    \mu_h \from F(h);t_{c'} \Rightarrow t_c ; G(h) \from F(c) \to G(c')
    \]
  for each morphism $h\from c \to c'$ in $\C$, such that for all $2$-cells $\phi\from h' \to h\from c \to c'$ the diagram
  \[
    \begin{tikzcd}[arrows=Rightarrow]
      F(h');t_{c'} \arrow[r, "\mu_{h'}"] \arrow[d, "F(\phi);t_{c'}"']
        & t_c;G(h') \arrow[d, "T_c;F(\phi)"] \\
      F(h);t_{c'} \arrow[r, "\mu_h"]
        & t_c;G(h)
    \end{tikzcd}
    \]
  commutes, and such that for all $h\from c\to c'$, $h'\from c' \to c''$, the diagrams
  \begin{mathpar}
    \begin{tikzcd}[column sep=34pt, arrows=Rightarrow]
      (F(h);F(h'));t_{c''} \arrow[r, "\assoc"] \arrow[d, "{m^F_{h,h'};t_{c''}}"']
        & F(h);(F(h');t_{c''}) \arrow[r, "F(h);\mu_{h'}"]
          & F(h);(t_{c'};G(h')) \arrow[d, "\assoc"] \\
      F(h;h');t_{c''} \arrow[d, "\mu_{h;h'}"']
        &
          & (F(h);t_{c'});G(h') \arrow[d, "\mu_h;G(h')"] \\
      t_c;G(h;h')
        & t_c;(G(h);G(h')) \arrow[l, "{t_c;m^G_{h,h'}}"]
          & (t_c;G(h));G(h') \arrow[l, "\assoc"]
    \end{tikzcd}
    \and
    \begin{tikzcd}[arrows=Rightarrow]
      l_c \arrow[r, "\lunit_{l_c}"] \arrow[d, "\runit_{l_c}"']
        & \id_c;l_c \arrow[dd, "e^F_c;l_c"] \\
      l_c;\id_c \arrow[d, "l_c;e^G_c"']
        & \\
      l_c;G(\id_c)
        & F(\id_c);l_c \arrow[l, "\mu_{\id_c}"]
    \end{tikzcd}
  \end{mathpar}
  commute.
\end{definition}

As we would expect, the horizontal/vertical composition of two oplax natural transformations is also an oplax natural transformation, and there is an identity oplax natural transformation $F \Rightarrow F$ for any lax functor $F$ of bicategories.

\begin{example}
  An oplax natural transformation between lax functors $\BB\X \to \Cat$ is the same thing as an oplax morphism between the corresponding actions, as defined in Definition \ref{DefOplaxMorphismOfActions}.
\end{example}
\begin{example}
  If $F\from \C \to \D$ is a lax functor of bicategories, then an oplax cocone under $F$ with tip $d\from \D$ is the same thing as an oplax natural transformation $F \to *_d$, where $*_d$ is the constant functor taking the value $d$.
\end{example}

We can then get a functoriality result similar to the usual one for ordinary colimits.

\begin{proposition}
  Let $F,G \from \C \Rightarrow \D$ be lax functors and let $t \from F \Rightarrow G$ be an oplax natural transformation.  
  Suppose that $F$ and $G$ have lax colimits $\tilde{F}$ and $\tilde{G}$.  
  Then $t$ naturally gives rise to a lax functor $\tilde{F} \to \tilde{G}$ in a way that respects composition and identity of lax natural transformations.
  \label{PropFunctorialityOfLaxColimits}
\end{proposition}
\begin{proof}
  Composing the limiting oplax cocone $k^G\from G \to *_{\tilde{G}}$ with $t$ gives us an oplax natural transformation $F \to *_{\tilde{G}}$ and hence a unique lax functor $(t;k^G)\hat{}\from \tilde{F} \to \tilde{G}$ such that $k^F_c;(t;k^G)\hat{}=t_c;k^G_c$ and $\nu^F_h(t;k^G)\hat{}=\nu^G_h$ for all objects $c$ and morphisms $h$ in $\C$.  

  Preservation of composition and identities is via the usual technique: if we have oplax natural transformations $t\from F \Rightarrow G$ and $t' \from G \Rightarrow H$, then the composite of the induced lax functors $\tilde{F} \to \tilde{G}$ and $\tilde{G}\to \tilde{H}$ satisfies the same property that uniquely defines the lax functor $\tilde{F} \to \tilde{H}$ induced from the composite of $t$ and $t'$.
\end{proof}

Another similar functoriality result tells us what happens to lax colimits when we precompose with a lax functor.

\begin{proposition}
  Let $F \from \C \to \D$, $G \from \D \to \E$ be lax functors.  
  Suppose that lax colimits exist for $G$ and for $F;G$.  
  Then we get a natural lax functor $\hat F$ from the lax colimit of $F;G$ to the lax colimit of $G$ such that for all objects $c$ and all morphisms $h$ in $\C$ we have
  \begin{mathpar}
    k^{F;G}_c;\hat{F} = k^g_{Fc}
    \and
    \nu^{F;G}_h \hat{F} = \nu^G_{Fh}\,,
  \end{mathpar}
  where $(k^{F;G},\nu^{F;G})$ is the limiting cocone for the lax colimit of $F;G$ and $(k^G,\nu^G)$ is the limiting cocone for the lax colimit of $G$.
  \label{PropHorizFunctorialityOfLaxColimits}
\end{proposition}
\begin{proof}
  Composing the limiting cocone $(k^G_d,\nu^G_k)$ for the lax colimit of $G$ with $F$ gives us an oplax cocone $(k^G_{Fc},\nu^G_{Fh})$ under $F;G$, inducing a functor from the lax colimit of $F;G$ to the lax colimit of $G$ that satisfies the required properties.
\end{proof}

\subsection{Lax $2$-colimits in $\Cat$}

Let $\C$ be a bicategory, and let $F\from \C \to \Cat$ be a bifunctor.

\begin{definition}[\cite{FibrationsInBicategories}]
  The \emph{Grothendieck construction} associates to the bifunctor $F$ a bicategory $\int F$, where
  \begin{itemize}
    \item the objects are pairs $(a, m)$, where $a$ is an object of $\C$ and $m$ an object of $F(a)$,
    \item the $1$-cells $(a, m) \to (b, n)$ are pairs $(h, f)$, where $h$ is a morphism $b \to a$ and $f$ is a morphism $m \to F(h)(n)$ and
    \item the $2$-cells $(h, f) \Rightarrow (k, g) \from (a,m) \to (b,n)$ are $2$-cells $\phi\from h \Rightarrow k$ in $\C$ that make the following diagram commute.
      \[
        \begin{tikzcd}
          m \arrow[r, "f"] \arrow[dr, "g"']
            & F(h)(n) \arrow[d, "(F\phi)_n"] \\
          %
            & F(k)(n)
        \end{tikzcd}
        \]
  \end{itemize}
  The identity $1$-cell $(a,m)\to(a,m)$ is given by $\left(\id_a,m \xrightarrow{(e_a)_m} F(\id)(m)\right)$.
  The composite of a $1$-cell $(h,f)\from (a,m) \to (b,n)$ with a $1$-cell $(k,g)\from (b,n) \to (c,p)$ is the pair $(k;h,f*g)$, where $f*g$ is the following composite.
  \[
    m \xrightarrow{f}
    F(h)(n) \xrightarrow{F(h)(g)}
    F(h)(F(k)(n)) \xrightarrow{m_{h,k}}
    F(k;h)(n)
    \]
\end{definition}
\begin{remark}
  For the Grothendieck construction see \cite[VI.8]{SGA1} and \cite[B1.3.1]{Elephant} -- in the original formulation, the category $F$ is a pseudofunctor $\C \to \Cat$ (i.e., a lax functor where the coherences $m$ and $e$ are isomorphisms), where $\C$ is an ordinary category -- then $\int F$ is an ordinary category.
  For the version where $\C$ is an arbitrary bicategory (and $\int F$ is a bicategory), see \cite{FibrationsInBicategories}.

  There are many different variations of this construction, where various combinations of arrows are reversed.  
  We have chosen the variation that suits our needs.
\end{remark}

\begin{definition}
  Given a bicategory $\C$, we write $\pi_*\C$ for the ordinary category whose objects are the objects of $\C$ and where the morphisms $a\to b$ are the connected components of the $1$-cells $a\to b$ in $\C$: i.e., equivalence classes of $1$-cells $a\to b$ under the equivalence relation generated by relating $f\from a \to b$ to $g \from a \to b$ if there is a $2$-cell $\phi \from f \to g$.
\end{definition}

\begin{proposition}[{\cite[\sec 3]{nlab:2-limit}}]
  Let $\C$ be a bicategory and let $F\from \C \to \Cat$ be a lax functor.  
  Then $\pi_*\int F$ is the lax colimit of $F$.
  \label{PropGrotLaxLimit}
\end{proposition}
\begin{proof}
  First, we define the lax cocone under $F$ with tip $\pi_*\int F$.
  Given an object $a$ of $\C$, we have a functor $k_a \from F(a) \to \pi_*\int F$ given by
  \begin{mathpar}
    k_a(m) = (a,m)
    \and
    k_a\left(m \xrightarrow{f} n\right) = \left(\id_a,m\xrightarrow{f}n\xrightarrow{(e_a)_n} F(\id)(n)\right)\,.
  \end{mathpar}
  We should check that this is indeed a functor; clearly it preserves the identity, so we need to check that it preserves composition.  
  Let $m\xrightarrow{f}n\xrightarrow{g}p$ be morphisms in $F(a)$.  
  We need to show that $k_a(f);k_a(g) = k_a(f;g)$ in $\pi_*\int F$, for which it suffices to exhibit a $2$-cell in $\int F$ mediating between these two $1$-cells, and indeed such a $2$-cell is given by $\lunit_{\id_a}\from \id_a \to \id_a;\id_a$ (see Figure \ref{FigKaFG=KaFKaG}).
  \begin{figure}
    \[
      \begin{tikzcd}[column sep=45pt]
        m \arrow[r, thick, dashed, "f"] \arrow[r]
          & n \arrow[r, thick, dashed, "(e_a)_n"] \arrow[d, "g"']
            & F(\id_a)(n) \arrow[d, thick, dashed, "F(\id_a)(g)"]
              & \\
        %
          & p \arrow[r, "(e_a)_p"]
            & F(\id_a)(p) \arrow[r, thick, dashed, "F(\id_a)(e_p)" xshift=-3pt] \arrow[dr, dotted, "F(\lunit_{\id_a})_p"']
              & F(\id_a)((F(\id_a)(p))) \arrow[d, thick, dashed, "{m_{\id_a,\id_a}}"] \\
        %
          &
            &
              & F(\id_a,\id_a)(p)
      \end{tikzcd}
      \]
    \caption[The composite $k_a(f);k_a(g)$ is equal to $k_a(f;g)$ in $\pi_*\int F$.]{The square commutes because $(e_a)$ is a natural transformation, while commutativity of the triangle is one of the conditions for a lax functor.  
The composite $k_a(f);k_a(g)$ is given by the pair $\left(\id_a;\id_a, {\begin{tikzcd}[ampersand replacement=\&]m \arrow[r, thick, dashed] \& F(\id_a,\id_a)(p)\end{tikzcd}}\right)$ (thick dashed arrows), while $k_a(f;g)$ is given by $\left(\id_a,m \to F(\id_a)(p)\right)$ (normal arrows). The $2$-cell $\lunit_{\id_a}\from \id_a\to\id_a$ from $\int F$ mediates between them (dotted arrow).}
      \label{FigKaFG=KaFKaG}
  \end{figure}

  Next, if $h\from a \to a'$ is a morphism in $\C$, then we define a natural transformation
  \[
    \nu_h \from F(h);k_{a'} \Rightarrow k_a
    \]
  by
  \[
    (\nu_h)_m = (h, \id_{F(h)(m)}) \from (a',F(h)(m)) \to (a,m)\,.
    \]
  We want to show that this is a natural transformation; i.e., that for any morphism $f \from m \to n$ in $F(a)$, the following diagram commutes.
  \[
    \begin{tikzcd}
      k_{a'}(F(h)(m)) \arrow[r, "(\nu_h)_m"] \arrow[d, "k_{a'}(F(h)(f))"']
        & k_a(m) \arrow[d, "k_a(f)"] \\
      k_{a'}(F(h)(n)) \arrow[r, "(\nu_h)_n"]
        & k_a(n)
    \end{tikzcd}
    \]
  Here, the top right composite is given by
  \[
    F(h)(m) \xrightarrow{F(h)(f)}
    F(h)(n) \xrightarrow{F(h)((e_{a})_n)}
    F(h)(F(\id_a)(n)) \xrightarrow{m_{h,\id_a}}
    F(h;\id_a)(n)\,,
    \]
  which is related to $F(h)(f)$ via $\runit_h$, while the bottom left composite is given by
  \[
    F(h)(m) \xrightarrow{(Fh)f}
    F(h)(n) \xrightarrow{(e_{a'})_{(Fh)f}}
    F(\id_{a'})(F(h)(n)) \xrightarrow{m_{\id_{a'},h}}
    F(\id_{a'};F(h))(n)\,,
    \]
  which is related to $F(h)(f)$ via $\lunit_h$.

  We now need to show that the appropriate diagrams commute to ensure that $\left(\pi_*\int F,k,\nu\right)$ is a lax cocone under $F$.
  For the first diagram, we must show that the following commutes for any $2$-cell $\phi\from h' \Rightarrow h\from a \to a'$ and any object $m$ of $F(a)$.
  \[
    \begin{tikzcd}
      k_{a'}(F(h')(m)) \arrow[r, "(\nu_{h'})_m"] \arrow[d, "k_{a'}((F\phi)_m)"']
        & k_a(m) \\
      k_{a'}(F(h)(m)) \arrow[ur, "(\nu_h)_m"']
        &
    \end{tikzcd}
    \]
  Since the arrow along the top is given by the identity on $F(h')(m)$, it will suffice to show that the arrow along the bottom is of the form $F(\psi)_m$, for some $2$-cell $\psi$ in $\C$, and is therefore equal to the identity in $\pi_*\int F$.
  Indeed, writing this composite out in full, we get
  \[
    F(h')(m) \xrightarrow{(F\phi)_m}
    F(h)(m) \xrightarrow{(e_{a'})_{F(h)(m)}}
    F(\id)(F(h)(n)) \xrightarrow{m_{\id,h}}
    F(\id;h)(n)\,.
    \]
  By the conditions on a lax functor, the composite of the last two morphisms is $F(\lunit_h)_m$, and so the whole thing is equal to $F(\phi;\lunit_h)_m$.

  Lastly, we need to show that the following diagrams commute for any $1$-cells $h\from a \to a'$, $h'\from a' \to a''$ and any object $m$ of $F(a)$.
  \begin{mathpar}
    \begin{tikzcd}[column sep=50pt]
      k_{a''}(F(h')(F(h)(m))) \arrow[r, "(\nu_h)_{F(h)(m)}"] \arrow[d, "{k_{a''}((m_{h,h'})_m)}"']
        & k_{a'}(F(h)(m)) \arrow[d, "(\nu_{h})_m"] \\
      k_{a''}(F(h;h')(m)) \arrow[r, "(\nu_{h;h'})_m"]
        & k_a(m)
    \end{tikzcd}
    \and
    \begin{tikzcd}[column sep=40pt]
      k_a(m) \arrow[r, "k_a((e_a)_m)"] \arrow[dr, "\id_{k_a(m)}"']
        & k_a(F(\id)(m)) \arrow[d, "(\nu_{\id})_m"] \\
      %
        & k_a(m)
    \end{tikzcd}
  \end{mathpar}
  This time, we can compute that both composites in the first diagram are equal to
  \[
    F(h')(F(h)(m)) \xrightarrow{m_{h,h'}} F(h;h')(m)\,,
    \]
  while in the second diagram the top left composite is equal to
  \[
    m \xrightarrow{(e_a)_m}
    F(\id)(m) \xrightarrow{(e_a)_{F(\id)(m)}}
    F(\id)(F(\id)(m)) \xrightarrow{\mu_{\id,\id}}
    F(\id;\id)(m)\,,
    \]
  and the diagonal arrow is given by
  \[
    m \xrightarrow{(\e_a)_m}
    F(\id)(m)\,.
    \]
  These $1$-cells in $\int F$ are related by $F(\lunit_m)$, so they correspond to the same morphism in $\pi_*\int F$.

  This completes the definition of the universal cocone under $\pi_*\int F$.

  Now, suppose that $(\D,l,\mu)$ is another lax cocone under $F$.  
  We define a functor $\hat{l}$ from $\pi_*\int F$ to $\D$ by
  \begin{mathpar}
    \hat{l}(a,m) = l_a(m)
    \and
    \hat{l}\left((a,m) \xrightarrow{(h,f)} (b,n)\right) =
    l_a(m) \xrightarrow{l_a(f)}
    l_a(F(h)(n)) \xrightarrow{(\mu_h)_n}
    l_b(n)\,.
  \end{mathpar}
  We need to show that $\hat{l}$ is a functor.  
  First, observe that it sends the identity on $(a,m)$ to the composite
  \[
    l_a(m) \xrightarrow{l_a((e_a)_m)}
    l_a(F(\id)(m)) \xrightarrow{(\mu_{\id})_m}
    l_a(m)\,,
    \]
  which is equal to the identity on $l_a(m)$ since $l$ is a cocone under $F$.
  Now suppose we have morphisms
  \[
    (a,m) \xrightarrow{(h,f)}
    (b,n) \xrightarrow{(k,g)}
    (c,p)\,.
    \]
  Using the formula above for the composition of morphisms in the Grothendieck construction, we see that $\hat{l}((h,f);(k;g))$ is given by the thick dotted composite in Figure \ref{FigLHatFunctor}, while $\hat{l}(h,f);\hat{l}(k,g)$ is given by the thin composite.  
  Therefore, these two are equal.
  \begin{figure}
    \[
      \begin{tikzcd}[column sep=40pt]
        l_a(m) \arrow[r, "l_a(f)"] \arrow[r, thick, dashed]
          & l_a(F(h)(n)) \arrow[r, "(\mu_h)_n"] \arrow[d, "l_a(F(h)(g))"', thick, dashed]
            & l_b(n) \arrow[d, "l_b(g)"] \\
        %
          & l_a(F(h)(F(k)(p))) \arrow[r, "(\mu_h)_{F(k)(p)}", dotted] \arrow[d, "l_a((m_{k,h})_p)"', thick, dashed]
            & l_b(F(k)(p)) \arrow[d, "(\mu_k)_p"] \\
        %
          & l_a(F(k;h)(p)) \arrow[r, "(\mu_{k;h})_p", thick, dashed]
            & l_c(p)
      \end{tikzcd}
      \]
    \caption{Proof that $\hat{l}$ respects composition.  
    The thick dotted line represents $\hat{l}((h,f);(k,g))$, while the thin solid line represents $\hat{l}(h,f);\hat{l}(k,g)$.  
    The top square commutes because $\mu_h$ is a natural transformation, while the bottom is one of the conditions on a cocone.}
    \label{FigLHatFunctor}
  \end{figure}

  Therefore, $\hat{l}$ is a functor.  
  Moreover, if $a$ is an object of $\C$ and $m$ and object of $F(a)$, we have $\hat{l}(k_a(m))=\hat{l}(a,m)=l_a(m)$, and if $f\from m \to n$ is a morphism in $F(a)$, then $\hat{l}(k_a(f))$ is given by the composite
  \[
    l_a(m) \xrightarrow{l_a(f)}
    l_a(n) \xrightarrow{l_a((e_a)_n)}
    l_a(F(\id)(n)) \xrightarrow{(\mu_{\id})_n}
    l_a(n)\,,
    \]
  which is equal to $l_a(f)$ by the cocone condition on $\mu$.
  Therefore, $k_a;\hat{l}=l_a$ for each object $a$ of $\C$.

  Lastly, if $h\from a \to b$ is a morphism in $\C$, then $\hat{l}((\nu_h)_m)=\hat{l}(h,\id_{F(h)(m)}) = (\mu_h)_m$, and so $\nu_hl = \mu_h$.  
  This completes the existence part of the proof.

  For uniqueness, suppose that $j$ is another functor $\pi_*\int F \to \D$ such that $k_a;j=l_a$ and $\nu_hj=\mu_h$ for each object $a$ of $\C$ and each morphism $h\from a \to b$ in $\C$.

  Let $(a,m)$ be an object of $\pi_*\int F$.  
  Then $j(a,m)=j(k_a(m))=l_a(m)=\hat{l}(a,m)$.  

  Now let $(h,f)\from (a,m) \to (b,n)$ be a morphism in $\pi_*\int F$.  
  We claim that we may decompose $(h,f)$ as the composite
  \[
    \begin{tikzcd}
      (a,m) \arrow[d, "k_a(f)"'] \arrow[dr, "{(h,f)}"]
        & \\
      (a,F(h)(n)) \arrow[r, "(\nu_h)_n"']
        & (b,n)
    \end{tikzcd}\,.
    \]
  Indeed, if we work out what that composite is, then we get
  \[
    m \xrightarrow{f}
    F(h)(n) \xrightarrow{(e_a)_n}
    F(\id)(F(h)(n)) \xrightarrow{(m_{\id,h})_n}
    F(\id;h)(n)\,,
    \]
  which is related to $(h,f)$ in $\int F$ by the $2$-cell $\lunit_h$ (since $F$ is a lax functor), and is therefore equal to $(h,f)$ in $\pi_*\int F$.

  We therefore have
  \[
    j(h,f) = j(k_a(f);(\nu_h)_n) = j(k_a(f));j((\nu_h)_n) = l_a(f);(\mu_h)_n = \hat{l}(h, f)\,,
    \]
  and therefore $j=\hat{l}$.  
\end{proof}

\subsection{Examples of lax $2$-colimits in $\Cat$}

Suppose that $M$ is a monad on a category $\C$, considered as a lax functor
\[
  F\from 1\to \Cat
  \]
such that $F(*)=\C$.

If we apply the Grothendieck construction to $F$, then the category we get has pairs $(*,a)$ for objects, where $a$ ranges over the objects of $\C$, and the morphisms from $(*,a)$ to $(*,b)$ are morphisms $a \to F(\id)(b)$ in $\C$, where $F(\id)=M$.  
In other words, $\int F$ is precisely the Kleisli category for $M$.
In this case, Proposition \ref{PropGrotLaxLimit} reduces to Proposition \ref{pKleisli}.

More generally, if $\X$ is a monoidal category and $F\from \BB \X \to \Cat$ is a lax functor, corresponding to a lax action of $\X$ upon $\C = F(*)$, then it is easy to see that $\pi_*\int F$ is isomorphic to the category $\C/\X$ that we defined earlier.  
Indeed, in both cases the objects may be identified with the objects of $\C$, and the morphisms from an object $a$ to an object $b$ may be written as equivalence classes of morphisms $a \to x.b$ in $\C$ under the equivalence relation generated by relating $f\from a \to x.b$ to $g\from a \to y.b$ if there is a morphism $h\from x \to y$ in $\X$ such that $g=f;(h.b)$.

In this case, we can recast Proposition \ref{PropGrotLaxLimit} to get a result about $\C/\X$.

\begin{corollary}
  Let $\blank.\blank$ be an action of a monoidal category $\X$ on a category $\C$.  
  If we identify this action with a lax functor $F \from \BB \X \to \Cat$, then $\C/\X$ is the lax colimit of this functor.

  In other words, there is a functor $J\from \C \to \C/\X$ and a natural transformation
  \[
    \phi_{x,a}\from J(x.a) \to Ja
    \]
  making the following diagrams commute
  \begin{mathpar}
    \begin{tikzcd}
      J(x.y.a) \arrow[r, "{\phi_{x,y.a}}"] \arrow[d, "{J(m_{x,y,a})}"']
        & J(y.a) \arrow[d, "{\phi_{y,a}}"] \\
      J((x\tensor y).a) \arrow[r, "{\phi_{x\tensor y,a}}"]
        & Ja
    \end{tikzcd}
    \and
    \begin{tikzcd}
      Ja \arrow[dr, "\id"] \arrow[d, "J(\lun_a)"']
        & \\
      J(I.a) \arrow[r, "{\phi_{I,a}}" xshift=-4pt]
       & Ja
    \end{tikzcd}\,,
  \end{mathpar}
  such that if $F\from \C \to \D$ is a functor and
  \[
    \psi_{x,a} \from F(x.a) \to Fa
    \]
   is a natural transformation making the following diagrams commute
  \begin{mathpar}
    \begin{tikzcd}
      F(x.y.a) \arrow[r, "{\psi_{x,y.a}}"] \arrow[d, "{F(m_{x,y,a})}"']
        & F(y.a) \arrow[d, "{\psi_{y,a}}"] \\
      F((x\tensor y).a) \arrow[r, "{\psi_{x\tensor y,a}}"]
        & Fa
    \end{tikzcd}
    \and
    \begin{tikzcd}
      Fa \arrow[dr, "\id"] \arrow[d, "F(\lun_a)"']
        & \\
      F(I.a) \arrow[r, "{\psi_{I,a}}" xshift=-4pt]
       & Fa
    \end{tikzcd}\,,
  \end{mathpar}
  then there is a unique functor $\hat{F}\from \C/\X\to \D$ such that $F=\hat{F}J$ and $\psi=\hat{F}\phi$.
  \label{CorTheConstructionUniversalProperty}
\end{corollary}

Corollary \ref{CorTheConstructionUniversalProperty} tells us that $\C/\X$ is generated as a category by the objects and morphisms in $\C$, together with the special morphisms making up the natural transformation $\phi_{x,a}\from x.a \to a$.  

This is significant, because it tells us that the category $\C/\X$ always satisfies a `factorization result' akin to those in \cite{SamsonGuyIAActive} and \cite{mcCHFiniteND}: if $f \from a \to b$ is a morphism in $\C/\X$, given by some morphism $\tilde{f}\from a \to x.b$ in $\C$, then we may write $f$ as the composite
\[
  a \xrightarrow{J(\tilde{f})} x.b \xrightarrow{\phi_{x,b}} b
  \]
in $\C$.  
This means that if we have a definability result for the category $\C$ -- for example, that every compact morphism in $\C$ is definable in some language $\L$ -- then we can automatically get a definability result for the category $\C/\X$ -- for example, that every compact morphism in $\C/\X$ (i.e., every morphism $a\to b$ in $\C/\X$ that is given by a compact morphism $a \to x.b$ in $\C$) is definable in the language $\L + \Phi_{x,a}$, where $\Phi_{x,a}$ is a new family of primitives in the language whose denotations are given by the $\phi_{x,a}$.
This is the same purpose that we used Proposition \ref{pKleisli} for when proving Theorem \ref{TheKleisliFullAbstraction}.

\subsection{Finite products distribute over lax colimits in $\Cat$}

The next result mirrors the result for distributivity of finite products over arbitrary colimits in a Cartesian closed category (essentially because the functor $A \times \blank$ is a left adjoint, so preserves colimits).

\begin{proposition}
  Let $F \from \C \to \Cat$ be a lax functor of bicategories.  
  Let $\A$ be a category.
  Then the lax colimit of $\A \times F$ (defined by $(\A \times F)(c) = \A \times F(c)$ and $(\A \times F)(f) = (\id,f)$) is given by the product of $\A$ with the lax colimit of $F$.
  \label{PropGrotCommutesProducts}
\end{proposition}
\begin{proof}
  It is easiest to compute the colimit of $\A \times F$ directly using Proposition \ref{PropGrotLaxLimit}. 
  The objects are tuples $(c,(a,m))$, where $c$ is an object of $\C$, $a$ is an object of $a$ and $m$ is an object of $F(c)$.
  $1$-cells from $(c',(a',m'))$ to $(c,(a,m))$ are given by pairs $(h,(q,f))$, where $h$ is a morphism $c \to c'$ in $\C$, $f$ is a morphism $a' \to F(h)(a)$ in $F(a')$ and $q$ is a morphism $a'\to a$ in $\A$.
  Two $1$-cells $(h,(q,f))$ and $(h',(q',f'))$ from $(c',(a',m'))$ to $(c,(a,m))$ define equivalent morphisms if there is a $2$-cell $\phi\from h \Rightarrow h'$ such that $f;(F\phi)_m=f'$, and if $a'=a$.  

  Meanwhile, the product of $\A$ with the lax colimit of $F$ has tuples $(a,(c,m))$ for objects, where $a$ is an object of $\A$, $c$ an object of $\C$ and $m$ an object of $F(c)$.
  $1$-cells $(a',(c',m')) \to (a,(c,m))$ are given by tuples $(q,(h,f))$, where $q$ is a morphism $a' \to a$, $h$ is a morphism $c \to c'$ and $f$ is a morphism $m' \to F(h)(m)$.
  Two $1$-cells $(q,(h,f))$ and $(q',(h',f'))$ from $(a',(c',m'))$ to $(a,(c,m))$ define equivalent morphisms if $a'=a$, and if there is a $2$-cell $\phi\from h\Rightarrow h'$ such that $f;(F\phi)_m=f'$. 
  These two categories are clearly isomorphic.
\end{proof}

\begin{corollary}
  Let $F\from \C\to\Cat$, $G\from \D \to \Cat$ be lax functors of bicategories.  
  Then
  \[
    \laxcolim_{c\from\C,d\from\D} F(c)\times G(d) \cong \laxcolim_{c\from \C} F(c) \times \laxcolim_{d\from\D} G(d)\,.
    \]
  \label{CorGrotBothSidesProductPreservation}
\end{corollary}
\begin{proof}
  By Proposition \ref{PropGrotCommutesProducts}, we have
  \begin{IEEEeqnarray*}{Cl}
    & \laxcolim_{c\from \C} F(c) \times \laxcolim_{d\from\D} G(d) \\
    \cong &
    \laxcolim_{c\from \C} (F(c) \times \laxcolim_{d\from\D} G(d)) \\
    \cong &
    \laxcolim_{c\from \C} \laxcolim_{d\from\D} (F(c)\times G(d)) \\
    \cong  &
    \laxcolim_{c\from\C,d\from\D} (F(c)\times G(d))\,.\qedhere
  \end{IEEEeqnarray*}
\end{proof}

In the particular case of a bifunctor out of $\BB\X$, for $\X$ a monoidal category, we get the following.

\begin{corollary}
  Suppose we have actions of a monoidal category $\X$ on a category $\C$ and of a monoidal category $\YY$ on a category $\D$.
  We get an action of $\X\times\YY$ on $\C\times \D$ by
  \[
    (x,y).(c,d) = (x.c,y.d)\,.
    \]
  Then
  \[
    (\C\times\D)/(\X\times\YY)\cong (\C/\X) \times (\D/\YY)\,.
    \]
  \label{CorQuotProductPreservation}
\end{corollary}
\begin{proof}
  Since the $\C/\X$ construction is a special case of a lax colimit in $\Cat$, this is a direct application of Corollary \ref{CorGrotBothSidesProductPreservation} where we are using the fact that $\BB(\X\times\YY)\cong \BB\X \times \BB\YY$.
\end{proof}

\subsection{Monoidal structure of $\C/\X$}

\begin{definition}
  Suppose that $\X$ is a symmetric monoidal category acting on a monoidal category $\C$, and suppose moreover that the underlying functor $\X\times\C \to \C$ is a lax monoidal functor.  
  
  Then the functor $\X\times\X\times\C\to \C$ that sends $(x,y,a)$ to $x.y.a$ is lax monoidal.  
  Since $\X$ is symmetric monoidal, the tensor product $\blank\tensor\blank\from \X \times\X \to \X$ is a strong monoidal functor and so the functor $\X \times \X \times \C \to \C$ that sends $(x,y,a)$ to $(x\tensor y).a$ is lax monoidal.

  We say that the action of $\X$ on $\C$ is \emph{monoidal} if, in addition to the underlying functor $\X \times \C \to \C$ being monoidal, the natural transformations $m_{x,y,a}$ and $e_a$ are monoidal natural transformations.
  \label{DefMonoidalLaxAction}
\end{definition}

\begin{example}
  If $\X,\C$ are symmetric monoidal categories, where $\C$ is symmetric monoidal closed, and $j\from \X \to \C$ is an oplax symmetric monoidal functor, then the parametric reader monad action
  \[
    x.a = jx\implies a
    \]
  is a symmetric monoidal lax action of $\oppcat\X$ on $\C$.
\end{example}

\begin{remark}
  This generalizes the definition of a monad being \emph{monoidal}; i.e., when the underlying functor $\C \to \C$ is a monoidal functor and the multiplication and unit for the monad are monoidal natural transformations.  
  However, although a monoidal monad can be defined to be a monoid in the category of monoidal functors $\C \to \C$ and monoidal natural transformations between them, a monoidal parametric monad parameterized by a symmetric monoidal category $\X$ cannot be defined as a monoidal functor from $\X$ into this category.  
  Indeed, that would mean we had a natural transformation
  \[
    x.a \tensor x.b \to x.(a\tensor b)\,,
    \]
  whereas we want to specify that there should be a natural transformation
  \[
    x.a \tensor y.b \to (x\tensor y).(a\tensor b)\,.
    \]
\end{remark}

\begin{proposition}
  Suppose that $\X$ is a symmetric monoidal category, and that we have a symmetric monoidal lax action of $\X$ on another monoidal category $\C$.  
  Then $\C/\X$ inherits the structure of a monoidal category and the natural functor $J\from \C \to \C/\X$ is strict monoidal.
\end{proposition}
\begin{proof}
  By taking the product of the action with itself, we get an action of $\X\times\X$ on $\C\times\C$; i.e., the action given by $(x,y).(a,b)=(x.a,y.b)$.  
  Since $\X$ is symmetric monoidal, the tensor product functor $\X\times\X \to \X$ is strong monoidal, so by composing it with the action of $\X$ on $\C$ we get a lax action of $\X\times\X$ on $\C$; i.e., the action given by $(x,y).a = (x\tensor y).a$, with coherence $m_{(x,y),(x',y').a}$ given by
  \[
    (x\tensor y).(x'\tensor y').a \xrightarrow{m_{(x\tensor y),(x'\tensor y'),a}}
    ((x\tensor y)\tensor (x'\tensor y')).a \to
    ((x\tensor x')\tensor (y\tensor y')).a\,,
    \]
  where the last arrow is given by the unique symmetric monoidal isomorphism.

  We claim that the functor $\blank\tensor\blank\from \C\times\C\to\C$ and the monoidal coherence
  \[
    m_{x,y,a,b} \from x.a \tensor y.b \to (x\tensor y).(a\tensor b)
    \]
  of the monoidal functor $\blank.\blank\from \X \times \C \to \C$ give rise to an oplax morphism from the action of $\X\times\X$ on $\C\times\C$ to the action of $\X\times\X$ on $\C$.

  Indeed, the first diagram in Definition \ref{DefOplaxMorphismOfActions} in this case is given by
  \[
    \begin{tikzcd}
      (x.y.a)\tensor(x'.y'.a') \arrow[r, "{m_{x,x',y.a,y'.a'}}"] \arrow[d, "{m_{x,y,a}\tensor m_{x',y',a'}}"']
        & (x\tensor x').(y.a \tensor y'.a') \arrow[d, "{(x\tensor x').m_{y,y',a,a'}}"] \\
      ((x\tensor y).a)\tensor ((x'\tensor y').a') \arrow[d, "{m_{x\tensor y,x'\tensor y',a,a'}}"']
        & (x \tensor x').(y\tensor y').(a \tensor a') \arrow[d, "{m_{x\tensor x',y\tensor y',a\tensor a'}}"] \\
      ((x\tensor y)\tensor (x'\tensor y')).(a\tensor a') \arrow[r]
        & ((x \tensor x')\tensor (y \tensor y')) . (a \tensor a') \\
    \end{tikzcd}\,,
    \]
  which is precisely the diagram saying that $m$ is a monoidal natural transformation.  
  Similarly, the second diagram from Definition \ref{DefOplaxMorphismOfActions} is the same as the diagram saying that $e$ is a monoidal natural transformation.  

  Therefore, by Proposition \ref{PropFunctorialityOfLaxColimits}, this lax morphism of actions gives rise to a functor
  \[
    (\C\times\C)/(\X\times\X) \to \C/(\X\times\X)\,.
    \]
  The lax monoidal functor $\X \times \X \to \X$ gives rise by Proposition \ref{PropHorizFunctorialityOfLaxColimits} to a functor $\C/(\X\times\X) \to \C/\X$, which we may compose with the functor above to give us a functor
  \[
    (\C\times\C)/(\X\times\X) \to \C/\X\,.
    \]
  Moreover, Corollary \ref{CorQuotProductPreservation} tells us that $(\C\times\C)/(\X\times\X)$ is isomorphic to $(\C/\X)\times(\C/\X)$, giving us our desired functor
  \[
    \blank\tensor\blank \from (\C/\X)\times(\C/\X) \to (\C/\X)\,.
    \]
  Moreover, the construction of this functor tells us that it commutes with the identity-on-objects functors out of the original categories:
  \[
    \begin{tikzcd}
      \C \times \C \arrow[r, "\blank\tensor\blank"] \arrow[d, "J\times J"]
        & \C \arrow[d, "J"] \\
      (\C/\X)\times (\C/\X) \arrow[r, "\blank\tensor\blank"]
        & \C/\X
    \end{tikzcd}\,,
    \]
  which means that the functor $J$ preserves the tensor product.
  This also means that we may lift the associators and unitors for the tensor product on $\C$ along the functor $J$ morphisms in $\C/\X$, and that these morphisms satisfy the appropriate coherence diagrams.
  It remains to show that they are still natural transformations in $\C/\X$, which we do in Figure \ref{FigAssocUnitorNaturalTransformations}.
  \begin{figure}
    \begin{mathpar}
      \begin{tikzcd}[column sep=100pt]
        (a'\tensor b')\tensor c' \arrow[r, "{\assoc_{a',b',c'}}"] \arrow[d, "(f\tensor g)\tensor h"']
          & a' \tensor (b'\tensor c') \arrow[d, "f\tensor(g\tensor h)"] \\
        (x.a \tensor y.b) \tensor z.c \arrow[r, "{\assoc_{x.a,y.b,z.c}}"] \arrow[d, "{m_{x,y,a,b}\tensor z.c}"']
          & x.a \tensor (y.b \tensor z.c) \arrow[d, "{x.a \tensor m_{y.b,z.c}}"] \\
        ((x\tensor y).(a\tensor b)) \tensor z.c \arrow[d, "{m_{x\tensor y,z,a\tensor b,c}}"']
          & x.a \tensor ((y\tensor z).(b\tensor c)) \arrow[d, "{m_{x,y\tensor z,a,b\tensor c}}"] \\
        ((x\tensor y)\tensor z).((a \tensor b)\tensor c) \arrow[r, "{\assoc_{x,y,z}.\assoc_{a,b,c}}"]
          & (x \tensor (y \tensor z)).(a \tensor (b \tensor c))
      \end{tikzcd}
      \and
      \begin{tikzcd}
        a' \arrow[r, "\lunit_{a'}"] \arrow[d, "f"']
          & I \tensor a \arrow[d, "I\tensor f"] \\
        x.a \arrow[r, "\lunit_{x.a}"] \arrow[d, "\lunit_x.\lunit_a"' description]
          & I\tensor (x.a) \arrow[d, "e\tensor x.a"] \\
        (I\tensor x).(I\tensor a)
          & (I.I) \tensor (x.a) \arrow[l, "{m_{I,x,I,a}}"']
      \end{tikzcd}
      \and
      \begin{tikzcd}
        a' \arrow[r, "\runit_{a'}"] \arrow[d, "f"']
          & a \tensor I \arrow[d, "f\tensor I"] \\
        x.a \arrow[r, "\runit_{x.a}"] \arrow[d, "\runit_x.\runit_a"' description]
          & (x.a)\tensor I \arrow[d, "x.a\tensor e"] \\
        (x\tensor I).(a\tensor I)
          & (x.a) \tensor (I.I) \arrow[l, "{m_{x,I,a,I}}"']
      \end{tikzcd}
    \end{mathpar}
    \caption[Proof that the associators and unitors in $\C/\X$ are indeed natural transformations.]{Proof that the associators and unitors in $\C/\X$ are indeed natural transformations.  
    Here, $f\from a'\to a$, $g\from b' \to b$, $h\from c'\to c$ are morphisms in $\C/\X$, considered as morphisms $a'\to x.a$, $b'\to y.b$, $c'\to z.c$ in $\C$.}
    \label{FigAssocUnitorNaturalTransformations}
  \end{figure}
\end{proof}

Having defined the monoidal structure on $\C/\X$ in a very formal way, let us unpack what it actually is.  
The tensor product on objects is defined exactly as in $\C$, while the tensor product of morphisms $f\from a' \to a$, $g \from b' \to b$ (considered as morphisms $f \from a' \to x.a$, $g\from b'\to y.b$ in $\C$) is given by the composite
\[
  a'\tensor b' \xrightarrow{f\tensor g}
  (x.a) \tensor (y.b) \xrightarrow{m_{x,y,a,b}}
  (x\tensor y).(a\tensor b)\,,
  \]
where the right hand arrow is the multiplicative coherence for the monoidal functor $\blank.\blank\from \X \times \C \to \C$.

The reason we need some kind of symmetry in $\X$ is for this to be a functor: indeed, suppose that we have morphisms $f' \from a'' \to a'$, $f \from a' \to a$, $g' \from b'' \to b'$ and $g \from b' \to b$ in $\C/\X$, considered as morphisms $f' \from a'' \to x'.a'$, $f \from a' \to x.a$, $g' \from b'' \to y'.b'$ and $g \from b' \to y'b$ in $\C$.  
Then $(f'\tensor g');(f\tensor g)$ is given by the composite
\begin{IEEEeqnarray*}{Cl}
  & a'' \tensor b'' \\
  \xrightarrow{\makebox[100pt]{$f'\tensor g'$}}
  & (x'.a') \tensor (y'.b') \\
  \xrightarrow{\makebox[100pt]{$m_{x',y',a',b'}$}}
  & (x'\tensor y').(a'\tensor b') \\
  \xrightarrow{\makebox[100pt]{$(x'\tensor y').(f\tensor g)$}}
  & (x'\tensor y').((x.a)\tensor (y.b)) \\
  \xrightarrow{\makebox[100pt]{$(x'\tensor y').m_{x,y,a,b}$}}
  & (x' \tensor y').(x\tensor y).(a\tensor b) \\
  \xrightarrow{\makebox[100pt]{$m_{x'\tensor y',x\tensor y,a\tensor b}$}}
  & ((x'\tensor y')\tensor (x\tensor y)).(a\tensor b)\,,
\end{IEEEeqnarray*}

while $(f';f)\tensor (g';g)$ is given by
\begin{IEEEeqnarray*}{Cl}
  & a'' \tensor b'' \\
  \xrightarrow{\makebox[100pt]{$f'\tensor g'$}}
  & (x'.a') \tensor (y'.b') \\
  \xrightarrow{\makebox[100pt]{$x'.f \tensor y'.g$}}
  & (x'.x.a) \tensor (y'.y.b) \\
  \xrightarrow{\makebox[100pt]{$m_{x',x,a} \tensor m_{y',y,b}$}}
  & ((x'\tensor x).a) \tensor ((y'\tensor y).b) \\
  \xrightarrow{\makebox[100pt]{$m_{x'\tensor x,y'\tensor y,a,b}$}}
  & ((x' \tensor x) \tensor (y' \tensor y)) \tensor (a \tensor b)\,.
\end{IEEEeqnarray*}

Since $\X$ is symmetric, and since the natural transformation $m_{x,y,a}$ is a monoidal natural transformation with respect to the monoidal functor structure on $x.y.a$ and $(x\tensor y).a$, the natural symmetric monoidal coherence
\[
  ((x'\tensor y')\tensor (x\tensor y)) \toisom ((x' \tensor x) \tensor (y' \tensor y))
  \]
mediates between these two composites, so they give us the same morphism in $\C/\X$.

\subsection{Symmetric monoidal structure of $\C/\X$}

\begin{definition}
  Let a symmetric monoidal category $\X$ act via a monoidal lax action on a monoidal category $\C$.  
  If $\C$ is symmetric monoidal, then we say that the action of $\X$ on $\C$ is \emph{symmetric monoidal} if the underlying functor $\X \times \C \to \C$ is a symmetric monoidal functor.
\end{definition}

\begin{example}
  If $j\from \X \to \C$ is a symmetric oplax monoidal functor, where $\C$ is symmetric monoidal closed, then the reader action of $\oppcat\X$ on $\C$ induced from $j$ is a symmetric monoidal lax action.
\end{example}

\begin{proposition}
  Let a symmetric monoidal category $\X$ act via a symmetric monoidal lax action on a symmetric monoidal category $\C$.  
  Then the category $\C/\X$ is symmetric monoidal.
\end{proposition}
\begin{proof}
  As with the associators and unitors, we can lift the symmetry isomorphisms from $\C$ on to $\C/\X$, and these isomorphisms will satisfy the appropriate coherence diagrams.

  We need only show that they are natural transformations in $\C/\X$.  
  Let $f\from a' \to a$, $g\from b' \to b$ be morphisms in $\C/\X$, considered as morphisms $f\from a' \to x.a$, $g\from b' \to y.b$ in $\C$.  
  Then we need show that the following diagram commutes.
  \[
    \begin{tikzcd}[column sep=80pt]
      a'\tensor b' \arrow[r, "{\sym_{a',b'}}"] \arrow[d, "f\tensor g"']
        & b' \tensor a' \arrow[d, "g\tensor f"] \\
      (x.a)\tensor(y.b) \arrow[r, "{\sym_{x.a,y.b}}", dotted] \arrow[d, "{m_{x,y,a,b}}"']
        & (y.b) \tensor (x.a) \arrow[d, "{m_{y,x,b,a}}"] \\
      (x\tensor y).(a\tensor b) \arrow[r, "{\sym_{x,y}.\sym_{a,b}}"]
        & (y \tensor x).(b\tensor a)
    \end{tikzcd}
    \]
  Indeed, the top square commutes because $\sym$ is a natural transformation in $\C$, and the bottom square commutes because the action is a symmetric monoidal functor.
\end{proof}

\subsection{Symmetric monoidal closed structure of $\C/\X$}

Let a symmetric monoidal category $\X$ act on a symmetric monoidal category $\C$ via a symmetric monoidal action.  
Since any lax monoidal functor between monoidal closed categories is automatically lax monoidal closed, we might expect that in such a situation the category $\C/\X$ would be symmetric monoidal closed.

In fact, this is not the case.  
For example, suppose that $\X$ is the unit monoidal category, and that $\X$ acts on the category of sets via the powerset monad.  
Then the category $\Set/\X$ is the Kleisli category for the powerset monad -- i.e., the category of sets and relations.  

Since the powerset functor is lax monoidal, it induces a monoidal structure on the Kleisli category -- i.e., the familiar Cartesian product.  
Indeed, the category of sets and relations \emph{is} monoidal closed under this choice of monoidal product, but the internal hom functor is also given by the Cartesian product, and does not agree with the function space functor on the original category of sets.

In fact, we can obtain a result for monoidal closedness of $\C/\X$, but under different conditions.  
Only $\C$ needs to be monoidal closed, but the action of $\X$ must preserve the internal hom strictly.

\begin{definition}
  Let a symmetric monoidal category $\X$ act on a symmetric monoidal closed category $\C$ via a symmetric monoidal lax action.
  We say that the action is \emph{symmetric monoidal closed} if there is a natural isomorphism
  \[
    s_{x,a,b} \from a\implies (x.b) \toisom x.(a\implies b)
    \]
  that makes the following diagram commute for any object $x$ of $\X$ and any objects $a,b$ of $\C$.
  \[
    \begin{tikzcd}[column sep=50pt]
      (x.(a\implies b)) \tensor (I.a) \arrow[r, "{m_{x,I,a\implies b,a}}"]
        & (x \tensor I).((a\implies b)\tensor a) \arrow[d, "{\runit\inv\tensor \ev_{a,b}}"] \\
      (a \implies x.b)\tensor a \arrow[u, "{s_{x,a,b}\tensor e_a}"] \arrow[r, "{\ev_{a,x.b}}"]
        & x.b
    \end{tikzcd}
    \]
  \label{DefSymmetricMonoidalClosedAction}
\end{definition}

\begin{example}
  If $j$ is an oplax symmetric monoidal functor from a symmetric monoidal category $\X$ to a symmetric monoidal closed category $\C$, then the reader action of $\oppcat X$ on $\C$ induced by $j$ is symmetric monoidal closed, via the isomorphism
  \[
    (jx \implies b) \cong jx \implies (a \implies b)\,.
    \]
\end{example}

\begin{proposition}
  Suppose that a symmetric monoidal category $\X$ acts on a symmetric monoidal closed category $\C$ via a symmetric monoidal closed action.  
  Then the category $\C/\X$ is symmetric monoidal closed.
\end{proposition}
\begin{proof}
  Given objects $a,b$ of $\C$, we define $a\implies b$ in $\C/\X$ to be the same as $a\implies b$ in $\C$, and we define
  \[
    \ev_{a,b} = J(\ev_{a,b}) \from (a\implies b)\tensor a \to b\,,
    \]
  to be given by the same morphism as in $\C$.

  Let $f \from c \tensor a \to b$ be a morphism in $\C/\X$.  
  It is necessary and sufficient to show that there is a unique morphism
  \[
    h \from c \to a \implies b
    \]
  in $\C/\X$ such that $f = (h\tensor a);\ev_{a,b}$.

  Suppose that $f$ is given by a morphism
  \[
    \hat{f} \from c \tensor a \to x.b
    \]
  in $\C$.  
  This then induces a morphism
  \[
    \tilde{f} = W(\hat{f}) \from c \to (a \implies x.b)\,,
    \]
  which we can compose with $s_{x,a,b}\inv$ to give us a morphism
  \[
    \hat{h} = W(\hat{f});s_{x,a,b} \from c \to x.(a\implies b)\,,
    \]
  which we may consider as a morphism $h\from c \to (a\implies b)$ in $\C/\X$.  
  We claim that $(h\tensor a);\ev_{a,b}=f$; indeed, the composite $(h\tensor \id_a);\ev_{a,b}$ is given in $\C$ by the composite
  \[
    \begin{tikzcd}[column sep=50pt]
      %
        & (x.(a\implies b)) \tensor (I.a) \arrow[r, "{m_{x,I,a\implies b,a}}"]
          & (x \tensor I).((a\implies b)\tensor a) \arrow[d, "{\id_{x\tensor I}\tensor \ev_{a,b}}"] \\
      c \tensor a \arrow[r, "\tilde{f}\tensor a"]
        & (a \implies x.b)\tensor a \arrow[u, "{s_{x,a,b}\tensor e_a}"]
          & (x\tensor I).b
    \end{tikzcd}\,.
    \]
  Using the diagram in Definition \ref{DefSymmetricMonoidalClosedAction}, we see that this is equal to the composite
  \[
    c\tensor a \xrightarrow{\tilde{f}\tensor a}
    (a \implies x.b) \tensor a \xrightarrow{\ev_{a,x.b}}
    x.b \xrightarrow{\runit_x.b}
    (x\tensor I).b\,.
    \]
  The composite of the first two morphisms is equal to $\hat{f}$, and so the whole thing defines the same morphism as $f$ in $\C/\X$ (using the equivalence relation on morphisms).

  Now suppose that $h\from c \to a\implies b$ is a morphism in $\C/X$, given by a morphism $\hat{h} \from c \to x.(a\implies b)$ in $\C$.
  Let $f \from c \tensor a \to b$ be given by the morphism
  \[
    \hat{f} = W\inv(\hat{h};s_{x,a,b}\inv)\,.
    \]
  Then $h$ may be recovered from $f$ just as in the first part of this proof, and the equation $f = (h\tensor a);\ev_{a,b}$ determines $f$.  
  It follows that the choice of $h$ is unique.
\end{proof}

\subsection{Cartesianness of the monoidal structure}

Suppose that a symmetric monoidal category $\X$ acts via a symmetric monoidal action on a Cartesian category $\C$.  
In this section we will consider what properties of the action we need in order to ensure that the induced monoidal structure on $\C/\X$ is Cartesian.

Since $\C$ is Cartesian, the diagonal map in $\C$ gives every object a natural comonoid structure:
\[
  a \xrightarrow{\Delta_a} a \times a\,.
  \]
Since the tensor product of objects in $\C/\X$ is given by the same object as in $\C$, we can lift this comonoid structure through the functor $J$ to give us a natural comonoid structure on each object of $\C/\X$:
\[
  a \xrightarrow{J\Delta_a} a \tensor a\,.
  \]
We use the symbol $\tensor$ rather than $\times$ in $\C/\X$, since this is not necessarily a category-theoretic product.

If every morphism $f\from a \to b$ in $\C/\X$ is a comonoid homomorphism with respect to this comonoid structure; i.e., if it makes the diagram
\[
  \begin{tikzcd}
    a \arrow[r, "f"] \arrow[d, "J\Delta_a"']
      & b \arrow[d, "J\Delta_b"] \\
    a \tensor a \arrow[r, "f\tensor f"]
      & b \tensor b
  \end{tikzcd}
  \]
commute, then we may identify $\C/\X$ with a full subcategory of its own category of comonoids, closed under tensor product.  
Since the category of comonoids in a monoidal category is automatically Cartesian, this will tell us that $\C/\X$ is Cartesian.  

What does it mean for every morphism in $\C/\X$ to be a comonoid homomorphism?  
Firstly, since $\C$ itself is Cartesian, every morphism in $\C$ is a comonoid homomorphism with respect to the diagonal, and it follows that every morphism of the form $Jf \from a \to b$ (where $f\from a \to b$ is a morphism in $\C$) is a comonoid homomorphism in $\C/\X$.  
Since every morphism in $\C/\X$ may be written as the composite of a morphism of the form $Jf$ with a morphism of the form $\phi_{x,b}$, it will suffice to show that the morphisms of the form $\phi_{x,a}$ are comonoid homomorphisms in $\C/\X$; i.e., that they make the following diagram commute.
\[
  \begin{tikzcd}[column sep=40pt]
    x.a \arrow[r, "{\phi_{x,a}}"] \arrow[d, "J\Delta_{x.a}"']
      & a \arrow[d, "J\Delta_a"] \\
    x.a \tensor x.a \arrow[r, "{\phi_{x,a}\tensor\phi_{x,a}}"]
      & a \tensor a
  \end{tikzcd}
  \]
Let us compute the two arms of this square directly in $\C$.  
The composite along the top right is given in $\C$ by the morphism
\[
  x.a \xrightarrow{x.\Delta_a}
  x.(a\times a)\,,
  \]
while that along the bottom right is given by the composite
\[
  x.a \xrightarrow{\Delta_{x.a}}
  x.a \times x.a \xrightarrow{m_{x,x,a,a}}
  (x\tensor x) . (a \times a)\,.
  \]
Then a sufficient condition for these two morphisms to be equal in $\C/\X$, and therefore for the category $\C/\X$ to be Cartesian, is that for every object $x$ of $\X$ there should be a morphism $f \from x \to x\tensor x$ that makes the following diagram commute.
\[
  \begin{tikzcd}
    x.a \arrow[dr, "f.\Delta_a"] \arrow[d, "\Delta_{x.a}"']
      & \\
    x.a \times x.a \arrow[r, "{m_{x,x,a,a}}"]
      & (x\tensor x). (a\times a)
  \end{tikzcd}
  \]
If $\X$ itself is Cartesian, then we have a natural choice of morphism $x \to x \times x$, but choosing $f$ to be the diagonal map does not automatically make the diagram above commute.  
For example, the Cartesian product is not the category-theoretic product in the category of sets and relations, even though this category is the Kleisli category for the symmetric monoidal monad $\powerset \from \Set \to \Set$, and may be identified with the category $\Set/1$, where $1$ and $\Set$ are both Cartesian.  
In this case, the diagram
\[
  \begin{tikzcd}
    \powerset(A) \arrow[dr, "\powerset(\Delta_A)"] \arrow[d, "\Delta_{\powerset(A)}"']
      & \\
    \powerset(A) \times \powerset(A) \arrow[r, "{m_{A,A}}" xshift=-3pt]
      & \powerset(A \times A)
  \end{tikzcd}
  \]
does not commute, the arrow at the top right sending a set $X\subseteq A$ to the set $\{(x,x) \suchthat x \in X\}$ while the composite at the bottom left sends $X$ to the set $\{(x,y)\suchthat x,y\in X\}$.

An alternative sufficient condition for $\C/\X$ to be Cartesian is for the morphism mediating between the two arms of the square to go in the other direction.  
In this case, we require that for each object $x$ of $\X$ there should be a morphism $g \from x\tensor x \to x$ that makes the following diagram commute.
\[
  \begin{tikzcd}
    x.a \arrow[r, "x.\Delta_a"] \arrow[d, "\Delta_{x.a}"']
      & x.(a\times a) \\
    x.a \times x.a \arrow[r, "{m_{x,x,a,a}}"]
      & (x \tensor x).(a \times a) \arrow[u, "g.(a\times a)"']
  \end{tikzcd}
  \]
This will be particularly useful for reader actions.  
Indeed, suppose that $j\from \X \to \C$ is an oplax symmetric monoidal functor between symmetric monoidal categories, where $\C$ is Cartesian closed.  
The functor $j$ induces a reader action of $\oppcat\X$ on $\C$ (i.e., $x.a = jx \to a$), and we want the category $\C/\oppcat\X$ to be Cartesian.  
For this, it will be sufficient to find, for each object $x$ of $\X$, a morphism $f \from x \to x\tensor x$ such that the following diagram commutes.
\[
  \begin{tikzcd}
    jx \to a \arrow[rr, "jx \to \Delta_a"] \arrow[d, "\Delta_{jx \to a}"']
      &[-10pt]
        &[+10pt] jx \to (a \times a) \\
    (jx \to a) \times (jx \to a) \arrow[r]
      & (jx \times jx) \to (a \times a) \arrow[r, "m^j_x \to (a\times a)"]
        & j(x\tensor x) \to (a \times a) \arrow[u, "jf \to (a \times a)"']
  \end{tikzcd}
  \]
For this to commute, it is sufficient that $jf$ should be equal to the following composite.
\[
  jx \xrightarrow{\Delta_{jx}}
  jx \times jx \xrightarrow{m^j_{x,x}}
  j(x \tensor x)\,.
  \]
We have proved:

\begin{theorem}
  Let $j\from \X \to \C$ be an oplax symmetric monoidal functor between symmetric monoidal categories, where $\C$ is Cartesian closed.  
  Then $j$ induces a reader action of $\oppcat\X$ on $\C$, and the category $\C/\oppcat \X$ is symmetric monoidal closed.

  Suppose that for every object $x$ of $\X$, there is a morphism $f \from x \to x\tensor x$ in $\X$ such that $jf=\Delta_{jx};m^j_{x,x}$.  
  Then $\C/\oppcat \X$ is Cartesian closed.
  \label{TheCartesianClosedCx}
\end{theorem}

In the next section, we will investigate a class of reader actions, and give a more concrete property for the category $\C/\oppcat\X$ to be Cartesian closed.

\section{Reader actions on $\Set$}

\subsection{Colimits of actions are monoidal functors}

\begin{proposition}
  Let $\C$ be a cocomplete monoidal category, and suppose that a symmetric monoidal category $\X$ acts on $\C$ via a symmetric monoidal action in the sense of Definition \ref{DefMonoidalLaxAction}.  
  Define a functor $\C \to \C$ by
  \[
    F a = \colim_{x\from \X} x.a\,.
    \]
  Then $F$ is a lax monoidal functor.
  \label{PropColimitOfActionIsMonoidalFunctor}
\end{proposition}
\begin{proof}
  We have morphisms
  \begin{IEEEeqnarray*}{Cl}
    & \colim_{x \from \X} x.a \tensor \colim_{y\from \X} y.b \\
    \to & \colim_{x,y \from \X} x.a \tensor y.b \\
    \to & \colim_{x,y \from \X} (x\tensor y).(a \tensor b) \\
    \hookrightarrow & \colim_{z \from \X} z.(a\tensor b)\,.
  \end{IEEEeqnarray*}
  and
  \[
    1 \xrightarrow{e}
    I.1 \hookrightarrow
    \colim_{x\from \X} x.1\,.
    \]
  Figures \ref{FigColimMonoidalMultiplicative} and \ref{FigColimMonoidalUnital} show that these satisfy the coherence conditions for a monoidal functor.
  \begin{SidewaysFigure}
    \begin{mathpar}
      \begin{tikzcd}[ampersand replacement=\&]
        %
        \&[-20pt] \left(\colim_{x\from \X}\limits x.a \tensor \colim_{y\from \X}\limits y.b\right) \tensor \colim_{z \from \X}\limits z.c \arrow[r, "\assoc" yshift=5pt] \arrow[d]
            \&[-20pt] \colim_{x \from \X}\limits x.a \tensor \left(\colim_{y\from \X}\limits y.b \tensor \colim_{z \from \X}\limits z.c \right) \arrow[d]
              \&[-20pt] \\
        %
          \& \left(\colim_{x,y\from \X}\limits x.a \tensor y.b\right) \tensor \colim_{z \from \X}\limits z.c \arrow[d]
            \& \colim_{x\from \X}\limits x.a \tensor \left(\colim_{y,z \from \X}\limits (y.b \tensor z.c)\right) \arrow[d]
              \& \\
        %
          \& \colim_{x,y,z \from \X}\limits (x.a \tensor y.b) \tensor z.c \arrow[r, "\Colim\assoc"] \arrow[d, "{\Colim m_{x,y,a,b}\tensor z.c}"']
            \& \colim_{x,y,z \from \X}\limits x.a \tensor (y.b \tensor z.c) \arrow[d, "{\Colim x.a \tensor m_{y,z,b,c}}"]
              \& \\
        \colim_{t,z\from \X}\limits t.(a \tensor b) \tensor z.c \arrow[d, "{\Colim m_{t,z,a\tensor b,c}}"']
          \& \colim_{x,y,z \from \X}\limits (x\tensor y).(a \tensor b) \tensor z.c \arrow[d, "{\Colim m_{x\tensor y,z,a\tensor b,c}}"'] \arrow[l, hookrightarrow]
            \& \colim_{x,y,z \from \X}\limits x.a \tensor (y\tensor z).(b\tensor c) \arrow[d, "{\Colim m_{x,y\tensor z,a,b\tensor c}}"] \arrow[r, hookrightarrow]
              \& \colim_{x,t\from \X}\limits x.a \tensor t.(b\tensor c) \arrow[d, "{\Colim m_{x,t,a,b\tensor c}}"] \\
        \colim_{t,z\from \X}\limits (t\tensor z).((a \tensor b)\tensor c) \arrow[d, hookrightarrow]
          \& \colim_{x,y,z\from\X}\limits ((x\tensor y) \tensor z).((a \tensor b) \tensor c) \arrow[r, "\Colim\assoc.\assoc" yshift=5pt] \arrow[l, hookrightarrow]
            \& \colim_{x,y,z\from\X}\limits (x \tensor (y\tensor z)).(a \tensor (b \tensor c)) \arrow[r, hookrightarrow]
              \& \colim_{x,t\from \X}\limits (x \tensor t).(a \tensor (b \tensor c)) \arrow[d, hookrightarrow] \\
        \colim_{u\from \X}\limits u.((a\tensor b) \tensor c) \arrow[rrr, "\Colim u.\assoc"]
          \&
            \&
              \& \colim_{u\from \X}\limits u.(a \tensor (b\tensor c))
      \end{tikzcd}
    \end{mathpar}
    \caption[Proof that the colimit of a monoidal action satisfies the multiplicative criterion for being a monoidal functor.]
    {Proof that $\colim_{x\from \X} x.a$ satisfies the multiplicative criterion for being a monoidal functor.  
    The bottom hexagon is the multiplicative coherence for $\blank.\blank$ as a monoidal functor, after applying the colimit.}
    \label{FigColimMonoidalMultiplicative}
  \end{SidewaysFigure}
  \begin{figure}
    \begin{mathpar}
      \begin{tikzcd}
        %
          & |[alias=Z]| \colim_{z\from \X}\limits z.(1 \tensor a)
            &[-20pt] \\
        |[alias=Y]| \colim_{y\from \X}\limits y.a \arrow[r, "\Colim\lunit.\lunit" yshift=2pt] \arrow[ur, "\Colim \id.\lunit", from=Y.north, to=Z.west, bend left=20] \arrow[d, "\Colim\lunit" description] \arrow[dd, "\lunit"', bend right=60]
          & \colim_{y\from \X}\limits (I \tensor y).(1\tensor a) \arrow[r, hookrightarrow] \arrow[u, hookrightarrow]
            & |[alias=X]| \colim_{x,y\from \X}\limits (x \tensor y).(1 \tensor a) \arrow[ul, hookrightarrow, from=X.north, to=Z.east, bend right=20] \\
        \colim_{y\from \X}\limits 1 \tensor y.a \arrow[r, "\Colim e \tensor \id"]
          & \colim_{y \from \X}\limits I.1 \tensor y.a \arrow[u, "{\Colim m_{I,y,1,a}}" description] \arrow[r, hookrightarrow]
            & \colim_{x,y\from\X}\limits x.1 \tensor y.a \arrow[u, "{\Colim m_{x,y,1,a}}"'] \\
        1 \tensor \colim_{y \from \X}\limits y.a \arrow[u, Sim] \arrow[r, "e \tensor \id"]
          & I.1 \tensor \colim_{y \from \X}\limits y.a \arrow[u, Sim] \arrow[r, hookrightarrow]
            & \colim_{x\from \X}\limits x.1 \tensor \colim_{y\from 1} y.a \arrow[u]
      \end{tikzcd}
      \and
      \begin{tikzcd}
        %
          & |[alias=Z]| \colim_{z\from \X}\limits z.(a \tensor 1)
            &[-20pt] \\
        |[alias=Y]| \colim_{x\from \X}\limits x.a \arrow[r, "\Colim\runit.\runit" yshift=2pt] \arrow[ur, "\Colim \id.\runit", from=Y.north, to=Z.west, bend left=20] \arrow[d, "\Colim\runit" description] \arrow[dd, "\runit"', bend right=60]
          & \colim_{x\from \X}\limits (x \tensor I).(a\tensor 1) \arrow[r, hookrightarrow] \arrow[u, hookrightarrow]
            & |[alias=X]| \colim_{x,y\from \X}\limits (x \tensor y).(a \tensor 1) \arrow[ul, hookrightarrow, from=X.north, to=Z.east, bend right=20] \\
        \colim_{x\from \X}\limits x.a \tensor 1 \arrow[r, "\Colim \id \tensor e"]
          & \colim_{x \from \X}\limits x.a \tensor I.1 \arrow[u, "{\Colim m_{x,I,a,1}}" description] \arrow[r, hookrightarrow]
            & \colim_{x,y\from\X}\limits x.a \tensor y.1 \arrow[u, "{\Colim m_{x,y,a,1}}"'] \\
        \left(\colim_{x \from \X}\limits x.a\right) \tensor 1 \arrow[u, Sim] \arrow[r, "\id \tensor e"]
          & \left(\colim_{x \from \X}\limits x.a\right)\tensor I.a \arrow[u, Sim] \arrow[r, hookrightarrow]
            & \colim_{x\from \X}\limits x.a \tensor \colim_{y\from 1} y.1 \arrow[u]
      \end{tikzcd}
    \end{mathpar}
    \caption[Proof that the colimit of a monoidal action satisfies the unital criteria for being a monoidal functor.]
    {Proof that $\colim_{x\from \X} x.a$ satisfies the unital criteria for being a monoidal functor.  
    The top-right squares are the unital coherence for $\blank.\blank$ as a monoidal functor, after applying the colimit.}
    \label{FigColimMonoidalUnital}
  \end{figure}
\end{proof}

\begin{proposition}
  If $\C$ is a cocomplete symmetric monoidal category and the action of $\X$ on $\C$ is symmetric monoidal, then
  \[
    \colim_{x\from \X} x.\blank
    \]
  is a symmetric monoidal functor.
\end{proposition}
\begin{proof}
  We have a commutative diagram
  \[
    \begin{tikzcd}
      \colim_{x\from \X}\limits x.a \tensor \colim_{y\from \X} y.b \arrow[r, "\sym"] \arrow[d]
        & \colim_{y\from \X}\limits y.b \tensor \colim_{x\from \X}x.a \arrow[d] \\
      \colim_{x,y\from \X}\limits x.a \tensor y.b \arrow[r, "\Colim\sym"] \arrow[d, "{\Colim m_{x,y,a,b}}"']
        & \colim_{x,y\from \X}\limits y.b \tensor x.a \arrow[d, "{\Colim m_{y,x,b,a}}"] \\
      \colim_{x,y\from \X}\limits (x\tensor y).(a\tensor b) \arrow[r, "\Colim sym.sym"] \arrow[d, hookrightarrow]
        & \colim_{x,y \from \X}\limits (y \tensor x).(b\tensor a) \arrow[d, hookrightarrow] \\
      \colim_{z \from \X}\limits z.(a \tensor b) \arrow[r, "\Colim \id.\sym"]
        & \colim_{z\from \X}\limits z.(b\tensor a)
    \end{tikzcd}\,.
    \]
  Here, commutativity of the middle square is by applying the colimit to the diagram for $\blank.\blank$ to be a symmetric monoidal functor.
\end{proof}

\subsection{Reader actions on $\Set$ vs change of base}

For our next result, we will go via the \Mellies category.  
First, we need a standard result about the Day convolution product.

\begin{proposition}[\cite{Pisani}]
  Let $\X,\YY$ be a monoidal category and let $[\YY,\Set]$ be equipped with the Day convolution product.  
  Then a lax monoidal functor $\X \times \YY \to \Set$ is the same thing (via currying) as a lax monoidal functor $\X \to [\YY,\Set]$.
\end{proposition}

In particular, if $\X$ acts on the category of sets via a monoidal action, then it gives rise to a functor $\Set \to [\X,\Set]$.

\begin{proposition}
  Let $j \from \X \to \Set$ be an oplax symmetric monoidal functor between symmetric monoidal categories.
  Let $\oppcat\X$ act on $\Set$ via the reader action induced by $j$.
  Then the \Mellies category $\Mell_{\oppcat\X}\Set$ is isomorphic to the category obtained via base change along the action regarded as a functor
  \[
    \blank.\blank\from \Set \to [\X,\Set]\,.
    \]
  \label{PropMelliesVsBaseChange}
\end{proposition}
\begin{proof}
  The objects of both categories are sets.  
  The morphism objects in the \Mellies category are given by
  \[
    \Mell_\X\Set(A,B)(x) = [A,[jx,B]]\,,
    \]
  while those in the base-changed category are given by
  \[
    (\blank.\blank)_*\Set(A,B)(x) = [jx, [A,B]]\,,
    \]
  and these may be related by the symmetry isomorphism
  \[
    s_{jx,A,B} \from [A,[jx,B]] \to [jx, [A,B]]\,.
    \]
  We need to show that this preserves composition of morphisms.  
  Recall that composition in the \Mellies category is given by
  \begin{IEEEeqnarray*}{Cl}
    & \int^{y,z} [A,[jy,B]] \times [B,[jz,C]] \times \X(y\tensor z,x) \\
    \to &
    \int^{y,z} [A,[j(y\tensor z),C]]\times \X(y\tensor z,x) \\
    \cong &
    [A,[jx,C]]\,,
  \end{IEEEeqnarray*}
  where the first arrow is induced via the \Mellies composition, while composition in the base changed category is given by
  \begin{IEEEeqnarray*}{Cl}
    &\int^{y,z}[jy,[A,B]] \times [jz,[B,C]] \times \X(y\tensor z,x) \\
    \xrightarrow{\makebox[146pt]{$\int^{y,z} m_{jy,jz,[A,B],[B,C]}\times \X(y\tensor z,x)$}} &
    \int^{y,z}[jy \times jz,[A,B]\times [B,C]] \times \X(y,\tensor z,x) \\
    \xrightarrow{\makebox[146pt]{$\int^{y,z} [m^j_{y,z}, ;] \times \X(y\tensor z,x)$}} &
    \int^{y,z}[j(y \tensor z),[A,C]]\times \X(y\tensor z,x) \\
    \cong & 
    [jx,[A,C]]\,.
  \end{IEEEeqnarray*}
  where $;$ is the internal composition in $\Set$.

  Using the diagram in Figure \ref{FigMelliesVsBaseChange}, we see that the expressions inside the coends are related by the symmetry isomorphisms as follows.
  \[
    \begin{tikzcd}[column sep=30pt]
      {[A,[jy,B]] \times [B,[jz,C]]} \arrow[r, "\text{\Mellies}" xshift=-2pt] \arrow[d, "{s_{jx,A,B}\times s_{jy,B,C}}"']
        & {[A,[j(y\tensor z),C]]} \arrow[d, "{s_{j(x\tensor y)},A,C}"] \\
      {[jy,[A,B]]\times [jz,[B,C]]} \arrow[r, "\text{B.c.}"]
        & {[j(y \tensor z),[A,C]]}
    \end{tikzcd}
    \]
  \begin{SidewaysFigure}
    \[
      \begin{tikzcd}[ampersand replacement=\&, column sep=73pt, row sep=20pt]
        {[A,[jy,B]] \times [B,[jz,C]]} \arrow[d, "{[A,[jy,B]] \times L_{B,[jz,C]}^{jy}}"', thick, dashed] \arrow[r, "{s_{y,A,B}\times s_{z,B,C}}", dotted]
          \& {[jy,[A,B]] \times [jz,[B,C]]} \arrow[d, "{m_{jy,jz,[A,B],[A,C]}}"]
            \& \\
        {[A,[jy,B]] \times [[jy,B],[jy,[jz,C]]]} \arrow[d, "{[A,[jy,B]]\times[[jy,B],W\inv]}"', thick, dashed]
          \& |[alias=Z]| {[jy \times jz,[A,B]\times[B,C]]} \arrow[d, "{[jy\times jz,;]}", dotted] \arrow[dr, "{[m_{y,z},;]}", bend left=17.5, from=Z.east]
            \& \\
        {[A,[jy,B]]} \times [[jy,B],[jy\times jz,C]] \arrow[d, ";"', thick, dashed]
          \& {[jy \times jz,[A,C]]} \arrow[r, "{[m_{y,z},[A,C]]}", dotted]
            \& {[j(y\tensor z),[A,C]]} \\[51pt]
        {[A,[jy \times jz,C]]} \arrow[ur, "{s_{jy\times jz,A,C}}", dotted] \arrow[r, "{[A,[m_{y,z},C]]}", thick, dashed]
          \& {[A,[j(y\tensor z),C]]} \arrow[ur, "{s_{j(y\tensor z),A,C}}", dotted]
            \&
      \end{tikzcd}
      \]
      \caption[Proof that \Mellies composition agrees with base-changed composition in the case of a symmetric reader action on $\Set$.]%
      {Proof that \Mellies composition agrees with base-changed composition in the case of a symmetric reader action on $\Set$.  
      The \Mellies composition is given by the thick dashed arrows, while the composition in the base-changed category is given by the thin arrows.
      The dotted lines at the top and at the bottom right -- given by the symmetry isomorphisms -- mediate between the two.
      We can verify that the main heptagon commutes by directly computing each direction: in each case, a pair $\langle f,g\rangle$ of functions is sent to the function $h \from A \to {[jy\times jz,C]}$ given by
      \[
        h(a)(Y,Z) = g(f(a)(Y))(Z)\,.
        \]}
      \label{FigMelliesVsBaseChange}
  \end{SidewaysFigure}
  It follows that the functor induced by $s$ is an isomorphism of $[\X,\Set]$-enriched categories.
\end{proof}

By applying base change along the colimit functor to both these categories, we get the following.

\begin{corollary}
  Let $j\from \X \to \Set$ be a symmetric monoidal functor.  
  Let $\oppcat\X$ act on $\Set$ via the induced reader action.  
  Then the category $\Set/\oppcat\X$ is isomorphic to the category obtained from $\Set$ by base change along the functor
  \[
    \colim_{x\from \X} x.\blank \from \Set \to \Set\,.
    \]
  \label{CorCxVsBaseChange}
\end{corollary}

Thus, the theory of reader actions on $\Set$ is subsumed into the theory of base change in $\Set$ along monoidal functors $\Set \to \Set$.

\subsection{From monoidal endofunctors on $\Set$ to reader actions}

We can get a result in the other direction; i.e., a kind of converse to Proposition \ref{PropColimitOfActionIsMonoidalFunctor} in the case of reader actions on $\Set$.

\begin{proposition}
  Let $F \from \Set \to \Set$ be a lax symmetric monoidal functor.  
  Then there is a symmetric monoidal category $\X$ and an oplax monoidal functor $j\from \X \to \Set$ such that for all sets $A$, we have
  \[
    FA \cong \colim_{x \from \X} [jx,A]\,,
    \]
  and the monoidal coherences of $F$ arise from the reader action of $\oppcat\X$ on $\Set$ as in Proposition \ref{PropColimitOfActionIsMonoidalFunctor}.
  \label{PropMonoidalFunctorIsColimitOfReaderAction}
\end{proposition}
\begin{proof}
  Consider $F$ as a pseudofunctor $\Set \to \Cat$, and let $\X=\oppcat{\left(\int F\right)}$ be the opposite of its Grothendieck construction.  
  Since $\Set$ is an ordinary category, so is $\X$.

  More concretely, the objects of $\X$ are pairs $(A,m)$, where $A$ is a set and $m\in FA$, and morphisms
  \[
    (A,m) \to (B,n)
    \]
  are given by functions $h\from A \to B$ such that $F(h)(m) = n$.
  $\X$ is normally called the \emph{category of elements of $F$}.

  We define a symmetric monoidal structure on $\X$ by defining
  \begin{mathpar}
    (A,m) \tensor (B,n) = (A \times B, m^F_{A,B}(m,n))
    \and
    I = (1, e^F)\,,
  \end{mathpar}
  using the fact that the monoidal coherences for $F$ are given by functions
  \begin{mathpar}
    m^F_{A,B} \from FA \times FB \to F(A\times B)
    \and
    e^F \from 1 \to F1\,.
  \end{mathpar}
  We now need to show that the monoidal coherences in $\Set$ give rise to morphisms
  \begin{mathpar}
    \assoc_{A,B,C} \from ((A \times B) \times C,m^F_{A\times B,C}(m^F_{A,B}(m,n),p))
    \to
    (A \times (B \times C),m^F_{A,B \times C}(m,m^F_{B,C}(n,p))
    \and
    \lunit_{A} \from (A,m) \to (1\times A,m^F_{1,A}(e^F,m))
    \and
    \runit_{A} \from (A,m) \to (A \times a,m^F_{A,1}(m,e^F))
    \and
    \sym_{A,B} \from (A \times B,m^F_{A,B}(m,n)) \to (B \times A,m^F_{B,A}(n,m))
  \end{mathpar}
  in $\X$.  
  Happily, the diagrams we need for this are precisely the coherence diagrams for $m^F,e^F$ that we get from $F$ being a lax symmetric monoidal functor.

  Since $\assoc,\lunit,\runit,\sym$ satisfy the pentagon, triangle and hexagon identities in $\Set$, so they do in $\X$.
  Therefore, $\X$ is a symmetric monoidal category.

  There is an obvious forgetful functor $\X \to \Set$ that is, in fact, strict monoidal.

  We claim that if $A$ is a set, then we have
  \[
    \colim_{(X,m)\from \oppcat\X} [X,A] \cong FA\,.
    \]
  Indeed, for each object $(X,m)$ of $\X$ we have a function
  \begin{IEEEeqnarray*}{cCc}
    [X,A] & \to & FA \\
    f & \mapsto & F(f)(m)\,.
  \end{IEEEeqnarray*}
  We claim that this defines a cocone under the functor $(X,m) \mapsto [X,A] \from \oppcat\X \to \Set$.  
  Indeed, if we have a morphism $h \from (Y,n) \to (X,m)$ then by definition we have
  \[
    F(h)(n) = m\,,
    \]
  and so we have a commutative triangle
  \[
    \begin{tikzcd}[column sep=50pt]
      {[X,A]} \arrow[r, "{f \mapsto  F(f)(m)}"] \arrow[d, "{f \mapsto h;f}"']
        & FA \\
      {[Y,A]} \arrow[ur, "{g \mapsto F(g)(n)}"']
        &
    \end{tikzcd}\,,
    \]
  since
  \[
    F(h;f)(n) = F(f)(F(h)(n)) = F(f)(m)\,.
    \]
  Therefore, we have an induced map
  \[
    \colim_{(X,m)\from\X}[X,A] \to FA
    \]
  that sends $((X,m),f)$ to $F(f)(m)$.

  Now we define a map in the other direction.
  \begin{IEEEeqnarray*}{cCc}
    FA & \to & \colim_{(X,m)\from \X} [X,A] \\
    m & \mapsto & ((A,m),\id_A)
  \end{IEEEeqnarray*}

  We claim that these two maps are inverses.  
  Indeed, we certainly have
  \[
    F(\id_A)(m) = id_{FA}(m) = m\,.
    \]
  In the other direction, we need to show that
  \[
    ((A,F(f)(m)),\id_A) = ((X,m),f)
    \]
  in the colimit.
  But indeed, we have a morphism $f\from (X,m) \to (A,F(f)(m))$ in $\oppcat\X$, and $f;\id_A=f$.
  Therefore, our map $\colim_{(X,m)\from \X}[X,A] \to FA$ was a bijection.

  Lastly, we need to show that this decomposition as a colimit gives rise to the monoidal structure on the functor $F$; i.e., that the following diagrams commute.
  \begin{mathpar}
    \begin{tikzcd}
      FA \times FB \arrow[r, "{m^F_{A,B}}"] \arrow[d, Isom']
        & F(A\times B) \arrow[d, Isom] \\
      \colim_{(X,m)\from \X}\limits [X,A] \times \colim_{(Y,n)\from\X}\limits [Y,B] \arrow[r]
        & \colim_{(Z,p)\from\X}\limits [Z,A\times B]
    \end{tikzcd}
    \and
    \begin{tikzcd}
      1 \arrow[r, "e^F"] \arrow[dr]
        & F1 \arrow[d, Isom] \\
      %
        & \colim_{(X,m)\from \X}\limits[X,1]
    \end{tikzcd}
  \end{mathpar}
  Here, the arrows marked with the isomorphism symbol $\cong$ are the isomorphisms that we have just defined, while the arrows at the bottom of the first diagram and at the bottom left of the second are as in Proposition \ref{PropColimitOfActionIsMonoidalFunctor}.

  We can check by hand that these diagrams commute: that in the first diagram, both directions send the pair $(m,n)\in FA \times FB$ to
  \[
    ((A\times B,m_{A,B}(m,n)),\id_{A\times B})\in \colim_{(Z,p)\from \X}[Z,A\times B]\,,
    \]
  and that in the second diagram both directions pick out the element
  \[
    ((1,e^F),\id_1)
    \]
  of $\colim_{(X,m)\from \X}[X,1]$.
\end{proof}

\begin{example}
  Let $\powerset_+\from \Set \to \Set$ be the non-empty powerset functor.  
  Then the category $\X$ of elements of $\powerset_+$ has pairs $(A,M)$ as elements, where $A$ is a (necessarily non-empty) set and $M$ is a non-empty subset of $A$.
  Morphisms $(A,M) \to (B,N)$ are functions $f \from A \to B$ such that $f(M)=N$.

  The nonempty powerset functor has a natural monoidal structure:
  \begin{mathpar}
    \powerset_+A \times \powerset_+ B \to \powerset_+(A \times B)
    \and
    1 \to \powerset_+1
  \end{mathpar}
  sending $(M,N)$ to $M\times N\subset A \times B$ and picking out the subset $1 \subset 1$.
  This then gives us a monoidal structure on this category of elements, inducing an action as in Proposition \ref{PropMonoidalFunctorIsColimitOfReaderAction}.
  The colimit of this action then gives us the original functor.  
  Therefore, by Corollary \ref{CorCxVsBaseChange}, the category $\Set/\X$ is isomorphic to the category $(\powerset_+)_*\Set$ obtained by base change along the non-empty powerset functor $\powerset_+\from \Set \to \Set$.
\end{example}

\begin{example}
  Let $\DG \from \Set \to \Set$ be the functor that sends a set $A$ to the set of discrete probability measures on $A$ and sends a function $f\from A \to B$ to the function $\DG(A) \to \DG(B)$ that sends a probability measure $\bP$ on $A$ to the probability measure $f_*\bP$ on $B$ given by
  \[
    f_*\bP(X) = \bP(f\inv(X))\,.
    \]
  Then the category of elements of $\DG$ is the category whose objects are pairs $(A,\bP)$ and where the morphisms $(A,\bP_A) \to (B,\bP_B)$ are functions $f\from A \to B$ such that $\bP_B=f_*\bP_A$.  
  In other words, it is the category of discrete probability spaces and probability-preserving functions.

  The functor $DG$ has a natural monoidal structure: given a discrete probability measure on a set $A$ and a discrete probability measure on a set $B$, we can get a discrete probability measure on the set $A \times B$ by
  \[
    \bP_{A\times B}(\{(a,b)\}) = \bP_A(\{a\})\times\bP_B(\{b\})\,.
    \]
  This then gives us a monoidal structure on the category of discrete probability spaces, where the tensor product of probability spaces $(A,\bP_A)$ and $(B,\bP_B)$ is given by the set $A \times B$, together with the probability measure as described above.
\end{example}

Now suppose that $F\from \Set \to \Set$ is a lax symmetric monoidal endofunctor.  
We have shown that $F$ gives rise to a strict symmetric monoidal functor $j\from \X \to \Set$, for some monoidal category $\X$, and that the induced category $\Set/\oppcat\X$ is isomorphic to the category obtained from $\Set$ by base change along $F$.  

Since the action of $\X$ on $\Set$ is the reader action of a symmetric monoidal functor, by Theorem \ref{TheCartesianClosedCx} we know that the category $\Set/\oppcat\X$ must be symmetric monoidal closed.  

Now recall that in order to apply the second part of Theorem \ref{TheCartesianClosedCx}, and deduce that $\Set/\oppcat\X$ is Cartesian closed, we must prove that for every object $x$ of $\X$, there is a morphism $f \from x \to x \tensor x$ such that $jf = \Delta_{jx};m^j_x$.  
This is particularly useful in our case, since the functor $\X \to \Set$ is faithful.

In other words, for every set $A$ and every $p\in FA$, the function $\Delta_A$ must define a morphism
\[
  (A,p) \to (A \times A, m^F_{A,A}(p,p))\,;
  \]
i.e., we must have
\[
  F(\Delta_A)(p) = m^F_{A,A}(p,p)\,.
  \]
Let us see what this means in some examples.
\begin{example}
  The finite powerset functor does not satisfy the condition given above; indeed, we have
  \begin{mathpar}
    \powerset_+(\Delta_A)(X) = \{(x,x)\in A \times A \suchthat x \in X\}
    \and
    m^{\powerset_+}_{A,A}(X,X) = X \times X = \{(x,y) \in A \times A \suchthat x \in X\}\,.
  \end{mathpar}
  Similarly, the discrete probability measure functor does not satisfy the condition we have given, since the diagonal map
  \[
    \Delta_A \from (A,\bP_A) \to (A \times A, \bP_{A\times A})
    \]
  does not preserve probability in general.
\end{example}

\begin{example}
  An alternative way of dealing with probability that does satisfy the condition for Cartesianness of the resulting category.
  If $(\Omega,\F,\bP)$ is a fixed probability space, then a \emph{discrete random variable taking values in a set $A$} is a measurable function
  \[
    V\from (\Omega,\F) \to (A,\powerset A)\,.
    \]
  Given $X \subset A$, we define
  \[
    \bP(V \in X) = \bP(V\inv(X))\,.
    \]
  This then gives us a discrete probability measure on $X$.

  Write $\RV_\Omega \from \Set \to \Set$ for the functor that sends a set $A$ to the set of all random variables taking values in the set $A$ and sends a function $f\from A \to B$ to the function
  \[
    \RV_\Omega(A) \to \RV_\Omega(B)
    \]
  given by composing on the right with $f$.

  Moreover, $\RV_\Omega$ is a lax monoidal functor: if $V,W$ are discrete random variables taking values in sets $A,B$, then we have a random variable $\langle V,W \rangle\from X \to A\times B$ given by pairing (and if $(a,b)\in A\times B$, then $(\langle V,W\rangle)\inv(\{(a,b)\}) = V\inv(\{a\})\cap W\inv(\{b\})$, so this is a measurable function).  

  Now if we apply the construction from Proposition \ref{PropMonoidalFunctorIsColimitOfReaderAction}, we get the category whose objects are pairs $(A,V)$, where $A$ is a set and $V$ a discrete random variable taking values in $A$ and where the morphisms $(A,V) \to (B,W)$ are functions $f \from A \to B$ such that $W=V;f$.
  This category, which we will call $\Rv_\Omega$, admits a strict monoidal forgetful functor into the category of sets, giving us a lax reader action of $\oppcat{\Rv_\Omega}$ on $\Set$.
  Once again, we can pass to the symmetric monoidal category $\Set/\oppcat{\Rv_\Omega}$, giving us a way of modelling probability that is equivalent to taking base change through the functor $\RV_\Omega$.

  The difference now is that the diagonal map
  \[
    \Delta_A \from (A,V) \to (A\times A, \langle V,V\rangle)
    \]
  \emph{is} probability preserving.  
  In the language of probability, the two copies of the random variable $V$ are \emph{dependent random variables}, so the probability of obtaining the reading $(a,a)$ from $\langle V,V\rangle$ is the same as the probability of obtaining the reading $v$ from $V$.
  Therefore, the category $\Set/\oppcat{\Rv_\Omega}$ will be Cartesian closed.
\end{example}

\subsection{Actions of categories with terminal objects}

The probability example above lends itself to further examination.  
Recall that if a monoidal category $\X$ acts on a category $\C$, and $x$ is a monoid in $\X$ with multiplication $m^x$ and unit $e^x$, then we get a monad on $\C$ given by the composite
\[
  1 \xrightarrow{x} \X \xrightarrow{\blank.\blank} \End[\C]
  \]
of lax monoidal functors.

More explicitly, this monad is given by
\[
  M_xa = x.a\,,
  \]
with the monadic coherences given by
\begin{mathpar}
  M_xM_xa = x.x.a \xrightarrow{m_{x,x}} (x \tensor x).a \xrightarrow{m^x.a} x.a = M_xa
  \and
  a \xrightarrow{e_a} I.a \xrightarrow{e^x.a} x.a = M_x a\,.
\end{mathpar}
In particular, if $j\from \X \to \C$ is an oplax monoidal (in particular, comonoid-preserving) functor, yielding a reader action of $\oppcat\X$ upon $\C$, and $x$ is a monoid in $\oppcat\X$ (i.e., a comonoid in $\X$), then the monad $M_x$ will be the reader monad corresponding to $jx$.

Now suppose that the category $\X$ has a terminal object $1$.  
Then $1$ automatically has the structure of a monoid in $\X$, via the unique morphisms
\begin{mathpar}
  1 \tensor 1 \xrightarrow{()} 1
  \and
  I \xrightarrow{()} 1\,.
\end{mathpar}

\begin{proposition}
  Let a monoidal category $\X$ act on a category $\C$ via a lax action.  
  Suppose that $\X$ has a terminal object $1$.
  Then the category $\C/\X$ is isomorphic to the Kleisli category for the monoid $M_1$.
\end{proposition}
\begin{proof}
  In both cases, the objects are the objects of $\C$.  
  Note also that we have an isomorphism
  \begin{mathpar}
    (\C/\X)(a,b) = \int^x\C(a,x.b) \cong \int^x \C(a,x.b) \times \C(x,1) \cong \C(a,1.b) = \Kl_{M_1}(a,b)\,.
  \end{mathpar}
  More concretely, this isomorphism sends a \Mellies morphism $f\from a \to x.b$ to the Kleisli morphism given by the composite
  \[
    a \xrightarrow{f} x.b \xrightarrow{().b} 1.b\,.
    \]
  By Proposition \ref{PropFunctorialityOfLaxColimits}, this gives us a functor
  \[
    \C/\X \to \Kl_{M_1}\,,
    \]
  which is fully faithful and the identity on objects.
\end{proof}

Now we can get a clearer idea of what is going on in the probability example: the new category $\Rv_\Omega$ \emph{almost} has an initial object (i.e., a terminal object in $\oppcat{\Rv_\Omega}$), given by the pair
\[
  (\Omega,\id_\Omega)\,.
  \]
At least, it does if the $\sigma$-algebra $\F$ on $\Omega$ is discrete.  
But in general, the identity function
\[
  \id_\Omega \from (\Omega,\F) \to (\Omega,\powerset\Omega)
  \]
is not measurable (indeed, it is measurable if and only if $\F=\powerset\Omega$), so $\id_\Omega$ is not a discrete random variable in general.

In the case that $\F=\powerset\Omega$, the category $\Rv_\Omega$ has an initial object, and the category
\[
  \Set/\oppcat{\Rv_\Omega}
  \]
is isomorphic to the Kleisli category for the reader monad for the set $\Omega$ on $\Set$.  
Now, from Theorem \ref{FunctionalCompletenessCcc}, we know that this Kleisli category is automatically Cartesian closed.  
The point here is that the category $\Rv_\Omega$ is just close enough to having an initial object for the category $\Set/\oppcat{\Rv_\Omega}$ to be Cartesian, even though it is not a Kleisli category.

\begin{remark}
  This subtlety with probability spaces is not present in the case of the nonempty powerset functor, which is why we it makes sense to model plain nondeterminism with a Kleisli category, as we did in Chapter \ref{ChapMonads}.
\end{remark}

\subsection{Reader actions and semantics}

Suppose that we have an oplax monoidal functor $j \from \X \to \Set$, giving rise to a lax reader action of $\oppcat\X$ on $\Set$.
Let $\G$ be a (Cartesian closed) model for a programming language $P$, and suppose that there is a functor $D\from \Set \to \G$ that gives us the denotation of the datatypes in $\G$.

For example, if $\G$ is the category of games, then we can define $D(A)$ to be the game given by $M_{D(A)} = \{q\}\cup A$ with $\lambda(q)=O$ and $\lambda(a)=P$ for $a\in A$, where $q$ is initial and justifies the moves $a\in A$ and the legal plays are those of the form $\epsilon$, $q$ or $qa$.

In general, such functions automatically carry an oplax monoidal structure, since the source category $\Set$ and target category $\G$ are Cartesian.  
So we get morphisms
\begin{mathpar}
  m_{A,B} = D(A \times B) \xrightarrow{\langle D(\pr_1),D(\pr_2) \rangle} D(A) \times D(B)
  \and
  e = D(1) \xrightarrow{()} 1\,.
\end{mathpar}

Then we may compose the functors $j$ and $D$ to get a new oplax monoidal functor $\X \to \G$, inducing a reader action of $\oppcat{\X}$ on $\G$.

Now the category $\G/\oppcat\X$ has its universal natural transformation
\[
  \phi_{x,A} \from (Djx \to A) \to A\,,
  \]
which, by the enriched Yoneda lemma, may be equivalently given by the natural transformation
\[
  \omega_x = 1 \xrightarrow{u_{D(jx)}} (D(jx) \to D(jx)) \xrightarrow{\phi_{x,D(jx)}} D(jx)\,.
  \]
We can use this natural transformation to model primitives in the language of the form
\[
  \choose_x \from jx\,,
  \]
for each object $x$ of $\X$, where we use $jx$ to refer to the datatype corresponding to the set $jx$.

The purpose of the final chapter will be to define a language with an operational semantics for which these primitives make sense, and prove a full abstraction result for it.

\bibliographystyle{alpha2}
\bibliography{../common/phd_bibliography}

\end{document}

\chapter{Promonads and parametric promonads}
\label{ChapPromonads}

The purpose of this short chapter is to shine some light on the definition of the \Mellies category for a parametric monad, showing why it is natural to think of it as being an analogue for the Kleisli category on a monad.

As a technical tool to prove the results we want, we shall introduce multicategories, which are a small generalization of monoidal categories.  
The main purpose of this generalization is to allow us to do without coends wherever possible: for example, while we need coends to make $[\X,\Set]$ into a monoidal category, we do not need them to make it into a multicategory.  

The first half of this chapter is, in the interests of completeness, fairly technical, and may be skimmed over on a first reading.  
In chapter \ref{SecEndoprofunctors}, we introduce the multicategory of endoprofunctors on a category $\C$, which generalizes the monoidal category of endofunctors on $\C$.  
As monoids in $\End(\C)$ are called monads on $\C$, so we will call monoids in $\Endoprof(\C)$ \emph{promonads} on $\C$.  
We will observe that a promonad may be regarded as a sort of category, and that the Kleisli category may be characterized as the embedding of monads on $\C$ into promonads on $\C$.  

The main result will then be to show that an $\X$-parametric promonad on a category $\C$ -- i.e., a multifunctor $\X \to \Endoprof(\C)$ -- may be regarded as a sort of $[\X,\Set]$-enriched category, and that the \Mellies category may similarly be regarded as an embedding of $\X$-parametric monads on $\C$ into $\X$-parametric promonads on $\C$.

\section{Multicategories}

\begin{definition}[\cite{Multicategories}]
  A \emph{multicategory} $\M$ is given by a set of objects $\text{Ob}(\M)$ whose elements are called \emph{objects} and, for each (possibly empty) list $a_1,\cdots,a_n$ of objects and each object $b$, a set
  \[
    \M_n(a_1,\cdots,a_n;b)
    \]
  whose elements are called the ($n$-ary) \emph{multimorphisms} $a_1,\cdots,a_n\to b$.

  Given collections $(a_{ij} \from i = 1,\cdots, n,j=1, \cdots, k_i), (b_i \from i = 1, \cdots, n), c$ of objects and multimorphisms
  \begin{mathpar}
    f_i \from a_{i1} , \cdots , a_{i,k_{i}} \to b_i
    \and
    g \from b_1, \cdots, b_n \to c\,,
  \end{mathpar}
  there is an operation that forms the \emph{composition}
  \[
    (f_1,\cdots, f_n);g \from a_{11},\cdots,a_{1k_1},\cdots,a_{n1}\cdots,a_{nk_n} \to c\,.
    \]
  Moreover, for each object $a$ of $\M$, there is a distinguished multimorphism $\id_a \from a \to a$ called the \emph{identity} on $a$.

  The composition and identity are subject to associativity and unitality conditions.
  Namely, let
  \begin{mathpar}
    \left(\mbox{\pbox{24pt}{$a_{pqr}$}} \suchthat \mbox{\pbox{80pt}{$p=1,\cdots,n$ \\ $q=1,\cdots,k_p$ \\ $r=1,\cdots, l_{pq}$}}\right)
    \and
    \left(\mbox{\pbox{24pt}{$b_{pq}$}} \suchthat \mbox{\pbox{80pt}{$p=1,\cdots,n$ \\ $q=1,\cdots,k_p$}}\right)
    \\
    (c_p\from p = 1,\cdots,n)
    \and
    d
  \end{mathpar}
  be collections of objects and let
  \begin{mathpar}
    f_{pq} \from a_{pq1},\cdots,a_{p,q,l_{pq}} \to b_{pq}
    \and
    g_p \from b_{p1},\cdots,b_{p,k_p} \to c_p
    \and
    h \from c_1,\cdots,c_n\to d
  \end{mathpar}
  be multimorphisms.
  Then we require that
  \begin{gather*}
    (((f_{11},\cdots,f_{1k_1});g_1),\cdots,((f_{n1},\cdots,f_{n,k_n});g_n));h\\
    =\\
    (f_{11},\cdots,f_{1k_1},\cdots,f_{n1},\cdots,f_{n,k_n});((g_1,\cdots,g_n);h)\,.
  \end{gather*}
  Furthermore, we require that if $f\from a_1,\cdots,a_n\to b$ is a multimorphism, then
  \begin{mathpar}
    (\id_{a_1},\cdots,\id_{a_n});f = f
    \and
    f = (f);\id_b\,.
  \end{mathpar}
\end{definition}

\begin{remark}
  We will use a slightly different form of commutative diagrams for multicategories, which should be fairly straightforward to understand.
  If, for example, we say that the following diagram commutes,
  \[
    \begin{tikzcd}[column sep=40pt]
      {a_1,\cdots,a_n} \arrow[r, "{f_1,\cdots,f_j}"] \arrow[d, "{g_1,\cdots,g_k}"']
        & {b_1,\cdots,b_j} \arrow[d, "h"] \\
      {c_1,\cdots,c_k} \arrow[r, "i"]
        & d
    \end{tikzcd}
    \]
  where the arities of $f_1,\cdots,f_j$ sum to $n$, as do the arities of $g_1,\cdots,g_k$, then we mean that the composite $(f_1,\cdots,f_k);h$ is equal to the composite $(g_1,\cdots,g_k);i$.
  We leave it to the reader to extend this to more complicated diagrams.
\end{remark}

\begin{example}
  If $\C$ is an ordinary category, then $\C$ may be regarded as a multicategory $\hat{C}$ in which $\hat{C}_1(a;b)=\C(a,b)$ and $\hat{C}_n(a_1,\cdots,a_n;b)=\emptyset$ for $n\ne 1$.  
  At the same time, if $\M$ is a multicategory, then it has an \emph{underlying ordinary category} $\M_1$ whose morphisms are the morphisms in $\M$ with a single source object.
\end{example}
\begin{example}
  If $\M$ is a \emph{monoidal} category, then $\M$ may be regarded as a multicategory $\tilde{\M}$ with
  \begin{mathpar}
    \tilde{\M}_n(a_1,\cdots,a_n;b) = \M(a_1 \tensor \cdots \tensor a_n, b) \quad n\ge 1
    \and
    \tilde{\M}_0(;b) = \M(I, b)
  \end{mathpar}
  If we're being careful, then we should note that the expression $a_1 \tensor \cdots \tensor a_n$ does not define a single object of $\M$, since the tensor product is not in general strictly associative.
  Since the tensor product is not necessarily strictly associative, it is not obvious exactly what we mean by $a_1 \tensor \cdots \tensor a_n$.
  It is enough to fix any one of the possible bracketings (e.g., to make $\blank\tensor\blank$ always associate to the right).  

  Composition is then given by
  \begin{IEEEeqnarray*}{CCCCCC}
    & a_{11} \tensor \cdots \tensor a_{1k_1} & \tensor & \mathmakebox[4em]{\cdots} & \tensor & a_{n1} \tensor \cdots \tensor a_{nk_n} \\
    \xrightarrow{\mathmakebox[4em]{}} &
    (a_{11} \tensor \cdots \tensor a_{1k_1}) & \tensor & \mathmakebox[4em]{\cdots} & \tensor & (a_{n1} \tensor \cdots \tensor a_{nk_n}) \\
    \xrightarrow{\mathmakebox[4em]{f_1\tensor \cdots \tensor f_n}} &
    b_1 & \tensor & \mathmakebox[4em]{\cdots} & \tensor & b_n \\
    \xrightarrow{\mathmakebox[4em]{g}} &
    &&c\,,&&
  \end{IEEEeqnarray*}
  where the first arrow is induced from the normal monoidal coherences (exactly which ones depends on how we choose to interpret the iterated tensor product).
  \label{ExMonoidalCategoryAsMulticategory}
\end{example}

\section{Representable multicategories}

We call a multicategory \emph{representable} if it isomorphic to a multicategory that arises from a monoidal category as in Example \ref{ExMonoidalCategoryAsMulticategory}.
The next theorem gives a criterion for a multicategory to be representable.

\begin{theorem}[\cite{RepresentableMulticategories}]
  \label{TheRepresentableMulticategories}
  Let $\M$ be a multicategory and suppose that for each natural number $n$ and each sequence $a_1,\cdots,a_n$ of objects of $\M$ there is an object $\tensor \vec{a}$ and a multimorphism
  \[
    \pi_{\vec a} \from a_1,\cdots,a_n \to \tensor \vec a\,.
    \]
  Suppose that the $\pi_{\vec a}$ are \emph{strongly universal} in the sense that if 
  \begin{mathpar}
    b_1,\cdots b_k, c_1
    \and
    \cdots,c_l
  \end{mathpar}
  are two (possibly empty) lists of objects, and $d$ is an object, then any multimorphism
  \[
    f \from b_1,\cdots,b_k,a_1,\cdots,a_n,c_1,\cdots,c_l \to d
    \]
  factors uniquely through $\pi_{\vec a}$; i.e., there is a unique morphism
  \[
    \hat{f} \from b_1,\cdots,b_k,\tensor\vec{a},c_1,\cdots,c_l \to d
    \]
  such that
  \[
    f=(\id_{b_1},\cdots,\id_{b_k},\pi_{\vec a},\id_{c_1},\cdots,\id_{c_l});\hat{f}\,.
    \]
  Given objects $a,b$ of $\M$, define $a\tensor b = \tensor a,b$, and let $I$ be the object $\tensor\epsilon$, where $\epsilon$ is the empty list.  
  Then the operation $\blank\tensor\blank$ and $I$ make $\M$ into a monoidal category $\overline{\M}$ such that $\tilde{\overline{\M}}$ and $\M$ are isomorphic multicategories.
  \begin{itemize}
    \item $\blank\tensor\blank$ and $I$ are the monoidal product and unit of a monoidal category on $\M_1$, the underlying category of $\M$.
    \item For any sequence $a_1,\cdots a_n$ of objects of $\M$ there is a canonical isomorphism
      \[
        a_1 \tensor \cdots \tensor a_n \cong \tensor \vec a\,,
        \]
      for any bracketing of the left hand side, and the associators and unitors in $\M_1$ are induced from these isomorphisms.
    \item The set of multimorphisms $a_1,\cdots,a_n \to b$ is in bijection with the set of morphisms $a_1\tensor\cdots\tensor a_n \to b$ for $n\ge1$, and the set of multimorphisms $\to b$ is in bijection with the set of morphisms $I \to b$, and these bijections commute with the multicategory composition in $\M$ and the composition in $\M_1$.
  \end{itemize}
\end{theorem}

\begin{definition}
  A \emph{symmetric multicategory} is a multicategory $\M$ together with an action of the symmetric group on the sets $\M_n(a_1,\cdots,a_n;b)$ that respects composition.  
  In other words, for each natural number $n$, each multimorphism $f\from a_1,\cdots,a_n\to b$ and each permutation $\sigma$ of $\{1,\cdots,n\}$ there is a multimorphism
  \[
    \sigma_* f \from a_{\sigma(1))},\cdots,a_{\sigma(n)} \to b
    \]
  such that if $(a_{ij}\from i=1,\cdots,n), (b_i\from i = 1,\cdots,n)$ are objects, 
  \begin{mathpar}
    f_i \from a_{i1},\cdots,a_{i,k_i} \to b_i
    \and
    g \from b_1,\cdots,b_n \to c
  \end{mathpar}
  are multimorphisms, $\sigma_i$ is a permutation of $\{1,\cdots,k_i\}$, and $\tau$ is a permutation of $\{1,\cdots,n\}$, then
  \[
    ({\sigma_1}_* f_q,\cdots,{\sigma_n}_*f_n);(\tau_*g) = (\tau*(\sigma_1,\cdots,\sigma_n))_*((f_1,\cdots,f_n);g)\,,
    \]
  where $\tau*(\sigma_1,\cdots,\sigma_n)$ is the permutation of
  \[
    \{(1,1),\cdots,(1,k_1),\cdots,(n,1),\cdots,(n,k_n)\}
    \]
  that sends $(i,j)$ to $(\tau(i),\sigma_i(j))$.

  Moreover, we require that for any morphism $f\from a_1,\cdots,a_n\to b$ and permutations $\sigma,\tau$ of $\{1,\cdots,n\}$ we have
  \begin{mathpar}
    \sigma_*\tau_*f = (\sigma\circ\tau)_*f
    \and
    \id_* f = f
  \end{mathpar}
\end{definition}

\begin{example}
  Any multicategory arising from an ordinary category is symmetric.
\end{example}
\begin{example}
  A monoidal category is a symmetric multicategory if and only if it is a symmetric monoidal category.  
\end{example}

\section{Product and unit multicategories}

\begin{definition}
  Let $\M,\N$ be multicategories.  
  The \emph{product multicategory} $\M\times\N$ has, as objects, pairs $(a,b)$, where $a$ is an object of $\M$ and $b$ an object of $\N$.  
  The multimorphisms are given by
  \begin{mathpar}
    (\M\times\N)_n((a_1,b_1),\cdots,(a_n,b_n);(c,d)) = \M_n(a_1,\cdots,a_n;c)\times\N_n(b_1,\cdots,b_n;d)\,.
  \end{mathpar}
  Composition and the identity are similarly defined pointwise.
\end{definition}

\begin{definition}
  The \emph{unit multicategory} $1$ has a single object $I$, and for each $n$, the set
  \[
    1_n(I,\cdots,I;I)
    \]
  is a singleton.

  This is a representable multicategory; indeed, it may be identified with the usual unit monoidal category.
\end{definition}

\section{Multifunctors \& multinatural transformations}

\begin{definition}
  Let $\M,\N$ be multicategories.  
  A \emph{multifunctor} from $\M$ to $\N$ is a map $F$ from the objects of $\M$ to the objects of $\N$ together with, for each list $a_1,\cdots,a_n,b$ of objects of $\M$, a function
  \[
    \M_n(a_1,\cdots,a_n;b) \to \N_n(F a_1,\cdots,F a_n; F b)
    \]
  that commutes with the composition operator.
\end{definition}
  
\begin{definition}
  Given multicategories $\M,\N$ and multifunctors 
  \[
    F,G\from \M \to \N\,,
    \]
  a \emph{multinatural transformation} $\phi\from F \Rightarrow G$ is given by morphisms 
  \[
    \phi_a \from F a \to G a
    \]
  for each object $a$ of $\M$, such that if $f\from a_1,\cdots a_n\to b$ is any morphism in $\M$, then the following diagram commutes.
  \[
    \begin{tikzcd}[column sep=36pt]
      Fa_1,\cdots, Fa_n \arrow[r, "{\phi_{a_1},\cdots,\phi_{a_n}}"] \arrow[d, "Ff"']
        & Ga_1,\cdots, Ga_n \arrow[d, "Gf"] \\
      Fb \arrow[r, "\phi_b"]
        & Gb
    \end{tikzcd}
    \]
\end{definition}

\begin{proposition}
  If $\M,\N$ are monoidal categories, considered as multicategories, then multifunctors $\M\to\N$ are the same thing as lax monoidal functors.  
  Multinatural transformations are the same thing as monoidal natural transformations.
\end{proposition}

\begin{definition}
  Let $\M,\N$ be multicategories, where $\M$ is symmetric.  
  Then the collection of multifunctors $\M\to\N$ forms a multicategory.  
  A multimorphism $F_1,\cdots,F_n\Rightarrow G$, where $F_1,\cdots,F_n,G$ are multifunctors $\M\to \N$, is given by a family
  \[
    \phi_a \from F_1(a),\cdots,F_n(a) \to G(a)
    \]
  such that for any multimorphism $f\from a_1,\cdots,a_m\to b$ in $\M$, the diagram
  \[
    \begin{tikzcd}[column sep=46pt]
      F_1(a_1),\cdots,F_n(a_1),\cdots,F_1(a_m),\cdots,F_n(a_m) \arrow[r, "{\phi_{a_1},\cdots,\phi_{a_m}}"] \arrow[d, "\sigma_*"']
        & G(a_1),\cdots,G(a_m) \arrow[dd, "Gf"] \\
      F_1(a_1),\cdots,F_1(a_m),\cdots,F_n(a_1),\cdots,F_n(a_m) \arrow[d, "{F_1f,\cdots,F_nf}"']
        & \\
      F_1(b),\cdots,F_n(b) \arrow[r, "\phi_b"]
        & G(b)
    \end{tikzcd}
    \]
  commutes, where $\sigma$ is the map
  \begin{mathpar}
    (1,1),\cdots,(n,1),\cdots,(1,m),\cdots,(n,m) \to (1,1),\cdots,(1,m),\cdots,(n,1),\cdots,(n,m)
  \end{mathpar}
  sending $(i,j)$ to $(i,j)$, considered as a permutation of $\{1,\cdots,mn\}$.
\end{definition}

\section{Monoids in multicategories}

\begin{definition}
  \label{DefMonoidMulticategory}
Let $\M$ be a multicategory.  

  Then a \emph{monoid} in $\M$ is an object $a$ of $\M$ together with multimorphisms
  \begin{mathpar}
    m\from a,a\to a
    \and
    e \from \to a
  \end{mathpar}
  satisfying the following associativity and unitality laws.
  \begin{mathpar}
    \begin{tikzcd}
      a,a,a \arrow[r, "{m,\id_a}"] \arrow[d, "{\id_a,m}"']
        & a,a \arrow[d, "m"] \\
      a,a \arrow[r, "m"]
        & a
    \end{tikzcd}
    \and
    \begin{tikzcd}
      a \arrow[r,"\id_a"] \arrow[d, "{e_a,\id_a}"']
        & a \\
      a,a \arrow[ur, "m"']
        &
    \end{tikzcd}
    \and
    \begin{tikzcd}
      a \arrow[r, "\id_a"] \arrow[d, "{\id_a,e_a}"']
        & a \\
      a,a \arrow[ur, "m"']
        &
    \end{tikzcd}
  \end{mathpar}
\end{definition}

Note that a monoid in a multicategory $\M$ may equivalently be defined as a multifunctor $1 \to \M$ \cite[2.1.11]{Multicategories}.

\section{Categories enriched over multicategories}

\begin{definition}
  \label{DefEnrichedCategoryMulticategory}
  Let $\V$ be a multicategory.  
  Then a \emph{$V$-enriched category} $\C$ is given by a collection $\text{Ob}(\C)$ of objects together with, for each pair $a,b$ of objects, an object
  \[
    \C(a,b)
    \]
  of $\V$ and, for objects $a,b,c$ of $\C$, composition and identity multimorphisms
  \begin{mathpar}
    ;_{a,b,c} \from \C(a,b),\C(b,c) \to \C(a,c)
    \and
    \eta_a \from \to \C(a,a)
  \end{mathpar}
  that satisfy the following associativity and unitality laws for all objects $a,b,c,d$ of $\C$.
  \begin{mathpar}
    \begin{tikzcd}[column sep=12ex]
      \C(a,b),\C(b,c),\C(c,d) \arrow[r, "{;_{a,b,c},\id_{\C(c,d)}}"] \arrow[d, "{\id_{\C(a,b)},;_{b,c,d}}"']
        & \C(a,c),\C(c,d) \arrow[d, "{;_{a,c,d}}"] \\
      \C(a,b),\C(b,d) \arrow[r, "{;_{a,b,d}}"]
        & \C(a,d)
    \end{tikzcd}
    \and
    \begin{tikzcd}
      \C(a,b) \arrow[r, "{\id_{\C(a,b)}}"] \arrow[d, "{\eta_a,\id_{\C(a,b)}}"']
        & \C(a,b) \\
      \C(a,a),\C(a,b) \arrow[ur, "{;_{a,a,b}}"']
        &
    \end{tikzcd}
    \and
    \begin{tikzcd}
      \C(a,b) \arrow[r, "{\id_{\C(a,b)}}"] \arrow[d, "{\id_{\C(a,b)},\eta_b}"']
        & \C(a,b) \\
      \C(a,b),\C(b,b) \arrow[ur, "{;_{a,b,b}}"']
        &
    \end{tikzcd}
  \end{mathpar}
\end{definition}

\begin{remark}
  These definitions clearly generalize the same definitions for categories enriched over a monoidal category.

  In particular, a monoid in a multicategory $\M$ is the same thing as an $\M$-enriched category with a single object.
\end{remark}

\begin{remark}
  If $\V$ is a symmetric multicategory and $\C$ is a $\V$-enriched category, then we may define the \emph{opposite category} $\oppcat{\C}$ whose objects are the objects of $\C$ and where
  \[
    \oppcat\C(a,b)=\C(b,a)\,.
    \]
  Composition is defined by
  \[
    \C(b,a),\C(c,b) \xrightarrow{\tau_*}
    \C(c,b),\C(b,a) \xrightarrow{;_{c,b,a}}
    \C(c,a)\,,
    \]
  where $\tau$ is the permutation that transposes the two values.
\end{remark}

\section{Multicategory-enriched functors and natural transformations}

\begin{definition}
  Let $\C,\D$ be categories enriched over some multicategory $\V$.  
  An \emph{$\V$-enriched functor} $F \from \C \to \D$ is a map $F$ from the objects of $\C$ to the objects of $\D$ together with, for each, pair $a,b$ of objects of $\C$, a (unary) multimorphism
  \[
    F \from \C(a,b) \to \D(F(a),F(b))
    \]
  such that for all $a, b, c$ the following diagrams commute.
  \begin{mathpar}
    \begin{tikzcd}[column sep=48pt]
      \C(a, b), \C(b, c) \arrow[r, "{;_{a,b,c}}"] \arrow[d, "{F,F}"']
        & \C(a,c) \arrow[d, "F"] \\
      \D(F(a), F(b)), \D(F(b), F(c)) \arrow[r, "{;_{F(a),F(b),F(c)}}"]
        & \D(F(a), F(c))
    \end{tikzcd}
    \and
    \begin{tikzcd}
       \arrow[r, "\eta_a"] \arrow[dr, "\eta_{F(a)}"']
        & \C(a,a) \arrow[d, "F"] \\
      %
        & \D(F(a),F(a))
    \end{tikzcd}
  \end{mathpar}
\end{definition}

\begin{definition}
  Let $\C,\D$ be categories enriched over a multicategory $\V$ and let $F,G\from \C \to \D$ be $\V$-enriched functors.
  An \emph{$\V$-enriched natural transformation} $\phi\from F \Rightarrow G$ is given by a family of $0$-ary multimorphisms
  \[
    \phi_a \from \to \D(F(a),G(a))
    \]
  such that for all objects $a,b$ the following diagram commutes.
  \[
    \begin{tikzcd}
      \C(a,b) \arrow[r, "{F,\phi_b}"] \arrow[d, "{\phi_a,G}"']
        & \D(F(a),F(b)),\D(F(b),G(b)) \arrow[d, "{;_{F(a),F(b),G(b)}}"] \\
      \D(F(a),G(a)),\D(G(a),G(b)) \arrow[r, "{;_{F(a),G(a),G(b)}}"]
        & \D(F(a), G(b))
    \end{tikzcd}
    \]
\end{definition}

\section{The categories enriched over a symmetric multicategory form a multicategory}

\begin{definition}
  Let $\V$ be a symmetric multicategory.
  Given $\V$-enriched categories $\C_1,\cdots,\C_n,\D$, a multimorphism
  \[
    F\from \C_1,\cdots,\C_n \to \D
    \]
  is given by a function
  \[
    F \from \text{Ob}(\C_1)\times \cdots \times \text{Ob}(\C_n) \to \text{Ob}(\D)
    \]
  together with, for each $a_i,b_i\in \text{Ob}(\C_i)$, a multimorphism
  \[
    F \from \C_1(a_1,b_1) , \cdots , \C_n(a_n,b_n) \to \D(F(a_1,\cdots,a_n),F(b_1,\cdots,b_n))\,,
    \]
  such that the diagrams in Figure \ref{fig:VEnrichedMulticategoryFunctors} commute.  
  \begin{SidewaysFigure}
    \centering
    \begin{mathpar}
      \begin{tikzcd}[ampersand replacement=\&, column sep=68pt]
        \C_1(a_1,b_1),\C_1(b_1,c_1),\cdots,\C_n(a_n,b_n),\C_n(b_n,c_n) \arrow[r, "{;_{a_1,b_1,c_1},\cdots,;_{a_n,b_n,c_n}}"] \arrow[d, "\sigma_*"']
          \& \C_1(a_1,c_1),\cdots,\C_n(a_n,c_n) \arrow[dd, "F"] \\
        \C_1(a_1,b_1),\cdots,\C_n(a_n,b_n),\C_1(b_1,c_1),\cdots,\C_n(b_n,c_n) \arrow[d, "F"']
          \& \\
        \D(F(a_1,\cdots,a_n),F(b_1,\cdots,b_n)),\D(F(b_1,\cdots,b_n),F(c_1,\cdots,c_n)) \arrow[r, "{;_{F(a_1,\cdots,a_n),F(b_1,\cdots,b_n),F(c_1,\cdots,c_n)}}" yshift=0.5em]
          \& \D(F(a_1,\cdots,a_n),F(c_1,\cdots,c_n))
      \end{tikzcd}
      \\
      \vspace{25pt}
      \\
      \begin{tikzcd}[ampersand replacement=\&]
        { } \arrow[r, "{\eta_{a_1},\cdots,\eta_{a_n}}"] \arrow[dr, "{\eta_{F(a_1,\cdots,a_n)}}"']
          \& \C_1(a_1,a_1),\cdots,\C_n(a_n,a_n) \arrow[d, "F"] \\
        %
          \& \D(F(a_1,\cdots,a_n),F(a_1,\cdots,a_n))
      \end{tikzcd}
    \end{mathpar}
    \caption[Definition of multimorphisms between categories enriched in multicategories.]
    {The rules for preservation of composition and identity by multimorphisms of $\V$-enriched functors are similar to those for ordinary enriched functors.  
    Note that it is essential for the $\V$ to be a symmetric multicategory.  
    This generalizes the usual construction for categories enriched over a symmetric monoidal category.}
    \label{fig:VEnrichedMulticategoryFunctors}
  \end{SidewaysFigure}

  In the case $n=1$, this is the same thing as a $\V$-enriched functor from $\C_1$ to $\D$.
\end{definition}

\section{Change of base}

Let $\M,\N$ be multicategories, let $F\from \M \to \N$ be a multifunctor and let $\C$ be an $\M$-enriched category.  
Then we can form an $\N$-enriched category $F_*\C$ whose objects are the objects of $\C$ and where the morphisms are given by the formula
\[
  F_*\C(a,b) = F(\C(a,b))\,.
  \]
We get composition and identities by applying the multifunctor $F$ to the composition and identity multimorphisms in $\C$.
By functoriality of $F$, these composition and identities are associative and unital, meaning that $F_*\C$ is indeed an $\N$-enriched category.  

This process is called \emph{base change along $F$}.

\section{Closed multicategories}

\begin{definition}[\cite{ClosedMulticategories}]
  We say that a multicategory $\M$ is \emph{closed} if for any pair $a,c$ of objects, there exists an object
  \[
    \underline{\M}(a,c)
    \]
  and a multimorphism
  \[
    \ev_{a,c} \from a,\underline{\M}(a,c) \to c
    \]
  such that for any sequence $b_1,\cdots,b_n$ of objects of $\M$, the function
  \begin{IEEEeqnarray*}{rCcCc}
    \kappa_{a,b_1,\cdots,b_n,c} & \from & \M_n(b_1,\cdots,b_n;\underline{\M}(a,c)) & \to & \M_{n+1}(a,b_1,\cdots,b_n;c) \\
    && f & \mapsto & (\id_a,f);\ev_{a,c}
  \end{IEEEeqnarray*}
  is a bijection.
\end{definition}

\begin{proposition}[\cite{ClosedMulticategories}]
  If $\V$ is a closed multicategory, then $\V$ gives rise to the structure of a $\V$-enriched category on the underlying category $\V_1$ of $\V$.  
  We will also call this category $\V_1$, relying on context to distinguish the two.  
  The objects of $\V_1$ are the objects of $\V$, while the morphisms are given by
  \[
    \V_1(a,b) = \underline{\V}(a,b)\,.
    \]
\end{proposition}

If $\V$ is a closed multicategory and $\C_1,\cdots,\C_n$ are $\V$-enriched categories, then a functor $\C_1,\cdots,\C_n \to \V$ is given by a map $\text{Ob}(\C_1)\times\cdots\times\text{Ob}(\C_n) \to \text{Ob}(\V)$ and, for each $a_i,b_i\in \text{Ob}(\C_i)$, a multimorphism
\[
  \C_1(a_1,b_1),\cdots,\C_n(a_n,b_n) \to \underline{\V}(F(a_1,\cdots,a_n),F(b_1,\cdots,b_n))
  \]
By the definition of a closed multicategory, this is equivalent to providing a multimorphism
\[
  F(a_1,\cdots,a_n),\C_1(a_1,b_1),\cdots,\C_n(a_n,b_n) \to F(b_1,\cdots,b_n)\,.
  \]
In what follows, we will denote these multimorphisms (and their various permutations) with the letter $p$.

We have seen so far that multicategories provide us with a rather straightforward generalization of monoidal categories.  
We might ask the question, then: why make this generalization? 

To answer this question, we introduce some natural multicategories that are not representable.

\section{The multicategory of endoprofunctors}
\label{SecEndoprofunctors}

Let $\C,\D$ be ordinary categories.  
Recall that a \emph{profunctor} $F \from \C \pto \D$ is an ordinary functor $\oppcat\C \times \D\to\Set$.

More generally, if $\C,\D$ are enriched over some symmetric closed multicategory $\V$, then a $\V$-enriched profunctor $F \from \C \pto \D$ is a $\V$-enriched functor $\oppcat\C\times\D\to\V_1$.

Let $F_1,\cdots,F_n,G \from \oppcat \C \times \C \to \V_1$ be $\V$-enriched profunctors $\C\pto \C$, where $\C$ is a $\V$-enriched category.

We then define a multimorphism $\phi\from F_1,\cdots,F_n \Rightarrow G$ to be given by a family of multimorphisms
\[
  \phi_{a,b_1,\cdots,b_{n-1},c} \from F_1(a,b_1),\cdots,F_n(b_{n-1},c) \to G(a,c)
  \]
that make the diagrams in Figure \ref{FigExtranatural} commute.

A $0$-ary multimorphism $\to G$ is an ordinary enriched natural transformation $\C(a,c) \to G(a,c)$.
\begin{figure}
  \begin{mathpar}
    \begin{tikzcd}[column sep=8pt]
      {\C(a,a'),F_1(a',b_1),\cdots,F_n(b_{n-1},c),\C(c,c')} \arrow[r, "{\id,\phi_{a',\vec{b},c},\id}", xshift=-2ex, yshift=1pt, shorten=-1.8ex] \arrow[d, "{p,\id,\cdots,\id,p}"']

        & \C(a,a'),G(a',c),G(c,c') \arrow[d, "p"] \\
      F_1(a,b_1),\cdots,F_n(b_{n-1},c') \arrow[r, "{\phi_{a,\vec{b},c'}}"]
        & \G(a,c')
    \end{tikzcd}
    \and
    \begin{tikzcd}[column sep=-134pt]
      & F_1(a,b_1),\C(b_1,b_1'),F_2(b_1',b_2),\cdots,F_{n-1}(b_{n-2}',b_{n-1}),\C(b_{n-1},b_{n-1}'),F_n(b_{n-1}',c)
      \arrow[dr, "{\id,p,\cdots,p}"] \arrow[dl, "{p,\cdots,p,\id}"'] & \\
      F_1(a,b_1'),\cdots,F_n(b_{n-1}',c) \arrow[dr, "{\phi_{a,\vec{b'},c}}"'] & &
      F_1(a,b_1),\cdots,F_n(b_{n-1},c) \arrow[dl, "{\phi_{a,\vec{b},c}}"]  \\
        & G(a,c) &
    \end{tikzcd}
  \end{mathpar}
  \caption[Extranatural transformations between endoprofunctors.]
  {Extranatural transformations between endoprofunctors.
  The coherences we require on the multimorphisms between endoprofunctors are essentially the axioms for an extranatural transformation as in \cite{ExtranaturalTransformations}.}
  \label{FigExtranatural}
\end{figure}

We say that $\phi_{a,b_1,\cdots,b_{n-1},c}$ is \emph{natural} in $a$ and $c$ and \emph{extranatural} in the $b_i$.

We will often drop the component objects from $\phi$ and from the profunctors in question where they can be inferred from context.

We compose these multimorphisms pointwise.
The following proposition shows that this is indeed a well-defined composition.

\begin{proposition}
  Let $\V$ be a symmetric closed multicategory and let $\C$ be a $\V$-enriched category.
  Let $F_1,\cdots F_n,G_1,\cdots,G_m,H$ be profunctors $\C \pto \C$, and let $0=k_0,\cdots,k_m=n$ be a (not necessarily strictly) increasing subsequence of $\{0,\cdots,n\}$.  
  Let $\phi^{(i)} \from F_{k_i + 1},\cdots,F_{k_{i+1}} \to \G_i, \psi\from G_1,\cdots,G_m \to G$ be multimorphisms of profunctors.

  Then the family of multimorphisms
  \[
    F_1,\cdots,F_n
    \xrightarrow{\phi^{(1)},\cdots,\phi^{(m)}}
    G_1,\cdots,G_m
    \xrightarrow{\psi}
    H
    \]
  forms a multimorphism $F_1,\cdots,F_n\to H$.
\end{proposition}
\begin{proof}
  For the first condition (naturality), we have
  \[
    \begin{tikzcd}[column sep=60pt]
      \C,F_1,\cdots,F_n,\C \arrow[r, "{\id,\phi^{(1)},\cdots,\phi^{(m)},\id}"] \arrow[d, "{p,\id,\cdots,\id,p}"']
        & \C,\G_1,\cdots,\G_m,\C \arrow[r, "{\id,\psi,\id}"] \arrow[d, "{p,\id,\cdots,\id,p}"]
          & \C,H,\C \arrow[d, "p"] \\
      F_1,\cdots,F_n \arrow[r, "{\phi^{(1)},\cdots,\phi^{(m)}}"]
        & G_1,\cdots,G_m \arrow[r, "\psi"]
          & H
    \end{tikzcd}\,,
    \]
  where commutativity of the left hand square is the naturality condition on $\phi^{(1)}$ and $\phi^{(m)}$, while commutativity of the right hand square is the naturality condition for $\psi$.

  For the second condition (extranaturality), see Figure \ref{FigExtranaturalityComposition}.
\end{proof}

\begin{SidewaysFigure}
  \[
    \begin{tikzcd}[ampersand replacement=\&, row sep=80pt, column sep=23pt]
      %
        \& F_1,\cdots,F_n \arrow[dr, "{\phi^{(1)},\cdots,\phi^{(m)}}"]
          \&
            \& \\
      %
        \& F_1,\cdots,F_{k_1},\C,\cdots,\C,F_{k_{m-1}},\cdots,F_n \arrow[u, "{\id_{F_1},\cdots,\id_{F_{k_1}},p,\cdots,p,\id_{F_{k_{m-1}+1}},\cdots,\id_{F_n}}"' {description, near start, xshift=6ex}] \arrow[dr, "{\phi^{(1)},\id,\cdots,\id,\phi^{(n)}}" description]
          \& G_1,\cdots,G_m \arrow[dr, "\psi"]
            \& \\
      F_1,\C_1,F_2,\cdots,F_{n-1},\C,F_n \arrow[uur, "{\id,p,\cdots,p}", bend left=5] \arrow[ddr, "{p,\cdots,p,\id}"', bend right=5] \arrow[ur, "{\id,p,\cdots,p,\id_{\C},\cdots,\id_{\C},\id,\cdots,p}"' {description, xshift=6ex}] \arrow[dr, "{p,\cdots,p,\id,\id_{\C},\cdots,\id_{\C},p,\cdots,p,\id}" {description, xshift=6ex}]
        \&
          \& G_1,\C,G_2,\cdots,G_{m-1},\C,G_m \arrow[u, "{\id,p,\cdots,p}" description] \arrow[d, "{p,\cdots,p,\id}" description]
            \& H \\
      %
        \& F_1,\cdots,F_{k_1},\C,\cdots,\C,F_{k_{m-1}},\cdots,F_n \arrow[d, "{\id_{F_1},\cdots,\id_{F_{k_1}},p,\cdots,p,\id_{F_{k_{m-1}+1}},\cdots,\id_{F_n}}"' {description, near start, xshift=6ex}] \arrow[ur, "{\phi^{(1)},\id,\cdots,\id,\phi^{(n)}}" description]
          \& G_1,\cdots,G_m \arrow[ur, "\psi"']
            \& \\
      %
        \& F_1,\cdots,F_n \arrow[ur, "{\phi^{(1)},\cdots,\phi^{(m)}}"']
          \&
            \&
    \end{tikzcd}
    \]
  \caption[Proof that extranaturality is preserved by composition.]
  {Proof that extranaturality is preserved by composition.  
  Commutativity of the central square is by extranaturality of the $\phi^{(i)}$, while that of the four-cornered triangle at the right is by extranaturality of $\psi$.  
  The triangles on the left commute automatically, while the parallelograms at the top and the bottom commute by naturality of the $\phi^{(i)}$.}
  \label{FigExtranaturalityComposition}
\end{SidewaysFigure}

This composition is associative, because it is given pointwise by composition in $\V$, and its unit is given by the identity natural transformation.  
This gives us a multicategory.  

Suppose that $\V$ is the category of sets, so that the multimorphisms
\[
  F_1,\cdots,F_n \to G
  \]
are ordinary extranatural transformations
\[
  \phi_{a,\vec{b},c} \from F_1(a,b_1),\cdots,F_n(b_{n-1},c) \to G(a,c)\,.
  \]
Then the definition of the \emph{coend}
\[
  \int_{b_1,\cdots,b_{n-1}\from\C} F_1(a,b_1)\times\cdots\times F_n(b_{n-1},c)
  \]
is that it is universal among all objects admitting such an extranatural transformation out of them.  
It follows that in this case (and more generally, if $\V$ is a cocomplete monoidal category), that the multicategory of endoprofunctors on $\C$ is representable, with monoidal product given by
\[
  F\tensor G(a, c) = \int_{b\from \C} F(a,b) \times G(b,c)\,.
  \]
This is the usual notion of composition for $\Set$-enriched profunctors.  
However, it relies on the appropriate coends existing in $\Set$; for more generally $\V$-enriched profunctors, the multicategory of endoprofunctors on $\C$ need not be representable, even if $\V$ is a monoidal category.

We have only considered profunctors going from a category into itself.  
Profunctors in general form a structure called an \emph{fc-multicategory} \cite{Multicategories}, but we shall not be using this notion.

\section{Functors are a special case of profunctors}

The reason why we refer to a functors $F\from \oppcat\C,\D \to \V$ as a \emph{profunctors} $\C \pto \D$ is that they generalize ordinary functors $\C \to \D$.  
Specifically, if $F \from \C \to \D$ is a functor, then we can identify it with the profunctor
\[
  \tilde{F}(d,c) = \D(d, F(c)) \from \oppcat \D, \C \to \V\,.
  \]
This gives us an embedding of the monoidal category of endofunctors $\C \to \C$ into the multicategory of endoprofunctors $\C \pto \C$:

\begin{proposition}
  Let $\V$ be a closed symmetric multicategory and let $\C$ be a $\V$-enriched category.
  Let $F_1,\cdots,F_n, G\from \C \to \C$ be functors.  
  Then the set of natural transformations $F_1\circ\cdots\circ F_n \to G$ is naturally in bijection with the set of extranatural transformations
  \[
    \hat{F_1},\cdots,\hat{F_n} \to \hat{G}\,.
    \]
  \label{PropFunctorsIntoProfunctors}
\end{proposition}
\begin{proof}
  We have a natural multimorphism
  \begin{IEEEeqnarray*}{RL}
    &\C(a,F_1(b_1)),\cdots,\C(b_{n-1},F_n(c))\\
    \xrightarrow{\mathmakebox[6em]{\id,\cdots,F_1\circ\cdots\circ F_{n-1}}}&
    \C(a,F_1(b_1)),\cdots,\C(F_1\circ\cdots\circ F_{n-1}(b_{n-1}),F_1\circ\cdots\circ F_n(c))\\
    \xrightarrow{\mathmakebox[6em]{;^*}}&
    \C(a,F_1\circ\cdots\circ F_n(c))\,,
  \end{IEEEeqnarray*}
  which is natural in $a,c$ and extranatural in the $b_i$.

  If $\phi\from F_1\circ\cdots\circ F_n\to G$ is a natural transformation, then it gives rise (via postcomposition) to a natural transformation
  \[
    \C(a,F_1(\cdots(F_n(c))\cdots)) \to \C(a,G(c))\,,
    \]
  which we can compose with the multimorphism above to get the required extranatural transformation
  \[
    \C(a,F_1(b_1)),\cdots,\C(b_{n-1},F_n(c)) \to \C(a,G(c))\,.
    \]
  In the other direction, suppose that we have some extranatural transformation
  \[
    \phi_{a,\vec{b},c} \from \C(a,F_1(b_1)),\cdots,\C(b_{n-1},F_n(c)) \to \C(a,G(c))\,.
    \]
  Then we can take components of the form
  \[
    \phi_{a,F_1\circ\cdots\circ F_n(c),\cdots,F_n(c),c}\from \C(a,F_1\circ\cdots\circ F_n(c)),\cdots,\C(F_n(c),F_n(c)) \to \C(a,G(c))
    \]
  and compose with $\id,\eta,\cdots,\eta$ to get our natural transformation
  \[
    \C(a,F_a\circ\cdots\circ F_n(c)) \to \C(a,\G(c))\,.
    \]
  It is easy to check that these two constructions are inverses and that they respect composition of natural transformations.
\end{proof}

\section{Promonads are categories}

Since a monad was defined to be a monoid in the category of endofunctors on a category $\C$, we can define a \emph{promonad} to be a monoid in the multicategory of endoprofunctors on $\C$.

\begin{proposition}[See, e.g., \cite{Promonad}]
  \label{PropPromonadsAreCategories}
  Let $\V$ be a symmetric closed multicategory.  
  Let $\C$ be a $\V$-enriched category.  
  Then a promonad $\D\from \C\pto \C$ is the same thing as a $\V$-enriched category $\D$ together with an identity-on-objects functor $j\from \C \to \D$.
\end{proposition}
\begin{proof}
  This is a matter of unwrapping the definitions.

  Let $\D\from \C \pto \C$ be such a promonad.  
  So $\D$ is given by a $\V$-enriched functor $\D \from \oppcat \C,\C \to \V_1$, together with extranatural transformations
  \begin{mathpar}
    m_{a,b,c} \from \D(a,b),\D(b,c) \to \D(a,c)
    \and
    e_{a,b} \from \C(a,b) \to \D(a,b)
  \end{mathpar}
  such that the following diagrams commute (see Definition \ref{DefMonoidMulticategory}).
  \begin{mathpar}
    \begin{tikzcd}[column sep=12ex]
      \D(a,b),\D(b,c),\D(c,d) \arrow[r, "{m_{a,b,c},\id}"] \arrow[d, "{\id,m_{b,c,d}}"']
        & \D(a,c),\D(c,d) \arrow[d, "{m_{a,c,d}}"] \\
      \D(a,b),\D(b,d) \arrow[r, "{m_{a,b,d}}"]
        & \D(a,d)
    \end{tikzcd}
    \and
    \begin{tikzcd}
      \C(a,b),\D(b,c) \arrow[r, "p"] \arrow[d, "{e_{a,b},\id}"']
        & \D(a,c) \\
      \D(a,b),\D(b,c) \arrow[ur, "{m_{a,b,c}}"']
        &
    \end{tikzcd}
    \and
    \begin{tikzcd}
      \D(a,b),\C(b,c) \arrow[r, "p"] \arrow[d, "{\id,e_{a,b}}"']
        & \D(a,c) \\
      \D(a,b),\D(b,c) \arrow[ur, "{m_{a,b,c}}"']
        &
    \end{tikzcd}
  \end{mathpar}
  If we set $a=b$ in the second diagram and $b=c$ in the third, and compose with the identity multimorphisms $\eta$, then these are exactly the diagrams (see Definition \ref{DefEnrichedCategoryMulticategory}) for $\D$ to have the structure of a $\V$-enriched category on the collection of objects of $\C$!

  Then the full versions of the second and third diagrams give us our desired enriched functor $\C\to \D$.  
  It is the identity on objects and is the multimorphism $e_{a,b}$ on morphisms.

  We can show that this is indeed a functor using the diagram in Figure \ref{FigPromonadFunctorProof}.
  \begin{figure}[hbt]
    \[
      \begin{tikzcd}
        \C(a,b),\C(b,c) \arrow[rr, "{;_{a,b,c}}"] \arrow[dd, "{e_{a,b},e_{b,c}}"'] \arrow[dr, "{\id_{e_{b,c}}}" description]
          &
            & \C(a,c) \arrow[dd, "{e_{a,c}}"] \\
        %
          & \C(a,b),\D(b,c) \arrow[dl, "{e_{a,b},\id}" description] \arrow[dr, "p" description]
            & \\
        \D(a,b),\D(b,c) \arrow[rr, "{m_{a,b,c}}"]
          &
            & \D(a,c)
      \end{tikzcd}
      \]
    \caption[Promonads are identity-on-objects functors.]
    {Proof that the identity-on-objects functor arising from a promonad is indeed a functor.  
    The proof uses naturality of $e_{a,b}$ for commutativity of the large triangle at the top right.}
    \label{FigPromonadFunctorProof}
  \end{figure}
\end{proof}

Consider the case that $\D$ is an actual functor, so that $\D(a,b)=\C(a,F(b))$ for some endofunctor $F\from\C\to\C$.  
Then, by Proposition \ref{PropFunctorsIntoProfunctors}, a promonad structure on $\D$ is the same thing as a monad structure on $F$.  
If we consider $\D$ as a category, then the objects of $\D$ are the objects of $\C$, and morphisms from $a$ to $b$ are morphisms from $a$ to $F(b)$ in $\C$; i.e., Kleisli morphisms for $F$.

If we work the definitions through the proof of Proposition \ref{PropFunctorsIntoProfunctors}, then we see that the composition of morphisms $f\from a \to F(b)$ and $g \from b \to F(c)$ in $\D$ is given by the composite
\[
  a \xrightarrow{f}
  Fb \xrightarrow{Fg}
  FFc \to
  Fc\,,
  \]
where the rightmost arrow arises from the promonad structure on $\D$.  
In other words, $\D$ is precisely the Kleisli category for the monad $F$:

\begin{slogan}
  \label{SlogKleisli}
  The Kleisli category is the category we get by considering functors as profunctors.
\end{slogan}

\section{The multicategory of functors}

Let $\X$ be a monoidal category and let $\M$ be a multicategory.
We define a multicategory $[\X,\M]$ where the objects are ordinary functors
\[
  \X \to \M_1
  \]
and where multimorphisms $F_1,\cdots,F_n\to G$ are natural transformations
\[
  \phi_{x_1,\cdots,x_n} \from F_1(x_1),\cdots,F_n(x_n)\to G(x_1\tensor\cdots\tensor x_n)\,.
  \]
\begin{remark}
  Suppose that $\M$ is the category of sets, regarded as a multicategory through its Cartesian structure.  
  Let $x_1,\cdots,x_n,y_1,\cdots,y_p$ be objects of $\X$.  
  Then for any collection of functors
  \[
    F_1,\cdots,F_n,G_1,\cdots,G_p,H \from \X \to \Set\,,
    \]
  the set of natural transformations
  \[
    \phi_{\vec{x},\vec{y}} \from \prod_i F_i(x_i) \times \prod_j G_j(y_j) \to H(x_1\tensor\cdots\tensor x_n\tensor y_1\tensor\cdots\tensor y_p)
    \]
  may be written as the end
  \[
    \int_{\vec{x},\vec{y}} \left[\prod_i F_i(x_i) \times \prod_j G_j(y_j) ,  H\left(\Tensor_i x_i \tensor \Tensor_j y_j\right)\right]\,.
    \]
  We may then perform some co/end calculus (See the similar computation in \cite{Pisani}, but note that that version is not quite sufficient to prove representability according to Theorem \ref{TheRepresentableMulticategories}).
  \begin{IEEEeqnarray*}{Cl}
    &\int_{\vec{x},\vec{y}} \left[\prod_i F_i(x_i) \times \prod_j G_j(y_j) ,  H\left(\Tensor_i x_i \tensor \Tensor_j y_j\right)\right]\\
    \cong&\int_{\vec{x},z,\vec{y}} \left[\X\left(\Tensor_i x_i,z\right),\left[\prod_i F_i(x_i) \times \prod_j G_j(y_j) ,  H\left(z \tensor \Tensor_j y_j\right)\right]\right]\\
    \cong&\int_{\vec{x},z,\vec{y}} \left[\prod_i F_i(x_i) \times \X\left(\Tensor_i x_i,z\right) \times \prod_j G_j(y_j),H\left(z \tensor \Tensor_j y_j\right)\right]\\
    \cong&\int_{z,\vec{y}}\left[\int^{\vec{x}} \left(\prod_i F_i(x_i) \times \X\left(\Tensor_i x_i,z\right)\right)\times\prod_j G_j(y_j),H\left(z\tensor\Tensor_j y_j\right)\right]
  \end{IEEEeqnarray*}
  In other words, this multicategory is representable by the Day convolution that we met in Definition \ref{DefDayConvolution}:
  \[
    (F\tensor_{\text{Day}} G)(z) = \int^{x,y} F(x) \times G(y)\times \X(x\tensor y,z)\,.
    \]
  However, this multicategory is not representable in general, particularly in the cases when we are working with enriched multicategories (not defined here), where the enriching multicategory is not cocomplete, or when the category $\M$ is not the enriching multicategory.
\end{remark}

\section{Monoids on functors are multifunctors}
It might seem strange that the objects of the multicategory of functors are ordinary functors rather than multifunctors.  
We appear to have ignored the monoidal structure of $\X$ and the multicategory structure of $\M$.  

One way to make sense of this fact is to note that an object of a category $\C$ is the same thing as a functor
\[
  1 \to \C\,.
  \]
In the same way, perhaps the correct way to think of an `element' of a multicategory is that it is a \emph{multifunctor}
\[
  1 \to \M\,;
  \]
i.e., a monoid in $\M$.

Then the following proposition tells us that the `elements' of $\X \to \M$ in this sense are the multifunctors.

\begin{proposition}
  Let $\X$ be a monoidal category and let $\M$ be a multicategory.  
  Then a monoid in $[\X,\M]$ is the same thing as a multifunctor $\X \to \M$.
  \label{PropMonoidInXMIsMultifunctorXToM}
\end{proposition}

This can be proved by setting $\N=1$ in the following stronger result.

\begin{proposition}[{\cite[2.8]{Pisani}}]
  Let $\X$ be a monoidal category and let $\M,\N$ be multicategories.  
  \label{PropPisani}
  Then a multifunctor $\N \to [\X,\M]$ is the same thing as a multifunctor $\N\times\X \to \M$.
\end{proposition}

\section{Two perspectives on monoids in $\Set$}

We now come to our main result of the chapter.  
We will approach it from an oblique perspective.  
First note the following two rather different generalizations of the notion of an internal monoid in $\Set$.

\begin{enumerate}
  \item A monoid in $\Set$ may be regarded as a lax monoidal functor (i.e., a multifunctor) $1 \to \Set$.  
    This generalizes to arbitrary lax monoidal functors $\X \to \Set$, for monoidal categories $\X$.
  \item A monoid in $\Set$ may also be regarded as a category with a single object.  
    This generalizes to arbitrary categories.
\end{enumerate}

We shall now attempt to unify these into a single grand unifying generalization of a monoid.
From Proposition \ref{PropPromonadsAreCategories}, we know that a category with one object is the same thing as a monoid in the category $\Endoprof_\Set(*)$, where $*$ is the category with a single object and only an identity morphism.

We can clearly generalize this to the idea of a monoid in $\Endoprof_\Set(\C)$ for an arbitrary category $\C$.  
This then generalizes to the universal idea of a multifunctor
\[
  \X \to \Endoprof_\Set(\C)\,,
  \]
(which we might call a \emph{parametric promonad on $\C$ parameterized by $\X$}), which generalizes both lax monoidal functors $\X \to \Set$ (when $\C=*$) and $\Set$-enriched categories (when $\X=1$).

However, we can also do things the other way round.  
From Proposition \ref{PropMonoidInXMIsMultifunctorXToM}, a lax monoidal functor $\X \to \Set$ is a monoid in the multicategory $[\X,\Set]$.  
This is the same thing as an $[\X,\Set]$-enriched category with a single object, so another way of generalizing monoids in $\Set$ is to generalize them to monoids in the multicategory
\[
  \Endoprof_{[\X,\Set]}(\C)
  \]
for some monoidal category $\X$ and some $[\X,\Set]$-enriched category $\C$.

This generalizes categories in the case that $\X=1$.  
It generalizes monoidal functors $\X \to \Set$ in the case that $\C$ is the $[\X,\Set]$-enriched category $*_{[\X,\Set]}$ with a single object $()$, where the morphisms $() \to ()$ are given by the functor $[I,\blank]$, for $I$ the monoidal unit in $\X$, this being the initial object in $[\X,\Set]$.

Our main result will tell us that it doesn't actually matter which way round we choose: these two ways of unifying the two generalization of a monoid in fact give the same result.  
The only ingredient we are missing is an appropriate change of base to move from ordinary $\Set$-enriched categories to $[\M,\Set]$-enriched categories.

\begin{definition}
  Let $\X$ be a monoidal category.  
  We have a multifunctor
  \[
    \X \to [\Set,\Set]
    \]
  given by
  \[
    x \mapsto \X(I,x) \times \blank\,.
    \]
  By Proposition \ref{PropPisani}, this may equivalently be given as a multifunctor
  \[
    \partial_\X\from\Set \to [\X,\Set]
    \]
  that sends a set $A$ to the functor
  \[
    \X(I,\blank)\times A\,.
    \]
\end{definition}

The important property of this particular multifunctor is as follows.

\begin{proposition}
  If $\C_1,\cdots,\C_n$ are categories, then $[\X,\Set]$-enriched functors ${\partial_\X}_*\C_1,\cdots,{\partial_\X}_*\C_n\to[\X,\Set]$ are the same thing as ordinary functors from $\C_1\times\cdots\times\C_n$ to $[\X,\Set]$.
\end{proposition}
\begin{proof}
  Let $\C_1,\cdots,\C_n$ be categories.  
  An $[\X,\Set]$-enriched functor
  \[
    F \from {\partial_\X}_*\C_1,\cdots,{\partial_\X}_*\C_n \to [\X,\Set]
    \]
  is given by a map
  \[
    F\from \text{Ob}(\C_1)\times\cdots\times\text{Ob}(\C_n) \to \text{Ob}([\X,\Set])\,,
    \]
  together with, for all objects $a_i,b_i$ of $\C_i$, a multimorphism
  \[
    {\partial_X}_*\C_1(a_1,b_1),\cdots,{\partial_X}_*\C_n(a_n,b_n),F(a_1,\cdots,a_n)\to F(b_1,\cdots,b_n)\,;
    \]
  i.e., a natural transformation
  \begin{mathpar}
    \left(\prod_i \X(I,x_i)\times\C(a_i,b_i)\right) \times F(a_1,\cdots,a_n)(y) \to F(b_1,\cdots,b_n)(x_1\tensor\cdots\tensor x_n\tensor y)\,.
  \end{mathpar}
  By the Yoneda lemma, such a natural transformation is the same thing as a natural transformation
  \[
    \prod_i \C(a_i,b_i) \times F(a_1,\cdots,a_n)(y) \to F(b_1,\cdots,b_n)(I \tensor \cdots \tensor I \tensor y)\,;
    \]
  i.e., a natural transformation
  \[
    \C(a_1,b_1)\times\cdots\times\C(a_n,b_n) \times F(a_1,\cdots,a_n)(y) \to F(b_1,\cdots,b_n)(y)\,.
    \]
  But this is precisely the data of an ordinary functor $\C_1\times\cdots\times\C_n \to [\X,\Set]$.

  By naturality of the Yoneda transformation (and the left unitor), this process preserves and reflects the property of respecting composition and units.
  \label{ProppartialProperty}
\end{proof}

\begin{theorem}[`Stokes's Theorem']
  \label{StokessTheorem}
  Let $\X$ be a monoidal category and let $\C$ be a category.  
  Then we have an isomorphism of multicategories
  \[
    [\X,\Endoprof_{\Set}(\C)] \cong \Endoprof_{[X,\Set]}({\partial_\X}_*\C)\,.
    \]
\end{theorem}
\begin{proof}
  Let $F \from \X\times\oppcat\C\times\C \to \Set$ be an ordinary functor.  
  We may view $F$ either as the object
  \[
    F(x, \blank, \blank) \from \X \to \Endoprof_\Set(\C)_1
    \]
  of $[\X,\Endoprof_{\Set}(\C)]$ or, by Proposition \ref{ProppartialProperty}, as the object
  \[
    F(\blank, a, b) \from \oppcat \C \times \C \to [\X,\Set]
    \]
  of $\Endoprof_{[\X,\Set]}({\partial_\X}_*\C)$.
  Moreover, every object of each of the two categories arises in such a way.  
  Our aim is to show that the two categories give rise to identical notions of multimorphisms between such $F$.

  Let $F_1,\cdots,F_n,G \from \X \times \oppcat\C\times\C\to \Set$ be functors.  
  Considering the $F_i$ as objects of $[\X,\Endoprof_\Set(\C)]$, a multimorphism $F_1,\cdots,F_n\to G$ is given by a transformation
  \[
    F_1(x_1,\blank,\blank),\cdots, F_n(x_n,\blank,\blank) \to G(x_1\tensor\cdots\tensor x_n,\blank,\blank)\,;
    \]
  natural in the $x_i$ i.e., a transformation
  \[
    F_1(x_1,a,b_1)\times\cdots\times F_n(x_n,b_{n-1},c) \to G(x_1\tensor\cdots\tensor x_n,a,c)\,.
    \]
  natural in the $x_i,a,c$ and extranatural in the $b_i$.

  A multimorphism $\to G$ is given by a multimorphism
  \[
    \to G(I,\blank,\blank)\,;
    \]
  i.e., a morphism
  \[
    \C(a,c) \to G(I,a,c)\,.
    \]

  Now let us consider the $F_i,G$ as objects of $\Endoprof_{[\X,\Set]}({\partial_\X}_*\C)$.  
  A multimorphism $F_1,\cdots,F_n\to G$ is given by a transformation
  \[
    F_1(\blank,a,b_1),\cdots,F_n(\blank,b_{n-1},c) \to G(\blank,a,c)
    \]
  natural in $a,c$ and extranatural in the $b_i$; i.e., a transformation
  \[
    F_1(x_1,a,b_1)\times\cdots\times F_n(x_n,b_{n-1},c) \to G(x_1\tensor\cdots\tensor x_n,a,c)
    \]
  natural in $a$, $c$ and the $x_i$ and extranatural in the $b_i$.  

  A multimorphism $\to G$ is given by an extranatural transformation
  \[
    \partial_\X(\C(a,c)) \to G(\blank,a,c)\,;
    \]
  i.e., a natural transformation
  \[
    \X(I,x)\times\C(a,c) \to \G(x,a,c)\,,
    \]
  which by the Yoneda lemma is the same thing as a natural transformation
  \[
    \C(a,c) \to \G(I,a,c)\,.
    \]
  Thus, the two multicategories are isomorphic.
\end{proof}

Now consider the case that we have a parametric monad $\blank.\blank\from\X \times \C \to \C$ on a category $\C$.  
By considering functors as profunctors, we may identify this with a multifunctor $\X \to \Endoprof_\Set(\C)$, which is the same thing as a monoid in $[\X,\Endoprof_\Set(\C)]$.
Then, by Theorem \ref{StokessTheorem}, we may identify this multifunctor with a monoid in $\Endoprof_{[\X,\Set]}({\partial_\X}_*\C)$; i.e., an $[\X,\Set]$-enriched promonad on ${\partial_\X}_*\C$.

But now, by Proposition \ref{PropPromonadsAreCategories}, this promonad is the same thing as an $[\X,\Set]$-enriched category that has the same objects as $\C$ and admits an identity-on-objects $[\X,\Set]$-enriched functor out of ${\partial_X}_*(\C)$.  

The objects of this $[\X,\Set]$-enriched category are the objects of $\C$.
By working the definitions through the proofs of Proposition \ref{PropFunctorsIntoProfunctors} and Theorem \ref{StokessTheorem}, we see that the object of morphisms from $a$ to $b$ is the functor
\[
  x \mapsto \C(a,x.b)\from \X \to \Set\,,
  \]
and that composition of morphisms is the multimorphism
\[
  \C(a,x.b) \times \C(b,y.c) \to \C(a,(x\tensor y).c)
  \]
in $[\X,\Set]$ given by sending morphisms $f\from a \to x.b$, $g\from b \to x.c$ to the composite
\[
  a
  \xrightarrow{f}
  x.b
  \xrightarrow{x.g}
  x.y.c \xrightarrow{m}
  (x\tensor y).c\,,
  \]
which is precisely the definition of composition in the \Mellies category.

We get a new analogue of Slogan \ref{SlogKleisli}.

\begin{slogan}
  The \Mellies category is precisely the $[\X,\Set]$-enriched category that we get by considering functors as profunctors.
\end{slogan}

\documentclass{report}[11pt]
\let\FEWFONTS=1

\usepackage[utf8]{inputenc}

\usepackage{graphicx} % support the \includegraphics command and options

\usepackage{parskip} % Activate to begin paragraphs with an empty line rather than an indent

%%% PACKAGES
\usepackage{booktabs} % for much better looking tables
\usepackage{array} % for better arrays (eg matrices) in maths
\ifdefined\BEAMER
\else
\usepackage{paralist} % very flexible & customisable lists (eg. enumerate/itemize, etc.)\prefix\t$.
\fi
\usepackage{verbatim} % adds environment for commenting out blocks of text & for better verbatim
\ifdefined\BEAMER
\else
\ifdefined\THESIS
\usepackage{subcaption}
\else
\usepackage{subfig} % make it possible to include more than one captioned figure/table in a single float
\fi
\fi
\usepackage{mathtools} % for the all important \coloneqq symbol
\usepackage{hyperref} % for hyperreferences
\usepackage{IEEEtrantools} % for \IEEEeqnarray
\usepackage{pbox} % for \pbox
\usepackage{multirow,bigdelim} % for \multirow
\usepackage{lettrine} % For the drop cap
\usepackage{mathpartir} % for \inferrule, \inferrule* and the mathpar environment
\usepackage{listings}

\usepackage{caption}
\captionsetup{singlelinecheck=off}

\ifdefined\NOTARTICLE
\else

%%% ToC (table of contents) APPEARANCE
\usepackage[nottoc,notlof,notlot]{tocbibind} % Put the bibliography in the ToC
\usepackage[titles,subfigure]{tocloft} % Alter the style of the Table of Contents
\renewcommand{\cftsecfont}{\rmfamily\mdseries\upshape}
\renewcommand{\cftsecpagefont}{\rmfamily\mdseries\upshape} % No bold!

\fi

%% Font things %%
\usepackage{amssymb}
\usepackage{cmll} % Linear logic symbols!
\ifdefined\FEWFONTS
\else
\usepackage{bm} % for bold Greek letters
\fi
\usepackage{stmaryrd}
\usepackage{bbm}

%% Get the sqsubsetneqq character from the mathabx package
\DeclareFontFamily{U}{mathb}{\hyphenchar\font45}
\DeclareFontShape{U}{mathb}{m}{n}{
      <5> <6> <7> <8> <9> <10> gen * mathb
      <10.95> mathb10 <12> <14.4> <17.28> <20.74> <24.88> mathb12
      }{}
\DeclareSymbolFont{mathb}{U}{mathb}{m}{n}

\DeclareMathSymbol{\sqsubsetneq}    {3}{mathb}{"88}
\DeclareMathSymbol{\varsqsubsetneq} {3}{mathb}{"8A}
\DeclareMathSymbol{\varsqsubsetneqq}{3}{mathb}{"92}
\DeclareMathSymbol{\sqsubsetneqq}   {3}{mathb}{"90}

%% Get the left and right moons from the wasysym package

\DeclareFontFamily{U}{wasy}{}
\DeclareFontShape{U}{wasy}{m}{n}{ <5> <6> <7> <8> <9> gen * wasy
      <10> <10.95> <12> <14.4> <17.28> <20.74> <24.88>wasy10  }{}
\DeclareFontShape{U}{wasy}{b}{n}{ <-10> sub * wasy/m/n
 <10> <10.95> <12> <14.4> <17.28> <20.74> <24.88>wasyb10 }{}
\DeclareFontShape{U}{wasy}{bx}{n}{ <-> sub * wasy/b/n}{}

\def\wasyfamily{\fontencoding{U}\fontfamily{wasy}\selectfont}
\def\leftmoon   {\mbox{\wasyfamily\char36}}
\def\rightmoon  {\mbox{\wasyfamily\char37}}

%% Lists %%
\usepackage{enumerate}

%% Graphics %%
\usepackage{tikz}
\usetikzlibrary{cd}
\usetikzlibrary{patterns}
\usetikzlibrary{calc}
\usetikzlibrary{decorations.pathmorphing}
\usetikzlibrary{positioning}

\tikzset{inlinearrows/.style={anchor=base,baseline,x=0.6\baselineskip,y=0.6\baselineskip}}

\ifdefined\BEAMER
\else

%% Theorems! %%
\usepackage{amsthm}
\theoremstyle{plain} % Theorems, lemmas, propositions etc.
\newtheorem{theorem}{Theorem}[section]
\newtheorem{lemma}[theorem]{Lemma}
\newtheorem{proposition}[theorem]{Proposition}
\newtheorem{corollary}[theorem]{Corollary}
\newtheorem{fact}[theorem]{Fact}
\newtheorem{construction}[theorem]{Construction}
\theoremstyle{definition} % Definitions etc.  
\newtheorem{definition}[theorem]{Definition}
\newtheorem{notation}[theorem]{Notation}
\theoremstyle{remark} % Remarks
\newtheorem{remark}[theorem]{Remark}
\newtheorem{remarks}[theorem]{Remarks}
\newtheorem{example}[theorem]{Example}
\newtheorem{question}[theorem]{Question}
\newtheorem{slogan}[theorem]{Slogan}

\newtheoremstyle{note} {3pt} {3pt} {\itshape} {} {\itshape} {:} {.5em} {} % For short notes
\theoremstyle{note}
\newtheorem{note}[theorem]{Note}

\fi

%% Exercises and answers %%
\usepackage{answers}

\newtheoremstyle{exercisestyle}% name
  {6pt}   % ABOVESPACE
  {6pt}   % BELOWSPACE
  {\itshape}  % BODYFONT
  {0pt}       % INDENT (empty value is the same as 0pt)
  {\bfseries} % HEADFONT
  {.}         % HEADPUNCT
  {3pt} % HEADSPACE
  {}          % CUSTOM-HEAD-SPEC

\theoremstyle{exercisestyle}
\newtheorem{exercise}{Exercise}
\newtheorem{answerthm}{Exercise}

\Newassociation{answer}{answerthm}{answers}
\newcommand{\answerthmparams}{}

%% Changes to enumerate things so they look better %%\sigma$

\makeatletter
\def\enumfix{%
\if@inlabel
 \noindent \par\nobreak\vskip-\topsep\hrule\@height\z@
\fi}

\let\olditemize\itemize
\def\itemize{\enumfix\olditemize}
\let\oldenumerate\enumerate
\def\enumerate{\enumfix\oldenumerate}

%% Random crap %%
\usepackage{xifthen}

\makeatletter
\def\thm@space@setup{%
  \thm@preskip=\parskip \thm@postskip=0pt
}
\makeatother

\makeatletter
\newcommand*{\relrelbarsep}{.386ex}
\newcommand*{\relrelbar}{%
  \mathrel{%
    \mathpalette\@relrelbar\relrelbarsep
  }%
}
\newcommand*{\@relrelbar}[2]{%
  \raise#2\hbox to 0pt{$\m@th#1\relbar$\hss}%
  \lower#2\hbox{$\m@th#1\relbar$}%
}
\providecommand*{\rightrightarrowsfill@}{%
  \arrowfill@\relrelbar\relrelbar\rightrightarrows
}
\providecommand*{\leftleftarrowsfill@}{%
  \arrowfill@\leftleftarrows\relrelbar\relrelbar
}
\providecommand*{\xrightrightarrows}[2][]{%
  \ext@arrow 0359\rightrightarrowsfill@{#1}{#2}%
}
\providecommand*{\xleftleftarrows}[2][]{%
  \ext@arrow 3095\leftleftarrowsfill@{#1}{#2}%
}
\makeatother

\newcommand{\catname}[1]{{\normalfont\textbf{#1}}}
\newcommand{\Rings}{\catname{CRing}}
\newcommand{\CAT}{\catname{CAT}}
%\newcommand{\Top}{\catname{Top}}
\newcommand{\Set}{\catname{Set}}
\newcommand{\Cat}{\catname{Cat}}
\newcommand{\MonCat}{\catname{MonCat}}
\newcommand{\SymmMonCat}{\catname{SymmMonCat}}
\newcommand{\Cont}{\catname{Cont}}
\newcommand{\Sch}{\catname{Sch}}
\newcommand{\Rel}{\catname{Rel}}
\newcommand{\Coh}{\catname{Coh}}
\newcommand{\Inj}{\catname{Inj}}
\newcommand{\Dcpo}{\catname{Dcpo}}
\newcommand{\Mod}[1][]{\ifthenelse{\isempty{#1}}{\catname{Mod}}{#1\catname{mod}}}
\DeclareMathOperator{\sh}{Sh}
\newcommand{\Sh}[1][]{\ifthenelse{\isempty{#1}}{\sh}{\sh(#1)}}
\newcommand{\map}[3]{#2\xrightarrow{#1} #3}
\newcommand*\from{\colon}
\newcommand*\bigto{\Rightarrow}
\newcommand{\cmap}[3]{#1\from{}#2\to{}#3}
\newcommand\oppcat[1]{#1^{\mathrm{op}}}
\newcommand{\object}{\colon}
\DeclareRobustCommand{\vmap}[3] {\begin{tikzcd} #2 \arrow[d, "#1"] \\ #3 \end{tikzcd}}
\newcommand{\partref}[1]{(\ref{#1})}
\newcommand{\intgrpd}[4] {#1 \xrightrightarrows[#3]{#4} #2}
\DeclareRobustCommand{\bigintgrpd}[4] {\begin{tikzcd}[ampersand replacement=\&] #1 \arrow[r, shift left=0.5ex, "#3"] \arrow[r, shift right=0.5ex, "#4"'] \& #2 \end{tikzcd}}

\usepackage{xspace}

\newcommand{\etale}{\'{e}tale\xspace}
\newcommand{\Etale}{\'{E}tale\xspace}

\def \inv {^{-1}}

\DeclareMathOperator{\id}{id}
\DeclareMathOperator{\op}{op}
\DeclareMathOperator{\pr}{pr}
\DeclareMathOperator{\inj}{in}
\DeclareMathOperator{\pre}{{pre}}
\DeclareMathOperator{\et}{{\acute{e}t}}

\DeclareMathOperator{\Hom}{Hom}
\DeclareMathOperator{\Spec}{Spec}

\DeclareMathOperator{\ol}{ol}

\def\presuper#1#2%
  {\mathop{}%
   \mathopen{\vphantom{#2}}^{#1}%
   \kern-\scriptspace%
   #2}
\def\presub#1#2%
  {\mathop{}%
   \mathopen{\vphantom{#2}}_{#1}%
   \kern-\scriptspace%
   #2}

\newsavebox{\overlongequation}
\newenvironment{longdiagram}
 {\begin{displaymath}\begin{lrbox}{\overlongequation}$\displaystyle}
 {$\end{lrbox}\makebox[0pt]{\usebox{\overlongequation}}\end{displaymath}}

%% Our things %%

\newcommand{\neggame}[1]{\presuper{\perp}{#1}}
\newcommand{\tensor}{\otimes}
\newcommand{\Tensor}{\bigotimes}
\newcommand{\sequoid}{\oslash}
\newcommand{\varsequoid}{\vartriangleleft}
\renewcommand{\implies}{\multimap}
\newcommand{\iimpl}{\Longrightarrow}
\newcommand{\comp}[2]{#1 \circ #2}
\newcommand{\icomp}[2]{\comp{#1}{#2}}
\newcommand{\cprd}{\sqcup}
\newcommand{\bigcprd}{\bigsqcup}
\newcommand{\G}{\mathcal G}
\newcommand{\W}{\mathcal W}
\newcommand{\suchthat}{\;\colon\;}
\newcommand{\varsuchthat}{\;\mid\;}
\newcommand{\esuchthat}{\;.\;}
\newcommand{\OP}{\{O,P\}}
\newcommand{\QA}{\{Q,A\}}
\renewcommand{\L}{\mathcal L}
\newcommand{\F}{\mathcal F}
\newcommand{\U}{\mathcal U}
\newcommand{\s}{\mathfrak s}
\renewcommand{\t}{\mathfrak t}
\renewcommand{\u}{\mathfrak u}
\renewcommand{\d}{\mathfrak d}
\newcommand{\e}{\mathfrak e}
\newcommand{\emptyplay}{\epsilon}
\newcommand{\bracketed}[1]{\left({#1}\right)}
\newcommand{\bneggame}[1]{{\bracketed{\neggame{#1}}}}
\newcommand{\prefix}{\sqsubseteq}
\newcommand{\ppprefix}{\sqsubset}
\newcommand{\pprefix}{\sqsubsetneqq}
\renewcommand{\ss}{\mathbf{s}}
\newcommand{\bN}{\mathbb{N}}
\newcommand{\bC}{\mathbb{C}}
\newcommand{\bB}{\mathbb{B}}
\newcommand{\bP}{\mathbb{P}}
\newcommand{\pfun}{\rightharpoonup}
\newcommand{\grel}[1]{\underline{#1}}
\DeclareMathOperator{\length}{length}
\renewcommand{\b}{\mathfrak b}
\renewcommand{\r}{\mathfrak r}
\newcommand{\bbeta}{{\bm{\beta}}}
\newcommand{\st}{{\Sigma^*}}
\let\sec\S
\renewcommand{\S}{{\mathfrak{S}}}
\DeclareMathOperator{\cc}{cc}
\DeclareMathOperator{\subs}{subs}
\DeclareMathOperator{\ret}{ret}
\DeclareMathOperator{\zz}{zz}
\newcommand{\aaa}{\mathbf{a}}
\newcommand{\bbb}{\mathbf{b}}
\newcommand{\ccc}{\mathbf{c}}
\newcommand{\ddd}{\mathbf{d}}
\newcommand{\B}{\mathcal B}
\newcommand{\BB}{\mathbf B}
\renewcommand{\H}{\mathcal H}
\DeclareMathOperator{\assoc}{assoc}
\DeclareMathOperator{\lunit}{lunit}
\DeclareMathOperator{\runit}{runit}
\DeclareMathOperator{\dom}{dom}
\DeclareMathOperator{\sym}{sym}
\newcommand{\braid}{\sym}
\newcommand{\blank}{\,\underline{\hspace{1.5ex}}\,}
\DeclareMathOperator{\cn}{cn}
\newcommand{\impliescn}{\protect\overset{\cn}{\implies}}
\newcommand{\C}{{\mathcal{C}}}
\newcommand{\D}{{\mathcal{D}}}
\newcommand{\E}{{\mathcal{E}}}
\newcommand{\V}{{\mathcal{V}}}
\newcommand{\EE}{{\mathbf{E}}}
\DeclareMathOperator{\ev}{ev}
\newcommand{\der}{{\mathtt{der}}}
\newcommand{\mult}{{\mathtt{mult}}}
\DeclareMathOperator{\wk}{wk}
\newcommand{\toisom}{{\xrightarrow{\cong}}}
\DeclareMathOperator{\passoc}{{\mathsf{passoc}}}
\DeclareMathOperator{\pcomm}{{\mathsf{pcomm}}}
\DeclareMathOperator{\run}{{\mathsf{r}}}
\DeclareMathOperator{\lun}{{\mathsf{l}}}
\newcommand{\fcoal}[1]{{\leftmoon #1 \rightmoon}}
\DeclareMathSymbol{\co}{\mathord}{operators}{"3C}
\DeclareMathSymbol{\nw}{\mathord}{operators}{"3E}
\newcommand{\T}{\mathfrak{T}}
\renewcommand{\subset}{\subseteq}
\newcommand{\Ord}{\catname{Ord}}
\newcommand{\FS}{\mathcal{FS}}
\DeclareMathOperator{\rank}{rank}
\DeclareMathOperator{\dist}{{\mathsf{dist}}}
\DeclareMathOperator{\dec}{{\mathsf{dec}}}
\DeclareMathOperator{\str}{str}
\DeclareMathOperator{\weak}{weak}
\DeclareMathOperator{\Strat}{Strat}
\DeclareMathOperator{\OppStrat}{OppStrat}
\newcommand{\seqs}[1]{{\overline{{#1}^{*}}}}
\def\flushRight{\leftskip0pt plus 1fill\rightskip0pt}
\def\Centering{\relax\ifvmode\centering\fi}
\newcommand{\deno}[1]{\left\llbracket#1\right\rrbracket}
\newcommand{\converges}{\Downarrow}
\newcommand{\diverges}{\Uparrow}
\newcommand{\mustconverge}{\converges^{\text{must}}}
\newcommand{\Iflt}{\mathtt{If{<}\;}}
\newcommand{\Ifgt}{\mathtt{If{>}\;}}
\newcommand{\inr}{{\mathsf{inr}}}
\newcommand{\inl}{{\mathsf{inl}}}
\newcommand{{\Na}}{\bN}
\newcommand{{\cell}}{{\mathsf{cell}}}
\newcommand{\fix}{{\mathsf{fix}}}
\newcommand{\eq}{{\mathsf{eq}}}
\DeclareMathOperator{\CCom}{CCom}
\newcommand{\power}{\mathfrak P}

% Slanty things
\newcommand*{\xslant}[2][76]{%
  \begingroup
    \sbox0{#2}%
    \pgfmathsetlengthmacro\wdslant{\the\wd0 + cos(#1)*\the\wd0}%
    \leavevmode
    \hbox to \wdslant{\hss
      \tikz[
        baseline=(X.base),
        inner sep=0pt,
        transform canvas={xslant=cos(#1)},
      ] \node (X) {\usebox0};%
      \hss
      \vrule width 0pt height\ht0 depth\dp0 %
    }%
  \endgroup
}

\makeatletter
\newcommand*{\xslantmath}{}
\def\xslantmath#1#{%
  \@xslantmath{#1}%
}
\newcommand*{\@xslantmath}[2]{%
  % #1: optional argument for \xslant including brackets
  % #2: math symbol
  \ensuremath{%
    \mathpalette{\@@xslantmath{#1}}{#2}%
  }%
}
\newcommand*{\@@xslantmath}[3]{%
  % #1: optional argument for \xslant including brackets
  % #2: math style
  % #3: math symbol
  \xslant#1{$#2#3\m@th$}%
}
\makeatother

\newcommand{\seqdeno}[1]{\xslantmath{\llbracket}#1\xslantmath{\rrbracket}}

% Empty set etc.

\let\oldemptyset\emptyset
\let\emptyset\varnothing

%% Constant width xrightarrows
\newlength{\arrow}
\settowidth{\arrow}{\scriptsize$1000$}
\newcommand*{\constantwidthxrightarrow}[1]{\xrightarrow{\mathmakebox[\arrow]{#1}}}

%% Landscape pages
\usepackage{everypage}
\usepackage{environ}
\usepackage{pdflscape}
\newcounter{abspage}

\ifdefined\NOTARTICLE

\else

\makeatletter
\newcommand{\newSFPage}[1]% #1 = \theabspage
  {\global\expandafter\let\csname SFPage@#1\endcsname\null}

\NewEnviron{SidewaysFigure}{\begin{figure}[p]
\protected@write\@auxout{\let\theabspage=\relax}% delays expansion until shipout
  {\string\newSFPage{\theabspage}}%
\ifdim\textwidth=\textheight
  \rotatebox{90}{\parbox[c][\textwidth][c]{\linewidth}{\BODY}}%
\else
  \rotatebox{90}{\parbox[c][\textwidth][c]{\textheight}{\BODY}}%
\fi
\end{figure}}

\AddEverypageHook{% check if sideways figure on this page
  \ifdim\textwidth=\textheight
    \stepcounter{abspage}% already in landscape
  \else
    \@ifundefined{SFPage@\theabspage}{}{\global\pdfpageattr{/Rotate 0}}%
    \stepcounter{abspage}%
    \@ifundefined{SFPage@\theabspage}{}{\global\pdfpageattr{/Rotate 90}}%
  \fi}
\makeatother

\fi

%% PCF Things

\newcommand{\nat}{{\mathtt{nat}}}
\newcommand{\bool}{{\mathtt{bool}}}

\newcommand{\Y}{\mathbf{Y}}
\newcommand{\opto}{\longrightarrow}
\newcommand{\oopto}{\dashrightarrow}
\newcommand{\n}{{\mathtt{n}}}
\DeclareMathOperator{\IfO}{{\mathsf{If0}}}
\DeclareMathOperator{\suc}{{\mathsf{succ}}}
\DeclareMathOperator{\pred}{{\mathsf{pred}}}
\newcommand{\0}{{\mathtt{0}}}

\newcommand{\iter}{{\mathtt{iter}}}
\newcommand{\rec}{\iter}
\newcommand{\Var}{{\mathtt{Var}}}
\DeclareMathOperator{\Varr}{Var}
\newcommand{\new}{{\mathtt{new}}}
\newcommand{\case}{{\mathtt{case}}}

\newcommand{\lmam}{\mathrel{\sqsubseteq_{m\&m}}}
\newcommand{\emam}{\mathrel{\equiv_{m\&m}}}
\newcommand{\lst}{\mathrel{\lesssim}}
\newcommand{\smam}{\mathrel{\sim_{m\&m}}}
\newcommand{\amam}{\mathrel{\approx_{m\&m}}}

\newcommand{\oes}{\sim}

%% Idealized Algol things

\newcommand{\com}{{\mathtt{com}}}
\newcommand{\skipp}{{\mathsf{skip}}}
\DeclareMathOperator{\seq}{{\mathsf{seq}}}
\DeclareMathOperator{\neww}{{\mathsf{new}}}
\DeclareMathOperator{\mkvar}{{\mathsf{mkvar}}}
\newcommand{\deref}{\texttt{@}}
\DeclareMathOperator{\dereff}{\mathsf{deref}}
\DeclareMathOperator{\assign}{\mathsf{assign}}
\newcommand{\ia}[2]{\langle #1 , #2 \rangle}
\newcommand{\stup}[3]{\langle #1 \mid #2 \mapsto #3 \rangle}

%% Hyland-Ong games things

\newbox\gnBoxA
\newdimen\gnCornerHgt
\setbox\gnBoxA=\hbox{$\ulcorner$}
\global\gnCornerHgt=\ht\gnBoxA
\newdimen\gnArgHgt
\def\pv #1{%
    \setbox\gnBoxA=\hbox{$#1$}%
    \gnArgHgt=\ht\gnBoxA%
    \ifnum     \gnArgHgt<\gnCornerHgt \gnArgHgt=0pt%
    \else \advance \gnArgHgt by -\gnCornerHgt%
    \fi \raise\gnArgHgt\hbox{$\ulcorner$} \box\gnBoxA %
    \raise\gnArgHgt\hbox{$\urcorner$}}
\def\ov #1{%
    \setbox\gnBoxA=\hbox{$#1$}%
    \gnArgHgt=\ht\gnBoxA%
    \ifnum     \gnArgHgt<\gnCornerHgt \gnArgHgt=0pt%
    \else \advance \gnArgHgt by -\gnCornerHgt%
    \fi \raise\gnArgHgt\hbox{$\llcorner$} \box\gnBoxA %
    \raise\gnArgHgt\hbox{$\lrcorner$}}
\newcommand{\ct}[1]{\lceil#1\rceil}
\DeclareMathOperator{\Int}{int}

%% Nondeterministic Factorization things

\newcommand{\code}{\mathsf{code}}
\newcommand{\Det}{\mathsf{Det}}

%% Flexible strategy things

\newcommand{\stle}{{\;\le_s\;}}
\newcommand{\steq}{{\;=_s\;}}
\newcommand{\exle}{\sqsubseteq}
\newcommand{\exlub}{\bigsqcup}
\newcommand{\dv}{{\text{\lightning}}}
\DeclareMathOperator{\pocl}{pocl}
\newcommand{\plot}{\mathrel{\triangleleft}}
\newcommand{\shad}{\mathfrak{S}}
%\newcommand{\tree}{\mathfrak{T}}
\newcommand{\Tau}{T}
\newcommand{\Epsilon}{E}
\newcommand{\sw}{\triangleleft}

%% Roman numerals

\newcommand{\RN}[1]{%
  \textup{\uppercase\expandafter{\romannumeral#1}}%
}
\newcommand{\RNl}[1]{%
  \mathrel{\raisebox{1pt}{$\overline{\underline{#1}}$}}
}

%% Game language things

\newcommand{\ul}[1]{{\underline{#1}}}
\newcommand{\A}{{\mathcal{A}}}
\renewcommand{\P}{\mathcal P}
\newcommand{\M}{\mathcal M}
\newcommand{\N}{\mathcal N}
\newcommand{\X}{\mathcal X}
\newcommand{\YY}{\mathcal Y}
\newcommand{\hole}{\blank}
\newcommand{\Tct}{\xrightarrow{T}t}
\newcommand{\teamconverge}[2]{\xrightarrow{#1}#2}

%% Inference rule things
\newcommand{\rulename}[1]{\LeftTirNameStyle{#1}}
\newcommand{\ts}{\mathbin{\vdash}}
\newcommand{\nts}{\mathbin{\not\vdash}}

%% Double category things
\newcommand{\hc}[2]{\left({#1}\middle|{#2}\right)}
\newcommand{\vc}[2]{\left(\frac{#1}{#2}\right)}

%% What is going on?
\DeclareMathOperator{\Kl}{Kl}
\DeclareMathOperator{\Mell}{Mell}
\newcommand{\powerset}{\mathcal P}
\DeclareMathOperator{\ask}{{\mathsf{ask}}}
\newcommand{\sleep}{{\mathsf{sleep}}}
\newcommand{\true}{\mathbbm{t}}
\newcommand{\false}{\mathbbm{f}}
\DeclareMathOperator{\If}{\mathsf{If}}
\newcommand{\Then}{\mathrel{\mathsf{then}}}
\newcommand{\Else}{\mathrel{\mathsf{else}}}
\newcommand\cat{\mathbin{+\mkern-10mu+}}

%% Profunctor arrows

\makeatletter
\def\slashedarrowfill@#1#2#3#4#5{%
  $\m@th\thickmuskip0mu\medmuskip\thickmuskip\thinmuskip\thickmuskip
   \relax#5#1\mkern-7mu%
   \cleaders\hbox{$#5\mkern-2mu#2\mkern-2mu$}\hfill
   \mathclap{#3}\mathclap{#2}%
   \cleaders\hbox{$#5\mkern-2mu#2\mkern-2mu$}\hfill
   \mkern-7mu#4$%
}
\def\rightslashedarrowfill@{%
  \slashedarrowfill@\relbar\relbar\mapstochar\rightarrow}
\newcommand\xslashedrightarrow[2][]{%
  \ext@arrow 0055{\rightslashedarrowfill@}{#1}{#2}}
\makeatother
\newcommand{\pto}{{\xslashedrightarrow{} }}

%% Profunctors 
\DeclareMathOperator{\Prof}{Prof}
\DeclareMathOperator{\End}{End}
\DeclareMathOperator{\Endoprof}{Endoprof}

%% Our

\def\searchmacro#1{
  \AtBeginOfFiles{%
    \ifdefined#1
      \expandafter\def\csname \currfilename:found\endcsname{}%
    \fi}
  \AtEndOfFiles{%
    \ifdefined#1
      \unless\ifcsname \currfilename:found\endcsname
        \immediate\write\finder{found in '\currfilename'}%
    \fi\fi}}

%% Isomorphism arrows on commutative diagrams %%
\tikzset{Isom/.style={every to/.append style={edge node={node [sloped, above, allow upside down, auto=false]{$\cong$}}}},
         Isom'/.style={every to/.append style={edge node={node [sloped, above, allow upside down, auto=false, rotate=180]{$\cong$}}}},
         Sim/.style={every to/.append style={edge node={node [sloped, above, allow upside down, auto=false]{$\sim$}}}},
         Sim'/.style={every to/.append style={edge node={node [sloped, above, allow upside down, auto=false, rotate=180]{$\sim$}}}}}

%% Adjunctions
\newcommand{\adjunction}[4]{%
  {#1} \underset{\underset{#3}{\longleftarrow}}{\overset{\overset{#2}{\longrightarrow}}{\bot}} {#4}}        

%% Important!
\newcommand\Mellies{Melli\`{e}s\xspace}

\makeatletter
\newcommand{\colim@}[2]{%
  \vtop{\m@th\ialign{##\cr
    \hfil$#1\operator@font colim$\hfil\cr
    \noalign{\nointerlineskip\kern1.5\ex@}#2\cr
    \noalign{\nointerlineskip\kern-\ex@}\cr}}%
}
\newcommand{\colim}{%
  \mathop{\mathpalette\colim@{\rightarrowfill@\textstyle}}\nmlimits@
}
\makeatother

\makeatletter
\newcommand{\laxcolim@}[2]{%
  \vtop{\m@th\ialign{##\cr
    \hfil$#1\operator@font colim_l$\hfil\cr
    \noalign{\nointerlineskip\kern1.5\ex@}#2\cr
    \noalign{\nointerlineskip\kern-\ex@}\cr}}%
}
\newcommand{\laxcolim}{%
  \mathop{\mathpalette\laxcolim@{\rightarrowfill@\textstyle}}\nmlimits@
}
\makeatother

\DeclareMathOperator{\Colim}{colim}

\DeclareMathOperator{\DG}{DG}
\DeclareMathOperator{\RV}{RV}
\newcommand{\Rv}{\catname{Rv}}

\let\choose\undefined
\DeclareMathOperator{\choose}{\mathsf{choose}}
\DeclareMathOperator{\tr}{tr}
\DeclareMathOperator{\test}{test}

%% Slot game things %%
\newcommand{\circled}[1]{\raisebox{.5pt}{\textcircled{\raisebox{-.9pt} {#1}}}}
\newcommand{\slot}{{\circled{\$}}}

\DeclareMathOperator{\may}{may}
\DeclareMathOperator{\must}{must}

\newcommand{\encode}[1]{\lceil#1\rceil}
\DeclareMathOperator{\app}{\mathsf{app}}
\DeclareMathOperator{\lett}{\mathsf{let}}
\newcommand{\inn}{\mathrel{\mathsf{in}}}
\DeclareMathOperator{\byval}{\mathsf{byval}}

\DeclareMathOperator{\rread}{read}
\DeclareMathOperator{\wwrite}{write}

\DeclareSymbolFont{bbsymbol}{U}{bbold}{m}{n}
\DeclareMathSymbol{\bbsemicolon}{\mathbin}{bbsymbol}{"3B}
\newcommand{\semicom}{\bbsemicolon}

\newcommand{\ms}{\makebox[-1pt]{}}

\DeclareMathOperator{\Acc}{Acc}
\DeclareMathOperator{\im}{Im}
\DeclareMathOperator{\wit}{wit}

%%% END Article customizations


\usepackage{lua-visual-debug}

\begin{document}

\chapter{Parametric Monads and Full Abstraction}

Let a monoidal category $\X$ act on a category $\G$, where $\G$ is a suitable model of some programming language.  
In this chapter we will investigate the adequacy and full abstraction properties of the resulting category $\G/\X$, as we did with Kleisli categories in Chapter \ref{ChapMonads}.  
Once again, we will pass to a special case of the general theory.  
As in Chapter \ref{ChapMonads}, we will require $\G$ to be a Cartesian closed category that admits a computationally adequate denotational semantics of Idealized Algol, and we shall require that $\G$ may be regarded as being enriched in algebraic directed-complete partial orders in such a way that every compact morphism between the denotations of types is the denotation of some term.
As hinted at in Chapter \ref{ChapParametricMonads}, we shall require the action of $\X$ on $\G$ to be a reader action corresponding to an oplax symmetric monoidal functor that satisfies the condition in Theorem \ref{TheCartesianClosedCx}, so that the category $\G/\X$ is Cartesian closed.

We fix a symmetric monoidal category $\X$ and an oplax monoidal functor $j\from \X \to \Set$ such that for any object $p$ of $\X$ there are morphisms $h\from p \to p \tensor p$ and $h_0 \from p \to I$ such that the composite
\[
  j(p) \xrightarrow{jh} j(p\tensor p) \xrightarrow{m^j_{p,p}} j(p) \times j(p)
  \]
is equal to the diagonal on $j(p)$.

We fix a model $\G$ of Idealized Algol as above, and suppose that the datatypes in $\G$ are interpreted via an oplax monoidal functor $\Set \to \G$.  
Then we get an oplax monoidal functor $\X \to \G$, inducing a reader action of $\oppcat\X$ on $\G$ such that the category $\G/\oppcat\X$ is Cartesian closed.  
We will define a language and an interpretation of this language in the category $\G/\oppcat\X$.

\newcommand{\IAX}{{IA${}_{X}$}\xspace}
\newcommand{\IAXX}{{IA${}_{\X}$}\xspace}
\section{The language \IAXX}

\begin{definition}[{The language \IAXX}]
  The language \IAXX is formed by taking Idealized Algol, and adding to it new constants
  \[
    \choose_p
    \]
  for each object $p$ of $\X$ such that $j(p)\in\{\bC,\bB,\bN\}$, with typing rule
  \[
    \inferrule{ }{\Gamma \ts \choose_p \from j(p)}\,.
    \]
\end{definition}

The interpretation of $\choose_p$ is that it requests an element $a$ of the set $j(p)$.

Let $\G$ be a model of Idealized Algol as described above, and suppose that there is an oplax monoidal functor $\Set \to \G$ that is used to interpret datatypes.  
We will use an underline to indicate thies functor; so, for example, the object of $\G$ that is used to denote the natural number type is written $\ul\bN$.

By our description of $\G/\oppcat \X$ as a lax colimit in $\Cat$ (i.e., Corollary \ref{CorTheConstructionUniversalProperty}), we have a natual functor $J\from \G\to\G/\oppcat \X$ and a natural transformation $\phi_{p,a}\from J(jp\to a) \to Ja$.  
Our denotational semantics of \IAXX is then given in the category $\G/\oppcat X$ as follows.  
The denotation of any type $T$ of Idealized Algol is given by $J(\deno{T}_\G)$, where $\deno{T}_\G$ is the original denotation in $\G$.
The denotation of any sequent $\Gamma\ts M$ is given by $\deno{\Gamma\ts M} = J(\deno{\Gamma\ts M}_\G)$, where $\deno{-}_\G$ is the original denotation in $\G$.
The denotation of $\choose_p$ is given by the morphism $\omega_p \from 1 \to j(p)$ given by the composite
\[
  I \xrightarrow{\Lambda(\id_{Jjp})} (Jjp \to Jjp)  \to J(jp \to jp) \xrightarrow{\phi_{p,jp}} Jjp\,.
  \]

This denotation may alternatively be defined in a non-compositional way: given a term $\Gamma\ts M\from T$ in context of \IAXX, we can write
\[
  M = N[\choose_p/x_p]\,,
  \]
where $(x_p)$ is a finite collection of free variables in $M$.

Since the categories $\G$ and $\G/\X$ are Cartesian closed, the $\beta$-rule is valid in the semantics, and so if $N$ is a term of \IAXX that refers to $(\choose_p)_{p\in\P}$ for some finite collection $\P$ of objects of $\X$, then we may write the denotation of $\Gamma \ts N$ as the composite
\[
  \deno{\Gamma} \xrightarrow{\langle \id,(\omega_p)\rangle} \deno{\Gamma,(x_p)} \xrightarrow{\deno{\Gamma,(x_p)\ts N[x_p/\choose_p}} \deno{T}\,,
  \]
where the denotation at the right is that of ordinary Idealized Algol.

This is a morphism in $\G/\oppcat\X$.  
If we consider it as a morphism in $\G$, we see that it is given by the curried form of the composite
\[
  \deno{\Gamma}\times j\left(\Tensor_p p\right) \xrightarrow{\deno{\Gamma}\times m^j} \deno \Gamma \times \prod_p j(p) \xrightarrow{\deno{\Gamma,(x_p)\ts N[x_p/\choose_p]}} \deno T\,.
  \]

The example to have in mind is that of probability; here, the objects of our category $\X$ are discrete random variables, with $j(p)$ giving the codomain of the random variable, and the term $\choose_p\from j(p)$ can be thought of as choosing an element of that set.

\section{Operational Semantics}

We inductively define a relation
\[
  \Gamma,s\ts M\converges_U c,s'\,,
  \]
where $\Gamma$ is a $\Var$-context, $s,s'$ are $\Gamma$-stores, $\Gamma\ts M,\Gamma\ts c$ are \IAXX terms-in-context such that $c$ is an IA canonical form, and $U$ is a sequence of pairs of the form $(p:a)$, where $p$ is an object of $\X$ and $a\in j(p)$.
The definition of this rule is shown in Figure \ref{FigIaxxOpSem}.

\begin{figure}
  \begin{mathpar}
    \inferrule*{ }{\Gamma,s\ts c \converges_\epsilon c,s}
    \and
    \inferrule*{\Gamma,s \ts M \converges_U \lambda x.M',s' \\ \Gamma,s' \ts M'[N/x] \converges_V c,s''}{\Gamma,s \ts MN \converges_{U\cat V} c,s''}
    \and
    \inferrule*{\Gamma,s \ts M(\Y M) \converges_U c,s'}{\Gamma,s \ts \Y M \converges_U c,s'}
    \and
    \inferrule*{\Gamma,s\ts M \converges_U n,s'}{\Gamma,s\ts \suc M \converges_U n+1,s'}
    \\\and
    \inferrule*{\Gamma,s\ts M \converges_U n+1,s'}{\Gamma,s\ts \pred M \converges_U n,s'}
    \and
    \inferrule*{\Gamma,s\ts M \converges_U 0,s'}{\Gamma,s\ts \pred M \converges_U 0,s'}
    \and
    \inferrule*{\Gamma,s\ts M \converges_U \skipp,s' \\ \Gamma,s'\ts N \converges_V c,s''}{\Gamma,s \ts M;N \converges_{U\cat V} c,s''}
    \and
    \inferrule*{\Gamma,s\ts M \converges_U \true,s' \\ \Gamma,s' \ts N \converges_V c,s''}{\Gamma,s \ts \If M \Then N \Else P \converges_{U\cat V} c,s''}
    \and
    \inferrule*{\Gamma,s\ts M \converges_U \false,s' \\ \Gamma,s' \ts P \converges_V c,s''}{\Gamma,s \ts \If M \Then N \Else P \converges_{U\cat V} c,s''}
    \and
    \inferrule*{\Gamma,s\ts M \converges_U 0,s' \\ \Gamma,s' \ts N \converges_V c,s''}{\Gamma,s \ts \IfO M \Then N \Else P \converges_{U\cat V} c,s''}
    \and
    \inferrule*{\Gamma,s\ts M \converges_U n+1,s' \\ \Gamma,s' \ts P \converges_V c,s''}{\Gamma,s \ts \IfO M \Then N \Else P \converges_{U \cat V} c,s''}
    \and
    \inferrule*[right=$x\in\Gamma$]{\Gamma,s\ts E \converges_U n,s' \\ \Gamma,s' \ts V \converges_V x,s''}{\Gamma,s\ts V \gets E \converges_{U \cat V} \skipp,(s''\vert x \mapsto n)}
    \and
    \inferrule*[right={$s'(x)=n$}]{\Gamma,s\ts V \converges_U x,s'}{\Gamma,s\ts !V \converges_U n,s'}
    \and
    \inferrule*{\Gamma,x\from\Var,(s\vert x\mapsto 0)\ts M \converges_U c,(s'\vert x\mapsto n)}{\Gamma,s\ts \neww \lambda x.M \converges_U c,s'}
    \and
    \inferrule*{\Gamma,s\ts E \converges_U n,s' \\ \Gamma,s'\ts V \converges_V \mkvar W R,s'' \\ \Gamma,s'' \ts Wn \converges_W \skipp,s'''}
    {\Gamma,s \ts V\gets E \converges_{U \cat V \cat W} \skipp,s'''}
    \and
    \inferrule*{\Gamma,s\ts V \converges_U \mkvar W R,s' \\ \Gamma,s'\ts R \converges_V n,s''}{\Gamma,s\ts !V \converges_{U \cat V} n,s''}
    \\\and
    \inferrule*[right=$a\in j(p)$]{ }{\Gamma,s\ts \choose_p \converges_{(p:a)} a,s}
  \end{mathpar}
  \caption[Operational semantics for \IAXX]{Operational semantics for \IAXX.}
  \label{FigIaxxOpSem}
\end{figure}

We notice immediately that all but one of these rules are exactly the same as the corresponding rules from \IAX, the only difference being the form the form that the associated sequence takes.  
The one difference is the rule for $\choose_p$.

\section{Translation into \IAX}

We make the connection between the languages \IAX and \IAXX more explicit in the following series of lemmas.
In this section, we will make use of an encoding between an object of the form $jp$ and an Idealized Algol datatype $N$.  
The case we have in mind is when $jp$ is some finite set and $N$ is the natural number object, so that we can choose some way of representing elements of $jp$ as elements of $N$.

\begin{definition}
  Let $p_1,\cdots,p_n$ be a sequence of objects of $\X$ and let $N$ be an object of $\X$ such that $j(N)$ is a datatype of Idealized Algol (i.e., $j(N)\in\{\bC,\bB,\bN\}$).  
  Suppose we have a morphism
  \[
    f \from N \to p_1 \tensor \cdots \tensor p_n
    \]
  in $\X$ such that $jf$ is a surjection.  
  Define functions $\pi_i\from j(N) \to j(p_i)$ to be given by the composites
  \[
    j(N) \xrightarrow{jf} j(p_1 \tensor \cdots \tensor p_n) \xrightarrow{m^j} j(p_1) \times \cdots \times j(p_n) \xrightarrow{\pr_i} j(p_i)\,.
    \]

  Let $u\in j(N)^*$ be a sequence of elements of $j(N)$, and let $U$ be a sequence of pairs $(p:a)$, where each $p$ is one of the $p_i$.  
  We say that $u$ \emph{covers $U$ with respect to $f$} if $U$ and $u$ have the same length and if whenever $U^{(k)}=(p_i:a)$, we have $a = \pi_i(u^{(k)})$.
  \label{DefCovers}
\end{definition}

Recall that, in the definition of the category $\G/\oppcat\X$, the \Mellies morphisms are left unchanged by precomposing with a morphism in the image of the functor $j$; therefore, if $M\from T$ is a closed term of \IAXX referring to $\choose_{p_1},\cdots,\choose_{p_n}$, then we may write the denotation of $M$ as the composite
\[
  j(N) \xrightarrow{jf} j(p_1 \tensor \cdots \tensor p_n) \xrightarrow{m^j} j(p_1) \times \cdots \times j(p_n) \xrightarrow{\deno{x_1,\cdots,x_n\ts M[x_i/\choose_{p_i}]}_\G} \deno T\,;
  \]
i.e., as
\[
  j(N) \xrightarrow{\langle \pi_1,\cdots,\pi_n\rangle} j(p_1) \times \cdots \times j(p_n) \xrightarrow{\deno{x_1,\cdots,x_n\ts M[x_i/\choose_{p_i}]}_\G} \deno T\,.
  \]

\begin{lemma}
  Let $\Gamma \ts M$ be an \IAXX term-in-context, where $M$ refers to terms $\choose_{p_1},\cdots,\choose_{p_n}$, and no other $\choose$ terms.
  Let $N$ be an object of $\X$ such that $j(N)$ is an IA datatype and let $f\from N \to p_1\tensor \cdots \tensor p_n$ be a morphism, as in Definition \ref{DefCovers}.  
  Suppose that the functions $\pi_i$ are all definable in Idealized Algol; that is, that there are terms $\Pi_i\from j(N) \to j(p_i)$ of IA such that the following inference is valid.
  \[
    \inferrule{ \Gamma,s\ts M \converges m,s'}{\Gamma,s\ts \Pi_i M\converges \pi_i(m),s'}
    \]

  Let $s,s'$ be $\Gamma$-stores, and let $\Gamma\ts c$ be a canonical form.  
  Suppose $U$ is a sequence such that we have
  \[
    \Gamma,s\ts M\converges_U c,s'\,.
    \]
  Then
  \[
    \Gamma,s\ts M[\neww (\lambda v.v\gets \ask_{j(N)};\Pi_i \oc v)/\choose_{p_i}] \converges_u c,s'
    \]
  in IA${}_{j(N)}$ for all sequences $u\in j(N)^*$ that cover $U$.
  \label{LemSoundnessIaxx}
\end{lemma}
\begin{proof}
  Induction on the derivation of $\Gamma,s\ts M\converges_U c,s'$.  
  Suppose that the last rule in the derivation takes the following form.
  \[
    \inferrule{\Gamma_1,s^{(0)}\ts M_1\converges_{U_1} c_1,s^{(1)} \\ \cdots \\ \Gamma_n,s^{(n-1)}\ts M \converges_{U_n} c_n,s^{(n)}}
    {\Gamma,s^{(0)} \ts M \converges_{U_1 \cat \cdots \cat U_n} c,s^{(n)}}
    \]
  Suppose a sequence $u$ covers $U_1\cat \cdots \cat U_n$.  
  Then we may write $u = u_1 \cat \cdots \cat u_n$, where $u_i$ covers $U_i$.  

  By induction, then, we may derive that
  \[
    \Gamma_k,s^{(k)}\ts M_k[\neww (\lambda v.v\gets \ask_{j(N)};\Pi_i \oc v)/\choose_{p_i}]  \converges_{u_k} c_k,s^{(k)}
    \]
  for $k=1,\cdots,n$.
  Now note that Lemma \ref{LemFirstSubstitution} still holds if we use the terms $\choose_{p_i}$ instead of the $\ask_X$; this means that we have a valid IA${}_j(N)$ inference given by
  \[
    \inferrule*{\Gamma_1,s^{(0)}\ts M_1[\new(\lambda v.v\gets \ask_{j(N)};\Pi_i \oc v)/\choose_{p_i}] \converges_{u_1} c_1,s^{(1)} \\ \cdots \\ \Gamma_n,s^{(n-1)}\ts M_n[\new(\lambda v.v\gets \ask_{j(N)};\Pi_i \oc v)/\choose_{p_i}] \converges_{u_n} c_n,s^{(n)}}
    {\Gamma,s^{(0)}\ts M[(\lambda z.\Pi_i) \ask_{j(N)}.\choose_{p_i}] \converges_u c,s^{(n)}}\,,
    \]
  from which we can deduce that
  \[
    \Gamma,s^{(0)}\ts M[\new(\lambda v.v\gets \ask_{j(N)};\Pi_i \oc v)/\choose_{p_i}] \converges_i c,s^{(n)}\,,
    \]
  as desired.

  The other possibility is that the final step in the derivation takes the form
  \[
    \inferrule*{ }{\Gamma,s \ts \choose_{p_j} \converges_{(p_j:a)} a,s}\,.
    \]
  Let $U$ be a (length $1$) sequence covering $(p_j:a)$.  
  So $U=t$, where $t\in j(P)$ is such that $\pi_j(t)=a$.

  Then 
  \[
    \choose_{p_j}[\neww(\lambda v.v\gets \ask_{j(N)};\Pi_i \oc v)/\choose_{p_i}] = \neww(\lambda v.v\gets \ask_{j(N)};\Pi_j \oc v)\,,
    \]
  and we may derive
  \[
    \Gamma,s\ts \new(\lambda v.v\gets \ask_{j(N)};\Pi_j\oc v) \converges_t a,s\,.\qedhere
    \]
\end{proof}

To prove the converse, we prove a lemma about substitution analogous to  Lemma \ref{LemSecondSubstitution}.

\begin{lemma}
  Let
  \[
    \inferrule{\Gamma,s^{(0)}\ts M_1\converges_{u_1} c_1,s^{(1)} \\ \cdots \\ \Gamma,s^{(n-1)}\ts M_n\converges_{u_n} c_n,s^{(n)}}
    {\Gamma,s^{(0)} \ts M \converges_{u_1\cat \cdots \cat u_n} c,s^{(n)}}
    \]
  be an inference of IA${}_{j(N)}$, where every instance of $\ask_{j(N)}$ occurs as part of some term $\neww (\lambda v.v\gets \ask_{j(N)};\Pi_i \oc v)$, and suppose that $M\ne \neww(\lambda v.v\gets\ask_{j(N)};\Pi_j\oc v)$ for any $j$.  
  Suppose we have sequences $U_1,\cdots,U_n$ such that $u_k$ covers $U_k$ for $k=1,\cdots,n$.
  Then
  \[
    \inferrule{\Gamma,s^{(0)}\ts M_1[\choose_{p_i}/\neww(\lambda v.v\gets \ask_{j(N)};\Pi_i\oc v)] \converges_{U_1} c_1,s^{(1)} \\ \cdots \\ \Gamma,s^{(n-1)}\ts M_n[\choose_{p_i}/\neww(\lambda v.v\gets \ask_{j(N)};\Pi_i\oc v)] \converges_{U_n} c_n,s^{(n)}}
    {\Gamma,s^{(0)} \ts M[\choose_{p_i}/\neww(\lambda v.v\gets \ask_{j(N)};\Pi_i\oc v)] \converges_{U_1\cat \cdots \cat U_n} c,s^{(n)}}
    \]
  is a valid inference of \IAXX.
  \label{LemThirdSubstitution}
\end{lemma}
\begin{proof}
  As in Lemma \ref{LemSecondSubstitution}, we can prove this by looking at cases.  
  For example, consider the sequencing rule
  \[
    \inferrule{\Gamma,s\ts M\converges_u \skipp,s' \\ \Gamma,s' \ts N \converges_v c,s''}{\Gamma,s\ts M;N \converges c,s''}\,.
    \]
  We have
  \begin{IEEEeqnarray*}{Cl}
    &(M;N)[\choose_{p_i}/\neww(\lambda v.v\gets \ask_{j(N)};\Pi_i\oc v)] \\
    = &M[\choose_{p_i}/\neww(\lambda v.v\gets \ask_{j(N)};\Pi_i\oc v)];\\&N[\choose_{p_i}/\neww(\lambda v.v\gets \ask_{j(N)};\Pi_i\oc v)]\,,
  \end{IEEEeqnarray*}
  and so we certainly get a rule
  \[
    \inferrule*{\Gamma,s\ts M[\choose_{p_i}/\neww(\lambda v.v\gets \ask_{j(N)};\Pi_i\oc v)]\converges_U \skipp,s' \\ \Gamma,s' \ts N[\choose_{p_i}/\neww(\lambda v.v\gets \ask_{j(N)};\Pi_i\oc v)] \converges_V c,s''}{\Gamma,s\ts (M;N)[\choose_{p_i}/\neww(\lambda v.v\gets \ask_{j(N)};\Pi_i\oc v)] \converges c,s''}\,.
    \]
  The only case where we need to be careful is for the $\neww$ rule:
  \[
    \inferrule*{\Gamma,x,(s\vert x\mapsto 0)\ts M\converges_u c,(s'\vert x\mapsto n)}
    {\Gamma,s\ts \neww \lambda x.M \converges_u c,s'}\,.
    \]
  If $\neww\lambda x.M \ne \neww(\lambda v.v\gets \ask_{j(N)};\Pi_j \oc v)$, then we have
  \begin{IEEEeqnarray*}{Cl}
    &(\neww \lambda x.M)[\choose_{p_i}/\neww(\lambda v.v\gets \ask_{j(N)};\Pi_i\oc v)] \\
    = & \neww \lambda x. (M[\choose_{p_i}/\neww(\lambda v.v\gets \ask_{j(N)};\Pi_i\oc v)])\,.
  \end{IEEEeqnarray*}
  Then we can apply the rule
  \[
    \inferrule*{\Gamma,x,(s\vert x\mapsto 0)\ts M[\choose_{p_i}/\neww(\lambda v.v\gets \ask_{j(N)};\Pi_i\oc v)] \converges_U c,(s'\vert x\mapsto n)}
    {\Gamma,s\ts (\neww\lambda x.M)[\choose_{p_i}/\neww(\lambda v.v\gets \ask_{j(N)};\Pi_i\oc v)] \converges U c,s'}.\qedhere
    \]
\end{proof}

We now prove the converse to Lemma \ref{LemSoundnessIaxx}.

\begin{lemma}
  Let $\Gamma,y_1\from j(p_1),\cdots,y_n\from j(p_n) \ts M \from T$ be a term-in-context of ordinary Idealized Algol, where $\Gamma$ is a $\Var$-context.  
  Let $U$ be a sequence and let $N,\pi_i,\Pi_i$ be as above.
  Suppose that there exists some sequence $u\in j(N)^*$ such that $u$ covers $U$ and such that
  \[
    \Gamma,s\ts M[\neww(\lambda v.v\gets \ask_{j(N)};\Pi_i \oc v)/y_i]\converges_u c,s'\,.
    \]
  Then
  \[
    \Gamma,s\ts M[\choose_{p_i}/y_i] \converges_U c,s'\,.
    \]
  \label{LemAdequacyIaxx}
\end{lemma}
\begin{proof}
  Induction on the derivation.
  Suppose that $M$ is not one of the $y_i$; then $M[\neww(\lambda v.v\gets\ask_{j(N)};\Pi_i \oc v)/y_i]$ is not equal to $\neww(\lambda v.v\gets\ask_{j(N)};\Pi_i \oc v)$.
  Moreover, every instance of $\ask_{j(N)}$ in $M$ occurs as part of an expression of the form $\neww(\lambda v.v\gets\ask_{j(N)};\Pi_i \oc v)$, and so we win by Lemma \ref{LemThirdSubstitution} and the inductive hypothesis.

  Otherwise, $M=\new\lambda v.v\gets \ask_{j(N)};\Pi_j\oc v)$ for some $j$.  
  Now, if we have
  \[
    \Gamma,s\ts \neww(\lambda v.v\gets\ask_{j(N)};\Pi_j\oc v)\converges_u c,s'\,,
    \]
  then a simple examination of the reduction tells us that we must have $s'=s$, and that $u$ must have length $1$ -- say $u=m$ -- where the single element $n$ of $u$ satisfies $\pi_j(m)=c$.

  But now we certainly have
  \[
    \Gamma,s\ts \choose_{p_j} \converges_{(p_j:c)} c,s\,,
    \]
  and the sequence $m$ covers the sequence $(p_j:c)$.
\end{proof}

Lemmas \ref{LemSoundnessIaxx} and \ref{LemAdequacyIaxx} together prove the following.

\begin{lemma}
  Let $\Gamma,x_1,\cdots,x_n\ts M$ be a term-in-context of Idealized Algol, where $\Gamma$ is a $\Var$-context.
  Then the following are equivalent.

  i) $\Gamma,s\ts M[\choose_{p_i/x_i}]\converges_U c,s'$.
  
  ii) $\Gamma,s\ts M[\neww(\lambda v.v\gets\ask_{j(N)};\Pi_i \oc v/x_i]\converges_u c,s'$ for all $u$ covering $U$.

  iii) $\Gamma,s\ts M[\neww(\lambda v.v\gets\ask_{j(N)};\Pi_i \oc v/x_i]\converges_u c,s'$ for some $u$ covering $U$.
  \label{LemComputationalAdequacyIaxx}
\end{lemma}
\begin{proof}
  (i) $\Rightarrow$ (ii): Lemma \ref{LemSoundnessIaxx}.

  (ii) $\Rightarrow$ (iii): By assumption, the function $j(f) \from N \to j(\Tensor_i p_i)$ is surjective, so for any $U$ there is some $u\in j(N)^*$ covering $U$.

  (iii) $\Rightarrow$ (i): Lemma \ref{LemAdequacyIaxx}.
\end{proof}

\section{Computational Adequacy}

We are now ready to make the definitions we need to state our Computational Adequacy result.

Recall that if $\sigma$ was a Kleisli morphism $1 \to \bC$ (i.e., a morphism $1 \to (X \to \bC)$ in the original category, where $X$ was an Idealized Algol datatype), then we wrote $\sigma\downarrow_u$ if the composite
\[
  1 \xrightarrow{\sigma} (X \to \bC) \xrightarrow{\eta_u} (\Varr \to \bN) \xrightarrow{\neww} \bN \xrightarrow{t_{|u|}} \bC
  \]
was not equal to $\bot$, where $\eta_u$ was the denotation of the Idealized Algol term-in-context
\[
  f \from X \to \com \ts \lambda v.v\gets 0;f(v\gets\suc\oc v;\tr_u \oc v);\oc v \from \Var \to \nat\,.
  \]

We want to extend this definition to morphisms in the category $\G/\oppcat \X$.  
There are a couple of problems here.  

Firstly, the morphisms in $\G/\oppcat \X$ are equivalence classes of \Mellies morphisms, and the equivalence relation does not respect this predicate $\downarrow_u$ -- especially since the $X$ in the above formula could change when we choose a different representative of the equivalence class.

Secondly, a morphism $1 \to \bC$ in $\G/\oppcat\X$ is given by an (equivalence class of) morphisms $1 \to (j(p) \to \bC)$ in $\G$, and the object $j(p)$ need not be an Idealized Algol datatype.

To solve the second problem, we make an additional small assumption on our category $\G$.  
We require that there exist morphisms
\[
  \xi_u \from (X \to \bC) \to \bC
  \]
for any set $X$ and any finite sequence $u\in X^*$ such that for any function $f \from X \to Y$, we have $(f\to \bC);\xi_u = \xi_{f_*u}$, where $f_*u$ is the sequence formed by applying $f$ pointwise to $u$, and such that if $X$ is an IA datatype, then
\[
  \xi_u = (X \to \bC) \xrightarrow{\eta_u} (\Varr \to \bN) \xrightarrow{\neww} \bN \xrightarrow{t_{|u|}} \bC\,.
  \]

\begin{example}
  In the category of games, the morphisms $\xi_u$ are the strategies containing the plays $\epsilon$, $qq$ and plays of the form
  \begin{mathpar}
    \begin{array}{ccc}
      X           & \bC & \bC \\
                  &     &  q  \\
                  &  q  &     \\
      q           &     &     \\
      u^{(0)}     &     &     \\
      \vdots      &     &     \\
      q           &     &     \\
      u^{(k)}       &     &    
    \end{array}
    \and
    \begin{array}{ccc}
      X           & \bC & \bC \\
                  &     &  q  \\
                  &  q  &     \\
      q           &     &     \\
      u^{(0)}     &     &     \\
      \vdots      &     &     \\
      q           &     &     \\
      u^{(|u|-1)} &     &     \\
                  &  a  &     \\
                  &     &  a  
    \end{array}
  \end{mathpar}
  (so the strategy has no reply if player $O$ asks the question in $X$ fewer than $|u|$ times, or tries to ask it more than $|u|$ times).
\end{example}

\begin{definition}
  Given a set $X$ and a \Mellies morphism $\sigma \from 1 \to (X \to \bC)$, we say that $\sigma$ \emph{accepts} a sequence $u\in X^*$ if $\sigma;\xi_u\ne\bot$.
  We write $\Acc(\sigma)$ for the set of all sequences accepted by $\sigma$.
  \label{DefAcc}
\end{definition}

Recall that a morphism $1 \to \bC$ in $\G/\oppcat\X$ is given by an equivalence class of \Mellies morphisms $1 \to (j(p) \to \bC)$ in $\G$, where $p$ ranges over the objects of $\X$, and where the equivalence relation is generated by identifying all pairs of morphisms $\sigma \from 1 \to (j(p) \to \bC)$ and $\tau \from 1 \to (j(q) \to \bC)$ such that there is a morphism $f \from p \to q$ such that $\tau;(j(f)\to\bC)=\sigma$.

\begin{definition}
  We define an equivalence relation on pairs $(p,\U)$, where $p$ is an object of $\X$ and $\U\subset j(p)^*$ is a set of finite sequences drawn from $j(p)$ to be the equivalence relation generated by identifying $(p,\U)$ and $(q,\V)$ whenever there is a morphism $f\from p \to q$ in $\X$ such that for all $u\in j(p)^*$, we have $u\in\U$ if and only if $j(f)_*u\in\V$.
  \label{DefEquivalenceOfPairs}
\end{definition}

It is instructive to consider the equivalence relation on pairs $(p,\U)$ in the case that $\X = \Rv_\Omega$ is the category of random variables on some probability space $\Omega$.
Given a random variable $V$ on a set $X$, we get an induced random variable taking values in $X^*$.  
If we have a random variable $W$ on a set $Y$ and a function $f\from X \to Y$ such that $W=f\circ X$, and if $\U\subset X^*$ and $\V\subset Y^*$ are such that $u\in\U$ if and only if $f_*u\in\V$, then the induced probabilities of the sets $\U$ and $\V$ are the same.  
So, in this case, the equivalence relation on sets of sequences is subsumed into the very natural equivalence relation of having the same probability.

\begin{proposition}
  Let $\sigma\from 1 \to (j(p) \to A)$, $\tau\from 1 \to (j(q) \to A)$ be two representatives of the same morphism $1 \to A$ in $\G/\oppcat\X$.  
  Then $(p,\Acc(\sigma))$ and $(q,\Acc(\tau)$ are equivalent.
  \label{PropAccEquivalent}
\end{proposition}
\begin{proof}
  Since the relation on pairs $(p,\U)$ is an equivalence relation, it suffices to assume that $\sigma$ and $\sigma'$ are related by the relation that generates the equivalence relation on \Mellies morphisms; i.e., that there is a morphism $f\from p \to q$ such that $\sigma = \tau;(j(f)\to \bC)$.  

  Let $u\in j(p)^*$.  
  Then we have
  \begin{IEEEeqnarray*}{rCl}
    u\in\Acc(\sigma) & \Leftrightarrow & \sigma;\xi_u\ne\bot \\
    & \Leftrightarrow & \tau;(j(f)\to A);\xi_u\ne\bot \\
    & \Leftrightarrow & \tau\xi_{j(f)_*u}\ne\bot \\
    & \Leftrightarrow & j(f)_*u \in \Acc(\tau)\,.
  \end{IEEEeqnarray*}
  Therefore, $(p,\Acc(\sigma))$ and $(q,\Acc(\tau))$ are equivalent.
\end{proof}

We can now state and prove our Computational Adequacy result.
For this result, given a term $M\from \com$ mentioning objects $p_1,\cdots,p_n$, we shall assume the existence of some IA datatype $N$ admitting a morphism $f \from N \to p_1 \tensor \cdots \tensor p_n$ such that the definable projections $\pi_i$ on to the objects $j(p_i)$ are IA-definable.

\begin{definition}
  Let $M$ be a closed term of \IAXX of type $\com$ mentioning $p_1,\cdots,p_n$.
  Let $S(M)$ be the set of all sequences $U$ such that $M\converges_U\skipp$.

  We define $B(M)$, the \emph{behaviours of $M$}, to be the equivalence class corresponding to the pair
  \[
    (p_1\tensor \cdots \tensor p_n, \U)\,,
    \]
  where $\U$ is the set of all sequences $u\in j(p_1\tensor \cdots \tensor p_n)^*$ that cover some sequence $U\in S(M)$, via the projections
  \[
    j(p_1\tensor \cdots \tensor p_n) \xrightarrow{m^j} j(p_1) \times \cdots \times j(p_n) \xrightarrow{\pr_i} j(p_i)\,.
    \]
\end{definition}


\begin{theorem}[Computational Adequacy for \IAXX]
  Let $M \from \com$ be a closed term of \IAXX referring to $p_1,\cdots,p_n$.  
  Suppose the denotation of $M$ is given by a morphism $1 \to (j(p) \to \bC)$ in $\G/\oppcat\X$.  

  Then $(p,\Acc(\deno M))$ is equivalent to $B(M)$.
  \label{TheComputationalAdequacyIAXX}
\end{theorem}
\begin{proof}
  By Proposition \ref{PropAccEquivalent}, we may assume that the denotation of $M$ is in a particular form, namely the (curried form of) the composite
  \[
    j(N) \xrightarrow{\langle\pi_1,\cdots,\pi_n\rangle} j(p_1) \times \cdots \times j(p_n) \xrightarrow{\deno{x_1,\cdots,x_n\ts M[x_i/\choose_{p_i}]}_\G} \bC\,.
    \]
  But if we consider this as a Kleisli morphism in the category $\Kl_{R_{j(N)}}\G$, then this is the denotation of the IA${}_{j(N)}$ term
  \[
    M[\neww(\lambda v.v\gets \ask_{j(N)};\Pi_i\oc v)/\choose_{p_i}]\,.
    \]
  By Lemma \ref{LemComputationalAdequacyIaxx}, if $u\in j(N)^*$ is a sequence, then
  \[
    M[\neww(\lambda v.v\gets \ask_{j(N)};\Pi_i\oc v)/\choose_{p_i}]\converges_u \skipp
    \]
  if and only if $u$ covers a sequence $U$ such that $M\converges_U\skipp$.
  By our Computational Adequacy result for \IAX (Propositions \ref{PropKleisliSoundness} and \ref{PropKleisliAdequacy}), this means that for all $u\in j(N)^*$, $u\in \Acc(\deno M)$ (for this particular form of $\deno M$) if and only if $u\in \U$.
  Therefore, $(N,\Acc(\deno M)) = (N,\U')$, where $\U'\subset j(N)^*$ is the set of all sequences $u$ that cover some $U$ such that $M\converges_U\skipp$ via the projections $\pi_i$.
  Lastly, we note that $(N,\U')$ is equivalent to $B(M)$, through the morphism $f\from N \to p_1\tensor\cdots\tensor p_n$.
\end{proof}

\section{Equational Soundness}

We transfer to an Equational Soundness result in our standard way.
First, we make a definition of observational equivalence of \IAXX terms.

\begin{definition}
  Let $M$ be a closed term of \IAXX of type $\com$ mentioning $p_1,\cdots,p_n$.
  Let $S(M)$ be the set of all sequences $U$ such that $M\converges_U\skipp$.

  We define $B(M)$, the \emph{behaviours of $M$}, to be the equivalence class corresponding to the pair
  \[
    (p_1\tensor \cdots \tensor p_n, \U)\,,
    \]
  where $\U$ is the set of all sequences $u\in j(p_1\tensor \cdots \tensor p_n)^*$ that cover some sequence $U\in S(M)$, via the projections
  \[
    j(p_1\tensor \cdots \tensor p_n) \xrightarrow{m^j} j(p_1) \times \cdots \times j(p_n) \xrightarrow{\pr_i} j(p_i)\,.
    \]
\end{definition}

\begin{definition}[Observational Equivalence]
  Let $M,M'\from T$ be closed terms of \IAXX.  
  We say that $M$ and $M'$ are \emph{observationally equivalent} if $B(C[M])$ and $B(C[M'])$ are equivalent for all contexts $C\from \com$ with a hole of type $T$.
\end{definition}

We then make definitions that will mirror this equivalence in the denotational semantics.

\begin{definition}[Equivalence of morphisms $1 \to \bC$]
  Let $\sigma,\tau\from 1 \to \bC$ be morphisms in $\G/\oppcat\X$, considered as morphisms $\sigma\from j(p) \to \bC$ and $\tau \from j(q) \to \bC$ in $\G$.  
  We say that $\sigma\approx\tau$ if $(p,\Acc(\sigma))$ is equivalent to $(q,\Acc(\tau))$.
\end{definition}

\begin{definition}[Intrinsic Equivalence]
  Let $\sigma,\tau\from A \to B$ be morphisms in $\G/\oppcat\X$.  
  Then we say that $\sigma\sim\tau$ if for all $\alpha\from (A\to B) \to \bC$, we have $\Lambda(\sigma);\alpha \approx \Lambda(\tau);\alpha$.
\end{definition}

Now we can prove Equational Soundness as we did in Proposition \ref{PropEquationalSoundness}.

\begin{theorem}[Equational Soundness for \IAXX]
  Let $M,M'\from T$ be closed terms of \IAXX such that $\deno{M}\sim\deno{N}$.
  Then $M$ and $M'$ are observationally equivalent.
\end{theorem}
\begin{proof}
  First suppose that $M$ and $M'$ are not observationally equivalent -- so there is some context $C$ such that $B(C[M])$ and $B(C[M'])$ are inequivalent.
  Now $B(C[M])$ is equivalent to $(N,\U)$ and $B(C[M'])$ is equivalent to $(N,\U')$, where $\U\subset j(N)^*$ is the set of sequences that cover some $U\in S(C[M])$ and $\U'$ the set of sequences that cover some $U\in S(C[M'])$ via the projections $\pi_i$.

  Let $\alpha$ be the denotation of the term-in-context $f\from T \ts C[f]$.  
  Then $\Lambda(\deno{M});\alpha$ is the denotation of $C[M]$ and $\Lambda(\deno{M'});\alpha$ the denotation of $C[M']$.  
  By Theorem \ref{TheComputationalAdequacyIAXX}, the sets $(N,\Acc(\Lambda(\deno{M});\alpha))$ and $(N,\Acc(\Lambda(\deno{M'})))$ are inequivalent, and so $\deno M \not\sim\deno{M'}$.
\end{proof}

Our setup is too general for us to prove Full Abstraction, but we will be able to prove a Full Abstraction result in an important special case: that of Probabilistic Algol.

\section{Probability}

We now specialize to the case where $\X$ is a category of random variables on some fixed probability space $(\Omega,\F,\bP)$, in order to model a probabilistic language.
For our purposes, it will suffice to take $\Omega$ to be the real interval $(0,1)$ with its Borel $\sigma$-algebra and measure.
A \emph{random variable} on $\Omega$ is a measurable function $V \from \Omega \to X$.  
Given such a random variable, and $A\subset X$, we write $\bP(V \in A)$ for $\bP(V\inv(A))$, and $\bP(V = x)$ for $\bP(V \in \{x\})$.

The category $\X = \Rv_\Omega^{FS}$ will then be the category whose objects are random variables of \emph{finite support}; that is, measurable functions $V\from\Omega \to X$, where $X$ is a discrete space, such that there is some finite subset $Y\subset X$ such that $\bP(V\in Y) = 1$.

The morphisms in $\Rv_\Omega^{FS}$ from $V\from \Omega \to X$ to $W \from \Omega \to Y$ are probability-preserving functions $X \to Y$; i.e., functions $X \to Y$ such that for all $A \subset Y$, we have $\bP(f(V) \in A) = \bP(W \in A)$.

Recall that the tensor product of two random variables $V\from \Omega \to X$ and $W \from \Omega \to Y$ is their pairing $V \tensor W = \langle V,W\rangle \from \Omega \to X \times Y$.

We define a language Probabilistic Algol (PA) to be the sublanguage of IA${}_{\Rv_\Omega^{FS}}$ generated by the terms of Idealized Algol and the terms
\[
  \choose_{V_p}\,,
  \]
where $p\in [0,1]$, and where we have identified $V_p$ is the Bernoulli random variable
\[
  \Omega \to \bB
  \]
that returns $\true$ if its input is less than $p$ and $\false$ if it is greater than or equal to $p$.

The denotation of any term of PA, then, will be an (equivalence class of) morphisms
\[
  \ul{\bB^n} \xrightarrow{m} \ul{\bB}^n \to \deno{T}\,,
  \]
together with some random variable taking values in $\bB^n$.
We have used an underline to distinguish between the object $\ul{\bB^n}$ of $\G$ corresponding to the set $\bB^n$ of $n$-tuples of booleans and the $n$-fold Cartesian power $\ul{\bB}^n$ in $\G$ of the object $\bB$ corresponding to the set of booleans.

Lastly, given such a random variable $V\from\Omega \to \bB^n$, there is a random variable $\tilde{V} \from \Omega \to \bN$ such that for each $\vec{v}\in\bB^n$, we have
\[
  \bP\left(\tilde{V} = \sum_{i=1}^n 2^{i-1}\vec{v}_i\right) = \bP(V = \vec{v})
  \]
and such that $\bP(\tilde{V}=k)=0$ for any $k\ge 2^n$.
Then we have a function $f \from \bN \to \bB^n$ that sends $\sum_{i=1}^n 2^{i-1}a_i$ to $(a_1,\cdots,a_n)$ and sends $k\ge 2^n$ to some fixed value (say, $(\false,\cdots,\false)$), and then we have
\[
  f\circ\tilde{V}=V\,.
  \]
Moreover, $\tilde{V}$ has finite support.

Now suppose that $X$ is a finite discrete probability space.  
Then the set $X^\omega$ of all infinite sequences of elements of $X$ may be given the product topology, and equipped with the resulting Borel $\sigma$-algebra.  
A basic open subset of $X^\omega$ is a set $\S\subset X^\omega$ for which there exists some $n$ such that if $s\in \S$ and $t$ is a sequence such that $s$ and $t$ are identical on the first $n$ terms, then $t\in\S$.
We can define a pre-probability measure on these basic open sets by setting
\[
  \bP(\S) = \sum_{u\in \S\vert_n} \prod_{i=0}^{n-1} \bP(u^{(i)})\,,
  \]
where $\S\vert_n$ is the set of all length-$n$ prefixes of elements of $\S$.

Then the Carath\'{e}odory Extension Theorem tells us that there is a unique extension of this to a probability measure on the whole space (see, for example, \cite[1.1.4]{StochasticCalculusII}).

If $V \from \Omega \to Z$ is a finitely-supported random variable, then $V$ induces a probability measure on its support $\im(V)\subset Z$.
This gives us a probability measure on $\im(V)^*$, which we can extend to a probability measure on $Z^*$ by setting
\[
  \bP(A) = \bP(A \cap \im(V)^*)
  \]
for any $A \subset Z^*$.

\begin{definition}
  Let $V\from \Omega \to X$ be a finitely supported random variable and let $\U\subset X^*$ be a set of sequences.  
  Then we define
  \[
    \bP(V,\U) = \bP(\U^\omega)\,,
    \]
  where $\U^\omega\subset X^\omega$ is the set of all infinite sequences having some prefix in $\U$.
  Note that $\U^\omega$ is an open subset of $X^\omega$, so is in particular measurable.

  An easier way to define $\bP(V,\U)$ is that it is the sum of the probabilities of all the sequences in $\U$; i.e.:
  \[
    \bP(V,\U) = \sum_{u\in \U} \prod_{i=0}^{|u|-1} \bP(u^{(i)})\,,
    \]
  where the infinite sum refers to the supremum of the sums of all finite subsets of $\U$.
\end{definition}

\begin{proposition}
  Suppose that $(V,\U)$ and $(W,\V)$ are equivalent pairs, in the sense of Definition \ref{DefEquivalenceOfPairs}, where $V \from \Omega \to X$, $W \from \Omega \to Y$ are finitely-supported random variables, and $\U\subset X^*$, $\V\subset Y^*$ are sets of sequences.
  Then $\bP(V,\U) = \bP(W, \V)$.
  \label{PropProbabilityWellDefined}
\end{proposition}
\begin{proof}
  Without loss of generality, we may assume that there is a probability-preserving function $f \from X \to Y$; i.e., a function such that for any $A \subset Y$ we have $\bP(W \in A) = \bP(f(V) \in A)$ and such that $\U = f_*\inv(\V)$.
  Then we have
  \begin{IEEEeqnarray*}{rCl}
    \bP(V,\U) & = & \bP(V,f_*\inv(\V)) \\
    & = & \sum_{\stackrel{u\in X^*}{f_*u\in\V}} \prod_{i=0}^{|u|-1} \bP(V = u^{(i)}) \\
    & = & \sum_{v\in \V}\sum_{\stackrel{u\in X^*}{f_*u = v}} \prod_{i=0}^{|u|-1} \bP(V = u^{(i)}) \\
    & = & \sum_{v\in \V}\prod_{i=0}^{|v|-1} \sum_{\stackrel{x\in X }{ f(x) = v}} \bP(V = x) \\
    & = & \sum_{v\in \V} \prod_{i=0}^{|v|-1} \bP(f(V) = v^{(i)}) \\
    & = & \sum_{v\in \V} \prod_{i=0}^{|v|-1} \bP(W = v^{(i)}) \\
    & = & \bP(W,\V)\,.\hspace{1em plus 1fill}\qedhere
  \end{IEEEeqnarray*}
\end{proof}

We now make a definition relating the operational behaviour we have defined for \IAXX to probability.

\begin{definition}
  Let $M\from \com$ be a closed term of PA mentioning probabilities $p_1,\cdots,p_n$.
  We define $\bP(M\converges)$ to be $\bP(B(M))$; i.e.,
  \[
    \bP(V_{p_1}\tensor \cdots \tensor V_{p_n},\U)\,,
    \]
  where $\U$ is the set of all sequences $u\in j(V_{p_1}\tensor \cdots \tensor V_{p_n})^*$ that cover some sequence $U$ such that $M\converges_U\skipp$.
  \label{DefProbConverges}
\end{definition}

We can define a corresponding notion for morphisms in the denotational semantics.

\begin{definition}
  Let $\sigma \from 1 \to \bC$ be a morphism in $\G/\oppcat{(\Rv_\Omega^{FS})}$, considered as a morphism $\sigma \from 1 \to (X \to \bC)$ in $\G$, together with a finitely-supported random variable $V \from \Omega \to X$.  

  Then we define $\bP(\sigma\downarrow)$ to be
  \[
    \bP(V, \Acc(\sigma))\,.
    \]
  \label{DefProbNonBot}
\end{definition}

\begin{remark}
  By Propositions \ref{PropProbabilityWellDefined} and \ref{PropAccEquivalent}, Definitions \ref{DefProbConverges} and \ref{DefProbNonBot} are well defined.
\end{remark}

Now we are ready to prove computational adequacy.

\begin{proposition}[Computational Adequacy for PA]
  Let $M\from \com$ be a closed term of PA.  
  Then $\bP(M\converges) = \bP(\deno M \downarrow)$.
  \label{PropComputationalAdequacyPa}
\end{proposition}
\begin{proof}
  By Theorem \ref{TheComputationalAdequacyIAXX}, $B(M)$ is equivalent to $(p,\Acc(\deno M))$.  
  Therefore, by Proposition \ref{PropProbabilityWellDefined}, $\bP(M\converges) = \bP(B(M)) = \bP(V,\Acc(\deno M)) = \bP(\deno M\downarrow)$.
\end{proof}

We can define observational equivalence for terms.

\begin{definition}
  Let $M,N\from T$ be closed terms of PA.  
  Then we say that $M$ and $N$ are (probabilistically) \emph{observationally equivalent} if for all contexts $C\from \com$ with a hole of type $T$, we have
  \[
    \bP(C[M]\converges) = \bP(C[N]\converges)\,.
    \]
\end{definition}

We then have the usual corresponding definition in the denotational semantics.

\begin{definition}
  Let $\sigma,\tau\from A \to B$ be morphisms in $\G/\oppcat{(\Rv_\Omega^{FS})}$.  
  We write $\sigma\sim_{\bP}\tau$ if for all morphisms $\alpha \from (A \to B) \to \bC$ in $\G/\oppcat{(\Rv_\Omega^{FS})}$ we have
  \[
    \bP(\Lambda(\sigma);\alpha\downarrow) = \bP(\Lambda(\tau);\alpha\downarrow)\,.
    \]
\end{definition}

Then, by our standard argument, we may derive Equational Soundness from Computational Adequacy.

\begin{proposition}
  Let $M,N\from T$ be closed terms of PA such that $\deno M \sim_{\bP} \deno N$.  
  Then $M$ and $N$ are probabilistically observationally equivalent.
\end{proposition}

Our next goal will be to prove the converse to this result: Full Abstraction.

\section{Full Abstraction for Probabilistic Algol}

\begin{proposition}
  Let $V\from \Omega \to X$ be a finitely supported random variable.  
  Then there exist $p_1,\cdots,p_n$ and a function
  \[
    f \from \bB^n \to X
    \]
  such that for all $x\in X$ we have $\bP(V = x) = \bP(f(V_{p_1},\cdots,V_{p_n}) = x)$.
  \label{PropFiniteSupport}
\end{proposition}
\begin{proof}
  Recall that the $V_p$ are not independent in our formulation; indeed, if $p<q$, then $V_p = \true \Rightarrow V_q = \true$.

  Enumerate those elements $x\in X$ such that $\bP(V = x) \ne 0$ as $x_1,\cdots,x_n$, and for each $k = 1,\cdots,n$, define
  \[
    p_k = \sum_{i=1}^n \bP(X = x_i)\,.
    \]
  Note that we must have $p_n = 1$.  
  Then we define
  \[
    f(\vec{b}) = \begin{cases}
      x_1 & \text{if $\vec{b} = \vec{\false}$} \\
      \min\{k \suchthat b_k = \true\} & \text{otherwise}\,.
    \end{cases}
    \]
  Fix $x\in X$.  
  If $x$ is not one of the $x_i$, then we have $\bP(f(V_{p_1},\cdots,V_{p_n}) = x) = 0 = \bP(V = x)$.  
  Otherwise, suppose $x = x_k$.  
  If $\omega\in\Omega$ and $p_{k-1}\le\omega<p_k$, then $V_{p_k}(\omega) = \true$, and $V_{p_i}(\omega) = \false$ for all $i \le k$.  
  So $f((V_{p_1}\tensor\cdots\tensor V_{p_n})(\omega)) = x_k$.
  If $\omega<p_{k-1}$, then $V_{p_{k-1}}(\omega) = \true$, so $f((V_{p_1}\tensor\cdots\tensor V_{p_n})(\omega)) \ne x_k$.
  If $\omega\ge p_k$, then $V_{p_k}(\omega) = \false$, so $f((V_{p_1}\tensor\cdots\tensor V_{p_n})(\omega)) \ne x_k$.  
  Therefore, 
  \[
    \bP(f(V_{p_1},\cdots,V_{p_k}) = x_k) = \bP([p_{k-1},p_k)) = p_k - p_{k-1} = \bP(X = x_k)\,.
    \]
  It follows that $f$ is probability preserving in the sense required.
\end{proof}

The most important consequence of Proposition \ref{PropFiniteSupport} is that it tells us that any morphism $A\to B$ in $\G/\oppcat{(\Rv_{\Omega}^{FS})}$ may be considered as a pair 
\[
  (V_{p_1}\tensor\cdots\tensor V_{p_n},f \from A \to (\bB^n \to B))
  \]
for appropriately chosen $p_1,\cdots,p_n$.

\begin{definition}
  Let $\sigma\from A \to B$ be a morphism in $\G/\oppcat{(\Rv_{\Omega}^{FS})}$.  
  We say that $\sigma$ is \emph{compact} if it is compact when considered as a morphism in $\G$.
\end{definition}

\begin{remark}
  When we say `considered as a morphism in $\G$' in the above definition, we mean `in at least one of its possible interpretations as a morphism in $\G$'.  
  Note, however, that the continuous image of a compact element is compact, and so if we pass to a new representative of $\sigma$ by composing on the left by the image of some morphism in $\Rv_{\Omega}^{FS}$, then the resulting representative of $\sigma$ will also be compact.

  Note that if $\G$ is the category of games, this compactness property is invariant under the choice of representative for $\sigma$.
\end{remark}

\begin{proposition}[Compact definability]
  Let $T$ be an Idealized Algol type and let $\sigma \from 1 \to \deno{T}$ be a compact morphism in $\G/\oppcat{(\Rv_{\Omega}^{FS})}$.  
  Then there is some closed term $M\from T$ such that $\sigma = \deno{M}$.
  \label{PropProbabilityCompactDefinability}
\end{proposition}
\begin{proof}
  Let $(V, \sigma \from 1 \to (X \to \deno{T})$ be a compact representative of $\sigma$, where $X$ is a set and $V$ is a finitely-supported random variable taking values in $X$.  
  By Proposition \ref{PropFiniteSupport}, we may choose $p_1,\cdots,p_n$ such that there is a probability-preserving function
  \[
    f \from V_{p_1} \tensor \cdots \tensor V_{p_n} \to \deno{T}\,.
    \]
  After composing on the right by $(\sigma \to \deno{T})$, we may assume that $\sigma$ is of the form
  \[
    (V_{p_1}\tensor\cdots\tensor V_{p_n}, \sigma \from 1 \to (\ul{\bB^n} \to \deno{T}))\,.
    \]

  Now this $\sigma$ necessarily factors as
  \[
    1 \xrightarrow{\hat{\sigma}}
    (\ul{\bB}^n \to \deno T) \xrightarrow{(m \to \deno T)}
    (\ul{\bB^n} \to \deno T)\,,
    \]
  where $\hat{\sigma}$ is compact.  
  Then, by compact definability in $\G$, $\hat\sigma$ is the denotation of some term $N \from \bool \to \cdots \to \bool \to T$, and it follows that our original morphism $\sigma$ in $\G/\oppcat{(\Rv_{\Omega}^{FS})}$ is the denotation of
  \[
    N\,\choose_{p_1}\,\cdots\,\choose_{p_n}\,.\qedhere
    \]
\end{proof}

\begin{theorem}[Full Abstraction for PA]
  Let $M,N \from T$ be observationally equivalent terms of PA.  
  Then $\deno M \sim_{\bP} \deno N$.
  \label{TheFullAbstractionPa}
\end{theorem}
\begin{proof}
  Suppose that $\deno M \not\sim_{\bP} \deno N$.
  So there is some $\alpha \from \deno T \to \bC$ such that $\bP(\deno{M};\alpha\downarrow) \ne \bP(\deno{N};\alpha\downarrow)$.

  Let $\bP(\deno M;\alpha\downarrow) = p$ and $\bP(\deno N;\alpha\downarrow) = q$, and suppose without loss of generality that $p>q$.
  Now there must be some finite subset $\V$ of $\Acc(\deno M;\alpha\downarrow)$ such that the combined probability of the sequences in $\V$ is still greater than $q$.  
  For each $u\in \V$, we can choose some compact $\alpha_u\subset \alpha$ such that $u$ is still accepted by $\deno M;\alpha_u$, by algebraicity.
  Since the set of compact elements below $\alpha$ is directed, there is some $\alpha'\subset \alpha$ such that $\alpha_u\subset \alpha'$ for each $u\in \V$.  
  Then we have
  \begin{mathpar}
    \bP(\deno{M};\alpha'\downarrow) > q
    \and
    \bP(\deno{N};\alpha'\downarrow) \le q\,,
  \end{mathpar}
  and therefore $\bP(\deno{M};\alpha'\downarrow) \ne \bP(\deno{N};\alpha'\downarrow)$.

  By Proposition \ref{PropProbabilityCompactDefinability}, $\alpha'$ is the denotation of some term $L\from T \to \com$, and our Computational Adequacy result (Proposition \ref{PropComputationalAdequacyPa}) then tells us that 
  \[
    \bP(L\,M\converges)=\bP(\deno M;\alpha'\downarrow)\ne\bP(\deno N;\alpha'\downarrow) = \bP(L\,N\converges)\,.
    \]
  Therefore, $M$ and $N$ are not observationally equivalent.
\end{proof}

\section{Comparison with a Kleisli Category Model}

An alternative way to model probability is using the Kleisli category construction that we considered in Chapter \ref{ChapMonads}.  
Specifically, we can consider the language IA${}_\bB$ as a probabilistic Algol variant, by treating the term $\ask_\bB$ as a coin flip that returns $\true$ or $\false$ each with probability $\frac12$.

Given a closed term $M\from\com$ of IA${}_\bB$, we define
\[
  \bP(M\converges) = \sum_{u\in\bB^*\colon M\converges_u\skipp} 2^{-|u|}\,,
  \]
since $2^{-|u|}$ is the probability of a particular sequence $u$ of $\true$ and $\false$ values occurring.
Here, the infinite sum means the supremum over all sums of finite subsets.
Similarly, given a Kleisli morphism $\sigma\from 1 \to \bC$ -- i.e., a morphism $\sigma\from 1 \to (\bB \to \bC)$ in $\G$, we can define
\[
  \bP(\sigma\downarrow) = \sum_{u\in\Acc(\sigma)} 2^{-|u|}\,.
  \]
Our Computational Adequacy result for IAX (Propositions \ref{PropKleisliSoundness} and \ref{PropKleisliAdequacy}) then gives us a Computational adequacy result for this model.

\begin{proposition}
  Let $M\from \com$ be a closed term of IA${}_\bB$.  
  Then $\bP(M\converges) = \bP(\deno M\downarrow)$.
\end{proposition}
\begin{proof}
  Propositions \ref{PropKleisliSoundness} and \ref{PropKleisliAdequacy} tell us that the set of sequences $u$ such that $M\converges_u\skipp$ is the same as the set $\Acc(\deno{M})$.
\end{proof}

We can define probabilistic observational equivalence and the probabilistic intrinsic equivalence $\sim_{\bP}$ in exactly the same way as we did for PA.  
Then the same argument we used in Theorem \ref{TheFullAbstractionPa} proves Full Abstraction for this model.

\begin{theorem}
  Let $M,N\from T$ be closed terms of IA${}_\bB$.  
  Then $M$ and $N$ are probabilistically observationally equivalent if and only if $\deno{M}\sim_{\bP}\deno{N}$.
\end{theorem}

This model is much easier to realize.  
What do we gain, then, from our more complicated model?  

The most obvious answer is that our original model allowed us to work with arbitrary probabilities, rather than using a fixed coin with probability $\frac12$.  
This is not such a great advantage as it might seem, since if $p\in[0,1]$ is any real number whose binary expansion is computable as a function $\bN \to \bB$, then we can simulate $\choose_p$ within the probabilistic version of IA${}_\bB$\footnote{Idea: build up a sequence of intervals $I_n$ of width $2^n$ by repeatedly flipping the coin and using the result to choose either the lower or the upper half of $I_{n-1}$.  
If at any point the upper bound of $I_n$ is less than or equal to $p$ (which can be checked by computing the first $n$ terms of the binary expansion of $p$, then return $\true$.  
If at any point the lower bound of $I_n$ is greater than or equal to $p$, return $\false$.}.

Perhaps a better way of thinking about the difference between the two models, then, is to consider what the denotation of a term actually looks like.  
In the language IA${}_\bB$, we can define a term that converges to $\true$ with probability $\frac23$ and to $\false$ with probability $\frac13$ by
\[
  \Y_\bool(\lambda b.\If \choose_{\frac12} \Then \true \Else (\If \choose_{\frac12} \Then b \Else \false))\,.
  \]
Here, we have renamed $\ask_{\bB}$ to $\choose_{\frac12}$ to give a better idea of what the term does in the probabilistic setup.

Now the denotation of this term in $\Kl_{R_{\bB}}\G$ is given by the denotation of the term-in-context
\[
  c\from\bool\ts\Y_{\bool}(\lambda b.\If c \Then \true \Else (\If c \Then b \Else \false))
  \]
in $\G$.
If $\G$ is the category of games, then this morphism is the strategy with maximal plays taking one of the following two forms.
\begin{mathpar}
  q(q\false q\true)^nq\true \true
  \and
  q(q\false q\true)^nq\false q\false\false
\end{mathpar}
In other words, it is not at all clear from the denotation that the term should give $\true$ with probability $\frac23$ and $\false$ with probability $\frac13$.

In $\G/\Rv_{\Omega}^{FS}$, however, we can model a term that has the same behaviour using the morphism
\[
  \left(V_{\frac23}, \id_\bB\right)\,,
  \]
which makes it much clearer what the probabilistic behaviour is.

\section{Game Semantics and Probability}

We now specialize to the case that $\G$ is the category of arenas and single-threaded strategies that we developed in Chapters \ref{ChapGames} and \ref{ChapFullAbstraction}.
This will allow us to capture the close relationship between the sequences $u$ that we have been considering and the plays in a strategy.

\begin{definition}
  Let $X$ be a set and let $u\in X^*$ be a sequence.  
  Consider $X$ as an arena $\ul X$.  
  Then we write $qu$ for the play in $\ul X$ given by
  \[
    q\,u^{(1)}\,\cdots\,q\,u^{(|u|-1)}\,.
    \]
  Note that any $P$-position in $\ul X$ is of the form $qu$ for some sequence $u$.  
\end{definition}

\begin{definition}
  Let $A$ be an arena, let $V$ be a random variable taking values in a set $X$, and let $\sigma \from X \to A$ be a single-threaded strategy.  
  We may consider $X$ as a game $\ul X$.  
  Let $s$ be a legal play of $A$.  
  If $t\in\sigma$, we write $t/s$ if $t\vert_A = s$ and if the last move of $t$ is the last move of $s$ (this implies in particular that $t\vert_{\ul X}$ is a $P$-position in $\ul X$).
  Then we define
  \[
    \Acc_s(\sigma) = \{u\in X^*\suchthat \exists t\in\sigma\esuchthat t/s,\,t\vert_{\ul X} = qu\}\,.
    \]
  We define
  \[
    \bP_V(s\in\sigma) = \bP(V, \Acc_s(\sigma))\,.
    \]
\end{definition}

We would like use this definition to define $\bP(s\in\sigma)$ for $\sigma$ a morphism in $\G/\oppcat{(\Rv_{\Omega}^{FS})}$, but we first need to check that this is well-defined with respect to the equivalence relation on \Mellies morphisms.

\begin{proposition}
  Let
  \begin{mathpar}
    (V\from \Omega \to X, \sigma \from \ul X \to A)
    \and
    (W\from \Omega \to Y, \sigma \from \ul Y \to A)
  \end{mathpar}
  be representatives of the same morphism $A \to B$ in $\G/\oppcat{(\Rv_{\Omega}^{FS})}$, where $X$ and $Y$ are sets.  
  Let $s$ be a legal play of $A$.
  Then $(V,\Acc_s(\sigma))$ and $(W,\Acc_s(\sigma'))$ are equivalent as pairs in the sense of Definition \ref{DefEquivalenceOfPairs}.
\end{proposition}
\begin{proof}
  It suffices by induction to prove this in the case that the two representatives are related by the relation that generates the equivalence relation on morphisms; i.e., that there is a probability-preserving function $f\from X \to Y$ such that $\sigma=\sigma';(j(f) \to A)$.

  Let $u\in X^*$.  
  Then we have
  \begin{IEEEeqnarray*}{rCl}
    u \in \Acc_s(\sigma) & \Leftrightarrow & \exists t\in \sigma \esuchthat t/s,\,t\vert_{\ul X}=qu \\
    & \Leftrightarrow & \exists t \in \sigma';(j(f) \to A) \esuchthat t/s,\,t\vert_{\ul X}=qu \\
    & \Leftrightarrow & \exists t' \in \sigma' \esuchthat t/s,\,t\vert_{\ul Y} = q(f_*u) \\
    & \Leftrightarrow & f_*u \in \Acc_s(\sigma')\,.
  \end{IEEEeqnarray*}
  Therefore, $(V,\Acc_s(\sigma))$ and $(W,\Acc_s(\sigma'))$ are equivalent.
\end{proof}

It follows by Proposition \ref{PropProbabilityWellDefined} that $\bP_V(s\in\sigma)=\bP_W(s\in\sigma')$ for all $s$.  
Therefore, the following is well-defined.

\begin{definition}
  Let $\sigma \from A \to B$ be a morphism in $\G/\oppcat{(\Rv_{\Omega}^{FS})}$, where $\G$ is the category of arenas and single-threaded strategies, and suppose that $\sigma$ is given (after currying) by a morphism
  \[
    \tilde\sigma \from \ul X \to (A \to B)
    \]
  in $\G$, together with a random variable $V$ taking values in $X$.
  Then we define
  \[
    \bP(s\in\sigma) = \bP_V(s\in\tilde\sigma)\,.
    \]
\end{definition}

\begin{definition}
  Let $\sigma,\sigma'\from A \to B$ be morphisms in $\G/\oppcat{(\Rv_{\Omega}^{FS})}$, where $\G$ is the category of arenas and single-threaded strategies.  
  We say that $\sigma\approx_{\bP}\sigma'$ if for all legal plays $s$ of $A \to B$ we have
  \[
    \bP(s\in\sigma) = \bP(s\in\sigma')\,.
    \]
\end{definition}

We now relate this definition to \Mellies composition of strategies.

\begin{definition}
  Let $A,B,C$ be arenas and let $s$ be a play in $A\to C$.  
  We write
  \[
    \wit_B(s) = \{\s\in\Int(A,B,C)\suchthat \s\vert_{A,C}=s\}\,.
    \]
\end{definition}

\begin{proposition}
  Let $\sigma\from A \to B$, $\tau\from B \to C$ be morphisms in $\G/\oppcat{(\Rv_{\Omega}^{FS})}$.  
  Let $s$ be a legal play in $A \to C$.
  Then
  \[
    \bP(s\in\sigma;\tau) = \sum_{\s\in\wit_B(s)}\bP(\s\vert_{A,B}\in\sigma)\bP(\s\vert_{B,C}\in\tau)\,.
    \]
\end{proposition}
\begin{proof}
  Suppose that $\sigma$ and $\tau$ are given by (equivalence classes of) pairs
  \begin{mathpar}
    (V\from\Omega\to X, \tilde \sigma \from A \to (\ul X \to B))
    \and
    (W\from\Omega\to Y, \tilde \tau \from B \to (\ul Y \to C))\,,
  \end{mathpar}
  where $V$ and $W$ are random variables and $\tilde \sigma,\tilde\tau$ are strategies in $\G$.
  Then the composition $\sigma;\tau$ in $\G/\oppcat{(\Rv_{\Omega}^{FS})}$ is given by $V\tensor W\from\Omega \to X\times Y$, together with the \Mellies composition
  \[
    A \xrightarrow{\tilde\sigma}
    (\ul{X} \to B) \xrightarrow{\ul{X} \to \tilde\tau}
    (\ul X \to (\ul Y \to C)) \to
    ((\ul X \times \ul Y) \to C) \xrightarrow{\mu\to C}
    (\ul{X\times Y} \to C)\,,
    \]
  where $\mu$ is $\langle\ul{\pr_X},\ul{\pr_Y}\rangle$.

  First suppose that $s$ is a sequence in $A \to C$, and let $\s\in\wit_B(s)$.  
  Let $\t$ be a sequence in $\sigma\|(\ul X \to \tau)$ such that $\t\vert_{A,B,C}=\s$.  
  Then $\t\vert_{A,\ul X \to B}\in\sigma$ and $\t\vert_{B,\ul Y,C}\in\tau$.

  Moreover, since we have $\t\vert_{A,B}=\s\vert_{A,B}$ and $\t\vert_{B,C}=\s\vert_{B,C}$, we must have
  \begin{mathpar}
    \t\vert_{\ul X} \in \Acc_{\s\vert_{A,B}}(\sigma)
    \and
    \t\vert_{\ul Y} \in \Acc_{\s\vert_{B,C}}(\tau)\,,
  \end{mathpar}
  where we have identified a play $qu$ occurring in the arena $\ul X$ with its underlying sequence $u$ of elements of $X$, and likewise for $Y$.

  This gives us a function
  \[
    \{\t\in\sigma\|(\ul X \to \tau)\suchthat \t\vert_{A,B,C}=\s\} \to \Acc_{\s\vert_{A,B}}(\sigma)\times\Acc_{\s\vert_{B,C}}(\tau)\,.
    \]
  We claim that this function is a bijection.

  Indeed, suppose that $\t,\t'$ are two interactions in $\sigma\|(\ul X \to \tau)$ such that $\t\vert_{A,B,C}=\t'\vert_{A,B,C}=\s$, $\t\vert_{\ul X} = \t'\vert_{\ul X}$ and $\t\vert_{\ul Y}=\t'\vert_{\ul Y}$.
  We claim that $\t = \t'$.

  Indeed, suppose for a contradiction that $\t\ne\t'$: then there are prefixes $\r p\prefix \t$ and $\r q \prefix \t'$, where $\r$ is the longest common subsequence of $\t$ and $\t'$ and $p\ne q$ are moves.

  By our earlier analysis (see, for example, the proof of \ref{PropComposition}), we know that $p$ and $q$ must either both occur in the $A \to (\ul X \to B)$-component, or both in the $(\ul X \to B) \to (\ul X \to (\ul Y \to C))$-component.
  But since $\t,\t'\in\sigma\|(\ul X \to \tau)$ are both interactions of deterministic strategies, we also know that they must both be $O$-moves in that component -- otherwise, they would have to be equal.
  In particular, neither $p$ nor $q$ may be a move in the middle component $\ul X \to B$, since then it would be a $P$-move in one of the two components.

  Therefore, $p$ and $q$ are both $O$-moves in one of the outer components $A$ and $\ul X \to (\ul Y \to C)$.  
  By Corollary \ref{CorSwitchingCondition}, we know that only Player $P$ may switch between games in $\ul X \to (\ul Y \to C)$, and therefore $p$ and $q$ must occur in the same component game -- i.e., both in $A$, both in $\ul X$, both in $\ul Y$ or both in $C$.  
  But now the conditions that $\t\vert_{A,B,C}=\t'\vert_{A,B,C}$, $\t\vert_{\ul X}=\t'\vert_{\ul X}$ and $\t\vert_{\ul Y}=\t'\vert_{\ul Y}$ mean that we must have $p=q$, which is the desired contradiction.

  For surjectivity, let $u\in\Acc_{\s\vert_{A,B}}(\sigma)$ and $v\in\Acc_{\s\vert_{B,C}}(\tau)$.  
  We seek a sequence $\t\in\sigma\|(\ul X \to \tau)$ such that $\t\vert_{A,B,C}=\s$, $\t\vert_{\ul X}=qu$ and $\t\vert_{\ul Y}=qv$.
\end{proof}

\section{Danos-Harmer Probabilistic Game Semantics}

\bibliographystyle{alpha2}
\bibliography{../common/phd_bibliography}

\end{document}

\chapter{Further Directions}
\label{ChapFurtherDirections}

\section{Stateless Languages}
\label{SecStatelessLanguages}

In this thesis, the base language we have used has been the stateful language Idealized Algol.  
One area to explore is whether the techniques we have developed allow us to build models of effectful versions of stateless languages such as PCF.

One immediate problem we face is that the addition of nondeterministic effects to PCF allows us to distinguish terms that cannot be distinguished within PCF itself.  
Consider, for instance, the PCF term
\[
  M = \lambda x^{\bool}.\If x \Then \Omega \Else (\If x \Then \true \Else \Omega)
  \]
is observationally equivalent (in PCF) to $\Omega$ -- informally, because the term $x$ must have the same `value' both time it is called, so if it is false the first time, then it must be false the second time.
However, the same term inside PCF with finite nondeterminism is not observationally equivalent to $\Omega$: if we substitute nondeterministic choice in for $x$, then it could be false the first time and true the second, causing the term to converge to $\true$.

This means that if $\G$ is a `truly fully abstract' model of PCF -- i.e., one in which observational equivalence of terms corresponds to equality of morphisms rather than intrinsic equivalence\footnote{As an example, take any model that is full abstract in our sense, and take the quotient by the intrinsic equivalence relation.} -- then there can be no natural inclusion functor from $\G$ into a model of PCF with finite nondeterminism.  
Indeed, in such a model $\deno{M}$ and $\deno{\Omega}$ are the same morphism, so they would have to be sent to the same morphism in the model of nondeterministic PCF.

This is a problem for us, since both our techniques -- the Kleisli category and the $\G/\X$ construction -- have a natural functor coming out of the original category $\G$.

All is not lost, however, since we can still try and impose some condition on the original model $\G$ of PCF to say that it is still fine-grained enough to distinguish between certain morphisms that are denotations of observationally equivalent PCF terms.  
One convenient way to do this is to ask that $\G$ should embed into some model $\G^{IA}$ of Idealized Algol.  
An obvious case when this holds is when $\G$ is the (unquotiented) Hyland-Ong category of arenas and innocent strategies, and $\G^{IA}$ is the category of arenas and single-threaded strategies.

For example, fix $X\in\{\bB,\bN\}$ and let \emph{Nondeterministic PCF} be the language PCF, together with a constant $\ask_X \from \bool$ that plays the role of a nondeterministic oracle.  
I.e., it is the language with types given by the following inductive grammar.
\[
  T \Coloneqq \bool \mid \nat \mid T \to T
  \]
And with a type theory as shown in Figure \ref{FigNDPcfTypeTheory}.
\begin{figure}
  \begin{mathpar}
    \inferrule*{ }{\Gamma,x\from T \ts x \from T}
    \and
    \inferrule*{\Gamma\ts M \from S \to T \\ \Gamma \ts N \from S}{\Gamma\ts MN \from T}
    \and
    \inferrule*{\Gamma,x\from S \ts M \from T}{\Gamma \ts \lambda x^S.M \from S \to T}
    \and
    \inferrule*{ }{\Gamma\ts\true\from\bool}
    \and
    \inferrule*{ }{\Gamma\ts\false\from\bool}
    \\
    \inferrule*[right={$T\in\{\bool,\nat\}$}]{\Gamma\ts M \from \bool \\ \Gamma \ts N \from T \\ \Gamma \ts P \from T}{\Gamma\ts \If M \Then N \Else P \from T}
    \\
    \inferrule*{ }{\Gamma\ts n\from\nat}
    \and
    \inferrule*{\Gamma\ts M \from \nat}{\Gamma\ts \suc M \from \nat}
    \and
    \inferrule*{\Gamma\ts M \from \nat}{\Gamma\ts \pred M\from \nat}
    \\
    \inferrule*[right={$T\in\{\bool,\nat\}$}]{\Gamma\ts M \from \nat \\ \Gamma\ts N \from T \\ \Gamma \ts P \from T}{\Gamma \ts \IfO M \Then N \Else P \from T}
    \and
    \inferrule*{\Gamma \ts M \from T \to T}{\Gamma \ts \Y_T M \from T}
    \and
    \inferrule*{ }
    {\ask_X \from\bool}
  \end{mathpar}
  \caption[Type theory for a variant of PCF with nondeterminism.]
  {Type theory for a variant of PCF with nondeterminism.
  Note that if the nondeterminism is infinite (i.e., $X=\bN$), then it is common (see \cite{LairdOrdinalGames} for nondeterminism and \cite{ProbabilisicPcf} for the probabilistic case) to modify the conditional construct $\IfO M\Then N \Else P$ so that the value of $M$ is passed into $P$ (so that the typing rule for $\IfO$ becomes
  \[
    \inferrule*[right={$T\in\{\bool,\nat\}$}]{\Gamma\ts M \from \nat \\ \Gamma \ts N \from T \\ \Gamma \ts P \from \nat \to T}
    {\Gamma \ts \IfO M \Then N \Else P \from T}\,.
    \]
  The idea behind this is that if $M$ evaluates to $0$ then $N$ will be evaluated, and if it evaluates to $n+1$ then $P\,n$ will be evaluated.
  In ordinary deterministic PCF, this does not change the expressive power of the language, since we could always write
  \[
    \IfO M \Then N \Else (P\,M)\,.
    \]
  However, this does not work if $M$ may exhibit nondeterministic behaviour, since it may evaluate to two different values the two times it is called.
  We will stick with the variant shown above for the sake of simplicity, but nothing changes in our proofs if we choose the other version.}
  \label{FigNDPcfTypeTheory}
\end{figure}
We endow this language with an operational semantics of may testing.  
Define a \emph{canonical form} of the language to be a term taking one of the following forms.
\begin{itemize}
  \item At type $\bool$, the constants $\true$ and $\false$;
  \item at type $\nat$, the numerals $n$; and
  \item at type $S \to T$, terms of the form $\lambda x^S.M$.
\end{itemize}
We then define a relation $M \converges c$ (read `$M$ \emph{may converge to} $c$'), for $M$ a term of Nondeterministic PCF and $c$ a canonical form, inductively as in Figure \ref{FigNDPCFOpSem}.
\begin{figure}
  \begin{mathpar}
    \inferrule*{ }{c\converges c}
    \and
    \inferrule*{M\converges\lambda x.M' \\ M'[N/x] \converges c}{M\,N\converges c}
    \\
    \inferrule*{M\converges \true \\ N \converges c}{\If M \Then N \Else P \converges c}
    \and
    \inferrule*{M\converges \false \\ P \converges c}{\If M \Then N \Else P \converges c}
    \\
    \inferrule*{M \converges n}{\suc M \converges n+1}
    \and
    \inferrule*{M \converges n+1}{\pred M \converges n}
    \and
    \inferrule*{M \converges 0}{\pred M \converges 0}
    \\
    \inferrule*{M \converges 0 \\ N \converges c}{\IfO M \Then N \Else P \converges c}
    \and
    \inferrule*{M \converges n+1 \\ P \converges c}{\IfO M \Then N \Else P \converges c}
    \\
    \inferrule*{M(\Y M) \converges c}{\Y M \converges c}
    \and
    \inferrule*{ }{\ask_X \converges \true}
    \and
    \inferrule*{ }{\ask_X \converges \false}
  \end{mathpar}
  \caption{Big-step operational semantics for Nondeterministic PCF and may testing}
  \label{FigNDPCFOpSem}
\end{figure}

It is a quick check to see that every term of Nondeterministic PCF may be regarded as a term of IA${}_X$, giving us a denotational semantics of the language within $\Kl_{R_X}\G$, where $\G$ is the category of arenas and single-threaded strategies.
But we can be more precise than this: we know from \cite{hoPcf} that the denotation of any term of ordinary PCF is an innocent strategy, so that the denotational semantics of PCF within $\G$ factors through the inclusion $\G_{inn}\hookrightarrow\G$, where $\G_{inn}$ is the category of arenas and innocent strategies.
In particular, the denotation within $\Kl_{R_X}\G$ of any term of ordinary PCF lives within $\Kl_{R_X}\G_{inn}$.
Moreover, the denotation of the new term $\ask_X$, being given by the identity morphism in $\G$, also lives in $\Kl_{R_X}\G_{inn}$, meaning that the whole of Nondeterministic PCF is interpreted within this category.

It is also quick to check that our operational semantics in Figure \ref{FigNDPCFOpSem} may be regarded as a subset of the rules for IA${}_X$ with May Testing, as we studied in \sec \ref{SecMayTesting}.
Specifically, if $M$ is a term of PCF and $c$ a canonical form, then $M\converges c$ if and only if
\[
  \blank,()\ts M\converges c,()
  \]
in IA${}_X$.

We want to use our Computational Adequacy result for IA${}_X$ with May Testing (Corollary \ref{CorMayAdequacy}) to get a Computational Adequacy result for Nondeterministic PCF.
The only slight problem is that Corollary \ref{CorMayAdequacy} is stated for terms of type $\com$, whereas our language has no such type.  
We get around this by using $\nat$ as our main ground type rather than $\com$, and by using the following easy lemma.

\begin{lemma}
  Let $M\from\nat$ be a term of IA${}_X$.  
  Then there exists $n$ such that $\Gamma,s\ts M\converges n,s'$ if and only if
  \[
    \Gamma,s\ts \IfO M \Then \skipp \Else \skipp \converges \skipp, s'\,.
    \]
\end{lemma}

We will write $\delta\from \bN \to \bC$ for the denotation of the term-in-context 
\[
  x\from\nat \ts \IfO x \Then \skipp \Else \skipp \from \com\,.
  \]
Then we immediately get the following result as a special case of Corollary \ref{CorMayAdequacy}.

\begin{corollary}[Computational Adequacy for Nondeterministic PCF]
  Let $M\from\nat$ be a closed term of Nondeterministic PCF.
  Consider the denotation $\deno M \from 1 \to \bN$ in $\Kl_{R_X}\G_{inn}$ as a morphism $1 \to (X \to \bN)$ in $\G_{inn}$, and thence as a morphism $1 \to (X \to \bN)$ in $\G$.

  Then there exists some sequence $u\in X^*$ such that the composite
  \[
    1 \xrightarrow{\deno M}
    (X \to \bN) \xrightarrow{X \to \delta}
    (X \to \bC) \xrightarrow{\eta_u}
    (\Varr \to \bN) \xrightarrow{\deno{\neww}}
    \bN \xrightarrow{t_{|u|}}
    \bC
    \]
  is not equal to $\bot$, if and only if $M\converges n$ for some $n$.
  \label{CorMayAdequacyPCF}
\end{corollary}

Note that Corollary \ref{CorMayAdequacyPCF}, which does not depend on any specific details of the models $\G_{inn}$ and $\G$ beyond the fact that they are computationally adequate for ordinary PCF and Idealized Algol, uses morphisms $\delta$, $\eta_u$, $\deno{\neww}$ and $\t_{|u|}$ that do not occur in the category $\G_{inn}$.

In the case that $\G$ is the category of arenas and single-threaded strategies, and $\G_{inn}$ the category of arenas and innocent strategies, we can restate Corollary \ref{CorMayAdequacyPCF} in a way that does not refer to Idealized Algol terms at all.

\begin{corollary}
  Let $M\from\nat$ be a closed term of Nondeterministic PCF and consider the denotation $\deno M \from 1 \to \bN$ in $\Kl_{R_X}\G_{inn}$ as a morphism $1 \to (X \to \bN)$ in $\G_{inn}$.  

  Then, for all $n\in\bN$, there exists some sequence $s\in\deno M$ such that $s\vert_{\bN} = qn$ if and only if $M\converges n$.
  \label{CorMayAdequacyPcfComb}
\end{corollary}
\begin{proof}
  Corollary \ref{CorMayAdequacyPCF}, together with our earlier analysis, immediately tells us that we have
  \[
    \exists s\in \deno M \esuchthat s\vert_\bN = qn\text{ for some $n$} \Leftrightarrow \exists n\esuchthat M\converges n\,.
    \]
  The problem is that the two $n$'s might be different.  
  However, we can obtain the desired result for a given $n$ from this one by composing with an appropriate term $\nat \to \nat$ that converges if and only if its input is equal to $n$.
\end{proof}

The passage to full abstraction, via innocent definability for PCF is essentially the same as for Idealized Algol (though if $X=\bN$, then we need to cut down to a category of \emph{recursive} strategies).

If we so desire, we can continue with the programme from \sec\ref{SecMayTesting}, declaring two Kleisli strategies $A \to (X \to B)$ to be equivalent if they contain the same plays after restriction to $A \to B$.
We can then do away with the plays in $X$ altogether and give a model that involves only nondeterministic strategies.

At this point, we run into a well-known problem: under what circumstances should a nondeterministic strategy be described as \emph{innocent}? 
The answer that we get from our model is that it is innocent if and only if it takes the form
\[
  A \xrightarrow{\sigma} (X \to B) \xrightarrow{\deno{\ask_X} \to B} B\,,
  \]
where $\sigma$ is an innocent deterministic strategy.  
This is the definition considered -- and ultimately rejected for being too indirect -- by Harmer in his thesis \cite[\sec 3.7]{RusssThesis}.  
This definition is correct, but, perhaps surprisingly, does not coincide with the same definition that we would get by naively applying the usual definition of innocence to nondeterministic strategies.

Consider, for example, the denotation of the term 
\[
  \If \ask_\bB \Then (\lambda f.f\true) \Else (\lambda f.f\false)\,,
  \]
which has maximal plays taking one of the following two forms.
\begin{mathpar}
  \begin{tikzcd}[row sep=5pt]
    %
      &
        & q \\
    %
      & q \arrow[ur, bend left=10]
        & \\
    |[alias=A]| q \arrow[ur, bend left=10]
      &
        & \\
    |[alias=B]| \true \arrow[u, bend left=45, from=B.west, to=A.west]
      &
        & \\
    \vdots
      &
        & \\
    |[alias=C]| q  \arrow[uuuur, from=C.east, bend right=10]
      & 
        & \\
    |[alias=D]| \true \arrow[u, bend left=45, from=D.west, to=C.west]
      &
        & \\
    %
      & a \arrow[uuuuuu, bend right=10]
        & \\
    %
      &
        & a \arrow[uuuuuuuu, bend right=10]
  \end{tikzcd}
  \and
  \begin{tikzcd}[row sep=5pt]
    %
      &
        & q \\
    %
      & q \arrow[ur, bend left=10]
        & \\
    |[alias=A]| q \arrow[ur, bend left=10]
      &
        & \\
    |[alias=B]| \false \arrow[u, bend left=45, from=B.west, to=A.west]
      &
        & \\
    \vdots
      &
        & \\
    |[alias=C]| q  \arrow[uuuur, from=C.east, bend right=10]
      & 
        & \\
    |[alias=D]| \false \arrow[u, bend left=45, from=D.west, to=C.west]
      &
        & \\
    %
      & a \arrow[uuuuuu, bend right=10]
        & \\
    %
      &
        & a \arrow[uuuuuuuu, bend right=10]
  \end{tikzcd}
\end{mathpar}
Note that this strategy displays typically non-innocent behaviour: if player $P$ has played $\true$ on the left, then she must play $\true$ again whenever player $O$ asks, even though the original move $\true$ occurs outside the current $P$-view.

In the Kleisli category, the innocence of this strategy becomes clear: plays are either of the form
\[
  \begin{tikzcd}[row sep=5pt]
    %
      &
        &
          & q \\
    |[alias=Y]| q  \arrow[urrr, bend left=10] \\
    |[alias=Z]| \true \arrow[u, bend left=45, from=Z.west, to=Y.west] \\
    %
      &
        & q \arrow[uuur, bend left=10]
          & \\
    %
      & |[alias=A]| q \arrow[ur, bend left=10]
        &
          & \\
    %
      & |[alias=B]| \true \arrow[u, bend left=45, from=B.west, to=A.west]
        &
          & \\
    %
      & \vdots
        &
          & \\
    %
      & |[alias=C]| q  \arrow[uuuur, from=C.east, bend right=10]
        & 
          & \\
    %
      & |[alias=D]| \true \arrow[u, bend left=45, from=D.west, to=C.west]
        &
          & \\
    %
      &
        & a \arrow[uuuuuu, bend right=10]
          & \\
    %
      &
        &
          & a \arrow[uuuuuuuuuu, bend right=10]
  \end{tikzcd}
  \]
or of the form
\[
  \begin{tikzcd}[row sep=5pt]
    %
      &
        &
          & q \\
    |[alias=Y]| q  \arrow[urrr, bend left=10] \\
    |[alias=Z]| \false \arrow[u, bend left=45, from=Z.west, to=Y.west] \\
    %
      &
        & q \arrow[uuur, bend left=10]
          & \\
    %
      & |[alias=A]| q \arrow[ur, bend left=10]
        &
          & \\
    %
      & |[alias=B]| \false \arrow[u, bend left=45, from=B.west, to=A.west]
        &
          & \\
    %
      & \vdots
        &
          & \\
    %
      & |[alias=C]| q  \arrow[uuuur, from=C.east, bend right=10]
        & 
          & \\
    %
      & |[alias=D]| \false \arrow[u, bend left=45, from=D.west, to=C.west]
        &
          & \\
    %
      &
        & a \arrow[uuuuuu, bend right=10]
          & \\
    %
      &
        &
          & {a\,,} \arrow[uuuuuuuuuu, bend right=10]
  \end{tikzcd}
  \]
where now the move $\true$ or $\false$ in the nondeterministic oracle game (on the very left) locks the `branching-time information' into the $P$-view.

There is a surprising and elegant definition of nondeterministic innocence (see Levy \cite{levy2014morphisms}, with the correction given by Tsukada and Ong in \cite[Proposition 46]{TsukadaSheaves}), which we shall not detail here.
Tsukada and Ong have also demonstrated that it is instructive to consider strategies as presheaves over plays, where we assign to each individual play a set of possible branches that lead us to that play.  
In this formalism, the innocent nondeterministic strategies are precisely those presheaves that are sheaves with respect to a certain natural Grothendieck topology.

What, then, can our Kleisli category model do for us?
One answer is that it provides a general setting in which we can prove results such as adequacy without worrying about the specific details of a nondeterministic construction.  
For example, we can prove a computational adequacy result for Tsukada and Ong's model $\G_{TO}$ from Corollary \ref{CorMayAdequacyPcfComb} by noting that the denotational semantics in that category factors through the natural functor $\Kl_{R_\bB}\G \to \G_{TO}$.  
Of course, Tsukada and Ong have provided their own proof of Computational Adequacy in the finite nondeterminism case, but this method could be useful if, for example, we wanted to extend their model to deal with countable nondeterminism.

\section{Stateful Effects}

In Chapters \ref{ChapGames} and \ref{ChapFullAbstraction}, we presented a categorical algebra of scoped state via sequoidal categories.  
A natural question to ask is whether we could use the techniques we have developed in the remainder of this thesis in the stateful case.  
Indeed, the state monad
\[
  A \mapsto (W \to (A \times W))
  \]
can be constructed in categories of games.  
More ambitiously, we might hope to capture a parametric notion of local state, via the action
\[
  (R,W).A = R \to (A \times W)\,.
  \]
of $\oppcat \G \times \G$ on $\G$.

In this section, we will present some of the problems involved in this case.  

One issue, which we have touched on already, is that the state monad is not a monoidal functor, which was the condition we needed in order to ensure that its Kleisli category inherited a monoidal structure.  
Indeed, there is not obvious natural morphism
\[
  (W \to (A \times W)) \times (W \to (B \times W)) \to (W \to ((A \times B) \times W)\,.
  \]

The parametric version suffers from a different problem.  
Suppose we have two \Mellies morphisms
\begin{mathpar}
  f \from A \to (R \to (B \times W))
  \and
  g \from B \to (R' \to (C \times W'))\,.
\end{mathpar}
Then their composition will be given by a \Mellies morphism
\[
  f;g \from A \to ((R \times R') \to (C \times (W \times W')))\,:
  \]
in other words, we will have combined the two inputs $R$ and $R'$ and the two outputs $W$ and $W'$, without any way of specifying that the output state $W$ from $f$ should be used as the input state $R'$ from $g$.

The difficulty is that we might want $W$ and $R'$ to represent different variables, in which case this is the right thing to do, but we might want them to represent the same storage cell, in which case, the output from $W$ should be fed into $R'$.  
It's not clear how to deal with this in a category-theoretic way.

Linear type theory provides a perspective on this problem.  
Note that the state monad and the parametric action above can both be defined inside an arbitrary monoidal category.  
However, if we want our language to exhibit any typically stateful behaviour, then we need to be able to refer to the same variable more than once: there is little point writing a value into a cell if we are not able to read from it later.  
This is not a problem in a monolithic stateful system -- such as we have with the state monad -- but if we want to refer to individual storage cells using variables in the language, then we need our category to admit diagonals, to allow us to refer to a storage cell twice.
Since the state action above does not interact in any way with the Cartesian structure of the category (i.e., the diagonals and terminal morphisms), it cannot hope to capture local state in this way.

One avenue that could shed some light on this issue is the sequoidal semantics which we developed for game semantics, in which the stateful behaviour is intimately connected to the Cartesian structure of the category, via the exponential.

There is also a \emph{local state monad}, due to Plotkin and Power \cite{PlotkinPower} and based on a construction by Levy \cite{PaulsThesis}, which is defined on the category $[\Inj,\Set]$ of functors from $\Inj$, the category of natural numbers and injections, into the category of sets.  
It is given by the following formula, where $V$ denotes a fixes set of values.
\[
  (TA)(n) = V^n \to \left( \int^{p\from\Inj} V^p \times A(p) \times \Inj(n, p) \right)
  \]
Here, if $A \from \Inj \to \Set$ is a functor, then we think of $A(n)$ as the set of all values that $A$ takes on in the presence of $n$ local variables taking values in $V$.

We cannot construct this coend in the category of games, but there is a clear link with our constructions with probabilistic monads.  
Note that the `state action'
\[
  (W, A) \mapsto W \to (A \times W)
  \]
has mixed variance in $W$, suggesting a modification of our $\C/\X$ construction that uses a coend rather than an ordinary colimit.

\section{Relational Models}

The archetypal Cartesian closed category is the category $\Set$ of sets and functions.  
Most of the work in this thesis has been to do with reader actions (including reader monads as a special case), and we have already provided a classification of the reader actions on $\Set$ -- up to the equivalence relation that identifies actions of monoidal categories $\X$ and $\YY$ if the resulting categories $\Set/\oppcat\X$ and $\Set/\oppcat\YY$ are isomorphic -- in Propositions \ref{PropColimitOfActionIsMonoidalFunctor} and \ref{PropMonoidalFunctorIsColimitOfReaderAction}: namely, they are classified by lax monoidal functors $F\from\Set \to \Set$, in such a way that if a reader action given by an oplax monoidal functor $j\from\X \to \Set$ corresponds to the functor $F \from \Set \to \Set$, then $\Set/\oppcat\X$ is isomorphic to the category $F_*\Set$ formed from $\Set$ by base change along $F$.
Thus, the theory in the case of $\Set$ reduces to the theory of $\Set$-enriched base change.

There are, however, many more examples in the literature of models of programming languages that ultimately derive from Kleisli categories for monads not of the reader kind.  
The problem with this, from our point of view, is that the resulting Kleisli categories will not be automatically Cartesian.  
This means that we need different methods to ensure that we end up with a category suitable for modelling a compositional semantics.

For example, the Kleisli category for the powerset monad on $\Set$ is the category $\Rel$ of sets and relations.  
This is a monoidal closed category, so can be used to model multiplicative linear type theory.
By using a linear exponential comonad based on finite multisets, we can transform it into a Cartesian closed category.

An important category derived from the category of sets and relations is the category $\Coh$ of coherence spaces.  
A coherence space is a set with some additional structure on it -- namely, the structure of an undirected graph -- and a morphism between coherence spaces is a relation between sets that respects this structure in a particular way (see \cite{MelliesCoherence} for details).  

Danos and Ehrhard in \cite{ProbabilisticCoherenceSpaces} develop a probabilistic version of coherence spaces, which is still monoidal closed.  
By using the finite multiset comonad, we can transform it into a Cartesian closed category.  
A striking result of Ehrhard, Pagani and Tasson \cite{ProbabilisicPcf} is that this category is already fully abstract for a probabilistic variant of PCF.

This suggests an alternative way to get round the Cartesianness hurdle when adding effects: start with a model of the base language that is constructed by applying a linear exponential comonad to a monoidal closed category, and then delay application of the linear exponential comonad until after we have added the effect.

Unfortunately, there is no real guarantee that a linear exponential comonad on the base category will give us a linear exponential comonad on a Kleisli category or one of the other effectful categories we have considered.  
Some work is required to investigate the conditions in which we can lift an exponential construction in this way, given the functorial results from Chapter \ref{ChapParametricMonads}.  
This will pave the way for us to understand the relational and coherence-space models using the ideas developed in this thesis.

\section{Adequacy Proofs}

In Chapters \ref{ChapMonads} and \ref{ChapParametricMonadsFullAbstraction} we proved Computational Adequacy via a factorization and operational techniques.  
This is good, because it allows us to assume very little about the underlying model, other than that it is itself computationally adequate.  
Nevertheless, our techniques rely very heavily on the fact that our base language is Idealized Algol.  
Although we outlined ways to get around this in Section \ref{SecStatelessLanguages}, it would also be useful to have a conventional logical relations-based proof of Computational Adequacy.

Assume that $\G$ is a Cartesian closed category enriched in directed-complete partial orders (dcpos).
If $M\from\G \to \G$ is a monoidal monad, then the Kleisli category associated to $M$ inherits an order, whereby $\sigma\le\tau\from A \to B$ in $\Kl_M\G$ if $\sigma\le\tau\from A \to MB$ in $\G$.
Moreover, since composition in the Kleisli category is given by a composition in the base category, this partial order will be respected by composition, as will be limits of directed sets.

Suppose again that $\G$ is a dcpo-enriched category, and that a monoidal category $\X$ acts on $\G$ via a lax action.  
Then the construction of \Mellies yields an $[\X,\Dcpo]$-enriched category $\Mell_\X\G$.
Since $\Dcpo$ is cocomplete \cite{CocompleteDcpo}, we can construct the coends we need to define the Day convolution product in $[\X,\Dcpo]$.

If $\X$ is small, or if the required colimits exist for other reasons, then we may take the change of base with respect to the colimit functor $[\X,\Dcpo] \to \Dcpo$ to $\Mell_\X\G$ to give us a $\Dcpo$-enriched category $\G/\X$.

Concretely, the underling set of morphisms in $\G/\X$ contains all the usual morphisms from the $\Set$-enriched case, while the order structure is given by a transfinite construction as in \cite[2.17]{Fiech}.  
In particular, if $\sigma,\tau\from A \to B$ in $\G/\X$ are given by morphisms
\begin{mathpar}
  \tilde{\sigma}\from A \to X.B
  \and
  \tilde{\tau} \from A \to Y.B
\end{mathpar}
in $\G$, and if there are morphisms $h\from X \to Z$, $k\from Y \to Z$ in $\X$ such that
\[
  \sigma;(h.B)\le\tau;(k.B)\,,
  \]
then $\sigma\le\tau$ in $\G/\X$.

A slight problem is that it is not true in general that the colimit of cpos with bottom elements has itself a bottom element.  
For example, let $-[n]$ be the partial order $\{-n<\cdots<0\}$.  
Then, $-[n]$ is a dcpo with a bottom element for each $n$, but the colimit of the chain
\[
  -[0] \hookrightarrow -[1] \hookrightarrow -[2] \hookrightarrow \cdots
  \]
is the partial order of non-positive integers.

However, if all the morphisms in a diagram of dcpos are such that they preserve bottom elements, then the colimit of that diagram will have a bottom element.

In particular, if the action of $\X$ on $\G$ is a reader-style action that factors through the category of sets (as in all our examples), then every arrow in the colimit diagram that defines $\G/\X(A,B)$ will be of the form
\[
  \G(A, \ul Y \to B) \xrightarrow{\G(A,\ul f.B)} \G(A, \ul X \to B)
  \]
for some sets $X,Y$, considered as objects $\ul X,\ul Y$ of $\G$, and some function $f \from X \to B$ considered as a morphism $\ul f \from \ul A \to \ul B$ such that composition with $\ul f$ preserves bottom elements.
Therefore, the map $\G(A,\ul f.B)$ of dcpos will also preserve bottom elements.

A second problem is that the setwise colimit of the $\G(A,X.B)$, with the order from \cite{Fiech}, is not necessarily itself a dcpo.  
In fact, to get the correct colimit of the diagram, we need to perform an extra step of \emph{completion}.  
However, with a little more care, we can show that if the action is a reader action that factors through the category of sets, then the colimit is already directed complete under the order, at least in the category of games.

Having given the definition of the order-enriched structure of $\G$, we may define the interpretation of the fixpoint combinator $\Y$ using the properties of this new order, which will be enough to prove adequacy along the lines of Theorem \ref{TheComputationalAdequacyIA}.

It is worth mentioning that although we can prove computational adequacy for our various Kleisli categories (and categories of the form $\G/\X$), it does not automatically follow that we can prove adequacy in the same way for the various quotiented categories.  
An important example is our model of countable nondeterminism with must testing.  
It is well known (see, e.g., \cite{Apt,PlotkinApt}) that countable nondeterminism is inextricably linked to discontinuity of composition, which robs us of an important ingredient in the standard order-theoretic proof of adequacy.

The solution is to prove adequacy in the normal way for the semantics of IA${}_\bN$ in $\G_\bN$, and then to apply the argument from Corollary \ref{CorAdeqMust} to reduce this to a Computational Adequacy result for the semantics of Idealized Algol with countable nondeterminism and must testing.
Here, we are using an idea due to Levy \cite{LevyGsInfinite}: that if we want to prove adequacy for countable nondeterminism, then we need a form of \emph{explicit forcing}, in which we uncover some information about the points at which we have made an infinite nondeterministic branch.  
In our case, the explicit forcing is done through the language IA${}_\bN$, which creates a complete record of the nondeterministic values we have chosen along the way.  
We have then applied the second part of Levy's technique -- \emph{hiding} -- to remove the explicit information and yield a pure, computationally adequate model of countable nondeterminism.


\Urlmuskip=0mu plus 9mu\relax
\bibliographystyle{alpha2}
\bibliography{../common/phd_bibliography}

\end{document}
