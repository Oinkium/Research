\chapter{Further Directions}
\label{ChapFurtherDirections}

We conclude by examining a few avenues that we are left unexplored in this thesis.

\section{Stateless Languages}
\label{SecStatelessLanguages}

So far, the base language we have used has been the stateful language Idealized Algol.  
One further question to ask is whether the techniques we have developed allow us to build models of effectful versions of stateless languages such as PCF.

One immediate problem we face is that the addition of nondeterministic effects to PCF allows us to distinguish terms that cannot be distinguished within PCF itself.  
For instance, the PCF term
\[
  M = \lambda x^{\bool}.\If x \Then \Omega \Else (\If x \Then \true \Else \Omega)
  \]
is observationally equivalent (in PCF) to $\Omega$ -- informally, because the term $x$ must have the same `value' both time it is called, so if it is false the first time, then it must be false the second time.
However, the same term inside PCF with finite nondeterminism is not observationally equivalent to $\Omega$: if we substitute a nondeterministic oracle in for $x$, then the nondeterministic oracle could return false the first time and true the second, causing the term to converge to $\true$.
Note that this problem does not arise in Idealized Algol: nondeterminism does not allow us to distinguish terms that were indistinguishable in the deterministic language, as we proved in Section \ref{SecIntrinsicEquivalenceRelationKleisli}.

This means that if $\G$ is a `truly Fully Abstract' model of PCF -- i.e., one in which observational equivalence of terms corresponds to equality of morphisms rather than intrinsic equivalence\footnote{As an example, take any model that is full abstract in our sense, and take the quotient by the intrinsic equivalence relation.} -- then there can be no natural inclusion functor from $\G$ into a model of PCF with finite nondeterminism.  
Indeed, in such a model $\deno{M}$ and $\deno{\Omega}$ are the same morphism, so they would have to be sent to the same morphism in the model of nondeterministic PCF.

This is a problem for us, since it means that our model of PCF cannot be the Kleisli category for a monad on $\G$ (or a category of the form $\G/\X$), since then we \emph{do} have a functor coming out of $\G$.

A way to get around this program is to insist that our category $\G$ is not `truly Fully Abstract', but is still fine-grained enough to distinguish between terms that will eventually be distinguished when we add in the nondeterminism.  
A convenient requirement to make is that $\G$ should be embeddable into some model $\G^{IA}$ of Idealized Algol (as is the case for Hyland and Ong's game semantics).  
The idea is that since $\G^{IA}$ must necessarily distinguish between observationally inequivalent terms of nondeterministic PCF, then $\G$ must distinguish between them too.

The other advantage of going via an Idealized Algol model is that we can automatically apply many of the results that we have already obtained.

Let us try this out with an example.  
Fix $X\in\{\bB,\bN\}$ and let \emph{Nondeterministic PCF} be the language PCF, together with a constant $\ask_X \from \bool$ that plays the role of a nondeterministic oracle.  
It is the language with types given by
\[
  T \Coloneqq \bool \mid \nat \mid T \to T\,,
  \]
and with a type theory as shown in Figure \ref{FigNDPcfTypeTheory}.
\begin{figure}
  \begin{mathpar}
    \inferrule*{ }{\Gamma,x\from T \ts x \from T}
    \and
    \inferrule*{\Gamma\ts M \from S \to T \\ \Gamma \ts N \from S}{\Gamma\ts MN \from T}
    \and
    \inferrule*{\Gamma,x\from S \ts M \from T}{\Gamma \ts \lambda x^S.M \from S \to T}
    \and
    \inferrule*{ }{\Gamma\ts\true\from\bool}
    \and
    \inferrule*{ }{\Gamma\ts\false\from\bool}
    \\
    \inferrule*[right={$T\in\{\bool,\nat\}$}]{\Gamma\ts M \from \bool \\ \Gamma \ts N \from T \\ \Gamma \ts P \from T}{\Gamma\ts \If M \Then N \Else P \from T}
    \\
    \inferrule*{ }{\Gamma\ts n\from\nat}
    \and
    \inferrule*{\Gamma\ts M \from \nat}{\Gamma\ts \suc M \from \nat}
    \and
    \inferrule*{\Gamma\ts M \from \nat}{\Gamma\ts \pred M\from \nat}
    \\
    \inferrule*[right={$T\in\{\bool,\nat\}$}]{\Gamma\ts M \from \nat \\ \Gamma\ts N \from T \\ \Gamma \ts P \from T}{\Gamma \ts \IfO M \Then N \Else P \from T}
    \and
    \inferrule*[right={$S,T\in\{\bool,\nat\}$}]{\Gamma\ts M \from S \\ \Gamma,x\from S\ts N \from T}{\Gamma\ts \lett x=M\inn N\from T}
    \and
    \inferrule*{\Gamma \ts M \from T \to T}{\Gamma \ts \Y_T M \from T}
    \and
    \inferrule*{ }
    {\ask_X \from\bool}
  \end{mathpar}
  \caption[Type theory for a variant of PCF with nondeterminism.]
  {Type theory for a variant of PCF with nondeterminism.
  Note that although ordinary deterministic PCF does not include the $\lett$ construct, it has to be included in the nondeterministic (see \cite{LairdOrdinalGames}) and probabilistic (see \cite{ProbabilisicPcf}) versions.  

  In ordinary PCF, we may simulate the term $\lett x=M \inn N$ by the term $(\lambda x.N) M$, which is less efficient than the version with $\lett$ (If $N$ contains multiple occurrences of the variable $x$, then we end up compute $M$ multiple times), but which still gives the same result.

  In nondeterministic variants of PCF, however, evaluating $M$ multiple times may give a different answer each time, in contrast to the version with $\lett$, in which we get a single answer and use it multiple times.}
  \label{FigNDPcfTypeTheory}
\end{figure}
We endow this language with an operational semantics of may testing.  
Define a \emph{canonical form} of the language to be a term taking one of the following forms.
\begin{itemize}
  \item At type $\bool$, the constants $\true$ and $\false$;
  \item at type $\nat$, the numerals $n$; and
  \item at type $S \to T$, terms of the form $\lambda x^S.M$.
\end{itemize}
We then define a relation $M \converges c$ (read `$M$ \emph{may converge to} $c$'), for $M$ a term of Nondeterministic PCF and $c$ a canonical form, inductively as in Figure \ref{FigNDPCFOpSem}.
This rule is the restriction of the $\converges$ rule from Nondeterministic Idealized Algol to the set of those terms that are terms of Nondeterministic PCF.
\begin{figure}
  \begin{mathpar}
    \inferrule*{ }{c\converges c}
    \and
    \inferrule*{M\converges\lambda x.M' \\ M'[N/x] \converges c}{M\,N\converges c}
    \\
    \inferrule*{M\converges \true \\ N \converges c}{\If M \Then N \Else P \converges c}
    \and
    \inferrule*{M\converges \false \\ P \converges c}{\If M \Then N \Else P \converges c}
    \\
    \inferrule*{M \converges n}{\suc M \converges n+1}
    \and
    \inferrule*{M \converges n+1}{\pred M \converges n}
    \and
    \inferrule*{M \converges 0}{\pred M \converges 0}
    \\
    \inferrule*{M \converges 0 \\ N \converges c}{\IfO M \Then N \Else P \converges c}
    \and
    \inferrule*{M \converges n+1 \\ P \converges c}{\IfO M \Then N \Else P \converges c}
    \\
    \inferrule*{M(\Y M) \converges c}{\Y M \converges c}
    \and
    \inferrule*{ }{\ask_X \converges \true}
    \and
    \inferrule*{ }{\ask_X \converges \false}
  \end{mathpar}
  \caption{Big-step operational semantics for Nondeterministic PCF and may testing}
  \label{FigNDPCFOpSem}
\end{figure}

It is a quick check to see that every term of Nondeterministic PCF may be regarded as a term of IA${}_X$, giving us a denotational semantics of the Nondeterministic PCF within $\Kl_{R_X}\G$, where $\G$ is the category of arenas and single-threaded strategies.
But we can be more precise than this: we know from \cite{hoPcf} that the denotation of any term of ordinary PCF is an innocent strategy, so that the denotational semantics of PCF within $\G$ factors through the inclusion $\G_{inn}\hookrightarrow\G$, where $\G_{inn}$ is the category of arenas and innocent strategies.
In particular, the denotation within $\Kl_{R_X}\G$ of any term of ordinary PCF lives within $\Kl_{R_X}\G_{inn}$.
Moreover, the denotation of the new term $\ask_X$, being given by the identity morphism in $\G$, also lives in the category
\[
  \G_{inn,X} \coloneqq \Kl_{R_X}\G_{inn}\,,
  \]
giving us a denotational semantics of Nondeterministic PCF within $\G_{inn,X}$.

It is also quick to check that our operational semantics in Figure \ref{FigNDPCFOpSem} may be regarded as a subset of the rules for IA${}_X$ with May Testing, as we studied in \sec \ref{SecMayTesting}.
Specifically, if $M$ is a term of PCF and $c$ a canonical form, then $M\converges c$ if and only if
\[
  \blank,()\ts M\converges c,()
  \]
in IA${}_X$.

We want to use our Computational Adequacy result for IA${}_X$ with May Testing (Corollary \ref{CorMayAdequacy}) to get a Computational Adequacy result for Nondeterministic PCF.
The only slight problem is that Corollary \ref{CorMayAdequacy} is stated for terms of type $\com$, whereas our language has no such type.  
We get around this by using $\nat$ as our main ground type rather than $\com$, and by using the following easy lemma.

\begin{lemma}
  Let $M\from\nat$ be a term of IA${}_X$.  
  Then there exists $n$ such that $\Gamma,s\ts M\converges n,s'$ if and only if
  \[
    \Gamma,s\ts \IfO M \Then \skipp \Else \skipp \converges \skipp, s'\,.
    \]
\end{lemma}

We will write $\delta\from \bN \to \bC$ for the denotation of the term-in-context 
\[
  x\from\nat \ts \IfO x \Then \skipp \Else \skipp \from \com\,.
  \]
Then we immediately get the following result as a special case of Corollary \ref{CorMayAdequacy}.

\begin{corollary}[Computational Adequacy for Nondeterministic PCF]
  Let $M\from\nat$ be a closed term of Nondeterministic PCF.
  Consider the denotation $\deno M \from 1 \to \bN$ in $\Kl_{R_X}\G_{inn}$ as a morphism $1 \to (X \to \bN)$ in $\G_{inn}$, and thence as a morphism $1 \to (X \to \bN)$ in $\G$.

  Then there exists some sequence $u\in X^*$ such that the composite
  \[
    1 \xrightarrow{\deno M}
    (X \to \bN) \xrightarrow{X \to \delta}
    (X \to \bC) \xrightarrow{\eta_u}
    (\Varr \to \bN) \xrightarrow{\deno{\neww}}
    \bN \xrightarrow{t_{|u|}}
    \bC
    \]
  is not equal to $\bot$, if and only if $M\converges n$ for some $n$.
  \label{CorMayAdequacyPCF}
\end{corollary}

Note that Corollary \ref{CorMayAdequacyPCF}, which does not depend on any specific details of the models $\G_{inn}$ and $\G$ beyond the fact that they are computationally adequate for ordinary PCF and Idealized Algol, uses morphisms $\delta$, $\eta_u$, $\deno{\neww}$ and $\t_{|u|}$ that do not occur in the category $\G_{inn}$.

In the case that $\G$ is the category of arenas and single-threaded strategies, and $\G_{inn}$ the category of arenas and innocent strategies, we can restate Corollary \ref{CorMayAdequacyPCF} in a way that does not refer to Idealized Algol terms at all.

\begin{corollary}
  Let $M\from\nat$ be a closed term of Nondeterministic PCF and consider the denotation $\deno M \from 1 \to \bN$ in $\Kl_{R_X}\G_{inn}$ as a morphism $1 \to (X \to \bN)$ in $\G_{inn}$.  

  Then, for all $n\in\bN$, there exists some sequence $s\in\deno M$ such that $s\vert_{\bN} = qn$ if and only if $M\converges n$.
  \label{CorMayAdequacyPcfComb}
\end{corollary}
\begin{proof}
  Corollary \ref{CorMayAdequacyPCF}, together with our earlier analysis, immediately tells us that we have
  \[
    \exists s\in \deno M \esuchthat s\vert_\bN = qn\text{ for some $n$} \Leftrightarrow \exists n\esuchthat M\converges n\,.
    \]
  The problem is that the two $n$'s might be different.  
  However, we can obtain the desired result for a given $n$ from this one by composing with an appropriate term $\nat \to \nat$ that converges if and only if its input is equal to $n$.
\end{proof}

The passage to full abstraction, via innocent definability for PCF is essentially the same as for Idealized Algol (though, as before, if $X=\bN$, then we need to cut down to a category of \emph{recursive} strategies).

If we so desire, we can continue with the programme from \sec\ref{SecMayTesting}, declaring two Kleisli strategies $A \to (X \to B)$ to be equivalent if they contain the same plays after restriction to $A \to B$.
We can then do away with the plays in $X$ altogether and give a model that involves only nondeterministic strategies.

At this point, we run into a well-known problem: under what circumstances should a nondeterministic strategy be described as \emph{innocent}? 
The answer that we get from our model is that it is innocent if and only if it takes the form
\[
  A \xrightarrow{\sigma} (X \to B) \xrightarrow{\deno{\ask_X} \to B} B\,,
  \]
where $\sigma$ is an innocent deterministic strategy.  
This is the definition considered -- and ultimately rejected for being too indirect -- by Harmer in his thesis \cite[\sec 3.7]{RusssThesis}.  
This definition is correct, but, perhaps surprisingly, does not coincide with what we would get by naively applying the usual definition of innocence to nondeterministic strategies.

Consider, for example, the denotation of the term 
\[
  \If \ask_\bB \Then (\lambda f.f\true) \Else (\lambda f.f\false)\,,
  \]
which has maximal plays taking one of the following two forms.
\begin{mathpar}
  \begin{tikzcd}[row sep=5pt]
    %
      &
        & q \\
    %
      & q \arrow[ur, bend left=10]
        & \\
    |[alias=A]| q \arrow[ur, bend left=10]
      &
        & \\
    |[alias=B]| \true \arrow[u, bend left=45, from=B.west, to=A.west]
      &
        & \\
    \vdots
      &
        & \\
    |[alias=C]| q  \arrow[uuuur, from=C.east, bend right=10]
      & 
        & \\
    |[alias=D]| \true \arrow[u, bend left=45, from=D.west, to=C.west]
      &
        & \\
    %
      & a \arrow[uuuuuu, bend right=10]
        & \\
    %
      &
        & a \arrow[uuuuuuuu, bend right=10]
  \end{tikzcd}
  \and
  \begin{tikzcd}[row sep=5pt]
    %
      &
        & q \\
    %
      & q \arrow[ur, bend left=10]
        & \\
    |[alias=A]| q \arrow[ur, bend left=10]
      &
        & \\
    |[alias=B]| \false \arrow[u, bend left=45, from=B.west, to=A.west]
      &
        & \\
    \vdots
      &
        & \\
    |[alias=C]| q  \arrow[uuuur, from=C.east, bend right=10]
      & 
        & \\
    |[alias=D]| \false \arrow[u, bend left=45, from=D.west, to=C.west]
      &
        & \\
    %
      & a \arrow[uuuuuu, bend right=10]
        & \\
    %
      &
        & a \arrow[uuuuuuuu, bend right=10]
  \end{tikzcd}
\end{mathpar}
Note that this strategy displays typically non-innocent behaviour: if player $P$ has played $\true$ on the left, then she must play $\true$ again whenever player $O$ asks, even though the original move $\true$ occurs outside the current $P$-view.

In the Kleisli category model, the innocence of this strategy becomes clear: plays are either of the form
\[
  \begin{tikzcd}[row sep=5pt]
    %
      &
        &
          & q \\
    |[alias=Y]| q  \arrow[urrr, bend left=10] \\
    |[alias=Z]| \true \arrow[u, bend left=45, from=Z.west, to=Y.west] \\
    %
      &
        & q \arrow[uuur, bend left=10]
          & \\
    %
      & |[alias=A]| q \arrow[ur, bend left=10]
        &
          & \\
    %
      & |[alias=B]| \true \arrow[u, bend left=45, from=B.west, to=A.west]
        &
          & \\
    %
      & \vdots
        &
          & \\
    %
      & |[alias=C]| q  \arrow[uuuur, from=C.east, bend right=10]
        & 
          & \\
    %
      & |[alias=D]| \true \arrow[u, bend left=45, from=D.west, to=C.west]
        &
          & \\
    %
      &
        & a \arrow[uuuuuu, bend right=10]
          & \\
    %
      &
        &
          & a \arrow[uuuuuuuuuu, bend right=10]
  \end{tikzcd}
  \]
or of the form
\[
  \begin{tikzcd}[row sep=5pt]
    %
      &
        &
          & q \\
    |[alias=Y]| q  \arrow[urrr, bend left=10] \\
    |[alias=Z]| \false \arrow[u, bend left=45, from=Z.west, to=Y.west] \\
    %
      &
        & q \arrow[uuur, bend left=10]
          & \\
    %
      & |[alias=A]| q \arrow[ur, bend left=10]
        &
          & \\
    %
      & |[alias=B]| \false \arrow[u, bend left=45, from=B.west, to=A.west]
        &
          & \\
    %
      & \vdots
        &
          & \\
    %
      & |[alias=C]| q  \arrow[uuuur, from=C.east, bend right=10]
        & 
          & \\
    %
      & |[alias=D]| \false \arrow[u, bend left=45, from=D.west, to=C.west]
        &
          & \\
    %
      &
        & a \arrow[uuuuuu, bend right=10]
          & \\
    %
      &
        &
          & {a\,,} \arrow[uuuuuuuuuu, bend right=10]
  \end{tikzcd}
  \]
where now the move $\true$ or $\false$ in the nondeterministic oracle game (on the very left) locks the `branching-time information' into the $P$-view.

Moving away from our Kleisli-category model, there is a surprising and elegant definition of nondeterministic innocence within the usual category of games (see Levy \cite{levy2014morphisms}, with the correction given by Tsukada and Ong in \cite[Proposition 46]{TsukadaSheaves}), which we shall not detail here.
Tsukada and Ong have also demonstrated that it is instructive to consider strategies as presheaves over plays, where we assign to each individual play a set of possible branches that lead us to that play.  
In this formalism, the innocent nondeterministic strategies are precisely those presheaves that are sheaves with respect to a certain natural Grothendieck topology.

Even given this theory, however, our Kleisli category model is still worthwhile, since it provides a general setting in which we can prove results such as adequacy without worrying about the specific details of a nondeterministic construction.  
For example, we can prove a computational adequacy result for Tsukada and Ong's model $\G_{TO}$ from Corollary \ref{CorMayAdequacyPcfComb} by noting that the denotational semantics in that category factors through the natural functor $\Kl_{R_\bB}\G \to \G_{TO}$.  
Tsukada and Ong have provided their own proof of Computational Adequacy in the finite nondeterminism case, but this method could be useful if, for example, we wanted to extend their model to deal with countable nondeterminism.

\section{Stateful Effects}

In Chapters \ref{ChapGames} and \ref{ChapFullAbstraction}, we presented a categorical algebra of scoped state via sequoidal categories.  
The category theory that we used in those chapters is fairly non-standard, and a natural question to ask is whether we can come up with a treatment of state from the point of view of monads or lax actions.

Indeed, there is a well known \emph{state monad}
\[
  A \mapsto (W \to (A \times W))\,,
  \]
which can be constructed in categories of games.  
More ambitiously, we might hope to capture a parametric notion of local state, via the action
\[
  (R,W).A = R \to (A \times W)\,.
  \]
of $\oppcat \G \times \G$ on $\G$.

In this section, we will present some of the problems involved in this case.  

One issue, which we have touched on already, is that the state monad is not a monoidal functor, which was the condition we needed in order to ensure that its Kleisli category inherited a monoidal structure.  
Indeed, there is no obvious natural morphism
\[
  (W \to (A \times W)) \times (W \to (B \times W)) \to (W \to ((A \times B) \times W))\,.
  \]

The parametric version suffers from a different problem.  
Suppose we have two \Mellies morphisms
\begin{mathpar}
  f \from A \to (R \to (B \times W))
  \and
  g \from B \to (R' \to (C \times W'))\,.
\end{mathpar}
Then their composition will be given by a \Mellies morphism
\[
  f;g \from A \to ((R \times R') \to (C \times (W \times W')))\,:
  \]
in other words, we will have combined the two inputs $R$ and $R'$ and the two outputs $W$ and $W'$.

This works well up to a point, but the main point of state is that we want to be able to write to a variable once and then read from it later.  
So we'd need some way of wiring up the output on the right to the input on the left, and it's not clear how to deal with this in a category-theoretic way.

Linear type theory provides a perspective on this problem.  
Note that the state monad and the parametric action above can both be defined inside an arbitrary monoidal category.  
However, if we want our language to exhibit any typically stateful behaviour, then we need to be able to refer to the same variable more than once: there is little point writing a value into a cell if we are not able to read from it later.  
This is not a problem in a monolithic stateful system -- such as we have with the state monad -- but if we want to refer to individual storage cells using variables in the language, then we need our category to admit diagonals, to allow us to refer to a storage cell twice.
Since the state action above does not interact in any way with the Cartesian structure of the category (i.e., the diagonals and terminal morphisms), it cannot hope to capture local state in this way.

One avenue that could shed some light on this issue is the sequoidal semantics which we developed for game semantics, in which the stateful behaviour is intimately connected to the Cartesian structure of the category, via the exponential.

There is also a \emph{local state monad}, due to Plotkin and Power \cite{PlotkinPower} and based on a construction by Levy \cite{PaulsThesis}, which is defined on the category $[\Inj,\Set]$ of functors from $\Inj$, the category of natural numbers and injections, into the category of sets.  
It is given by the following formula, where $V$ denotes a fixed set of values.
\[
  (TA)(n) = V^n \to \left( \int^{p\from\Inj} V^p \times A(p) \times \Inj(n, p) \right)
  \]
Here, if $A \from \Inj \to \Set$ is a functor, then we think of $A(n)$ as the set of all values that $A$ takes on in the presence of $n$ local variables taking values in $V$.

We cannot construct this coend in the category of games, but there is a clear link with our constructions with probabilistic monads.  
Note that the `state action'
\[
  (W, A) \mapsto W \to (A \times W)
  \]
has mixed variance in $W$, suggesting a modification of our $\C/\X$ construction that uses a coend rather than an ordinary colimit.

\section{Relational Models}

The archetypal Cartesian closed category is the category $\Set$ of sets and functions.  
Most of the work in this thesis has been to do with reader actions (including reader monads as a special case), and we have already provided a classification of the reader actions on $\Set$ in Propositions \ref{PropColimitOfActionIsMonoidalFunctor} and \ref{PropMonoidalFunctorIsColimitOfReaderAction}: namely, they are classified by lax monoidal functors $F\from\Set \to \Set$, in such a way that if a reader action given by an oplax monoidal functor $j\from\X \to \Set$ corresponds to the functor $F \from \Set \to \Set$, then $\Set/\oppcat\X$ is isomorphic to the category $F_*\Set$ formed from $\Set$ by base change along $F$.
Thus, the theory in the case of $\Set$ reduces to the theory of $\Set$-enriched base change.

There are, however, many more examples in the literature of models of programming languages that ultimately derive from Kleisli categories for monads not of the reader kind.  
The problem with this, from our point of view, is that the resulting Kleisli categories will not be automatically Cartesian.  
This means that we need different methods to ensure that we end up with a category suitable for modelling a compositional semantics.

For example, the Kleisli category for the powerset monad on $\Set$ is the category $\Rel$ of sets and relations.  
This is a monoidal closed category, so can be used to model multiplicative linear type theory.
By using a linear exponential comonad based on finite multisets, we can transform it into a Cartesian closed category.

An important category derived from the category of sets and relations is the category $\Coh$ of coherence spaces.  
A coherence space is a set with some additional structure on it -- namely, the structure of an undirected graph -- and a morphism between coherence spaces is a relation between sets that respects this structure in a particular way (see \cite{MelliesCoherence} for details).  

Danos and Ehrhard in \cite{ProbabilisticCoherenceSpaces} develop a probabilistic version of coherence spaces, which is still monoidal closed.  
By using the finite multiset comonad, we can transform it into a Cartesian closed category.  
A striking result of Ehrhard, Pagani and Tasson \cite{ProbabilisicPcf} is that this category is already fully abstract for a probabilistic variant of PCF.

This suggests an alternative way to get round the Cartesianness hurdle when adding effects: start with a model of the base language that is constructed by applying a linear exponential comonad to a monoidal closed category, and then delay application of the linear exponential comonad until after we have added the effect.

Unfortunately, there is no real guarantee that a linear exponential comonad on the base category will give us a linear exponential comonad on a Kleisli category or one of the other effectful categories we have considered.  
Some work is required to investigate the conditions in which we can lift an exponential construction in this way, given the functorial results from Chapter \ref{ChapParametricMonads}.  
This will pave the way for us to understand the relational and coherence-space models using the ideas developed in this thesis.

\section{Adequacy Proofs}

In Chapters \ref{ChapMonads} and \ref{ChapParametricMonadsFullAbstraction} we proved Computational Adequacy via a factorization and operational techniques.  
This is good, because it allows us to assume very little about the underlying model, other than that it is itself computationally adequate.  
Nevertheless, our techniques rely very heavily on the fact that our base language is Idealized Algol.  
Although we outlined ways to get around this in Section \ref{SecStatelessLanguages}, it would also be useful to have a conventional logical relations-based proof of Computational Adequacy.

Assume that $\G$ is a Cartesian closed category enriched in directed-complete partial orders (dcpos).
If $M\from\G \to \G$ is a monoidal monad, then the Kleisli category associated to $M$ inherits an order, whereby $\sigma\le\tau\from A \to B$ in $\Kl_M\G$ if $\sigma\le\tau\from A \to MB$ in $\G$.
Moreover, since composition in the Kleisli category is given by a composition in the base category, this partial order will be respected by composition, as will be limits of directed sets.

Suppose again that $\G$ is a dcpo-enriched category, and that a monoidal category $\X$ acts on $\G$ via a lax action.  
Then the construction of \Mellies yields an $[\X,\Dcpo]$-enriched category $\Mell_\X\G$.
Since $\Dcpo$ is cocomplete \cite{CocompleteDcpo}, we can construct the coends we need to define the Day convolution product in $[\X,\Dcpo]$.

If $\X$ is small, or if the required colimits exist for other reasons, then we may take the change of base with respect to the colimit functor $[\X,\Dcpo] \to \Dcpo$ to $\Mell_\X\G$ to give us a $\Dcpo$-enriched category $\G/\X$.

Concretely, the underling set of morphisms in $\G/\X$ contains all the usual morphisms from the $\Set$-enriched case, while the order structure is given by a transfinite construction as in \cite[2.17]{Fiech}.  
In particular, if $\sigma,\tau\from A \to B$ in $\G/\X$ are given by morphisms
\begin{mathpar}
  \tilde{\sigma}\from A \to X.B
  \and
  \tilde{\tau} \from A \to Y.B
\end{mathpar}
in $\G$, and if there are morphisms $h\from X \to Z$, $k\from Y \to Z$ in $\X$ such that
\[
  \sigma;(h.B)\le\tau;(k.B)\,,
  \]
then $\sigma\le\tau$ in $\G/\X$.

A slight problem is that it is not true in general that the colimit of cpos with bottom elements has itself a bottom element.  
For example, let $-[n]$ be the partial order $\{-n<\cdots<0\}$.  
Then, $-[n]$ is a dcpo with a bottom element for each $n$, but the colimit of the chain
\[
  -[0] \hookrightarrow -[1] \hookrightarrow -[2] \hookrightarrow \cdots
  \]
is the partial order of non-positive integers.

However, if all the morphisms in a diagram of dcpos are such that they preserve bottom elements, then the colimit of that diagram will have a bottom element.

In particular, if the action of $\X$ on $\G$ is a reader-style action that factors through the category of sets (as in all our examples), then every arrow in the colimit diagram that defines $\G/\X(A,B)$ will be of the form
\[
  \G(A, \ul Y \to B) \xrightarrow{\G(A,\ul f.B)} \G(A, \ul X \to B)
  \]
for some sets $X,Y$, considered as objects $\ul X,\ul Y$ of $\G$, and some function $f \from X \to B$ considered as a morphism $\ul f \from \ul A \to \ul B$ such that composition with $\ul f$ preserves bottom elements.
Therefore, the map $\G(A,\ul f.B)$ of dcpos will also preserve bottom elements.

A second problem is that the setwise colimit of the $\G(A,X.B)$, with the order from \cite{Fiech}, is not necessarily itself a dcpo.  
In fact, to get the correct colimit of the diagram, we need to perform an extra step of \emph{completion}.  
However, with a little more care, we can show that if the action is a reader action that factors through the category of sets, then the colimit is already directed complete under the order, at least in the category of games.

Having given the definition of the order-enriched structure of $\G$, we may define the interpretation of the fixpoint combinator $\Y$ using the properties of this new order, which will be enough to prove adequacy along the lines of Theorem \ref{TheComputationalAdequacyIA}.

It is worth mentioning that although we can prove computational adequacy for our various Kleisli categories (and categories of the form $\G/\X$), it does not automatically follow that we can prove adequacy in the same way for the various quotiented categories.  
An important example is our model of countable nondeterminism with must testing.  
It is well known (see, e.g., \cite{Apt,PlotkinApt}) that countable nondeterminism is inextricably linked to discontinuity of composition, which robs us of an important ingredient in the standard order-theoretic proof of adequacy.

The solution is to prove adequacy in the normal way for the semantics of IA${}_\bN$ in $\G_\bN$, and then to apply the argument from Corollary \ref{CorAdeqMust} to reduce this to a Computational Adequacy result for the semantics of Idealized Algol with countable nondeterminism and must testing.
Here, we are using an idea due to Levy \cite{LevyGsInfinite}: that if we want to prove adequacy for countable nondeterminism, then we need a form of \emph{explicit forcing}, in which we uncover some information about the points at which we have made an infinite nondeterministic branch.  
In our case, the explicit forcing is done through the language IA${}_\bN$, which creates a complete record of the nondeterministic values we have chosen along the way.  
We have then applied the second part of Levy's technique -- \emph{hiding} -- to remove the explicit information and yield a pure, computationally adequate model of countable nondeterminism.

\section{Afterword}

The aim of this thesis was to demonstrate how techniques of categorical algebra can be applied to the study of Full Abstraction for programming languages.
I can think of two main benefits of this combination.
The first is a technical one: the results that we have proved are very generally applicable, and it would not be unreasonable to hope that they could be applied in the future to help construct new Fully Abstract models of programming languages.  
The second reason is less precise, but equally important: throughout this thesis, we have endeavoured to clear up some of the mystery surrounding Full Abstraction, particularly in Game Semantics.  
The idea of using monads and Kleisli categories to model computational effects has a long history, but work in Full Abstraction, though it sometimes uses category-theoretic techniques, has often seemed fairly \emph{ad hoc}.  
My hope is that the work presented here -- including both the sequoidal-exponential semantics for Idealized Algol and the work on Kleisli categories and parametric monads -- has gone some way towards explaining where some of the constructions used in Game Semantics come from.

One of the themes in this thesis has been an attempt to push results as far as they can go: thus, we have tried as far as possible to use, for example, Computational Adequacy results for a base language in order to prove Computational Adequacy for an extended language, rather than writing on a stand-alone proof.
To an extent, we have been forced to do this by the level of generality we have been working at, but I hope that our focus on syntactic rather than semantic results has helped shift the focus away from the properties of any particular model.

