
\subsection{Game semantics for countable nondeterminism with must testing}

In general, it is harder to model terms up to must equivalence than it is up to may equivalence.
For the finite nondeterminism case, the model of Harmer and McCusker \cite{mcCHFiniteND} uses nondeterministic strategies to model terms, as with may-testing, but these are not enough on their own: for instance, the terms $\skipp$ and $\If \ask_\bB \Then \skipp \Else \Omega$ are both denoted by the strategy
\[
  \{\epsilon,qa\}
  \]
for $\bC$, but while these terms are may equivalent, $\skipp \mustconverge$, while $\If \ask_\bB \Then \skipp \Else \Omega\diverges$.
Harmer and McCusker's solution is to augment each strategy $\sigma$ to include a set $D_\sigma$ of $O$-positions after which play may diverge.  
Then the denotation of $\skipp$ is
\[
  (\{\epsilon,qa\},\emptyset)\,,
  \]
while the denotation of $\If \ask_\bB \Then \skipp \Else \Omega$ is
\[
  (\{\epsilon,qa\},\{q\}\}\,,
  \]
where the divergence set $\{q\}$ signifies the fact that there is a possible divergence after the initial question move.

For countable nondeterminism, there is an additional subtlety.  
Consider the term
\[
  c \from\com \ts M = (\Y_{\nat \to \com} \lambda f.\lambda n.\IfO n \Then \skipp \Else c;f(\pred n)) \ask_\bN \from \com\,.
  \]
So $M$ nondeterministically chooses an arbitrarily large number and then calls its argument that many times.
Then consider also the term
\[
  c \from \com \ts N = \IfO \ask_\bN \Then M \Else \Y_{\com} (\lambda f.c;f) \from \com\,,
  \]
which exhibits all the behaviours of $M$, but may also call its argument infinitely many times.

The denotation of $M$ is as the strategy with maximal plays
\[
  q(qa)^na
  \]
for all $n\in\bN$, and no divergences.  
The denotation of $N$ should be the same as that of $M$, except that we must add an extra infinite sequence of increasing plays to capture the divergent behaviour.  
But this infinite sequence of plays is precisely those plays of the form
\[
  q(qa)^mq\,,
  \]
and these are already contained in the denotation of $N$.  
In other words, $M$ and $N$ have the same denotation, because the semantics is not able to distinguish between infinite (divergent) computations, and those that are convergent but may take an arbitrarily long time.
This is a problem, because while $(\lambda c.M)\skipp\mustconverge$, $(\lambda c.N)\skipp\diverges$.

Following Roscoe \cite{RoscoeCspInfinite} and Levy \cite{LevyGsInfinite}, Jim Laird and I published a fully abstract game semantics for IA with countable nondeterminism \cite{CslPaper}, in which a strategy is allowed to contain infinite as well as finite plays.
Our semantics distinguishes between the terms $M$ and $N$ above, in that only $\deno N$ contains the infinite play
\[
  qqaqaqa\cdots\,.
  \]
A problem then arises
