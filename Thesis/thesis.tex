\chapter{Monads and Kleisli categories}
\label{ChapMonads}

\section{Monads}

Let $\C$ be a category.  
Then the category $[\C,\C]$ of functors $\C\to\C$ and natural transformations has a (strict) monoidal structure given by composition.  
A \emph{monad} \cite[\sec VI]{WorkingMathematician} in $\C$ is a monoid in $[\C,\C]$.

In other words, a monad is a functor $M\from \C\to\C$ together with natural  transformations $m_a \from MMa \to Ma$ and $u_a \from a \to Ma$ such that the following diagrams commute for all objects $a$ of $\C$.

\begin{mathpar}
  \begin{tikzcd}
    MMM a \arrow[r, "Mm_a"] \arrow[d, "m_{Ma}"]
      & MM a \arrow[d, "m_a"] \\
    MM a \arrow[r, "m_a"]
      & M a
  \end{tikzcd}
  \and
  \begin{tikzcd}
    M a \arrow[r, "M u_a"] \arrow[dr, "id"']
      & M M a \arrow[d, "m_a"] \\
    %
      & M a
  \end{tikzcd}
  \and
  \begin{tikzcd}
    M a \arrow[r, "u_{M a}"] \arrow[dr, "id"']
      & M M a \arrow[d, "m_a"] \\
    %
      & M a
  \end{tikzcd}
\end{mathpar}

\begin{example}
  In the category of sets, the \emph{nonempty powerset functor} $\powerset_+$ sends a set $A$ to the set of nonempty subsets of $A$.  
  This functor has the structure of a monad on $\Set$, since we have a natural transformation (union) from $\powerset_+\powerset_+A$ to $\powerset_+A$ and a natural transformation (singleton) from $A$ to$ \powerset_+A$ that obey the diagrams given above.
\end{example}
\begin{example}
  Let $\M$ be a monoidal category and let $x$ be a monoid in $\M$.  
  The \emph{writer monad} $W_x$ on $\M$ is defined by $W_x y = y \tensor x$, with natural transformations
  \begin{mathpar}
    m_y \from y \tensor x \tensor x \to y \tensor x
    \and
    u_y \from y \to y \tensor x
  \end{mathpar}
  given by the monoid structure on $x$.

  Going the other way, if $\M$ is monoidal closed with inner hom $\implies$, and if $z$ is a comonoid in $\M$, then the \emph{reader monad} $R_z$ is given by $R_z y = z \implies y$.  
  Then the monadic coherences
  \begin{mathpar}
    m_y \from z \implies z \implies y \to z \implies y
    \and
    u_y \from y \to z \implies y
  \end{mathpar}
  are induced from the comonoid structure on $z$.

  If the monoidal structure on $\M$ is Cartesian, then a comonoid in $\M$ is the same thing as an object $a$ of $\M$ (with the diagonal $a \to a \times a$ and terminal $a \to 1$ morphisms).  
  In such a case, every object $a$ of $\M$ gives rise to a reader monad on $\M$.
\end{example}
\begin{example}
  If $\adjunction{\C}{L}{R}{\D}$ is an adjunction with counit $\epsilon\from LR\to 1$ and unit $\eta\from 1 \to RL$, then the composite $RL\from \C \to \C$ has the structure of a monoid on $\C$, where the multiplication and unit are given by
  \begin{mathpar}
    R\epsilon L \from RLRL \to RL
    \and
    \eta \from 1 \to RL\,.
  \end{mathpar}
  We will see in the next chapter that every monad is induced by an adjunction in this way.

  Let us look at an example of a monad that arises from an adjunction in this way.  
  Recall that the definition of a monoidal closed category $\M$ is that it admits an adjunction between functors $\blank\tensor w$ and $w\implies\blank$ for any object $w$ of $M$.  
  If $\M$ is a monoidal closed category and $w$ is an object of $\M$, then the functor $S_w$ on $\M$ defined by
  \[
    S_w x = w \implies (x \tensor w)
    \]
  is the composition of these adjoint functors and therefore inherits the structure of a monad, known as the \emph{state monad}.
\end{example}
\begin{example}
  Another example that arises from an adjunction is the \emph{list monad} on $\Set$ that arises from the adjunction between the category $\Set$ of sets and the category of monoids.
  The underlying set of the free monoid on a set $A$ is the set $A^*$ of finite lists of elements of $A$, and the functor $A\mapsto A^*$ inherits a monoid structure where the multiplication $m_A\from (A^*)^* \to A^*$ concatenates a list of lists into a single list and the unit $u_a \from A \to A^*$ forms a list with a single element.  
\end{example}
\begin{example}
  A monad on $\oppcat\C$ is called a \emph{comonad} on $\C$.  
  The carrier of a comonad is still a functor $M\from \C\to\C$, but now the multiplication and unit are natural transformations $M\Rightarrow MM$ and $M\Rightarrow 1$, rather than the other way round.  

  An adjunction $\adjunction\C LR\D$ gives rise to a comonad structure on $LR$ in much the same way as it gives rise to a monad structure on $RL$.  
  So, for example, we have the \emph{store comonad} $S_r'$ for any object $r$ of a monoidal closed category $\M$, given by
  \[
    S_r'x = (r \implies x) \tensor r\,.
    \]
\end{example}

\section{Kleisli Categories}

Let $\C$ be a category and let $M$ be a monad on $\C$.  
Then \cite{Kleisli} there is a category $\Kl_M$, called the \emph{Kleisli category} of $M$, whose objects are the objects of $\C$ and where a morphism from an object $a$ to an object $b$ is a morphism $a \to Mb$ in $\C$.

Identity arrows are given by the morphisms $u_c\from c \to M c$ (considered as morphisms $c\to c$ in $\Kl_M$) and the composition of arrows $f\from a \to Mb$ and $g \from b \to Mc$ is given by the following composite in $\C$.
\[
  a \xrightarrow{f}
  Mb \xrightarrow{Mg}
  MMc \xrightarrow{m_c}
  M c
  \]
There is a natural identity-on-objects functor $J\from \C \to \Kl_M$ that sends a morphism $f\from a \to b$ in $\C$ to the composite
\[
  a \xrightarrow{f}
  b \xrightarrow{u_b}
  M b\,,
  \]
considered as a morphism $a\to b$ in $\Kl_M$.

In the other direction, we have a functor $ S \from \Kl_M \to \C$ that sends an object $a$ of $\Kl_M$ to the object $Ma$ of $\C$ and sends a morphism $f\from a \to M b$ from $a$ to $b$ in $\Kl_M$ to the composite
\[
  Ma \xrightarrow{M f}
  MMb \xrightarrow{m_b}
  Mb
  \]
in $\C$.  
Note that $ S J=M$, by one of our coherence conditions on $m$ and $u$.
Meanwhile, $J S $ is the functor $\Kl_M\to\Kl_M$ that sends an object $a$ to $Ma$ and sends a morphism $a\to b$ given by a morphism $f\from a \to Mb$ in $\C$ to the morphism $Mf\from Ma \to MMb$, considered as a morphism $Ma \to Mb$ in $\Kl_M$.
\begin{proposition}[\cite{Kleisli}]
  $ S $ is a right adjoint to $J$.
  The unit of the adjunction is $u\from \id \Rightarrow M$.  
  The counit $e_a\from J( S  a) \to a$ is given by the identity morphism $Ma \to Ma$ in $\C$, considered as a morphism $Ma \to a$ in $\Kl_M$.
  \label{prop:KleisliHasAdjunction}
\end{proposition}

\section{Multiplicative natural transformations}

Given a monad $M$ on a category $\C$ and a functor $F\from \C \to \D$, where $\D$ is another category, we say that a natural transformation $\psi_a \from FMa \to Fa$ is \emph{$M$-multiplicative} if it makes the following diagrams commute.
\begin{mathpar}
  \begin{tikzcd}
    FMMa \arrow[r, "\psi_{Ma}"] \arrow[d, "Fm_a"']
      & FMa \arrow[d, "\psi_a"] \\
    FMa \arrow[r, "\psi_a"]
      & Fa
  \end{tikzcd}
  \and
  \begin{tikzcd}
    Fa \arrow[r, "F u_a"] \arrow[dr, "\id"']
      & FMa \arrow[d, "\psi_a"] \\
    %
      & Fa
  \end{tikzcd}
\end{mathpar}

Given two triples $(\D, F,\psi), (\D', F',\psi')$, where $F\from \C \to \D, F'\from \C'\to\D'$ are functors and $\psi\from FM\Rightarrow F, \psi'\from F'M\Rightarrow F'$ are functors, we define a \emph{morphism} from $(\D', F',\psi')$ to $(\D, F, \psi)$ to be a functor $H\from \D' \to \D$ such that $F=HF'$ and $\psi=H\psi'$.  
This gives us a category.

A defining property of the Kleisli category is that it is an initial object in the category of such triples $(\D,F,\psi)$:

\begin{proposition}[\cite{StreetMonads}]
  i) Given an object $a$ of $\C$, consider the identity morphism $Ma \to Ma$ as a morphism $\phi_a \from JMa \to Ja$ in $\Kl_M$.  
  Then $\phi_a$ is an $M$-multiplicative natural transformation.

  ii) Let $\D$ be a category, let $F\from \C \to \D$ be a functor and suppose that $\psi_a\from FMa \to Ma$ is an $M$-multiplicative natural transformation.
  Then there is a unique functor $\hat{F}\from \Kl_M \to \D$ such that $F=\hat{F}J$ and $\psi = \hat{F}\phi$.
  \label{pKleisli}
\end{proposition}

Another way to characterize the Kleisli category $\Kl_M$ is to say that the adjunction we described above is initial among all adjunctions giving rise to the monad $M$.  
This can be deduced from Proposition \ref{pKleisli} using the following result.

\begin{lemma}[\cite{StreetMonads}]
  Let $\C$ be a category and let $M$ be a monad on $\C$.  
  If $\adjunction{\C}{L}{R}{\D}$ is an adjunction (with counit $\epsilon$ and unit $\eta$), we say it \emph{gives rise to $M$} if $M=RL$, $m=R\epsilon L$ and $u=\eta$.

  Any such adjunction gives rise to an $M$-multiplicative natural transformation $\psi\from LM \Rightarrow L$.  
  This gives us a fully faithful functor from the category of adjunctions giving rise to $M$ to the category of triples $(\D,F,\psi)$ where $\psi$ is $M$-multiplicative.
\end{lemma}

The proof of Proposition \ref{pKleisli} essentially comes down to the following factorization result.  
If $f\from a \to b$ is a morphism in $\Kl_M$, then $f$ may be factorized as
\[
  f = a \xrightarrow{Jf}
  Mb \xrightarrow{\phi_b}
  b\,,
  \]
where we use `$f$' to refer both to the morphism $a\to b$ in $\Kl_M$ and to the underlying morphism $a \to Mb$ in $\C$.
Indeed, if we compute this composite inside $\C$, we get
\[
  a \xrightarrow{f}
  Mb \xrightarrow{u_{Mb}}
  MMb \xrightarrow{M\id}
  MMb \xrightarrow{m_b}
  Mb\,,
  \]
which is equal to $f$ by the coherence conditions on $m$ and $u$.
This means that the Kleisli category may be thought of as being freely generated from the original category $\C$ and an $M$-multiplicative natural transformation $\phi$.

\begin{example}
  The morphisms in the Kleisli category for the nonempty powerset monad $\powerset_+$ on $\Set$ are functions $A \to \powerset_+B$, which can be thought of as nondeterministic functional programs.  
  Given a set $A$, the morphism $\phi_A\from \powerset_+A \to A$ in $\Kl_{\powerset_+}$ can be interpreted as a `nondeterministic choice' function that accepts a nonempty set of elements of $A$ and nondeterministically chooses one of them.
  The factorization then means that the category is freely generated over $\C$ by these nondeterministic choice morphisms.
\end{example}
\begin{example}
  Let $\C$ be a Cartesian closed category and let $z$ be some fixed object of $\C$.  
  Then the Kleisli category for the reader monad $R_z$ on $\C$ is generated over $\C$ by a natural transformation $\phi_y\from (z \to y) \to y$.  
  As a Cartesian closed category, $\C$ may be regarded as being enriched over itself.
  By the enriched Yoneda lemma, then, such a natural transformation is always given by precomposition with some fixed morphism $\ask_z\from 1 \to z$.  
  This means that $\Kl_{R_z}$ is suitable for modelling any situation in which we are generally working in $\C$, but need the ability to request a value of type $z$ (for example, a config file, a piece of user input, a random number or something else that isn't being passed into the function in question).
  \label{ExReaderMonadKleisli}
\end{example}

A particularly important fact about the reader monad in Cartesian closed categories is the following.

\begin{theorem}[\cite{FunctionalCompleteness}]
  Let $\C$ be a Cartesian closed category and let $z$ be an object of $\C$.  
  Then the Kleisli category $\Kl_{R_z}$ for the reader monad over $z$ on $\C$ is Cartesian closed.
  \label{FunctionalCompletenessCcc}
\end{theorem}

The \emph{functional completeness} theorem \cite{FunctionalCompleteness} can be thought of as a special case of our remarks above.

\section{Denotational Semantics}

Having given an overview of the general theory of monads and Kleisli categories, we now examine the relationship between Kleisli categories and Full Abstraction.  From now till the end of the chapter, we fix an base Cartesian closed category $\G$ that admits a denotational semantics of Idealized Algol satisfying Computational Adequacy.  
In addition, we require that $\G$ can be interpreted as being enriched in algebraic partial orders, in such a way that any compact morphism between the denotation of IA types is the denotation of some IA term.
The prototypical example, of course, will be the category of games and visible strategies that we met in Chapter \ref{ChapGames}, but we will not exploit any properties of this model beyond the ones we have already mentioned, mentioning it only in examples where appropriate.

In addition, we will specialize to reader monads.  
The rationale behind this is twofold: firstly, we do not wish to assume too much about the underlying category $\G$ beyond the fact that it gives us a good model of Idealized Algol.  
Reader monads can be constructed for any object inside any Cartesian closed category.
Secondly, in order to get a compositional semantics for the $\lambda$-calculus, it is important that the Kleisli categories we consider are themselves Cartesian closed.  
Theorem \ref{FunctionalCompletenessCcc} tells us that this is the case for the Kleisli categories of reader monads on categories that are themselves Cartesian closed, but this may not be the case for other monads we can build from the Cartesian closed structure on our base category, such as state monads.

Let $X\in \{\bB,\bN,\bC\}$ be a set that has an interpretation as an Idealized Algol type $X$, and write $X$ also for the corresponding object of $\G$.
We shall write $\G_X$ as a shorthand for $\Kl_{R_X}\G$, the Kleisli category for the reader monad on $\G$ that corresponds to the object $X$.  
The purpose of the rest of this chapter will be to define a new language, give it a denotational semantics in $\G_X$, and prove a full abstraction result for this denotational semantics.

\newcommand{\IAX}{{IA${}_X$}\xspace}
\begin{definition}[{The language \IAX}]
  The language \IAX is formed by taking Idealized Algol, and adding to it a new constant
  \[
    \ask_X
    \]
  with typing rule
  \[
    \inferrule{ }{\Gamma \ts \ask_X \from X}\,.
    \]
\end{definition}

From Proposition \ref{pKleisli}, we know that there is a distinguished natural transformation $\phi_A \from (X \to A) \to A$ in $\G_X$; in particular, we have a morphism
\[
  \phi = \Lambda(\lunit_X;\id_X);\phi_X
  \]
(where $\Lambda(\lunit_X;\id_X)\from 1 \to (X \to X)$ is the interpretation of the $\lambda$-term $\lambda x.x$), which will be the denotation of the term $\ask_X$.
Together with the existing denotational semantics of Idealized Algol within $\G$, this gives us an inductively defined denotational semantics of \IAX within $\G_X$.

Clearly any term-in-context of \IAX is of the form
\[
  \Gamma \ts M[\ask_X/x]\from T\,,
  \]
where
\[
  \Gamma,x\from X \ts M \from T
  \]
is a judgement of Idealized Algol.
Given such a term-in-context, we know that the denotation of
\[
  \Gamma\ts (\lambda x.M) \ask_X \from T
  \]
is given by the composite
\[
  1 \xrightarrow{\phi}
  X \xrightarrow{\deno{\Gamma,x \ts M}}
  \deno{T}\,.
  \]
Now this last term is $\beta$-equivalent to our original term-in-context $\Gamma\ts M$.  
Since $\G_X$ is Cartesian closed (by Theorem \ref{FunctionalCompletenessCcc}), the $\beta$ rule is valid in $\G_X$, and this means that the composite above is an alternative definition of the denotation of $\Gamma\ts M$.

\section{Operational Semantics}

We now define the operational semantics of \IAX and prove a computational adequacy result for our denotational semantics.

\begin{definition}[Operational semantics of \IAX]
  Let $X^*$ be the free monoid on the set $X$; i.e., the set of all finite strings of elements of $X$.
  Given $u,v\in X^*$ we shall write $u\cat v$ for their product in $X^*$; i.e., the concatenation of the two strings.
  We shall write $\epsilon$ for the unit in $X^*$; i.e., the empty string.

  If $u\in X^*$,we write $|u|$ for the length of $u$.
  If $0\le n < |u|$, then we write $u^{(n)}$ for the corresponding element of $u$, numbering from $0$.

  We inductively define a relation $\Gamma,s\ts M\converges_u c,s'$, where $\Gamma$ is a $\Var$-context, $M,c$ are terms of \IAX with all free variables in $\Gamma$, where $c$ is an IA canonical form, $s,s'$ are $\Gamma$-stores and $u\in X^*$.
  The definition of this relation is shown in Figure \ref{FigIaxOpSem}.

  \begin{figure}
    \small
    \begin{mathpar}
      \inferrule*{ }{\Gamma,s\ts c \converges_\epsilon c,s}
      \and
      \inferrule*{\Gamma,s \ts M \converges_u \lambda x.M',s' \\ \Gamma,s' \ts M'[N/x] \converges_v c,s''}{\Gamma,s \ts MN \converges_{u\cat v} c,s''}
      \and
      \inferrule*{\Gamma,s \ts M(\Y M) \converges_u c,s'}{\Gamma,s \ts \Y M \converges_u c,s'}
      \and
      \inferrule*{\Gamma,s\ts M \converges_u n,s'}{\Gamma,s\ts \suc M \converges_u n+1,s'}
      \\\and
      \inferrule*{\Gamma,s\ts M \converges_u n+1,s'}{\Gamma,s\ts \pred M \converges_u n,s'}
      \and
      \inferrule*{\Gamma,s\ts M \converges_u 0,s'}{\Gamma,s\ts \pred M \converges_u 0,s'}
      \and
      \inferrule*{\Gamma,s\ts M \converges_u \skipp,s' \\ \Gamma,s'\ts N \converges_v c,s''}{\Gamma,s \ts M;N \converges_{u\cat v} c,s''}
      \and
      \inferrule*{\Gamma,s\ts M \converges_u \true,s' \\ \Gamma,s' \ts N \converges_v c,s''}{\Gamma,s \ts \If M \Then N \Else P \converges_{u\cat v} c,s''}
      \and
      \inferrule*{\Gamma,s\ts M \converges_u \false,s' \\ \Gamma,s' \ts P \converges_v c,s''}{\Gamma,s \ts \If M \Then N \Else P \converges_{u\cat v} c,s''}
      \and
      \inferrule*{\Gamma,s\ts M \converges_u 0,s' \\ \Gamma,s' \ts N \converges_v c,s''}{\Gamma,s \ts \IfO M \Then N \Else P \converges_{u\cat v} c,s''}
      \and
      \inferrule*{\Gamma,s\ts M \converges_u n+1,s' \\ \Gamma,s' \ts P \converges_v c,s''}{\Gamma,s \ts \IfO M \Then N \Else P \converges_{u\cat v} c,s''}
      \\
      \inferrule{\Gamma,s\ts M\converges c',s' \\ \Gamma,s' \ts N[c'/x]\converges c,s''}{\Gamma,s\ts \lett x=M \inn N \converges c,s''}
      \\
      \inferrule*[right=$x\in\Gamma$]{\Gamma,s\ts E \converges_u n,s' \\ \Gamma,s' \ts V \converges_v x,s''}{\Gamma,s\ts V \gets E \converges_{u\cat v} \skipp,(s''\vert x \mapsto n)}
      \and
      \inferrule*[right={$s'(x)=n$}]{\Gamma,s\ts V \converges_u x,s'}{\Gamma,s\ts !V \converges_u n,s'}
      \and
      \inferrule*{\Gamma,x\from\Var,(s\vert x\mapsto 0)\ts M \converges_u c,(s'\vert x\mapsto n)}{\Gamma,s\ts \neww \lambda x.M \converges_u c,s'}
      \and
      \inferrule*{\Gamma,s\ts E \converges_u n,s' \\ \Gamma,s'\ts V \converges_v \mkvar W R,s'' \\ \Gamma,s'' \ts Wn \converges_w \skipp,s'''}
      {\Gamma,s \ts V\gets E \converges_{u\cat v\cat w} \skipp,s'''}
      \and
      \inferrule*{\Gamma,s\ts V \converges_u \mkvar W R,s' \\ \Gamma,s'\ts R \converges_v n,s''}{\Gamma,s\ts !V \converges_{u\cat v} n,s''}
      \\\and
      \inferrule*[right=$x\in X$]{ }{\Gamma,s\ts \ask_X \converges_x x,s}
    \end{mathpar}
    \caption[Operational semantics for \IAX]{Operational semantics for \IAX.  
    All the rules except the last one are deterministic and may be obtained from the corresponding rules of Idealized Algol by suitably annotating the $\converges$ relation with sequences from $X^*$.}
    \label{FigIaxOpSem}
  \end{figure}

  Closer examination of the rules in Figure \ref{FigIaxOpSem} reveals an alternative, indirect definition of the operational semantics of \IAX.
  Note that each rule from ordinary Idealized Algol takes the form
  \[
    \inferrule{\Gamma,s^{(0)}\ts M_1\converges c_1,s^{(1)} \\ \cdots \\ \Gamma,s^{(n-1)}\ts M_n\converges c_n,s^{(n)}}
    {\Gamma,s^{(0)} \ts M \converges c,s^{(n)}}\,,
    \]
  Here, we have interpreted each IA rule as an infinite scheme of rules ranging over the different terms $M_i,M$ that the rule can apply to.
  We first extend this scheme of rules to a scheme of rules for \IAX, by allowing the $M_i,M$ to range over terms of \IAX.
  We then replace the rule with the new rule
  \[
    \inferrule{\Gamma,s^{(0)}\ts M_1 \converges_{u_1} c_1,s^{(1)} \\ \cdots \\ \Gamma,s^{(n-1)}\ts M_n \converges_{u_n} c_n,s^{(n)}}
    {\Gamma,s^{(0)} \ts M \converges_{u_1 \cat \cdots \cat u_n} c,s^{(n)}}\,,
    \]
  to give us an operational rule for \IAX (if $n=0$, then we treat the empty string $\epsilon$ as the empty concatenation).
  Lastly, we add the rule for the new constant $\ask_X$:
  \[
    \inferrule*[right=$x\in X$]{ }{\Gamma,s \ts \ask_X \converges_x x,s}\,.
    \]
  This rule is the only nondeterministic one in our language, as well as being the only one in which the sequence annotating the $\converges$ symbol at the bottom is not formed from concatenating together the sequences on the top.
\end{definition}

\begin{example}
  If $X=\bC$, then, since $X$ has a single element $a$, a sequence $u=\underbrace{a\cdots a}_n$ of elements of $X$ may be identified with its length $n$.
  In this case, the language \IAX gives us a way to model time complexity, and the term $\ask_X$ may be considered as a subroutine $\sleep\from \com$ whose semantics is to wait for some fixed period of time before continuing.  
  In this case, 
  \[
    \Gamma,s\ts M \converges_n c,s'
    \]
  is interpreted to say that `starting with the store $s$, $M$ converges to $c$ in time $n$, leaving store $s'$'.
\end{example}
\begin{example}
  If $X=\bN$, we can interpret the language \IAX as giving us a way to model user input.  
  When we call $\ask_\bN$, we are asking the user to provide us as a string of text (which we interpret as a binary string and hence as a natural number).  
  Then we interpret the judgement
  \[
    \Gamma,s\ts M \converges_u c,s'
    \]
  as saying that `starting with the store $s$, and the sequence $u$ of pieces of user input, $M$ converges to $c$, leaving store $s'$'.
\end{example}
\begin{example}
  If $X\in\{\bB,\bN\}$, then the language \IAX gives us a way to model nondeterminism, where $\ask_X$ behaves as a \emph{nondeterministic oracle}; i.e., a device that nondeterministically returns an element of $X$.

  If $X=\bB$ then we have a model of binary (i.e., finite) nondeterminism, whereas if $X=\bN$ then we have a model of countable nondeterminism.

  In these cases, we interpret the relation
  \[
    \Gamma,s\ts M \converges_u c,s'
    \]
  as saying that $M$ converges to $c$ in the case that the sequence of values returned by the nondeterministic oracle is given by the sequence $u$.
\end{example}

\section{Soundness}
\label{SecSoundness}

To prove our adequacy result for the operational semantics of \IAX, we first give some definitions.

\begin{definition}
  We inductively define terms in context $\tr_u\from \nat \to X$ of \emph{ordinary deterministic} Idealized Algol for each $u\in X^*$ as follows.
  \begin{mathpar}
    \tr_\epsilon = \lambda n.\Omega
    \and
    \tr_{xu} = \lambda n.\lett y=n \inn \IfO y \Then x \Else \tr_u(\pred y)
  \end{mathpar}
\end{definition}
In other words, $\tr_u$ is the function that will return the $i$-th term of the sequence $u$ when given $i$ as an input, or will fail to terminate if passed some index $k\ge|u|$.

\begin{proposition}
  Let $u \in X^*$ and let $n<|u|$.  
  Then it is possible to deduce that
  \[
    \inferrule{\Gamma,s \ts M \converges n,s'}{\Gamma,s \ts \tr_u M \converges u^{(n)},s'}
    \]
  in Idealized Algol.
  \label{PropTr}
\end{proposition}

We start our soundness proof with a small lemma to help us deal with substitution.

\begin{lemma}
  Let
  \[
    \inferrule{\Gamma,s^{(0)}\ts M_1\converges c_1,s^{(1)} \\ \cdots \\ \Gamma,s^{(n-1)}\ts M_n\converges c_n,s^{(n)}}
    {\Gamma,s^{(0)} \ts M \converges c,s^{(n)}}
    \]
  be an inference, where the $M_i$,$M$ are terms of \IAX and the whole inference satisfies one of the patterns of the Idealized Algol rules.  
  Let $Q$ be a fixed term of type $X$.  
  Then
  \[
    \inferrule{\Gamma,s^{(0)}\ts M_1[Q/\ask_X]\converges c_1,s^{(1)} \\ \cdots \\ \Gamma,s^{(n-1)}\ts M_n[Q/\ask_X]\converges c_n,s^{(n)}}
    {\Gamma,s^{(0)} \ts M[Q/\ask_X] \converges c,s^{(n)}}
    \]
  is a valid inference of Idealized Algol.
  \label{LemFirstSubstitution}
\end{lemma}
\begin{proof}
  Informally, this is true because the term $\ask_X$ is not mentioned anywhere in the IA rules, so substitution of the term $Q$ for $\ask_X$ could not possibly break the pattern.  
  Formally, we can show this by inspection on each of the different rules.  
  For instance, if the original rule is the one for sequencing:
  \[
    \inferrule{\Gamma,s\ts M \converges \skipp,s' \\ \Gamma,s'\ts N \converges c,s''}
    {\Gamma,s \ts M;N \converges c,s''}\,,
    \]
  then we have $(M;N)[Q/\ask_X] = M[Q/\ask_X];N[Q/\ask_X]$ and the inference
  \[
    \inferrule{\Gamma,s\ts M[Q/\ask_X] \converges \skipp,s' \\ \Gamma,s'\ts N[Q/\ask_X]\converges c,s''}
    {\Gamma,s \ts M[Q/\ask_X];N[P/\ask_X] \converges c,s''}
    \]
  is still a valid instance of the sequencing rule.
\end{proof}

We can now state and prove our soundness lemma.
As suggested by the substitution lemmas we have just proved, we will be taking a term $M$ of nondeterministic IA and replacing all instances of $\ask_X$ in $M$ with some particular term $Q$.  
The $Q$ that we will use is
\[
  \Gamma,v\from\Var \ts v \gets \succ \oc v;\tr_w \oc v\from X\,,
  \]
where $w$ is some sequence of elements of $X$.  
The idea is that this term will trace the effect of a particular evaluation path of $\ask_X$, similarly to how we sometimes `mock' a random number generator when writing unit tests.  

Each time we call the term above, we increment the value of $v$, and successive invocations will end up spelling out a particular substring $u$ of $w$ -- depending on the initial value of $v$ and assuming that $w$ is long enough.  
This will allow us to study one particular evaluation path of the original term $M$ -- namely, the one where the successive values returned by $\ask_X$ spell out the sequence $u$.

\begin{lemma}
  Suppose that
  \[
    \Gamma,s\ts M\converges_u c,s'
    \]
  is derivable in \IAX.
  Fix $k\in\bN$ and let $w\in X^*$ be a sequence such that $u$ is the substring of $w$ starting at position $k+1$ (i.e., $u^{(j)}=w^{(k+j+1)}$ for each $j=0,\cdots,|u|-1$).
  Then
  \[
    \Gamma,v\from\Var,(s\vert v\mapsto k) \ts M[v\gets \suc!v;\tr_w !v/\ask_v] \converges c,(s'\vert v\mapsto k + |u|)
    \]
  in Idealized Algol.
  \label{LemSoundnessMonads}
\end{lemma}
\begin{proof}
  Structural induction on the derivation.  

  Suppose that the last rule we use comes from one of the Idealized Algol rules.
  That is, there is an inference
  \[
    \inferrule{\Gamma,s^{(0)}\ts M_1\converges c_1,s^{(1)} \\ \cdots \\ \Gamma,s^{(n-1)}\ts M_n\converges c_n,s^{(n)}}
    {\Gamma,s^{(0)} \ts M \converges c,s^{(n)}}\,,
    \]
  derived from one of the Idealized Algol schemas, and we have replaced it with the rule
  \[
    \inferrule{\Gamma,s^{(0)}\ts M_1 \converges_{u_1} c_1,s^{(1)} \\ \cdots \\ \Gamma,s^{(n-1)}\ts M_n \converges_{u_n} c_n,s^{(n)}}
    {\Gamma,s^{(0)} \ts M \converges_{u_1 \cat \cdots \cat u_n} c,s^{(n)}}\,,
    \]
  where each of the relations $\Gamma,s^{(i-1)} \ts M_i \converges_{u_i} c_i,s^{(i)}$ is derivable in \IAX.

  Fix $k\in \bN$ and a sequence $w$ such that $u_1\cat \cdots \cat u_n$ is a subsequence of $w$ starting at position $k+1$.  
  In particular, for each $i=1,\cdots,n$, $u_i$ is a subsequence of $w$ starting at position $k+\sum_{j=1}^{i-1}|u_j|+1$.

  Then by the inductive hypothesis, we know that for each $i=1,\cdots,n$, the relation
  \small
  \[
    \Gamma,v,(s^{(i-1)}\vert v \mapsto k+\sum_{j=1}^{i-1}|u_j|) \ts M_i[v\gets \suc !v;\tr_w !v/\ask_v] \converges c,(s^{(i)}\vert v\mapsto k + \sum_{j=1}^i |u_j|)
    \]
  \normalsize
  is derivable in Idealized Algol.  
  Then we may apply the Idealized Algol inference and Lemma \ref{LemFirstSubstitution} to deduce that
  \[
    \Gamma,v,(s^{(0)}\vert v\mapsto k) \ts M[v\gets \suc !v;\tr_w !v/\ask_v] \converges c, (s^{(n)} \vert v\mapsto k + \sum_{i=1}^{n} |u_n|)\,,
    \]
  as desired.

  Now suppose instead that the last rule was the new one for $\ask_X$; i.e., 
  \[
    \inferrule{ }{\Gamma,s\ts \ask_X \converges_x x,s}\,,
    \]
  where $x\in X$.  
  Fix some $k\in \bN$ and some $w$ such that the single term $x$ is a subsequence of $w$ starting at position $k+1$; i.e., that $x=w^{(k+1)}$.
  Then we would like to derive that
  \[
    \Gamma,v,(s\vert v\mapsto k) \ts v \gets \suc !v;\tr_w !v \converges x,(s\vert v\mapsto k+1)\,,
    \]
  which we can do using the derivation in Figure \ref{FigTheOneForTheLemma}, where we have used Proposition \ref{PropTr} to deal with the $\tr_w$ term.

  This completes the induction.
\end{proof}

\begin{SidewaysFigure}
  \small
  \begin{subfigure}{\textheight}
    \[
      \inferrule*
      {
        \inferrule*
        {
          \inferrule*
          {
            \inferrule*
            {
              \inferrule*
              {
              }
              {
                \Gamma,v,(s\vert v\mapsto k) \ts v \converges v,(s\vert v\mapsto k)
              }
            }
            {
              \Gamma,v,(s\vert v\mapsto k) \ts !v \converges k,(s\vert v\mapsto k)
            }
          }
          {
            \Gamma,v,(s\vert v\mapsto k) \ts \suc !v \converges k+1,(s\vert v\mapsto k)
          }
          \\
          \inferrule*
          {
          }
          {
            \Gamma,v,(s\vert v\mapsto k) \ts v\converges v,(s\vert v\mapsto k)
          }
        }
        {
          \Gamma,v,(s\vert v\mapsto k) \ts v\gets \suc !v \converges \skipp,(s\vert v\mapsto k+1)
        }
        \\
          \inferrule*
          {
            \inferrule*
            {
              \inferrule*
              {
              }
              {
                \Gamma,v,(s\vert v\mapsto k+1) \ts v\converges v,(s\vert v\mapsto k+1)
              }
            }
            {
              \Gamma,v,(s\vert v\mapsto k+1) \ts !v \converges k+1,(s\vert v\mapsto k+1)
            }
          }
          {
            \Gamma,v,(s\vert v\mapsto k+1) \ts \tr_w !v \converges x,(s\vert v\mapsto k+1)
          }
        }
      {
        \Gamma,v,(s\vert v\mapsto k) \ts v\gets \suc !v;\tr_w !v \converges x,(s\vert v\mapsto k+1)
      }
      \]
    \normalsize
    \caption{IA derivation that if $s(v)=k$, then $\Gamma,s\ts v \gets \suc !v;\tr_w !v$ converges to the $k+1$-th term of $w$, leaving state $v\mapsto k+1$.}
    \label{FigTheOneForTheLemma}
  \end{subfigure}
  \vspace{48pt}\\
  \begin{subfigure}{\textheight}
    \[
      \inferrule*
      {
        \inferrule*[sep=0em]
        {
          \inferrule*[left={Lem. \ref{LemSoundnessMonads}}]
          {
          }
          {
            v,(v\mapsto 0)\ts M[v\gets\suc!v;\tr_{u^\top} !v/\ask_x]\converges \skipp,(v\mapsto |u|)
          }
          \\
          \inferrule*
          {
            \inferrule*
            {
            }
            {
              v,(v\mapsto |u|) \ts v \converges v,(v\mapsto |u|)
            }
          }
          {
            v,(v\mapsto |u|)\ts !v \converges |u|,(v\mapsto |u|)
          }
        }
        {
          v,(v\mapsto 0) \ts M[v\gets \suc !v;\tr_{u^\top} !v/\ask_X];v \converges |u|,(v\mapsto |u|)
        }
      }
      {
        \blank,()\ts\new (\lambda v.M[v\gets \suc !v;\tr_{u^\top} !v/\ask_X];!v) \converges |u|,()
      }\qedhere
      \]
    \caption{IA derivation used in the proof of Proposition \ref{PropKleisliSoundness}.}
    \label{FigKleisliSoundnessDerivation}
  \end{subfigure}
  \caption{Some useful IA derivations}
\end{SidewaysFigure}

In light of Lemma \ref{LemSoundnessMonads}, we can state our soundness result.

First recall the statement of Computational Adequacy for $\G$ (Theorem \ref{TheComputationalAdequacyIA} if $\G$ is the category of games):

\begin{theorem}
  Let $M\from \com$ be a closed term of Idealized Algol.  
  Then
  \[
    \blank,()\ts M\converges \skipp,()\,.
    \]
  if and only if $\deno{M} \ne \bot$.
\end{theorem}

\begin{definition}
  Let $u\in X^*$.
  Let $u^\top$ be the sequence formed by appending some fixed value $\top\in X$ to the start of $u$, so that $u$ is the subsequence of $u^\top$ running from position $1$ up to position $|u|$.
  Define a morphism
  \[
    \eta_u = \deno{f\from X \to \com \ts \lambda v.f(v\gets \suc !v;\tr_{u^\top} !v);!v} \from (X \to \bC) \to (\Var \to \bN)
    \]
  in $\G$.
\end{definition}

\begin{definition}
  Let $n$ be a natural number.  
  We inductively define terms $\test_n\from \nat \to \com$ by
  \begin{mathpar}
    \test_0 = \lambda m.\IfO m \Then \skipp \Else \Omega
    \and
    \test_{n+1} = \lambda m.\lett x=m \inn \IfO x \Then \Omega \Else \test_n(\pred x)
  \end{mathpar}
  So $\test_n$ converges if its input evaluates to $n$ and diverges otherwise.

  We then define $t_n\from \bN \to \bC$ to be the denotation of $\test_n$ in $\G$.
\end{definition}

\begin{definition}
  Let $\sigma \from 1 \to \bC$ be a morphism in $\G_X$, considered as a morphism $(X \to \bC)$ in $\G$.  
  We say that $\sigma\downarrow_u$ if the composite
  \[
    1 \xrightarrow{\sigma}
    (X \to \bC) \xrightarrow{\eta_u}
    (\Var \to \bN) \xrightarrow{\deno{\neww}}
    \bN \xrightarrow{t_{|u|}}
    \bC
    \]
  is not equal to $\bot$.
  \label{DefDenotationalUConvergence}
\end{definition}

\begin{proposition}
  Let $M\from\com$ be a closed term of \IAX, let $u\in X^*$ be a sequence and suppose that
  \[
    \blank,()\ts M\converges_u\skipp,()\,.
    \]
  Let the denotation $\deno{M} \from 1 \to \com$ in $\G_X$ be considered as a morphism $1 \to (X \to \bC)$ in $\G$.
  Then $\deno{M}\downarrow_u$; i.e., the composite
  \[
    1 \xrightarrow{\deno{M}}
    (X \to \bC) \xrightarrow{\eta_u}
    (\Var \to \bN) \xrightarrow{\deno{\neww}}
    \bN \xrightarrow{t_{|u|}}
    \bC
    \]
  is not equal to $\bot$.
  \label{PropKleisliSoundness}
\end{proposition}
\begin{proof}
  Since the $\beta$ rule is valid in $\G_X$, this composite is equal to the denotation of the term
  \[
    \test_{|u|}(\new (\lambda v.M[v\gets \suc !v;\tr_{u^\top} !v/\ask_X];!v))
    \]
  in IA.
  By the adequacy result for Idealized Algol, it suffices to show that this term converges to $\skipp$; i.e., that the term
  \[
    \new (\lambda v.M[v\gets\suc !v;\tr_{u^\top} !v/\ask_X];!v)
    \]
  converges to $|u|$ in IA.
  
  We can prove this using the derivation tree in Figure \ref{FigKleisliSoundnessDerivation}.\qedhere
  \begin{SidewaysFigure}
  \end{SidewaysFigure}
\end{proof}

\begin{remark}
  Definition \ref{DefDenotationalUConvergence} looks a bit odd.  
  This is a result of working at such a high level of generality: since we have not assumed much about $\G$ beyond the fact that it is a suitable model of Idealized Algol, then we have to define everything in terms of Idealized Algol denotations.

  If $\G$ is the category of games and visible strategies, then the statements of Proposition \ref{PropKleisliSoundness} (and our later Adequacy and Full Abstraction results) become clearer.  
  Observe that if $\sigma \from 1 \to (X \to \bC)$ is a strategy in $\G$ (considered as a strategy for $\oc X \implies \bC$, then the maximal plays in the interaction
  \[
    \sigma || (\eta_u;\deno{\neww})
    \]
  take the form
  \[
    \begin{array}{ccc}
      X         & \bC & \bN \\
                &     &  q  \\
                &  q  &     \\
      q         &     &     \\
      u^{(0)}   &     &     \\
      \vdots    &     &     \\
      q         &     &     \\
      u^{(k-1)} &     &     \\
                &  a  &     \\
                &     &  {k\,,}  \\
    \end{array}
    \]
  for $k\le |u|$, where the component in $X,\bC$ is a valid play of $\sigma$.
  Moreover, the strategy $t_n$ is the one with maximal plays of the form
  \[
    \begin{array}{cc}
      \bN & \bC \\
          &  q  \\
      q   &     \\
      n   &     \\
          &  a
    \end{array}
    \]
  or
  \[
    \begin{array}{cc}
      \bN & \bC \\
          &  q  \\
      q   &     \\
      m   &
    \end{array}
    \]
  for $m\ne n$.

  This means that the composite
  \[
    1 \xrightarrow{\sigma}
    (X \to \bC) \xrightarrow{\eta_u}
    (\Var \to \bN) \xrightarrow{\deno{\neww}}
    \bN \xrightarrow{t_|u|}
    \bC
    \]
  is not equal to $\bot$ if and only if $\sigma$ contains the sequence
  \[
    \begin{array}{cc}
      X         & \bC \\
                &     \\
                &  q  \\
      q         &     \\
      u^{(0)}   &     \\
      \vdots    &     \\
      q         &     \\
      u^{(|u|-1)}&     \\
                &  a  \\
                &     \\
    \end{array}\,.
    \]
  Since complete plays in the game $\oc X \implies \bC$ are always of the form $q$, followed by some sequence of pairs of the form $qx_i$ for $x_i\in X$, followed by $a$, it is very natural to consider conditions on those $x_i$ when dealing with a strategy $\sigma\from \oc X \implies \bC$.
\end{remark}

\section{Computational Adequacy}

Now we want to prove Computational Adequacy; i.e., the converse to Proposition \ref{PropKleisliSoundness}.
To do this, we need to prove a converse to Lemma \ref{LemSoundnessMonads}.

First of all, we need to prove a reverse result to Lemma \ref{LemFirstSubstitution} that deals with substitution in the opposite direction.  
That is, instead of telling us what happens when we substitute a term for $\ask_X$, we will look at what happens when we substitute a term for $v\gets \suc !v;\tr_u !v$.  

In most cases, this will not disrupt the structure of the IA rule.  
For instance, we always have
\begin{mathpar}
  (!V)[Q/v\gets \suc !v;\tr_u !v] = !(V[Q/v\gets \suc !v;\tr_u !v])\,,
\end{mathpar}
and so the derivation
\[
  \inferrule*[right={$s'(v)=n$}]{\Gamma,s\ts V[Q/v\gets \suc !v;\tr_u !v] \converges v,s'}
  {\Gamma,s \ts !V[Q/v\gets \suc !v;\tr_u !v] \converges n,s'}
  \]
still follows the pattern of the Idealized Algol rule for variable dereference.

There is only one case where this breaks down.  
Consider the following instance of the sequencing rule.
\[
  \inferrule
  {\Gamma,v,s\ts v\gets \suc !v\converges \skipp,s' \\ \Gamma,v,s'\ts \tr_u !v \converges x,s''}
  {\Gamma,v,s\ts v\gets \suc !v;\tr_u !v \converges x,s''}
  \]
In this case, substituting some term $Q$ for $v\gets \suc !v;\tr_u !v$ in the top two terms will have no effect, whereas it will replace the whole of the bottom with $Q$, invalidating the whole inference.

We have proved the following.

\begin{lemma}
  Let 
  \[
    \inferrule{\Gamma,s^{(0)}\ts M_1\converges c_1,s^{(1)} \\ \cdots \\ \Gamma,s^{(n-1)}\ts M_n\converges c_n,s^{(n)}}
    {\Gamma,s^{(0)} \ts M \converges c,s^{(n)}}
    \]
  be an inference of Idealized Algol.  
  Let $u\in X^*$ and let $Q\from X$ be a term of \IAX.
  Fix an unused variable name $v$ and suppose that $M \ne v\gets \suc !v;\tr_u !v$.
  Then
  \[
    \inferrule{\Gamma,s^{(0)}\ts M_1[Q/v\gets \suc!v;\tr_u !v]\converges c_1,s^{(1)} \\ \cdots \\ \Gamma,s^{(n-1)}\ts M_n[Q/v\gets\suc !v;\tr_u !v]\converges c_n,s^{(n)}}
    {\Gamma,s^{(0)} \ts M[Q/v\gets\suc !v;\tr_u !v] \converges c,s^{(n)}}
    \]
  conforms to the same Idealized Algol pattern.
  In particular, if $w_1,\cdots,w_n\in X^*$, then
  \[
    \inferrule{\Gamma,s^{(0)}\ts M_1[Q/v\gets \suc!v;\tr_u !v]\converges_{w_1} c_1,s^{(1)} \\ \cdots \\ \Gamma,s^{(n-1)}\ts M_n[Q/v\gets\suc !v;\tr_u !v]\converges_{w_n} c_n,s^{(n)}}
    {\Gamma,s^{(0)} \ts M[Q/v\gets\suc !v;\tr_u !v] \converges_{w_1\cat\cdots\cat w_n} c,s^{(n)}}
    \]
  is a valid inference of \IAX.
  \label{LemSecondSubstitution}
\end{lemma}

We need one more lemma to help us deal with substitution.

\begin{lemma}
  Suppose that $\Gamma,y\ts M\from T$ is a typing judgement of Idealized Algol, where $\Gamma$ is a $\Var$-context and $y$ is a free variable of type $X$.  
  Fix $u\in X^*$.  
  Suppose that $M\ne y$ and that we have some inference
  \[
    \inferrule{\Gamma,s^{(0)}\ts N_1\converges c_1,s^{(1)} \\ \cdots \\ \Gamma,s^{(n-1)}\ts N_n\converges c_n,s^{(n)}}
    {\Gamma,s^{(0)} \ts M[v\gets\suc !v;\tr_u !v/y] \converges c,s^{(n)}}\,.
    \]
  of Idealized Algol.
  Then each $N_i$ may be written as $M_i[v\gets \suc !v;\tr_u !v/y]$ for some $\Gamma,y \ts M_i$.
  \label{LemThirdSubstitution}
\end{lemma}
\begin{proof}
  This can be checked case-by-case.  
  The most interesting is the case for sequencing: if $M[v\gets \suc !v;\tr_u !v/y] \ne v\gets \suc !v;\tr_u !v$, then we must have
  \[
    M[v\gets \suc !v;\tr_u !v/y] = N[v\gets \suc !v;\tr_u !v/y];P[v\gets \suc !v;\tr_u !v/y]\,,
    \]
  which is deduced from $N[v\gets \suc !v;\tr_u !v/y]$ and $P[v\gets \suc !v;\tr_u !v/y]$.
\end{proof}

Now we can state and prove our adequacy lemma.

\begin{lemma}
  Suppose that $w\in X^*$ is a sequence of length greater than or equal to $k,l$ and that
  \[
    \Gamma,v,(s|v\mapsto k) \ts M[v\gets\suc !v;\tr_w !v/y] \converges c,(s'|v\mapsto l)
    \]
  is derivable in Idealized Algol, where $v$ is not free in $M$ and $y$ is a variable name of type $X$.
  Then $l\ge k$ and
  \[
    \Gamma,s \ts M[\ask_X/y] \converges_u c,s'
    \]
  in \IAX, where $u$ is the subsequence of $w$ consisting of all terms from $k+1$ up to $l$.
  \label{LemAdequacyMonads}
\end{lemma}
\begin{proof}
  Induction on the derivation.

  Suppose that $M\ne y$.  
  Then, by Lemma \ref{LemThirdSubstitution}, the last step in the derivation of $M[v\gets \suc !v;\tr_w !v/y]$ must be of the form
  \[
    \inferrule{\Gamma,s^{(0)}\ts M_1[v\gets \suc!v;\tr_w !v/y]\converges c_1,s^{(1)} \\ \cdots \\ \Gamma,s^{(n-1)}\ts M_n[v\gets\suc !v;\tr_w !v/y]\converges c_n,s^{(n)}}
    {\Gamma,s^{(0)} \ts M[v\gets\suc !v;\tr_w !v/y] \converges c,s^{(n)}}\,,
    \]
  where each $M_i[v\gets \suc !v;\tr_w !v/y]$ is derivable in Idealized Algol.

  By the inductive hypothesis, $s^{(i-1)}(v) \le s^{(i)}(v)$ for each $i$ and so $s^{(0)}(v) \le s^{(n)}(v)$, as desired (in the case that there are no premises -- i.e., the case of the rule for canonical forms -- we have $s^{(0)}(v)=s^{(0)}(v)$).
  Moreover, by the inductive hypothesis, it is derivable that
  \[
    \Gamma,s^{(i-1)} \ts M_i[\ask_X/y] \converges_{u_i} c_i,s^{(i)}\,,
    \]
  where $u_i$ is the subsequence of $w$ going from term $s^{(i-1)}(v)+1$ up to $s^{(i)}(v)$.

  Now for any term $\Gamma,y\ts P$, we have
  \[
    P[\ask_X/y] = P[v\gets \suc !v;\tr_u !v/y][\ask_X/v\gets \suc !v;\tr_u !v]\,,
    \]
  and so by Lemma \ref{LemSecondSubstitution} we may derive
  \[
    \Gamma,s^{(0)} \ts M[\ask_X/y]\converges_{u_1\cat \cdots \cat u_n} c,s^{(n)}\,.
    \]
  But $u_1\cat \cdots \cat u_n$ is precisely the subsequence of $w$ going from term $s^{(0)}(v)+1$ up to $s^{(n)}(v)$!

  This completes the first case.  
  The second case is where $M=y$.  
  Suppose, then, that
  \[
    \Gamma,v,(s\vert v\mapsto k) \ts v\gets \suc !v;\tr_w !v \converges x,(s'\vert v\mapsto l)
    \]
  is derivable in Idealized Algol.

  Since IA is a deterministic language (so if $\Gamma,s\ts M\converges c,s'$ and $\Gamma,s\ts M \converges c',s''$ then $c=c'$ and $s'=s''$), then the derivation of this term must agree with the valid IA derivation given in Figure \ref{FigTheOneForTheLemma}.  
  It follows that $l=k+1$ and that $x$ is the $(k+1)$-th term of $w$, so the single-term sequence $x$ is the subsequence of $w$ going from $k+1$ to $l$.
  
  Then we have the derivation
  \[
    \inferrule{ }{\Gamma,s \ts \ask_X \converges_x x,s'}
    \]
  in \IAX.
  This completes the induction.
\end{proof}

We can now prove computational adequacy for our model.
We have proved one direction already in Proposition \ref{PropKleisliSoundness}, so it suffices to prove the other direction.

\begin{proposition}[Computational adequacy]
  Let $M\from \com$ be a closed term of \IAX.  
  Consider the denotation $\deno M\from 1 \to \bC$ in $\G_X$ as a morphism $1 \to (X \to \bC)$ in $\G$.  
  Let $u\in X^*$ be a sequence and suppose that $\deno M \downarrow_u$; i.e., that the composite
  \[
    1 \xrightarrow{\deno{M}}
    (X \to \bC) \xrightarrow{\eta_u}
    (\Var \to \bN) \xrightarrow{\deno{\neww}}
    \bN \xrightarrow{t_{|u|}}
    \bC
    \]
  is not equal to $\bot$.  
  Then
  \[
    \blank,()\ts M\converges_u \skipp,()\,.
    \]
  \label{PropKleisliAdequacy}
\end{proposition}
\begin{proof}
  As before, the composite given in the statement is the denotation of the term
  \[
    \test_{|u|} (\new (\lambda v.M[v\gets \suc !v;\tr_{u^\top} !v/\ask_X];!v))\,.
    \]
  By the adequacy result for Idealized Algol, the fact that this denotation is not equal to $\bot$ means that the term converges to $\skipp$, from which we can deduce that
  \[
    \new (\lambda v.M[v\gets \suc !v;\tr_{u^\top} !v/\ask_X];!v
    \]
  converges to $|u|$.

  It is easy to see that this is equivalent to asking whether we can derive the following relation in Idealized Algol.
  \[
    v,(v\mapsto 0) \ts M[v\gets \suc !v;\tr_{u^\top} !v/\ask_X] \converges \skipp,(v\mapsto |u|)
    \]
  Now $u$ is the subsequence of $u^\top$ going from position $1$ to position $|u|$.  
  So Lemma \ref{LemAdequacyMonads} tells us that we must have
  \[
    \blank,()\ts M \converges_u \skipp,()
    \]
  in \IAX.
\end{proof}

\begin{remark}
  Recall that if $\G$ is the category of games, then we have $\sigma\downarrow_u$ if and only if $\sigma$ contains the play 
  \[
    \begin{array}{cc}
      X         & \bC \\
                &     \\
                &  q  \\
      q         &     \\
      u^{(0)}   &     \\
      \vdots    &     \\
      q         &     \\
      u^{(|u|-1)}&     \\
                &  a  \\
                &     \\
    \end{array}\,.
    \]
  Therefore, we have now proved that we have $\blank,() \ts M \converges_u \skipp,()$ if and only if $\deno{M}$ contains that sequence.
\end{remark}

\section{Full Abstraction}

To prove full abstraction of our semantics for \IAX, we introduce the usual intrinsic equivalence on terms.

\begin{definition}
  Let $\sigma,\tau\from A \to B$ be morphisms in $\G_X$.  
  By currying, we may consider $A$ and $B$ as morphisms $1 \to (A \to B)$ in $\G_X$.
  We say that $\sigma\sim\tau$ if for all morphisms $\alpha\from (A \to B) \to \bC$ and for all sequences $u\in X^*$, if we regard $\sigma;\alpha,\tau;\alpha\from 1 \to \bC$ as morphisms $1 \to (X \to \bC)$ in $\G$, then the composites
  \begin{mathpar}
    1 \xrightarrow{\sigma;\alpha}
    (X \to \bC) \xrightarrow{\eta_u}
    (\Var \to \bN) \xrightarrow{\deno{\neww}}
    \bN \xrightarrow{t_{|u|}}
    \bC
    \and
    1 \xrightarrow{\tau;\alpha}
    (X \to \bC) \xrightarrow{\eta_u}
    (\Var \to \bN) \xrightarrow{\deno{\neww}}
    \bN \xrightarrow{t_{|u|}}
    \bC
  \end{mathpar}
  are equal.
\end{definition}

\begin{theorem}[Full abstraction]
  Let $M,N\from T$ be closed terms of \IAX.  
  Then $M,N$ are observationally equivalent -- i.e., for all contexts $C[-]\from\com$ of \IAX with a hole of type $T$ and for all sequences $u\in X^*$, 
  \[
    \blank,()\ts C[M] \converges_u \skipp,() \Longleftrightarrow \blank,()\ts C[N] \converges_u \skipp,()\,\text{--}
    \]
  if and only if $\deno{M}\sim\deno{N}$.
  \label{TheKleisliFullAbstraction}
\end{theorem}
\begin{proof}
  First, suppose that $\deno{M}\sim\deno{N}$.  
  Let $C[-]\from\com$ be a context with a hole of type $T$.  
  Then the denotation of $t\ts C[t]$ is a morphism $\alpha\from \deno{T} \to \bC$.
  Moreover, the denotation of $C[M]$ is the composite $\deno{M};\alpha$ and that of $C[N]$ is the composite $\deno{N};\alpha$ since we are working in a Cartesian closed category.

  Then the composites
  \begin{mathpar}
    1 \xrightarrow{\sigma;\alpha}
    (X \to \bC) \xrightarrow{\eta_u}
    (\Var \to \bN) \xrightarrow{\deno{\neww}}
    \bN \xrightarrow{t_{|u|}}
    \bC
    \and
    1 \xrightarrow{\tau;\alpha}
    (X \to \bC) \xrightarrow{\eta_u}
    (\Var \to \bN) \xrightarrow{\deno{\neww}}
    \bN \xrightarrow{t_{|u|}}
    \bC
  \end{mathpar}
  are equal, so $C[M]\converges_u\skipp$ if and only if $C[N]\converges_u\skipp$ by Propositions \ref{PropKleisliSoundness} and \ref{PropKleisliAdequacy}.

  Conversely, suppose that $M\not\sim N$.  
  So there is some $\alpha\from \deno T \to \bC$ in $\G_X$ and some sequence $u$ such that (without loss of generality),
  \begin{mathpar}
    (\deno M;\alpha);\eta_u;\deno{\neww};t_{|u|} \ne \bot
    \and
    (\deno N;\alpha);\eta_u;\deno{\neww};t_{|u|} = \bot\,.
  \end{mathpar}
  Here, we have enclosed $\deno M;\alpha$ and $\deno N;\alpha$ in brackets to indicate that the composition is taken in the Kleisli category $\G_X$, and then the whole thing is considered as a morphism $1 \to (X \to \bC)$ in $\G$.

  More specifically, these composites are given by the composites
  \begin{mathpar}
    1 \xrightarrow{\deno M}
    (X \to \deno T) \xrightarrow{X \to \alpha}
    (X \to (X \to \bC)) \xrightarrow{\mu}
    (X \to \bC)
    \and
    1 \xrightarrow{\deno N}
    (X \to \deno T) \xrightarrow{X \to \alpha}
    (X \to (X \to \bC)) \xrightarrow{\mu}
    (X \to \bC)
  \end{mathpar} 
  in $\G$, where $\mu$ indicates precomposition with the diagonal.

  Now $\alpha$ is the least upper bound of its compact approximants, so it follows that there is some compact $\alpha'\subseteq \alpha$ such that
  \begin{mathpar}
    \deno M;(X \to \alpha');\mu;\eta_u;\deno{\neww};t_{|u|} \ne \bot
    \and
    \deno N;(X \to \alpha');\mu;\eta_u;\deno{\neww};t_{|u|} = \bot\,.
  \end{mathpar}
  Then, by compact definability in $\G$, $\alpha'$ is the denotation of some IA term $x\from T \ts C[x]\from X \to \com$, which is therefore the denotation of the term $x\from T \ts C[x]\,\ask_X \from \com$ in $\G_X$.
  So we get
  \begin{mathpar}
    \deno{C[M]};\eta_u;\deno{\neww};t_{|u|} \ne \bot
    \and
    \deno{C[N]};\eta_u;\deno{\neww};t_{|u|} = \bot\,,
  \end{mathpar}
  and so $C[M]\converges_u\skipp$ by Proposition \ref{PropKleisliAdequacy}, while $C[N] \not\converges_u \skipp$ by Proposition \ref{PropKleisliSoundness}.
  Therefore, $M$ and $N$ are observationally inequivalent in \IAX.
\end{proof}

\section{Comparison with Ghica's slot games}

For this section, let us suppose that $\G$ is the category of games and visible strategies, and that $X=\bC$.  
As we remarked above, this means that \IAX can be interpreted as a language for modelling time complexity.

We compare our approach to a different one, due to Dan Ghica \cite{SlotGames}.  
Given a game $A$, Ghica defines a \emph{play with costs} in $A$ to be a justified sequence $s\in (M_A + \{\slot\})^*$ such that $s\vert_{M_A}\in P_A$.
Here, $\slot$ is a special symbol called a \emph{slot} or \emph{token-action}, which can be interleaved throughout the play $s\vert_{M_A}$ from $A$.  
We shall also impose the requirement (not found in \cite{SlotGames}) that an occurrence of the special symbol $\slot$ must take place either after an $O$-move in $A$ or after another instance of $\slot$.
The token actions do not carry justification pointers.

Following Ghica, we define a \emph{strategy with costs} to be a prefix-closed set $\sigma$ of plays with costs such that the set $\sigma\vert_{M_A} = \{s\vert_{M_A}\suchthat s\in \sigma\}$ is a valid visible strategy for $A$ and such that determinism applies to token actions as well as ordinary moves -- i.e., if $s$ is an $O$-play with costs in $A$ such that $s\slot a\in\sigma$ for some ordinary move $a$, then $sb\not\in\sigma$ for all ordinary moves $b$.

The identity strategy with costs is the usual identity strategy, without any token actions.  
We can similarly define an \emph{interaction sequence with costs} to be a sequence $\s\in(M_A+M_B+M_C+\{\slot\})^*$ such that $\s\vert_{A,B,C}$ is an interaction sequence for $A,B,C$ and such that each occurrence of the token action $*$ occurs after a move that could be considered as an $O$-move in either $A\implies B$ or $B\implies C$.

Given such an interaction sequence $\s\in(M_A+M_B+M_C+\{\slot\})^*$ between the games $A$, $B$ and $C$, write $\s\vert_{A,B}$ for the subsequence consisting of all those moves in $A$ and $B$, together with all token actions such that the previous move was an $O$-move in $A\implies B$.  
Define $\s\vert_{B,C}$ similarly.  
Then if $\sigma\from A \implies B$ and $\tau\from B \implies C$ are strategies with costs, we define $\sigma\|\tau$ to be the set of all such sequences $\s$ such that $\s\vert_{A,B}\in\sigma$ and $\s\vert_{B,C}\in\tau$.  
Lastly, we define $\sigma;\tau$ to be the set of all sequences obtained by taking a sequence $\s\in\sigma\|\tau$ and removing all the moves in $B$ (but retaining all the token actions, including those that arise between moves in $B$).  
The usual arguments apply to show that this is indeed a category.

This seems like a purely combinatorial construction, but it can actually be subsumed into our category-theoretic apparatus.

\begin{proposition}
  Let $A$ be a game.  
  Then there is a bijection
  \[
    c \from \{\text{normal strategies for $\oc\bC \implies A$}\}\leftrightarrow\{\text{strategies with costs for $A$}\}
    \]
  Moreover, this bijection respects composition and identity in the Kleisli category: let $\sigma \from\oc\bC \implies (A \to B)$ and $\tau \from \oc\bC \implies (B \to C)$ be strategies.  
  Write $\sigma;\tau$ for the Kleisli composition of $\sigma$ and $\tau$ in $\G_{\bC}$; i.e., the composite
  \[
    \oc\bC \xrightarrow{\mu}
    \oc\bC \tensor \oc\bC \xrightarrow{\sigma\tensor\tau}
    (A \implies B) \tensor (B \implies C) \xrightarrow{;}
    (A \implies C)\,.
    \]
  Then $c(\sigma;\tau)=c(\sigma);c(\tau)$.  
  Moreover, $c(\id_A)$ is the identity in the category of games and strategies with costs.
\end{proposition}
\begin{proof}
  The map $c$ is the unique functor given by the functional completeness theorem (Proposition \ref{pKleisli}) that sends the canonical strategy $\phi \from 1 \to \bC$ to the strategy with costs for $\bC$ with maximal play
  \[
    q\slot a\,.
    \]
  More synthetically, we get from a strategy for $\sigma\from\oc \bC \implies A$ to a strategy with costs for $A$ by replacing each occurrence of the pair $qa$ occurring in the $\bC$ component with the token action $\slot$ in each play of $\sigma$.

  This functor is the identity on objects, and it is fully faithful, since it has an obvious inverse, given by taking a strategy with costs and replacing each occurrence of the token action with a pair of moves $qa$ in $\oc\bC$, where $q$ justifies $a$ and is itself justified by the most recently occurring $O$-move in $A$.
  Since each token action must always occur after an opponent move or after another token action, and since player $O$ has no reply to the move $q$ other than the move $a$, this always gives us a legal strategy.
\end{proof}

Therefore, we see that the category of games and strategies with costs is isomorphic to the Kleisli category $\G_\bC$, which we have already shown to be fully abstract for a language with time complexity.

\section{Alternative Reduction Rules - May Testing}
\label{SecMayTesting}

We remarked above that if $X\in\{\bB,\bN\}$, then \IAX is a model of nondeterminism, finite in the case of $\bB$ and countable in the case of $\bN$.  
However, we have given a rather strange operational semantics for these languages (the relation $\converges_u$) that does not accurately reflect how they are normally used.

For example, the terms
\begin{mathpar}
  \If \ask_\bB \Then \true \Else \false\from \bool
  \and
  \If \ask_\bB \Then \false \Else \true\from \bool
\end{mathpar}
are not observationally equivalent in our operational semantics; indeed, we have
\begin{mathpar}
  \If \ask_\bB \Then \true \Else \false \converges_{\true} \true
  \and
  \If \ask_\bB \Then \false \Else \true \converges_{\true} \false\,.
\end{mathpar}

However, these terms (which both nondeterministically choose either the true or the false value), \emph{should} be observationally equivalent in any sensible nondeterministic semantics.  
The issue is the labelling on the reduction relations, which is saving too much information about the reduction of the term.
Indeed, this is sort of the point of nondeterminism: we should be able to make nondeterministic choices without recording which value we used for that choice.

\begin{definition}[\cite{mcCHFiniteND}]
  If $X\in\{\bB,\bN\}$, we define an operational relation $\converges$ (`may converge') on the language \IAX as follows.  
  The rules for $\converges$ are identical to the operational rules for Idealized Algol, with the addition of the following rule for the primitive $\ask_X$.
  \[
    \inferrule*[right=$x\in X$]{ }{\Gamma,s\ts \ask_X \converges x,s}
    \]
\end{definition}
These rules are exactly the same as our original operational semantics for \IAX, but with the sequences $u$ removed.  
Moreover, if we have a valid derivation of $\Gamma,s\ts M\converges u,s'$, then it is clear (by induction) that we may annotate all the occurrences of $\converges$ with suitable sequences in order to obtain a derivation of $\Gamma,s\ts M\converges_u c,s'$ for some $u\in X^*$.
So we get the following alternative definition of may convergence.
\begin{definition}
  We say that $\Gamma,s\ts M\converges c,s'$ if there is some $u\in X^*$ such that $\Gamma,s\ts M\converges_u c,s'$.
\end{definition}

We can reflect this operational relation in the semantics by modifying the definition of intrinsic equivalence.  

Firstly, an obvious consequence of Propositions \ref{PropKleisliSoundness} and \ref{PropKleisliAdequacy} is that

\begin{corollary}[Computational Adequacy for May Testing]
  Let $M\from\com$ be a closed term of \IAX.  
  Consider the denotation $\deno M\from 1 \to \bC$ in $\G_X$ as a morphism $1 \to (X \to \bC)$ in $\G$.  
  Then there exists some sequence $u\in X^*$ such that the composite
  \[
    1 \xrightarrow{\deno M}
    (X \to \bC) \xrightarrow{\eta_u}
    (\Var \to \bN) \xrightarrow{\deno{\neww}}
    \bN \xrightarrow{t_{|u|}}
    \bC
    \]
  is not equal to $\bot$, if and only if
  \[
    \blank,()\ts M \converges \skipp,()\,.
    \]
  \label{CorMayAdequacy}
\end{corollary}

In light of this result, we can define a new intrinsic equivalence on morphisms in $\G_X$:

\begin{definition}
  Let $\sigma,\tau\from A \to B$ be morphisms in $\G_X$, considered as morphisms $1 \to (A \to B)$ in $\G_X$.  
  We say that $\sigma \sim_{\may} \tau$ if for all $u\in X^*$ there exists $v\in X^*$ such that the composites
  \begin{mathpar}
    1 \xrightarrow{\sigma;\alpha}
    (X \to \bC) \xrightarrow{\eta_u}
    (\Var \to \bN) \xrightarrow{\deno{\neww}}
    \bN \xrightarrow{t_{|u|}}
    \bC
    \and
    1 \xrightarrow{\tau;\alpha}
    (X \to \bC) \xrightarrow{\eta_v}
    (\Var \to \bN) \xrightarrow{\deno{\neww}}
    \bN \xrightarrow{t_{|v|}}
    \bC
  \end{mathpar}
  are equal, and vice versa.
  \label{DefMayIntrinsicEquivalence}
\end{definition}

Then exactly the same argument as before gives us a full abstraction result for may-equivalence.

\begin{theorem}
  Let $M,N\from T$ be closed terms of \IAX. Then $M,N$ are may-observationally equivalent -- i.e., for all contexts $C[-]\from \com$ of \IAX with a hole of type $T$, \[
    \blank,()\ts C[M] \converges \skipp,() \Leftrightarrow \blank,()\ts C[N] \converges \skipp,()\,\text{--}
    \]
  if and only if $\deno M \sim_{\may} \deno N$.
\end{theorem}
\begin{proof}
  Suppose that $\deno M\sim_{\may}\deno N$.  
  Let $C[-]$ be a context of \IAX, interpreted as a morphism $\alpha \from \deno T \to \bC$.

  Suppose that $\blank,()\ts C[M] \converges \skipp,()$.  
  So there is some sequence $u\in X^*$ such that $\blank,()\ts C[M] \converges_u \skipp,()$ and therefore the composite
  \[
    1 \xrightarrow{\deno M;\alpha}
    (X \to \bC) \xrightarrow{\eta_u}
    (\Var \to \bN) \xrightarrow{\deno{\neww}}
    \bN \xrightarrow{t_{|u|}}
    \bC
    \]
  is not equal to $\bot$.  
  This means that there exists some $v\in X^*$ such that the composite
  \[
    1 \xrightarrow{\deno N;\alpha}
    (X \to \bC) \xrightarrow{\eta_v}
    (\Var \to \bN) \xrightarrow{\deno{\neww}}
    \bN \xrightarrow{t_{|v|}}
    \bC
    \]
  Therefore, $\blank,()\ts C[N] \converges_v \skipp,()$, and so $\blank,()\ts C[N] \converges \skipp,()$.
  The reverse direction is identical.

  Conversely, suppose that $M,N$ are may-observationally equivalent.  
  Then, as before, we can take $\alpha$ to be compact, whence definable, in Definition \ref{DefMayIntrinsicEquivalence}, and the proof continues as in the first part, but in reverse.
\end{proof}

Let us examine what this means in the category of games.
If $\sigma\from \oc X \implies \bC$ is a strategy, then, by our discussion at the end of \sec\ref{SecSoundness}, we know that, for any sequence $u$, the composite
\[
  1 \xrightarrow{\sigma}
  (X \to \bC) \xrightarrow{\eta_u}
  (\Var \to \bN) \xrightarrow{\deno\neww}
  \bN \xrightarrow{t_{|u|}}
  \bC
  \]
is not equal to $\bot$ if and only if $\sigma$ contains the play
\[
  \begin{array}{cc}
    X         & \bC \\
              &     \\
              &  q  \\
    q         &     \\
    u^{(0)}   &     \\
    \vdots    &     \\
    q         &     \\
    u^{(|u|-1)}&     \\
              &  a  \\
              &     \\
  \end{array}\,.
  \]
Moreover, since any complete play in $\oc X \implies \bC$ must take this form, we can see there exists such a $u$ making the composite above not equal to $\bot$ if and only if
\[
  qa \in \{s\vert_\bC \suchthat s\in\sigma\text{ is complete}\}\,.
  \]
This suggests a general equivalence relation on Kleisli morphisms in $\G_X$: given a strategy $\sigma\from \oc X \implies A$, we write $\sigma\vert_A$ for the set
\[
  \{s\vert_A \suchthat s\in \sigma\text{ is complete}\}\,.
  \]
We say that two strategies $\sigma,\tau\from \oc X \implies A$ are may-equivalent, and write $\sigma\approx_{\may}\tau$, if
\[
  \sigma\vert_A = \tau\vert_A\,.
  \]
In this case, Corollary \ref{CorMayAdequacy} tells us that $M\converges \skipp$ if and only if $\sigma\approx_{\may}\tau$.

We need to show that this respects composition, so that we get a category if we take the quotient by this equivalence relation.
\begin{proposition}
  Let $\sigma,\sigma' \from A \to B$, $\tau,\tau'\from B \to C$ be morphisms in $\G_X$.
  Suppose that $\sigma\approx_{\may}\sigma'$ and $\tau\approx_{\may}\tau'$.  
  Then $\sigma;\tau \approx_{\may} \sigma';\tau'$.
\end{proposition}
\begin{proof}
  A complete play in $\sigma;\tau$ is given by a sequence $\s\vert_{X,A,C}$, where $\s\in (M_X + M_A + M_B + M_C)^*$ is a legal interaction of a complete play in $\tau$ with a collection of complete plays in $\sigma$, $B$-components being identified.
  Then $\s\vert_{A,C}$ can alternatively be characterized as $\t\vert_{A,C}$, where $\t\in (M_A + M_B + M_C)^*$ is a legal interaction of a sequence from $\tau\vert_{B,C}$ with a collection of sequences from $\sigma\vert_{A,B}$.  

  It follows that if $\sigma\vert_{A,B}=\sigma'\vert_{A,B}$ and $\tau\vert_{B,C}=\tau'\vert_{B,C}$, then $\sigma;\tau\vert_{A,C} = \sigma';\tau'\vert_{A,C}$.
\end{proof}

Now we can also see that this equivalence we have just defined is subsumed into the intrinsic equivalence.
\begin{proposition}
  Let $\sigma,\tau\from \oc X \implies A$ be strategies.  
  If $\sigma\approx_{\may}\tau$ then $\sigma\sim_{\may} \tau$.
\end{proposition}
\begin{proof}
  Given strategies $\sigma,\tau\from A$ in $\G_X$, we have $\sigma \sim_{\may} \tau$ if and only if $\sigma;\alpha\approx_{\may}\tau;\alpha$ for any morphism $\alpha\from \oc A \implies \bC$ in $\G_X$.
  If $\sigma\approx_{\may}\tau$, we have
  \[
    \sigma;\alpha\vert_{\bC} = (\sigma\vert_{\bC});\alpha = (\tau\vert_{\bC});\alpha = \tau;\alpha\vert_{\bC}\,.\qedhere
    \]
\end{proof}

Note that it is also the case that if $\alpha\approx_{\may}\alpha'$ and $\sigma\approx_{\may}\tau$ then $\sigma;\alpha \approx_{\may} \tau;\alpha'$.  
Therefore, the Full Abstraction result we have just proved applies to the quotiented category.

The definition of the relation $\approx_{\may}$ suggests that we might forget about the $\oc X$ component of a strategy $\sigma\from\oc X \implies A$ altogether, and consider only the set $\sigma\vert_A$.  
This set is not a strategy, since it does not satisfy the determinism requirement, but it satisfies every other requirement.

\begin{definition}
  Given a game $A$, a \emph{nondeterministic strategy} is a prefix-closed set of even-length legal plays from $A$.
\end{definition}

We can compose nondeterministic strategies using `parallel composition plus hiding', just as for deterministic ones, and we get a Cartesian closed category in the same way.  
We interpret all the Idealized Algol terms in the usual way as deterministic strategies, interpreting the nondeterministic primitive $\ask_X$ as the nondeterministic strategy for $X$ with maximal plays
\[
  qx
  \]
for every $x\in X$.

Now, if we have a Kleisli morphism $\sigma\from \oc X \implies A$, then $\sigma\vert_A$ is a nondeterministic strategy for $A$.  
The composition of nondeterministic strategies respects the composition and identity in the Kleisli category; moreover, any two Kleisli morphisms that give rise to the same nondeterministic strategy are intrinsically equivalent.

It is already known (see \cite{mcCHFiniteND}) that this model is fully abstract for (finitely or countably) nondeterministic Idealized Algol with may-contextual equivalence.
We have just provided an alternative proof of this fact.

\section{Alternative Reduction Rules - Must Testing}

A more interesting, and more complicated, reduction rule for nondeterministic IA is the \emph{must-convergence} relation.

We shall define this indirectly via its negation.

\begin{definition}
  We define a relation
  \[
    \Gamma,s\ts M\diverges
    \]
  ($M$ `may diverge') between $\Var$-contexts $\Gamma$, $\Gamma$-stores $s$ and terms $\Gamma\ts M\from T$ of Idealized Algol.

  The rules for this relation may be deduced from the rules for Idealized Algol.  
  If
  \[
    \inferrule{\Gamma,s^{(0)}\ts M_1\converges c_1,s^{(1)} \\ \cdots \\ \Gamma,s^{(n-1)}\ts M_n\converges c_n,s^{(n)}}
    {\Gamma,s^{(0)} \ts M \converges c,s^{(n)}}\,,
    \]
  is an IA rule, then for each $j=1,\cdots,n$ we have a rule
  \[
    \inferrule*[sep=1.9em]{\Gamma,s^{(0)}\ts M_1 \converges c_1,s^{(1)} \and \cdots \\ \Gamma,s^{(j-2)} \ts M_{j-1} \converges c_{j-1},s^{(j-1)} \\ \Gamma,s^{(j-1)}\ts M_j\diverges}
    {\Gamma,s^{(0)} \ts M \diverges}\,.
    \]
  In other words, if the evaluation of the term $M$ diverges at any step, then the whole thing diverges.  
  Note that since the rules for the canonical forms have no premises, there are no rules relating them to the $\diverges$ predicate.

  Crucially, these rules are interpreted not inductively but coinductively; in other words, the derivation trees for $\diverges$ are allowed to have infinite height.
  Indeed, since every rule for $\diverges$ has at least one premise, any valid tree must be infinitely tall.

  If there is no such derivation tree, then we say that $M$ \emph{must converge}, and write $\Gamma,s\ts M\mustconverge$.
  \label{DefMayDivergence}
\end{definition}

\begin{example}
  If $\Gamma\ts c$ is a canonical form, then $\Gamma\ts c\mustconverge$, since there is no inference that we can apply that will yield the term $\Gamma\ts c\diverges$ on the bottom.
\end{example}

\begin{example}
  $\blank,()\ts \Y(\lambda x.x)\diverges$ because it has the following infinite derivation tree.
  \[
    \inferrule*
    {
      \inferrule*
      {
        \inferrule*
        {
        }
        {
          \blank,()\ts (\lambda x.x) \converges (\lambda x.x),()
        }
        \\
        \inferrule*
        {
          \vdots
        }
        {
          \blank,() \ts \Y(\lambda x.x)\diverges
        }
      }
      {
        \blank,()\ts (\lambda x.x)(\Y(\lambda x.x)) \diverges
      }
    }
    {
      \blank,()\ts \Y(\lambda x.x)\diverges
    }
    \]
\end{example}

\begin{remark}
It is possible to characterize the $\mustconverge$ predicate directly via an inductive relation (see \cite{RusssThesis}, for instance).  
\end{remark}

We will next give an alternative definition of $\mustconverge$ that will relate it to our existing rules $\converges_u$.

\begin{definition}
  We say that $\Gamma,s\ts M\mustconverge$ if for every infinite sequence $w\in X^\omega$ there is some finite prefix $u\pprefix w$ such that $\Gamma,s\ts M\converges_u c,s'$ for some canonical form $c$ and some $\Gamma$-store $s'$.
  \label{DefMustConvergence}
\end{definition}

Before we show that the two definitions are equivalent, we shall examine why this one makes sense.  
Recall that we said that the statement $\Gamma,s\ts M\converges_u c,s'$ meant `$M$ will converge to $c$ in the case that the values we obtain by querying the nondeterministic oracle form the sequence $u$'.  
Our initial thought, then, might be to declare that $\Gamma,s\ts M\mustconverge$ if for all $u\in X^*$, $\Gamma,s\ts M\converges_u c,s'$ for some $c,s'$.  
However, this is not the case even for terms that clearly have to converge.
For example, if we have
\[
  M = \If \ask_\bB \Then \true \Else \false\,,
  \]
then $M\converges_\true \true$ and $M\converges_\false \false$, and it is certainly true that $M\mustconverge$.
However, it is not the case that, for example, $M\converges_\epsilon c$; nor is it the case that $M\converges_{\false\false\true\false\true}c$ for any $c$.
In the first case, the sequence is too short: we can extend it to the sequence $\true$ and then we get convergence.  
In the second case, the sequence is too long: we have already converged after the first term $\false$, so we never get the chance to read any other terms.

The idea is that if we fix beforehand an infinite sequence made up of all the values that the nondeterministic oracle could possibly give, then we know that the term will always end up reading some finite sequence of values from the beginning of the sequence and then terminating at a value.

\begin{proposition}
  The two definitions of the $\mustconverge$ predicate given in Definitions \ref{DefMayDivergence} and \ref{DefMustConvergence} are equivalent.
\end{proposition}
\begin{proof}
  Suppose that $\Gamma,s\ts M\diverges$.  
  We coinductively construct a sequence $w\in X^\omega$ such that $\Gamma,s\ts M\not\converges_u c,s'$ for any finite prefix $u\pprefix w$ and any $c,s'$.
  We shall use the notation $\Gamma,s\ts M\diverges^w$ to indicate this $w$.

  Given any rule
  \[
    \inferrule{\Gamma,s^{(0)}\ts M_1 \converges c_1,s^{(1)} \\ \cdots \\ \Gamma,s^{(j-2)} \ts M_{j-1} \converges c_{j-1} \\ \Gamma,s^{(j-1)}\ts M_j\diverges}
    {\Gamma,s^{(0)} \ts M \diverges}\,,
    \]
  by our definition of $\converges$, there are $u_1,\cdots,u_{j-1}\in X^*$ such that $\Gamma,s^{(i-1)}\ts M_i\converges_{u_i}c_i,s^{(i)}$ for each $i=1,\cdots,j-1$.  

  The corresponding coinductive rule is then
  \[
    \inferrule*[sep=8pt]{\Gamma,s^{(0)}\ts M_1 \converges_{u_1} c_1,s^{(1)} \\ \cdots \\ \Gamma,s^{(j-2)} \ts M_{j-1} \converges_{u_{j-1}} c_{j-1} \\ \Gamma,s^{(j-1)}\ts M_j\diverges^w}
    {\Gamma,s^{(0)} \ts M \diverges^{u_1\cat \cdots \cat u_{j-1}\cat w}}\,.
    \]
  In other words, the sequence $w$ at the bottom of the tree may be characterized as the concatenation of all the sequences $u$ constructed for the $\converges$ relation in the tree, working bottom to top and left to right.
  If the $w$ so formed is a finite sequence, then pad it with arbitrary values so that it becomes infinite.

  Now suppose that $u\pprefix w$ is some finite prefix such that $\Gamma,s\ts M\converges_u c,s'$ for some $c,s'$.  
  We claim that $\Gamma,s\ts M\not\diverges$, giving us a contradiction.  

  The proof of the claim is via induction on the derivation of $\Gamma,s\ts M\converges_u c,s'$.  

  Suppose that the last rule takes the form
  \[
    \inferrule*{\Gamma,s^{(0)}\ts M_1\converges_{u_1} c_1,s^{(1)} \\ \cdots \\ \Gamma,s^{(n-1)}\ts M_n\converges_{u_n} c_n,s^{(n)}}
    {\Gamma,s^{(0)} \ts M \converges_{u_1\cat \cdots\cat u_n} c,s^{(n)}}\,.
    \]
  Then inspection of the IA rules tells us that the last step in our derivation of $\Gamma,s^{(0)} \ts M\diverges$ must look like
  \[
    \inferrule*[sep=8pt]{\Gamma,s^{(0)}\ts M_1 \converges_{u_1} c_1,s^{(1)} \\ \cdots \\ \Gamma,s^{(j-2)} \ts M_{j-1} \converges_{u_{j-1}} c_{j-1} \\ \Gamma,s^{(j-1)}\ts M_j\diverges^w}
    {\Gamma,s^{(0)} \ts M \diverges^{u_1\cat \cdots \cat u_{j-1}\cat w}}
    \]
  \emph{for the same $M_i$ and $c_i$}.
  But we had $\Gamma,s^{(j-1)}\ts M_j\converges_{u_j} c_j,s^{(j)}$.  
  By hypothesis, $u_1\cat \cdots \cat u_n$ is a finite prefix of $u_1 \cat \cdots \cat u_{j-1} \cat w$, and so $u_j$ is a finite prefix of $w$.  
  Therefore, by the inductive hypothesis applied to $M_j$, we cannot have $\Gamma,s^{(j-1)}\ts M_j\diverges$, leading to the desired contradiction.

  For the converse, define an \emph{$n$-truncated} derivation tree to be a derivation tree of height $n$ such that premises of the form $\Gamma,s\ts M\diverges^w$ may occur at the top level without proof.  
  We now define, for any infinite sequence $w\in X^\omega$, any triple $\Gamma,s\ts M$ and any $i\ge 0$, either
  \begin{itemize}
    \item a proof of height $\le i$ that $\Gamma,s\ts M\converges_u c,s'$ for some $u\pprefix w$ (and some $c,s'$); or
    \item an $i$-truncated proof $F_i(w,\Gamma,s\ts M)$ of $\Gamma,s\ts M\diverges^v$ for some (possibly infinite) $v\prefix w$.
  \end{itemize}
  Moreover, the $F_i(w,\Gamma,s\ts M)$ will form an infinite sequence, so that if $i<j$ then $F_i(w,\Gamma,s\ts M)$ is the result of truncating $F_j(w,\Gamma,s\ts M)$ at the $i$-th level.

  For $i=0$, $F_i(w,\Gamma,s\ts M)$ is the tree of the form
  \[
    \Gamma,s\ts M\diverges^w\,,
    \]
  where this statement is treated as an unproven (and in many cases untrue) hypothesis.

  Now we define $F_{i+1}(w,\Gamma,s\ts M)$ as follows.  
  If $M$ is a canonical form, then $F_{i+1}(w,\Gamma,s\ts M)$ is the derivation.
  \[
    \inferrule*{ }{\Gamma,s\ts M\converges_\epsilon M,s}\,.
    \]
  If $M=\ask_X$, then $F_{i+1}(w,\Gamma,s\ts M)$ is the derivation
  \[
    \inferrule*{ }{\Gamma,s\ts M\converges_{w^{(0)}} w^{(0)},s}\,.
    \]
  Otherwise, based on the structure of $M$, we choose a first premise $M_1$ for $M$.  
  In most cases, there is a unique \IAX rule that applies to $M$, but sometimes (i.e., if $M$ is $\pred M'$, $\If M' \Then N' \Else P'$, $\IfO M' \Then N' \Else P'$, $!V$ or $V\gets E$) there may be more than one possible rule.  
  However, in all of these cases, the first premise of each rule is the same.  
  For instance, if $M=\If M' \Then N' \Else P'$ then the first premise of both the possible IA rules is $M'$, and the value that this converges to then determines which instance of the $\If$ rule we use.

  We evaluate $F_i(w,\Gamma,s\ts M_1)$.  
  If $F_i(w,\Gamma,s\ts M_1)$ is a complete proof that $\Gamma,s\ts M_1 \converges_{u_1} c_1,s^{(1)}$ for $u\pprefix w$, then write $w=u_1w_2$ and move on to the next premise, computing $F_i(w_2,\Gamma,s\ts M_2)$.  
  Note that now the value of $c_1$ completely determines the \IAX rule we are using; for example, if $M=\If M' \Then N' \Else P'$, and $\Gamma,s\ts M' \converges_{u} c,s'$, then $c$ is either $\true$ or $\false$, and there is a unique \IAX rule that applies in each case.

  We keep going through the premises of $M$ in order.  
  If we satisfy them all, so that $F_i(w_i,\Gamma,s^{(i-1)} \ts M_i)$ is a proof that $\Gamma,s^{(i-1)} \ts M_i \converges_{u_i} c_i,s^{(i)}$ for each $M_i$, then we can put these together to get a derivation of
  \[
    \Gamma,s \ts M\converges_{u_1\cat \cdots \cat u_n} c,s^{(n)}\,,
    \]
  of height $\le i+1$, where $u_1\cat \cdots \cat u_n$ is a finite prefix of $w$.
  
  Otherwise, at some point $F_i(w,\Gamma,s^{(j-1)}\ts M_j)$ must be an $i$-truncated proof that
  \[
    \Gamma,s^{(j-1)}\ts M_j\diverges^v
    \]
  for some $v\prefix w$.
  In that case, since we have a proof of height $\le i$ for $\Gamma,s^{(k-1)}\ts M_k \converges_{u_k} c_k,s^{(k)}$ for each $k<j$, we can apply the appropriate rule for $\diverges^v$ to get an $(i+1)$-truncated proof that
  \[
    \Gamma,s\ts M\diverges^{u_1\cat \cdots \cat u_{j-1} \cat v}\,,
    \]
  where $u_1\cat \cdots \cat u^{j-1} \cat v$ is a prefix of $w$.

  Now, given $w$ and $\Gamma,s\ts M$, either $F_m(\Gamma,s\ts M)$ is a proof that $\Gamma,s\ts M\converges_u c,s'$ for $u\pprefix w$ and $m$ sufficiently large, or the $F_i(w,\Gamma,s\ts M)$ form an infinite sequence of trees of increasing height that all extend one another.  
  In the second case, we can stitch the trees together to form an infinite derivation tree for $\Gamma,s\ts M\diverges$.

  It follows that if there is some $w$ such that $\Gamma,s\ts M\not\converges_u c,s'$ for any finite prefix $u\pprefix w$, then there is an infinite derivation tree for $\Gamma,s\ts M\diverges$.
\end{proof}

We can now get an adequacy result for must testing.

\begin{definition}
  Given a morphism $\sigma\from 1\to\bC$ in $\G_X$, considered as a morphism $X \to \bC$ in $\G$, we write $\sigma\downarrow_{\must}$ if whenever $w\in X^\omega$ is an infinite sequence, then $w$ has a finite prefix $u$ such that the composite
  \[
    1 \xrightarrow{\sigma}
    (X \to \bC) \xrightarrow{\eta_u}
    (\Var \to \bN) \xrightarrow{\deno{\neww}}
    \bN \xrightarrow{t_{|u|}}
    \bC
    \]
  is not equal to $\bot$.
  \label{DefDenoMust}
\end{definition}

\begin{corollary}
  Let $M\from \com$ be a closed term of \IAX and consider the denotation $\deno M \from 1 \to \bC$ in $\G_X$ as a morphism $1 \to (X \to \bC)$ in $\G$.
  Then $\blank,()\ts M\mustconverge$ if and only if $\deno{M}\downarrow_{\must}$.
  \label{CorAdeqMust}
\end{corollary}

Once again, this slightly abstruse Definition \ref{DefDenoMust} becomes much more natural when $\G$ is the category of games, though in this case things are a bit more subtle.  
Let $\sigma$ be a strategy for $\oc X \implies \bC$.  
We observed earlier that $\sigma;\eta_u;\deno{\new};t_{|u|}\neq\bot$ if and only if the sequence
\[
  \begin{array}{ccc}
    \oc X       & \implies & \bC \\
                &          &  q  \\
      q         &          &     \\
    u^{(0)}     &          &     \\
    \vdots      &          &     \\
      q         &          &     \\
   u^{(|u|-1}   &          &     \\
                &          &  a
  \end{array}
  \]
is contained in $\sigma$.  
Now fix some infinite sequence $w\in X^\omega$ and consider what it means if we say that such a sequence is not contained in $\sigma$ for any $u\pprefix w$.  
By the definition of a strategy, there are two reasons why this might happen.  
Either there is a partiality in the strategy, so that for some finite $v\pprefix w$, player $P$ has no reply at all to the $O$-position
\[
  \begin{array}{ccc}
    \oc X       & \implies & \bC \\
                &          &  q  \\
      q         &          &     \\
    v^{(0)}     &          &     \\
    \vdots      &          &     \\
      q         &          &     \\
   v^{(|v|-1}   &          &     \\
                &          & {a\,.}
  \end{array}
  \]
Or, $\sigma$ contains an infinite increasing sequence of plays, whose limit is the infinite sequence
\[
  \begin{array}{ccc}
    \oc X       & \implies & \bC \\
                &          &  q  \\
      q         &          &     \\
    v^{(0)}     &          &     \\
      q         &          &     \\
    v^{(1)}     &          &     \\
    \vdots      &          &     
  \end{array}
  \]
\begin{definition}
  We say that a strategy $\sigma\from \oc X \implies \bC$ is \emph{winning} if $\sigma$ is total and contains no infinite plays.
\end{definition}

We have shown the following adequacy result.
\begin{corollary}
  Let $M\from \com$ be a closed term of \IAX and consider the denotation $\deno M \from 1 \to \bC$ in $\G_X$ as a morphism $X \to \bC$ in $\G$.
  Then 
  \[
    \blank,()\ts M\mustconverge
    \]
  if and only if this strategy is winning.
\end{corollary}

\section{Full Abstraction for Finite Nondeterminism Under Must Testing}

We now extend our earlier definitions using the new $\mustconverge$ rule.

\begin{definition}
  Given closed terms $M,N\from T$ of \IAX, we write 
  \[
    M\equiv_{\must} N
    \]
  if for all contexts $-\from T\ts C[-]\from\com$, $C[M]\mustconverge$ if and only if $C[N]\mustconverge$.
\end{definition}

We write $M\emam N$ if $M\equiv_{\may} N$ and $M\equiv_{\must} N$.

\begin{definition}
  Given morphisms $\sigma,\tau \from 1 \to \bC$ in $\G_X$, we write $\sigma\approx_{\must}\tau$ if $\sigma\downarrow_{\must}$ and $\tau\downarrow_{\must}$ or if $\sigma\not\downarrow_{\must}$ and $\tau\not\downarrow_{\must}$.  
\end{definition}

We write $\sigma\amam \tau$ if $\sigma\approx_{\may}\tau$ and $\sigma\approx_{\must}\tau$.

\begin{definition}
  Given morphisms $\sigma,\tau\from A \to B$ in $\G_X$, considered as morphisms $1 \to (A \to B)$, we write $\sigma\sim_{\must}\tau$ if whenever $\alpha\from (A \to B) \to \bC$ is a morphism in $\G_X$, then $\sigma;\alpha\approx_{\must}\tau;\alpha$.
\end{definition}

We write $\sigma\smam \tau$ if $\sigma\sim_{\may}\tau$ and $\sigma\sim_{\must} \tau$.

\begin{remark}
  In the category of games we can simplify the definition of $\smam$ to saying that $\sigma\smam \tau$ if for all $\alpha\from (A \to B) \to \bC$, $\sigma;\alpha\approx_{\may}\tau;\alpha$ and either $\sigma;\alpha$ and $\tau;\alpha$ are both winning, or neither is.
\end{remark}

Now we might expect to be able to prove a full abstraction result as before, namely that $\deno M\smam \deno N$ if and only if $M\emam N$.  
In the case of finite nondeterminism, this is possible.

\begin{theorem}
  Let $M,N\from T$ be closed terms of IA${}_{\bB}$.
  Then $M\emam N$ if and only if $\deno M\smam \deno N$.
  \label{FullAbstractionFiniteNondeterminism}
\end{theorem}
\begin{proof}
  We already know that $\deno M\sim_{\may}\deno N$ if and only if $M$ and $N$ are may-contextually equivalent, so it is sufficient to prove that $\deno M\sim_{\must}\deno N$ if and only if $M\equiv_{\must}N$.

  One direction is easy, as usual.  
  Suppose that $\deno M\sim_{\must} \deno N$, and let $C[-]$ be a context.  
  Suppose that $C[M]\mustconverge$, and fix $w\in X^\omega$.
  Then $\deno{C[M]} = \deno{C};\deno{M}$, so there is some finite prefix $u\pprefix w$ such that the composite
  \[
    1 \xrightarrow{\deno{C};\deno{M}}
    (X \to \bC) \xrightarrow{\eta_u}
    (\Var \to \bN) \xrightarrow{\deno{\neww}}
    \bN \xrightarrow{t_{|u|}}
    \bC
    \]
  is not equal to $\bot$.  
  Since $M\sim_{\must} N$, there is some finite prefix $v\pprefix w$ and some compact $\alpha'\from \deno{T}\to\bC$ such that the composite
  \[
    1 \xrightarrow{\deno{C};\deno{N}}
    (X \to \bC) \xrightarrow{\eta_v}
    (\Var \to \bN) \xrightarrow{\deno{\neww}}
    \bN \xrightarrow{t_{|v|}}
    \bC
    \]
  is not equal to $\bot$, and therefore $C[N]\mustconverge$.  
  The other direction is completely symmetric.

  The converse is harder.  
  Let $\alpha \from \deno{T} \to \bC$ be an arbitrary morphism.
  Suppose that for all $w\in \bB^\omega$ there is some finite $u\pprefix w$ such that the composite
  \[
    \deno{M};(X \to \alpha);\mu;\eta_u;\deno{\neww};t_{|u|}
    \]
  is not equal to $\bot$.  
  Let $V$ be the set of all sequences $v\in \bB^*$ such that $v$ has no prefix $u$ with this property.  
  Then $V$ is a finitely branching tree with no infinite path so, by K\"{o}nig's Lemma, it is finite.  
  Therefore, there is some finite set of sequences $U\subset \bB^*$ such that for every sequence $w\in \bB^\omega$, $w$ has some prefix $u\in U$ such that $\deno{M};(X \to \alpha);\mu;\eta_u;\deno{\neww};t_{|u|}\ne\bot$.

  Since $\G$ is enriched in algebraic cpos, for each $u\in U$ there is some compact $\alpha^u\subset\alpha$ such that
  \[
    \deno{M};(X \to \alpha^u);\mu;\eta_u;\deno{\neww};t_{|u|}
    \]
  is not equal to $\bot$.  
  Since the set of compact elements below $\alpha$ is directed, there is some compact $\alpha'\subset\alpha$ such that $\alpha^u\subset\alpha'$ for each $u\in U$.  

  So we now know that: for any infinite sequence $w\in \bB^\omega$, there is some finite prefix $u\pprefix w$ such that the composite
  \[
    \deno{M};(X \to \alpha');\mu;\eta_u;\deno{\neww};t_{|u|}
    \]
  is not equal to $\bot$.  
  Now if $\deno N\not\sim_{\must}\deno M$, then there is some infinite $w\in \bB^\omega$ is such that for every $v\pprefix w$ the composite
  \[
    \deno{N};(X \to \alpha);\mu;\eta_v;\deno{\neww};t_{|v|}
    \]
  is equal to $\bot$; since $\alpha'\subset \alpha$, the same is true if we replace $\alpha$ with $\alpha'$.

  Now, since $\alpha'$ is compact, it is definable as a term in IA and hence as a term $L$ in IA${}_\bB$.  
  By Corollary \ref{CorAdeqMust}, this means that $L M \mustconverge$, while $L N\diverges$, so $M\not\equiv_{\must}N$.
\end{proof}

This last result makes crucial use of the finiteness of $\bB$ in order to apply K\"{o}nig's lemma.  
In the case of IA${}_\bN$, the argument we have just used does not apply, since we can certainly have infinite bounded-height trees if we allow infinite branching.

In fact, this point is essential: not only does this specific argument not work for our model of IA${}_\bN$, the model is not necessarily Fully Abstract at all! 
In particular, if $\G$ is the category of games, then the semantics of must testing in $\G_{\bN}$ is not Fully Abstract.
The next result will show that no model of countable nondeterminism can be Fully Abstract, subject to some rather mild assumptions.  
We will then explore how we can come up with a model for which these assumptions do not hold, and eventually prove a Full Abstraction result.

\section{The Kleene Tree}
\label{SecKleene}

In order to show that our model is not Fully Abstract, we start by introducing the Kleene tree.

We define a \emph{binary tree} to be a prefix-closed subset of $\bB^*$.  
Using the binary representation, we can identify $\bB^*$ with $\bN$, meaning that every such tree can be specified by its characteristic function $\chi \from \bN \to \bB$.

We say that the tree is \emph{computable} if there is some IA term $M\from \nat \to \bool$ such that $M n$ converges to $\chi(n)$ for all $n\in\bN$.

In this section, we will show the construction of the \emph{Kleene tree} (see \cite[4.5]{Joshua} and \cite{KleeneTree}).  
This is a tree $K$ that has the following seemingly paradoxical properties.
\begin{itemize}
  \item It is computable.
  \item It has an infinite path; that is, there are $b_0,b_1,\cdots\in\bB$ such that for all $k$, the string $b_0\cdots b_{k-1}$ is contained in $K$.
  \item It has no computable infinite path.  
    I.e., if $M \from \bool \to \nat$ is a term of IA such that $M i$ converges to some $a_i\in \bB$ for each $i\in \bN$, then there is some $k$ such that the string $a_0 \cdots a_{k-1}$ is not contained in $K$.
\end{itemize}

In order to construct $K$, we are first going to assume that we have fixed some sensible encoding of IA terms as binary strings and hence as natural numbers.  
We shall further assume the existence of an \emph{interpreter} for IA terms of type $\bool$; that is, a term $\Int \from \nat \to \nat$ such that:
\begin{itemize}
  \item if $n$ is the encoding of some $M \from \bool$ such that $M \converges \true$, then $\Int n \converges 0$;
  \item if $n$ is the encoding of some $M \from \bool$ such that $M \converges \false$, then $\Int n \converges 1$; 
  \item if $n$ is the encoding of some $M \from \bool$ such that $M$ diverges, then $\Int n$ diverges; and
  \item if $n$ is not the encoding of a term of type $\bool$, then $\Int n\converges 255$.
\end{itemize}

The construction of $\Int$ is more a matter of software engineering than anything else.  
The idea is to use pattern matching and recursion to implement the small-step semantics of Idealized Algol within the encoding we have fixed.

In fact, we shall need a slightly modified version of $\Int$ that accepts an extra \emph{timeout} parameter.  
That is, we require some $\Int'\from \nat \to \nat \to \nat$ such that
\begin{itemize}
  \item if $n$ is the encoding of some $M \from \bool$ such that $M\converges\true$ in under $k$ steps, then $\Int' k\,n \converges 0$;
  \item if $n$ is the encoding of some $M \from \bool$ such that $M\converges\false$ in under $k$ steps, then $\Int' k\,n \converges 1$; and
  \item if $n$ is the encoding of some $M \from \bool$ such that $M$ does not converge after $k$ steps, or if $n$ is not the encoding of a term of type $\bool$, then $\Int' k\,n \converges 255$.
\end{itemize}

This modification to the original $\Int$ is not too hard to make, but now we can be assured that $\Int'$ will always converge, even if fed the encoding of a divergent program.  
This will be important later.

\begin{definition}[Kleene tree]
  We construct the tree $K$ as follows.  
  Suppose $b=b_0,\cdots,b_{k-1}$ is a finite sequence of binary digits.  
  Then $b$ is contained in $K$ if and only if for all $n=0,\cdots,k-1$, if $n$ is the encoding of a program $M\from \nat \to \bool$ such that $M n$ converges in under $k$ steps, then $M n$ converges to a value different from $b_n$.  
\end{definition}

\begin{remark}
  We can use $\Int'$ to construct an IA term $\nat \to \bool$ that computes the characteristic function of $K$.  
  Since the syntax of IA is not very transparent, we shall describe the operation of the term informally.  
  Given an input $n$, treated as a sequence $b_0 \cdots \b_{k-1}$, we iterate from $0$ up to $k-1$.  
  For each number $p$, if $p$ is the encoding of a term $M \from \nat \to \bool$, we pass the encoding of $M p$ to $\Int'$, giving $k$ as the timeout parameter.  
  If we get $0$ (true) or $1$ (false) back, then we compare against $b_p$, returning $\false$ if the values are equal and $\true$ if they are different.  
  If we get the error value $255$ back, then we return $\true$.

  Lastly, if we have got $\true$ back for each $p$, we return $\true$; otherwise, we return $\false$.
\end{remark}

\begin{proposition}
  i) $K$ has an infinite path.

  ii) $K$ has no computable infinite path.
\end{proposition}
\begin{proof}
  (i): For each $p$ such that $p$ is the encoding of some $M \from \nat \to \bool$ such that $M p$ converges, let $a_p$ be $\true$ if $M p \converges \false$ and $\false$ if $M p \converges \true$.  
  Otherwise -- i.e., if $M p$ does not converge or if $p$ is not the encoding of a term $\nat \to \bool$ -- then return any value -- $\false$, say.
  Then any sequence $a_0 \cdots a_{k-1}$ must be contained in $K$.

  (ii): Let $M \from \nat \to \bool$ be such that there are $b_i\in\bB$ for which $M i \converges b_i$ for all $i$, let $p$ be the encoding of $M$ and suppose that $M p$ converges in $k$ steps.  
  Let $N=\max\{k-1,p\}$.
  We claim that $b_0\cdots b_{N}$ is not contained in $K$.  
  Indeed, $p$ is the encoding of $M$ and $M p$ converges to $b_p$ in under $N$ steps.
\end{proof}

\section{Non-computable functions and Observational Equivalence}
\label{SecNonComputable}

The presence of non-computable functions in the semantics for a language does not normally present a problem for Full Abstraction.  
In previous sections, we have been able to appeal to continuity results that tell us that the intrinsic equivalence relation in our model is determined entirely by some collection of \emph{compact} (and hence computable) morphisms.  
In the presence of countable nondeterminism, these continuity results no longer hold, and the intrinsic equivalence becomes sensitive to noncomputable elements in the semantics.

To investigate this, we define a language IA+Functions by extending IA with, for each (set-theoretic) function $f\from \bN \to \bB$, a primitive $\Phi_f \from \nat \to \bool$ with the following operational semantics.
\[
  \inferrule*{\Gamma,s\ts N \converges n,s'}{\Gamma,s \ts \mathop{\Phi_f}N \converges f(n),s'}
  \]

We make some new claims.

\begin{proposition}
  i) The Game Semantics model from Sections \ref{ChapGames} and \ref{ChapFullAbstraction} is Computationally Adequate and Fully Abstract for IA+Functions.

  ii) Two terms of IA are observationally equivalent within IA if and only if they are observationally equivalent when we consider them as terms of IA+Functions.
\end{proposition}
\begin{proof}
  (i): The denotation of $\Phi_f$ is the strategy with maximal plays of the form
  \[
    \begin{array}{cc}
      \bN & \bN \\
          &  q  \\
       q  &     \\
       n  &     \\
          & f(n)
    \end{array}\,,
    \]
  for $n\in \bN$.  
  In the notation of Section \ref{DenotationalSemanticsOfIdealizedAlgol}, this is the strategy given by the formula
  \[
    ( f(0), f(1), f(2), \cdots )\,.
    \]
  Computational adequacy for this function follows by exactly the same arguments as for the functions $\suc$ and $\pred$; note that we used no special properties of the successor or predecessor functions in Chapter \ref{ChapFullAbstraction}, so exactly the same arguments go through for any function $\bN \to \bN$.  
  Full Abstraction also goes through without modification.

  (ii): This follows from (i), since the two languages share a Fully Abstract model.
  \label{PropPlusFunctions}
\end{proof}

In the presence of countable nondeterminism, however, we shall show that the equivalent of (ii) does not hold; i.e., that there are observationally equivalent terms of nondeterministic IA that are observationally distinguishable within IA+Functions+Countable nondeterminism.

This will then give the reason why our proposed semantics cannot be fully abstract: our starting model is fully abstract for both IA and IA+Functions and we have already proved that the nondeterministic version is Equationally Sound for nondeterministic IA (and, by a similar argument, for IA+Functions+Countable nondeterminism).  
Equational Soundness tells us that any observationally distinguishable terms of IA+Functions+Countable nondeterminism must therefore be distinguished by the model, proving that it cannot be fully abstract for IA+Countable nondeterminism.

Let us now examine the two terms that we shall be considering.  
First, we define $M \from (\nat \to \bool) \to \com$ by the formula
\begin{align*}
  M=& \lambda f.\Let k=?\In \\
  & \hspace{10pt} \If K((f\,0)\cdots(f\,(k-1))) \Then (\skipp\Or\skipp) \Else (\skipp \Or \Omega)\,,
\end{align*}
where
\[
  ((f\,0)\cdots(f\,(k-1)))
  \]
is shorthand for the natural number corresponding to the binary sequence given by $(f\,0)$ up to $(f\,(k-1))$, $\Or$ is binary choice and $K$ is the Kleene tree from the previous section.
Now define $N \from (\nat \to \bool) \to \com$ by the formula
\begin{align*}
  N=& \lambda f.\Let k=?\In (f\,0);\cdots;(f\,(k-1));\\
  &\hspace{10pt}\skipp \Or \Omega\,,
\end{align*}
where we have used the notation $P;Q$ (for $P \from \nat$) as a shorthand for
\[
  \IfO P \Then Q \Else Q\,;
  \]
i.e., the program that evaluates $P$, forgets the result and then evaluates $Q$.
The reason for evaluating $(f\,0), \cdots, (f\,(k-1))$ before either returning or diverging is to make sure that any side effects of the evaluation of $f$ are the same for both programs (we shall assume that $K$ operates in such a way that it evaluates each of the $(f\,i)$ exactly once).

We now claim that $M$ and $N$ are observationally equivalent in nondeterministic IA.  
We will postpone the formal proof of this fact until we have a working Fully Abstract semantics for that language, but a rough argument is as follows.  
$M$ and $N$ can only be distinguished by an $f$ such that
\[
  K((f\,0)\cdots(f\,(k-1)))
  \]
evaluates to $\true$ for all $k$.  
Otherwise, both terms may diverge.  
But we have shown in the previous section that there is no computable $f$ with this property.

Meanwhile, $M$ and $N$ are not observationally equivalent in nondeterministic IA+Functions; indeed, if $\kappa\from \bN \to \bB$ is an infinite path in the Kleene tree, then $M$ and $N$ are distinguished by $\Phi_\kappa$.

\section{Full Abstraction for Countable Nondeterminism under Must Testing}

The previous two sections make it clear that the problem is the existence in the semantics of morphisms that behave like certain noncomputable functions.  
The presence of such morphisms is normally an orthogonal issue to full abstraction, since we can usually assume that the strategy $\alpha$ in the definition of intrinsic equivalence is compact.
However, in our case, the presence of these noncomputable morphisms is a big problem.

In order to get around this problem, then, we will clearly need to disallow noncomputable functions $\bN\to\bN$, which quickly entails that we should disallow all morphisms that are not computable.

Such models were considered by Hyland and Ong \cite{hoPcf} and by Abramsky, Jagadeesan and Malacaria \cite{ajmPcf} in their original proofs of Full Abstraction for PCF.
In these models, a game (or arena) $A$ comes equipped with an encoding function $M_A \to \bN$; a strategy is said to be \emph{recursive} if it can be defined by a recursive function $\bN \to \bN$.
The remarkable thing about these models is that they satisfy a property called \emph{universality}: \emph{every} morphism into the denotation of one of the types of the language is definable.

Suppose our model $\G$ of Idealized Algol satisfies universality.  
Then it is relatively easy to pass from Computational Adequacy to Full Abstraction.

\begin{theorem}
  Let $M,N\from T$ be terms of \IAX and suppose that our base semantics of IA in $\G$ satisfies universality.  
  Then $M\emam N$ if and only if $\deno M \smam \deno N$.
\end{theorem}
\begin{proof}
  See the proof of Theorem \ref{FullAbstractionFiniteNondeterminism} for the proof that if $\deno M \smam \deno N$ then $M\emam N$.  

  Conversely, if $\deno M \not\smam \deno N$, then without loss of generality there is some $\alpha\from\deno T \to \bC$ such that $\deno M;\alpha \downarrow_{\must}$ and $\deno N;\alpha\not\downarrow_{\must}$.  
  But by universality $\deno M;\alpha = \deno{LM}$ and $\deno N;\alpha=\deno{LN}$ for some term $L\from T \to \com$ of IA.  
  Therefore, the result follows from Corollary \ref{CorAdeqMust}.
\end{proof}

The next step is to find an example of such a $\G$.
As we said earlier, we will do this by cutting our game semantics down so that it only contains \emph{computable} strategies.
The first step is to fix an enumeration $e_A\from M_A \to \bN$ of the moves of each game $A$.  
We do this for the ground type games and then extend to the connectives.  
For example:
\begin{mathpar}
  e_{\bC}(q)=0
  \and
  e_{\bC}(a)=1
  \and
  e_{\bB}(q)=0
  \and
  e_{\bB}(\true)=1
  \and
  e_{\bB}(\false)=2
  \\
  e_{\bN}(q)=0
  \and
  e_{\bN}(n)=n+1
  \and
  e_{\Var}(q)=0
  \and
  e_{\Var}(n)=3n+1
  \and
  e_{\Var}(q_n)=3n+2
  \and
  e_{\Var}(a_n)=3n+3
  \\
  e_{A\tensor B}(\inj_A(a))=2e_A(a)
  \and
  e_{A\tensor B}(\inj_B(b))=2e_B(b)+1
\end{mathpar}
...and so on.

We can extend the enumeration of moves to an enumeration of (justified) plays in $A$.  
We then say that a strategy $\sigma$ is \emph{recursive} if it is a recursively enumerable subset of $P_A$.
Recursive visible strategies give us a Cartesian closed subcategory of the original category of games and visible strategies.

We now appeal to some results about these recursive strategies.  
For example, the following was proved in \cite{hoPcf}.

\begin{theorem}[Universality for PCF]
  Let $A$ be the denotation of a PCF type $T$
  Then every recursive innocent strategy for $A$ is innocently intrinsically equivalent to the denotation of some PCF term $M\from T$.
\end{theorem}

This result is not quite enough for us, since the innocent intrinsic equivalent is coarser than that for visible strategies.  
However, if we add the constant $\lett$ to PCF, then we can get a stronger result that gives us definability \emph{on the nose}.

\begin{theorem}[{Unpublished, but see \cite[\sec 7.1.5]{LongleyBook}}]
  Let $A$ be the denotation of a PCF type $T$.
  Then every recursive innocent strategy for $A$ is the denotation of a term $M\from T$ of PCF+$\lett$.
\end{theorem}

Moreover, these results are all easily extended to the IA ground types that are not present in PCF.  
So we see that any recursive innocent $\sigma\from \deno T$ is the denotation of some Idealized Algol term $M \from T$.

Lastly, we observe that our innocent factorization result (Proposition \ref{PropInnocentFactorization}) will always produce a recursive innocent strategy if the original strategy is recursive.  
It follows, then, that any recursive \emph{visible} $\sigma\from \deno T$ is the denotation of some Idealized Algol term $M \from T$.
Thus, the category $\G^{rec}$ of games and recursive visible strategies is universal for Idealized Algol, and therefore its Kleisli category $\G^{rec}_X$ is fully abstract for \IAX.

As an application of this fully abstract semantics, we clear up a loose end from Section \ref{SecNonComputable}.

\begin{lemma}
  The terms $M,N\from (\nat \to \bool) \to \com$, as defined in Section \ref{SecNonComputable}, are observationally equivalent in Idealized Algol with countable nondeterminism.
\end{lemma}
\begin{proof}
  The denotations for $M$ and $N$ both have maximal plays that look like this:
  \[
    \begin{array}{cccc}
      \bN & \bN & \bB & \bC \\
          &     &     &  q  \\
       q  &     &     &     \\
       k  &     &     &     \\
          &     &  q  &     \\
          &  q  &     &     \\
          &  0  &     &     \\
          &     & a_0 &     \\
          &\vdots&    &     \\
          &     &  q  &     \\
          &  q  &     &     \\
          &(k-1)&     &     \\
          &     &a_{k-1}&   \\
       q  &     &     &     \\
       n  &     &     &     \\
          &     &     & (a)
    \end{array}\,.
    \]
  Here, the copy of $\bN$ on the left denotes the nondeterministic oracle given by the Kleisli category.  
  The final move $a$ is in brackets to indicate that it does not always occur, depending on the moves that have gone before.

  More specifically, in $\deno M$, the final move $a$ always occurs if $a_0\cdots a_{k-1}$ is a finite path in the Kleene tree; otherwise, it may or may not occur, depending on whether $n$ is $0$ or not.

  Meanwhile, the final move $a$ is present in $\deno N$ if and only if the number $n$ is equal to $0$.
  This means that $\deno N$ is a subset of $\deno M$.

  Fix some strategy $\alpha\from ((\bN \to \bC)\to\bC) \to \bC$ in $\G_X$ and suppose that
  \[
    \deno M;\alpha\not\amam \deno N;\alpha\,.
    \]
  First note that we always have $\deno M;\alpha\approx_{\may}\deno N;\alpha$ for arbitrary $\alpha$, which we can deduce by ignoring the copy of $\bN$ on the left and considering $\deno M$ and $\deno N$ as nondeterministic strategies.  
  In this setting, $\deno M$ and $\deno N$ are equal, since both have maximal plays of the form
  \[
    \begin{array}{ccc}
       \bN & \bB & \bC \\
           &     &  q  \\
           &     &     \\
           &     &     \\
           &  q  &     \\
        q  &     &     \\
        0  &     &     \\
           & a_0 &     \\
     \vdots&\vdots&   \\
           &  q  &     \\
        q  &     &     \\
      (k-1)&     &     \\
           &a_{k-1}&   \\
           &     &     \\
           &     &     \\
           &     & (a)
    \end{array}\,,
    \]
  where $k$ is arbitrary.  

  Therefore, we must have $\deno M;\alpha \not\approx_{\must}\deno N;\alpha$.

  Since $\deno N \subseteq \deno M$, it must be the case that $\deno M;\alpha\downarrow_{\must}$ and $\deno N;\alpha\not\downarrow_{\must}$; i.e., that $\deno M;\alpha$ is a winning strategy, when considered as a strategy $\bN \to \bC$ in $\G$, but $\deno N;\alpha$ is not.

  Since $\deno M$ and $\deno N$ do not differ until the very last move, this means that $\alpha$ must respond to the initial move $q$ in the rightmost copy of $\bC$ with the move $q$ in the copy of $\bC$ on the left (otherwise, $\deno N;\alpha$ is winning), and that $\alpha$ must be total (otherwise, $\deno M;\alpha$ is not winning).
  $\alpha$ replies to each move $i$ in $\bN$ with some move in $\bB$, and this defines a function $\bN \to \bB$.  
  Since $\alpha$ is total, this function must trace out an infinite path in the Kleene tree.

  But if $\alpha$ were recursive, then we would be able to recover this infinite path from $\alpha$ via a recursive algorithm, and we know that this is not possible
  Therefore, $\alpha$ is not recursive.  

  It follows that $M$ and $N$ are observationally equivalent in Idealized Algol with countable nondeterminism under may-and-must testing.
\end{proof}

Note that we certainly have $\sigma_\phi;\deno M\downarrow_{\must}$, while $\sigma_\phi;\deno N\not\downarrow_{\must}$, demonstrating that the recursivity requirement is essential to the full abstraction result in this case.

\section{The Intrinsic Equivalence Relation}
\label{SecIntrinsicEquivalenceRelationKleisli}

We conclude this chapter by examining the intrinsic equivalence relation in $\G_X$.

\begin{proposition}
  Let $\sigma,\tau\from A \to B$ be two morphisms in $\G_X$,
  If $\sigma\sim\tau$ when considered as morphisms $X \to (A \to B)$ in $\G$, then $\sigma\sim\tau$ in $\G_X$.
\end{proposition}
\begin{proof}
  Suppose that $\alpha \from (A \to B) \to \bC$ separates $\sigma$ and $\tau$ in $\G_X$, so without loss of generality, we have
  \begin{mathpar}
    1 \xrightarrow{(\sigma;\alpha)}
    (X \to \bC) \xrightarrow{\eta_u}
    (\Var \to \bN) \xrightarrow{\deno{\neww}}
    \bN \xrightarrow{t_{|u|}}
    \bC = \bot
    \and
    1 \xrightarrow{(\tau;\alpha)}
    (X \to \bC) \xrightarrow{\eta_u}
    (\Var \to \bN) \xrightarrow{\deno{\neww}}
    \bN \xrightarrow{t_{|u|}}
    \bC \ne \bot
  \end{mathpar}
  for some $u$.  
  Expanding the Kleisli compositions $\sigma;\alpha$ and $\tau;\alpha$, we get
  \begin{mathpar}
    \sigma;(X \to \alpha);\mu;\eta_u;\deno{\neww};t_{|u|} = \bot
    \\
    \tau;(X \to \alpha);\mu;\eta_u;\deno{\neww};t_{|u|} \ne \bot\,,
  \end{mathpar}
  and so $(X\to\alpha);\mu;\eta_u;\deno{\neww};t_{|u|}$ distinguishes $\sigma$ and $\tau$, when considered as morphisms $X \to (A \to B)$ in $\G$.
\end{proof}

As a special case, if $\sigma,\tau\from A \to B$ are morphisms in $\G$, then $\sigma\sim\tau$ in $\G$ if and only if $J\sigma\sim J\tau$ in $\G_X$.  
Combined with our Full Abstraction results, this tells us that \IAX is a conservative extension of Idealized Algol.

What is more, we have
\[
  \sigma \sim \tau \Rightarrow J\sigma \sim J\tau \Rightarrow J\sigma \smam J\tau \Rightarrow \sigma\sim\tau\,.
  \]
This proves that the finite and countable nondeterministic variants of Idealized Algol are conservative extensions of Idealized Algol.
