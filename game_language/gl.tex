\documentclass{article}

\usepackage[utf8]{inputenc}

\usepackage{graphicx} % support the \includegraphics command and options

\usepackage{parskip} % Activate to begin paragraphs with an empty line rather than an indent

%%% PACKAGES
\usepackage{booktabs} % for much better looking tables
\usepackage{array} % for better arrays (eg matrices) in maths
\ifdefined\BEAMER
\else
\usepackage{paralist} % very flexible & customisable lists (eg. enumerate/itemize, etc.)\prefix\t$.
\fi
\usepackage{verbatim} % adds environment for commenting out blocks of text & for better verbatim
\ifdefined\BEAMER
\else
\ifdefined\THESIS
\usepackage{subcaption}
\else
\usepackage{subfig} % make it possible to include more than one captioned figure/table in a single float
\fi
\fi
\usepackage{mathtools} % for the all important \coloneqq symbol
\usepackage{hyperref} % for hyperreferences
\usepackage{IEEEtrantools} % for \IEEEeqnarray
\usepackage{pbox} % for \pbox
\usepackage{multirow,bigdelim} % for \multirow
\usepackage{lettrine} % For the drop cap
\usepackage{mathpartir} % for \inferrule, \inferrule* and the mathpar environment
\usepackage{listings}

\usepackage{caption}
\captionsetup{singlelinecheck=off}

\ifdefined\NOTARTICLE
\else

%%% ToC (table of contents) APPEARANCE
\usepackage[nottoc,notlof,notlot]{tocbibind} % Put the bibliography in the ToC
\usepackage[titles,subfigure]{tocloft} % Alter the style of the Table of Contents
\renewcommand{\cftsecfont}{\rmfamily\mdseries\upshape}
\renewcommand{\cftsecpagefont}{\rmfamily\mdseries\upshape} % No bold!

\fi

%% Font things %%
\usepackage{amssymb}
\usepackage{cmll} % Linear logic symbols!
\ifdefined\FEWFONTS
\else
\usepackage{bm} % for bold Greek letters
\fi
\usepackage{stmaryrd}
\usepackage{bbm}

%% Get the sqsubsetneqq character from the mathabx package
\DeclareFontFamily{U}{mathb}{\hyphenchar\font45}
\DeclareFontShape{U}{mathb}{m}{n}{
      <5> <6> <7> <8> <9> <10> gen * mathb
      <10.95> mathb10 <12> <14.4> <17.28> <20.74> <24.88> mathb12
      }{}
\DeclareSymbolFont{mathb}{U}{mathb}{m}{n}

\DeclareMathSymbol{\sqsubsetneq}    {3}{mathb}{"88}
\DeclareMathSymbol{\varsqsubsetneq} {3}{mathb}{"8A}
\DeclareMathSymbol{\varsqsubsetneqq}{3}{mathb}{"92}
\DeclareMathSymbol{\sqsubsetneqq}   {3}{mathb}{"90}

%% Get the left and right moons from the wasysym package

\DeclareFontFamily{U}{wasy}{}
\DeclareFontShape{U}{wasy}{m}{n}{ <5> <6> <7> <8> <9> gen * wasy
      <10> <10.95> <12> <14.4> <17.28> <20.74> <24.88>wasy10  }{}
\DeclareFontShape{U}{wasy}{b}{n}{ <-10> sub * wasy/m/n
 <10> <10.95> <12> <14.4> <17.28> <20.74> <24.88>wasyb10 }{}
\DeclareFontShape{U}{wasy}{bx}{n}{ <-> sub * wasy/b/n}{}

\def\wasyfamily{\fontencoding{U}\fontfamily{wasy}\selectfont}
\def\leftmoon   {\mbox{\wasyfamily\char36}}
\def\rightmoon  {\mbox{\wasyfamily\char37}}

%% Lists %%
\usepackage{enumerate}

%% Graphics %%
\usepackage{tikz}
\usetikzlibrary{cd}
\usetikzlibrary{patterns}
\usetikzlibrary{calc}
\usetikzlibrary{decorations.pathmorphing}
\usetikzlibrary{positioning}

\tikzset{inlinearrows/.style={anchor=base,baseline,x=0.6\baselineskip,y=0.6\baselineskip}}

\ifdefined\BEAMER
\else

%% Theorems! %%
\usepackage{amsthm}
\theoremstyle{plain} % Theorems, lemmas, propositions etc.
\newtheorem{theorem}{Theorem}[section]
\newtheorem{lemma}[theorem]{Lemma}
\newtheorem{proposition}[theorem]{Proposition}
\newtheorem{corollary}[theorem]{Corollary}
\newtheorem{fact}[theorem]{Fact}
\newtheorem{construction}[theorem]{Construction}
\theoremstyle{definition} % Definitions etc.  
\newtheorem{definition}[theorem]{Definition}
\newtheorem{notation}[theorem]{Notation}
\theoremstyle{remark} % Remarks
\newtheorem{remark}[theorem]{Remark}
\newtheorem{remarks}[theorem]{Remarks}
\newtheorem{example}[theorem]{Example}
\newtheorem{question}[theorem]{Question}
\newtheorem{slogan}[theorem]{Slogan}

\newtheoremstyle{note} {3pt} {3pt} {\itshape} {} {\itshape} {:} {.5em} {} % For short notes
\theoremstyle{note}
\newtheorem{note}[theorem]{Note}

\fi

%% Exercises and answers %%
\usepackage{answers}

\newtheoremstyle{exercisestyle}% name
  {6pt}   % ABOVESPACE
  {6pt}   % BELOWSPACE
  {\itshape}  % BODYFONT
  {0pt}       % INDENT (empty value is the same as 0pt)
  {\bfseries} % HEADFONT
  {.}         % HEADPUNCT
  {3pt} % HEADSPACE
  {}          % CUSTOM-HEAD-SPEC

\theoremstyle{exercisestyle}
\newtheorem{exercise}{Exercise}
\newtheorem{answerthm}{Exercise}

\Newassociation{answer}{answerthm}{answers}
\newcommand{\answerthmparams}{}

%% Changes to enumerate things so they look better %%\sigma$

\makeatletter
\def\enumfix{%
\if@inlabel
 \noindent \par\nobreak\vskip-\topsep\hrule\@height\z@
\fi}

\let\olditemize\itemize
\def\itemize{\enumfix\olditemize}
\let\oldenumerate\enumerate
\def\enumerate{\enumfix\oldenumerate}

%% Random crap %%
\usepackage{xifthen}

\makeatletter
\def\thm@space@setup{%
  \thm@preskip=\parskip \thm@postskip=0pt
}
\makeatother

\makeatletter
\newcommand*{\relrelbarsep}{.386ex}
\newcommand*{\relrelbar}{%
  \mathrel{%
    \mathpalette\@relrelbar\relrelbarsep
  }%
}
\newcommand*{\@relrelbar}[2]{%
  \raise#2\hbox to 0pt{$\m@th#1\relbar$\hss}%
  \lower#2\hbox{$\m@th#1\relbar$}%
}
\providecommand*{\rightrightarrowsfill@}{%
  \arrowfill@\relrelbar\relrelbar\rightrightarrows
}
\providecommand*{\leftleftarrowsfill@}{%
  \arrowfill@\leftleftarrows\relrelbar\relrelbar
}
\providecommand*{\xrightrightarrows}[2][]{%
  \ext@arrow 0359\rightrightarrowsfill@{#1}{#2}%
}
\providecommand*{\xleftleftarrows}[2][]{%
  \ext@arrow 3095\leftleftarrowsfill@{#1}{#2}%
}
\makeatother

\newcommand{\catname}[1]{{\normalfont\textbf{#1}}}
\newcommand{\Rings}{\catname{CRing}}
\newcommand{\CAT}{\catname{CAT}}
%\newcommand{\Top}{\catname{Top}}
\newcommand{\Set}{\catname{Set}}
\newcommand{\Cat}{\catname{Cat}}
\newcommand{\MonCat}{\catname{MonCat}}
\newcommand{\SymmMonCat}{\catname{SymmMonCat}}
\newcommand{\Cont}{\catname{Cont}}
\newcommand{\Sch}{\catname{Sch}}
\newcommand{\Rel}{\catname{Rel}}
\newcommand{\Coh}{\catname{Coh}}
\newcommand{\Inj}{\catname{Inj}}
\newcommand{\Dcpo}{\catname{Dcpo}}
\newcommand{\Mod}[1][]{\ifthenelse{\isempty{#1}}{\catname{Mod}}{#1\catname{mod}}}
\DeclareMathOperator{\sh}{Sh}
\newcommand{\Sh}[1][]{\ifthenelse{\isempty{#1}}{\sh}{\sh(#1)}}
\newcommand{\map}[3]{#2\xrightarrow{#1} #3}
\newcommand*\from{\colon}
\newcommand*\bigto{\Rightarrow}
\newcommand{\cmap}[3]{#1\from{}#2\to{}#3}
\newcommand\oppcat[1]{#1^{\mathrm{op}}}
\newcommand{\object}{\colon}
\DeclareRobustCommand{\vmap}[3] {\begin{tikzcd} #2 \arrow[d, "#1"] \\ #3 \end{tikzcd}}
\newcommand{\partref}[1]{(\ref{#1})}
\newcommand{\intgrpd}[4] {#1 \xrightrightarrows[#3]{#4} #2}
\DeclareRobustCommand{\bigintgrpd}[4] {\begin{tikzcd}[ampersand replacement=\&] #1 \arrow[r, shift left=0.5ex, "#3"] \arrow[r, shift right=0.5ex, "#4"'] \& #2 \end{tikzcd}}

\usepackage{xspace}

\newcommand{\etale}{\'{e}tale\xspace}
\newcommand{\Etale}{\'{E}tale\xspace}

\def \inv {^{-1}}

\DeclareMathOperator{\id}{id}
\DeclareMathOperator{\op}{op}
\DeclareMathOperator{\pr}{pr}
\DeclareMathOperator{\inj}{in}
\DeclareMathOperator{\pre}{{pre}}
\DeclareMathOperator{\et}{{\acute{e}t}}

\DeclareMathOperator{\Hom}{Hom}
\DeclareMathOperator{\Spec}{Spec}

\DeclareMathOperator{\ol}{ol}

\def\presuper#1#2%
  {\mathop{}%
   \mathopen{\vphantom{#2}}^{#1}%
   \kern-\scriptspace%
   #2}
\def\presub#1#2%
  {\mathop{}%
   \mathopen{\vphantom{#2}}_{#1}%
   \kern-\scriptspace%
   #2}

\newsavebox{\overlongequation}
\newenvironment{longdiagram}
 {\begin{displaymath}\begin{lrbox}{\overlongequation}$\displaystyle}
 {$\end{lrbox}\makebox[0pt]{\usebox{\overlongequation}}\end{displaymath}}

%% Our things %%

\newcommand{\neggame}[1]{\presuper{\perp}{#1}}
\newcommand{\tensor}{\otimes}
\newcommand{\Tensor}{\bigotimes}
\newcommand{\sequoid}{\oslash}
\newcommand{\varsequoid}{\vartriangleleft}
\renewcommand{\implies}{\multimap}
\newcommand{\iimpl}{\Longrightarrow}
\newcommand{\comp}[2]{#1 \circ #2}
\newcommand{\icomp}[2]{\comp{#1}{#2}}
\newcommand{\cprd}{\sqcup}
\newcommand{\bigcprd}{\bigsqcup}
\newcommand{\G}{\mathcal G}
\newcommand{\W}{\mathcal W}
\newcommand{\suchthat}{\;\colon\;}
\newcommand{\varsuchthat}{\;\mid\;}
\newcommand{\esuchthat}{\;.\;}
\newcommand{\OP}{\{O,P\}}
\newcommand{\QA}{\{Q,A\}}
\renewcommand{\L}{\mathcal L}
\newcommand{\F}{\mathcal F}
\newcommand{\U}{\mathcal U}
\newcommand{\s}{\mathfrak s}
\renewcommand{\t}{\mathfrak t}
\renewcommand{\u}{\mathfrak u}
\renewcommand{\d}{\mathfrak d}
\newcommand{\e}{\mathfrak e}
\newcommand{\emptyplay}{\epsilon}
\newcommand{\bracketed}[1]{\left({#1}\right)}
\newcommand{\bneggame}[1]{{\bracketed{\neggame{#1}}}}
\newcommand{\prefix}{\sqsubseteq}
\newcommand{\ppprefix}{\sqsubset}
\newcommand{\pprefix}{\sqsubsetneqq}
\renewcommand{\ss}{\mathbf{s}}
\newcommand{\bN}{\mathbb{N}}
\newcommand{\bC}{\mathbb{C}}
\newcommand{\bB}{\mathbb{B}}
\newcommand{\bP}{\mathbb{P}}
\newcommand{\pfun}{\rightharpoonup}
\newcommand{\grel}[1]{\underline{#1}}
\DeclareMathOperator{\length}{length}
\renewcommand{\b}{\mathfrak b}
\renewcommand{\r}{\mathfrak r}
\newcommand{\bbeta}{{\bm{\beta}}}
\newcommand{\st}{{\Sigma^*}}
\let\sec\S
\renewcommand{\S}{{\mathfrak{S}}}
\DeclareMathOperator{\cc}{cc}
\DeclareMathOperator{\subs}{subs}
\DeclareMathOperator{\ret}{ret}
\DeclareMathOperator{\zz}{zz}
\newcommand{\aaa}{\mathbf{a}}
\newcommand{\bbb}{\mathbf{b}}
\newcommand{\ccc}{\mathbf{c}}
\newcommand{\ddd}{\mathbf{d}}
\newcommand{\B}{\mathcal B}
\newcommand{\BB}{\mathbf B}
\renewcommand{\H}{\mathcal H}
\DeclareMathOperator{\assoc}{assoc}
\DeclareMathOperator{\lunit}{lunit}
\DeclareMathOperator{\runit}{runit}
\DeclareMathOperator{\dom}{dom}
\DeclareMathOperator{\sym}{sym}
\newcommand{\braid}{\sym}
\newcommand{\blank}{\,\underline{\hspace{1.5ex}}\,}
\DeclareMathOperator{\cn}{cn}
\newcommand{\impliescn}{\protect\overset{\cn}{\implies}}
\newcommand{\C}{{\mathcal{C}}}
\newcommand{\D}{{\mathcal{D}}}
\newcommand{\E}{{\mathcal{E}}}
\newcommand{\V}{{\mathcal{V}}}
\newcommand{\EE}{{\mathbf{E}}}
\DeclareMathOperator{\ev}{ev}
\newcommand{\der}{{\mathtt{der}}}
\newcommand{\mult}{{\mathtt{mult}}}
\DeclareMathOperator{\wk}{wk}
\newcommand{\toisom}{{\xrightarrow{\cong}}}
\DeclareMathOperator{\passoc}{{\mathsf{passoc}}}
\DeclareMathOperator{\pcomm}{{\mathsf{pcomm}}}
\DeclareMathOperator{\run}{{\mathsf{r}}}
\DeclareMathOperator{\lun}{{\mathsf{l}}}
\newcommand{\fcoal}[1]{{\leftmoon #1 \rightmoon}}
\DeclareMathSymbol{\co}{\mathord}{operators}{"3C}
\DeclareMathSymbol{\nw}{\mathord}{operators}{"3E}
\newcommand{\T}{\mathfrak{T}}
\renewcommand{\subset}{\subseteq}
\newcommand{\Ord}{\catname{Ord}}
\newcommand{\FS}{\mathcal{FS}}
\DeclareMathOperator{\rank}{rank}
\DeclareMathOperator{\dist}{{\mathsf{dist}}}
\DeclareMathOperator{\dec}{{\mathsf{dec}}}
\DeclareMathOperator{\str}{str}
\DeclareMathOperator{\weak}{weak}
\DeclareMathOperator{\Strat}{Strat}
\DeclareMathOperator{\OppStrat}{OppStrat}
\newcommand{\seqs}[1]{{\overline{{#1}^{*}}}}
\def\flushRight{\leftskip0pt plus 1fill\rightskip0pt}
\def\Centering{\relax\ifvmode\centering\fi}
\newcommand{\deno}[1]{\left\llbracket#1\right\rrbracket}
\newcommand{\converges}{\Downarrow}
\newcommand{\diverges}{\Uparrow}
\newcommand{\mustconverge}{\converges^{\text{must}}}
\newcommand{\Iflt}{\mathtt{If{<}\;}}
\newcommand{\Ifgt}{\mathtt{If{>}\;}}
\newcommand{\inr}{{\mathsf{inr}}}
\newcommand{\inl}{{\mathsf{inl}}}
\newcommand{{\Na}}{\bN}
\newcommand{{\cell}}{{\mathsf{cell}}}
\newcommand{\fix}{{\mathsf{fix}}}
\newcommand{\eq}{{\mathsf{eq}}}
\DeclareMathOperator{\CCom}{CCom}
\newcommand{\power}{\mathfrak P}

% Slanty things
\newcommand*{\xslant}[2][76]{%
  \begingroup
    \sbox0{#2}%
    \pgfmathsetlengthmacro\wdslant{\the\wd0 + cos(#1)*\the\wd0}%
    \leavevmode
    \hbox to \wdslant{\hss
      \tikz[
        baseline=(X.base),
        inner sep=0pt,
        transform canvas={xslant=cos(#1)},
      ] \node (X) {\usebox0};%
      \hss
      \vrule width 0pt height\ht0 depth\dp0 %
    }%
  \endgroup
}

\makeatletter
\newcommand*{\xslantmath}{}
\def\xslantmath#1#{%
  \@xslantmath{#1}%
}
\newcommand*{\@xslantmath}[2]{%
  % #1: optional argument for \xslant including brackets
  % #2: math symbol
  \ensuremath{%
    \mathpalette{\@@xslantmath{#1}}{#2}%
  }%
}
\newcommand*{\@@xslantmath}[3]{%
  % #1: optional argument for \xslant including brackets
  % #2: math style
  % #3: math symbol
  \xslant#1{$#2#3\m@th$}%
}
\makeatother

\newcommand{\seqdeno}[1]{\xslantmath{\llbracket}#1\xslantmath{\rrbracket}}

% Empty set etc.

\let\oldemptyset\emptyset
\let\emptyset\varnothing

%% Constant width xrightarrows
\newlength{\arrow}
\settowidth{\arrow}{\scriptsize$1000$}
\newcommand*{\constantwidthxrightarrow}[1]{\xrightarrow{\mathmakebox[\arrow]{#1}}}

%% Landscape pages
\usepackage{everypage}
\usepackage{environ}
\usepackage{pdflscape}
\newcounter{abspage}

\ifdefined\NOTARTICLE

\else

\makeatletter
\newcommand{\newSFPage}[1]% #1 = \theabspage
  {\global\expandafter\let\csname SFPage@#1\endcsname\null}

\NewEnviron{SidewaysFigure}{\begin{figure}[p]
\protected@write\@auxout{\let\theabspage=\relax}% delays expansion until shipout
  {\string\newSFPage{\theabspage}}%
\ifdim\textwidth=\textheight
  \rotatebox{90}{\parbox[c][\textwidth][c]{\linewidth}{\BODY}}%
\else
  \rotatebox{90}{\parbox[c][\textwidth][c]{\textheight}{\BODY}}%
\fi
\end{figure}}

\AddEverypageHook{% check if sideways figure on this page
  \ifdim\textwidth=\textheight
    \stepcounter{abspage}% already in landscape
  \else
    \@ifundefined{SFPage@\theabspage}{}{\global\pdfpageattr{/Rotate 0}}%
    \stepcounter{abspage}%
    \@ifundefined{SFPage@\theabspage}{}{\global\pdfpageattr{/Rotate 90}}%
  \fi}
\makeatother

\fi

%% PCF Things

\newcommand{\nat}{{\mathtt{nat}}}
\newcommand{\bool}{{\mathtt{bool}}}

\newcommand{\Y}{\mathbf{Y}}
\newcommand{\opto}{\longrightarrow}
\newcommand{\oopto}{\dashrightarrow}
\newcommand{\n}{{\mathtt{n}}}
\DeclareMathOperator{\IfO}{{\mathsf{If0}}}
\DeclareMathOperator{\suc}{{\mathsf{succ}}}
\DeclareMathOperator{\pred}{{\mathsf{pred}}}
\newcommand{\0}{{\mathtt{0}}}

\newcommand{\iter}{{\mathtt{iter}}}
\newcommand{\rec}{\iter}
\newcommand{\Var}{{\mathtt{Var}}}
\DeclareMathOperator{\Varr}{Var}
\newcommand{\new}{{\mathtt{new}}}
\newcommand{\case}{{\mathtt{case}}}

\newcommand{\lmam}{\mathrel{\sqsubseteq_{m\&m}}}
\newcommand{\emam}{\mathrel{\equiv_{m\&m}}}
\newcommand{\lst}{\mathrel{\lesssim}}
\newcommand{\smam}{\mathrel{\sim_{m\&m}}}
\newcommand{\amam}{\mathrel{\approx_{m\&m}}}

\newcommand{\oes}{\sim}

%% Idealized Algol things

\newcommand{\com}{{\mathtt{com}}}
\newcommand{\skipp}{{\mathsf{skip}}}
\DeclareMathOperator{\seq}{{\mathsf{seq}}}
\DeclareMathOperator{\neww}{{\mathsf{new}}}
\DeclareMathOperator{\mkvar}{{\mathsf{mkvar}}}
\newcommand{\deref}{\texttt{@}}
\DeclareMathOperator{\dereff}{\mathsf{deref}}
\DeclareMathOperator{\assign}{\mathsf{assign}}
\newcommand{\ia}[2]{\langle #1 , #2 \rangle}
\newcommand{\stup}[3]{\langle #1 \mid #2 \mapsto #3 \rangle}

%% Hyland-Ong games things

\newbox\gnBoxA
\newdimen\gnCornerHgt
\setbox\gnBoxA=\hbox{$\ulcorner$}
\global\gnCornerHgt=\ht\gnBoxA
\newdimen\gnArgHgt
\def\pv #1{%
    \setbox\gnBoxA=\hbox{$#1$}%
    \gnArgHgt=\ht\gnBoxA%
    \ifnum     \gnArgHgt<\gnCornerHgt \gnArgHgt=0pt%
    \else \advance \gnArgHgt by -\gnCornerHgt%
    \fi \raise\gnArgHgt\hbox{$\ulcorner$} \box\gnBoxA %
    \raise\gnArgHgt\hbox{$\urcorner$}}
\def\ov #1{%
    \setbox\gnBoxA=\hbox{$#1$}%
    \gnArgHgt=\ht\gnBoxA%
    \ifnum     \gnArgHgt<\gnCornerHgt \gnArgHgt=0pt%
    \else \advance \gnArgHgt by -\gnCornerHgt%
    \fi \raise\gnArgHgt\hbox{$\llcorner$} \box\gnBoxA %
    \raise\gnArgHgt\hbox{$\lrcorner$}}
\newcommand{\ct}[1]{\lceil#1\rceil}
\DeclareMathOperator{\Int}{int}

%% Nondeterministic Factorization things

\newcommand{\code}{\mathsf{code}}
\newcommand{\Det}{\mathsf{Det}}

%% Flexible strategy things

\newcommand{\stle}{{\;\le_s\;}}
\newcommand{\steq}{{\;=_s\;}}
\newcommand{\exle}{\sqsubseteq}
\newcommand{\exlub}{\bigsqcup}
\newcommand{\dv}{{\text{\lightning}}}
\DeclareMathOperator{\pocl}{pocl}
\newcommand{\plot}{\mathrel{\triangleleft}}
\newcommand{\shad}{\mathfrak{S}}
%\newcommand{\tree}{\mathfrak{T}}
\newcommand{\Tau}{T}
\newcommand{\Epsilon}{E}
\newcommand{\sw}{\triangleleft}

%% Roman numerals

\newcommand{\RN}[1]{%
  \textup{\uppercase\expandafter{\romannumeral#1}}%
}
\newcommand{\RNl}[1]{%
  \mathrel{\raisebox{1pt}{$\overline{\underline{#1}}$}}
}

%% Game language things

\newcommand{\ul}[1]{{\underline{#1}}}
\newcommand{\A}{{\mathcal{A}}}
\renewcommand{\P}{\mathcal P}
\newcommand{\M}{\mathcal M}
\newcommand{\N}{\mathcal N}
\newcommand{\X}{\mathcal X}
\newcommand{\YY}{\mathcal Y}
\newcommand{\hole}{\blank}
\newcommand{\Tct}{\xrightarrow{T}t}
\newcommand{\teamconverge}[2]{\xrightarrow{#1}#2}

%% Inference rule things
\newcommand{\rulename}[1]{\LeftTirNameStyle{#1}}
\newcommand{\ts}{\mathbin{\vdash}}
\newcommand{\nts}{\mathbin{\not\vdash}}

%% Double category things
\newcommand{\hc}[2]{\left({#1}\middle|{#2}\right)}
\newcommand{\vc}[2]{\left(\frac{#1}{#2}\right)}

%% What is going on?
\DeclareMathOperator{\Kl}{Kl}
\DeclareMathOperator{\Mell}{Mell}
\newcommand{\powerset}{\mathcal P}
\DeclareMathOperator{\ask}{{\mathsf{ask}}}
\newcommand{\sleep}{{\mathsf{sleep}}}
\newcommand{\true}{\mathbbm{t}}
\newcommand{\false}{\mathbbm{f}}
\DeclareMathOperator{\If}{\mathsf{If}}
\newcommand{\Then}{\mathrel{\mathsf{then}}}
\newcommand{\Else}{\mathrel{\mathsf{else}}}
\newcommand\cat{\mathbin{+\mkern-10mu+}}

%% Profunctor arrows

\makeatletter
\def\slashedarrowfill@#1#2#3#4#5{%
  $\m@th\thickmuskip0mu\medmuskip\thickmuskip\thinmuskip\thickmuskip
   \relax#5#1\mkern-7mu%
   \cleaders\hbox{$#5\mkern-2mu#2\mkern-2mu$}\hfill
   \mathclap{#3}\mathclap{#2}%
   \cleaders\hbox{$#5\mkern-2mu#2\mkern-2mu$}\hfill
   \mkern-7mu#4$%
}
\def\rightslashedarrowfill@{%
  \slashedarrowfill@\relbar\relbar\mapstochar\rightarrow}
\newcommand\xslashedrightarrow[2][]{%
  \ext@arrow 0055{\rightslashedarrowfill@}{#1}{#2}}
\makeatother
\newcommand{\pto}{{\xslashedrightarrow{} }}

%% Profunctors 
\DeclareMathOperator{\Prof}{Prof}
\DeclareMathOperator{\End}{End}
\DeclareMathOperator{\Endoprof}{Endoprof}

%% Our

\def\searchmacro#1{
  \AtBeginOfFiles{%
    \ifdefined#1
      \expandafter\def\csname \currfilename:found\endcsname{}%
    \fi}
  \AtEndOfFiles{%
    \ifdefined#1
      \unless\ifcsname \currfilename:found\endcsname
        \immediate\write\finder{found in '\currfilename'}%
    \fi\fi}}

%% Isomorphism arrows on commutative diagrams %%
\tikzset{Isom/.style={every to/.append style={edge node={node [sloped, above, allow upside down, auto=false]{$\cong$}}}},
         Isom'/.style={every to/.append style={edge node={node [sloped, above, allow upside down, auto=false, rotate=180]{$\cong$}}}},
         Sim/.style={every to/.append style={edge node={node [sloped, above, allow upside down, auto=false]{$\sim$}}}},
         Sim'/.style={every to/.append style={edge node={node [sloped, above, allow upside down, auto=false, rotate=180]{$\sim$}}}}}

%% Adjunctions
\newcommand{\adjunction}[4]{%
  {#1} \underset{\underset{#3}{\longleftarrow}}{\overset{\overset{#2}{\longrightarrow}}{\bot}} {#4}}        

%% Important!
\newcommand\Mellies{Melli\`{e}s\xspace}

\makeatletter
\newcommand{\colim@}[2]{%
  \vtop{\m@th\ialign{##\cr
    \hfil$#1\operator@font colim$\hfil\cr
    \noalign{\nointerlineskip\kern1.5\ex@}#2\cr
    \noalign{\nointerlineskip\kern-\ex@}\cr}}%
}
\newcommand{\colim}{%
  \mathop{\mathpalette\colim@{\rightarrowfill@\textstyle}}\nmlimits@
}
\makeatother

\makeatletter
\newcommand{\laxcolim@}[2]{%
  \vtop{\m@th\ialign{##\cr
    \hfil$#1\operator@font colim_l$\hfil\cr
    \noalign{\nointerlineskip\kern1.5\ex@}#2\cr
    \noalign{\nointerlineskip\kern-\ex@}\cr}}%
}
\newcommand{\laxcolim}{%
  \mathop{\mathpalette\laxcolim@{\rightarrowfill@\textstyle}}\nmlimits@
}
\makeatother

\DeclareMathOperator{\Colim}{colim}

\DeclareMathOperator{\DG}{DG}
\DeclareMathOperator{\RV}{RV}
\newcommand{\Rv}{\catname{Rv}}

\let\choose\undefined
\DeclareMathOperator{\choose}{\mathsf{choose}}
\DeclareMathOperator{\tr}{tr}
\DeclareMathOperator{\test}{test}

%% Slot game things %%
\newcommand{\circled}[1]{\raisebox{.5pt}{\textcircled{\raisebox{-.9pt} {#1}}}}
\newcommand{\slot}{{\circled{\$}}}

\DeclareMathOperator{\may}{may}
\DeclareMathOperator{\must}{must}

\newcommand{\encode}[1]{\lceil#1\rceil}
\DeclareMathOperator{\app}{\mathsf{app}}
\DeclareMathOperator{\lett}{\mathsf{let}}
\newcommand{\inn}{\mathrel{\mathsf{in}}}
\DeclareMathOperator{\byval}{\mathsf{byval}}

\DeclareMathOperator{\rread}{read}
\DeclareMathOperator{\wwrite}{write}

\DeclareSymbolFont{bbsymbol}{U}{bbold}{m}{n}
\DeclareMathSymbol{\bbsemicolon}{\mathbin}{bbsymbol}{"3B}
\newcommand{\semicom}{\bbsemicolon}

\newcommand{\ms}{\makebox[-1pt]{}}

\DeclareMathOperator{\Acc}{Acc}
\DeclareMathOperator{\im}{Im}
\DeclareMathOperator{\wit}{wit}

%%% END Article customizations



\usepackage[T1]{fontenc}

\newcommand{\game}{\textsc{game}}
\newcommand{\play}{\textsc{play}}
\newcommand{\move}{\textsc{move}}
\newcommand{\team}{\textsc{team}}
\newcommand{\strategy}{\textsc{strategy}}

\begin{document}

\section{A language of games}
\label{sec:game-language}

\subsection{Terms}

In this section, we describe a general framework of the language of games.  
The underlying language we shall normally start with is the simply typed lambda calculus.  
To this, we fix a finite set $\A$ of countable sets and for each $A\in\A$ we add a type $\ul{A}$ to the language.  The types of the form $\ul{A}$ are called the \emph{ground types}.
We then add a set of constants to the language.  These take one of two forms:
\begin{description}
  \item[Value constants] A \emph{value constant} takes the form $\ul{a}\from\ul{A}$, where $A\in\A$ and $a\in A$.  For our purposes, we shall assume that our language includes a value constant for every element of every set $A\in\A$.
  \item[Function constants] Given $A\in\A$ and a type $T$, a \emph{function constant} takes the form $\ul{\phi}$, where $\phi$ is some function sending elements of $A$ to terms of type $T$.
\end{description}

\begin{example}
  In the usual presentation of PCF, we have a type $\nat=\ul{\bN}$.  Then numerals $\n$ are the value constants $\ul{n}$, while the terms $\suc\from\nat\to\nat$ and $\IfO_T\from\nat\to T\to T\to T$ are function constants corresponding to the functions
  \begin{gather*}
    \phi_\suc \from n \mapsto \ul{n + 1} \\
    \phi_{\IfO} \from n \mapsto \begin{cases}
      \lambda x . \lambda y . x & \text{if $n=0$} \\
      \lambda x . \lambda y . y & \text{otherwise}
    \end{cases}
  \end{gather*}
\end{example}

\begin{remark}
  Function constants may refer to other function constants.  To avoid circular definitions, define a directed graph on the set of function constants by drawing an edge from $\ul{\phi}$ to $\ul{\psi}$ if $\phi(n)$ contains an occurrence of $\ul{\psi}$ for some $n$.  We require that this graph be acyclic (in particular, having no loops) and that it be well-founded in the sense that there is no infinite sequence $\ul{\phi_1} \to \ul{\phi_2} \to \cdots$.
\end{remark}

This framework is enough to formalize simple deterministic languages such as PCF.  In order to turn it into a language that can model games, we shall add additional constants to our language:

\begin{definition}
  We fix a finite set $\P$ of \emph{players}.  
  
  \begin{description}
    \item[Move constants] A \emph{move constant} takes the form $S^b\from \ul{A}$, where $S\subset A$ is non-empty and $b\in\mathcal P$.
  \end{description}

  We read $S^b$ as `Player $b$ chooses a value from $S$'.

  Our language will be given by the tuple $(\A,\F, \M)$, where $\F$ the set of function constants and $\M$ the set of move constants.
\end{definition}

\begin{example}
  We can recover the language PCF by setting $\P = \{\RN{1}\}$ and defining numerals by $\n=\ul{n}$, using the function constants $\suc$ and $\IfO$ as defined above.  We can add a nondeterministic oracle to the language by setting $\wn = \bN^\RN{1}$.
\end{example}

\subsection{Operational semantics}

We define a notion of \emph{evaluation context} in the style of Felleisen:

\[
  C ::= \hole \mid C\; M\mid\ul{\phi}\;C
  \]
Here, $M$ ranges over terms in the language and $\ul{\phi}$ ranges over the function constants in $\F$.  We write $M=C[M']$ if the term $M$ is obtained from $C$ by replacing the unique occurrence of $\hole$ in $C$ with the term $M'$.

We call a program of ground type $\ul{A}$ a \emph{game}.  We call a subset $T\subset\P$ a \emph{team} and a subset $t\subset A$ a \emph{target}.

We now define a predicate $\Tct$ on terms as in Figure \ref{fig:big-step}.  
\begin{figure}
  \begin{mathpar}
    \inferrule*{b\in T \\ \exists a\in S\esuchthat C[\ul{a}]\Tct}{C[S^b]\Tct}
      \and \inferrule*{b \not\in T \\ \forall a\in S\esuchthat C[\ul{a}]\Tct}{C[S^b]\Tct}
      \and \inferrule*{C[M[N/x]]\Tct}{C[(\lambda x.M)N]\Tct}
      \and \inferrule*[right=$\ul{\phi}\in\F$]{C[\phi(a)]\Tct}{C[\ul{\phi}\;\ul{a}]\Tct}
  \end{mathpar}
  \caption{Big-step operational semantics for the language $(\A,\F,\M)$}
  \label{fig:big-step}
\end{figure}
To understand this semantics better, let's look at the small-step semantics defined in Figure \ref{fig:small-step}.
\begin{figure}
  \begin{mathpar}
    \inferrule*[left=move,right=$a\in S$]{ }{S^b\opto \ul{a}}
      \and \inferrule*[left=function]{ }{\ul{\phi}\;\ul{a}\opto\phi(a)}
      \and \inferrule*[left=beta-reduction]{ }{(\lambda x.M)N \opto M[N/x]}
      \and \inferrule*[left=context]{M \opto N}{C[M]\opto C[N]}
  \end{mathpar}
  \caption{Small-step operational semantics for the language $(\A,\F,\M)$}
  \label{fig:small-step}
\end{figure}
A term $M\from T$ now defines a game between team $T$ and the opposing team $\P\setminus T$ as follows: we evaluate the term step by step.  At each step, the proof that $M\opto M'$ must terminate with one of the three rules \rulename{move}, \rulename{function} or \rulename{beta-reduction}.  In the last two cases, there is a unique choice of reduction to make.  In the first case, there are multiple possibilities, one for each value $a\in S$.  

In such a case, if $b\in T$ then team $T$ are allowed to choose which value $a$ to use.  If $b\not\in T$, then the opposing team $\P\setminus T$ are allowed to choose.  Team $T$ wins if the term eventually converges to a value $\ul{a}$, where $a\in t$ and team $\P\setminus T$ wins if the term converges to $\ul{b}$, where $b\not\in t$.  Then the judgement $M\Tct$ precisely says that team $T$ has a strategy to ensure that the term $M$ ends up in $t$.  

\begin{example}
  Coming back to our nondeterministic PCF example, if $M\from\nat$ is a term, then the judgement that $M\teamconverge{\emptyset}{\{n\}}$ says that $M$ \emph{must converge} to $n$.  The judgement that $M\teamconverge{\{I\}}{\{n\}}$ says that $M$ \emph{may converge} to $n$.
\end{example}

We read $M\Tct$ as `Team $T$ can force the term $M$ into the target $t$'.

\begin{definition}[Observational equivalence]
  We define observational equivalence for a specified set $\T\subset\power(\P)$ of `friendly teams' and a specified set $\t \subset \prod_{A\in\A}\power(A)$ of specified targets.

  If $A\in\A$ and $M,M'\from\ul{A}$, we say that $M\equiv_e M'$ ($M$ is \emph{observationally equivalent to $M'$}) if for all teams $T\in\T$ and for all target sets $t\in\t$ we have $M\xrightarrow{T}t(A)$ if and only if $M'\xrightarrow{T}t(A)$.

  For general terms $M,M'$ of the same type, we say that $M\equiv_e M'$ if for all contexts $C$ such that $C[M],C[M']$ have ground type, $C[M]\equiv_e C[M']$.
\end{definition}

\begin{definition}[Observational order]
  Similarly, we say that a term $M$ is \emph{observationally less than or equal to} another term $M'$, and write $M\le_e M'$, if for all $T\in\T$ and for all $t\in\t$, $M\xrightarrow{T}t(A)$ implies $M'\xrightarrow{T}t(A)$.

  For general terms $M,M'$ of the same type, we say that $M\le_e M'$ if for all contexts $C$ such that $C[M],C[M']$ have ground type, $C[M]\le_e C[M']$.
\end{definition}

\begin{example}
  PCF with may equivalence is the case corresponding to the set $\T=\{\{\RN{1}\}\}$ and $\t=\{\bN\}$.
  
  PCF with must equivalence corresponds to $\T=\{\emptyset\}$ and $\t=\{\bN\}$.

  PCF with may-and-must equivalence corresponds to $\T=\{\emptyset, \{\RN{1}\}\}$ and $\t=\{\bN\}$..
\end{example}

\begin{lemma}
  Let $M,M:S$ be such that $M\le_e M'$, and let $F,F':S\to T$ be such that $F\le_e F'$.  
  Then $F M \le_e F' M'$.
\end{lemma}

\subsection{Recursion}

In order to recover the full language PCF, we need to add recursion combinators $\Y_T$ to our language, corresponding to the following small-step reduction:
\[
  \inferrule*[left=recursion]{ }{\Y_T M \opto M ( \Y_T M )}
  \]
giving us the big-step rule
\[
  \inferrule{C[M (\Y_T M)] \Tct}{C[\Y_T M] \Tct}
  \]
This rule introduces for the first time the possibility of \emph{divergence}, or non-termination.  
We will assume that a team $T$ never wants the program to diverge.  
We do not lose anything by making this assumption; indeed, we can model the predicate $M \xrightarrow{T} t \cup \{\bot\}$ (`Team $T$ has a strategy to force the term $M$ into the target $t$ or to make it diverge') as the negation of the predicate $M \xrightarrow{\P\setminus T}\A \setminus t$ (`The opposition to team $T$ has a strategy to force the term $M$ into the complement of the target $t$'): the games we are considering are all \emph{open} in the sense that they terminate after finitely many moves, so exactly one of the two teams has a strategy by the Gale-Stewart theorem.

It turns out that we can also model PCF -- and other languages we have constructed in this section -- as a subset of a larger language that does not involve any special recursion constants, but uses only the function, value and move constants that we have already introduced.  
The advantage of this is that, since function constants are not self-referential, they are much easier to model in a denotational semantics than recursion, which usually requires some order-theoretic properties of the underlying model.  
The only innovation we will need to make on the semantics side is to give denotational semantics for the move constants; this will be covered in Section \ref{sec:denotational-semantics}.

As an illustration of our technique, we show how PCF itself can be modelled inside another language.
First, we introduce a new function constant, $\iter_T\from \nat \to T \to (T \to T) \to T$, corresponding to iteration (realization of natural numbers as Church numerals).  
$\iter_T$ is defined by $\iter_T=\ul{\iota_T}$, where:
\[
  \iota_T(n) = \lambda z.\lambda s.\underbrace{s (s (s \cdots}_{n\text{ times}}z)\cdots))
    \]
In order to model divergence, we add special divergence constants $\Omega_T$ for each type $T$.  
We include no extra rules for these divergence constants; if we end up with $C[\Omega_T]$ for some context $C$, then we consider the term to be `stuck', and to have converged to no value.

Now let $M$ be a general PCF term (possibly involving instances of $\Y_T$ for some types $T$.  Since PCF is a deterministic language, if $M$ converges then it does so in a bounded number of steps $m$.  If we now replace each instance of $\Y_T$ in $M$ with the function
\[
  \lambda N.\iter_T\;\ul{m}\;\Omega_T N
  \]
then we claim that the term converges to the same value.  Indeed, we observe that $\iter_T\;\ul{m}\;\Omega_T N$ is equivalent to the term
\[
  \underbrace{N (N (N \cdots}_{m\text{ times}}\Omega_T)\cdots))
  \]
and is therefore itself equivalent to
\[
  N(\iter_T (\ul{m-1})\Omega_T N)
  \]
The term $\iter_T\;\ul{m}\;\Omega_T N$ therefore has the same behaviour as the term $\Y_T N$, except that the index $m$ decreases by $1$ each time we use the rule.  
Since the reduction of the term $N$ takes $m$ steps anyway, we will never `run out' of iterations.

However, although this allows us to model \emph{individual programs} in PCF, it does not allow us to model functions that may take an unbounded number of recursion steps depending on their argument.  
In particular, we cannot model $\Y_T$ itself using iteration alone.  
In order to do this, we add a nondeterministic oracle $\wn$ to the language, and using may-equivalence as our equivalence relation on terms.  
In our framework, this corresponds to adding a player so that $\P=\{\RN{1}\}$ and then setting $\wn = \mathbb N^{\RN{1}}$.
Then we can model $\Y_T$ as the term
\[
  \lambda N.\iter_T\;\wn\;\Omega_T N
  \]
Because of the may-equivalence, this now has the same operational semantics (within PCF) as our original $\Y_T$: if a PCF term converges, then it converges after $m$ steps for some $m$, and so there is the \emph{possibility} that the oracle $\wn$ will choose a large enough value such that the term must converge.  
So the term may converge to the same value as the original term with $\Y_T$.  
On the other hand, if a PCF term diverges, then there is no number we could replace the $\wn$ with that would cause the term to converge.  
In that case, for every $m$ that the oracle could choose, the term will eventually `run out' and reach the divergence term $\Omega_T$, at which point it will get stuck.

\begin{theorem}
  There is an embedding of the language
  \[
    \text{PCF} = \lambda + \mathbb N + \Y
    \]
  into the language
  \[
    \text{Nondeterministic iterative PCF} = \lambda + \mathbb N + \iter + \Omega + \wn
    \]
  under may-equivalence.  
  That is to say, there is an structurally-inductive map $r$ from PCF terms/contexts to terms/contexts of nondeterministic iterative PCF such that two PCF programs $M$ and $N$ are observationally equivalent if and only if the corresponding programs in nondeterministic iterative PCF are may-equivalent.
\end{theorem}

\begin{proof}
  The mapping $r$ replaces every instance of $\Y_T$ with $(\lambda N.\iter_T\;\wn\;\Omega_T N$).

  Note the following observational equivalence:
  \[
    \Y_T \equiv_e \lambda N.\iter\;\ul{m}\;(\Y_T N)\;N
    \]
  which holds for any $n$.  
  This is because both terms may converge 
\end{proof}

\section{\game{}s}

In this section, we define the notion of a \game, one of our two alternative formulations of our basic intuition of a `game'.

We fix some set $\P$ of players throughout.

\begin{definition}
  Let $A$ be a set.  
  A \emph{\game{} $\Gamma$ with result set $A$} is given by a tuple $(\M_\Gamma,P_\Gamma,J_\Gamma)$ where
  \begin{itemize}
    \item $\M_\Gamma$ is a set of \move{}s.
    \item $P_\Gamma$ is a prefix-closed set of finite sequences of pairs of the form $p\from m$, where $p\in\P$ is a player and $m\in\M_\Gamma$ is a move.  We define $T_\Gamma$, the set of \emph{terminal \play{}}s in $P_\Gamma$ to be the set of all $s\in P_\Gamma$ such that $sa\not\in P_\Gamma$ for any $a$.  
    \item $J_\Gamma\from T_\Gamma\to A$ is a judgement function that says what the result is after a complete sequence of moves.
  \end{itemize}
  such that
  \begin{itemize}
    \item There is always exactly one player whose turn it is.  
      Explicitly, if $sa,sb\in P_\Gamma$, then the moves $a$ and $b$ are made by the same player.
  \end{itemize}
\end{definition}

\begin{remark}
  $\Gamma$ may contain infinitely long chains of sequences of \move{}s.  
  In that case, there is no result value for the \play{}.
\end{remark}

\begin{remark}
  This construction gives rise to a monad $G$ on the category of (possibly large) sets, where a set $A$ is sent to the set $GA$ of games with result set $A$.  
  The natural transformation $GGA\to GA$ is the one that takes a game whose results are games with result set $A$, and forms a game that plays as follows: the players play through the first game; then, if and when they reach a terminal \play{}, they then play through the resulting game to reach a value in $A$.

  We will not make great use of this monad, but it will be implicit in a lot of what we do.  
  As we can model the powerset monad using nondeterministic strategies, so our game semantics will allow us to model this game monad.
\end{remark}

\begin{definition}
  Let $\Gamma$ be a game with result set $A$.  
  Let $T\subset \P$ be a team of players.  
  Then a $T$-strategy $S$ for $\Gamma$ is a prefix-closed subset of $P_\Gamma$ subject to the following rules:
  \begin{itemize}
    \item If $s\in S$ and $sa\in P_\Gamma$ is such that the move $a$ is made by a player \emph{not} in $T$, then $sa\in S$.
    \item If $sa,sb\in S$, and the moves $a,b$ are made by a player in $T$, then $a=b$.
    \item If $s\in S$ and there exists $sa\in P_\Gamma$ such that the move $a$ is made by a player in $T$, then there exists $b$ such that $sb\in P_\Gamma$.
  \end{itemize}
\end{definition}

\section{Game Semantics}

\subsection{Arenas}

Our basic game semantics will be the one given in \cite{hoPcf}.  
We first present the Hyland-Ong game semantics.  
Then, we shall enrich the semantics by appending \game{}s on to strategies.

\begin{definition}
  An \emph{arena} is a triple $A=(M_A,\lambda_A,\ts_A)$ where
  \begin{itemize}
    \item $M_A$ is a countable set of \emph{moves}.
    \item $\lambda_A\from M_A\to\OP\times\QA$ designates each move as either a \emph{$P$-move} or an \emph{$O$-move} and as either a \emph{question} or an \emph{answer}.  
      We will often split $\lambda_A$ into its two parts
      \begin{gather}
        \lambda_A^{OP} = \pr_1\circ\lambda_A\from M_A\to\OP\\
        \lambda_A^{QA} = \pr_2\circ\lambda_A\from M_A\to\QA
      \end{gather}
    \item $\ts_A$ is an \emph{enabling relation} between $M_A+*$ and $M_A$ satisfying:
      \begin{description}
        \item[Alternation] If $a \ts_A b$ then $\lambda_A^{OP} (a) = \neg(\lambda_A^{OP}(b))$
        \item[Initial moves] If $*\ts_A a$ then $\lambda_A(a) = (O,Q)$ and for all $b\in M_A$, $b\not{\ts_A} a$.
        \item[Answers are enabled by questions] If $a,b\in M_A$ are such that $a \ts_A b$ and $\lambda_A^{QA}(b)=A$ then $\lambda_A^{QA}(a)=Q$.
      \end{description}
      We say that a move $a$ is \emph{initial} if $*\ts_A a$.
  \end{itemize}
\end{definition}

\begin{definition}
  A \emph{justified play} $s$ in an arena $A$ is given by a (finite or infinite) sequence of moves of $A$, together with a \emph{justification pointer} for each non-initial move $a$ occurring in $s$ back to a previous move $b$ such that $b\ts_A a$.  
  In that case, we say that $b$ \emph{justifies} $a$.
  
  Furthermore, we require that if $s\ne \emptyplay$ then the first move in $s$ is initial.
\end{definition}

\begin{definition}
  We say that a move $a$ \emph{hereditarily justifies} a move $b$ if there is a sequence of justification pointers in $J(s)$ that leads from $b$ to $a$.
\end{definition}

\subsection{Compound arenas}

Let $A,B$ be arenas.  
Then there are two main ways to combine $A$ and $B$ to make a new arena.  
The first forms the \emph{product} $A\times B$ of $A$ and $B$:
\begin{itemize}
  \item $M_{A\times B} = M_A + M_B$.
  \item $\lambda_{A\times B} = [\lambda_A, \lambda_B]$.  
  \item $x\ts_{A\times B} m$ if and only if $x\ts_A m$ or $x\ts_B m$.
\end{itemize}
The second forms the \emph{function space} $A \iimpl B$ from $A$ to $B$:
\begin{itemize}
  \item $M_{A\iimpl B} = M_A + M_B$.
  \item $\lambda_{A\iimpl B} = [(\neg\times\id)\circ\lambda_A,\lambda_B]$.  
  \item $x\ts_{A\iimpl B} m$ if and only if
    \begin{description}
      \item[either] $x\ts_B m$
      \item[or] $x\ne *$ and $x\ts_A m$
      \item[or] $*\ts_B x$ and $*\ts_A m$.
    \end{description}
\end{itemize}

\subsection{Strategies}

In order to capture the possibility of divergence, we follow \cite{mcCHFiniteND} and extend the usual definition of a strategy as a set of traces by adding in an extra set of positions designating which plays may lead to a divergence.  

Blah blah blah blah blah.

\subsection{Composition of strategies}

Same as usual.

\section{Strategies as \game{}s}

\subsection{\game{}-plays and \game{}-strategies}

We now enrich the structure of strategies by equipping them with certain classes of \game{}s. We fix a set $\P$ of players throughout.

\begin{definition}
  Let $A$ be an arena and let $\M$ be a set of moves.  
  Then a \game{}-play over $M_A$ is a (possibly infinite) sequence $s$ of moves that are either:
  \begin{itemize}
    \item Moves $a\in M_A$, called \emph{moves} or:
    \item Moves $p\from m\in \P\times\M$, called \emph{\move{}s}
  \end{itemize}
  such that
  \begin{itemize}
    \item The first move in $s$ is an $O$-move in $A$.
    \item If a move of the form $(p\from m)$ is followed by a move $a\in M_A$, then $a$ is a $P$-move.
    \item Every $P$-move in $A$ is followed immediately (if at all) by an $O$-move in $A$.
  \end{itemize}

  The \emph{shadow} of a \game{}-play is the sequence formed by removing all moves of the form $p\from m$.  
  A \emph{justified \game{}-play for $A$} is a \game{}-play $s$ over $A$, together with the structure of a justified play over the shadow of $s$.
\end{definition}

\begin{definition}
  A \emph{\game{}-strategy} for an arena $A$ is a set $\M_\sigma$ of moves and a prefix-closed set of \game{}-plays over $\M$ such that when we forget about the \play{}s $\Sigma_s^t$, the resulting collection of justified sequences is a strategy for $A$, and such that if $s.(p\from m),s.(q\from n)\in S$, then $p=q$.
\end{definition}

\subsection{Composition}

Possibly slightly tricky; the idea is that the process of `hiding' moves concatenates \play{}s together.

\subsection{Ordering strategies}

\begin{definition}
  Let $\sigma$ be a \game{}-strategy and let $T\subset \P$ be a team of players.

  A $T$-\strategy{} for $\sigma$\footnote{Yes, that is a \strategy{} for a strategy.} is a subset $S$ of $\sigma$ subject to the following rules:
  \begin{itemize}
    \item If $s\in S$ and $sa\in\sigma$, where $a$ is either an $O$-move in $M_A$ or a move $(p:m)$, where $m\not\in T$, then $sa\in S$.
    \item If $sa,sb\in S$, where $a,b$ are $P$-moves in $M_A$, then $a=b$.
    \item If $s.(p\from m),s.(p\from n)\in S$, and $p\in T$, then $m=n$.
    \item If $s\in S$ and $sa\in\sigma$, where $a$ is either a $P$-move or a move of the form $(p\from m)$ for $m\in T$, then $sb\in S$ for some $b$.
  \end{itemize}
\end{definition}

\begin{definition}
  Let $\sigma$ be a \game{}-strategy, and let $S$ be a \strategy{} for $\sigma$.  
  Then, if $s$ is an $O$-play in $A$ that is the shadow of some \game{}-play in $\sigma$, we write $R_S^s$ for the set of moves $a$ such that $sa$ is the shadow of some \game{}-play in $S$.

  We write $R_\sigma^s$ for the set
  \[
    \{R_S^s\suchthat \textrm{$S$ is a \strategy{} for $\sigma$ and $s$ is the shadow of a play in $S$}, R_S^s\ne\emptyset\}
    \]
\end{definition}

\begin{definition}
  Let $\sigma,\tau$ be \game{}-strategies.  
  We say that $\sigma \le_T \tau$ if the shadow of $\sigma$ is a subset of the shadow of $\tau$ and if for all $O$-moves $s$ that are the shadow of a play in $\sigma$, we have $R_\sigma^s\subset R_\tau^s$.
\end{definition}

\subsection{Example -- one-player \game{}s}

Suppose that $\P=\{\RN{1}\}$.  
Then there are two possibilities for $T$: $T=\emptyset$ and $T=\{\RN{1}\}$.

We show that these correspond to the may- and must-orderings defined by Harmer and McCusker.

\begin{proposition}
  Suppose $\{P\}=\{\RN{1}\}$.  Let $A$ be an arena and let $\sigma$ be a strategy for $A$.  
  Define
  \begin{itemize}
    \item $T_\sigma$ to be the shadow of $\sigma$
    \item $D_\sigma$ to be the set of all $O$-positions $d$ in $T_\sigma$ such that there is either an infinite position in $\sigma$ having shadow $d$ or such that there is a terminal position in $\sigma$ having shadow $d$ and ending with a \move{}.
  \end{itemize}

  Now let $\sigma,\tau$ be \game{}-strategies for $A$.

  i) $\sigma \le_{\{\RN{1}\}} \tau$ if and only if $T_\sigma \subset T_\tau$.
  
  ii) $\sigma \le_\emptyset \tau$ if and only if ...
\end{proposition}

\begin{proof}

  (i): \textbf{If. }
  Suppose that $T_\sigma\subset T_\tau$.  
  We want to show that $\sigma\le_{\{\RN{1}\}}$, for which it suffices to show that for all $O$-positions $s\in T_\sigma$ we have $R_\sigma^s\subset R_\tau^s$.  
  Let $s\in T_\sigma$ be an $O$-position and let $S$ be a \strategy{} for $\sigma$ such that $s$ is the shadow of some \game{}-play in $S$ and $R_S^s\ne\emptyset$.  
  Let $a\in R_S^s$.  
  By the definition of a \strategy{} for $\sigma$, we can see that in fact $R_S^s=\{a\}$.  

  Since $T_\sigma\subset T_\tau$, we know that there is some \game{}-play in $\tau$ whose shadow is $sa$.  
  Then it is easy to construct a \strategy{} $S'$ for $\tau$ such that $a\in R_{S'}^s$.  
  Moreover, by the same argument, we see that $R_{S'}^s=\{a\}$.  
  Therefore, $R_\sigma^s\subset R_\tau^s$.  

  \textbf{Only if. }
  Suppose that $\sigma\le_{\{\RN{1}\}}\tau$.  
  Let $sa\in T_\sigma$ be a non-empty $P$-position.  
  Then $sa$ is the shadow of some position in $\sigma$.  
  We can construct a \strategy{} $S$ for $\sigma$ such that $sa$ is in the shadow of $S$; moreover, we have $R_S^s=\{a\}$.  
  Therefore, $\{a\}=R_{S'}^s$ for some \strategy{} $S'$ for $\tau$.  
  It follows that $sa\in T_\tau$.

  (ii): \textbf{If. }
  

\end{proof}

\begin{example}
  Fix two players B (black) and W (white) and a set of moves $\M$ consisting of all legal chess moves.  
  Then we can model games played on a chess board using the arena given by
  \[
    (\{q,ww,dr,bw\}, \{q\mapsto(O,Q),\;ww,dr,bw\mapsto(P,A)\},\{q\ts ww,dr,bw\}, \{q\})
    \]
  A justified play over the sequence $q.ww$ is a collection of justification pointers $j$ from $ww$ to $q$, each annotated with a sequence of chess moves carried out by one or other of the two players.  
  The game of chess itself is then described by three particular justified plays:
  \begin{itemize}
    \item One over the sequence $q.ww$, where the labels of the justification pointers from $ww$ to $q$ are precisely those sequences of moves leading to a win for white.
    \item One over the sequence $q.dr$, where the labels of the justification pointers from $dr$ to $q$ are precisely those sequences of moves leading to a draw.
    \item One over the sequence $q.bw$, where the labels of the justification pointers from $bw$ to $q$ are precisely those sequences of moves leading to a win for black.
  \end{itemize}
\end{example}


\bibliographystyle{alpha}
\bibliography{../common/phd_bibliography}

\end{document}
