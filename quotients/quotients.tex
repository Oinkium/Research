%%%%%%%%%%%%%%%%%%%% author.tex %%%%%%%%%%%%%%%%%%%%%%%%%%%%%%%%%%%
%
% sample root file for your "contribution" to a proceedings volume
%
% Use this file as a template for your own input.
%
%%%%%%%%%%%%%%%% Springer %%%%%%%%%%%%%%%%%%%%%%%%%%%%%%%%%%


\documentclass{svproc}
%
% RECOMMENDED %%%%%%%%%%%%%%%%%%%%%%%%%%%%%%%%%%%%%%%%%%%%%%%%%%%
%

% to typeset URLs, URIs, and DOIs
\usepackage{url}
\usepackage{amsmath}
\usepackage{amssymb}
\usepackage{tikz}
\usetikzlibrary{cd}
\usepackage{mathpartir}
\def\UrlFont{\rmfamily}

\newcommand\C{\mathcal{C}}
\newcommand\from{\,\colon\,}
\DeclareMathOperator{\End}{End}
\newcommand{\passoc}{\texttt{passoc}}
\newcommand{\assoc}{\texttt{assoc}}
\newcommand\tensor\otimes
\newcommand\run{\texttt{r}}
\newcommand\lunit{\texttt{lunit}}
\newcommand\runit{\texttt{runit}}
\renewcommand\implies\multimap

\begin{document}
\mainmatter              % start of a contribution
%
\title{A Kleisli-like construction for Parametric Monads, with Examples}
%
\titlerunning{Dum-di-dum}  % abbreviated title (for running head)
%                                     also used for the TOC unless
%                                     \toctitle is used
%
\author{}
%
\authorrunning{} % abbreviated author list (for running head)
%
%%%% list of authors for the TOC (use if author list has to be modified)
\tocauthor{}
%
\institute{University of Bath, Claverton Down Road, Bath.  
BA2 7AY,\\
\email{wjg27@bath.ac.uk},\\ WWW home page:
\texttt{http://people.bath.ac.uk/wjg27}}

\maketitle              % typeset the title of the contribution

\begin{abstract}
\keywords{denotational semantics, category theory, game semantics}
\end{abstract}

\section{Introduction}

This paper is about general frameworks for modelling effects in a categorical semantics.  
The seminal paper in this field is Moggi's \emph{Computational Lambda-Calculus and Monads} \cite{Moggi}, which first put forward the idea that if $\C$ is a categorical model of a programming language, then we can simulate an effect in that language via a suitable \emph{monad} on $\C$.  
Given such a monad, its Kleisli category $Kl_M\C$ is then a model of a language formed by adding the effect to the original language.  

Spivey \cite{Spivey} and Wadler \cite{Wadler1,Wadler2} have transferred this theory to the real world by using monads as programming tools, particularly in the Haskell programming language.  



\section{Lax quotients of lax actions}

Let $\C'$ be a monoidal category and let $\C$ be a category.  
Then a \emph{lax iright action} of $\C'$ on $\C$ is a functor $\_.\_\from \C\times\C'\to \C$ that gives rise (through currying) to a lax monoidal functor $\C'\to \End[\C,\C]$.  
In other words, we have natural transformations $\passoc_{A,X,Y}\from A.X.Y\to A.(X\tensor Y)$ and $\run_A\from A \to A.I$ making the following diagrams commute for all objects $A$ of $\C$ and $X,Y,Z$ of $\C'$.

\begin{mathpar}
  \begin{tikzcd}[column sep=6em]
    A.X.Y.Z \arrow[dr,"\passoc_{A.X,Y,Z}"'] \arrow[r, "\passoc_{A,X\tensor Y,Z}"]
      & A.(X \tensor Y).Z \arrow[r, "\passoc_{A,X\tensor Y,Z}"]
        & A.((X \tensor Y) \tensor Z) \arrow[d, "A.\assoc_{X,Y,Z}"] \\
    %
      & A.X.(Y\tensor Z) \arrow[r, "\passoc_{A,X,Y\tensor Z}"]
        & A.(X\tensor(Y\tensor Z))
  \end{tikzcd}
  \and
  \begin{tikzcd}
    A.X \arrow[r, "\run_A.X"] \arrow[dr, "A.\lunit_X"']
      & A.I.X \arrow[d, "\passoc_{A,I,X}"] \\
    %
      & A.(I\tensor X)
  \end{tikzcd}
  \and
  \begin{tikzcd}
    A.X \arrow[r, "\run_{A.X}"] \arrow[dr, "A.\runit_X"']
      & A.X.I \arrow[d, "\passoc{A,X,I}"] \\
    %
      & A.(X \tensor I)
  \end{tikzcd}
\end{mathpar}

\begin{example}
  \begin{itemize}
    \item Any monad $M$ is an action of the trivial category on its underlying category $\C$, regarding $MA$ as the action $A.I$.
      The natural transformation $\passoc_{A,I,I}$ is precisely the monad action on $A$:
      \[
        A.I.I = MMA \to M A = A.I = A.(I\tensor I)\,.
        \]
      Dually, any lax action gives rise to a monad on the category being acted upon, by setting $MA = A.I$.
    \item If $\C$ is also a monoidal category and $J\from \C'\to \C$ is a lax monoidal functor with monoidal coherence $\mu$, then there is an action of $\C'$ on $\C$ given by
      \[
        A.X = A \tensor JX\,,
        \]
      and
      \[
        \passoc_{A,X,Y}=
        (A \tensor JX) \tensor JY \xrightarrow{\assoc_{A,JX,JY}}
        A \tensor (JX \tensor JY) \xrightarrow{A \tensor \mu_{X,Y}}
        A \tensor J(X\tensor Y)\,.
        \]

    \item If $\C$ is a monoidal closed category, and $j$ an oplax monoidal functor with coherence $\nu$, then there is an action of $\C'^{co}$ (i.e., $\C'$ with the opposite monoidal product) on $\C$ given by
      \[
        A.X = jX \implies A
        \]
      and
      \[
        \passoc_{A,X,Y}=jY \implies (jX \implies A) \to
        (jY \tensor jX) \implies A \xrightarrow{\nu_{Y,X}\implies A}
        j(Y \tensor X) \implies A\,.
        \]
      We sometimes refer to a right action of the opposite category $\C'^{co}$ on $\C$ as a \emph{left action} of $\C'$ on $\C$.  
      In that case, we write $X.A$ instead of $A.X$, so that the coherence becomes
      \[
        \passoc_{Y,X,A}\from Y.X.A \to (Y\tensor X).A\,.
        \]
    \item The intersection of the first two examples is the \emph{writer monad} given by $M_WX = X\tensor W$ for any monoid $W$ in $\C$.  
      The intersection of the first and third examples is the \emph{reader monad} given by $M^RX = R \implies X$ for any comonoid $R$ in $\C$ (and, in particular, for any object $R$ if $\C$ if the monoidal product in $\C$ is Cartesian).
      
      We shall therefore call an action of the form $A \tensor JX$ a \emph{writer-type action} and one of the form $JX \implies A$ a \emph{reader-type action}.
    \item We can define an \emph{oplax action} of $\C'$ on $\C$ to be a lax action of $\C'$ upon the opposite category $\C^{op}$.  
      In this case, the coherence $\passoc$ goes from $A.(X\tensor Y)$ to $A.X.Y$.  
      As monads are lax actions of the trivial category, so comonads are oplax actions of the trivial category.
  \end{itemize}
\end{example}

\bibliographystyle{spbasic}
\bibliography{quotients}

\end{document}
